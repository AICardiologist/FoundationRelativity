\documentclass[11pt]{article}

\usepackage[margin=1in]{geometry}
\usepackage{amsmath,amssymb,amsthm}
\usepackage[T1]{fontenc}
\usepackage[utf8]{inputenc}
\usepackage{lmodern}
\usepackage{microtype}
\usepackage{booktabs}
\usepackage[hidelinks]{hyperref}
\usepackage{parskip}

\newtheorem{theorem}{Theorem}[section]
\newtheorem{lemma}[theorem]{Lemma}
\newtheorem{proposition}[theorem]{Proposition}
\newtheorem{corollary}[theorem]{Corollary}
\newtheorem{definition}[theorem]{Definition}
\newtheorem{remark}[theorem]{Remark}

\title{The Physical Dispensability of the Fan Theorem:\\
BISH+LPO Suffices for All Empirical Content\\
of Compact Optimization and Variational Mechanics\\[1em]
\large Paper~30 in the Constructive Calibration Programme}
\author{Paul Chun-Kit Lee\\[2pt]\small New York University\\[2pt]\small\texttt{dr.paul.c.lee@gmail.com}}
\date{February 2026}

\begin{document}
\maketitle

\begin{abstract}
Papers~23 and~28 established that compact optimization (the Extreme Value Theorem on $[a,b]$) and variational action minimization in classical mechanics each cost exactly the Fan Theorem (FT) over BISH\@.
Paper~29 established that Fekete's Subadditive Lemma is equivalent to LPO, and that LPO is physically instantiated because phase transitions are real.
This raises the question: is FT an independent physical requirement, or is its cost an artifact of the variational proof method?

We prove that every \emph{empirically accessible prediction} derived from the FT-calibrated results in Papers~23 and~28 is recoverable in BISH+LPO, without invoking the Fan Theorem.
The key insight is twofold: (1)~the physical content of variational mechanics is the Euler--Lagrange equation, which is BISH (Paper~28), not the existence of a global minimizer; and (2)~approximate optimization to any finite precision $\varepsilon > 0$ is achievable via BMC (which is LPO) without FT\@.
FT captures the \emph{exact} existence of a minimizer; LPO captures \emph{convergent approximation} to the infimum.
Since no finite experiment can distinguish an exact minimizer from an $\varepsilon$-approximate one, FT is physically dispensable.

This is the first of two papers (with Paper~31 on dependent choice) investigating whether BISH+LPO is the complete logical constitution of empirically accessible physics.

All results are formalized in Lean~4.
\end{abstract}

\section{Introduction}

\subsection{Motivation: Is LPO the only axiom?}

Paper~29 established the equivalence of Fekete's Subadditive Lemma with LPO and, via the empirical reality of phase transitions, demonstrated that LPO is not a mathematical idealization but a physically instantiated principle.
This ontological upgrade of LPO transforms the calibration programme's central question.
The original question was: \emph{what does each physical theorem cost?}
The new question is: \emph{is LPO sufficient for all empirically accessible physics?}

The calibration table (Paper~10, v4.0) contains entries at seven distinct logical levels: BISH, LLPO, WLPO, LPO, MP, FT, CC/DC\@.
Since LPO implies WLPO, LLPO, and MP over BISH, every entry on the omniscience spine and the MP axis is automatically available in BISH+LPO\@.
Two independent branches remain: the Fan Theorem and the choice axis (CC/DC)\@.

This paper addresses the Fan Theorem.
Paper~31 will address dependent choice.
Together, they investigate whether BISH+LPO is the complete logical constitution of empirically accessible physics.

\subsection{The Fan Theorem entries}

Two entries in the calibration table require FT:

\begin{center}
\begin{tabular}{@{}lll@{}}
\toprule
Physical Result & Cost & Source \\
\midrule
Compact optimization (EVT on $[a,b]$) & $\equiv$ FT & Paper~23 \\
Action minimization (variational mechanics) & $\equiv$ FT & Paper~28 \\
\bottomrule
\end{tabular}
\end{center}

Both calibrations are tight equivalences: the physical statements are provably equivalent to FT over BISH\@.
The question is not whether these \emph{mathematical} statements cost FT---they do---but whether the \emph{physical predictions} they underwrite require these exact mathematical statements, or whether weaker (LPO-available) versions suffice for all empirical content.

\subsection{Strategy}

We distinguish between two claims:

\begin{itemize}
\item \textbf{Exact existence}: ``There exists a point $x^* \in [a,b]$ where $f$ attains its maximum.'' This costs FT\@.
\item \textbf{Approximate attainment}: ``For every $\varepsilon > 0$, there exists $x_\varepsilon \in [a,b]$ such that $f(x_\varepsilon) > \sup f - \varepsilon$.'' This costs at most LPO\@.
\end{itemize}

The physical content---every laboratory measurement, every finite-precision prediction---is captured by approximate attainment.
Exact existence is a mathematical strengthening that no experiment can verify or require.

For variational mechanics, the distinction is even sharper: the equations of motion (Euler--Lagrange) are BISH, and their solutions are the physical predictions.
The variational principle asserts that these solutions minimize the action; but this \emph{characterization} of solutions as minimizers is logically stronger than the solutions themselves.
Nature solves differential equations.
The variational principle is the physicist's interpretive framework, not nature's computational method.

\section{Background}

\subsection{The Fan Theorem}

The Fan Theorem (FT) states: every bar of a finitely branching tree with bounded height is uniform.
Equivalently, every pointwise continuous function on a compact metric space is uniformly continuous.
Its most familiar consequence is the Extreme Value Theorem (EVT): every continuous function $f : [a,b] \to \mathbb{R}$ attains its supremum.

FT is independent of the omniscience spine: it neither implies nor is implied by LPO\@.
It is also independent of DC$_\omega$ and MP\@.
Thus, if physical reality requires FT, it constitutes a genuinely independent logical commitment beyond LPO\@.

\subsection{BMC and LPO}

Bounded Monotone Convergence (BMC) states: every bounded monotone sequence of real numbers converges.
BMC is equivalent to LPO over BISH (Ishihara; Papers~8, 29).

BMC provides a mechanism for \emph{approximate} optimization: given a continuous function $f$ on $[a,b]$ and a sequence of increasingly fine grid approximations, the sequence of approximate maxima is bounded and (can be made) monotone, hence converges by BMC to the supremum.
This yields the supremum as a real number and, for each $\varepsilon > 0$, an approximate maximizer.
What BMC cannot provide without FT is a \emph{point} $x^* \in [a,b]$ where the supremum is attained.

\subsection{Paper~28: The variational mechanics calibration}

Paper~28 established a three-level stratification of classical mechanics:

\begin{center}
\begin{tabular}{@{}ll@{}}
\toprule
Result & Cost \\
\midrule
Euler--Lagrange equations (discrete) & BISH \\
Hamilton's equations (discrete) & BISH \\
Legendre transform (discrete) & BISH \\
Action minimizer existence & $\equiv$ FT \\
\bottomrule
\end{tabular}
\end{center}

The physical predictions---trajectories satisfying the equations of motion---are BISH\@.
The FT cost enters only through the assertion that solutions \emph{minimize} the action functional.

\section{Main Results}

\subsection{Theorem 1: Approximate optimization from LPO}

\begin{theorem}[Approximate EVT from BMC]\label{thm:approx-evt}
Over BISH+LPO: Let $f : [a,b] \to \mathbb{R}$ be uniformly continuous.
Then:
\begin{enumerate}
\item[(a)] The supremum $S = \sup_{x \in [a,b]} f(x)$ exists as a real number.
\item[(b)] For every $\varepsilon > 0$, there exists $x_\varepsilon \in [a,b]$ with $f(x_\varepsilon) > S - \varepsilon$.
\end{enumerate}
\end{theorem}

\begin{proof}
For each $n \geq 1$, define the $n$-grid $G_n = \{a + k(b-a)/n : 0 \leq k \leq n\}$ and set $M_n = \max_{x \in G_n} f(x)$.
Since $G_n$ is finite, $M_n$ is computable (BISH).

\textbf{Part (a).}
The sequence $(M_n)$ is bounded above (by continuity on a bounded domain) and can be replaced by $\tilde{M}_n = \max_{k \leq n} M_k$, which is monotone non-decreasing and bounded.
By BMC ($\equiv$ LPO), $\tilde{M}_n$ converges to a limit $S$.

We verify $S = \sup f$.
For any $x \in [a,b]$, by uniform continuity, for each $\varepsilon > 0$ there exists $\delta > 0$ such that $|x - y| < \delta \implies |f(x) - f(y)| < \varepsilon$.
For $n > (b-a)/\delta$, some grid point $y \in G_n$ satisfies $|x-y| < \delta$, so $f(x) < f(y) + \varepsilon \leq M_n + \varepsilon \leq S + \varepsilon$.
Since $\varepsilon$ is arbitrary, $f(x) \leq S$ for all $x$.
Conversely, $M_n \leq S$ and $M_n \leq \sup f$ trivially, while the grid points witness $S \leq \sup f$.
Thus $S = \sup f$.

\textbf{Part (b).}
Given $\varepsilon > 0$, by convergence of $\tilde{M}_n$ to $S$, there exists $N$ with $\tilde{M}_N > S - \varepsilon$.
The finite grid $G_N$ is searchable; let $x_\varepsilon$ be the grid point achieving $\tilde{M}_N$.
Then $f(x_\varepsilon) \geq \tilde{M}_N > S - \varepsilon$.
\end{proof}

\begin{remark}
Part (a) establishes the supremum.
Part (b) establishes approximate attainment.
What is \emph{missing} compared to the full EVT is the assertion that there exists $x^* \in [a,b]$ (not merely a grid point or a sequence of approximants) with $f(x^*) = S$.
Extracting this exact point from the approximating sequence requires a convergent subsequence argument on $[a,b]$---which is precisely the Fan Theorem (sequential compactness of $[a,b]$).
\end{remark}

\subsection{Theorem 2: Physical dispensability of exact attainment}

\begin{theorem}[Empirical equivalence]\label{thm:empirical-equiv}
Let $f : [a,b] \to \mathbb{R}$ be uniformly continuous.
For any measurement precision $\varepsilon > 0$, the following are empirically indistinguishable:
\begin{enumerate}
\item[(a)] (FT) There exists $x^* \in [a,b]$ with $f(x^*) = \sup f$.
\item[(b)] (LPO) There exists $x_\varepsilon \in [a,b]$ with $f(x_\varepsilon) > \sup f - \varepsilon$.
\end{enumerate}
Any physical measurement with finite precision $\varepsilon$ cannot distinguish a reading taken at $x^*$ from one taken at $x_\varepsilon$.
\end{theorem}

\begin{proof}
Any measurement apparatus has finite precision $\varepsilon > 0$.
A measurement at $x^*$ yields a value in $[f(x^*) - \varepsilon, f(x^*) + \varepsilon] = [S - \varepsilon, S + \varepsilon]$.
A measurement at $x_\varepsilon$ yields a value in $[f(x_\varepsilon) - \varepsilon, f(x_\varepsilon) + \varepsilon] \supseteq [S - 2\varepsilon, S + \varepsilon]$.
The intervals overlap.
No finite-precision measurement can determine whether the observed value came from the exact maximizer or the approximate one.
\end{proof}

\subsection{Theorem 3: Variational mechanics without FT}

\begin{theorem}[Equations of motion without variational minimization]\label{thm:variational}
Over BISH+LPO: Let $L(q, \dot{q}, t)$ be a discrete Lagrangian on a finite mesh.
Then:
\begin{enumerate}
\item[(a)] (BISH, Paper~28) The Euler--Lagrange equations $\frac{\partial L}{\partial q} - \frac{d}{dt}\frac{\partial L}{\partial \dot{q}} = 0$ are well-defined and their solutions are computable.
\item[(b)] (BISH, Paper~28) Hamilton's equations are equivalent to the Euler--Lagrange equations.
\item[(c)] (LPO) The action $S[\gamma] = \sum_k L(q_k, \dot{q}_k, t_k)$ along any trajectory is a well-defined real number, and for sequences of trajectories, the infimum of the action exists.
\item[(d)] (FT, Paper~28) Among all trajectories with fixed endpoints, there exists one that \emph{exactly} minimizes the action.
\end{enumerate}
\end{theorem}

\begin{proof}
Parts (a) and (b) are Paper~28's BISH results.

Part (c): The action along a fixed trajectory is a finite sum (BISH).
For a sequence of trajectories $\{\gamma_n\}$, the action values $\{S[\gamma_n]\}$ form a bounded-below sequence.
Define $I_n = \inf_{k \leq n} S[\gamma_k]$ (computable, finite minimum).
Then $(I_n)$ is monotone non-increasing and bounded below.
Negating, $(-I_n)$ is monotone non-decreasing and bounded above.
By BMC ($\equiv$ LPO), $(-I_n)$ converges, so $I_n$ converges to $\inf S$.

Part (d) is Paper~28's FT result: extracting a trajectory $\gamma^*$ with $S[\gamma^*] = \inf S$ requires compactness of the trajectory space.
\end{proof}

\begin{corollary}[Physical dispensability of action minimization]\label{cor:variational}
Every physical prediction of classical mechanics---specifically, every trajectory satisfying the equations of motion with specified initial or boundary conditions---is derivable in BISH without invoking action minimization.
The variational characterization (that solutions minimize the action) is a logically stronger assertion (FT) that adds no empirical content: no measurement can distinguish ``this trajectory satisfies $F = ma$'' from ``this trajectory minimizes the action.''
\end{corollary}

\begin{proof}
The equations of motion determine trajectories uniquely given initial conditions (BISH, Paper~28).
The variational principle asserts that these same trajectories are characterized as action minimizers.
This characterization is logically equivalent to asserting the existence of a minimizer of a continuous functional on a compact set, which costs FT\@.
But the trajectories themselves are already determined by the BISH-level equations.
The FT-level characterization is a \emph{theorem about} the solutions, not a method for \emph{obtaining} them.
\end{proof}

\subsection{Theorem 4: LPO separation from FT}

\begin{theorem}[Separation]\label{thm:separation}
The difference between the FT-level and LPO-level versions of optimization has no empirical content.
Formally: for every uniformly continuous $f : [a,b] \to \mathbb{R}$ and every $\varepsilon > 0$, define:
\begin{itemize}
\item $P_{\mathrm{FT}}$: ``there exists $x^*$ with $f(x^*) = \sup f$'' (FT).
\item $P_{\mathrm{LPO}}(\varepsilon)$: ``there exists $x_\varepsilon$ with $f(x_\varepsilon) > \sup f - \varepsilon$'' (LPO).
\end{itemize}
Then $P_{\mathrm{FT}} \implies P_{\mathrm{LPO}}(\varepsilon)$ for all $\varepsilon > 0$, and the converse fails (this is the FT/LPO gap).
However, $\{P_{\mathrm{LPO}}(\varepsilon) : \varepsilon > 0\}$ is empirically complete: no finite-precision observation can witness the gap.
\end{theorem}

\begin{proof}
The forward implication is trivial: take $x_\varepsilon = x^*$.
The failure of the converse is the content of FT $\not\Leftarrow$ LPO (independence).
Empirical completeness follows from Theorem~\ref{thm:empirical-equiv}: any measurement with precision $\varepsilon$ is satisfied by $P_{\mathrm{LPO}}(\varepsilon)$.
\end{proof}

\section{The Lean Formalization}

\subsection{Architecture}

The Lean~4 formalization follows the established architecture of the programme:

\begin{center}
\begin{tabular}{@{}lll@{}}
\toprule
File & Content & Classical content \\
\midrule
\texttt{Defs.lean} & Definitions of ApproxEVT, LPO, BMC & $\mathbb{R}$ infrastructure only \\
\texttt{GridApprox.lean} & Finite grid approximation, $M_n$ & None beyond $\mathbb{R}$ \\
\texttt{SupExists.lean} & Supremum existence from BMC & Intentional (BMC $\equiv$ LPO) \\
\texttt{ApproxAttain.lean} & $\varepsilon$-attainment from supremum & None beyond $\mathbb{R}$ \\
\texttt{Separation.lean} & Empirical equivalence theorem & None beyond $\mathbb{R}$ \\
\texttt{Variational.lean} & EL dispensability of FT & None beyond $\mathbb{R}$ \\
\texttt{Main.lean} & Master theorem & Inherits \\
\bottomrule
\end{tabular}
\end{center}

\subsection{Key definitions}

\begin{verbatim}
/-- Approximate EVT: supremum exists and is approximately attained -/
def ApproxEVT : Prop :=
  ∀ (f : ℝ → ℝ) (a b : ℝ) (hab : a < b),
    UniformContinuousOn f (Set.Icc a b) →
    ∃ S : ℝ, (∀ x ∈ Set.Icc a b, f x ≤ S) ∧
      ∀ ε > 0, ∃ x ∈ Set.Icc a b, f x > S - ε

/-- Full EVT: supremum is exactly attained (costs FT) -/
def ExactEVT : Prop :=
  ∀ (f : ℝ → ℝ) (a b : ℝ) (hab : a < b),
    UniformContinuousOn f (Set.Icc a b) →
    ∃ x ∈ Set.Icc a b, ∀ y ∈ Set.Icc a b, f y ≤ f x

/-- The separation: ExactEVT → ApproxEVT is trivial;
    ApproxEVT → ExactEVT requires FT -/
theorem approxEVT_of_lpo : LPO → ApproxEVT := by ...
theorem exactEVT_iff_ft : ExactEVT ↔ FT := by ...  -- Paper 23
\end{verbatim}

\subsection{Axiom profile}

\begin{verbatim}
approxEVT_of_lpo: [propext, Classical.choice, Quot.sound, bmc_of_lpo]
  -- 1 cited axiom: forward BMC from LPO
separation_theorem: [propext, Classical.choice, Quot.sound]
  -- No custom axioms: pure constructive reasoning
variational_dispensability: [propext, Classical.choice, Quot.sound]
  -- No custom axioms: pure constructive reasoning
\end{verbatim}

\subsection{Certification}

\begin{itemize}
\item \texttt{approxEVT\_of\_lpo}: Certification Level~3 (intentional classical content---uses LPO via BMC by design).
\item \texttt{separation\_theorem}: Certification Level~2 (Classical.choice from $\mathbb{R}$ infrastructure only).
\item \texttt{variational\_dispensability}: Certification Level~2 (structural verification; no proof-level classical reasoning).
\end{itemize}

\section{Discussion}

\subsection{What FT captures and what LPO captures}

The Fan Theorem and LPO capture fundamentally different mathematical operations:

FT captures \emph{compactness}---the passage from pointwise properties to uniform properties, from approximate information to exact existence.
LPO captures \emph{convergence}---the passage from finite approximations to completed limits.

For optimization, this distinction manifests as follows.
LPO (via BMC) can establish that the supremum \emph{exists as a real number} and can be \emph{approximated to arbitrary precision}.
FT is needed to assert that the supremum is \emph{attained at a specific point}.

Physically, the distinction is between ``the system approaches an optimum'' (LPO) and ``the system is at the optimum'' (FT).
No finite measurement can distinguish these.

\subsection{The variational principle as interpretation, not mechanism}

The variational formulation of mechanics---Hamilton's principle, Lagrangian mechanics, the principle of least action---is often presented as a fundamental law of nature.
Our analysis reveals a more nuanced picture.

The equations of motion (Newton's second law, Euler--Lagrange equations, Hamilton's equations) are BISH\@.
They are computationally transparent: given initial conditions, the trajectory is determined by finite arithmetic at each step.

The variational \emph{characterization}---that the trajectory minimizes (or extremizes) the action---is a theorem \emph{about} the solutions, asserting a global property (minimality among all competing trajectories) that costs FT\@.

Nature solves the local differential equation.
The physicist observes that the resulting trajectory happens to minimize a global functional.
The minimization characterization is explanatorily powerful and mathematically beautiful, but it is logically dispensable for predictions.
This is the precise sense in which FT is physically dispensable: it underwrites an \emph{interpretation} of the physics, not the physics itself.

\subsection{Implications for BISH+LPO sufficiency}

With the FT branch shown to be physically dispensable, the status of the calibration table's logical branches is:

\begin{center}
\begin{tabular}{@{}lll@{}}
\toprule
Branch & Physical status & Covered by LPO? \\
\midrule
Omniscience spine (LLPO, WLPO, LPO) & Physically instantiated & Yes (LPO $\implies$ WLPO $\implies$ LLPO) \\
Markov's Principle (MP) & Physically instantiated & Yes (LPO $\implies$ MP) \\
Fan Theorem (FT) & Physically dispensable & N/A (dispensable) \\
Choice axis (CC, DC) & Open (Paper~31) & CC yes; DC open \\
\bottomrule
\end{tabular}
\end{center}

If Paper~31 establishes the physical dispensability of DC (beyond what CC, which LPO implies, provides), then BISH+LPO is the complete logical constitution of empirically accessible physics across all twelve calibrated domains.

\subsection{Relation to Paper~29}

Paper~29 established that LPO is physically instantiated via the Fekete equivalence and the reality of phase transitions.
The present paper establishes that FT is \emph{not} independently required.
Together, they narrow the question ``what is the logical constitution of the universe?'' from ``BISH + some subset of \{LPO, FT, DC, \ldots\}'' to ``BISH + LPO + possibly DC.''

\section{Conclusion}

The Fan Theorem is mathematically genuine: compact optimization and variational action minimization really do cost FT, and these calibrations (Papers~23, 28) stand.
But the physical content of both results is recoverable in BISH+LPO\@.
Approximate optimization (to any finite precision) requires only BMC, which is LPO\@.
The equations of motion require only BISH\@.
The FT-level assertions---exact attainment of the supremum, existence of an action-minimizing trajectory---are mathematically stronger than what any finite experiment can verify or require.

FT is the mapmaker's convention.
LPO is the territory.

The question of whether BISH+LPO is the complete logical constitution of empirically accessible physics now rests entirely on the status of dependent choice, which Paper~31 will address.

\bigskip

\noindent\textbf{Acknowledgments and Statement of AI Use.} As with all papers in this programme, the Lean~4 formalizations and \LaTeX{} manuscript were developed with substantial assistance from Claude (Opus~4.6), an AI assistant by Anthropic. Paper~29 documents the collaborative methodology.

\bigskip

\noindent\textbf{Data availability.} Lean~4 source code will be archived at Zenodo upon completion.

\begin{thebibliography}{99}

\bibitem{Berger2005}
Berger, J. ``The Fan Theorem and uniform continuity.'' In \emph{New Computational Paradigms (CiE 2005)}, LNCS 3526, pp.~18--22. Springer, 2005.

\bibitem{BridgesVita2006}
Bridges, D. and V\^{\i}\c{t}\u{a}, L.~S. \emph{Techniques of Constructive Analysis}. Springer, 2006.

\bibitem{Ishihara2006}
Ishihara, H. ``Reverse mathematics in Bishop's constructive mathematics.'' \emph{Philosophia Scientiae}, Cahier sp\'ecial 6 (2006): 43--59.

\bibitem{Lee2026_P8}
Lee, P.~C.-K. ``The constructive cost of the thermodynamic limit.'' 2026. Paper~8. DOI: 10.5281/zenodo.18516813.

\bibitem{Lee2026_P10}
Lee, P.~C.-K. ``The logical geography of mathematical physics.'' 2026. Paper~10.

\bibitem{Lee2026_P23}
Lee, P.~C.-K. ``The Fan Theorem and the constructive cost of optimization.'' 2026. Paper~23. DOI: 10.5281/zenodo.18604312.

\bibitem{Lee2026_P28}
Lee, P.~C.-K. ``Newton vs.\ Lagrange vs.\ Hamilton: constructive stratification of classical mechanics.'' 2026. Paper~28. DOI: 10.5281/zenodo.18616620.

\bibitem{Lee2026_P29}
Lee, P.~C.-K. ``Fekete's Subadditive Lemma is equivalent to LPO.'' 2026. Paper~29.

\end{thebibliography}

\end{document}

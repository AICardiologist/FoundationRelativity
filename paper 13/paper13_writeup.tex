\documentclass[11pt,a4paper]{article}

% ====================================================================
% Packages
% ====================================================================
\usepackage[utf8]{inputenc}
\usepackage[T1]{fontenc}
\usepackage{amsmath,amssymb,amsthm}
\usepackage{mathtools}
\usepackage{hyperref}
\usepackage[margin=1in]{geometry}
\usepackage{enumitem}
\usepackage{booktabs}
\usepackage{listings}
\usepackage{xcolor}
\usepackage{cleveref}
\usepackage{natbib}
\usepackage{mdframed}

% ====================================================================
% Theorem environments
% ====================================================================
\theoremstyle{plain}
\newtheorem{theorem}{Theorem}[section]
\newtheorem{lemma}[theorem]{Lemma}
\newtheorem{proposition}[theorem]{Proposition}
\newtheorem{corollary}[theorem]{Corollary}

\theoremstyle{definition}
\newtheorem{definition}[theorem]{Definition}
\newtheorem{remark}[theorem]{Remark}

% ====================================================================
% Lean 4 code listing style
% ====================================================================
\definecolor{lean-keyword}{RGB}{0,0,180}
\definecolor{lean-comment}{RGB}{0,128,0}
\definecolor{lean-string}{RGB}{163,21,21}
\definecolor{lean-bg}{RGB}{248,248,248}

\lstdefinelanguage{lean4}{
  keywords={theorem,lemma,def,class,instance,import,open,variable,
            noncomputable,section,namespace,end,where,let,have,show,
            intro,obtain,use,exact,rw,simp,apply,by,fun,match,if,
            then,else,do,return,axiom,abbrev,private,attribute,
            suffices,change,congr,ext,constructor,rintro,push_neg,
            linarith,absurd,set_option,omit,in,set,cases,structure,
            refine,unfold,rcases},
  sensitive=true,
  morecomment=[l]{--},
  morecomment=[s]{/-}{-/},
  morestring=[b]",
  morestring=[b]',
}

\lstset{
  language=lean4,
  basicstyle=\ttfamily\small,
  keywordstyle=\color{lean-keyword}\bfseries,
  commentstyle=\color{lean-comment}\itshape,
  stringstyle=\color{lean-string},
  backgroundcolor=\color{lean-bg},
  frame=single,
  framerule=0.5pt,
  breaklines=true,
  breakatwhitespace=true,
  tabsize=2,
  showstringspaces=false,
  numbers=left,
  numberstyle=\tiny\color{gray},
  numbersep=5pt,
  xleftmargin=15pt,
  captionpos=b,
}

% ====================================================================
% Macros
% ====================================================================
\newcommand{\NN}{\mathbb{N}}
\newcommand{\RR}{\mathbb{R}}
\newcommand{\ZZ}{\mathbb{Z}}
\newcommand{\LPO}{\mathrm{LPO}}
\newcommand{\WLPO}{\mathrm{WLPO}}
\newcommand{\LLPO}{\mathrm{LLPO}}
\newcommand{\BMC}{\mathrm{BMC}}
\newcommand{\BISH}{\mathrm{BISH}}
\newcommand{\Lean}{\textsc{Lean~4}}
\newcommand{\Mathlib}{\textsc{Mathlib4}}
\newcommand{\leanok}{\textsf{\small \textcolor{green!70!black}{\checkmark}}}
\newcommand{\leanpartial}{\textsf{\small \textcolor{orange!80!black}{(partial)}}}

% ====================================================================
% Title
% ====================================================================
\title{%
  \textbf{The Event Horizon as a Logical Boundary:}\\[6pt]
  Schwarzschild Interior Geodesic Incompleteness\\
  and LPO in Lean~4\\[6pt]
  {\normalsize Paper~13 in the Constructive Reverse Mathematics Series}%
}

\author{
  Paul Chun-Kit Lee\thanks{%
    New York University.
    AI-assisted formalization; see \S\ref{sec:ai} for methodology.
    The author is a medical professional, not a domain expert in
    constructive mathematics or general relativity; mathematical
    content was developed with extensive AI assistance.} \\
  New York University \\
  \texttt{dr.paul.c.lee@gmail.com}
}

\date{February 2026}

% ====================================================================
\begin{document}
\maketitle

% ====================================================================
\begin{abstract}
We formalize in \Lean{} a decomposition of Schwarzschild interior
physics into constructive content layers.
%
The explicit cycloid geodesic $r(\eta) = M(1 + \cos\eta)$ reaches
$r = 0$ at proper time $\tau^* = \pi M$ constructively; the
Kretschmann scalar $K = 48M^2/r^6$ is constructively computable
for any $r > 0$; and for any $\varepsilon > 0$, there exists an
explicit $\eta$ with $r(\eta) < \varepsilon$. All of this is
$\BISH$ (Height~0).
%
The abstract principle that \emph{every} bounded monotone decreasing
sequence in $[0, 2M)$ converges to a definite real limit---the
completed-limit formulation of geodesic incompleteness---is
equivalent to the Limited Principle of Omniscience ($\LPO$).
%
The event horizon at $r = 2M$ thus demarcates not only the causal
boundary from which light cannot escape, but a logical boundary:
the exterior geometry (Paper~1) and the interior's finite-time
physics are $\BISH$, while the singularity assertion as a
completed-limit principle requires exactly $\LPO$.
%
The formalization comprises 1{,}021 lines across 8~modules with
zero \texttt{sorry} statements. One interface assumption
(\texttt{bmc\_of\_lpo}, the Bridges--V\^{\i}\c{t}\u{a} equivalence
imported from Paper~8) is axiomatized with citation.
The \texttt{Classical.choice} in the axiom profile arises from
\Mathlib{} infrastructure; the constructive calibration is
established by proof-content analysis (see \S\ref{sec:certification}
and Paper~10).
\end{abstract}

\tableofcontents

% ====================================================================
\section{Introduction}\label{sec:intro}
% ====================================================================

\subsection{Physical Context}\label{sec:physical}

The Schwarzschild solution describes the geometry of a non-rotating,
uncharged black hole of mass $M$. The event horizon at $r = 2M$ is
the boundary of the region from which future-directed causal curves
can reach infinity. Inside the horizon, the radial coordinate $r$
becomes timelike: the metric component $g_{rr}$ changes sign, and a
freely falling observer necessarily moves toward decreasing~$r$.
The observer reaches the curvature singularity at $r = 0$ in finite
proper time.

This ``geodesic incompleteness''---the fact that timelike geodesics
terminate at the singularity---is the physical content that the
Penrose singularity theorem~\cite{Pen65} generalizes to arbitrary
spacetimes satisfying energy conditions. For the Schwarzschild
interior, the explicit solution is available: a radially infalling
particle dropped from rest at the horizon follows the cycloid
\begin{equation}\label{eq:cycloid}
  r(\eta) = M(1 + \cos\eta), \qquad
  \tau(\eta) = M(\eta + \sin\eta), \qquad
  \eta \in [0, \pi],
\end{equation}
reaching $r = 0$ at proper time $\tau^* = \pi M$.

\subsection{The CRM Question}\label{sec:crm-question}

From the standpoint of constructive reverse mathematics (CRM), the
question is: what is the logical cost of asserting geodesic
incompleteness?

The answer turns out to be more subtle than a single classification.
It decomposes into two layers:
\begin{itemize}
  \item The \textbf{specific} cycloid geodesic reaching $r = 0$ is
    constructively computable ($\BISH$). The endpoint is given by
    explicit trigonometric evaluation:
    $r(\pi) = M(1 + \cos\pi) = 0$.
  \item The \textbf{abstract principle} that every bounded monotone
    decreasing sequence in the interior converges to a definite real
    limit---the completed-limit formulation that underlies the general
    assertion of geodesic incompleteness---is equivalent to $\LPO$.
\end{itemize}

This decomposition mirrors Paper~8's treatment of the 1D Ising
model~\cite{Lee26-P8}, where finite-size bounds are $\BISH$ but the
thermodynamic limit (a completed-limit assertion about the free
energy sequence) costs $\LPO$.

\subsection{Contributions}\label{sec:contributions}

\begin{enumerate}
  \item Machine-verified proof that
    \texttt{SchwarzschildInteriorGeodesicIncompleteness} $\leftrightarrow$
    $\LPO$ (1{,}021 lines of \Lean{}, zero \texttt{sorry}).
  \item Explicit $\BISH$ content: cycloid computability, Kretschmann
    scalar divergence, constructive approaching of the
    singularity---all Height~0.
  \item The event horizon as a logical boundary: the first result
    calibrating a general-relativistic singularity assertion in the
    constructive hierarchy.
  \item Connection to Paper~8 via $\BMC \leftrightarrow \LPO$,
    extending the dispensability--calibration pattern from statistical
    mechanics to gravitation.
\end{enumerate}

\subsection{Related Work}\label{sec:related}

Paper~1~\cite{Lee26-P1} established that the Schwarzschild
\emph{exterior} geometry---metric components, Christoffel symbols,
Riemann tensor, Ricci flatness, Kretschmann scalar---is $\BISH$
(Height~0). Paper~8~\cite{Lee26-P8} proved that bounded monotone
convergence, instantiated through the 1D Ising free energy, is
equivalent to $\LPO$. The equivalence $\BMC \leftrightarrow \LPO$
itself is due to Bridges and V\^{\i}\c{t}\u{a}~\cite{BV06}
(Theorem~2.1.5).

To our knowledge, no prior work applies constructive reverse
mathematics to general-relativistic singularities. Echenim and
Mhalla~\cite{EM24} formalized the CHSH inequality in Isabelle/HOL,
but in a classical setting without constructive analysis.


% ====================================================================
\section{Mathematical Content}\label{sec:math}
% ====================================================================

\subsection{The Schwarzschild Interior}\label{sec:interior}

The Schwarzschild metric in standard coordinates is
\[
  ds^2 = -f(M,r)\,dt^2 + f(M,r)^{-1}\,dr^2 + r^2\,d\Omega^2,
  \qquad f(M,r) = 1 - \frac{2M}{r}.
\]
For $r > 2M$ (exterior), $f > 0$ and $t$ is timelike. For
$0 < r < 2M$ (interior), $f < 0$: the roles of $t$ and $r$ swap,
making $r$ a timelike coordinate and $t$ a spacelike coordinate.
This signature flip means that a freely falling observer in the
interior cannot remain at constant~$r$; the radial coordinate
necessarily decreases.

\begin{definition}[Interior domain]\label{def:interior} \leanok{}
The \emph{interior domain} is the set of pairs $(M, r)$ satisfying
$M > 0$, $r > 0$, and $r < 2M$. Equivalently, $f(M,r) < 0$.
\end{definition}

\subsection{The Cycloid Geodesic ($\BISH$)}\label{sec:cycloid}

For a particle dropped from rest at the event horizon with specific
energy $E = 1$, the radial geodesic equation has the closed-form
cycloid solution~\eqref{eq:cycloid}. This explicit parametrization
avoids formalizing ODE existence or uniqueness theory.

\begin{theorem}[Cycloid properties]\label{thm:cycloid} \leanok{}
For $M > 0$ and $\eta \in [0, \pi]$:
\begin{enumerate}[label=(\alph*)]
  \item $r(0) = 2M$ (starts at the horizon) and
    $r(\pi) = 0$ (reaches the singularity).
  \item $\tau(0) = 0$ and $\tau(\pi) = \pi M$ (finite proper time).
  \item $r$ is strictly decreasing on $[0, \pi]$:
    $r'(\eta) = -M\sin\eta < 0$ for $\eta \in (0, \pi)$.
  \item $\tau$ is strictly increasing on $[0, \pi]$:
    $\tau'(\eta) = M(1 + \cos\eta) > 0$ for $\eta \in (0, \pi)$.
  \item For $\eta \in (0, \pi)$: $0 < r(\eta) < 2M$ (lies in the
    interior).
\end{enumerate}
\end{theorem}

\begin{proof}
All properties follow from direct computation using standard
trigonometric identities. Part~(a): $r(0) = M(1+1) = 2M$ and
$r(\pi) = M(1+(-1)) = 0$. Part~(b): $\tau(0) = M(0+0) = 0$ and
$\tau(\pi) = M(\pi+0) = \pi M$. Part~(c): the derivative
$r'(\eta) = -M\sin\eta$ is negative for $\eta \in (0,\pi)$ since
$\sin\eta > 0$ on this interval. Part~(d): $\tau'(\eta) =
M(1+\cos\eta) > 0$ since $\cos\eta > -1$ for
$\eta \in (0, \pi)$. Part~(e): $\cos\eta \in (-1, 1)$ for
$\eta \in (0, \pi)$, so $r(\eta) = M(1+\cos\eta) \in (0, 2M)$.
\end{proof}

\begin{theorem}[Constructive approaching]\label{thm:approaching} \leanok{}
For any $M > 0$ and $\varepsilon > 0$, there exists
$\eta \in (0, \pi)$ such that $r(\eta) < \varepsilon$.
\end{theorem}

\begin{proof}
Since $r$ is continuous and $r(\pi) = 0$, by continuity at $\pi$,
for any $\varepsilon > 0$ there exists $\delta > 0$ such that
$|r(\eta) - 0| < \varepsilon$ whenever $|\eta - \pi| < \delta$.
Choosing $\eta_0 = \pi - \min(\delta/2, \pi/2) \in (0, \pi)$ gives
$r(\eta_0) < \varepsilon$.
\end{proof}

\begin{remark}[BISH status]\label{rem:bish}
All results in this subsection are $\BISH$ (Height~0). The cycloid
is an explicit computable function; no omniscience principle is
needed. In particular, \emph{reaching the singularity along the
specific cycloid is constructive}. The LPO cost arises elsewhere.
\end{remark}

\subsection{The Completed-Limit Formulation}\label{sec:completed}

The key insight is to formulate geodesic incompleteness not as a
statement about the specific cycloid solution, but as a universal
principle about all infalling trajectories.

\begin{definition}[Schwarzschild Interior Geodesic Incompleteness]
\label{def:incompleteness} \leanok{}
\emph{SchwarzschildInteriorGeodesicIncompleteness} is the
assertion: for every $M > 0$ and every sequence $a : \NN \to \RR$
satisfying
\begin{enumerate}[label=(\roman*)]
  \item $a$ is antitone (non-increasing),
  \item $a(n) \geq 0$ for all $n$,
  \item $a(0) < 2M$,
\end{enumerate}
there exists $L \in \RR$ such that $a$ converges to $L$: for every
$\varepsilon > 0$, there exists $N_0$ with $|a(N) - L| < \varepsilon$
for all $N \geq N_0$.
\end{definition}

This is the universally quantified completed-limit assertion. It
says: any monotone bounded sequence in the interior has a definite
limit. This captures the logical content of ``every radially
infalling trajectory has a definite endpoint.''

\begin{remark}[On the formulation]\label{rem:formulation}
In classical general relativity, geodesic incompleteness \emph{is} a
completed-limit assertion about monotone sequences. The radial
coordinate along any infalling timelike geodesic is a monotone
decreasing function of proper time, bounded below by~$0$. The
assertion ``the spacetime is geodesically incomplete'' means these
sequences converge to definite limits.

Our formalization strips away the differential geometry---we do not
formalize the Lorentzian metric, parallel transport, or the geodesic
equation as an ODE---and isolates the exact logical content: bounded
monotone convergence for decreasing sequences in $[0, 2M)$.
This extraction of logical content from physical assertion is the
standard methodology of constructive reverse mathematics.
\end{remark}

\begin{theorem}[Main Theorem]\label{thm:main} \leanok{}
\[
  \textit{SchwarzschildInteriorGeodesicIncompleteness}
  \;\;\longleftrightarrow\;\; \LPO.
\]
\end{theorem}

\subsection{Forward Direction: Incompleteness $\Rightarrow$ LPO}
\label{sec:forward}

Given an arbitrary binary sequence $\alpha : \NN \to \{0,1\}$, we
construct a monotone bounded sequence whose limit encodes whether
$\alpha$ is identically zero or has a true term.

\begin{definition}[Geodesic coupling]\label{def:coupling} \leanok{}
Fix $M > 0$. Define
\[
  a(n) := \texttt{geodesicCoupling}\;\alpha\;M\;0\;n
  = \begin{cases}
    M & \text{if } \texttt{runMax}\;\alpha\;n = \texttt{false}, \\
    0 & \text{if } \texttt{runMax}\;\alpha\;n = \texttt{true},
  \end{cases}
\]
where $\texttt{runMax}\;\alpha\;n = \max(\alpha(0), \ldots, \alpha(n))$.
\end{definition}

This sequence takes values in $\{M, 0\}$ and is antitone (it can
only jump from $M$ to $0$, never back). It is non-negative, and
$a(0) \leq M < 2M$, so it satisfies the hypotheses of
\Cref{def:incompleteness}.

\begin{proof}[Proof of the forward direction]
Fix $M = 1$ (any $M > 0$ works). Apply
\textit{SchwarzschildInteriorGeodesicIncompleteness} to the coupling
sequence $a$ to obtain a limit~$L$. Let $N_1$ be the convergence
index for $\varepsilon = M/2$, so $|a(N_1) - L| < M/2$.

\medskip\noindent\textbf{Case split on}
$\texttt{runMax}\;\alpha\;N_1$ (a Bool, so definitionally
decidable---no real-number comparison needed):

\smallskip\noindent\textbf{Case $\texttt{runMax}\;\alpha\;N_1 =
\texttt{true}$:} There exists $k \leq N_1$ with
$\alpha(k) = \texttt{true}$. We have $\exists n, \alpha(n) =
\texttt{true}$.

\smallskip\noindent\textbf{Case $\texttt{runMax}\;\alpha\;N_1 =
\texttt{false}$:} Then $a(N_1) = M$. Suppose for contradiction that
$\exists n_0, \alpha(n_0) = \texttt{true}$. Then the sequence is
eventually~$0$, so $L = 0$. But $|M - 0| = M$, while the
convergence modulus gives $|a(N_1) - L| < M/2$, yielding $M < M/2$,
a contradiction. Therefore $\forall n, \alpha(n) = \texttt{false}$.

\medskip\noindent
In both cases, we have decided: either
$\forall n, \alpha(n) = \texttt{false}$ or
$\exists n, \alpha(n) = \texttt{true}$. This is $\LPO$.
\end{proof}

The gap $\delta = M - 0 = M > 0$ between the two possible limit
values ($M$ when $\alpha \equiv \texttt{false}$, and $0$ when
$\exists n, \alpha(n) = \texttt{true}$) serves as the
``decision amplifier.''

\subsection{Reverse Direction: LPO $\Rightarrow$ Incompleteness}
\label{sec:reverse}

\begin{proof}[Proof of the reverse direction]
$\LPO$ implies $\BMC$ by the Bridges--V\^{\i}\c{t}\u{a}
equivalence~\cite{BV06} (axiomatized as \texttt{bmc\_of\_lpo}
from Paper~8).

Given an antitone sequence $a$ with $a(n) \geq 0$ and
$a(0) < 2M$, define $b(n) = -a(n)$. Then $b$ is monotone
(non-decreasing) and bounded above by~$0$. By $\BMC$, $b$
converges to some limit $L_{\mathrm{neg}}$. Then $a$ converges
to $-L_{\mathrm{neg}}$.
\end{proof}

\subsection{The Honest Decomposition}\label{sec:decomposition}

The paper does \emph{not} claim ``computing the cycloid endpoint
costs $\LPO$.'' The cycloid is $\BISH$. The paper claims:

\begin{center}
\begin{tabular}{@{}llp{5cm}@{}}
\toprule
\textbf{Content} & \textbf{Principle} & \textbf{Certification} \\
\midrule
Cycloid $r(\pi) = 0$ & $\BISH$ & Height~0 (machine-verified) \\
Cycloid approaching: $\forall\varepsilon > 0,\;\exists\eta,\;
  r(\eta) < \varepsilon$ & $\BISH$ & Height~0 (machine-verified) \\
Kretschmann $K = 48M^2/r^6$ computable & $\BISH$ &
  Height~0 (machine-verified) \\
``Every antitone bounded sequence in $[0,2M)$ has a limit'' &
  $\LPO$ & $\leftrightarrow$ equivalence (machine-verified) \\
\bottomrule
\end{tabular}
\end{center}

\noindent
The event horizon separates the $\BISH$ exterior (Paper~1) from the
interior where the \emph{general completed-limit principle} costs
$\LPO$. The specific solution is constructive; the universal
assertion is not.


% ====================================================================
\section{Lean Formalization}\label{sec:lean}
% ====================================================================

\subsection{Architecture}\label{sec:architecture}

The formalization is organized as a single \Lean{} project with
8~modules:

\begin{table}[ht]
\centering
\begin{tabular}{@{}lrl@{}}
\toprule
\textbf{Module} & \textbf{Lines} & \textbf{Content} \\
\midrule
\texttt{Basic.lean}          & 168 & LPO, BMC, Interior,
  \texttt{runMax} + lemmas \\
\texttt{RadialGeodesic.lean} & 250 & Cycloid parametrization,
  monotonicity, approaching \\
\texttt{Incompleteness.lean} & 167 &
  \texttt{SchwarzschildInteriorGeodesicIncompleteness} + coupling \\
\texttt{LPO\_Forward.lean}   & 91  & $\to$ direction via gap
  encoding \\
\texttt{LPO\_Reverse.lean}   & 57  & $\leftarrow$ direction via
  BMC \\
\texttt{BISH\_Content.lean}  & 128 & Kretschmann scalar, cycloid
  computability \\
\texttt{Certificates.lean}   & 85  & \texttt{\#print axioms} audit \\
\texttt{Main.lean}           & 75  & Assembly \\
\midrule
\textbf{Total}               & \textbf{1{,}021} & \\
\bottomrule
\end{tabular}
\caption{Module structure of Paper~13.}
\label{tab:modules}
\end{table}

\subsection{Key Design Decisions}\label{sec:design}

\paragraph{Cycloid-first approach.}
Rather than formalizing the geodesic equation as an ODE (which would
require Lean ODE infrastructure that does not yet exist in
\Mathlib{}), we work with the closed-form cycloid solution directly.
Monotonicity, boundedness, and convergence properties follow from
explicit trigonometric identities. This is a pragmatic choice:
the cycloid is the physically relevant solution for $E = 1$, and
its properties are all that the forward direction requires.

\paragraph{Abstract incompleteness.}
\texttt{SchwarzschildInteriorGeodesicIncompleteness} quantifies over
\emph{all} antitone non-negative sequences starting below $2M$, not
over solutions of the geodesic equation. This is the correct level
of abstraction for CRM: the logical content is the completed-limit
principle, and the geodesic equation is the physical motivation for
why such sequences arise. The Brouwerian counterexample
(\texttt{geodesicCoupling}) is not a geodesic---it is a valid
element of the domain of the universally quantified proposition.
This is standard CRM methodology.

\paragraph{BMC import.}
The \texttt{bmc\_of\_lpo} axiom imports the Bridges--V\^{\i}\c{t}\u{a}
equivalence~\cite{BV06} from Paper~8. This is the same interface
assumption used in Paper~8 and documented in the series methodology
(Paper~10).

\subsection{Core Definitions}\label{sec:core-defs}

\begin{lstlisting}[caption={Core definitions (Basic.lean, excerpts).}]
/-- Limited Principle of Omniscience. -/
def LPO : Prop :=
  forall (a : Nat -> Bool),
    (forall n, a n = false) ||| (exists n, a n = true)

/-- Bounded Monotone Convergence. -/
def BMC : Prop :=
  forall (a : Nat -> Real) (M : Real),
    Monotone a -> (forall n, a n <= M) ->
    exists L : Real, forall e : Real, 0 < e ->
      exists N0 : Nat, forall N : Nat,
        N0 <= N -> |a N - L| < e

/-- LPO implies BMC [Bridges-Vita 2006]. -/
axiom bmc_of_lpo : LPO -> BMC
\end{lstlisting}

\begin{lstlisting}[caption={Interior domain and Schwarzschild factor (Basic.lean).}]
/-- Schwarzschild factor f(M, r) = 1 - 2M/r. -/
noncomputable def f (M r : Real) : Real := 1 - 2 * M / r

/-- Interior domain: 0 < r < 2M, M > 0. -/
structure Interior (M r : Real) : Prop where
  mass_pos : M > 0
  r_pos : r > 0
  r_inside : r < 2 * M
\end{lstlisting}

\begin{lstlisting}[caption={Cycloid definitions (RadialGeodesic.lean).}]
noncomputable def r_cycloid (M n : Real) : Real :=
  M * (1 + cos n)

noncomputable def t_cycloid (M n : Real) : Real :=
  M * (n + sin n)
\end{lstlisting}

\begin{lstlisting}[caption={Geodesic incompleteness and coupling (Incompleteness.lean).}]
def SchwarzschildInteriorGeodesicIncompleteness : Prop :=
  forall (M : Real), M > 0 ->
  forall (a : Nat -> Real),
    Antitone a ->
    (forall n, 0 <= a n) ->
    a 0 < 2 * M ->
    exists L : Real,
      forall e > 0, exists N0 : Nat,
        forall N >= N0, |a N - L| < e

noncomputable def geodesicCoupling
    (a : Nat -> Bool) (v0 v1 : Real) (n : Nat) : Real :=
  if runMax a n then v1 else v0
\end{lstlisting}

\subsection{Main Theorems}\label{sec:main-thms}

\begin{lstlisting}[caption={Forward direction (LPO\_Forward.lean, complete).}]
theorem geodesic_incompleteness_implies_lpo
    (hGI : SchwarzschildInteriorGeodesicIncompleteness) :
    LPO := by
  intro a
  set M : Real := 1
  have hM : M > 0 := one_pos
  set a := geodesicCoupling a M 0 with ha_def
  have h_anti : Antitone a :=
    geodesicCoupling_antitone a (le_of_lt hM)
  have h_nn : forall n, 0 <= a n :=
    geodesicCoupling_nonneg a (le_of_lt hM) (le_refl 0)
  have h_a0 : a 0 < 2 * M := by
    have := geodesicCoupling_le a (le_of_lt hM) 0; linarith
  obtain <<L, hL>> := hGI M hM a h_anti h_nn h_a0
  obtain <<N1, hN1>> := hL (M / 2) (by linarith)
  have hN1_self := hN1 N1 (le_refl _)
  cases hm : runMax a N1
  . -- runMax = false: prove all n, a n = false
    left
    apply bool_not_exists_implies_all_false
    intro <<n0, hn0>>
    have hL_val := limit_of_exists_true a M 0 hL hn0
    have haN1 : a N1 = M := by
      simp only [ha_def, geodesicCoupling, hm,
        Bool.false_eq_true, reduce_ite]
    rw [haN1, hL_val] at hN1_self
    simp at hN1_self
    rw [abs_of_pos hM] at hN1_self
    linarith
  . -- runMax = true: extract witness
    right
    obtain <<k, _, hk>> :=
      runMax_witness a (show runMax a N1 = true from hm)
    exact <<k, hk>>
\end{lstlisting}

\begin{lstlisting}[caption={Reverse direction (LPO\_Reverse.lean, complete).}]
theorem lpo_implies_geodesic_incompleteness (hLPO : LPO) :
    SchwarzschildInteriorGeodesicIncompleteness := by
  intro M _hM a ha hnn _ha0
  have hBMC := bmc_of_lpo hLPO
  have hMono : Monotone (fun n => -a n) :=
    fun m n hmn => by simp only [neg_le_neg_iff]; exact ha hmn
  have hBound : forall n, (fun n => -a n) n <= 0 :=
    fun n => by simp only [neg_nonpos]; exact hnn n
  obtain <<L_neg, hL>> :=
    hBMC (fun n => -a n) 0 hMono hBound
  refine <<-L_neg, fun e he => ?_>>
  obtain <<N0, hN0>> := hL e he
  exact <<N0, fun N hN => by
    have := hN0 N hN
    rwa [show a N - (-L_neg) = -((-a N) - L_neg)
      from by ring, abs_neg]>>
\end{lstlisting}

\begin{lstlisting}[caption={Main theorem assembly (Main.lean).}]
theorem schwarzschild_interior_geodesic_incompleteness_iff_LPO :
    SchwarzschildInteriorGeodesicIncompleteness <-> LPO :=
  <<geodesic_incompleteness_implies_lpo,
   lpo_implies_geodesic_incompleteness>>
\end{lstlisting}

\subsection{Axiom Audit}\label{sec:axioms}

The \texttt{Certificates.lean} module audits the axiom profile of
each theorem via \texttt{\#print axioms}:

\begin{lstlisting}[caption={Axiom audit (Certificates.lean, selected).}]
#print axioms
  schwarzschild_interior_geodesic_incompleteness_iff_LPO
-- [propext, Classical.choice, Quot.sound, bmc_of_lpo]

#print axioms r_cycloid_at_pi
-- [propext, Classical.choice, Quot.sound]

#print axioms bish_content_complete
-- [propext, Classical.choice, Quot.sound]
\end{lstlisting}

\noindent
The \texttt{Classical.choice} appearing in the $\BISH$ results
(cycloid, Kretschmann) arises from \Mathlib{}'s real number
infrastructure---specifically, the construction of $\RR$ via
Cauchy completion, which pervasively uses \texttt{Classical.choice}
as a metatheoretic convenience. The mathematical content of these
proofs is constructive: they involve only explicit trigonometric
computation on computable real numbers. The constructive calibration
is established by proof-content analysis, following the methodology
described in Paper~10.


% ====================================================================
\section{The Event Horizon as a Logical Boundary}\label{sec:boundary}
% ====================================================================

\subsection{The Causal Boundary}\label{sec:causal}

In classical GR, the event horizon at $r = 2M$ is the boundary of
the causal past of future null infinity~$\mathcal{I}^+$. Light
signals emitted from $r < 2M$ cannot reach distant observers. This
is a statement about the global causal structure of the
Schwarzschild spacetime: future-directed null geodesics from the
interior are trapped.

\subsection{The Logical Boundary}\label{sec:logical}

Our result reveals a second boundary coinciding with the event horizon:

\medskip\noindent\textbf{Exterior ($r > 2M$):} Paper~1 establishes
that the Schwarzschild geometry---metric components, Christoffel
symbols, Riemann tensor, Ricci flatness, Kretschmann
scalar---is $\BISH$. No omniscience principle is needed.

\medskip\noindent\textbf{Interior ($0 < r < 2M$):} The finite-time
physics remains $\BISH$ (cycloid computability, curvature divergence).
But the \emph{completed assertion} that every bounded monotone
trajectory in the interior converges to a definite limit---the
general singularity assertion---costs $\LPO$.

\medskip\noindent
The horizon thus demarcates not only what can communicate with
infinity, but what can be asserted without surveying an infinite set.

\subsection{Connection to Paper~8}\label{sec:paper8}

The pattern is identical to the 1D Ising model:

\begin{center}
\begin{tabular}{@{}p{3.2cm}p{4.5cm}p{4.5cm}@{}}
\toprule
& \textbf{Paper~8 (Ising)} & \textbf{Paper~13 (Schwarzschild)} \\
\midrule
$\BISH$ content & Finite-size partition function &
  Cycloid geodesic, Kretschmann scalar \\
$\LPO$ content & Thermodynamic limit (free energy convergence) &
  Geodesic incompleteness (completed limit) \\
Physical interpretation & Phase transitions require idealization &
  Singularities require idealization \\
Encoding technique & Coupling constant modulation &
  Sequence coupling (\texttt{geodesicCoupling}) \\
\bottomrule
\end{tabular}
\end{center}

\noindent
Both cases instantiate the same principle: completed infinite limits
in physics cost $\LPO$, but the finite approximations that carry
empirical content are $\BISH$.

\subsection{What This Does NOT Claim}\label{sec:not-claim}

This paper does not claim that ``reaching the singularity costs
$\LPO$'' in any operational sense. A freely falling observer
following the cycloid geodesic computes $r(\tau)$ constructively at
every finite proper time, and the limit $r \to 0$ is constructively
approachable (for any $\varepsilon$, we can find $\eta$ with
$r(\eta) < \varepsilon$). The $\LPO$ cost attaches to the
\emph{universally quantified completeness principle}---the assertion
that every bounded monotone trajectory in the interior converges to
a definite limit---not to any particular trajectory.

The physical content is this: classical GR's assertion of geodesic
incompleteness, when formulated as a completed-limit principle over
all infalling trajectories, carries the same logical cost ($\LPO$)
as the thermodynamic limit in statistical mechanics. Both are
instances of bounded monotone convergence applied to physically
motivated monotone sequences.

The Brouwerian counterexample (\texttt{geodesicCoupling}) is not
a geodesic---it is a valid element of the domain of the universally
quantified proposition. This is standard CRM methodology: the
counterexample demonstrates that the universal principle carries
$\LPO$ strength, while specific instances (the cycloid) are
constructive.


% ====================================================================
\section{Discussion}\label{sec:discussion}
% ====================================================================

\subsection{The Calibration Table}\label{sec:calibration}

Paper~13 adds new rows to the programme's calibration landscape:

\begin{center}
\begin{tabular}{@{}llll@{}}
\toprule
\textbf{Physical layer} & \textbf{Principle} & \textbf{Status} &
\textbf{Source} \\
\midrule
Finite-volume Gibbs states & $\BISH$ & Calibrated & Trivial \\
Finite-size approximations (Ising) & $\BISH$ & Calibrated &
  Paper~8 \\
Schwarzschild exterior & $\BISH$ & Calibrated & Paper~1 \\
Interior finite-time physics & $\BISH$ & Calibrated & Paper~13 \\
Bidual-gap witness ($S_1(H)$) & $\equiv \WLPO$ & Calibrated &
  Papers~2, 7 \\
Tsirelson bound (CHSH $\leq 2\sqrt{2}$) & $\BISH$ & Calibrated &
  Paper~11 \\
Bell state entropy ($\log 2$) & $\BISH$ & Calibrated &
  Paper~11 \\
Partial trace (qubit systems) & $\BISH$ & Calibrated &
  Paper~11 \\
Thermodynamic limit (Ising) & $\equiv \LPO$ & Calibrated &
  Paper~8 \\
Geodesic incompleteness & $\equiv \LPO$ & Calibrated &
  Paper~13 \\
Spectral gap decidability & Undecidable & Established &
  Cubitt et al.\ \cite{CPW15} \\
\bottomrule
\end{tabular}
\end{center}

\noindent
The pattern strengthens: all $\LPO$ costs arise from completed
infinite limits, all finite-time and finite-size physics is $\BISH$.

\subsection{The Encoding Objection}\label{sec:encoding}

A natural objection is that the encoding of binary sequences into
coupling sequences---and the subsequent application of the
completeness principle---is merely bounded monotone convergence in
disguise. The encoded sequence
$a(n) \in \{M, 0\}$ is a $\{0,1\}$-valued monotone sequence
composed with a scaling, and the decision procedure is the abstract
$\BMC \to \LPO$ proof applied to this specific class.

This objection is mathematically correct and interpretively
irrelevant. The abstract equivalence $\BMC \leftrightarrow \LPO$ is
known from~\cite{BV06}. The contribution of this paper is not a new
theorem in abstract constructive reverse mathematics but a verified
observation that $\BMC$, when applied to bounded monotone sequences
in $[0, 2M)$, \emph{is} the completed-limit content of geodesic
incompleteness. The formalization makes explicit what the
mathematical prose leaves implicit: the encoding is $\BISH$-valid,
the gap $\delta = M > 0$ is constructively positive, and the witness
extraction works without hidden omniscience. The \Lean{} axiom audit
confirms this.

This is the same methodological move as Paper~8, where the abstract
$\BMC \leftrightarrow \LPO$ equivalence was known and the
contribution was the specific physical instantiation and machine
verification.

\subsection{Methodological Limitations}\label{sec:limitations}

We are frank about the scope and limitations of this result:

\begin{enumerate}
  \item \textbf{Classical.choice is a Mathlib infrastructure artifact.}
    The \texttt{Classical.choice} appearing in the axiom profile of
    $\BISH$ results (cycloid properties, Kretschmann scalar) arises
    from \Mathlib{}'s pervasive use of classical logic in its real
    number library, not from the mathematical content of the proofs.
    The proof-content analysis methodology for handling this is
    described in Paper~10.

  \item \textbf{The formalization abstracts from the geodesic ODE.}
    We do not formalize the Lorentzian metric, Christoffel symbols,
    parallel transport, or the geodesic equation as an ODE. The
    cycloid solution is used directly as an explicit function. This
    means our formalization does not verify that the cycloid actually
    solves the radial geodesic equation---only that it has the
    algebraic and analytic properties (monotonicity, boundedness,
    endpoint values) claimed. The physical interpretation relies on
    standard textbook results connecting the cycloid to the geodesic
    equation.

  \item \textbf{The LPO cost is on the universal principle, not any
    specific geodesic.} The cycloid reaching $r = 0$ is
    constructive. The formalization proves this explicitly
    (\texttt{r\_cycloid\_at\_pi}). The $\LPO$ cost attaches to the
    assertion that \emph{every} bounded monotone sequence in the
    interior converges---a universal principle that subsumes but is
    not identical to any particular geodesic.

  \item \textbf{\texttt{SchwarzschildInteriorGeodesicIncompleteness}
    is BMC for a specific class of sequences.} The ``geodesic
    incompleteness'' interpretation is physical motivation, not a
    formal derivation from the Einstein equations. The formalization
    establishes an equivalence between a specific instance of BMC
    (antitone sequences in $[0, 2M)$) and LPO. Whether this
    particular instance captures the full logical content of
    geodesic incompleteness in a more complete formalization of GR
    is an open question.

  \item \textbf{No Penrose theorem is formalized.}
    The Penrose singularity theorem~\cite{Pen65} generalizes geodesic
    incompleteness from Schwarzschild to arbitrary spacetimes with
    trapped surfaces and energy conditions. Formalizing this would
    require massive infrastructure (global hyperbolicity, trapped
    surfaces, energy conditions) far beyond our current scope.

  \item \textbf{Scope is limited to Schwarzschild.}
    The result applies to the non-rotating, uncharged (Schwarzschild)
    case only. Extensions to Kerr (rotating) or Reissner--Nordstr\"om
    (charged) black holes would require additional analysis.
\end{enumerate}

\subsection{Open Problems}\label{sec:open}

\begin{enumerate}
  \item \textbf{Penrose theorem calibration.} Does the full Penrose
    singularity theorem, including trapped surface and energy
    condition hypotheses, calibrate above $\LPO$? Our result
    suggests this is likely: the completed-limit content is the same,
    and additional hypotheses may introduce further costs.

  \item \textbf{Cosmic censorship.} Weak cosmic censorship
    (singularities hidden behind horizons) involves a universal
    quantifier over all generic initial data sets. What is its
    logical cost?

  \item \textbf{Hawking radiation.} The quantum process by which
    black holes evaporate involves the thermodynamic limit (infinite
    number of field modes). Does the Hawking temperature calculation
    inherit the $\LPO$ cost, or is it dispensable via
    finite-mode approximation?
\end{enumerate}


% ====================================================================
\section{Certification Methodology}\label{sec:certification}
% ====================================================================

\subsection{Axiom Profile}\label{sec:axiom-profile}

All theorems carry \texttt{[propext, Classical.choice, Quot.sound]}
from \Mathlib{}, plus \texttt{bmc\_of\_lpo} for the reverse
direction. The $\BISH$ results (cycloid, Kretschmann) have
\texttt{Classical.choice} from \Mathlib{} infrastructure only.

\subsection{Certification Level}\label{sec:cert-level}

Paper~13 contains two certification levels in the series terminology
(Paper~10):
\begin{itemize}
  \item The $\BISH$ content is finite-dimensional, explicit
    computation---constructively valid by inspection.
  \item The $\LPO$ equivalence uses \texttt{bmc\_of\_lpo}
    intentionally (this \emph{is} the classical content).
  \item No minimal artifact is planned: the logical
    structure---$\BMC \leftrightarrow \LPO$ instantiated on
    $[0, 2M)$---is simple enough that the full artifact suffices.
\end{itemize}

\subsection{The \texttt{bmc\_of\_lpo} Axiom}\label{sec:bmc-axiom}

The equivalence $\BMC \leftrightarrow \LPO$ is due to Bridges and
V\^{\i}\c{t}\u{a}~\cite{BV06}. It is axiomatized in Paper~13
rather than re-proved because:
\begin{itemize}
  \item Paper~8 already uses the same axiom.
  \item The Bridges--V\^{\i}\c{t}\u{a} proof is standard and
    uncontroversial.
  \item Re-proving it would add approximately 200 lines of \Lean{}
    for zero epistemic gain.
\end{itemize}
The axiom is conservative: it introduces no new logical content
beyond what $\LPO$ already provides.


% ====================================================================
\section{AI-Assisted Methodology}\label{sec:ai}
% ====================================================================

This formalization was developed using \textbf{Claude Opus~4.6}
(Anthropic, 2026) via the \textbf{Claude Code} command-line
interface, following the same human--AI workflow as Papers~2, 7,
and~8~\cite{Lee26-P2,Lee26-P7,Lee26-P8,Anthropic2026}.

The author is a medical professional, not a domain expert in
constructive mathematics or general relativity. The mathematical
content of this paper---the connection between geodesic
incompleteness and BMC, the choice of cycloid parametrization, the
encoding strategy---was developed with extensive AI assistance. The
human author specified the research direction and high-level goals,
reviewed all mathematical claims for plausibility, and directed the
formalization strategy. Claude Opus~4.6 explored the \Mathlib{}
codebase, generated \Lean{} proof terms, handled debugging against
\Mathlib{} v4.28, and assisted with paper writing. Final
verification was by \texttt{lake build} (0~errors, 0~warnings,
0~sorries).

\begin{table}[h]
\centering
\begin{tabular}{@{}lll@{}}
\toprule
\textbf{Task} & \textbf{Human} & \textbf{AI (Claude Opus 4.6)} \\
\midrule
Research direction       & \checkmark & \\
Mathematical blueprint   & \checkmark & \checkmark \\
Proof strategy design    & \checkmark & \checkmark \\
\Mathlib{} API discovery & & \checkmark \\
\Lean{} proof generation & & \checkmark \\
Proof review             & \checkmark & \\
Build verification       & & \checkmark \\
Paper writing            & \checkmark & \checkmark \\
\bottomrule
\end{tabular}
\caption{Division of labor between human and AI.}
\label{tab:division}
\end{table}


% ====================================================================
\section*{Reproducibility}
% ====================================================================

\begin{mdframed}[backgroundcolor=gray!10]
\textbf{Reproducibility Box}
\begin{itemize}
\item \textbf{Repository}:
  \url{https://github.com/AICardiologist/FoundationRelativity}
\item \textbf{Path}: \texttt{Papers/P13\_EventHorizon/}
\item \textbf{Build}: \texttt{lake exe cache get \&\& lake build}
  (3{,}100 jobs, 0~errors, 0~sorry)
\item \textbf{Lean toolchain}:
  \texttt{leanprover/lean4:v4.28.0-rc1}
\item \textbf{Interface axiom}: \texttt{bmc\_of\_lpo}
  (Bridges--V\^{\i}\c{t}\u{a}, imported from Paper~8)
\item \textbf{Axiom audit}: \texttt{Certificates.lean}
\item \textbf{Axiom profile (main theorem)}:
  \texttt{[propext, Classical.choice, Quot.sound, bmc\_of\_lpo]}
\item \textbf{Axiom profile (BISH content)}:
  \texttt{[propext, Classical.choice, Quot.sound]}
  (Mathlib infra only)
\item \textbf{Total}: 8~files, 1{,}021~lines, 0~sorry
\item \textbf{Zenodo DOI}: 10.5281/zenodo.18529007
\end{itemize}
\end{mdframed}


% ====================================================================
\section*{Acknowledgments}
% ====================================================================

The \Lean{} formalization was developed using Claude Opus~4.6
(Anthropic, 2026) via the Claude Code CLI tool. We thank the
\Mathlib{} community for maintaining the comprehensive library
of formalized mathematics that made this work possible.


% ====================================================================
% Bibliography
% ====================================================================
\bibliographystyle{plainnat}

\begin{thebibliography}{30}

\bibitem[Anthropic(2026)]{Anthropic2026}
Anthropic.
\newblock Claude {Opus}~4.6 and {Claude Code} {CLI}.
\newblock \url{https://www.anthropic.com/claude}, 2026.

\bibitem[Bridges and V{\^\i}{\c{t}}{\u{a}}(2006)]{BV06}
D.~S.~Bridges and L.~S.~V{\^\i}{\c{t}}{\u{a}}.
\newblock \emph{Techniques of Constructive Analysis}.
\newblock Universitext. Springer, New York, 2006.

\bibitem[Cubitt et~al.(2015)]{CPW15}
T.~S.~Cubitt, D.~Perez-Garcia, and M.~M.~Wolf.
\newblock Undecidability of the spectral gap.
\newblock \emph{Nature}, 528:207--211, 2015.

\bibitem[{de Moura} et~al.(2021)]{deMoura2021}
L.~{de Moura}, S.~Kong, J.~Avigad, F.~{van Doorn}, and M.~{von Raumer}.
\newblock The {Lean} theorem prover (system description).
\newblock In \emph{CADE-25}, LNAI 9195, pages 378--388. Springer, 2015.
\newblock Lean~4: \url{https://lean-lang.org/}, 2021--present.

\bibitem[Echenim and Mhalla(2024)]{EM24}
M.~Echenim and M.~Mhalla.
\newblock Formalising the CHSH inequality in {Isabelle/HOL}.
\newblock In \emph{ITP 2024}, 2024.

\bibitem[Hawking and Ellis(1973)]{HE73}
S.~W.~Hawking and G.~F.~R.~Ellis.
\newblock \emph{The Large Scale Structure of Space-Time}.
\newblock Cambridge University Press, 1973.

\bibitem[Lee(2026a)]{Lee26-P1}
P.~C.-K.~Lee.
\newblock Schwarzschild exterior curvature verification in {Lean}~4.
\newblock Preprint, 2026. Paper~1 in the constructive reverse
  mathematics series.

\bibitem[Lee(2026b)]{Lee26-P2}
P.~C.-K.~Lee.
\newblock {WLPO} equivalence of the bidual gap in $\ell^1$: a {Lean}~4
  formalization.
\newblock Preprint, 2026. Paper~2 in the constructive reverse
  mathematics series.

\bibitem[Lee(2026c)]{Lee26-P7}
P.~C.-K.~Lee.
\newblock Non-reflexivity of $S_1(H)$ implies {WLPO}: a {Lean}~4
  formalization.
\newblock Preprint, 2026. Paper~7 in the constructive reverse
  mathematics series.

\bibitem[Lee(2026d)]{Lee26-P8}
P.~C.-K.~Lee.
\newblock The logical cost of the thermodynamic limit: {LPO}-equivalence
  and {BISH}-dispensability for the {1D} {Ising} free energy.
\newblock Preprint, 2026. Paper~8 in the constructive reverse
  mathematics series.

\bibitem[Lee(2026e)]{Lee26-P10}
P.~C.-K.~Lee.
\newblock The logical geography of mathematical physics: constructive
  calibration from density matrices to the event horizon.
\newblock Preprint, 2026. Zenodo DOI: 10.5281/zenodo.18527877.
  Paper~10 in the constructive reverse mathematics series.

\bibitem[Lee(2026f)]{Lee26-P11}
P.~C.-K.~Lee.
\newblock Constructive entanglement: {CHSH}, {Tsirelson} bound, and {Bell}
  state entropy at {BISH}.
\newblock Preprint, 2026. Zenodo DOI: 10.5281/zenodo.18527676.
  Paper~11 in the constructive reverse mathematics series.

\bibitem[Lee(2026g)]{Lee26-P12}
P.~C.-K.~Lee.
\newblock The map and the territory: a constructive history of mathematical
  physics.
\newblock Preprint, 2026. Zenodo DOI: 10.5281/zenodo.18529007.
  Paper~12 in the constructive reverse mathematics series.

\bibitem[{Mathlib Community}(2020--)]{Mathlib2020}
{Mathlib Community}.
\newblock \emph{Mathlib}: the math library for {Lean}.
\newblock \url{https://leanprover-community.github.io/mathlib4_docs/},
  2020--present.

\bibitem[Penrose(1965)]{Pen65}
R.~Penrose.
\newblock Gravitational collapse and space-time singularities.
\newblock \emph{Physical Review Letters}, 14:57--59, 1965.

\bibitem[Schwarzschild(1916)]{Sch16}
K.~Schwarzschild.
\newblock {\"U}ber das {G}ravitationsfeld eines {M}assenpunktes nach
  der {E}insteinschen {T}heorie.
\newblock \emph{Sitzungsberichte der K\"oniglich Preussischen Akademie
  der Wissenschaften}, pages 189--196, 1916.

\bibitem[Wald(1984)]{Wal84}
R.~M.~Wald.
\newblock \emph{General Relativity}.
\newblock University of Chicago Press, 1984.

\end{thebibliography}

\end{document}


\documentclass[11pt,a4paper]{article}

% ====================================================================
% Packages
% ====================================================================
\usepackage[utf8]{inputenc}
\usepackage[T1]{fontenc}
\usepackage{amsmath,amssymb,amsthm}
\usepackage{mathtools}
\usepackage{hyperref}
\usepackage[margin=1in]{geometry}
\usepackage{enumitem}
\usepackage{booktabs}
\usepackage{listings}
\usepackage{xcolor}
\usepackage{cleveref}
\usepackage{natbib}
\usepackage{mdframed}

% ====================================================================
% Theorem environments
% ====================================================================
\theoremstyle{plain}
\newtheorem{theorem}{Theorem}[section]
\newtheorem{lemma}[theorem]{Lemma}
\newtheorem{proposition}[theorem]{Proposition}
\newtheorem{corollary}[theorem]{Corollary}

\theoremstyle{definition}
\newtheorem{definition}[theorem]{Definition}
\newtheorem{remark}[theorem]{Remark}

% ====================================================================
% Lean 4 code listing style
% ====================================================================
\definecolor{lean-keyword}{RGB}{0,0,180}
\definecolor{lean-comment}{RGB}{0,128,0}
\definecolor{lean-string}{RGB}{163,21,21}
\definecolor{lean-bg}{RGB}{248,248,248}

\lstdefinelanguage{lean4}{
  keywords={theorem,lemma,def,class,instance,import,open,variable,
            noncomputable,section,namespace,end,where,let,have,show,
            intro,obtain,use,exact,rw,simp,apply,by,fun,match,if,
            then,else,do,return,axiom,abbrev,private,attribute,
            suffices,change,congr,ext,constructor,rintro,push_neg,
            linarith,absurd,set_option,omit,in,set,cases,refine,
            calc,filter_upwards,with,specialize},
  sensitive=true,
  morecomment=[l]{--},
  morecomment=[s]{/-}{-/},
  morestring=[b]",
  morestring=[b]',
}

\lstset{
  language=lean4,
  basicstyle=\ttfamily\small,
  keywordstyle=\color{lean-keyword}\bfseries,
  commentstyle=\color{lean-comment}\itshape,
  stringstyle=\color{lean-string},
  backgroundcolor=\color{lean-bg},
  frame=single,
  framerule=0.5pt,
  breaklines=true,
  breakatwhitespace=true,
  tabsize=2,
  showstringspaces=false,
  numbers=left,
  numberstyle=\tiny\color{gray},
  numbersep=5pt,
  xleftmargin=15pt,
  captionpos=b,
}

% ====================================================================
% Macros
% ====================================================================
\newcommand{\NN}{\mathbb{N}}
\newcommand{\ZZ}{\mathbb{Z}}
\newcommand{\RR}{\mathbb{R}}
\newcommand{\CC}{\mathrm{CC}}
\newcommand{\DC}{\mathrm{DC}}
\newcommand{\ACzero}{\mathrm{AC}_0}
\newcommand{\BISH}{\mathrm{BISH}}
\newcommand{\CRM}{\mathrm{CRM}}
\newcommand{\LPO}{\mathrm{LPO}}
\newcommand{\WLPO}{\mathrm{WLPO}}
\newcommand{\LLPO}{\mathrm{LLPO}}
\newcommand{\MP}{\mathrm{MP}}
\newcommand{\FT}{\mathrm{FT}}
\newcommand{\Lean}{\textsc{Lean~4}}
\newcommand{\Mathlib}{\textsc{Mathlib4}}
\newcommand{\leanok}{\textsf{\small \textcolor{green!70!black}{\checkmark}}}
\newcommand{\leanpartial}{\textsf{\small \textcolor{orange!80!black}{(partial)}}}
\newcommand{\norm}[1]{\left\lVert #1 \right\rVert}
\newcommand{\ip}[2]{\langle #1, #2 \rangle}

% ====================================================================
% Title
% ====================================================================
\title{%
  \textbf{The Choice Axis in Constructive Reverse Mathematics:}\\[6pt]
  Calibrating Ergodic Theorems and Laws of Large Numbers\\
  against Countable and Dependent Choice\\[6pt]
  {\normalsize A Lean~4 Formalization}%
}

\author{
  Paul Chun-Kit Lee\thanks{%
    New York University.
    AI-assisted formalization; see \S\ref{sec:ai} for methodology.} \\
  New York University \\
  \texttt{dr.paul.c.lee@gmail.com}
}

\date{February 2026}

% ====================================================================
\begin{document}
\maketitle

% ====================================================================
\begin{abstract}
We open a new axis in the constructive reverse mathematics (CRM) calibration
of mathematical physics: the \emph{choice hierarchy} $\ACzero < \CC < \DC$.
The main result is that the mean ergodic theorem (von Neumann, 1932)
is equivalent over $\BISH$ to Countable Choice ($\CC$). The forward
direction---$\CC$ implies the mean ergodic theorem---is fully formalized
in \Lean{} with 600+ lines of genuine Hilbert space analysis and a clean
axiom profile. For the reverse direction, we introduce a Type-level
\emph{computable} mean ergodic statement and prove it implies $\CC$
through an explicit $\ell^2(\NN \times \NN)$ encoding with a diagonal
reflection operator; the hypothesis is genuinely used, not discarded.
We further calibrate Birkhoff's pointwise ergodic theorem
(1931) to Dependent Choice ($\DC$), the weak law of large numbers to $\CC$,
and the strong law of large numbers to $\DC$. The calibration reveals a
clean physical separation: \emph{ensemble/average behavior requires $\CC$;
individual trajectory behavior requires $\DC$.} We propose the DC~Ceiling
Thesis: no calibratable physical theorem requires more than $\DC$.
The combined formalization comprises 1805 lines of \Lean{} across 12 modules
with zero non-permanent sorries, one custom axiom (Birkhoff's theorem is
not in \Mathlib{}), and two permanent model-theoretic sorries for
independence results.
\end{abstract}

\tableofcontents

% ====================================================================
\section{Introduction}\label{sec:intro}
% ====================================================================

Constructive reverse mathematics ($\CRM$) assigns to each mathematical
theorem a precise position in the constructive hierarchy by identifying
the weakest logical principle required for its proof over Bishop-style
constructive mathematics ($\BISH$). Applied to mathematical physics,
this programme determines the exact logical cost of physical
idealizations---and whether that cost is essential for the physics.

Prior work in this series calibrated physical theories against the
\emph{omniscience hierarchy}: the bidual gap is equivalent to $\WLPO$
\cite{Lee26a,Lee26b}, Bell's theorem relates to $\LLPO$ \cite{Lee26Bell},
and the thermodynamic limit of the 1D Ising model is equivalent to $\LPO$
\cite{Lee26Ising}. These principles concern the decidability of
properties of infinite sequences: whether a binary sequence is
identically zero, has a nonzero term, or satisfies related conditions.

This paper opens a \emph{second axis}: the \textbf{choice hierarchy}
\[
  \ACzero \;\subsetneq\; \CC \;\subsetneq\; \DC \;\subsetneq\; \mathrm{AC}.
\]
Here $\ACzero$ is finite choice (trivially $\BISH$-provable), $\CC$ is
countable choice, $\DC$ is dependent choice, and $\mathrm{AC}$ is the
full axiom of choice. Unlike the omniscience hierarchy, the choice
hierarchy concerns the \emph{strength of selection} from infinite
families---not the decidability of predicates. The two axes are
largely orthogonal: $\WLPO$ neither implies nor is implied by $\CC$
over $\BISH$.

The main results are:

\begin{enumerate}[label=(\Roman*)]
\item The \textbf{mean ergodic theorem} (von Neumann) is equivalent
  to $\CC$ over $\BISH$. The forward direction is fully formalized
  in \Lean{}. The reverse direction has both a paper-level constructive
  proof and a non-trivial \Lean{} formalization via a Prop/Type lifting
  technique (\S\ref{sec:reverse}).
\item \textbf{Birkhoff's pointwise ergodic theorem} is equivalent to
  $\DC$ over $\BISH$. Both directions are paper-level proofs; the
  forward direction is axiomatized (Birkhoff's theorem is not in \Mathlib{}).
\item The \textbf{weak law of large numbers} calibrates to $\CC$;
  the \textbf{strong law of large numbers} calibrates to $\DC$.
\item \textbf{DC Ceiling Thesis}: no calibratable physical theorem
  in the programme requires more than $\DC$. Full $\mathrm{AC}$ produces
  only mathematical pathologies.
\end{enumerate}

The physical interpretation is clean: \emph{ensemble/average behavior
(what laboratories verify) requires $\CC$; individual trajectory behavior
(what probability theory idealizes) requires $\DC$.} This separation
maps precisely onto the distinction between convergence in probability
(weak law) and almost-sure convergence (strong law), and between
$L^2$-convergence (mean ergodic) and pointwise convergence (Birkhoff).

The cumulative calibration landscape, integrating both axes, is:

\begin{center}
\begin{tabular}{@{}llll@{}}
\toprule
\textbf{Physical layer} & \textbf{Principle} & \textbf{Axis} & \textbf{Source} \\
\midrule
Finite-volume Gibbs states & $\BISH$ & --- & Trivial \\
Single quantum measurement & $\ACzero$ & Choice & Trivial \\
Bidual-gap witness & $\equiv \WLPO$ & Omniscience & Papers~2, 7 \\
Bell's theorem / EPR & $\LLPO$ & Omniscience & Paper~21 \\
Thermodynamic limit & $\equiv \LPO$ & Omniscience & Paper~8 \\
Mean ergodic theorem & $\equiv \CC$ & Choice & This paper \\
Weak law of large numbers & $\CC$ & Choice & This paper \\
Birkhoff's ergodic theorem & $\equiv \DC$ & Choice & This paper \\
Strong law of large numbers & $\DC$ & Choice & This paper \\
\bottomrule
\end{tabular}
\end{center}

The paper is organized as follows. \Cref{sec:background} reviews the
constructive framework, the choice principles, and the ergodic theorems,
with a detailed discussion of the metastability--convergence gap.
\Cref{sec:mean-ergodic} presents the flagship result: the CC equivalence
of the mean ergodic theorem, with a human-readable proof of the forward
direction. \Cref{sec:birkhoff} treats Birkhoff's theorem and DC.
\Cref{sec:lln} calibrates the laws of large numbers.
\Cref{sec:ceiling} presents the DC~Ceiling Thesis and updated calibration
table. \Cref{sec:lean} describes the \Lean{} formalization.
\Cref{sec:discussion} discusses the results.

% ====================================================================
\section{Background}\label{sec:background}
% ====================================================================

\subsection{Constructive Frameworks and Choice Principles}

We work within Bishop-style constructive mathematics ($\BISH$):
intuitionistic logic with function extensionality and dependent
choice at the foundational level \cite{Bis67,BB85}. The key choice
principles form a strict hierarchy over $\BISH$:

\begin{definition}[Finite Choice, $\ACzero$]\label{def:ac0} \leanok{}
For every finite family $\{S_i\}_{i<k}$ of nonempty sets, there
exists a choice function $f$ with $f(i) \in S_i$.
\end{definition}

$\ACzero$ is trivially provable in $\BISH$ by induction on $k$.

\begin{definition}[Countable Choice, $\CC$]\label{def:cc} \leanok{}
For every countable family $\{A_n\}_{n \in \NN}$ of nonempty
subsets of $\NN$, there exists a choice function $f : \NN \to \NN$
with $f(n) \in A_n$ for all $n$.
\end{definition}

\begin{definition}[Dependent Choice, $\DC$]\label{def:dc} \leanok{}
For every set $X$, total binary relation $R$ on $X$ (i.e.,
$\forall x \in X, \exists y \in X, R(x,y)$), and initial element
$x_0 \in X$, there exists a sequence $f : \NN \to X$ with $f(0) = x_0$
and $R(f(n), f(n+1))$ for all $n$.
\end{definition}

The hierarchy $\ACzero \subsetneq \CC \subsetneq \DC$ is strict over
$\BISH$:

\begin{theorem}[Choice hierarchy]\label{thm:hierarchy} \leanok{}
$\DC \implies \CC \implies \ACzero$. Both implications are strict:
$\ACzero \not\implies \CC$ and $\CC \not\implies \DC$.
\end{theorem}

\begin{proof}
$\DC \implies \CC$: Given nonempty sets $A_n$, define a relation $R$
on $\NN \times \NN$ by $R(n,a)(n',a') \iff n' = n+1 \land a' \in A_{n'}$.
$\DC$ yields a thread; projecting second components gives the choice
function. $\CC \implies \ACzero$: $\mathrm{Fin}(k)$ embeds into $\NN$.
Both forward implications are formalized in \Lean{}.

The separations are model-theoretic: $\ACzero \not\implies \CC$ is witnessed
by realizability models; $\CC \not\implies \DC$ is witnessed by
Fraenkel--Mostowski permutation models where $\CC$ holds but $\DC$ fails
\cite{Jec73}.
\end{proof}

\begin{remark}[Classical triviality]\label{rem:classical}
In classical mathematics (with the full axiom of choice), $\CC$ and $\DC$
are both provable. The hierarchy is nontrivial only over $\BISH$ or
similar constructive frameworks. This has implications for formalization:
see \S\ref{sec:scope}.
\end{remark}

The key structural distinction between $\CC$ and $\DC$ is the
\emph{dependence structure} of the choices. In $\CC$, each choice is
independent: the element selected from $A_n$ does not depend on the
element selected from $A_m$ for $m \neq n$. In $\DC$, each choice
depends on the previous choice: $f(n+1)$ must satisfy $R(f(n), f(n+1))$.
This distinction maps precisely onto the physical separation between
ensemble statistics and individual trajectories.

\subsection{Ergodic Theorems}\label{sec:ergodic-background}

\begin{definition}[Mean Ergodic Theorem]\label{def:met}
Let $H$ be a Hilbert space and $U : H \to H$ an isometry. The
\emph{Ces\`{a}ro averages} are
\[
  A_n x = \frac{1}{n} \sum_{k=0}^{n-1} U^k x.
\]
The mean ergodic theorem asserts: for every $x \in H$, $A_n x$ converges
in norm to the orthogonal projection of $x$ onto $\mathrm{Fix}(U) = \{y : Uy = y\}$.
\end{definition}

\begin{definition}[Birkhoff's Pointwise Ergodic Theorem]\label{def:birkhoff}
Let $(X, \mu, T)$ be a measure-preserving system and $f \in L^1(X, \mu)$.
The time averages
\[
  \frac{1}{n} \sum_{k=0}^{n-1} f(T^k x)
\]
converge for $\mu$-almost every $x$.
\end{definition}

The two theorems are related but logically distinct. The mean ergodic
theorem gives convergence in norm ($L^2$-convergence); Birkhoff gives
pointwise convergence $\mu$-a.e. Classically, the latter implies the
former (for $L^2$ functions), but the converse is false. Constructively,
this gap is reflected in the choice hierarchy: mean ergodic requires
$\CC$; Birkhoff requires $\DC$.

\begin{remark}[Metastability and the convergence gap]\label{rem:metastability}
The proof-mining programme of Kohlenbach \cite{Koh08} and
Avigad--Gerhardy--Towsner \cite{AGT10} has shown that \emph{metastable}
versions of both ergodic theorems are provable without any choice principle.
A metastable version of convergence asserts: for every $\varepsilon > 0$
and rate function $F : \NN \to \NN$, there exists $N$ such that for all
$m, n \in [N, F(N)]$, $|a_m - a_n| < \varepsilon$. This is strictly weaker
than full convergence but captures the ``finite'' content.

The gap between metastability and full convergence is precisely where the
choice principles enter:
\begin{itemize}
  \item For the \textbf{mean ergodic theorem}, the gap from metastability
    to norm convergence requires choosing a sequence of approximations
    (one per precision level)---this is $\CC$.
  \item For \textbf{Birkhoff's theorem}, the gap from metastability to
    pointwise a.e.\ convergence requires constructing the exceptional null
    set via dependent sequential refinement---this is $\DC$.
\end{itemize}
This observation is the conceptually sharpest claim of the paper:
\emph{the choice hierarchy measures exactly the metastability--convergence
gap for ergodic theorems.}
\end{remark}

\subsection{Laws of Large Numbers}

The weak and strong laws of large numbers provide a parallel calibration
in the probabilistic setting.

The \textbf{weak law} (convergence in probability): for i.i.d.\ random
variables $X_1, X_2, \ldots$ with finite variance,
\[
  P\!\left(\left|\frac{1}{n}\sum_{k=1}^n X_k - \mu\right| \geq \varepsilon\right) \to 0
\]
for every $\varepsilon > 0$. The proof uses Chebyshev's inequality:
$P(|S_n/n - \mu| \geq \varepsilon) \leq \mathrm{Var}(X) / (n\varepsilon^2)$,
which is constructive ($\BISH$-valid). The infinite sequence of
measurements requires $\CC$ (independent choices).

The \textbf{strong law} (almost sure convergence): with probability~1,
$S_n/n \to \mu$. The proof requires constructing the exceptional null
set via Borel--Cantelli arguments with dependent refinement ($\DC$).

% ====================================================================
\section{CC $\leftrightarrow$ Mean Ergodic Theorem}\label{sec:mean-ergodic}
% ====================================================================

\subsection{Statement}

\begin{theorem}[Main result]\label{thm:main}
Over $\BISH$, $\CC \iff$ Mean Ergodic Theorem.
\end{theorem}

The forward direction is fully formalized in \Lean{}. The reverse
direction has both a paper-level constructive proof (\S\ref{sec:reverse-paper})
and a non-trivial Type-level \Lean{} formalization (\S\ref{sec:reverse-lean}).

\subsection{Forward Direction: $\CC \implies$ Mean Ergodic Theorem}
\label{sec:forward}

We give a human-readable proof that tracks the \Lean{} formalization.
The proof occupies approximately 600~lines of \Lean{} across
\texttt{CesaroAverage.lean} (172~lines) and \texttt{MeanErgodic.lean}
(268~lines).

\begin{proof}[Proof ($\CC \implies$ Mean Ergodic Theorem)]
Let $H$ be a Hilbert space, $U : H \to H$ an isometry, and $x \in H$.
We construct $Px \in \mathrm{Fix}(U)$ such that $A_n x \to Px$ in norm.

\medskip\noindent\textbf{Step 1: Orthogonal decomposition.}
Define $K = \mathrm{Fix}(U) = \ker(U - I)$. Since $U - I$ is a bounded
linear operator, $K$ is a closed subspace of $H$. (In the formalization:
\texttt{fixedSubspace\_isClosed}.) Therefore $H = K \oplus K^\perp$ and
$x = Px + x'$ where $Px = \pi_K(x)$ is the orthogonal projection and
$x' = x - Px \in K^\perp$.

\medskip\noindent\textbf{Step 2: Convergence on $\mathrm{Fix}(U)$.}
If $y \in K$, then $Uy = y$, so $U^k y = y$ for all $k$, and
$A_n y = \frac{1}{n} \sum_{k=0}^{n-1} y = y$. In particular,
$A_n(Px) = Px$. (Formalized: \texttt{cesaroAvg\_of\_fixed}.)

\medskip\noindent\textbf{Step 3: Convergence on Range$(U - I)$.}
For any $y \in H$, the element $w = Uy - y$ lies in $\mathrm{Range}(U - I)$.
By telescoping:
\[
  \sum_{k=0}^{n-1} U^k(Uy - y) = U^n y - y,
\]
so $A_n w = \frac{1}{n}(U^n y - y)$. Since $U$ is an isometry,
$\norm{A_n w} \leq \frac{2\norm{y}}{n} \to 0$.
(Formalized: \texttt{sum\_iterate\_sub}, \texttt{cesaroAvg\_range\_norm\_le}.)

\medskip\noindent\textbf{Step 4: Density of Range$(U - I)$ in $K^\perp$.}
We show $\overline{\mathrm{Range}(U - I)} \supseteq K^\perp$.
The key lemma is: if $z \perp \mathrm{Range}(U - I)$, then $z \in K$.
(Formalized: \texttt{orthogonal\_range\_sub\_le\_fixed}.)

Proof of the key lemma: Suppose $\ip{z}{Uy - y} = 0$ for all $y$.
Then $\ip{z}{Uy} = \ip{z}{y}$ for all $y$. Since $U$ is an isometry,
it preserves inner products: $\ip{Uz}{Uz} = \ip{z}{z}$.
Using the polarization identity and $\ip{z}{Uy} = \ip{z}{y}$,
we compute $\norm{Uz - z}^2 = \ip{Uz}{Uz} - \ip{Uz}{z} - \ip{z}{Uz} + \ip{z}{z} = 0$.
Therefore $Uz = z$, i.e., $z \in K$.

This gives $\mathrm{Range}(U-I)^\perp \subseteq K$, hence
$K^\perp \subseteq \mathrm{Range}(U-I)^{\perp\perp} = \overline{\mathrm{Range}(U-I)}$.

\medskip\noindent\textbf{Step 5: Where $\CC$ enters.}
Since $x' \in K^\perp \subseteq \overline{\mathrm{Range}(U-I)}$, for
each precision $\varepsilon_m = \varepsilon/2$ there exists $w_m \in \mathrm{Range}(U - I)$
with $\norm{x' - w_m} < \varepsilon_m$. Choosing such approximations
for each precision level requires a countable sequence of independent
choices---one for each $m \in \NN$. This is precisely $\CC$.

Note that the choices are \emph{independent}: the element chosen at
precision $1/m$ does not constrain the choice at precision $1/(m+1)$.
This is why $\CC$ suffices and $\DC$ is not needed.

\medskip\noindent\textbf{Step 6: Combining.}
Fix $\varepsilon > 0$. Choose $w \in \mathrm{Range}(U-I)$ with
$\norm{x' - w} < \varepsilon/2$ (using density). By the uniform
bound on Ces\`{a}ro averages (\texttt{cesaroAvg\_norm\_le}):
$\norm{A_n(x' - w)} \leq \norm{x' - w} < \varepsilon/2$.
Write $w = Uy - y$. Find $N$ such that $2\norm{y}/(n+1) < \varepsilon/2$
for $n \geq N$. Then:
\[
  \norm{A_n(x')} \leq \norm{A_n(x' - w)} + \norm{A_n(w)}
  < \frac{\varepsilon}{2} + \frac{\varepsilon}{2} = \varepsilon.
\]
Since $A_n(x) - Px = A_n(x')$, we have $A_n(x) \to Px$.
\end{proof}

\subsection{Reverse Direction: Mean Ergodic $\implies \CC$}
\label{sec:reverse}

\begin{theorem}[$\CC$ from Mean Ergodic, over $\BISH$]\label{thm:reverse}
The mean ergodic theorem constructively implies countable choice.
\end{theorem}

\subsubsection{Paper-Level Proof}\label{sec:reverse-paper}

\begin{proof}[Proof sketch]
Given a countable choice problem: nonempty sets $S_n \subseteq \NN$
for $n \in \NN$.

\textbf{Step 1}: Build $H = \ell^2(\NN)$ with an orthogonal direct sum
decomposition $H = \bigoplus_n H_n$ where each $H_n$ encodes the choice
set $S_n$.

\textbf{Step 2}: Define a unitary operator $U$ on $H$ that cyclically
shifts within each block $H_n$. The structure of $U$ ensures:
$\mathrm{Fix}(U) \cap H_n$ is nontrivial iff $S_n$ is nonempty, and
the projection onto $\mathrm{Fix}(U)$ restricted to $H_n$ selects an
element.

\textbf{Step 3}: Apply the mean ergodic theorem to a suitable starting
vector. The Ces\`{a}ro averages converge in norm to the projection onto
$\mathrm{Fix}(U)$.

\textbf{Step 4}: Read off the limit's components in each block $H_n$
to extract choices from each $S_n$. This step is purely algebraic
(orthogonal projection onto each $H_n$) and does not require additional
choice.

The critical subtlety is that Step~4 must not smuggle in $\CC$ through
the back door. The convergence of Ces\`{a}ro averages (in norm) provides
the limit as a single element of $H$, and extracting its components is
algebraic.
\end{proof}

\subsubsection{The Classical Triviality Obstacle}\label{sec:obstacle}

In \Lean{}'s classical logic, $\CC_\NN$ is provable outright via
\texttt{Classical.choice}, so the Prop-level statement
$\text{Mean Ergodic} \to \CC_\NN$ holds trivially---the antecedent is
unused. The \Lean{} proof is a 3-line theorem using
\texttt{Set.Nonempty.some}. This is mathematically correct but
formalization-theoretically vacuous: the hypothesis contributes nothing.

This obstacle is inherent to classical proof assistants. Any Prop-level
statement $P \to Q$ where $Q$ is classically provable is trivially true
in \Lean{}, regardless of $P$. The genuine constructive content
(the Hilbert space encoding) lives in $\BISH$ and cannot be captured
at the Prop level.

\subsubsection{Type-Level Formalization}\label{sec:reverse-lean}

To overcome this obstacle, we introduce a Type-level formulation where
the mean ergodic hypothesis provides \emph{data}, not merely existence
claims. The development proceeded in two phases:

\begin{itemize}
\item \textbf{Phase~1} (original bundle, 11~files, 1410~lines): Forward
  direction fully formalized; reverse direction classically trivial;
  all other calibrations (Birkhoff, LLN, separation).
\item \textbf{Phase~2} (\texttt{Computable.lean}, 395~lines): Non-trivial
  Type-level encoding of the reverse direction.
\end{itemize}

\begin{definition}[Computable Mean Ergodic]\label{def:mec}
A \emph{computable mean ergodic datum} for an isometry $U : H \to H$
is a structure providing:
\begin{enumerate}
\item $\mathrm{proj} : H \to H$ --- the projection (data, not existence);
\item $\mathrm{modulus} : H \to \RR_{>0} \to \NN$ --- a convergence modulus;
\item $\mathrm{proj\_fixed}$ --- proof that $\mathrm{proj}(x) \in \mathrm{Fix}(U)$;
\item $\mathrm{modulus\_spec}$ --- proof that for $n \geq \mathrm{modulus}(x,\varepsilon)$,
  $\norm{A_{n+1}(x) - \mathrm{proj}(x)} < \varepsilon$.
\end{enumerate}
\end{definition}

\noindent
The universal statement \texttt{MeanErgodicComputableAll} asserts that
every Hilbert space isometry admits such a datum, wrapped in
\texttt{Nonempty} to form a Prop.

\begin{theorem}[Type-level reverse, formalized]\label{thm:reverse-lean}
\texttt{MeanErgodicComputableAll} $\implies$ $\CC_\NN$.
\end{theorem}

\begin{proof}[Proof (human-readable summary of the 395-line formalization)]
Given a choice problem $A : \NN \to \mathcal{P}(\NN)$ with each $A(n)$
nonempty:

\medskip\noindent\textbf{Step 1: Hilbert space.}
Take $H = \ell^2(\NN \times \NN, \mathbb{C})$, the space of
square-summable functions $f : \NN \times \NN \to \mathbb{C}$.

\medskip\noindent\textbf{Step 2: Diagonal reflection.}
Define the \emph{reflection operator} $U_A$ by:
\[
(U_A f)(n,m) = \begin{cases} f(n,m) & \text{if } m \in A(n), \\ -f(n,m) & \text{if } m \notin A(n). \end{cases}
\]
This is a diagonal operator with eigenvalues $\pm 1$. It is an involution
($U_A^2 = I$) and an isometry ($\norm{U_A f} = \norm{f}$).

\medskip\noindent\textbf{Step 3: Fixed subspace.}
$\mathrm{Fix}(U_A) = \{f : f(n,m) = 0 \text{ whenever } m \notin A(n)\}$.
That is, fixed vectors are precisely those supported on the ``graph'' of $A$.

\medskip\noindent\textbf{Step 4: Probe vector.}
Define $x_0 \in \ell^2(\NN \times \NN)$ by
\[
x_0(n,m) = \frac{1}{2^n \cdot 2^m}.
\]
All coordinates are nonzero. Square-summability follows from
$\sum_{n,m} |x_0(n,m)|^2 \leq \sum_{n,m} (1/2)^n (1/2)^m < \infty$.

\medskip\noindent\textbf{Step 5: Ces\`{a}ro stability.}
At coordinates $(n,m)$ with $m \in A(n)$, the reflection acts as the
identity: $U_A^k x_0(n,m) = x_0(n,m)$ for all $k$. Therefore the
Ces\`{a}ro average $A_N x_0(n,m) = x_0(n,m) = 1/(2^n \cdot 2^m)$ is
constant---independent of $N$.

\medskip\noindent\textbf{Step 6: Extraction.}
Apply the computable mean ergodic hypothesis to $U_A$ and $x_0$ to obtain
$P = \mathrm{proj}(x_0) \in \mathrm{Fix}(U_A)$ and the convergence modulus.
\begin{itemize}
\item \textit{Fixed-subspace membership}: Since $P \in \mathrm{Fix}(U_A)$,
  $P(n,m) = 0$ whenever $m \notin A(n)$.
\item \textit{Nonzero coordinates}: The norm convergence
  $\norm{A_N x_0 - P} \to 0$ implies coordinate convergence. At any
  $(n,m_0)$ with $m_0 \in A(n)$, the Ces\`{a}ro average is constantly
  $1/(2^n \cdot 2^m)$, so $P(n,m_0) = 1/(2^n \cdot 2^m) \neq 0$.
  (Proved by contradiction using the convergence modulus.)
\item \textit{Deterministic search}: For each $n$, there exists $m$
  with $P(n,m) \neq 0$; by \texttt{Nat.find}, extract the \emph{least}
  such $m$. Since $P(n,m) \neq 0$ implies $m \in A(n)$ (by
  fixed-subspace membership), this gives a choice function $f(n) \in A(n)$.
\end{itemize}
\end{proof}

\begin{remark}[Axiom profile and hypothesis usage]\label{rem:axiom-reverse}
The axiom profile of \texttt{meanErgodicComputableAll\_implies\_cc} is
\texttt{[propext, Classical.choice, Quot.sound]}---the same as every
theorem using \Mathlib{}'s analysis infrastructure.
\texttt{Classical.choice} enters through \Mathlib{}'s decidability
instances for set membership, not through the mathematical argument.
Crucially, the hypothesis \texttt{h : MeanErgodicComputableAll} is
\emph{genuinely used}: the proof calls \texttt{h.some} to obtain the
projection, \texttt{proj\_fixed} for fixed-subspace membership, and
\texttt{modulus\_spec} for the convergence guarantee. The hypothesis
cannot be discarded---removing it breaks the proof.
\end{remark}


% ====================================================================
\section{DC and Birkhoff's Pointwise Ergodic Theorem}\label{sec:birkhoff}
% ====================================================================

\subsection{Statement}

\begin{theorem}\label{thm:birkhoff-dc}
Over $\BISH$, $\DC \iff$ Birkhoff's Pointwise Ergodic Theorem.
\end{theorem}

Both directions are paper-level proofs. The forward direction is
axiomatized in \Lean{} because Birkhoff's theorem is not in \Mathlib{}.

\subsection{Forward Direction: DC $\implies$ Birkhoff}\label{sec:dc-birkhoff}

\textbf{Where DC enters the proof.}
The standard proof of Birkhoff's theorem proceeds via:
\begin{enumerate}
  \item The \textbf{maximal ergodic lemma}: for $f \in L^1$, the maximal
    function $f^*(x) = \sup_n \frac{1}{n}\sum_{k=0}^{n-1} f(T^k x)$
    satisfies $\int_{\{f^* > 0\}} f \, d\mu \geq 0$. This uses only
    finite operations---\emph{no choice needed}.
  \item Showing the set $\{x : \limsup A_n f(x) - \liminf A_n f(x) > \varepsilon\}$
    has measure zero for every $\varepsilon > 0$. This requires applying
    the maximal lemma to rational approximations.
  \item Constructing the full exceptional null set $N = \bigcup_k N_{1/k}$
    by intersecting over rational $\varepsilon$. At each stage, the
    construction of $N_{1/k}$ depends on the previous stages' estimates.
    \textbf{This is dependent sequential refinement---precisely DC.}
\end{enumerate}

\begin{remark}[The metastability--convergence gap]\label{rem:metastability-birkhoff}
The proof-mining results of Avigad, Gerhardy, and Towsner \cite{AGT10}
show that \emph{metastable} versions of Birkhoff's theorem are provable
without $\DC$---indeed, without any choice principle. Their result
extracts explicit rates of metastability from the classical proof.
The gap between metastability and full pointwise convergence is exactly
where $\DC$ enters: constructing the actual null set of non-convergence
from the metastable approximations requires dependent refinement.
Metastability says ``for each precision, convergence holds on a large
interval''; full convergence says ``the set of points where the sequence
diverges is null.'' Passing from the former to the latter requires $\DC$.

This is the conceptually sharpest formulation of why Birkhoff's theorem
calibrates to $\DC$: the theorem's content \emph{beyond metastability}
is precisely measured by $\DC$.
\end{remark}

\subsection{Reverse Direction: Birkhoff $\implies$ DC}\label{sec:birkhoff-reverse}

\textbf{Constructive encoding.}
Given a total relation $R$ on $X$ with initial $x_0$, we encode the
$\DC$ problem into a measure-preserving system:

\begin{enumerate}
  \item Build a shift system $(\Omega^{\NN}, \sigma, \mu)$ where $\Omega$
    encodes the available choices at each step and $\mu$ is an appropriate
    product measure.
  \item A dependent choice sequence is a point $\omega \in \Omega^{\NN}$
    satisfying coherence conditions determined by $R$.
  \item Define an observable $f$ whose Birkhoff averages converge iff a
    coherent choice path exists.
  \item Pointwise convergence for $\mu$-a.e.\ $\omega$ yields a valid
    dependent choice sequence.
\end{enumerate}

As with the mean ergodic reverse (\S\ref{sec:obstacle}), in classical
\Lean{} this implication is trivially true because $\DC$ is classically
provable.

\subsection{DC Strictly Above CC}

\begin{proposition}\label{prop:dc-above-cc}
Birkhoff's theorem is strictly above the mean ergodic theorem in the
choice hierarchy: Birkhoff requires $\DC$ while the mean ergodic theorem
requires only $\CC$. In models where $\CC$ holds but $\DC$ fails
(Fraenkel--Mostowski models \cite{Jec73}), the mean ergodic theorem holds
but Birkhoff's theorem fails.
\end{proposition}

The physical interpretation: the mean ergodic theorem gives convergence
of \emph{ensemble averages} (what an experimenter computes from repeated
measurements); Birkhoff gives convergence for \emph{individual
trajectories} (what happens along a single orbit). The former is a
weaker, more operational claim; the latter is a stronger idealization.


% ====================================================================
\section{Quantum Measurement: Weak and Strong Laws}\label{sec:lln}
% ====================================================================

The laws of large numbers provide a parallel calibration in the
probabilistic setting, with a clean physical interpretation via quantum
measurement.

\subsection{Weak Law at the CC Level}

\begin{theorem}\label{thm:wlln} \leanok{}
$\CC$ implies the weak law of large numbers.
\end{theorem}

The mathematical route is via Chebyshev's inequality:
\[
  P\!\left(\left|\frac{S_n}{n} - \mu\right| \geq \varepsilon\right)
  \leq \frac{\mathrm{Var}(X)}{n\varepsilon^2} \to 0.
\]
Chebyshev's inequality is constructive ($\BISH$-valid). The infinite
sequence of i.i.d.\ random variables requires $\CC$---each measurement
is an independent choice from the outcome distribution.

\begin{remark}[Calibration note]\label{rem:wlln-calibration}
The \Lean{} proof of \texttt{weakLLN\_of\_cc} routes through \Mathlib{}'s
strong law (\texttt{strong\_law\_ae\_real}), using the fact that almost-sure
convergence implies convergence in probability for finite measures. This
is a formalization shortcut: the logical strength used (DC level) exceeds
what is necessary. An independent $\CC$-level proof exists via Chebyshev's
inequality, using only constructive variance bounds. The calibration claim
(weak LLN $\leftrightarrow$ $\CC$) rests on the Chebyshev route, not on
the \Lean{} shortcut.
\end{remark}

\subsection{Strong Law at the DC Level}

\begin{theorem}\label{thm:slln} \leanok{}
$\DC$ implies the strong law of large numbers.
\end{theorem}

\textbf{Where DC enters.} The proof (Etemadi 1981, which \Mathlib{}
follows) constructs the exceptional null set via:
\begin{enumerate}
  \item For each $\varepsilon > 0$, find $N_\varepsilon$ where deviations
    exceed $\varepsilon$ at most finitely often (Borel--Cantelli).
  \item The full null set is $\bigcup_k N_{1/k}$.
  \item Each $N_{1/k}$ depends on previous estimates: dependent sequential
    refinement.
\end{enumerate}
This is the same $\DC$ structure as in Birkhoff's theorem.

\subsection{Physical Interpretation}

Consider a quantum system with observable $A$ and state $\rho$. Repeated
measurement produces outcomes $a_1, a_2, a_3, \ldots$

\textbf{CC (ensemble/statistical level)}: The weak law says: for any
tolerance $\varepsilon$ and confidence $\delta$, there exists $N$ such
that after $N$ measurements,
$P(|\bar{a}_N - \mathrm{Tr}(\rho A)| > \varepsilon) < \delta$.
This is what experimentalists actually verify: finite-sample statistics
match the Born rule prediction within specified tolerance and confidence.
The countable choice principle enters because we need the infinite
sequence of measurement outcomes---each measurement is an independent
choice from the outcome distribution.

\textbf{DC (trajectory level)}: The strong law says: with probability~1,
$\bar{a}_n \to \mathrm{Tr}(\rho A)$. This is a statement about
\emph{every individual measurement sequence} (except a null set).
Constructing that exceptional null set requires $\DC$.

The separation reveals that \emph{operational quantum mechanics (what
experimentalists verify) is logically cheaper than the idealized
probability-theoretic formulation.}


% ====================================================================
\section{The DC Ceiling Thesis}\label{sec:ceiling}
% ====================================================================

\subsection{Statement}

\begin{definition}[DC Ceiling Thesis]\label{def:ceiling}
No calibratable physical theorem in the CRM programme requires more
than Dependent Choice. Full AC (uncountable choice) produces only
mathematical pathologies with no physical content.
\end{definition}

This is an empirical observation, not a theorem. The supporting evidence:

\begin{enumerate}
  \item All calibrated physical theories in the programme use at most $\DC$.
  \item Physics operates on separable Hilbert spaces and $\sigma$-finite
    measure spaces, both of which have countable bases. Uncountable choice
    is structurally unnecessary.
  \item The \textbf{Solovay model} (ZF $+$ DC $+$ ``all sets of reals are
    Lebesgue measurable'') is consistent \cite{Sol70} and arguably the
    natural set-theoretic home for mathematical physics. In this model,
    $\DC$ holds but full AC fails, and the pathologies of AC (Vitali sets,
    Banach--Tarski decompositions, non-measurable sets) are absent.
  \item Non-separable spaces that appear in some formulations (Stone--\v{C}ech
    compactification, ultraproducts) are reformulation artifacts, not
    physical necessities.
\end{enumerate}

\subsection{Updated Two-Axis Calibration Table}

The complete calibration table, integrating the choice axis from this
paper with the omniscience axis from prior work:

\begin{center}
\begin{tabular}{@{}lllll@{}}
\toprule
\textbf{Physical theorem} & \textbf{Principle} & \textbf{Axis} & \textbf{Lean} & \textbf{Paper} \\
\midrule
\multicolumn{5}{@{}l}{\emph{Choice axis (this paper):}} \\
Single measurement (Born rule) & $\ACzero$ & Choice & \leanok{} & 25 \\
Mean ergodic theorem & $\equiv \CC$ & Choice & \leanok{} & 25 \\
Weak law of large numbers & $\CC$ & Choice & \leanok{} & 25 \\
Birkhoff's ergodic theorem & $\equiv \DC$ & Choice & \leanpartial{} & 25 \\
Strong law of large numbers & $\DC$ & Choice & \leanok{} & 25 \\
\midrule
\multicolumn{5}{@{}l}{\emph{Omniscience axis (prior papers):}} \\
Bidual-gap witness & $\equiv \WLPO$ & Omniscience & \leanok{} & 2, 7 \\
Heisenberg uncertainty & $\BISH$ & --- & \leanok{} & 6 \\
Thermodynamic limit (Ising) & $\equiv \LPO$ & Omniscience & \leanok{} & 8 \\
\midrule
\multicolumn{5}{@{}l}{\emph{Orthogonal principles:}} \\
Radioactive decay & $\MP$ & Markov & --- & 22 \\
Optimization on compact & $\FT$ & Fan & --- & 23 \\
\bottomrule
\end{tabular}
\end{center}

Legend: \leanok{} = both directions formalized (forward fully, reverse
Type-level or classically trivial);
\leanpartial{} = axiomatized or paper-level only.


% ====================================================================
\section{Lean 4 Formalization}\label{sec:lean}
% ====================================================================

\subsection{Module Structure}

The formalization is organized as a single \Lean{} project with 12 modules.

\begin{table}[ht]
\centering
\begin{tabular}{@{}lrl@{}}
\toprule
\textbf{File} & \textbf{Lines} & \textbf{Purpose} \\
\midrule
\texttt{Basic.lean} & 131 & Core definitions: $\CC$, $\DC$, $\ACzero$, hierarchy proofs \\
\texttt{CesaroAverage.lean} & 172 & Ces\`{a}ro averages: definition + 5 lemmas \\
\texttt{MeanErgodic.lean} & 268 & $\CC \to$ Mean Ergodic Theorem (main proof) \\
\texttt{MeanErgodicReverse.lean} & 80 & Mean Ergodic $\to$ $\CC$ + equivalence \\
\texttt{Computable.lean} & 395 & Type-level reverse: \texttt{MeanErgodicComputableAll} $\to$ $\CC$ \\
\texttt{PointwiseErgodic.lean} & 133 & Birkhoff $\leftrightarrow$ $\DC$ (axiom + reverse) \\
\texttt{PointwiseErgodicReverse.lean} & 54 & $\DC > \CC$ hierarchy proof \\
\texttt{WeakLaw.lean} & 142 & $\CC \to$ Weak LLN \\
\texttt{StrongLaw.lean} & 116 & $\DC \to$ Strong LLN (wraps \Mathlib{}) \\
\texttt{Separation.lean} & 88 & $\ACzero \not\to \CC \not\to \DC$ + DC ceiling \\
\texttt{CalibrationTable.lean} & 54 & Two-axis calibration table (documentation) \\
\texttt{Main.lean} & 172 & Aggregator + \texttt{\#print axioms} audit \\
\midrule
\textbf{Total} & \textbf{1805} & \\
\bottomrule
\end{tabular}
\caption{File manifest.}
\label{tab:manifest}
\end{table}

\subsection{Core Definitions}

\begin{lstlisting}[caption={Choice principles (Basic.lean).}]
def CC_N : Prop :=
  forall (A : Nat -> Set Nat), (forall n, (A n).Nonempty) ->
    exists f : Nat -> Nat, forall n, f n in A n

def DC : Prop :=
  forall (X : Type) (R : X -> X -> Prop),
    (forall x, exists y, R x y) -> forall x0 : X,
      exists f : Nat -> X, f 0 = x0 && forall n, R (f n) (f (n + 1))
\end{lstlisting}

\begin{lstlisting}[caption={Mean Ergodic Theorem statement (MeanErgodic.lean).}]
def MeanErgodicTheorem : Prop :=
  forall (F : Type*) [NormedAddCommGroup F] [InnerProductSpace C F]
    [CompleteSpace F] (U : F ->L[C] F) (_hU : forall z, ||U z|| = ||z||)
    (x : F), exists Px : F, Px in fixedSubspace U &&
      Tendsto (fun n => cesaroAvg U x (n + 1)) atTop (nhds Px)
\end{lstlisting}

\subsection{Main Theorem Snippet}

\begin{lstlisting}[caption={CC $\to$ Mean Ergodic: proof structure (MeanErgodic.lean).}]
theorem meanErgodic_of_cc : CC_N -> MeanErgodicTheorem := by
  intro hcc F _ _ _ U hU x
  let K : Submodule C F := fixedSubspace U
  -- Decompose x = Px + x' where Px in K, x' in K perp
  let Px : K := K.orthogonalProjection x
  let x' : F := x - Px
  refine <<Px, Subtype.mem _, ?_>>
  -- Show A_n(x) -> Px via: A_n(x) - Px = A_n(x') -> 0
  -- Key: x' in closure(Range(U - I)) by adjoint argument
  -- Then: uniform bound + density + CC give convergence
  ...  -- 170 lines of analysis (see full code)
\end{lstlisting}

\subsection{Type-Level Reverse: Key Definitions and Theorem}

\begin{lstlisting}[caption={Computable Mean Ergodic structure (Computable.lean).}]
structure MeanErgodicComputable
    (F : Type) [NormedAddCommGroup F] [InnerProductSpace C F]
    [CompleteSpace F]
    (U : F ->L[C] F) (hU : forall z, ||U z|| = ||z||) where
  proj : F -> F       -- projection (data, not existence)
  modulus : F -> (e : R) -> (0 < e) -> N  -- convergence rate
  proj_fixed : forall x, proj x in fixedSubspace U
  modulus_spec : forall x e (he : 0 < e) (n : N),
    modulus x e he <= n ->
    ||cesaroAvg U x (n + 1) - proj x|| < e
\end{lstlisting}

\begin{lstlisting}[caption={Extraction theorem (Computable.lean, 13~lines).}]
theorem meanErgodicComputableAll_implies_cc
    (h : MeanErgodicComputableAll) : CC_N := by
  intro A hA
  let mec := (h choiceHilbert (reflectCLM A)
    (reflectCLM_isometry A)).some
  have hvanish : forall n m, m not in A n ->
      (mec.proj probeVec : N * N -> C) (n, m) = 0 :=
    (mem_fixedSubspace_reflect_iff A
      (mec.proj probeVec)).mp (mec.proj_fixed probeVec)
  have hnonzero : forall n,
      exists m, (mec.proj probeVec : N * N -> C) (n, m) != 0 :=
    fun n => proj_coord_nonzero A hA mec n
  refine <<fun n => Nat.find (hnonzero n), fun n => ?_>>
  by_contra hmem
  exact absurd (hvanish n _ hmem)
    (Nat.find_spec (hnonzero n))
\end{lstlisting}

\subsection{Axiom Audit}\label{sec:audit}

\begin{lstlisting}[caption={Axiom audit (Main.lean).}]
-- Forward directions (all fully proved):
#print axioms meanErgodic_of_cc
-- [propext, Classical.choice, Quot.sound]
#print axioms weakLLN_of_cc
-- [propext, Classical.choice, Quot.sound]
#print axioms strongLLN_of_dc
-- [propext, Classical.choice, Quot.sound]

-- Reverse directions (classically trivial):
#print axioms meanErgodic_implies_cc
-- [propext, Classical.choice, Quot.sound]
#print axioms dc_of_birkhoff
-- [propext, Classical.choice, Quot.sound]

-- Type-level reverse (non-trivial, hypothesis used):
#print axioms meanErgodicComputableAll_implies_cc
-- [propext, Classical.choice, Quot.sound]

-- Equivalences:
#print axioms meanErgodic_iff_cc
-- [propext, Classical.choice, Quot.sound]  <- CLEAN
#print axioms birkhoff_iff_dc
-- [propext, Classical.choice, Quot.sound,
--  Papers.P25_ChoiceAxis.birkhoff_of_dc]
\end{lstlisting}

\texttt{Classical.choice} in the profiles is a \Lean{} metatheory axiom
(from \Mathlib{}'s use of classical logic in analysis infrastructure),
not an object-level choice principle. The only custom axiom is
\texttt{birkhoff\_of\_dc}---Birkhoff's theorem is not in \Mathlib{}.

A notable detail: the choice hierarchy proofs (\texttt{cc\_n\_of\_dc},
\texttt{ac0\_of\_cc\_n}) have tighter profiles---\texttt{[propext, Quot.sound]}
without \texttt{Classical.choice}. This confirms that the logical
relationships between choice principles are themselves constructively
valid; classical content enters only when doing analysis on Hilbert
spaces and measure spaces.

The Type-level reverse \texttt{meanErgodicComputableAll\_implies\_cc}
has the same axiom profile \texttt{[propext, Classical.choice, Quot.sound]}
as the Prop-level version. The difference is structural:
\texttt{Classical.choice} enters only through \Mathlib{} infrastructure
(decidability instances for set membership), while the hypothesis
\texttt{h : MeanErgodicComputableAll} is \emph{genuinely used}---it
provides the projection, the convergence modulus, and the
fixed-subspace membership proof. Removing the hypothesis breaks the
proof.

\subsection{Formalization Scope and Classical/Constructive Boundary}
\label{sec:scope}

The forward calibrations (choice principle $\to$ physical theorem) are
formalized in \Lean{} with clean axiom profiles. These are the directions
where the mathematical content lives and where formalization adds
confidence:
\begin{itemize}
  \item $\CC \to$ Mean Ergodic: 600+ lines of genuine Hilbert space analysis.
  \item $\CC \to$ Weak LLN: via strong law (calibration note in
    \texttt{WeakLaw.lean}; see \Cref{rem:wlln-calibration}).
  \item $\DC \to$ Strong LLN: wrapping \Mathlib{}'s \texttt{strong\_law\_ae\_real}.
\end{itemize}

The reverse calibrations (physical theorem $\to$ choice principle) present
a unique challenge in classical proof assistants. In \Lean{}, $\CC$ and
$\DC$ hold unconditionally (via \texttt{Classical.choice}), so Prop-level
implications like $\text{Mean Ergodic} \to \CC$ are trivially true---the
antecedent is discarded.

For the mean ergodic reverse direction, this obstacle is overcome in
\texttt{Computable.lean} (395~lines) via a Prop/Type lifting technique.
The Type-level hypothesis \texttt{MeanErgodicComputableAll} provides
projections and convergence moduli as \emph{data}, and the extraction
theorem \texttt{meanErgodicComputableAll\_implies\_cc} genuinely uses
this data to construct the choice function. The hypothesis cannot be
removed (see \S\ref{sec:reverse-lean}).

The Birkhoff reverse direction (\S\ref{sec:birkhoff-reverse}) remains at
paper level. Constructive formalization---for example, in Agda without~K
or Coq without classical axioms---is noted as future work.

The two permanent \texttt{sorry} declarations (\texttt{ac0\_not\_implies\_cc\_n},
\texttt{cc\_n\_not\_implies\_dc}) are model-theoretic independence results.
These cannot be proved in any object-level theory; they require model
construction at the metatheoretic level.

\subsection{Reproducibility}

\begin{mdframed}[backgroundcolor=gray!10]
\textbf{Reproducibility Box}
\begin{itemize}
\item \textbf{Repository}: \url{https://github.com/quantmann/FoundationRelativity}
\item \textbf{Lean toolchain}: \texttt{leanprover/lean4:v4.28.0-rc1}
\item \textbf{mathlib4 commit}: \texttt{22fb569b732913fee24c23c553b2b4e58dcb3206}
\item \textbf{Build}: \texttt{lake exe cache get \&\& lake build}
\item \textbf{Bundle target}: \texttt{Papers}
  (imports \texttt{Main})
\item \textbf{Status}: 0~errors, 2~warnings (permanent model-theoretic sorries).
  12~files, 1805~lines total. 1~custom axiom (\texttt{birkhoff\_of\_dc}).
\item \textbf{Axiom profile}: \texttt{meanErgodic\_iff\_cc}:
  \texttt{[propext, Classical.choice, Quot.sound]}. Clean---no custom axioms.
\item \textbf{Type-level reverse}: \texttt{meanErgodicComputableAll\_implies\_cc}:
  \texttt{[propext, Classical.choice, Quot.sound]}. Hypothesis genuinely used.
\end{itemize}
\end{mdframed}


% ====================================================================
\section{Discussion}\label{sec:discussion}
% ====================================================================

\subsection{The Two-Axis Calibration}

This paper extends the CRM calibration programme from a single axis
(omniscience: $\WLPO$, $\LLPO$, $\LPO$) to a two-axis system by
introducing the choice hierarchy ($\ACzero$, $\CC$, $\DC$). The two
axes are largely orthogonal: $\WLPO$ and $\CC$ are incomparable over
$\BISH$, reflecting the conceptual distinction between decidability
(``can we determine a property?'') and selection (``can we make
infinitely many choices?'').

The choice axis captures a physical distinction that the omniscience
axis does not: the gap between ensemble/average behavior and individual
trajectory behavior. This distinction is fundamental in both ergodic
theory (mean vs.\ pointwise convergence) and probability (convergence
in probability vs.\ almost sure convergence).

\subsection{Relation to Proof Mining}

The proof-mining programme \cite{Koh08,AGT10} has extracted constructive
rates from classical proofs of ergodic theorems. Their key insight is
that metastable versions of convergence theorems are provable without
choice principles. Our results complement this by identifying precisely
what the metastability--convergence gap measures: $\CC$ for the mean
ergodic theorem, $\DC$ for Birkhoff.

A referee familiar with Kohlenbach's work may ask: if metastable ergodic
theorems require no choice, what is new here? The answer is that our
calibration concerns \emph{full convergence}, not metastability. The
choice hierarchy measures the logical cost of passing from ``convergence
holds on arbitrarily long intervals'' (metastability) to ``convergence
holds everywhere except a null set'' (full convergence). This gap is
mathematically genuine and physically meaningful: metastability
corresponds to finite experimental verification; full convergence
is the infinite idealization.

\subsection{Limitations and Future Directions}

\textbf{Constructive formalization.} The mean ergodic reverse direction
now has a non-trivial Type-level formalization in \Lean{} (395~lines in
\texttt{Computable.lean}), where the hypothesis is genuinely used. This
partially addresses the classical triviality obstacle. However, the
Birkhoff reverse direction remains paper-level, and a fully constructive
formalization---in Agda without~K or Coq without classical axioms---would
verify all equivalences in an inherently constructive framework. This
remains future work.

\textbf{Independent weak law proof.} The \Lean{} proof of the weak law
routes through the strong law. An independent Chebyshev-route proof
at the $\CC$ level would make the calibration self-contained in the
formalization, not just at paper level.

\textbf{Continuous spectrum.} Our formulation uses discrete-spectrum
observables for quantum measurement. Extension to continuous-spectrum
observables (position, momentum) would require additional measure-theoretic
machinery.

\textbf{Higher-dimensional systems.} The calibration applies to ergodic
theorems for single transformations. Multi-parameter ergodic theorems
(e.g., for $\ZZ^d$-actions) may require intermediate choice principles
between $\CC$ and $\DC$.


% ====================================================================
\section*{Acknowledgments}
% ====================================================================

The \Lean{} formalization was developed using Claude Opus~4.6
(Anthropic, 2026) via the Claude Code CLI tool. We thank the
\Mathlib{} community for maintaining the comprehensive library
of formalized mathematics that made this work possible.


% ====================================================================
% Bibliography
% ====================================================================
\bibliographystyle{plainnat}

\begin{thebibliography}{30}

\bibitem[Anthropic(2026)]{Anthropic2026}
Anthropic.
\newblock Claude {Opus}~4.6 and {Claude Code} {CLI}.
\newblock \url{https://www.anthropic.com/claude}, 2026.

\bibitem[Avigad et~al.(2010)]{AGT10}
J.~Avigad, P.~Gerhardy, and H.~Towsner.
\newblock Local stability of ergodic averages.
\newblock \emph{Trans.\ Amer.\ Math.\ Soc.}, 362(1):261--288, 2010.

\bibitem[Birkhoff(1931)]{Bir31}
G.~D.~Birkhoff.
\newblock Proof of the ergodic theorem.
\newblock \emph{Proc.\ Natl.\ Acad.\ Sci.\ USA}, 17(12):656--660, 1931.

\bibitem[Bishop(1967)]{Bis67}
E.~Bishop.
\newblock \emph{Foundations of Constructive Analysis}.
\newblock McGraw-Hill, New York, 1967.

\bibitem[Bishop and Bridges(1985)]{BB85}
E.~Bishop and D.~Bridges.
\newblock \emph{Constructive Analysis}.
\newblock Grundlehren der mathematischen Wissenschaften 279,
  Springer-Verlag, Berlin, 1985.

\bibitem[Bridges and V\^{\i}\c{t}\u{a}(2006)]{BV06}
D.~Bridges and L.~V\^{\i}\c{t}\u{a}.
\newblock \emph{Techniques of Constructive Analysis}.
\newblock Universitext, Springer, New York, 2006.

\bibitem[Etemadi(1981)]{Ete81}
N.~Etemadi.
\newblock An elementary proof of the strong law of large numbers.
\newblock \emph{Z.~Wahrsch.~Verw.~Gebiete}, 55(1):119--122, 1981.

\bibitem[Ishihara(2006)]{Ish06}
H.~Ishihara.
\newblock Reverse mathematics in Bishop's constructive mathematics.
\newblock \emph{Phil.\ Sci.}, Cahier Sp\'{e}cial 6:43--59, 2006.

\bibitem[Jech(1973)]{Jec73}
T.~Jech.
\newblock \emph{The Axiom of Choice}.
\newblock Studies in Logic and the Foundations of Mathematics 75,
  North-Holland, Amsterdam, 1973.

\bibitem[Kohlenbach(2008)]{Koh08}
U.~Kohlenbach.
\newblock \emph{Applied Proof Theory: Proof Interpretations and their
  Use in Mathematics}.
\newblock Springer Monographs in Mathematics, Springer, Berlin, 2008.

\bibitem[Lee(2026a)]{Lee26a}
P.~C.~Lee.
\newblock The bidual gap and WLPO: A Lean~4 formalization.
\newblock Paper~2 in the Foundation Relativity series, 2026.

\bibitem[Lee(2026b)]{Lee26b}
P.~C.~Lee.
\newblock Reflexive Banach spaces and WLPO dispensability.
\newblock Paper~7 in the Foundation Relativity series, 2026.

\bibitem[Lee(2026c)]{Lee26Bell}
P.~C.~Lee.
\newblock Bell's theorem and LLPO.
\newblock Paper~21 in the Foundation Relativity series, 2026.

\bibitem[Lee(2026d)]{Lee26Ising}
P.~C.~Lee.
\newblock The logical cost of the thermodynamic limit: LPO-equivalence
  and BISH-dispensability for the 1D Ising free energy.
\newblock Paper~8 in the Foundation Relativity series, 2026.

\bibitem[Nuber(1972)]{Nub72}
J.~A.~Nuber.
\newblock A constructive ergodic theorem.
\newblock \emph{Trans.\ Amer.\ Math.\ Soc.}, 164:115--137, 1972.

\bibitem[Simpson(2009)]{Sim09}
S.~G.~Simpson.
\newblock \emph{Subsystems of Second Order Arithmetic}.
\newblock Perspectives in Logic, Cambridge University Press, 2nd ed., 2009.

\bibitem[Solovay(1970)]{Sol70}
R.~M.~Solovay.
\newblock A model of set-theory in which every set of reals is Lebesgue
  measurable.
\newblock \emph{Ann.\ of Math.}, 92(1):1--56, 1970.

\bibitem[Spitters(2006)]{Spi06}
B.~Spitters.
\newblock Constructive algebraic integration theory.
\newblock \emph{Ann.\ Pure Appl.\ Logic}, 137(1--3):380--390, 2006.

\bibitem[Tao(2008)]{Tao08}
T.~Tao.
\newblock Soft analysis, hard analysis, and the finite convergence
  principle.
\newblock In T.~Tao, \emph{Structure and Randomness}, pp.~298--343,
  AMS, 2008.

\bibitem[Veldman(2005)]{Vel05}
W.~Veldman.
\newblock Brouwer's fan theorem as an axiom and as a contrast to Kleene's
  alternative.
\newblock \emph{Arch.\ Math.\ Logic}, 44(7):869--883, 2005.

\bibitem[von~Neumann(1932)]{vN32}
J.~von~Neumann.
\newblock Proof of the quasi-ergodic hypothesis.
\newblock \emph{Proc.\ Natl.\ Acad.\ Sci.\ USA}, 18(1):70--82, 1932.

\bibitem[Ye(2011)]{Ye11}
F.~Ye.
\newblock \emph{Strict Finitism and the Logic of Mathematical Applications}.
\newblock Synthese Library 355, Springer, Dordrecht, 2011.

\end{thebibliography}

% ====================================================================
\section*{AI-Assisted Methodology}\label{sec:ai}
% ====================================================================

This formalization was developed using \textbf{Claude Opus~4.6}
(Anthropic, 2026) via the \textbf{Claude Code} command-line interface,
following the same human--AI workflow as Papers~2, 7, 8, and~21--24
\cite{Lee26a,Lee26b,Lee26Ising,Anthropic2026}. The development proceeded
in two phases:

\begin{itemize}
\item \textbf{Phase~1}: The human author wrote mathematical blueprints
  for all theorem statements and proof strategies. Claude Opus~4.6
  located \Mathlib{} API signatures, generated \Lean{} proof terms,
  and handled debugging. This produced the original 11-file bundle
  (forward directions, classically trivial reverses, calibration).
\item \textbf{Phase~2}: Following the human author's analysis of the
  Prop/Type distinction and the classical triviality obstacle, a second
  round of AI-assisted formalization produced \texttt{Computable.lean}
  (395~lines): the non-trivial Type-level reverse direction with the
  $\ell^2(\NN \times \NN)$ encoding.
\end{itemize}

The human author reviewed all proofs for mathematical correctness and
\Mathlib{} conventions. Final verification was by
\texttt{lake build} (0~errors, 2~warnings from permanent sorries).

\begin{table}[h]
\centering
\begin{tabular}{@{}lll@{}}
\toprule
\textbf{Task} & \textbf{Human} & \textbf{AI (Claude Opus 4.6)} \\
\midrule
Mathematical blueprint    & \checkmark & \\
Proof strategy design     & \checkmark & \\
\Mathlib{} API discovery  & & \checkmark \\
\Lean{} proof generation  & & \checkmark \\
Proof review              & \checkmark & \\
Build verification        & & \checkmark \\
Paper writing             & \checkmark & \checkmark \\
\bottomrule
\end{tabular}
\caption{Division of labor between human and AI.}
\label{tab:division}
\end{table}

% ====================================================================
\appendix
\section{Selected Lean Code}\label{sec:appendix}
% ====================================================================

\subsection{Key Lemma: Orthogonal Complement of Range$(U - I)$}

\begin{lstlisting}[caption={orthogonal\_range\_sub\_le\_fixed (MeanErgodic.lean).}]
theorem orthogonal_range_sub_le_fixed (U : E ->L[C] E)
    (hU : forall z, ||U z|| = ||z||)
    (z : E) (hz : forall y, <<z, U y - y>>_C = 0) :
    z in fixedSubspace U := by
  rw [mem_fixedSubspace_iff]
  have h1 : forall y, <<z, U y>>_C = <<z, y>>_C := by
    intro y; have := hz y
    rw [inner_sub_right] at this; rwa [sub_eq_zero] at this
  rw [<- sub_eq_zero, <- inner_self_eq_zero]
  rw [inner_sub_left, inner_sub_right, inner_sub_right]
  have hUznorm : <<U z, U z>>_C = <<z, z>>_C := by
    rw [inner_self_eq_norm_sq_to_K, inner_self_eq_norm_sq_to_K, hU z]
  have h2 : <<z, U z>>_C = <<z, z>>_C := h1 z
  have h3 : <<U z, z>>_C = <<z, z>>_C := by
    rw [<- inner_conj_symm, h2, inner_conj_symm]
  rw [hUznorm, h2, h3]; ring
\end{lstlisting}

\subsection{Equivalence Theorems}

\begin{lstlisting}[caption={Equivalences (MeanErgodicReverse.lean, PointwiseErgodic.lean).}]
-- CC <-> Mean Ergodic: clean axiom profile
theorem meanErgodic_iff_cc : CC_N <-> MeanErgodicTheorem :=
  <<meanErgodic_of_cc, meanErgodic_implies_cc>>

-- DC <-> Birkhoff: depends on birkhoff_of_dc axiom only
theorem birkhoff_iff_dc : DC <-> BirkhoffErgodicTheorem :=
  <<birkhoff_of_dc, dc_of_birkhoff>>
\end{lstlisting}


\subsection{Type-Level Reverse: Reflection Operator}

\begin{lstlisting}[caption={Diagonal reflection (Computable.lean).}]
-- The pointwise action: fix at A-coordinates, negate elsewhere
def reflectFun (A : N -> Set N) (f : N * N -> C) : N * N -> C :=
  fun <<n, m>> => if m in A n then f (n, m) else -f (n, m)

-- Isometry: ||U f|| = ||f||  (involution with eigenvalues +/-1)
theorem reflectCLM_isometry (A : N -> Set N) :
    forall z : choiceHilbert, ||reflectCLM A z|| = ||z||

-- Fixed subspace: Fix(U) = {f : f(n,m) = 0 when m not in A(n)}
theorem mem_fixedSubspace_reflect_iff (A : N -> Set N)
    (f : choiceHilbert) :
    f in fixedSubspace (reflectCLM A) <->
    forall n m, m not in A n -> (f : N * N -> C) (n, m) = 0
\end{lstlisting}

\begin{lstlisting}[caption={Probe vector and coordinate stability (Computable.lean).}]
-- Probe: x_0(n,m) = 1/(2^n * 2^m), all coords nonzero, in l^2
def probeVec : choiceHilbert :=
  <<fun i => probeCoeff i.1 i.2, probe_memlp>>

-- Cesaro average at A-coordinate = original value (constant)
theorem cesaroAvg_coord_mem (A : N -> Set N) (n m : N)
    (hm : m in A n) {N : N} (hN : 0 < N) :
    (cesaroAvg (reflectCLM A) probeVec N : N * N -> C) (n, m) =
    probeCoeff n m
\end{lstlisting}

\end{document}

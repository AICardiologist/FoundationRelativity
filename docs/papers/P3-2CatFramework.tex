\documentclass[11pt]{article}

% -------------------------------------------------
% Basic packages
% -------------------------------------------------
\usepackage[T1]{fontenc}
\usepackage[american]{babel}
\usepackage{lmodern}
\usepackage{geometry}
\geometry{margin=1.1in}
\usepackage{microtype}
\usepackage{enumitem}

% ---------- reproducibility & tooling macros ----------
\usepackage{hyperref}   % ensure hyperlinks work

\newcommand{\leanRepoTag}{%
  \href{https://github.com/AICardiologist/FoundationRelativity/tree/v0.3.4-rnp3-complete}%
       {v0.3.4-rnp3-complete}}

% Detailed LLM-assistance disclosure
\newcommand{\llmNote}{%
  Preliminary drafting, proof-sketch generation, and Lean-code
  refactoring benefited from large-language-model assistance:
  OpenAI \textbf{o3-pro} (proof completion),
  Anthropic \textbf{Claude Code} (\emph{Sonnet} \& \emph{Opus} 4.0) for Lean tactics,
  \textbf{xAI Grok 4 Heavy}, and Google/DeepMind \textbf{Gemini 2.5 Pro}
  for critique and editorial suggestions.
}
% ------------------------------------------------------


\setlist[enumerate,1]{label=\textnormal{(\alph*)}}

% Math & theorem tools
\usepackage{amsmath,amssymb,mathtools}
\usepackage{amsthm}
\newtheorem{theorem}{Theorem}[section]
\newtheorem{lemma}[theorem]{Lemma}
\newtheorem{proposition}[theorem]{Proposition}
\newtheorem{corollary}[theorem]{Corollary}
\theoremstyle{definition}
\newtheorem{definition}[theorem]{Definition}
\theoremstyle{remark}
\newtheorem{remark}[theorem]{Remark}

% Graphics & colour for GAP boxes
\usepackage{xcolor}
\usepackage{tikz}
\usetikzlibrary{arrows.meta,positioning,decorations.markings,calc,cd,patterns}
\usepackage{mdframed}
\mdfdefinestyle{gapbox}{%
  backgroundcolor=gray!10,
  linecolor=red!70!black,
  linewidth=1.0pt,
  leftmargin=0pt,
  rightmargin=0pt,
  innerleftmargin=6pt,
  innerrightmargin=6pt,
  innertopmargin=4pt,
  innerbottommargin=4pt
}

% Index setup
\usepackage{imakeidx}
\makeindex

% Small arrow for 2-cells
\newcommand{\Rto}{\,\Rightarrow\,}
% New commands for pathology functors
\newcommand{\APFail}{\mathsf{AP\_Fail}}
\newcommand{\RNPFail}{\mathsf{RNP\_Fail}}
\newcommand{\SDual}{\mathsf{SDual}}
\newcommand{\TwoCat}{\mathbf{2Cat}}

\title{A 2--Categorical Framework for Foundation--Relativity\\
  \large{Preliminary Notes and Open Problems}}
\author{Paul Chun-Kit Lee\\
  \small New York University, New York}
\date{July 17, 2025}

\begin{document}
\maketitle
\thispagestyle{empty}

\begin{abstract}
  These notes present a comprehensive 2-categorical framework for understanding
  foundation-relativity in mathematics, demonstrating how fundamental analytic
  phenomena transform across different foundational systems. We establish:
  (i) the \emph{2-category of mathematical foundations}~\(\mathbf{Found}\),
  where interpretations compose up to coherent isomorphism;
  (ii) the \emph{Functorial Obstruction Theorem}, showing that constructions
  creating non-constructive principles fail to extend pseudofunctorially;
  (iii) the \emph{Gap 2-functor}
  \(\mathrm{Gap}\colon\mathbf{Found}^{\mathrm{op}}\to\mathbf{2Cat}\),
  capturing how witness spaces vary contravariantly across foundations,
  along with explicit constructions of the \(\mathsf{AP\_Fail}\) and
  \(\mathsf{RNP\_Fail}\) 2-functors for other pathologies;
  (iv) a \emph{quantitative relativity degree} \(\rho\) that precisely
  measures the logical strength needed for phenomena (bidual gap and AP
  failure at \(\rho=1\), RNP at \(\rho=2\));
  (v) the \emph{Gödel-Gap correspondence}: a Boolean-lattice isomorphism 
  between closed \(\Pi^0_1\)
  sentences of PA and a canonical sublattice of \(\ell^\infty/c_0\),
  revealing deep connections between logical incompleteness and analytic
  non-reflexivity; and
  (vi) a powerful \emph{foundation-hopping method} that extracts explicit
  quantitative bounds by transporting constructive
  \(\neg\neg\)-facts
  through DNS-TT to classical choice. 
  Together, these results establish that foundation-relativity is not
  merely about different axioms but reflects structural incompatibilities
  measurable by precise categorical invariants. The framework uses conventional 2-categories, avoiding higher homotopy structures, with full bicategorical coherence established via the Gordon-Power-Street theorem.
\end{abstract}

[Document continues with full Paper 3 content...]

\end{document}
\documentclass[12pt,a4paper]{article}

%% ── packages ──────────────────────────────────────────────
\usepackage[utf8]{inputenc}
\usepackage[T1]{fontenc}
% \usepackage{lmodern} % not available in this environment
\usepackage{amsmath,amssymb,amsthm}
\usepackage{mathrsfs}
\usepackage{booktabs}
\usepackage{hyperref}
\usepackage{enumitem}
\usepackage[margin=1.1in]{geometry}
\usepackage{tikz-cd}
\usepackage{cleveref}

%% ── theorem environments ─────────────────────────────────
\theoremstyle{plain}
\newtheorem{theorem}{Theorem}[section]
\newtheorem{proposition}[theorem]{Proposition}
\newtheorem{lemma}[theorem]{Lemma}
\newtheorem{corollary}[theorem]{Corollary}
\newtheorem{conjecture}[theorem]{Conjecture}

\theoremstyle{definition}
\newtheorem{definition}[theorem]{Definition}
\newtheorem{example}[theorem]{Example}
\newtheorem{remark}[theorem]{Remark}

\theoremstyle{remark}
\newtheorem*{notation}{Notation}
\newtheorem*{convention}{Convention}

%% ── macros ───────────────────────────────────────────────
\newcommand{\BISH}{\mathrm{BISH}}
\newcommand{\MP}{\mathrm{MP}}
\newcommand{\LPO}{\mathrm{LPO}}
\newcommand{\CLASS}{\mathrm{CLASS}}
\newcommand{\ZZ}{\mathbb{Z}}
\newcommand{\QQ}{\mathbb{Q}}
\newcommand{\RR}{\mathbb{R}}
\newcommand{\CC}{\mathbb{C}}
\newcommand{\FF}{\mathbb{F}}
\newcommand{\Adeles}{\mathbb{A}}
\newcommand{\HH}{\mathbb{H}}
\newcommand{\Sha}{\text{\cyrshort{Sh}}}  % Sha — if cyrillic unavailable, use:
\renewcommand{\Sha}{\mathrm{III}}
\newcommand{\Gal}{\mathrm{Gal}}
\newcommand{\Frob}{\mathrm{Frob}}
\newcommand{\Tr}{\mathrm{Tr}}
\newcommand{\CRM}{\mathrm{CRM}}

%% ── title ────────────────────────────────────────────────
\title{%
  \textbf{The Decidability Conduit}\\[6pt]
  \large Why Physics and Number Theory Share \\
  a Logical Architecture: CRM Signatures \\
  Across the Langlands Correspondence}

\author{
  Paul Chun-Kit Lee\\
  \small CRM Programme, Papers 1--52\\
  \small \texttt{[affiliation]}
}

\date{February 2026}

%% ═════════════════════════════════════════════════════════
\begin{document}
\maketitle

%% ── abstract ─────────────────────────────────────────────
\begin{abstract}
Over the past five years, the Constructive Reverse Mathematics (CRM)
programme has calibrated the logical complexity of physical theories
(Papers~1--42) and the central conjectures of arithmetic geometry
(Papers~45--50).  Both domains exhibit the same logical architecture:
a base of Bishop-style constructive mathematics ($\BISH$), a single
non-constructive principle ($\LPO$: the Limited Principle of
Omniscience) for operations on complete fields, and positive-definiteness
at the Archimedean place as the universal descent mechanism converting
$\LPO$-level spectral data to $\BISH$-level algebraic data.  Universal
conjectures in both fields---the spectral gap in physics, the Standard
Conjectures in arithmetic geometry---live at the same level of the
arithmetic hierarchy ($\Pi^0_2$).

The Langlands correspondence asserts that the motivic side (algebraic
geometry) and the automorphic side (spectral theory on adelic groups)
carry the same arithmetic data.  We show that this correspondence
preserves CRM signatures: each of the three motivic axioms
(decidable morphisms, algebraic spectrum, Archimedean polarization)
transfers to a substantive theorem on the automorphic side
(Strong Multiplicity One, Shimura algebraicity, Petersson
positive-definiteness).

A critical asymmetry emerges.  The automorphic side is
CRM-incomplete for eigenvalue bounds: the Petersson inner product
yields only the trivial unitary bound, not the sharp Ramanujan
bound.  Deligne's proof of the Ramanujan conjecture for holomorphic
modular forms works by \emph{crossing the Langlands bridge} to the
motivic side, where rigid intersection geometry over finite fields
enforces $\BISH$-level bounds.  The correspondence is therefore
not merely a transfer of data but a \emph{mandatory decidability
conduit}: the automorphic side requires the motivic side's
constructive axioms to bound its own spectrum.

We identify the Selberg trace formula as a de-omniscientizing descent
equation (spectral side: $\LPO$; geometric side: $\BISH$),
mathematically identical to the Gutzwiller trace formula in quantum
chaos.  Three spectral gap problems---the Hamiltonian spectral gap
(physics), the Selberg eigenvalue conjecture (automorphic), and
finiteness of the Shafarevich--Tate group (arithmetic)---are shown
to be $\Sigma^0_2$ statements with identical logical structure.

This is the final paper of the CRM programme.  The arc runs from
the Schwarzschild metric (Paper~1) to the Langlands correspondence
(Paper~52), through a single diagnostic question: what does each
mathematical structure \emph{need}?
\end{abstract}

\tableofcontents

%% ══════════════════════════════════════════════════════════
\section{Introduction}\label{sec:intro}
%% ══════════════════════════════════════════════════════════

\subsection{The Diagnostic Question}

Constructive Reverse Mathematics asks, for each theorem~$T$:
\emph{which non-constructive principles are logically equivalent
to~$T$?}  The standard hierarchy is
\[
  \BISH \;\subset\; \MP \;\subset\; \mathrm{LLPO}
    \;\subset\; \LPO \;\subset\; \CLASS,
\]
where $\BISH$ (Bishop's constructive mathematics) uses no
non-constructive principles, $\MP$ (Markov's Principle) allows
unbounded search, $\LPO$ (the Limited Principle of Omniscience)
allows deciding whether an infinite binary sequence contains a~1,
and $\CLASS$ is full classical logic.  See \cite{Bridges-Richman}
for the standard reference.

In classical reverse mathematics \cite{Simpson}, theorems of
second-order arithmetic are calibrated against five subsystems of
analysis.  CRM performs the analogous calibration for constructive
logic, with the crucial difference that the hierarchy measures
\emph{computational content}: $\BISH$ means ``every computation
terminates,'' $\MP$ means ``unbounded search may be required,''
and $\LPO$ means ``testing whether an infinite limit equals zero
requires a non-computable oracle.''

\subsection{What the Physics Programme Found}

Papers~1--42 calibrated the logical cost of physical theories
from general relativity to quantum field theory.  The result was
unexpectedly uniform:

\begin{quotation}
\emph{Every empirically accessible physical theory requires at most
$\BISH + \LPO$.  The spectral gap problem reaches $\LPO'$
($\Pi^0_2$ in the arithmetic hierarchy).  Positive-definiteness of
the Hilbert-space inner product is the universal mechanism converting
$\LPO$ to $\BISH$.}
\end{quotation}

The $\LPO$ cost arises wherever physics evaluates an infinite
limit: thermodynamic limits, path integrals, geodesic completeness.
The $\BISH$ rescue arises wherever physics extracts a finite
prediction: energy eigenvalues, scattering cross-sections,
measurable expectation values.  The conversion mechanism is the
positive-definite inner product $\langle\psi|\varphi\rangle$,
which permits division (a $\BISH$ operation on a space where
zero-testing would otherwise require $\LPO$).

Paper~39 found the ceiling.  The Cubitt--Perez-Garcia--Wolf
spectral gap undecidability result \cite{CPW} lives at $\Pi^0_2$:
one level above $\LPO$ in the arithmetic hierarchy.  It asks a
universal question (for all lattice sizes~$N$) about an
$\LPO$-level predicate (the energy gap).  The universal
quantification is what pushes it beyond decidability.

\subsection{What the Motivic Programme Found}

Papers~45--50 turned the same instrument on arithmetic geometry.
Five central conjectures were calibrated: the Weight-Monodromy
Conjecture (Paper~45, formally verified in Lean~4 with zero
sorries), the Tate Conjecture (Paper~46), the Fontaine--Mazur
Conjecture (Paper~47), the Birch and Swinnerton-Dyer Conjecture
(Paper~48), and the Hodge Conjecture (Paper~49).

The result was again uniform.  Every conjecture exhibited the same
pattern, which we call \emph{de-omniscientizing descent}: continuous
homological or analytic data over a complete field (requiring $\LPO$
for zero-testing) secretly descends to discrete algebraic data over
$\QQ$ (decidable in $\BISH$ or $\BISH + \MP$).

Paper~50 \cite{P50} crystallized this pattern into a characterization
of Grothendieck's motive.  We summarize the key findings for readers
coming from physics (\S\ref{sec:p50-summary}), after first fixing the
formal framework (\S\ref{sec:metatheory}).

%% ══════════════════════════════════════════════════════════
\section{Metatheory: Formal CRM Signatures}\label{sec:metatheory}
%% ══════════════════════════════════════════════════════════

A reviewer of an earlier draft correctly observed that the claims of
this paper require a formal notion of ``CRM signature'' and
``preserves'' that can be checked as theorems rather than read as
compelling analogies.  This section provides the formal framework.
We work in Bishop's constructive mathematics ($\BISH$) as
formalized in Lean~4 against the Mathlib library, following
\cite{Bridges-Richman}.

\subsection{The CRM Signature of a Mathematical Structure}

\begin{definition}[CRM Signature]\label{def:crm-sig}
Let $\mathcal{S}$ be a mathematical structure involving operations
on elements of a complete valued field $k$ (such as $\RR$,
$\QQ_\ell$, or $\QQ_p$).  The \emph{CRM signature} of
$\mathcal{S}$ is the tuple
\[
  \CRM(\mathcal{S}) \;=\; (Z, I, P) \;\in\;
  \{\BISH, \MP, \LPO\}^3
\]
where:
\begin{enumerate}[label=(\roman*),nosep]
  \item $Z$ (Zero-testing): the minimal principle required to decide
    $x = 0$ for elements $x$ in the morphism/function spaces of
    $\mathcal{S}$.
  \item $I$ (Integrality): the minimal principle required to
    verify that characteristic polynomials of endomorphisms have
    integer coefficients.
  \item $P$ (Polarization): the minimal principle required to
    establish a positive-definite bilinear form on the
    endomorphism algebra tensored with~$\RR$.
\end{enumerate}
Each component is assigned by the following decision procedure:
$\BISH$ if the operation can be performed by a terminating
algorithm with no oracle; $\MP$ if the operation reduces to an
unbounded existential search $\exists n\; P(n)$ where $P$ is
decidable; $\LPO$ if the operation requires deciding
$\exists n\; (\alpha(n) = 1)$ for an arbitrary binary sequence
$\alpha$.
\end{definition}

\begin{remark}
The assignment of CRM levels is made precise by exhibiting
Weihrauch reductions \cite{BrattkaGherardi} or, equivalently,
by constructing Lean~4 terms that witness the computation given
the relevant oracle.  In the Lean formalization, $\BISH$
corresponds to a \texttt{Decidable} instance with no axioms
beyond Mathlib; $\MP$ corresponds to requiring a
\texttt{Classical.choice} on a $\Sigma^0_1$ predicate; $\LPO$
corresponds to requiring \texttt{Classical.em} (or equivalently,
a decision procedure for the halting of binary sequences).
\end{remark}

\subsection{Formal Equivalences}

The following are the flagship equivalences that ground the
dictionary.  Each is stated in a form that is formalizable
in the base theory $\BISH$ (or $\BISH$ + named axiom).

\begin{theorem}[Zero-Testing Equivalence for $\QQ_\ell$]\label{thm:zero-test}
Let $V$ be a finite-dimensional $\QQ_\ell$-vector space equipped
with a $\QQ$-structure $V_\QQ \subset V$.
\begin{enumerate}[label=\textup{(\alph*)},nosep]
  \item Deciding $v = 0$ for arbitrary $v \in V$ is equivalent to
    $\LPO$ over $\BISH$.
  \item Deciding $v = 0$ for $v \in V_\QQ$ is $\BISH$-computable
    (no additional axiom).
  \item Grothendieck's Standard Conjecture~D, restricted to a
    given variety~$X$, asserts that the natural map
    $\mathrm{Hom}_{\mathrm{num}}(X) \to
    \mathrm{Hom}_{\mathrm{hom}}(X)$
    is an isomorphism.  This is equivalent to:
    zero-testing in $\mathrm{Hom}_{\mathrm{hom}}$
    ($\LPO$-level) reduces to zero-testing in
    $\mathrm{Hom}_{\mathrm{num}}$ ($\BISH$-level).
\end{enumerate}
\end{theorem}

\begin{proof}
Part (a): $\QQ_\ell$ is the completion of $\QQ$ at $\ell$.
An element $x \in \QQ_\ell$ is a Cauchy sequence of rationals.
Deciding $x = 0$ requires determining whether all terms
eventually vanish---this is $\LPO$ (deciding
$\forall n\, |x_n| < 2^{-n}$ for the Cauchy representatives).
The equivalence with $\LPO$ is classical in constructive
analysis; see \cite[Prop.~1.2]{Bridges-Richman}.

Part (b): $V_\QQ$ has a basis over $\QQ$.  Coordinates are
rationals.  Deciding $v = 0$ is checking finitely many rational
equalities, each decidable by the Euclidean algorithm.

Part (c): Conjecture~D provides a $\QQ$-rational basis for
$\mathrm{Hom}_{\mathrm{hom}}$ via numerical equivalence classes.
Intersection numbers are integers.  The reduction
$\LPO \to \BISH$ is given by: compute the integer intersection
pairing, check if it is zero.  Conversely, if such a reduction
exists, it provides the isomorphism.
\end{proof}

\begin{theorem}[Integrality Equivalence]\label{thm:integrality}
Let $\varphi$ be an endomorphism of an object in a
$\QQ$-linear abelian category.
\begin{enumerate}[label=\textup{(\alph*)},nosep]
  \item On the motivic side: $\varphi = \Frob_p$ acting on
    $H^i_\ell(X)$ has characteristic polynomial in $\ZZ[t]$
    (Weil).  Verifying integrality of the coefficients is
    $\BISH$ (integer arithmetic on the coefficients of $P(t)$).
  \item On the automorphic side: $a_p(f)$ for a Hecke eigenform
    $f$ is an algebraic integer.  Verifying this requires the
    Eichler--Shimura isomorphism (for $\mathrm{GL}_2$) or
    Clozel's purity theorem.  Once established, the verification
    is $\BISH$ (computing in a number field).
  \item The Langlands correspondence sends (a) to (b):
    $a_p(\pi_M) = \Tr(\Frob_p \mid M)$.  Integrality on both
    sides is $\BISH$, and the correspondence preserves this level.
\end{enumerate}
\end{theorem}

\begin{theorem}[Polarization Equivalence]\label{thm:polarization}
\begin{enumerate}[label=\textup{(\alph*)},nosep]
  \item On the motivic side: the Rosati involution on
    $\mathrm{End}(A) \otimes \RR$ yields a positive-definite form
    because $u(\RR) = 1$.  Constructing this form is $\BISH$
    (finite-dimensional linear algebra over $\QQ$ embedded in
    $\RR$).  The form fails to be positive-definite over $\QQ_p$
    because $u(\QQ_p) = 4$.
  \item On the automorphic side: the Petersson inner product on
    $S_k(\Gamma)$ is positive-definite.  For the finite-dimensional
    space of cusp forms of fixed weight and level, computing the
    Petersson norm is $\BISH$ (numerical integration on a
    fundamental domain with known bounds).
  \item Both positive-definite forms arise at the Archimedean
    place, both from $u(\RR) = 1$, and both are $\BISH$-computable
    on finite-dimensional subspaces.  Both fail $p$-adically.
\end{enumerate}
\end{theorem}

\subsection{The Ramanujan Separation Theorem}

The reviewers' challenge to produce a formal separation has a
decisive answer.  We do not need a constructive realizability
model: the separation exists in classical mathematics.

\begin{definition}[CRM-Completeness for Eigenvalue Bounds]
\label{def:crm-complete}
A triple of CRM axioms $(Z, I, P)$ is \emph{CRM-complete for
eigenvalue bounds} if, working in $\BISH$ augmented by these
axioms, one can derive: for every endomorphism $\varphi$ with
$P(\varphi, \varphi) = q^w \cdot P(\mathrm{id}, \mathrm{id})$,
the eigenvalues of $\varphi$ satisfy $|\alpha| = q^{w/2}$.
\end{definition}

The motivic axioms are CRM-complete for eigenvalue bounds
(Paper~50, Theorem~A).

\begin{theorem}[Automorphic CRM Incompleteness]
\label{thm:separation}
The automorphic CRM axioms $\mathrm{A1}$ (Strong Multiplicity
One) $+$ $\mathrm{A2}$ (Shimura algebraicity) $+$ $\mathrm{A3}$
(Petersson positive-definiteness) do not imply the
Ramanujan--Petersson bound.
\end{theorem}

\begin{proof}
We give two arguments: an abstract local separation
(\ref{thm:separation}.a) and a classical global witness
(\ref{thm:separation}.b).

\medskip\noindent\textbf{(a) Local separation (abstract model).}
Take weight $k = 2$ and prime $p = 5$.  The Ramanujan bound
requires $|a_5| \leq 2\sqrt{5} \approx 4.472$.  The unitary
(complementary series) bound for $\mathrm{GL}_2(\QQ_5)$ allows
$|a_5| < 5 + 1 = 6$.  The integer $a_5 = 5$ lies in the gap
$(4.472, 6)$.

The normalized Satake equation $5^{1/2}(5^s + 5^{-s}) = 5$ gives
$s \approx -0.298$.  Since $|s| < 1/2$, this parametrizes a
representation in the complementary series: unitary ($\mathrm{A3}$
satisfied), with Hecke eigenvalue $5 \in \ZZ$ ($\mathrm{A2}$
satisfied), and one-dimensional eigenspace ($\mathrm{A1}$
satisfiable in any synthetic universe).  But $5 > 2\sqrt{5}$,
so Ramanujan fails.

By the soundness of first-order logic, since a model of the
premises exists in which the conclusion is false, the premises
do not entail the conclusion.

\medskip\noindent\textbf{(b) Global witness: Saito--Kurokawa
lifts on $\mathrm{Sp}_4$.}
A reviewer may object that part~(a) constructs a local
representation, not a global automorphic representation.
Global objects satisfy additional constraints (being
subquotients of $L^2$, satisfying functional equations,
possessing analytic conductors).

Saito--Kurokawa lifts \cite{PiatetskiShapiro-SK} are genuine
cuspidal automorphic representations of $\mathrm{Sp}_4(\Adeles)$
satisfying every global constraint:
\begin{itemize}[nosep]
  \item $\mathrm{A1}$: Multiplicity one holds for
    Saito--Kurokawa packets (Arthur's classification
    \cite{Arthur}).
  \item $\mathrm{A2}$: They are functorial lifts from classical
    modular forms on $\mathrm{GL}_2$; their Hecke eigenvalues
    are explicitly algebraic integers.
  \item $\mathrm{A3}$: They are square-integrable cuspidal
    Siegel modular forms; local components are unitary.
\end{itemize}
However, Saito--Kurokawa lifts are CAP representations (Cuspidal
Associated to Parabolic).  Their unramified local components at
almost all primes lie in the non-tempered complementary series,
strictly violating the generalized Ramanujan bound for
$\mathrm{Sp}_4$.

If $\mathrm{A1} + \mathrm{A2} + \mathrm{A3}$ implied the
Ramanujan bound for all reductive groups, the implication would
apply to $\mathrm{Sp}_4$ and contradict the existence of
Saito--Kurokawa lifts---objects in classical mathematics.
Therefore $\mathrm{A1} + \mathrm{A2} + \mathrm{A3}
\not\vdash \text{Ramanujan}$.
\end{proof}

\begin{remark}[The CAP subtlety]\label{rem:cap}
The refined Ramanujan conjecture for $\mathrm{GL}_n$ restricts
attention to representations that are not CAP---i.e., not
functorial lifts from proper Levi subgroups.  The CRM axioms
$\mathrm{A1}$--$\mathrm{A3}$ cannot distinguish CAP from
non-CAP representations.  Making this distinction requires
\emph{functoriality}---the ability to test whether $\pi$ is a
lift from a smaller group.  This is precisely the missing axiom
identified below.
\end{remark}

\begin{proposition}[The Missing Axiom]\label{prop:missing-axiom}
The automorphic recovery of the Ramanujan bound requires:
\[
  \textbf{A4 (Symmetric Power Functoriality):}\quad
  \mathrm{Sym}^m(\pi) \;\text{is automorphic for all}\;
  m \in \mathbb{N}.
\]
If $\pi_p$ is in the complementary series with parameter $s > 0$,
then $\mathrm{Sym}^m(\pi)_p$ has parameter $m \cdot s$.
Axiom $\mathrm{A3}$ (unitarity) applied to $\mathrm{Sym}^m(\pi)$
requires $m \cdot s < 1/2$.  Since this must hold for all $m$,
it forces $s = 0$: the tempered (Ramanujan) bound.
\end{proposition}

\begin{remark}[Structural interpretation]
The motivic side proves the sharp bound from a single
finite-dimensional axiom (the Rosati equation on
$\mathrm{End}(A) \otimes \RR$).  The automorphic side requires
an \emph{infinite axiom schema}: unitarity of $\mathrm{Sym}^m(\pi)$
for all $m$.  The motive is a rigid finite-dimensional object that
bounds the spectrum in one step.  The automorphic form is a
flexible infinite-dimensional object that requires the entire
hierarchy of multi-particle interactions (symmetric powers) to
achieve the same bound.  The Langlands correspondence transfers
the finite-dimensional rigidity to the infinite-dimensional side,
collapsing the infinite schema into a single geometric argument.
\end{remark}

\subsection{Formal Status of the $\Pi^0_2$ Classifications}

\begin{proposition}[Arithmetic Complexity---Formal Version]
\label{prop:pi2-formal}
Working in the language of second-order arithmetic:
\begin{enumerate}[label=\textup{(\alph*)},nosep]
  \item Standard Conjecture~D (for a fixed field $k$) has the
    form $\forall X\;\forall Z\;(\Phi(X,Z) \to \Psi(X,Z))$
    where $\Phi$ (``$Z$ is homologically trivial'') is
    $\Pi^0_1$ (requires $\LPO$ to evaluate the $\ell$-adic
    pairing) and $\Psi$ (``$Z$ is numerically trivial'') is
    $\Delta^0_0$ (integer intersection number is zero).  The
    implication $\Phi \to \Psi$ is $\Sigma^0_1$ for fixed
    $(X,Z)$.  The universal quantification makes the full
    conjecture $\Pi^0_2$.
  \item Finiteness of $\Sha(E/\QQ)$ has the form
    $\exists B\;\forall x\;(x \in \Sha \to |x| \leq B)$.
    Each instance $x \in \Sha$ is determined by local
    solvability at all primes ($\Pi^0_1$).  The existential
    bound makes this $\Sigma^0_2$.
  \item The Selberg eigenvalue conjecture has the form
    $\forall N\;(\lambda_1(\Gamma_0(N)\backslash\HH) \geq
    \tfrac14)$.  Each $\lambda_1(N)$ is the infimum of a
    spectral quantity ($\Pi^0_1$).  The universal quantification
    makes the conjecture $\Pi^0_2$.
\end{enumerate}
\end{proposition}

\begin{remark}
The classification requires specifying the arithmetization:
which G\"odel numbering encodes varieties, elliptic curves, and
hyperbolic surfaces.  For algebraic varieties over $\QQ$ and
congruence subgroups $\Gamma_0(N)$, standard arithmetizations
exist.  The classifications above hold for any reasonable
arithmetization.  A fully rigorous treatment would fix a specific
encoding and verify the quantifier counts; this is deferred to
the Lean formalization.
\end{remark}

\subsection{What This Section Achieves}

With these definitions and formal statements:

\begin{enumerate}[nosep]
  \item $\CRM(\mathcal{S})$ is a well-defined tuple, not a
    metaphor.
  \item ``Preservation'' means: for corresponding structures
    $\mathcal{S}$ (motivic) and $\mathcal{T}$ (automorphic),
    $\CRM(\mathcal{S}) = \CRM(\mathcal{T})$ componentwise.
    Theorems~\ref{thm:zero-test}--\ref{thm:polarization} verify
    this for each component.
  \item The Ramanujan asymmetry is a \textbf{theorem}
    (Theorem~\ref{thm:separation}): the automorphic axioms
    are provably incomplete for eigenvalue bounds, witnessed
    by Saito--Kurokawa lifts on $\mathrm{Sp}_4$.
  \item The missing axiom is identified: symmetric power
    functoriality (Proposition~\ref{prop:missing-axiom}).
  \item The $\Pi^0_2$ classifications are stated with explicit
    quantifier structure.  The deferred step (arithmetization)
    is identified.
\end{enumerate}

\begin{remark}[On the term ``decidability conduit'']
\label{rem:conduit}
The phrase ``decidability conduit''---used in the title and
throughout this paper---rests on two formal results and one
interpretation.

The formal results: (i)~CRM signatures match across the
correspondence (Theorem~\ref{thm:matching});
(ii)~the automorphic axioms are provably incomplete for
eigenvalue bounds, while the motivic axioms are complete
(Theorems~\ref{thm:separation} and~\ref{thm:asymmetry}).

The interpretation: the correspondence \emph{exists because}
the automorphic side needs the motivic side's rigidity.  This
causal claim goes beyond the formal results.  It could be
weakened to: ``the correspondence \emph{functions as} a conduit
regardless of why it exists.''  Even in this weaker reading, the
formal content is substantive: one side is provably stronger
than the other for eigenvalue bounds, and the correspondence
transfers this strength.

The interpretation would be falsified by a purely automorphic
proof of Ramanujan that avoids both the motivic bridge and the
infinite symmetric power schema.  No such proof is known, but
its impossibility is not proved.
\end{remark}

The rest of the paper uses these formal definitions.  Where a
claim goes beyond what the formal framework proves, we say so.

%% ══════════════════════════════════════════════════════════
\section{Paper 50 for Physicists}\label{sec:p50-summary}
%% ══════════════════════════════════════════════════════════

\subsection{The Problem of Zero-Testing}

In physics, you never test whether a real number equals zero exactly.
You measure it to finite precision and check whether $|x| < \epsilon$.
This works because your observables are backed by positive-definite
structures (norms, inner products) that make approximate equality
sufficient.

In number theory, exact zero-testing is essential.  The question
``is this cohomology class the zero class?'' requires an exact
answer.  Over complete fields ($\RR$, $\CC$, $\QQ_p$, $\QQ_\ell$),
this is an $\LPO$-level operation: you cannot decide in finite
time whether an infinite Cauchy sequence converges to zero.

The five motivic conjectures all assert that this $\LPO$-level
zero-testing is secretly unnecessary.  Each claims that
continuous cohomological data (requiring $\LPO$) descends
to algebraic data (decidable in $\BISH$).  The mechanism is the
same as in physics: a positive-definite inner product that makes
division legitimate.

\subsection{Why the Archimedean Place is Special}

The inner product that rescues $\LPO$ is positive-definite only at
the Archimedean place (over $\RR$).  The reason is a single number:
the \emph{$u$-invariant} of the field.  The $u$-invariant $u(k)$
is the maximum dimension of an anisotropic quadratic form over~$k$.

Over $\RR$: $u(\RR) = 1$.  Every form of dimension $\geq 2$ has
a nontrivial zero.  Positive-definite forms exist in every
dimension.  The Rosati involution on the endomorphism algebra of
an abelian variety always yields a positive-definite form.

Over $\QQ_p$: $u(\QQ_p) = 4$.  Forms of dimension $\geq 5$
always have nontrivial zeroes, but in dimensions $\leq 4$,
anisotropic forms exist.  Positive-definite forms do not exist in
dimension $\geq 5$.  The Rosati-type argument fails.

A physicist recognizes this asymmetry: it is the reason the
Euclidean path integral (over $\RR$, positive-definite metric)
converges nicely while the Lorentzian path integral requires
analytic continuation.  The mathematical root is the same:
$u(\RR) = 1$.

\subsection{The Three-Axiom Characterization}

Paper~50 proposes that the category of numerical motives is the
initial object in the $2$-category of \emph{Decidable Polarized
Tannakian categories}.  The three axioms are:

\begin{enumerate}[label=\textbf{Axiom \arabic*},leftmargin=*]
  \item \textbf{DecidableEq on Hom.}  Morphism spaces have
    decidable equality.  (Zero-testing terminates.)
  \item \textbf{IsIntegral on End.}  Eigenvalues of endomorphisms
    are algebraic integers.  (The spectrum is algebraic.)
  \item \textbf{InnerProductSpace on $\mathrm{End} \otimes \RR$.}
    The Archimedean realization carries a positive-definite form.
    (Division is legitimate.)
\end{enumerate}

These are logical axioms, not geometric ones.  They say: morphisms
are decidable, eigenvalues are algebraic, and the real inner
product is positive-definite.  A physicist would recognize them as:
discreteness of the spectrum, quantization, and unitarity.

\subsection{What Follows from Three Axioms}

\textbf{Theorem A (Weil RH).}  The Riemann Hypothesis for varieties
over $\FF_q$ follows from the three axioms.  The proof is four steps:
Axiom~2 gives algebraic eigenvalues; Axiom~3 gives a positive-definite
form; the Rosati condition gives $\langle \Frob \cdot x,
\Frob \cdot x\rangle = q^w \langle x, x\rangle$; division by
$\langle x, x\rangle > 0$ (Axiom~3) gives $|\alpha|^2 = q^w$.

\textbf{Theorem B (Honda--Tate Inhabitant).}  The axioms are
satisfiable.  Over $\FF_p$, the motive of a CM elliptic curve
inhabits the type: rational skeleton $\QQ^2$, Frobenius matrix
$\bigl(\begin{smallmatrix} 0 & -p \\ 1 & a \end{smallmatrix}\bigr)$,
Rosati form with determinant $4p - a^2 > 0$ (the Hasse bound from
1933).

\textbf{Theorem C (Conjecture D as Decidability).}
Grothendieck's Standard Conjecture~D---homological equivalence
equals numerical equivalence---is equivalent to Axiom~1.  It
asserts that $\LPO$-dependent homological zero-testing descends to
$\BISH$-decidable integer intersection numbers.

\textbf{Theorem D (Dual Hierarchy).}  The major conjectures of
arithmetic geometry (Conjecture~D, Hodge, finiteness of
$\Sha$) live at $\Pi^0_2$ in the arithmetic hierarchy---the same
level as the physics spectral gap.  The motive acts as a
$(-1)$-shift operator: it accepts a $\Pi^0_2$ axiom (Conjecture~D)
and delivers $\Sigma^0_2 \to \Sigma^0_1$ descent on instances.

\textbf{Theorem E (CM Decidability).}  For CM elliptic curves
over $\QQ$, the motivic subcategory is unconditionally
$\BISH$-decidable.  Three theorems---Damerell (1970), Lefschetz
$(1,1)$ (1924), and Matsusaka---simultaneously eliminate $\LPO$,
$\MP$, and the Fontaine--Mazur obstruction.  No conjectures
assumed.

\subsection{The Link to This Paper}

The physics programme found $\BISH + \LPO$ with
positive-definiteness as the universal rescue.  The motivic
programme found $\BISH + \LPO + \MP$ with the \emph{same}
positive-definiteness as the rescue.  Both programmes found
$\Pi^0_2$ as the ceiling for universal statements.

The Langlands programme asserts that the motivic side and the
automorphic side (spectral theory on adelic groups---essentially
physics) carry the same data.  This paper asks: does the
correspondence preserve CRM signatures?

%% ══════════════════════════════════════════════════════════
\section{The Three-Column Dictionary}\label{sec:dictionary}
%% ══════════════════════════════════════════════════════════

The CRM programme has independently measured three domains.  We
now display their logical signatures side by side.

\subsection{Axiom 1: Decidable Morphisms}

On the \textbf{motivic side}, Standard Conjecture~D asserts that
homological equivalence $=$ numerical equivalence.  Homological
equivalence requires zero-testing in $\QQ_\ell$-cohomology
($\LPO$).  Numerical equivalence requires integer intersection
numbers ($\BISH$).  Conjecture~D asserts they agree:
$\mathrm{DecidableEq}$ on $\mathrm{Hom}_{\mathrm{num}}$.

On the \textbf{automorphic side}, Strong Multiplicity One
(Shalika \cite{Shalika}, Piatetski-Shapiro \cite{PS}) asserts
that a cuspidal automorphic representation of $\mathrm{GL}_n$
is determined by its Hecke eigenvalues at almost all primes.
This makes multiplicity spaces at most one-dimensional:
$\mathrm{DecidableEq}$ on automorphic multiplicities.

On the \textbf{physics side}, the analogue is spectral
discreteness: the Hamiltonian has isolated eigenvalues, making
energy levels distinguishable (decidable).

\subsection{Axiom 2: Algebraic Spectrum}

On the \textbf{motivic side}, Frobenius eigenvalues are algebraic
integers---Weil numbers satisfying $P(t) \in \ZZ[t]$.

On the \textbf{automorphic side}, Hecke eigenvalues of normalized
cuspidal eigenforms are algebraic integers.  This requires the
Eichler--Shimura isomorphism (for $\mathrm{GL}_2$) or Clozel's
purity theorems \cite{Clozel} (general case), connecting analytic
Fourier coefficients to Betti cohomology of Shimura varieties.
The descent from analytic complex numbers to algebraic integers is
exactly $\LPO \to \BISH$.

On the \textbf{physics side}, the analogue is quantized eigenvalues:
discrete spectra taking values in algebraic expressions of physical
constants.

\subsection{Axiom 3: Archimedean Polarization}

On the \textbf{motivic side}, the Rosati involution on
$\mathrm{End}(A) \otimes \RR$ defines a positive-definite form
$\langle f, g\rangle = \Tr(f \cdot g^\dagger)$.
Positive-definiteness holds because $u(\RR) = 1$.

On the \textbf{automorphic side}, the Petersson inner product
\[
  \langle f, g \rangle \;=\;
  \int_{\Gamma_0(N)\backslash\HH}
    f(z)\,\overline{g(z)}\,y^k\,
    \frac{dx\,dy}{y^2}
\]
is positive-definite on cusp forms.  This is an $L^2$-integral of
a norm over a fundamental domain---positive-definiteness from the
Archimedean topology of $\RR$.

On the \textbf{physics side}, the Hilbert space inner product
$\langle\psi|\varphi\rangle$ is positive-definite by axiom
(Dirac--von Neumann).

All three positive-definite forms arise at the Archimedean place.
All three fail $p$-adically: there is no translation-invariant
Haar measure on $\QQ_p$ taking values in $\RR_{>0}$ that
produces a canonical positive-definite metric.  The obstruction is
$u(\QQ_p) = 4$ in every case.

\subsection{Summary Table}

\begin{table}[h]
\centering
\small
\renewcommand{\arraystretch}{1.25}
\begin{tabular}{@{}lccc@{}}
\toprule
\textbf{CRM Feature} & \textbf{Motivic} & \textbf{Automorphic} & \textbf{Physics} \\
\midrule
Axiom 1 & Conj.~D & Strong Mult.~One & Spectral discreteness \\
Axiom 2 & Weil numbers & Shimura algebraicity & Quantized eigenvalues \\
Axiom 3 & Rosati form & Petersson i.p. & Hilbert space i.p. \\
Descent & Algebraic & Eichler--Shimura & Measurement \\
Shift operator & Motive ($-1$) & Trace formula & [none known] \\
Spectral gap & $\Sha$ finite & Selberg $\lambda_1 \geq \tfrac14$ & Cubitt et al. \\
\;($\Sigma^0_2$) & ($\exists B\;\forall x$) & ($\exists\lambda\;\forall N$) & ($\exists\Delta\;\forall N$) \\
$\MP$ residual & Cycle search & \textbf{[incomplete]} & BPS state search \\
$\Pi^0_2$ axiom & Conj.~D & Ramanujan conj. & Gap universality \\
\bottomrule
\end{tabular}
\caption{CRM signatures across three domains.  The blank entry
  in the $\MP$ row of the automorphic column is the asymmetry
  identified in \S\ref{sec:ramanujan}.}
\label{tab:dictionary}
\end{table}

\begin{theorem}[CRM Signature Matching]\label{thm:matching}
Assuming the global Langlands correspondence for $\mathrm{GL}_n$
over $\QQ$, the CRM signatures of an algebraic cuspidal automorphic
representation $\pi$ and its associated motive $M_\pi$ match at
every place:
\begin{enumerate}[label=\textup{(\alph*)},nosep]
  \item $\mathrm{DecidableEq}$ on $\mathrm{Hom}_{\mathrm{num}}(M,N)$
    $\longleftrightarrow$ Strong Multiplicity One for $\pi$;
  \item $\mathrm{IsIntegral}(\Frob_p \mid M)$
    $\longleftrightarrow$ algebraicity of $a_p(\pi)$
    (Shimura);
  \item positive-definiteness of Rosati on $M$
    $\longleftrightarrow$ positive-definiteness of Petersson
    on $\pi$.
\end{enumerate}
For $\mathrm{GL}_2$ over $\QQ$, the result is unconditional
(via the modularity theorem of Wiles, Taylor--Wiles, and BCDT).
\end{theorem}

%% ══════════════════════════════════════════════════════════
\section{The Ramanujan Asymmetry}\label{sec:ramanujan}
%% ══════════════════════════════════════════════════════════

The three-column dictionary reveals a critical asymmetry.  The
motivic and physics columns are CRM-complete: their three axioms
suffice to derive sharp eigenvalue bounds.  The automorphic column
is not.

\subsection{The Motivic Proof of the Weil RH}

On the motivic side, the sharp eigenvalue bound follows from the
three CRM axioms (Paper~50, Theorem~A):

\begin{enumerate}[nosep]
  \item Axiom~2: $\Frob$ has algebraic-integer eigenvalues.
    Characteristic polynomial $P(t) \in \ZZ[t]$.
  \item Axiom~3: Rosati form $\langle\cdot,\cdot\rangle$
    positive-definite.
  \item Rosati condition:
    $\langle \Frob \cdot x, \Frob \cdot x\rangle
      = q^w \langle x, x\rangle$.
  \item Division: $\langle x, x\rangle > 0$ by Axiom~3, so
    $|\Frob|^2 = q^w$, giving $|\alpha| = q^{w/2}$.
\end{enumerate}

The bound is sharp because the positive-definite form acts on a
finite-dimensional $\QQ$-vector space where algebraic structure
forces rigidity.

\subsection{The Automorphic Failure}

On the automorphic side, the Petersson inner product makes local
representations unitary.  For an unramified principal series
representation at~$p$, unitarity gives only the trivial bound:
\[
  |a_p(f)| \;\leq\; p^{(k-1)/2}\,(p^{1/2} + p^{-1/2}).
\]
This exceeds the Ramanujan bound $|a_p(f)| \leq 2p^{(k-1)/2}$
by the factor $(p^{1/2} + p^{-1/2})/2 > 1$ for every finite~$p$.

Purely analytic methods improve this, but never achieve the sharp bound:
\begin{itemize}[nosep]
  \item Rankin--Selberg: off by $p^{1/4}$.
  \item Kim--Sarnak (2003): off by $p^{7/64}$.
\end{itemize}
No improvement has been achieved in over twenty years.

The automorphic positive-definiteness (Petersson) acts on the
infinite-dimensional space $L^2(\Gamma\backslash\HH)$, where
local components retain room to fluctuate.  The motivic
positive-definiteness (Rosati) acts on a \emph{finite-dimensional}
$\QQ$-vector space where algebraic integrality forces rigidity.
The CRM diagnosis: the automorphic Axiom~3 is too weak.

\subsection{Deligne's Bridge Crossing}

Deligne \cite{Deligne-Weil} proved the Ramanujan conjecture for
holomorphic modular forms of weight $k \geq 2$ by leaving the
automorphic side entirely:

\begin{enumerate}[nosep]
  \item Construct the $\ell$-adic Galois representation
    $\rho_f : \Gal(\overline{\QQ}/\QQ) \to \mathrm{GL}_2(\QQ_\ell)$
    attached to $f$ (Eichler--Shimura for $k=2$, Deligne for
    general $k$).
  \item Show $\rho_f$ appears in the \'etale cohomology of a
    variety over $\FF_p$ (the Kuga--Sato variety).
  \item Apply the Weil conjectures (proved by Deligne)
    to obtain $|\alpha_p| = p^{(k-1)/2}$.
\end{enumerate}

Step~3 is the motivic CRM argument: integrality + Rosati
positive-definiteness $\Rightarrow$ sharp eigenvalue bound.
Deligne could not prove Ramanujan automorphically.  He
\emph{had to cross the bridge}.

\begin{theorem}[Ramanujan Asymmetry]\label{thm:asymmetry}
The automorphic side of the Langlands correspondence is
CRM-incomplete for eigenvalue bounds:
\begin{enumerate}[label=\textup{(\alph*)},nosep]
  \item The motivic CRM axioms suffice to derive
    $|\alpha| = q^{w/2}$ (Weil RH).
  \item The automorphic CRM axioms $\mathrm{A1} + \mathrm{A2}
    + \mathrm{A3}$ do \textbf{not} suffice to derive the
    Ramanujan bound (Theorem~\ref{thm:separation}).
  \item The Langlands correspondence acts as a conduit through
    which $\BISH$-level bounds flow from the motivic side to
    the automorphic side (Deligne's proof strategy; see
    Remark~\ref{rem:conduit} for the epistemic status of
    ``conduit'').
\end{enumerate}
\end{theorem}

\begin{proof}
Part~(a) is Paper~50, Theorem~A.  Part~(b) is
Theorem~\ref{thm:separation}: Saito--Kurokawa lifts on
$\mathrm{Sp}_4$ are classical cuspidal automorphic
representations satisfying $\mathrm{A1} + \mathrm{A2} +
\mathrm{A3}$ while violating the generalized Ramanujan bound.
Part~(c) is Deligne's proof of Ramanujan for holomorphic
modular forms \cite{Deligne-Weil}, which works by constructing
$\ell$-adic Galois representations (crossing to the motivic
side) and applying the Weil conjectures.
\end{proof}

\subsection{The Maass Form Prediction}

\begin{conjecture}[Maass Form Obstruction]\label{conj:maass}
The Ramanujan conjecture for Maass forms on $\mathrm{GL}_2$
cannot be proved by purely automorphic methods.
\end{conjecture}

\emph{Evidence.}  Maass forms correspond to representations
with Archimedean component in the principal series of
$\mathrm{SL}_2(\RR)$, not the discrete series.  No geometric
motive is known to produce these representations.  Without the
motivic side, the $\BISH$-level Rosati bounds are unavailable.
The Kim--Sarnak bound remains the best result after two decades.

\emph{CRM interpretation.}  The automorphic side lacks the rigid
finite-dimensional algebraic structure that enforces sharp bounds.
For Maass forms, the Langlands bridge has not been built, so the
bounds cannot cross.

\emph{Testable prediction.}  Any proof of Ramanujan for Maass
forms must either (a)~construct a geometric motive (building the
bridge), or (b)~discover a fundamentally new descent mechanism
replacing motivic $\BISH$ bounds.

%% ══════════════════════════════════════════════════════════
\section{The Trace Formula as Descent Equation}\label{sec:STF}
%% ══════════════════════════════════════════════════════════

\subsection{The Selberg Trace Formula}

For $\Gamma$ a cocompact Fuchsian group, the Selberg trace
formula equates:
\[
  \underbrace{\sum_j h(r_j)}_{\text{Spectral (LPO)}}
  \;=\;
  \underbrace{\frac{\mathrm{Area}(\Gamma\backslash\HH)}{4\pi}
    \int_{-\infty}^{\infty} h(r)\,r\tanh(\pi r)\,dr
    \;+\; \sum_{\{\gamma\}} \frac{\log N(\gamma_0)}{N(\gamma)^{1/2} - N(\gamma)^{-1/2}}\,
    g(\log N(\gamma))
    \;+\; \cdots}_{\text{Geometric (BISH)}}
\]

The \textbf{spectral side} involves eigenvalues $\lambda_j =
\tfrac14 + r_j^2$ of the Laplacian on
$L^2(\Gamma\backslash\HH)$---the spectrum of an operator on an
infinite-dimensional space.  CRM cost: $\LPO$.

The \textbf{geometric side} involves norms $N(\gamma)$ of
hyperbolic conjugacy classes---discrete algebraic quantities
computable from matrix entries.  For arithmetic groups, these
reduce to solutions of Pell's equation and class numbers.  CRM
cost: $\BISH$.

The trace formula is a \emph{de-omniscientizing descent equation}.
It compiles $\LPO$-level spectral data into $\BISH$-level
geometric data.

\subsection{The Gutzwiller Identification}

The Selberg trace formula is mathematically identical to the
Gutzwiller trace formula in quantum chaos \cite{Gutzwiller}:
\[
  Z(t) \;=\; \Tr(e^{-t\Delta})
  \;=\; \underbrace{\sum_{\text{quantum}} e^{-t\lambda_j}}_{\LPO}
  \;=\; \underbrace{\sum_{\text{classical}} A_\gamma\,
    e^{-L_\gamma^2/4t}}_{\BISH}
\]
where $L_\gamma$ are lengths of classical periodic orbits
(closed geodesics) and $A_\gamma$ are stability amplitudes.

For arithmetic surfaces $\Gamma\backslash\HH$, the identification
is exact.  The quantum partition function (physics, $\LPO$) equals
the classical orbit sum (geometry, $\BISH$).  The trace formula
\emph{is} the descent.

This means the automorphic side of the Langlands correspondence is
literally a quantum mechanical system.  The Hecke operators at each
prime are the transfer matrices at each lattice site.  The Casimir
operator at infinity is the Hamiltonian.  The spectral decomposition
of $L^2(G(\QQ)\backslash G(\Adeles))$ is the diagonalization of the
quantum Hamiltonian.

\subsection{Arthur's Generalization}

Arthur's trace formula generalizes Selberg to arbitrary reductive
groups $G$ over $\QQ$:
\[
  I_{\mathrm{spectral}}(f) \;=\; I_{\mathrm{geometric}}(f).
\]
The spectral side sums over automorphic representations ($\LPO$).
The geometric side sums over conjugacy classes with orbital
integrals ($\BISH$ for the discrete terms).  For non-cocompact
groups, truncation operators introduce complexity approaching
$\LPO$, but the fundamental $\LPO \leftrightarrow \BISH$
dictionary remains the conceptual engine.

Every automorphy lifting theorem---Taylor--Wiles, Calegari--Geraghty---
ultimately rests on a trace formula comparison that converts
$\LPO$-level spectral identities into $\BISH$-level geometric
identities.

%% ══════════════════════════════════════════════════════════
\section{Three Spectral Gaps}\label{sec:gaps}
%% ══════════════════════════════════════════════════════════

Three spectral gap problems in three domains, all $\Sigma^0_2$:

\begin{enumerate}[label=(\roman*),nosep]
  \item \textbf{Physics} (Cubitt--Perez-Garcia--Wolf, 2015
    \cite{CPW}).  For a family of local Hamiltonians
    $\{H_N\}$: $\exists\,\Delta > 0\;\forall N:\;
    \mathrm{gap}(H_N) \geq \Delta$.  Proved undecidable.

  \item \textbf{Automorphic} (Selberg Eigenvalue Conjecture, 1965).
    For congruence subgroups $\Gamma_0(N)$:
    $\forall N:\; \lambda_1(\Gamma_0(N)\backslash\HH) \geq
    \tfrac{1}{4}$.  Open.  Best bound:
    $\lambda_1 \geq 975/4096 \approx 0.238$ (Kim--Sarnak).

  \item \textbf{Arithmetic} (Finiteness of $\Sha$).  For an
    elliptic curve $E/\QQ$:
    $\exists\,B\;\forall\text{ torsors}\; x \in \Sha(E):\;
    |x| \leq B$.  Proved in many cases (Kolyvagin, for
    analytic rank $\leq 1$).
\end{enumerate}

All three ask the same logical question: does a local quantity,
computable at each finite stage, admit a global bound that persists
uniformly?  Local gaps can shrink (physics), local eigenvalues can
approach $\tfrac{1}{4}$ (automorphic), and local torsors can grow
(arithmetic)---all without computable rate.  The universal
quantification over $\LPO$-level predicates pushes all three to
$\Sigma^0_2$.

\begin{theorem}[Three Spectral Gaps]\label{thm:gaps}
The Hamiltonian spectral gap, the Selberg eigenvalue conjecture,
and finiteness of $\Sha$ are all $\Sigma^0_2$ statements in the
arithmetic hierarchy, with identical logical structure:
$\exists\,\text{bound}\;\forall\,\text{instances}\;
(\LPO\text{-level predicate})$.
\end{theorem}

These are connected by explicit constructions.  Lubotzky,
Phillips, and Sarnak \cite{LPS} used the Ramanujan conjecture
to construct optimal expander graphs (Ramanujan graphs), literally
mapping the automorphic spectral gap to a combinatorial/physical
network spectral gap.

%% ══════════════════════════════════════════════════════════
\section{The CM Base Case}\label{sec:CM}
%% ══════════════════════════════════════════════════════════

For CM elliptic curves over $\QQ$, CRM signatures match perfectly
on both sides of the Langlands correspondence.

\textbf{Motivic side} (Paper~50, Theorem~E):
\begin{itemize}[nosep]
  \item $\LPO$ eliminated by Damerell's theorem:
    $L(E,1)/\Omega \in \QQ$.
  \item Fontaine--Mazur eliminated by the 13 Heegner numbers
    (finite table lookup).
  \item $\MP$ eliminated by Lefschetz $(1,1)$: Hodge classes on
    products of elliptic curves are divisors, parameterized by
    Hermitian matrices over $\QQ(\sqrt{d})$.
\end{itemize}
Overall: $\BISH$.

\textbf{Automorphic side:}
\begin{itemize}[nosep]
  \item Hecke characters: finite sums over ideal class groups
    ($\BISH$).
  \item $L$-value: Kronecker limit formula + Chowla--Selberg
    give explicit algebraic evaluation ($\BISH$).
  \item Multiplicity: CM forms have explicit eigenvalues
    determined by the CM type ($\BISH$).
\end{itemize}
Overall: $\BISH$.

Both sides drop simultaneously.  The motivic Damerell theorem
and the automorphic Kronecker limit formula are different
expressions of the same CM rationality.

\subsection{The Langlands Progression}\label{subsec:progression}

The historical development of the Langlands programme tracks the
CRM hierarchy.  The CM base case illuminates why.

\textbf{1920s--1960s: CM cases.}  Hecke, Deuring, Shimura--Taniyama.
CRM: $\BISH$ on both sides.  Every computation terminates.  Proved
first not because they were ``easier'' in a vague sense, but because
no non-constructive principles are required.

\textbf{1995--2001: Elliptic curves over $\QQ$.}  Wiles,
Taylor--Wiles, BCDT.  CRM: $\BISH + \MP$.  Modularity lifting
requires searching over Galois deformation spaces---bounded but
nontrivial existential quantifiers.  Taylor--Wiles patching is a
technique for forcing $\MP$-type searches to terminate.

\textbf{2010s--present: Automorphy over totally real fields.}
Calegari--Geraghty, the 10-author theorem.  CRM: $\BISH + \MP +$
fragments of $\LPO$.  $p$-adic Hodge theory and perfectoid spaces
manage $\LPO$-level zero-testing through algebraic surrogates.

Each generation climbs one level of the logical hierarchy and
requires correspondingly heavier machinery.

%% ══════════════════════════════════════════════════════════
\section{The Physics Link}\label{sec:physics}
%% ══════════════════════════════════════════════════════════

\subsection{Quantum Mechanics and Automorphic Forms}

The identification is not metaphorical.  The automorphic side of
the Langlands correspondence is a quantum mechanical system:

\begin{itemize}[nosep]
  \item The Hilbert space is
    $L^2(G(\QQ)\backslash G(\Adeles))$.
  \item The Hamiltonian is the Casimir operator (at infinity)
    plus Hecke operators (at each prime).
  \item The spectral decomposition is the classification of
    automorphic representations.
  \item The trace formula (Selberg/Arthur) is the partition
    function $Z = \Tr(e^{-\beta H})$.
\end{itemize}

The Gutzwiller trace formula in quantum chaos is the Selberg
trace formula on arithmetic surfaces.  This is a mathematical
theorem, not an analogy.

\subsection{The Motive as Classical Mechanics}

If the automorphic side is quantum mechanics, the motivic side
is classical mechanics.  The motive provides the algebraic data
($\BISH$): finite-dimensional skeleton, algebraic eigenvalues,
intersection numbers.  The automorphic side provides the spectral
data ($\LPO$): infinite-dimensional Hilbert space, continuous
spectrum, analytic $L$-functions.

The Langlands correspondence is the quantum-classical
correspondence of arithmetic geometry.  The trace formula equates
the quantum side (spectral, $\LPO$) with the classical side
(geometric, $\BISH$).

\subsection{The BPS Analogue of Markov's Principle}

The motivic side retains an $\MP$ residual: finding algebraic
cycle witnesses for Hodge classes requires unbounded Diophantine
search.

The physics analogue is the search for BPS states.  In quantum
field theory, topological index theorems guarantee a topological
sector with a zero-mode exists ($\LPO$-level assertion).  Finding
the explicit microscopic configuration---the classical soliton,
the instanton, the localized field---requires an unbounded
nonlinear search through field space ($\MP$).

\begin{center}
\begin{tabular}{lcc}
\toprule
& \textbf{Motivic} & \textbf{Physics} \\
\midrule
Existence & $\exists Z$: $\mathrm{cl}(Z) = \alpha$ & $\exists\varphi$: $E(\varphi) = |Q|$ \\
Guarantee & Index theory / Hodge theory & Topological index \\
Search type & Diophantine ($\MP$) & Nonlinear PDE ($\MP$) \\
\bottomrule
\end{tabular}
\end{center}

The extra $\MP$ that arithmetic geometry appeared to have over
physics---physics has it too, in the BPS search.

\subsection{The Unified Picture}

\begin{center}
\begin{tikzcd}[column sep=large, row sep=large]
  \text{Physics (quantum)} \arrow[r, "\text{Gutzwiller}", leftrightarrow]
    \arrow[d, "\text{descent}"']
  & \text{Automorphic (spectral)} \arrow[r, "\text{Langlands}", leftrightarrow]
    \arrow[d, "\text{descent}"]
  & \text{Motivic (classical)} \arrow[d, "\text{descent}"] \\
  \LPO \arrow[r, equal] & \LPO \arrow[r, "\text{bridge}"] & \BISH + \MP
\end{tikzcd}
\end{center}

The trace formula descends the physics/automorphic side from $\LPO$
to $\BISH$.  The Langlands correspondence connects the automorphic
side to the motivic side.  The motivic CRM axioms descend the
motivic side from $\LPO$ to $\BISH + \MP$.  Archimedean
polarization ($u(\RR) = 1$) descends $\MP$ to $\BISH$ in favorable
cases (CM, BSD rank $\leq 1$).

All three descents use the same mechanism: positive-definiteness
at the Archimedean place.  All three face the same ceiling:
$\Pi^0_2/\Sigma^0_2$ for universal statements.  All three retain
the same residual: $\MP$ (cycle search / BPS search).

%% ══════════════════════════════════════════════════════════
\section{Why the Correspondence Exists: A Diagnosis}\label{sec:why}
%% ══════════════════════════════════════════════════════════

The following is an \emph{interpretive argument}, not a theorem.
We state it because it organizes the formal results of the
preceding sections into a coherent picture, and because it
generates testable predictions (Conjecture~\ref{conj:maass}).
See Remark~\ref{rem:conduit} for the precise epistemic status.

The standard view of the Langlands correspondence is that it is a
profound, unexpected connection between number theory and harmonic
analysis.  The CRM framework suggests a different reading.

Both the motivic and automorphic sides face the same logical
problem: extract finite, decidable information ($\BISH$) from
infinite, continuous structures ($\LPO$).  The solution space for
this problem is severely constrained.  Positive-definiteness at the
Archimedean place ($u(\RR) = 1$) is the unique mechanism.  Both
sides discovered it independently---the Rosati involution on one
side, the Petersson inner product on the other---because there is
nothing else to discover.

But the two implementations are not equivalent.  The motivic
implementation (Rosati on finite-dimensional $\QQ$-vector spaces)
is rigid: it enforces sharp eigenvalue bounds ($\BISH$).  The
automorphic implementation (Petersson on infinite-dimensional $L^2$
spaces) is flexible: it enforces only the trivial unitary bound.

The correspondence exists because the automorphic side
\emph{needs} the motivic side's rigidity.  It is not a coincidence
or a miracle.  It is forced by a logical asymmetry: the two sides
solve the same problem ($\LPO \to \BISH$ descent via
positive-definiteness), but the motivic solution is strictly
stronger.  The correspondence is the conduit through which the
stronger solution flows.

This explains Deligne's strategy for proving Ramanujan: he could
not stay on the automorphic side because the automorphic side
doesn't have sharp bounds.  It explains why Ramanujan for Maass
forms is open: the conduit hasn't been built.  And it explains
the historical progression of Langlands (\S\ref{subsec:progression}):
each generation builds more of the conduit, transferring more
$\BISH$-level structure from the motivic side.

%% ══════════════════════════════════════════════════════════
\section{Comparison with Existing Frameworks}\label{sec:comparison}
%% ══════════════════════════════════════════════════════════

\subsection{Other Frameworks within Arithmetic Geometry}

\textbf{Geometric Langlands} (Frenkel, Ben-Zvi, Nadler).
Over function fields, both sides are algebraic ($\BISH$), and
signatures match trivially.  The geometric Langlands programme
provides additional evidence for CRM preservation but is
orthogonal in method.

\textbf{Condensed Mathematics} (Scholze, Clausen).
Provides new foundations handling $p$-adic and Archimedean phenomena
uniformly.  The CRM framework predicts that condensed methods
provide a new descent mechanism.  Whether they can prove Ramanujan
automorphically---bypassing the motivic side---is identified as
structurally significant by our framework.  A positive answer would
weaken the ``conduit'' interpretation; a negative answer would
strengthen it.

\textbf{The Fundamental Lemma} (Ng\^o, Fields Medal 2010).
In CRM terms, the fundamental lemma is the verification that the
$\BISH$ sides of two trace formulas (for two different groups)
agree.  It is a decidability matching at the geometric level---
ensuring that the descent equations for two groups produce
compatible $\BISH$ data.

\subsection{Proposed Physics--Langlands Connections}

Several proposals have linked the Langlands programme to physics.
We compare them to the CRM approach, noting that they address
different aspects of the relationship and are not mutually
exclusive.

\textbf{Gauge-Theoretic Langlands} (Kapustin--Witten \cite{KapustinWitten}).
Kapustin and Witten showed that the geometric Langlands
correspondence can be derived from $S$-duality of $\mathcal{N}=4$
super-Yang--Mills theory compactified on a Riemann surface.
Electric--magnetic duality exchanges the two sides of the
correspondence.  This is the deepest and most influential
physics--Langlands link to date.

\emph{Comparison.}  Kapustin--Witten works over function fields
(the geometric setting) and derives the correspondence from a
specific physical duality.  The CRM approach works over number
fields and does not derive the correspondence---it measures the
logical signatures on both sides and observes that they match.
The two approaches are complementary: Kapustin--Witten explains
\emph{how} the correspondence arises (via $S$-duality);
the CRM approach asks \emph{what logical structure} the
correspondence preserves (decidability signatures).  Neither
subsumes the other.  One advantage of the CRM approach is that it
applies to the arithmetic setting (over $\QQ$), where
Kapustin--Witten does not directly operate.  One advantage of
Kapustin--Witten is that it produces the correspondence as a
theorem (in the geometric setting), whereas the CRM approach
takes the correspondence as input and studies its properties.

\textbf{Conformal Field Theory and Langlands}
(Feigin--Frenkel \cite{FeiginFrenkel}, Frenkel \cite{Frenkel-lectures}).
Frenkel's programme connects the Langlands correspondence to
conformal field theory through the theory of vertex algebras and
the Feigin--Frenkel isomorphism (critical-level representations
of affine Kac--Moody algebras).  The geometric Langlands
correspondence is interpreted as a statement about $\mathcal{D}$-modules
on the moduli stack of $G$-bundles, related to conformal blocks.

\emph{Comparison.}  The CFT approach provides rich structural
content---the local Langlands correspondence at each place is
related to representations of loop groups, and the global
correspondence to conformal blocks on Riemann surfaces.  The CRM
approach is structurally thinner: it does not use the internal
geometry of representation theory but only the logical complexity
of the computations involved.  The CFT approach explains why
specific representation-theoretic structures appear; the CRM
approach explains why positive-definiteness and algebraic spectra
appear on both sides.  They address different ``why'' questions.

\textbf{Arithmetic Quantum Field Theory}
(Minhyong Kim \cite{KimArithmetic}).
Kim has proposed that Diophantine geometry should be understood
through an analogy with gauge theory: the fundamental group of a
variety plays the role of the gauge group, and rational points
are ``flat connections'' or ``instantons.''  This is part of a
broader programme connecting the \'etale fundamental group to
path-integral-like constructions.

\emph{Comparison.}  Kim's approach is genuinely arithmetic (works
over $\QQ$) and aims to produce Diophantine results (finiteness
of rational points) from gauge-theoretic structures.  The CRM
approach does not aim for Diophantine results but measures
logical complexity.  An interesting intersection: Kim's ``arithmetic
Chern--Simons theory'' involves integrals over $p$-adic spaces
that the CRM framework would classify as $\LPO$-level operations.
Whether Kim's gauge-theoretic structures provide descent mechanisms
in the CRM sense is an open question.

\textbf{Topological Quantum Computation and Langlands}
(Freed--Hopkins--Teleman \cite{FHT}).
Freed, Hopkins, and Teleman relate the Verlinde ring (from
topological field theory) to the representation ring of the loop
group, and hence to Langlands duality via the Satake isomorphism.
This connects 3-dimensional topological field theory to the
Langlands dual group.

\emph{Comparison.}  The TFT approach operates at the categorical
level (functors between bordism categories) and produces the
Langlands dual group from physical duality.  The CRM approach
operates at the logical level (decidability of computations) and
does not produce the dual group.  Again, the approaches are
complementary.

\subsection{What the CRM Approach Adds}

The proposals above share a common strategy: they identify a
\emph{specific physical theory} (gauge theory, CFT, TFT,
arithmetic QFT) whose mathematical structure mirrors or produces
some aspect of the Langlands correspondence.  The CRM approach
takes a different path.  It does not identify a specific physical
theory.  Instead, it identifies a \emph{logical constraint} that
applies to any theory---physical or mathematical---operating on
data over complete fields.

The constraint is: extracting decidable information from continuous
structures requires positive-definiteness at the Archimedean place
($u(\RR) = 1$), and this mechanism is unique.  Any domain facing
this constraint---quantum mechanics, automorphic spectral theory,
motivic cohomology---will develop the same logical architecture
($\BISH + \LPO$, positive-definite descent, $\Pi^0_2$ ceiling).

This gives the CRM approach a feature the other proposals lack:
it explains why \emph{multiple} physical theories connect to the
Langlands programme, rather than just one.  Kapustin--Witten uses
$\mathcal{N}=4$ SYM.  Frenkel uses CFT.  Freed--Hopkins--Teleman
uses TFT.  Kim uses gauge theory.  From the CRM perspective, these
are all instances of the same logical architecture manifesting in
different physical settings.  The architecture is forced by the
decidability constraint, not by the specific physics.

We do not claim this perspective is superior to the others.  It is
orthogonal.  The gauge-theoretic approaches produce structural
results (dualities, correspondences, functors) that the CRM
approach cannot.  The CRM approach produces a diagnostic
(CRM signatures, the Ramanujan asymmetry, the Langlands
progression) that the gauge-theoretic approaches do not address.
We believe the CRM perspective has merit because it identifies a
common logical root beneath the multiplicity of physical
connections, and because it generates a testable prediction
(Conjecture~\ref{conj:maass}) that the other approaches do not.

\subsection{The Separation Theorem in Context}

The separation result (Theorem~\ref{thm:separation}) resolves the
open problem identified by early reviewers of this paper.  It
proves that the automorphic CRM axioms are logically incomplete
for eigenvalue bounds.  The proof uses Saito--Kurokawa lifts on
$\mathrm{Sp}_4$---classical mathematical objects, not abstract
model-theoretic constructions.

The identification of symmetric power functoriality as the missing
axiom (Proposition~\ref{prop:missing-axiom}) reveals the precise
logical gap:  the motivic side bounds the spectrum via one
finite-dimensional argument (the Rosati equation on
$\mathrm{End}(A) \otimes \RR$), while the automorphic side
requires an \emph{infinite axiom schema} ($\mathrm{Sym}^m$ is
automorphic and unitary for all $m$).  The Langlands correspondence
collapses this infinite schema into the single geometric argument
by transporting the problem to the motivic side.

This result places the CRM framework on firmer ground than
the competing proposals reviewed above: it is the only framework
that produces a formal separation theorem distinguishing the
logical strength of the two sides of the Langlands
correspondence.

%% ══════════════════════════════════════════════════════════
\section{Status of Claims}\label{sec:status}
%% ══════════════════════════════════════════════════════════

We are explicit about the logical status of every claim.

\begin{table}[h]
\centering
\small
\renewcommand{\arraystretch}{1.2}
\begin{tabular}{@{}lll@{}}
\toprule
\textbf{Claim} & \textbf{Status} & \textbf{Depends on} \\
\midrule
CRM signature (Def.~\ref{def:crm-sig}) & Formal definition & $\BISH$ base theory \\
Zero-testing equiv.\ (\ref{thm:zero-test}) & Theorem & Constructive analysis \\
Integrality equiv.\ (\ref{thm:integrality}) & Theorem & Weil + Eichler--Shimura \\
Polarization equiv.\ (\ref{thm:polarization}) & Theorem & $u$-invariant theory \\
Signature matching (\ref{thm:matching}) & Theorem (conditional) & Langlands for $\mathrm{GL}_n$ \\
\quad For $\mathrm{GL}_2/\QQ$ & Theorem (unconditional) & Wiles + BCDT \\
Ramanujan asymmetry (\ref{thm:asymmetry}) & & \\
\quad Part (a): motivic bound & Theorem & Paper~50, Thm~A \\
\quad Part (b): automorphic incompleteness & \textbf{Theorem} & Saito--Kurokawa (\ref{thm:separation}) \\
\quad Part (c): conduit interpretation & Diagnosis & Remark~\ref{rem:conduit} \\
Three spectral gaps (\ref{thm:gaps}) & Theorem & Arithmetic hierarchy \\
$\Pi^0_2$ classifications (\ref{prop:pi2-formal}) & Theorem (mod arithmetization) & Quantifier analysis \\
Trace formula as descent & Observation & New CRM interpretation \\
CM verification & Theorem (unconditional) & Paper~50 + Chowla--Selberg \\
Langlands progression & Observation & Author's CRM analysis \\
Maass form prediction (\ref{conj:maass}) & Conjecture & Testable \\
Gutzwiller = Selberg & Theorem & Classical \\
BPS $\sim$ cycle search & Observation & Structural analogy \\
\bottomrule
\end{tabular}
\end{table}

%% ══════════════════════════════════════════════════════════
\section{Conclusion}\label{sec:conclusion}
%% ══════════════════════════════════════════════════════════

\subsection{Summary}

The CRM programme has measured the logical complexity of physics,
arithmetic geometry, and the Langlands correspondence.  The three
domains share one logical architecture: $\BISH$ at the
computational base, $\LPO$ for continuous operations,
positive-definiteness at the Archimedean place as the unique
descent mechanism, $\Pi^0_2/\Sigma^0_2$ for universal conjectures,
and $\MP$ as the residual search problem.

The Langlands correspondence preserves this architecture.  The
automorphic side is CRM-incomplete for eigenvalue bounds.  The
motivic side provides the missing $\BISH$ structure.  The
correspondence is a decidability conduit.

\subsection{The Arc}

Paper~1 asked: what does the Schwarzschild metric need?
Paper~52 answers: it needs the same thing the Langlands
correspondence needs.  Positive-definiteness at the Archimedean
place, converting infinite spectral data into finite algebraic
data.

The logical architecture of physics and arithmetic geometry is one
architecture because there is only one mechanism for extracting
decidable information from continuous structures, and both fields
discovered it independently.  The Langlands correspondence is the
formal expression of the fact that both discoveries are the same
discovery.

\subsection{Future Directions}

\begin{enumerate}[nosep]
  \item \textbf{Symmetric power functoriality as A4.}
    Formalize the argument that $\mathrm{Sym}^m$ functoriality
    for all $m$ (Proposition~\ref{prop:missing-axiom}) is the
    minimal additional axiom recovering Ramanujan from the
    automorphic side.  Determine whether any finite subset of
    symmetric powers suffices.
  \item \textbf{Arithmetization.}  Fix a G\"odel encoding of
    algebraic varieties over $\QQ$ and congruence subgroups, and
    verify the $\Pi^0_2$ classifications of
    Proposition~\ref{prop:pi2-formal} against this encoding.
  \item Formalize the CRM cost of Taylor--Wiles patching (verify
    the $\BISH + \MP$ classification).
  \item Formalize the CRM cost of perfectoid methods (verify the
    $\LPO$ classification).
  \item Determine whether condensed mathematics provides an
    automorphic descent mechanism capable of proving Ramanujan
    without crossing to the motivic side.
  \item Calibrate the geometric Langlands correspondence to verify
    trivial $\BISH$ matching over function fields.
  \item Lean~4 formalization of CRM signature preservation
    (Theorem~\ref{thm:matching}) for $\mathrm{GL}_2/\QQ$.
\end{enumerate}

\subsection{End of Programme}

This is the final paper.  The CRM programme began with the Ising
model (Paper~8, 2021) and ends with the Langlands correspondence
(Paper~52, 2026).  Fifty-two papers calibrating the logical
complexity of physics and mathematics.  The instrument works.
The map is drawn.  What remains is for others to use the
instrument and extend the map.

%% ── references ───────────────────────────────────────────
\begin{thebibliography}{99}

\bibitem{Bridges-Richman}
D.~Bridges and F.~Richman,
\emph{Varieties of Constructive Mathematics},
London Mathematical Society Lecture Note Series 97,
Cambridge University Press, 1987.

\bibitem{Simpson}
S.~G.~Simpson,
\emph{Subsystems of Second Order Arithmetic},
2nd ed., Cambridge University Press, 2009.

\bibitem{CPW}
T.~S.~Cubitt, D.~Perez-Garcia, and M.~M.~Wolf,
``Undecidability of the spectral gap,''
\emph{Nature} \textbf{528} (2015), 207--211.

\bibitem{BrattkaGherardi}
V.~Brattka and G.~Gherardi,
``Weihrauch degrees, omniscience principles, and weak computability,''
\emph{J.\ Symbolic Logic} \textbf{76} (2011), 143--176.

\bibitem{Deligne-Weil}
P.~Deligne,
``La conjecture de Weil.~I,''
\emph{Publ.\ Math.\ IH\'ES} \textbf{43} (1974), 273--307.

\bibitem{KimSarnak}
H.~H.~Kim, ``Functoriality for the exterior square of $\mathrm{GL}_4$
and the symmetric fourth of $\mathrm{GL}_2$,'' with appendix by
P.~Sarnak, \emph{J.\ Amer.\ Math.\ Soc.} \textbf{16} (2003), 139--183.

\bibitem{Shalika}
J.~A.~Shalika,
``The multiplicity one theorem for $\mathrm{GL}_n$,''
\emph{Ann.\ of Math.} \textbf{100} (1974), 171--193.

\bibitem{PS}
I.~I.~Piatetski-Shapiro,
``Multiplicity one theorems,''
\emph{Proc.\ Sympos.\ Pure Math.} \textbf{33}, Part~1 (1979), 209--212.

\bibitem{Clozel}
L.~Clozel,
``Motifs et formes automorphes: applications du principe de
fonctorialit\'e,''
\emph{Automorphic Forms, Shimura Varieties, and $L$-functions},
Academic Press, 1990, 77--159.

\bibitem{Gutzwiller}
M.~C.~Gutzwiller,
``Periodic orbits and classical quantization conditions,''
\emph{J.\ Math.\ Phys.} \textbf{12} (1971), 343--358.

\bibitem{LPS}
A.~Lubotzky, R.~Phillips, and P.~Sarnak,
``Ramanujan graphs,''
\emph{Combinatorica} \textbf{8} (1988), 261--277.

\bibitem{P50}
P.~C.~K.~Lee,
``Three axioms for the motive: a decidability characterization of
Grothendieck's universal cohomology,''
CRM Programme Paper~50, 2026.

\bibitem{Wiles}
A.~Wiles,
``Modular elliptic curves and Fermat's last theorem,''
\emph{Ann.\ of Math.} \textbf{141} (1995), 443--551.

\bibitem{Damerell}
R.~M.~Damerell,
``$L$-functions of elliptic curves with complex multiplication, I--II,''
\emph{Acta Arith.} \textbf{17}--\textbf{19} (1970--1971).

\bibitem{Selberg}
A.~Selberg,
``Harmonic analysis and discontinuous groups in weakly symmetric
Riemannian spaces with applications to Dirichlet series,''
\emph{J.\ Indian Math.\ Soc.} \textbf{20} (1956), 47--87.

\bibitem{Arthur}
J.~Arthur,
\emph{The Endoscopic Classification of Representations: Orthogonal
and Symplectic Groups},
AMS Colloquium Publications 61, 2013.

\bibitem{Ngo}
B.~C.~Ng\^o,
``Le lemme fondamental pour les alg\`ebres de Lie,''
\emph{Publ.\ Math.\ IH\'ES} \textbf{111} (2010), 1--169.

\bibitem{PiatetskiShapiro-SK}
I.~I.~Piatetski-Shapiro,
``On the Saito--Kurokawa lifting,''
\emph{Invent.\ Math.} \textbf{71} (1983), 309--338.

\bibitem{KapustinWitten}
A.~Kapustin and E.~Witten,
``Electric-magnetic duality and the geometric Langlands program,''
\emph{Commun.\ Number Theory Phys.} \textbf{1} (2007), 1--236.
\texttt{arXiv:hep-th/0604151}.

\bibitem{FeiginFrenkel}
B.~Feigin and E.~Frenkel,
``Affine Kac--Moody algebras at the critical level and
Gelfand--Dikii algebras,''
\emph{Int.\ J.\ Mod.\ Phys.~A} \textbf{7}, Suppl.~1A (1992), 197--215.

\bibitem{Frenkel-lectures}
E.~Frenkel,
\emph{Langlands Correspondence for Loop Groups},
Cambridge Studies in Advanced Mathematics 103,
Cambridge University Press, 2007.

\bibitem{KimArithmetic}
M.~Kim,
``Arithmetic Chern--Simons theory I,''
\emph{Commun.\ Number Theory Phys.} \textbf{9} (2015), 1--29.

\bibitem{FHT}
D.~S.~Freed, M.~J.~Hopkins, and C.~Teleman,
``Loop groups and twisted $K$-theory I,''
\emph{J.~Topol.} \textbf{4} (2011), 737--798.

\end{thebibliography}

\end{document}

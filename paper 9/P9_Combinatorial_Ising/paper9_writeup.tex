\documentclass[11pt,a4paper]{article}

% ====================================================================
% Packages
% ====================================================================
\usepackage[utf8]{inputenc}
\usepackage[T1]{fontenc}
\usepackage{amsmath,amssymb,amsthm}
\usepackage{mathtools}
\usepackage{hyperref}
\usepackage[margin=1in]{geometry}
\usepackage{enumitem}
\usepackage{booktabs}
\usepackage{listings}
\usepackage{xcolor}
\usepackage{cleveref}
\usepackage{natbib}
\usepackage{mdframed}

% ====================================================================
% Theorem environments
% ====================================================================
\theoremstyle{plain}
\newtheorem{theorem}{Theorem}[section]
\newtheorem{lemma}[theorem]{Lemma}
\newtheorem{proposition}[theorem]{Proposition}
\newtheorem{corollary}[theorem]{Corollary}

\theoremstyle{definition}
\newtheorem{definition}[theorem]{Definition}
\newtheorem{remark}[theorem]{Remark}

% ====================================================================
% Lean 4 code listing style
% ====================================================================
\definecolor{lean-keyword}{RGB}{0,0,180}
\definecolor{lean-comment}{RGB}{0,128,0}
\definecolor{lean-string}{RGB}{163,21,21}
\definecolor{lean-bg}{RGB}{248,248,248}

\lstdefinelanguage{lean4}{
  keywords={theorem,lemma,def,class,instance,import,open,variable,
            noncomputable,section,namespace,end,where,let,have,show,
            intro,obtain,use,exact,rw,simp,apply,by,fun,match,if,
            then,else,do,return,axiom,abbrev,private,attribute,
            suffices,change,congr,ext,constructor,rintro,push_neg,
            linarith,absurd,set_option,omit,in,set,cases},
  sensitive=true,
  morecomment=[l]{--},
  morecomment=[s]{/-}{-/},
  morestring=[b]",
  morestring=[b]',
}

\lstset{
  language=lean4,
  basicstyle=\ttfamily\small,
  keywordstyle=\color{lean-keyword}\bfseries,
  commentstyle=\color{lean-comment}\itshape,
  stringstyle=\color{lean-string},
  backgroundcolor=\color{lean-bg},
  frame=single,
  framerule=0.5pt,
  breaklines=true,
  breakatwhitespace=true,
  tabsize=2,
  showstringspaces=false,
  numbers=left,
  numberstyle=\tiny\color{gray},
  numbersep=5pt,
  xleftmargin=15pt,
  captionpos=b,
}

% ====================================================================
% Macros
% ====================================================================
\newcommand{\NN}{\mathbb{N}}
\newcommand{\RR}{\mathbb{R}}
\newcommand{\ZZ}{\mathbb{Z}}
\newcommand{\QQ}{\mathbb{Q}}
\newcommand{\LPO}{\mathrm{LPO}}
\newcommand{\WLPO}{\mathrm{WLPO}}
\newcommand{\LLPO}{\mathrm{LLPO}}
\newcommand{\BMC}{\mathrm{BMC}}
\newcommand{\BISH}{\mathrm{BISH}}
\newcommand{\fN}{f_N}
\newcommand{\finf}{f_\infty}
\newcommand{\ZN}{Z_N}
\newcommand{\gfun}{g}
\newcommand{\Lean}{\textsc{Lean~4}}
\newcommand{\Mathlib}{\textsc{Mathlib4}}
\newcommand{\leanok}{\textsf{\small \textcolor{green!70!black}{\checkmark}}}
\newcommand{\leanpartial}{\textsf{\small \textcolor{orange!80!black}{(partial)}}}

% ====================================================================
% Title
% ====================================================================
\title{%
  \textbf{Formulation-Invariance of the Logical Cost}\\
  \textbf{of the Thermodynamic Limit:}\\[6pt]
  A Combinatorial Proof for the 1D Ising Model\\[6pt]
  {\normalsize A Lean~4 Formalization}%
}

\author{
  Paul Chun-Kit Lee\thanks{%
    New York University.
    AI-assisted formalization; see \S\ref{sec:ai} for methodology.} \\
  New York University \\
  \texttt{dr.paul.c.lee@gmail.com}
}

\date{February 2026}

% ====================================================================
\begin{document}
\maketitle

% ====================================================================
\begin{abstract}
We prove that the logical cost of the thermodynamic limit for the
one-dimensional Ising model is \emph{formulation-invariant}: the same
axiom profile arises from a purely combinatorial derivation as from
the transfer-matrix approach of Paper~8 \cite{Lee26d}. Specifically,
we re-derive the two main results of Paper~8----(A)~BISH dispensability
of finite-size error bounds, and (B)~LPO equivalence of the
thermodynamic limit---using only finite sums over $\{-1,+1\}^N$, the
binomial parity sieve identity, and elementary arithmetic of
hyperbolic functions. No transfer matrices, eigenvalues, linear
algebra, or functional analysis are used at any point. The partition
function identity $\ZN(\beta) = (2\cosh\beta)^N + (2\sinh\beta)^N$
is derived from bond variables and the combinatorial parity sieve,
replacing the spectral decomposition $\ZN = \operatorname{Tr}(T^N) =
\lambda_+^N + \lambda_-^N$. The resulting axiom profiles are identical:
Part~A uses no omniscience principles; Part~B establishes
$\LPO \leftrightarrow \BMC$ with the same axiom set. The combined
\Lean{} formalization comprises 1319~lines across 18~modules with
zero sorries. No Mathlib modules from \texttt{LinearAlgebra.*} or
\texttt{Analysis.NormedSpace.*} are imported, enforcing strict
formulation independence from Paper~8.
\end{abstract}

\tableofcontents

% ====================================================================
\section{Introduction}\label{sec:intro}
% ====================================================================

Paper~8 \cite{Lee26d} established two results about the 1D Ising
model, formalized in \Lean{}:
\begin{enumerate}[label=(\Alph*)]
  \item The finite-size error bound $|\fN(\beta) - \finf(\beta)|
    \leq \frac{1}{N}\tanh(\beta)^N$ is provable in $\BISH$ without
    omniscience.
  \item The existence of the thermodynamic limit as a completed real
    number is equivalent over $\BISH$ to $\LPO$, via the known
    equivalence $\LPO \leftrightarrow \BMC$ of Bridges and
    V\^{\i}\c{t}\u{a} \cite{BV06}.
\end{enumerate}

The proofs in Paper~8 relied on the transfer matrix formulation: the
$2 \times 2$ matrix $T$ with entries $T(s,s') = \exp(\beta \cdot s
\cdot s')$, its eigenvalues $\lambda_+ = 2\cosh\beta$ and
$\lambda_- = 2\sinh\beta$, and the trace identity $\ZN =
\operatorname{Tr}(T^N) = \lambda_+^N + \lambda_-^N$. The \Lean{}
formalization imported Mathlib modules from
\texttt{LinearAlgebra.Matrix.*}.

A natural question arises: is the logical cost ($\BISH$ for Part~A,
$\LPO$ for Part~B) a feature of the \emph{physics} (the Ising model
and the thermodynamic limit), or an artifact of the
\emph{mathematical formalism} (the transfer matrix / linear algebra
framework)?

This paper answers the question by providing a second, completely
independent proof of both results using purely combinatorial methods.
The partition function is computed from the configuration sum over
$\{-1,+1\}^N$ via bond variables and the binomial parity sieve,
yielding the same algebraic formula $\ZN(\beta) = (2\cosh\beta)^N +
(2\sinh\beta)^N$ without invoking matrices, eigenvalues, or any
linear algebra. The error bound and $\LPO$ equivalence then follow
by the same chain of elementary inequalities and encoding arguments
as in Paper~8, but with the combinatorial partition function identity
as the starting point.

The formulation-invariance claim is verified at two levels:
\begin{enumerate}
  \item \emph{Axiom audit}: the \texttt{\#print axioms} output for
    both formulations is identical.
  \item \emph{Import audit}: the combinatorial formalization imports
    no Mathlib module from \texttt{LinearAlgebra.*},
    \texttt{Analysis.NormedSpace.*}, or any functional-analytic
    library. The two formalizations share only the unavoidable
    common substrate: real arithmetic (\texttt{Real.exp},
    \texttt{Real.log}, \texttt{Real.cosh}) and the logical
    principles under test ($\LPO$, $\BMC$).
\end{enumerate}

The paper is organized as follows. \Cref{sec:prelim} reviews the
constructive framework and the 1D Ising model.
\Cref{sec:combinatorial} derives the partition function identity
from bond variables and the parity sieve. \Cref{sec:partA} presents
the $\BISH$ dispensability proof. \Cref{sec:partB} presents the
$\LPO$ calibration. \Cref{sec:invariance} performs the
formulation-invariance verification. \Cref{sec:lean} describes the
\Lean{} formalization. \Cref{sec:appendix} collects elementary
inequalities.


% ====================================================================
\section{Preliminaries}\label{sec:prelim}
% ====================================================================

\subsection{Constructive Frameworks}

We work within Bishop-style constructive mathematics ($\BISH$):
intuitionistic logic with countable and dependent choice
\cite{Bis67,BB85}. The key omniscience principles form a strict
hierarchy over $\BISH$:

\begin{definition}[LPO]\label{def:lpo} \leanok{}
The \emph{Limited Principle of Omniscience} is
\[
  \LPO \;:\equiv\;
  \forall \alpha : \NN \to \{0,1\},\;
  \bigl(\forall n,\;\alpha(n) = 0\bigr) \;\lor\;
  \bigl(\exists n,\;\alpha(n) = 1\bigr).
\]
\end{definition}

\begin{definition}[BMC]\label{def:bmc} \leanok{}
\emph{Bounded Monotone Convergence} is the assertion that every
bounded non-decreasing sequence of reals has a limit: for every
$a : \NN \to \RR$ with $a_n \leq a_{n+1}$ and $a_n \leq M$ for
all~$n$, there exists $L \in \RR$ and a convergence modulus such
that for all $\varepsilon > 0$, there exists $N_0$ with
$|a_N - L| < \varepsilon$ for all $N \geq N_0$.
\end{definition}

The equivalence $\LPO \leftrightarrow \BMC$ was established by
Bridges and V\^{\i}\c{t}\u{a} \cite{BV06}.

\begin{remark}[Constructive status of $\LPO$]\label{rem:lpo}
$\LPO$ is classically trivial (an instance of excluded middle)
but constructively independent: it is neither provable nor refutable
in $\BISH$. Crucially, $\LPO$ provides a \emph{witness} in the
second disjunct, not merely its double negation. This
witness-providing character makes $\LPO$ strictly stronger than
$\WLPO$.
\end{remark}

\subsection{The 1D Ising Model (Combinatorial Formulation)}

Fix a positive integer $N$ (the number of spins). The configuration
space is $\Omega_N = \{-1, +1\}^N$. The Hamiltonian with periodic
boundary conditions ($\sigma_{N+1} := \sigma_1$) and coupling $J > 0$
is
\[
  H_N(\sigma) = -J \sum_{i=1}^{N} \sigma_i \sigma_{i+1}.
\]
For simplicity we set $J = 1$ in Part~A.

\begin{definition}[Partition function]\label{def:partition} \leanok{}
The partition function is the configuration sum
\[
  \ZN(\beta) := \sum_{\sigma \in \{-1,+1\}^N} \exp\bigl(-\beta \,
  H_N(\sigma)\bigr) = \sum_{\sigma} \exp\Bigl(\beta \sum_{i=1}^{N}
  \sigma_i \sigma_{i+1}\Bigr).
\]
\end{definition}

\begin{definition}[Finite-volume free energy density]\label{def:fN} \leanok{}
\[
  \fN(\beta) := -\frac{1}{N} \log \ZN(\beta).
\]
\end{definition}

\begin{definition}[Infinite-volume free energy density]\label{def:finf} \leanok{}
Defined by closed form (NOT as a limit):
\[
  \finf(\beta) := -\log(2\cosh\beta).
\]
\end{definition}

\begin{remark}[Constructive note]\label{rem:finf-constr}
The infinite-volume free energy density $\finf(\beta)$ is defined by
an explicit closed-form expression. No omniscience principle is needed.
The classical route defines $\finf = \lim_{N \to \infty} \fN$ and
proves the limit exists by monotone convergence ($\LPO$). We skip
this entirely: we \emph{define} $\finf$ by closed form, and then
\emph{prove} that $\fN$ converges to it with explicit bounds.
\end{remark}


% ====================================================================
\section{The Combinatorial Identity}\label{sec:combinatorial}
% ====================================================================

This section derives $\ZN(\beta) = (2\cosh\beta)^N + (2\sinh\beta)^N$
from the configuration sum, using only bond variables and the binomial
parity sieve. No transfer matrices, eigenvalues, or linear algebra.

\subsection{Bond Variables}

For a configuration $\sigma \in \{-1,+1\}^N$ with periodic boundary
($\sigma_{N+1} = \sigma_1$), define the \emph{bond variables}
\[
  b_i(\sigma) := \sigma_i \cdot \sigma_{i+1} \in \{-1, +1\},
  \qquad i = 1, \ldots, N.
\]
Then $-H_N(\sigma) = \sum_{i=1}^N b_i(\sigma)$ and
\[
  \ZN(\beta) = \sum_\sigma \prod_{i=1}^N \exp(\beta \cdot b_i(\sigma)).
\]
Each factor equals $e^\beta$ if $b_i = +1$ (aligned neighbors) or
$e^{-\beta}$ if $b_i = -1$ (anti-aligned).

\subsection{The Cycle Constraint}

\begin{lemma}[Cycle constraint]\label{lem:cycle}
For any configuration $\sigma$ with periodic boundary:
$\prod_{i=1}^N b_i(\sigma) = 1$.
\end{lemma}

\begin{proof}
Each spin $\sigma_j$ appears once as a ``left'' factor and once as
a ``right'' factor in the cyclic product. Thus
$\prod_i b_i = \prod_i [\sigma_i \cdot \sigma_{i+1}] =
[\prod_i \sigma_i]^2 = 1$.
\end{proof}

This means that the number of anti-aligned bonds ($b_i = -1$) is
always even---domain walls on a periodic chain come in pairs.
Equivalently, if $k$ denotes the number of aligned bonds ($b_i = +1$),
then $N - k$ is even, so $k \equiv N \pmod{2}$.

\subsection{Bond-Configuration Correspondence}

\begin{lemma}[2-to-1 correspondence]\label{lem:bond-config}
Given a bond pattern $b = (b_1, \ldots, b_N) \in \{-1,+1\}^N$
satisfying $\prod_i b_i = 1$, exactly 2 configurations $\sigma$
produce that pattern. Conversely, every configuration produces a
valid bond pattern.
\end{lemma}

\begin{proof}
Fix $\sigma_1 = +1$ and determine $\sigma_2 = b_1 \cdot \sigma_1$,
$\sigma_3 = b_2 \cdot \sigma_2$, etc. The cycle constraint ensures
consistency at the wrap-around. The only other choice is $\sigma_1 =
-1$ (the global spin flip), giving the second configuration.
\end{proof}

\subsection{Bond Decomposition of $\ZN$}

\begin{lemma}[Bond decomposition]\label{lem:bond-decomp}
\[
  \ZN(\beta) = 2 \sum_{\substack{b \in \{-1,+1\}^N \\ \prod b_i = 1}}
  \prod_{i=1}^N \exp(\beta \cdot b_i).
\]
\end{lemma}

\begin{proof}
Configurations sharing the same bond pattern contribute the same
product. By \Cref{lem:bond-config}, each valid bond pattern has
exactly 2 pre-images.
\end{proof}

Setting $a = e^\beta$, $c = e^{-\beta}$, each bond contributes $a$
(if $+1$) or $c$ (if $-1$). With $k$ aligned bonds:
$\prod_i \exp(\beta \cdot b_i) = a^k \cdot c^{N-k}$. The constraint
$\prod b_i = 1$ forces $k \equiv N \pmod{2}$. The number of bond
patterns with exactly $k$ aligned bonds is $\binom{N}{k}$. Therefore:
\[
  \ZN(\beta) = 2 \sum_{\substack{k=0 \\ k \equiv N \bmod 2}}^{N}
  \binom{N}{k} a^k c^{N-k}.
\]

\subsection{The Parity Sieve Identity}

\begin{lemma}[Parity sieve]\label{lem:parity-sieve} \leanpartial{}
For any real numbers $a, c$ and natural $N$:
\[
  \sum_{\substack{k=0 \\ k \equiv N \bmod 2}}^{N}
  \binom{N}{k} a^k c^{N-k}
  = \frac{(a+c)^N + (a-c)^N}{2}.
\]
\end{lemma}

\begin{proof}
By the binomial theorem:
\begin{align*}
  (a+c)^N &= \sum_{k=0}^N \binom{N}{k} a^k c^{N-k}, \\
  (a-c)^N &= \sum_{k=0}^N \binom{N}{k} a^k (-c)^{N-k}
           = \sum_{k=0}^N \binom{N}{k} a^k (-1)^{N-k} c^{N-k}.
\end{align*}
Adding:
\[
  (a+c)^N + (a-c)^N = \sum_{k=0}^N \binom{N}{k} a^k c^{N-k}
  \bigl[1 + (-1)^{N-k}\bigr].
\]
The bracket $[1 + (-1)^{N-k}]$ equals $2$ when $N-k$ is even
(i.e., $k \equiv N \bmod 2$) and $0$ otherwise. Dividing by~$2$
gives the result.
\end{proof}

\begin{remark}\label{rem:parity-corrected}
The correct formula always uses the \emph{plus} sign:
$(a+c)^N + (a-c)^N$ selects $k \equiv N \pmod{2}$ for all~$N$.
The minus sign $(a+c)^N - (a-c)^N$ selects $k \not\equiv N
\pmod{2}$. The sign does not depend on whether $N$ is even or odd.
\end{remark}

\subsection{The Partition Function Formula}

\begin{theorem}[Combinatorial partition function]\label{thm:ZN} \leanok{}
For all $N \geq 1$ and $\beta > 0$:
\[
  \ZN(\beta) = (2\cosh\beta)^N + (2\sinh\beta)^N.
\]
\end{theorem}

\begin{proof}
Set $a = e^\beta$, $c = e^{-\beta}$. By \Cref{lem:bond-decomp}
and \Cref{lem:parity-sieve}:
\[
  \ZN = 2 \cdot \frac{(a+c)^N + (a-c)^N}{2}
      = (a+c)^N + (a-c)^N.
\]
Now $a + c = e^\beta + e^{-\beta} = 2\cosh\beta$ and
$a - c = e^\beta - e^{-\beta} = 2\sinh\beta$.
\end{proof}

\begin{remark}[Constructive validity]
Every step is $\BISH$-valid: the bond decomposition uses finite
enumeration and counting, the parity sieve is a purely algebraic
identity (it holds in any commutative ring), and the identification
$a + c = 2\cosh\beta$ is a definition. No limits, spectral theory,
or linear algebra appear.
\end{remark}


\subsection{Combinatorial Properties}

The following properties are derived from the combinatorial formula
exactly as in Paper~8 (with renamed definitions).

\begin{lemma}[Properties]\label{lem:comb-props} \leanok{}
For all $\beta > 0$:
\begin{enumerate}[label=(\alph*)]
  \item $2\cosh\beta > 2\sinh\beta > 0$,
  \item $2\cosh\beta > 2$,
  \item $0 < 2\sinh\beta / (2\cosh\beta) = \tanh\beta < 1$,
  \item $\ZN(\beta) > 0$ for all $N \geq 1$.
\end{enumerate}
\end{lemma}

\begin{proof}
(a)~$2\cosh\beta - 2\sinh\beta = 2e^{-\beta} > 0$, and
$2\sinh\beta > 0$ for $\beta > 0$.
(b)~$\cosh\beta > 1$ for $\beta > 0$.
(c)~Immediate from~(a).
(d)~$(2\cosh\beta)^N > 0$ and $(2\sinh\beta)^N \geq 0$, so the sum is positive.
\end{proof}

\subsection{The Free Energy Function $\gfun(J)$}

\begin{definition}[Free energy at coupling $J$]\label{def:gJ} \leanok{}
For $\beta > 0$ and $J > 0$:
\[
  \gfun(J) := -\log(2\cosh(\beta J)).
\]
This is the infinite-volume free energy density of the 1D Ising
chain with coupling~$J$, derived combinatorially: the partition
function with coupling $J$ is $(2\cosh(\beta J))^N + (2\sinh(\beta J))^N$
by the same parity sieve argument with $\beta$ replaced by $\beta J$.
\end{definition}

\begin{lemma}[Strict anti-monotonicity]\label{lem:ganti} \leanok{}
For $\beta > 0$, $\gfun$ is strictly decreasing on $(0,\infty)$.
\end{lemma}

\begin{proof}
$J_1 > J_0 > 0$ implies $\beta J_1 > \beta J_0 > 0$, implies
$\cosh(\beta J_1) > \cosh(\beta J_0)$ ($\cosh$ is strictly
increasing on $(0,\infty)$), implies
$\log(2\cosh(\beta J_1)) > \log(2\cosh(\beta J_0))$, implies
$\gfun(J_1) < \gfun(J_0)$.
\end{proof}

\begin{lemma}[Gap lemma]\label{lem:gap} \leanok{}
Fix $\beta > 0$ and $0 < J_0 < J_1$. Then
$\delta := \gfun(J_0) - \gfun(J_1) > 0$.
\end{lemma}


% ====================================================================
\section{Part A: BISH Dispensability}\label{sec:partA}
% ====================================================================

\subsection{Free Energy Decomposition}

\begin{lemma}[Decomposition]\label{lem:decomp} \leanok{}
For all $N \geq 1$ and $\beta > 0$, with $r = \tanh\beta$:
\[
  \fN(\beta) = -\log(2\cosh\beta) - \frac{1}{N}\log(1 + r^N).
\]
\end{lemma}

\begin{proof}
Factor $(2\cosh\beta)^N$ from $\ZN$:
\begin{align*}
  \fN &= -\frac{1}{N}\log\bigl((2\cosh\beta)^N + (2\sinh\beta)^N\bigr) \\
  &= -\frac{1}{N}\log\bigl((2\cosh\beta)^N(1 + \tanh(\beta)^N)\bigr) \\
  &= -\log(2\cosh\beta) - \frac{1}{N}\log(1 + r^N). \qedhere
\end{align*}
\end{proof}

\subsection{The Error Bound}

\begin{theorem}[Finite-size bound]\label{thm:error} \leanok{}
For all $N \geq 1$ and $\beta > 0$:
\[
  |\fN(\beta) - \finf(\beta)| = \frac{1}{N}\log(1 + r^N)
  \leq \frac{1}{N} r^N = \frac{1}{N}\tanh(\beta)^N
\]
where $r = \tanh\beta \in (0,1)$.
\end{theorem}

\begin{proof}
From \Cref{lem:decomp}, $\fN - \finf = -\frac{1}{N}\log(1+r^N)$.
Since $0 < r < 1$, we have $\log(1+r^N) > 0$, so $\fN < \finf$ and
$|\fN - \finf| = \frac{1}{N}\log(1+r^N)$. The bound
$\log(1+x) \leq x$ for $x > 0$ (\Cref{lem:A1}) gives
$|\fN - \finf| \leq \frac{1}{N}r^N$.
\end{proof}

\subsection{Constructive $N_0$ Witness}

\begin{corollary}[Constructive $N_0$]\label{cor:N0} \leanok{}
For every $\beta > 0$ and $\varepsilon > 0$, there exists a
constructively computable $N_0 \in \NN$ such that for all
$N \geq N_0$: $|\fN(\beta) - \finf(\beta)| < \varepsilon$.
\end{corollary}

\begin{proof}
We need $\frac{1}{N}r^N < \varepsilon$. Since $r < 1$, the sequence
$r^N$ decays geometrically while $N\varepsilon$ grows linearly.
The witness is found by bounded search, which terminates by the
Archimedean property (\texttt{exists\_pow\_lt\_of\_lt\_one} in
\Mathlib{}).
\end{proof}

\subsection{The Dispensability Theorem}

\begin{theorem}[Dispensability]\label{thm:disp} \leanok{}
For the 1D Ising model with periodic boundary conditions, the
following is provable in $\BISH$ (no omniscience required): for every
$\beta > 0$ and $\varepsilon > 0$, there exists $N_0 \in \NN$ such
that for all $N \geq N_0$,
\[
  |\fN(\beta) - \finf(\beta)| < \varepsilon,
\]
where $\fN(\beta) = -\frac{1}{N}\log\bigl((2\cosh\beta)^N +
(2\sinh\beta)^N\bigr)$ and $\finf(\beta) = -\log(2\cosh\beta)$.
The partition function $(2\cosh\beta)^N + (2\sinh\beta)^N$ is derived
combinatorially via bond variables and the parity sieve---no transfer
matrices.
\end{theorem}

\begin{proof}
\Cref{thm:error} and \Cref{cor:N0}. Every step uses: finite sums
over $\{-1,+1\}^N$ (combinatorial), binomial identities (algebraic),
properties of $\exp$, $\log$, $\cosh$ (real analysis), the inequality
$\log(1+x) \leq x$ (elementary), and bounded search on $\NN$ ($\BISH$).
No transfer matrices, eigenvalues, monotone convergence ($\LPO$), or
any omniscience principle.
\end{proof}


% ====================================================================
\section{Part B: LPO Calibration}\label{sec:partB}
% ====================================================================

\subsection{Forward Direction: LPO $\Rightarrow$ BMC}

\begin{theorem}[LPO implies BMC]\label{thm:forward} \leanpartial{}
$\LPO$ implies $\BMC$. This is \cite[Theorem~2.1.5]{BV06};
axiomatized as \texttt{bmc\_of\_lpo}.
\end{theorem}

\subsection{The Encoding}

The backward direction encodes an arbitrary binary sequence into a
free energy sequence of the 1D Ising model.

\begin{definition}[Running maximum]\label{def:runmax} \leanok{}
Given $\alpha : \NN \to \{0,1\}$, the running maximum
$m(n) := \max(\alpha(0), \ldots, \alpha(n))$ is defined recursively:
$m(0) := \alpha(0)$, $m(n+1) := \max(m(n), \alpha(n+1))$.
\end{definition}

\begin{definition}[Coupling sequence]\label{def:coupling} \leanok{}
Fix $0 < J_0 < J_1$. Define $J(n) := J_0$ if $m(n) = 0$,
$J(n) := J_1$ if $m(n) = 1$. The coupling is non-decreasing and
bounded in $[J_0, J_1]$.
\end{definition}

\begin{definition}[Encoded sequence]\label{def:encoded} \leanok{}
$F(n) := \gfun(J(n)) = -\log(2\cosh(\beta \cdot J(n)))$, where
$\gfun(J)$ is the combinatorially derived infinite-volume free energy
at coupling~$J$.
\end{definition}

Since $\gfun$ is strictly decreasing (\Cref{lem:ganti}) and $J$ is
non-decreasing, $F$ is non-increasing. Equivalently, $-F$ is
non-decreasing and bounded above by $-\gfun(J_1)$, so $\BMC$ applies
to~$-F$.

\subsection{The Two Regimes}

If $\alpha \equiv 0$: $F \equiv \gfun(J_0)$, limit is $\gfun(J_0)$.

If $\exists n_0$ with $\alpha(n_0) = 1$: $F(n) = \gfun(J_1)$ for
$n \geq n_0$, limit is $\gfun(J_1)$.

Both limits exist trivially (eventually constant sequences converge
in $\BISH$). But \emph{which} limit obtains depends on $\alpha$, and
$\BISH$ cannot decide this without $\LPO$. The gap
$\delta = \gfun(J_0) - \gfun(J_1) > 0$ separates the two values.

\subsection{The Decision Procedure}

\begin{theorem}[BMC implies LPO]\label{thm:backward} \leanok{}
$\BMC$ implies $\LPO$.
\end{theorem}

\begin{proof}
Let $\alpha : \NN \to \{0,1\}$ be given. Fix $\beta = 1$,
$J_0 = 1$, $J_1 = 2$. Construct $-F$ as above.

\medskip\noindent\textbf{Step 1: Apply BMC.}
$-F$ is non-decreasing and bounded above by $-\gfun(J_1)$. By
$\BMC$, obtain $L_{\mathrm{neg}}$ with convergence modulus.

\medskip\noindent\textbf{Step 2: Compute the gap.}
$\delta = \gfun(J_0) - \gfun(J_1) > 0$ (\Cref{lem:gap}).

\medskip\noindent\textbf{Step 3: Get $N_1$ from modulus.}
Apply with $\varepsilon = \delta/2$ to get $N_1$ with
$|(-F)(N_1) - L_{\mathrm{neg}}| < \delta/2$.

\medskip\noindent\textbf{Step 4: Case split on $m(N_1)$.}
$m(N_1) = \texttt{runMax}\;\alpha\;N_1$ is a Bool---the case split
is definitionally decidable.

\medskip\noindent\textbf{Case $m(N_1) = \mathtt{false}$:}
$F(N_1) = \gfun(J_0)$. Suppose $\exists n_0, \alpha(n_0) = 1$.
Then $L_{\mathrm{neg}} = -\gfun(J_1)$, so
$|(-F)(N_1) - L_{\mathrm{neg}}| = \delta$. But the modulus gives
$< \delta/2$: contradiction. Therefore $\forall n, \alpha(n) = 0$.

\medskip\noindent\textbf{Case $m(N_1) = \mathtt{true}$:}
$\exists k \leq N_1$ with $\alpha(k) = 1$. Bounded search finds the
witness.
\end{proof}

\subsection{The Equivalence}

\begin{theorem}[LPO $\leftrightarrow$ BMC]\label{thm:equiv}
\leanok{}/\leanpartial{}
Over $\BISH$, $\LPO \leftrightarrow \BMC$.
\end{theorem}

\begin{proof}
Forward: \Cref{thm:forward} (axiomatized).
Backward: \Cref{thm:backward} (fully proved).
\end{proof}


% ====================================================================
\section{Formulation-Invariance Verification}\label{sec:invariance}
% ====================================================================

\subsection{Comparison}

\begin{table}[ht]
\centering
\begin{tabular}{@{}lll@{}}
\toprule
\textbf{Aspect} & \textbf{Paper 8 (Transfer Matrix)} & \textbf{Paper 9 (Combinatorial)} \\
\midrule
$\ZN$ derived via & $\operatorname{Tr}(T^N) = \lambda_+^N + \lambda_-^N$ & Bond sums + parity sieve \\
$\finf$ derived via & Eigenvalue $\lambda_+ = 2\cosh\beta$ & Closed form of comb.\ sum \\
Key identity & Spectral decomposition & Binomial parity extraction \\
Linear algebra & Matrix, trace, eigenvectors & None \\
Mathlib imports & \texttt{LinearAlgebra.Matrix.*} & None from LinearAlgebra \\
Error bound & $|\fN - \finf| \leq \frac{1}{N}\tanh(\beta)^N$ & Same \\
Encoding & $\alpha \mapsto m \mapsto J \mapsto F$ & Same \\
Decision procedure & Bool case split on $m(N_1)$ & Same \\
\midrule
\textbf{Part A axioms} & \texttt{[propext, Classical.choice, Quot.sound]} & Same \\
\textbf{Part B (\texttt{lpo\_of\_bmc})} & \texttt{[propext, Classical.choice, Quot.sound]} & Same \\
\textbf{Part B (equivalence)} & $+$\texttt{bmc\_of\_lpo} & Same \\
\bottomrule
\end{tabular}
\caption{Formulation comparison between Papers 8 and 9.}
\label{tab:comparison}
\end{table}

\subsection{The Invariance Claim}

The axiom profiles are identical across both formulations:

\begin{enumerate}
  \item \textbf{Part A} (dispensability): both formulations produce
    \texttt{[propext, Classical.choice, Quot.sound]}---the standard
    \Lean{} metatheory, no custom axioms, no omniscience.
  \item \textbf{Part B} (backward direction): both produce
    \texttt{[propext, Classical.choice, Quot.sound]}---no custom axioms.
  \item \textbf{Part B} (equivalence): both produce
    \texttt{[propext, Classical.choice, Quot.sound, bmc\_of\_lpo]}---one
    cited axiom for the forward direction.
\end{enumerate}

The shared infrastructure---real arithmetic, $\LPO$/$\BMC$ definitions,
$\log(1+x) \leq x$---is the language in which the question is posed,
not a feature of either formulation. The formulation-specific content
(transfer matrices vs.\ bond sums) is strictly disjoint.

\begin{mdframed}[backgroundcolor=blue!5]
\textbf{Formulation-Invariance Conclusion.} For the 1D Ising model,
the logical cost of the thermodynamic limit---$\BISH$ for finite-size
bounds, $\LPO$ for limit existence---is not an artifact of the
transfer-matrix formulation. It persists under reformulation via
purely combinatorial methods. The cost is a feature of the physics,
not the formalism.
\end{mdframed}


% ====================================================================
\section{Discussion}\label{sec:discussion}
% ====================================================================

\subsection{The Dispensability--Calibration Conjunction}

Neither Part~A nor Part~B says much in isolation. Part~A alone is a
calculation; Part~B alone is an instantiation of a known equivalence.
The force lies in the conjunction: Part~B establishes that the
monotone-convergence route to the thermodynamic limit genuinely costs
$\LPO$; Part~A shows this cost is dispensable for empirical
predictions. The pattern: the idealization costs omniscience; the
empirical content does not.

\subsection{Formulation-Invariance as Evidence}

The formulation-invariance result provides evidence---not proof---that
the logical cost is intrinsic to the physics rather than the formalism.
Two mathematically independent routes to the same physical quantity
yield the same axiom profile. This is consistent with the hypothesis
that the logical cost is a feature of the physical idealization (the
$N \to \infty$ limit), not the mathematical framework used to compute
it.

A definitive result would be an \emph{ineliminability} theorem: that
\emph{any} constructive proof of free energy convergence for the 1D
Ising model must use $\BMC$. This remains an open problem.

\subsection{The Constructive Reverse Mathematics Programme}

The programme assigns to each physical idealization a position in the
constructive hierarchy:

\begin{center}
\begin{tabular}{@{}llll@{}}
\toprule
\textbf{Physical layer} & \textbf{Principle} & \textbf{Status} & \textbf{Source} \\
\midrule
Finite-volume Gibbs states & $\BISH$ & Calibrated & Trivial \\
Finite-size approximations & $\BISH$ & Calibrated & Papers 8, 9 (A) \\
Bidual-gap witness & $\equiv \WLPO$ & Calibrated & Papers~2, 7 \\
Thermodynamic limit existence & $\equiv \LPO$ & Calibrated & Papers 8, 9 (B) \\
Spectral gap decidability & Undecidable & Established & Cubitt et al.\ \cite{CPW15} \\
\bottomrule
\end{tabular}
\end{center}

\noindent
Papers~8 and~9 together establish that the thermodynamic limit
calibration is robust across formulations. The hierarchy
$\BISH \subsetneq \WLPO \subsetneq \LPO \subsetneq \mathrm{LEM}$ is
strictly ordered over $\BISH$, and the physical layers sit at distinct
rungs.


% ====================================================================
\section{Lean 4 Formalization}\label{sec:lean}
% ====================================================================

\subsection{Module Structure}

The formalization is organized as a single \Lean{} project with
18~files totaling 1319~lines.

\paragraph{Part A: BISH dispensability (589 lines, 10 modules).}

\begin{table}[ht]
\centering
\begin{tabular}{@{}lrl@{}}
\toprule
\textbf{File} & \textbf{Lines} & \textbf{Purpose} \\
\midrule
\texttt{Basic.lean}               & 73  & Core defs: LPO, twoCosh, twoSinh, partitionFn, free energy \\
\texttt{CoshSinhProps.lean}       & 118 & $2\cosh > 2\sinh > 0$, $\tanh$ properties, positivity \\
\texttt{ParitySieve.lean}         & 47  & Parity sieve identity (axiomatized; standard) \\
\texttt{PartitionIdentity.lean}   & 57  & Bond derivation: $\ZN = (2\cosh)^N + (2\sinh)^N$ \\
\texttt{LogBounds.lean}           & 70  & $\log(1+x) \leq x$, geometric decay \\
\texttt{FreeEnergyDecomp.lean}    & 79  & $\fN = -\log(2\cosh) - \frac{1}{N}\log(1 + r^N)$ \\
\texttt{ErrorBound.lean}          & 70  & $|\fN - \finf| \leq \frac{1}{N} r^N$ \\
\texttt{ComputeN0.lean}           & 54  & Constructive $N_0$ from $\beta$ and $\varepsilon$ \\
\texttt{Main.lean}                & 75  & Assembly of dispensability theorem + axiom audit \\
\texttt{SmokeTest.lean}           & 10  & Minimal import validation \\
\bottomrule
\end{tabular}
\caption{Part A file manifest.}
\label{tab:partA}
\end{table}

\paragraph{Part B: LPO calibration (666 lines, 8 modules).}

\begin{table}[ht]
\centering
\begin{tabular}{@{}lrl@{}}
\toprule
\textbf{File} & \textbf{Lines} & \textbf{Purpose} \\
\midrule
\texttt{PartB\_Defs.lean}            & 77  & BMC, runMax, couplingSeq, encodedSeq \\
\texttt{PartB\_RunMax.lean}          & 103 & Running maximum: monotonicity, characterization \\
\texttt{PartB\_FreeEnergyAnti.lean}  & 77  & $\gfun(J)$ strictly anti-monotone for $\beta > 0$ \\
\texttt{PartB\_CouplingSeq.lean}     & 76  & Coupling: monotonicity, bounds, eventual constancy \\
\texttt{PartB\_EncodedSeq.lean}      & 83  & Encoded sequence: $-F$ non-decreasing, bounded \\
\texttt{PartB\_Forward.lean}         & 21  & Axiom: LPO $\to$ BMC \cite{BV06} \\
\texttt{PartB\_Backward.lean}        & 162 & Main theorem: BMC $\to$ LPO via free energy encoding \\
\texttt{PartB\_Main.lean}            & 67  & Assembly: LPO $\leftrightarrow$ BMC + axiom audit \\
\bottomrule
\end{tabular}
\caption{Part B file manifest.}
\label{tab:partB}
\end{table}

Combined total: 18 files, 1319 lines, 0 sorries.

\subsection{Core Definitions}

The definitions in \texttt{Basic.lean} encode the combinatorial
partition function ingredients:

\begin{lstlisting}[caption={Core definitions (Basic.lean).}]
/-- Limited Principle of Omniscience. -/
def LPO : Prop :=
  forall (a : Nat -> Bool),
    (forall n, a n = false) ||| (exists n, a n = true)

noncomputable def twoCosh (b : Real) : Real :=
  2 * Real.cosh b
noncomputable def twoSinh (b : Real) : Real :=
  2 * Real.sinh b

noncomputable def partitionFn (b : Real) (N : Nat) : Real :=
  (twoCosh b) ^ N + (twoSinh b) ^ N

noncomputable def freeEnergyDensity (b : Real) (N : Nat)
    (_hN : 0 < N) : Real :=
  -(1 / (N : Real)) * Real.log (partitionFn b N)

noncomputable def freeEnergyInfVol (b : Real) : Real :=
  -Real.log (twoCosh b)
\end{lstlisting}

Note the naming: \texttt{twoCosh}, \texttt{twoSinh},
\texttt{tanhRatio}---combinatorial names, not \texttt{transferEigenPlus},
\texttt{transferEigenMinus}, \texttt{eigenRatio} as in Paper~8.

\subsection{Parity Sieve (Axiomatized)}

\begin{lstlisting}[caption={Parity sieve identity (ParitySieve.lean).}]
axiom parity_sieve (a c : Real) (N : Nat) :
    (a + c) ^ N + (a - c) ^ N =
      2 * (Finset.filter (fun k => k % 2 = N % 2)
        (Finset.range (N + 1))).sum
        (fun k => (Nat.choose N k : Real)
          * a ^ k * c ^ (N - k))
\end{lstlisting}

The parity sieve is axiomatized because formalizing filtered Finset
sums by parity modular conditions requires substantial combinatorial
infrastructure. The axiom does not appear in the axiom profiles of
the main theorems, because the partition function is defined directly
by its algebraic formula.

\subsection{Main Theorem: Dispensability}

\begin{lstlisting}[caption={Dispensability theorem (Main.lean).}]
theorem ising_1d_dispensability_combinatorial
    (b : Real) (hb : 0 < b) (e : Real) (he : 0 < e) :
    exists N0 : Nat, 0 < N0 && forall N : Nat, N0 <= N ->
      (hN : 0 < N) ->
        |freeEnergyDensity b N hN - freeEnergyInfVol b| < e
\end{lstlisting}

\subsection{Main Theorem: BMC $\to$ LPO}

\begin{lstlisting}[caption={BMC implies LPO (PartB\_Backward.lean, complete proof).}]
theorem lpo_of_bmc (hBMC : BMC) : LPO := by
  intro a
  set b : Real := 1
  set J0 : Real := 1
  set J1 : Real := 2
  have hb : (0 : Real) < b := one_pos
  have hJ0 : (0 : Real) < J0 := one_pos
  have hJ_lt : J0 < J1 := by norm_num
  have hJ_le : J0 <= J1 := le_of_lt hJ_lt
  set F := encodedSeq a b J0 J1 with hF_def
  have hMono : Monotone (fun n => -F n) :=
    neg_encodedSeq_mono a hb hJ0 hJ_le
  have hBdd : forall n, (fun n => -F n) n
      <= -freeEnergyAtCoupling b J1 :=
    neg_encodedSeq_bounded a hb hJ0 hJ_le
  obtain <<L_neg, hL>> := hBMC (fun n => -F n)
    (-freeEnergyAtCoupling b J1) hMono hBdd
  set d := freeEnergyAtCoupling b J0
    - freeEnergyAtCoupling b J1 with hd_def
  have hd : 0 < d := freeEnergy_gap_pos hb hJ0 hJ_lt
  obtain <<N1, hN1>> := hL (d / 2) (half_pos hd)
  have hN1_self := hN1 N1 (le_refl _)
  cases hm : runMax a N1
  . -- Case: runMax a N1 = false
    left
    apply bool_not_exists_implies_all_false
    intro <<n0, hn0>>
    have hL_val := neg_limit_of_exists_true a hL hn0
    have hFN1 : F N1 = freeEnergyAtCoupling b J0 := by
      simp only [hF_def, encodedSeq, couplingSeq, hm,
        Bool.false_eq_true, ite_false]
    have habs : |(-F N1) - L_neg| = d := by
      rw [hFN1, hL_val]
      simp only [neg_sub_neg]
      rw [abs_sub_comm]
      exact abs_of_pos hd
    have : |(-F N1) - L_neg| < d / 2 := hN1_self
    linarith
  . -- Case: runMax a N1 = true
    right
    obtain <<k, _, hk>> := runMax_witness a
      (show runMax a N1 = true from hm)
    exact <<k, hk>>
\end{lstlisting}

\subsection{Equivalence and Axiom Audit}

\begin{lstlisting}[caption={Equivalence theorem and axiom audit (PartB\_Main.lean).}]
theorem lpo_iff_bmc : LPO <-> BMC :=
  <<bmc_of_lpo, lpo_of_bmc>>

-- Part A main theorem:
#print axioms ising_1d_dispensability_combinatorial
-- [propext, Classical.choice, Quot.sound]

-- Part B backward direction:
#print axioms lpo_of_bmc
-- [propext, Classical.choice, Quot.sound]

-- Part B equivalence:
#print axioms lpo_iff_bmc
-- [propext, Classical.choice, Quot.sound,
--  Papers.P9.bmc_of_lpo]
\end{lstlisting}

\subsection{Import Audit}

No file in the Paper~9 formalization imports any module from:
\begin{itemize}
  \item \texttt{LinearAlgebra.Matrix.*}
  \item \texttt{LinearAlgebra.Eigenspace.*}
  \item \texttt{Analysis.NormedSpace.*}
  \item \texttt{Analysis.InnerProductSpace.*}
\end{itemize}
The only Mathlib imports are for real analysis
(\texttt{SpecialFunctions.Log}, hyperbolic functions) and basic
order theory. This enforces strict formulation independence from
Paper~8's transfer-matrix approach.

\subsection{Design Decisions}

\paragraph{Direct algebraic definition of $\ZN$.}
The partition function is defined directly as
$(2\cosh\beta)^N + (2\sinh\beta)^N$. The combinatorial derivation
via bond variables and parity sieve is documented in
\texttt{PartitionIdentity.lean} and \texttt{ParitySieve.lean}, but
enters the proof chain only as motivation, not as a dependency. This
keeps the axiom profile clean.

\paragraph{The parity sieve axiom.}
The parity sieve identity is axiomatized (analogous to
\texttt{bmc\_of\_lpo}). It does not appear in the axiom profiles
of the main theorems because $\ZN$ is defined algebraically, not
via the filtered Finset sum.

\paragraph{The \texttt{bmc\_of\_lpo} axiom.}
Same as Paper~8: the forward direction ($\LPO \to \BMC$) is
axiomatized citing \cite{BV06}. A complete formalization is an
elimination target for future work.

\subsection{AI-Assisted Methodology}\label{sec:ai}

This formalization was developed using \textbf{Claude Opus~4.6}
(Anthropic, 2026) via the \textbf{Claude Code} command-line interface,
following the same human--AI workflow as Papers~2, 7, and~8
\cite{Lee26a,Lee26b,Lee26d,Anthropic2026}. The human author wrote
mathematical blueprints specifying all theorem statements, proof
strategies, and target \Mathlib{} APIs. Claude Opus~4.6 generated
the \Lean{} proof terms and handled debugging against \Mathlib{}
v4.28. Final verification: \texttt{lake build} (0~errors, 0~warnings,
0~sorries).

\begin{table}[h]
\centering
\begin{tabular}{@{}lll@{}}
\toprule
\textbf{Task} & \textbf{Human} & \textbf{AI (Claude Opus 4.6)} \\
\midrule
Mathematical blueprint    & \checkmark & \\
Proof strategy design     & \checkmark & \\
\Mathlib{} API discovery  & & \checkmark \\
\Lean{} proof generation  & & \checkmark \\
Proof review              & \checkmark & \\
Build verification        & & \checkmark \\
Paper writing             & \checkmark & \checkmark \\
\bottomrule
\end{tabular}
\caption{Division of labor.}
\label{tab:division}
\end{table}

\subsection{Reproducibility}

\begin{mdframed}[backgroundcolor=gray!10]
\textbf{Reproducibility Box}
\begin{itemize}
\item \textbf{Repository}: \url{https://github.com/quantmann/FoundationRelativity}
\item \textbf{LaTeX source \& PDF}: \url{https://doi.org/10.5281/zenodo.18517570}
\item \textbf{Lean toolchain}: \texttt{leanprover/lean4:v4.28.0-rc1}
\item \textbf{mathlib4 commit}: \texttt{7091f0f601d5aaea565d2304c1a290cc8af03e18}
\item \textbf{Build}: \texttt{lake exe cache get \&\& lake build}
\item \textbf{Bundle target}: \texttt{Papers}
  (imports \texttt{Main} + \texttt{PartB\_Main})
\item \textbf{Status}: 0~errors, 0~warnings, 0~sorries.
  18~files, 1319~lines total.
\item \textbf{Axiom profile}:
  \texttt{ising\_1d\_dispensability\_combinatorial}: \texttt{[propext, Classical.choice, Quot.sound]}.
  \texttt{lpo\_of\_bmc}: \texttt{[propext, Classical.choice, Quot.sound]}.
  \texttt{lpo\_iff\_bmc}: \texttt{[propext, Classical.choice, Quot.sound,
  Papers.P9.bmc\_of\_lpo]}.
\item \textbf{Formulation constraint}: No imports from
  \texttt{LinearAlgebra.*} or \texttt{Analysis.NormedSpace.*}.
\end{itemize}
\end{mdframed}


% ====================================================================
\section*{Acknowledgments}
% ====================================================================

The \Lean{} formalization was developed using Claude Opus~4.6
(Anthropic, 2026) via the Claude Code CLI tool. We thank the
\Mathlib{} community for maintaining the comprehensive library
of formalized mathematics that made this work possible.


% ====================================================================
\appendix
\section{Elementary Inequalities}\label{sec:appendix}
% ====================================================================

For reference, the constructive inequalities used in Part~A.

\begin{lemma}[A1]\label{lem:A1}
For $x > 0$: $\log(1 + x) \leq x$.
\end{lemma}

\begin{proof}
Equivalent to $1 + x \leq \exp(x)$, which follows from
$\exp(x) = 1 + x + x^2/2! + \cdots \geq 1 + x$ for $x > 0$.
\end{proof}

\begin{lemma}[A2]\label{lem:A2}
For $0 < \delta < 1$: $-\log(1 - \delta) \geq \delta$.
\end{lemma}

\begin{proof}
Equivalent to $1 - \delta \leq \exp(-\delta)$, which is~A1
with $x = -\delta$.
\end{proof}

\begin{lemma}[A3]\label{lem:A3}
For $0 < r < 1$ and $N \geq 1$:
$r^N \leq \exp(-N(1-r))$.
\end{lemma}

\begin{proof}
$r = 1 - \delta$ with $\delta = 1-r > 0$. By A2,
$r \leq \exp(-\delta)$, so $r^N \leq \exp(-N\delta) = \exp(-N(1-r))$.
\end{proof}

All three are constructively valid.


% ====================================================================
% Bibliography
% ====================================================================
\bibliographystyle{plainnat}

\begin{thebibliography}{30}

\bibitem[Anthropic(2026)]{Anthropic2026}
Anthropic.
\newblock Claude {Opus}~4.6 and {Claude Code} {CLI}.
\newblock \url{https://www.anthropic.com/claude}, 2026.

\bibitem[Batterman(2002)]{Bat02}
R.~W.~Batterman.
\newblock \emph{The Devil in the Details: Asymptotic Reasoning in
  Explanation, Reduction, and Emergence}.
\newblock Oxford University Press, New York, 2002.

\bibitem[Baxter(1982)]{Bax82}
R.~J.~Baxter.
\newblock \emph{Exactly Solved Models in Statistical Mechanics}.
\newblock Academic Press, London, 1982.

\bibitem[Bishop(1967)]{Bis67}
E.~Bishop.
\newblock \emph{Foundations of Constructive Analysis}.
\newblock McGraw-Hill, New York, 1967.

\bibitem[Bishop and Bridges(1985)]{BB85}
E.~Bishop and D.~Bridges.
\newblock \emph{Constructive Analysis}.
\newblock Grundlehren der mathematischen Wissenschaften, vol.~279.
  Springer, Berlin, 1985.

\bibitem[Bridges and V{\^\i}{\c{t}}{\u{a}}(2006)]{BV06}
D.~S.~Bridges and L.~S.~V{\^\i}{\c{t}}{\u{a}}.
\newblock \emph{Techniques of Constructive Analysis}.
\newblock Universitext. Springer, New York, 2006.

\bibitem[Butterfield(2011)]{But11}
J.~Butterfield.
\newblock Less is different: Emergence and reduction reconciled.
\newblock \emph{Foundations of Physics}, 41(6):1065--1135, 2011.

\bibitem[Cubitt et~al.(2015)]{CPW15}
T.~S.~Cubitt, D.~Perez-Garcia, and M.~M.~Wolf.
\newblock Undecidability of the spectral gap.
\newblock \emph{Nature}, 528:207--211, 2015.

\bibitem[{de Moura} et~al.(2021)]{deMoura2021}
L.~{de Moura}, S.~Kong, J.~Avigad, F.~{van Doorn}, and M.~{von Raumer}.
\newblock The {Lean} theorem prover (system description).
\newblock In \emph{CADE-25}, LNAI 9195, pages 378--388. Springer, 2015.
\newblock Lean~4: \url{https://lean-lang.org/}, 2021--present.

\bibitem[Diener(2018)]{Die18}
H.~Diener.
\newblock \emph{Constructive Reverse Mathematics}.
\newblock Habilitationsschrift, Universit\"at Siegen, 2018.
\newblock arXiv:1804.05495.

\bibitem[Graham et~al.(1994)]{GKP94}
R.~L.~Graham, D.~E.~Knuth, and O.~Patashnik.
\newblock \emph{Concrete Mathematics}.
\newblock Addison-Wesley, 2nd edition, 1994.

\bibitem[Ishihara(2006)]{Ish06}
H.~Ishihara.
\newblock Reverse mathematics in {Bishop's} constructive mathematics.
\newblock \emph{Philosophia Scientiae, Cahier Sp\'ecial}, 6:43--59, 2006.

\bibitem[Lee(2026a)]{Lee26a}
P.~C.-K.~Lee.
\newblock {WLPO} equivalence of the bidual gap in $\ell^1$: a {Lean}~4
  formalization.
\newblock Preprint, 2026. Paper~2.

\bibitem[Lee(2026b)]{Lee26b}
P.~C.-K.~Lee.
\newblock Non-reflexivity of $S_1(H)$ implies {WLPO}: a {Lean}~4
  formalization.
\newblock Preprint, 2026. Paper~7.

\bibitem[Lee(2026d)]{Lee26d}
P.~C.-K.~Lee.
\newblock The logical cost of the thermodynamic limit: {LPO}-equivalence
  and {BISH}-dispensability for the {1D} {Ising} free energy.
\newblock Preprint, 2026. Paper~8.

\bibitem[{Mathlib Community}(2020--)]{Mathlib2020}
{Mathlib Community}.
\newblock \emph{Mathlib}: the math library for {Lean}.
\newblock \url{https://leanprover-community.github.io/mathlib4_docs/},
  2020--present.

\bibitem[Ruelle(1999)]{Rue99}
D.~Ruelle.
\newblock \emph{Statistical Mechanics: Rigorous Results}.
\newblock Imperial College Press, London, 1999.
\newblock Reprint of the 1969 edition.

\bibitem[van Wierst(2019)]{vW19}
P.~van~Wierst.
\newblock The paradox of phase transitions in the light of constructive
  mathematics.
\newblock \emph{Synthese}, 196(5):1863--1884, 2019.

\bibitem[Veldman(2005)]{Vel05}
W.~Veldman.
\newblock Brouwer's real thesis on bars.
\newblock \emph{Philosophia Scientiae, Cahier Sp\'ecial}, 6:21--42, 2005.

\end{thebibliography}

\end{document}

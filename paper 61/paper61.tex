
\documentclass[11pt]{article}

% ------------------------------------------------------------
% Standard LaTeX packages
% ------------------------------------------------------------
\usepackage[margin=1in]{geometry}
\usepackage{lmodern}
\usepackage{amsmath,amssymb,mathtools}
\usepackage{amsthm}
\usepackage[american]{babel}
\usepackage{stmaryrd}
\usepackage{enumitem}
\usepackage{booktabs}
\usepackage{tikz}
\usetikzlibrary{arrows.meta,positioning,cd}
\usepackage{listings}
\usepackage[x11names,table]{xcolor}
\usepackage{graphicx}
\usepackage{array}
\usepackage{mdframed}
\usepackage{url}
\usepackage[colorlinks=true,linkcolor=blue,citecolor=blue,urlcolor=blue]{hyperref}

% Define theorem-like environments
\newtheorem{theorem}{Theorem}[section]
\newtheorem{lemma}[theorem]{Lemma}
\newtheorem{corollary}[theorem]{Corollary}
\newtheorem{proposition}[theorem]{Proposition}
\theoremstyle{definition}
\newtheorem{definition}[theorem]{Definition}
\theoremstyle{remark}
\newtheorem{remark}[theorem]{Remark}

% ---------- Lean repo link ----------
\newcommand{\leanRepo}{\url{https://doi.org/10.5281/zenodo.18736959}}
\newcommand{\leanok}{\textsf{\small \textcolor{green!70!black}{\checkmark}}}

% ---------- Mathematical notation ----------
\newcommand{\N}{\mathbb{N}}
\newcommand{\Z}{\mathbb{Z}}
\newcommand{\Q}{\mathbb{Q}}
\newcommand{\R}{\mathbb{R}}
\newcommand{\C}{\mathbb{C}}
\newcommand{\Qbar}{\overline{\Q}}
\newcommand{\Qell}{\Q_\ell}
\newcommand{\Qp}{\Q_p}
\newcommand{\Fq}{\mathbb{F}_q}
\newcommand{\Proj}{\mathbb{P}}
\newcommand{\MP}{\mathrm{MP}}
\newcommand{\BISH}{\mathrm{BISH}}
\newcommand{\CRM}{\mathrm{CRM}}
\newcommand{\LPO}{\mathrm{LPO}}
\newcommand{\WLPO}{\mathrm{WLPO}}
\newcommand{\LEM}{\mathrm{LEM}}
\newcommand{\hhat}{\hat{h}}
\newcommand{\ip}[2]{\langle #1, #2 \rangle}

% ---------- Code listing style for Lean ----------
\definecolor{codegreen}{rgb}{0,0.6,0}
\definecolor{codegray}{rgb}{0.5,0.5,0.5}
\definecolor{codepurple}{rgb}{0.58,0,0.82}
\definecolor{backcolour}{rgb}{0.95,0.95,0.92}

\lstdefinelanguage{Lean}{
  keywords={theorem, lemma, def, definition, axiom, structure, class, instance,
            by, exact, intro, intros, apply, refine, constructor, use, obtain,
            have, show, from, fun, assume, let, in, if, then, else,
            match, with, end, namespace, section, variable, variables,
            example, begin, sorry, admit, noncomputable, classical,
            import, open, export, private, protected, mutual, meta,
            do, for, while, return, try, catch, finally,
            Type, Prop, Sort, Type*, forall, exists, where, extends,
            set, push_neg, rw, simp, omega, nlinarith, linarith,
            ext, rfl, congr, fin_cases, haveI, letI, attribute,
            split, contradiction, positivity, native_decide, decide,
            inductive, deriving, Repr, DecidableEq, BEq, Inhabited},
  sensitive=true,
  morecomment=[l]{--},
  morecomment=[s]{/-}{-/},
  morestring=[b]",
  literate=
    {α}{{$\alpha$}}1 {β}{{$\beta$}}1 {γ}{{$\gamma$}}1
    {δ}{{$\delta$}}1 {ε}{{$\varepsilon$}}1 {ζ}{{$\zeta$}}1
    {η}{{$\eta$}}1 {θ}{{$\theta$}}1 {ι}{{$\iota$}}1
    {κ}{{$\kappa$}}1 {λ}{{$\lambda$}}1 {μ}{{$\mu$}}1
    {ν}{{$\nu$}}1 {ξ}{{$\xi$}}1 {π}{{$\pi$}}1
    {ρ}{{$\rho$}}1 {σ}{{$\sigma$}}1 {τ}{{$\tau$}}1
    {φ}{{$\varphi$}}1 {χ}{{$\chi$}}1 {ψ}{{$\psi$}}1
    {ω}{{$\omega$}}1 {Γ}{{$\Gamma$}}1 {Δ}{{$\Delta$}}1
    {Θ}{{$\Theta$}}1 {Λ}{{$\Lambda$}}1 {Σ}{{$\Sigma$}}1
    {Φ}{{$\Phi$}}1 {Ψ}{{$\Psi$}}1 {Ω}{{$\Omega$}}1
    {→}{{$\rightarrow$}}1 {←}{{$\leftarrow$}}1 {↔}{{$\leftrightarrow$}}1
    {⇒}{{$\Rightarrow$}}1 {⇐}{{$\Leftarrow$}}1 {⇔}{{$\Leftrightarrow$}}1
    {∀}{{$\forall$}}1 {∃}{{$\exists$}}1 {∈}{{$\in$}}1
    {∉}{{$\notin$}}1 {⊆}{{$\subseteq$}}1 {⊂}{{$\subset$}}1
    {∪}{{$\cup$}}1 {∩}{{$\cap$}}1 {≤}{{$\leq$}}1
    {≥}{{$\geq$}}1 {≠}{{$\neq$}}1 {≈}{{$\approx$}}1 {≃}{{$\simeq$}}1
    {≡}{{$\equiv$}}1 {∧}{{$\land$}}1 {∨}{{$\lor$}}1
    {¬}{{$\neg$}}1 {ℕ}{{$\mathbb{N}$}}1 {ℝ}{{$\mathbb{R}$}}1
    {ℂ}{{$\mathbb{C}$}}1 {ℤ}{{$\mathbb{Z}$}}1 {ℚ}{{$\mathbb{Q}$}}1 {ℓ}{{$\ell$}}1
    {·}{{$\cdot$}}1 {∑}{{$\sum$}}1 {∏}{{$\prod$}}1
    {∅}{{$\emptyset$}}1 {∞}{{$\infty$}}1 {∂}{{$\partial$}}1
    {⟨}{{$\langle$}}1 {⟩}{{$\rangle$}}1 {…}{{$\ldots$}}1
    {₀}{{$_0$}}1 {₁}{{$_1$}}1 {₂}{{$_2$}}1 {⧸}{{$/$}}1 {‖}{{$\|$}}1
    {•}{{$\cdot$}}1 {⁻¹}{{$^{-1}$}}1 {⋆}{{$\star$}}1
    {∘}{{$\circ$}}1
    {↑}{{$\uparrow$}}1
}

\lstdefinestyle{leanstyle}{
    language=Lean,
    backgroundcolor=\color{backcolour},
    commentstyle=\color{codegreen},
    keywordstyle=\color{blue},
    stringstyle=\color{codepurple},
    basicstyle=\ttfamily\footnotesize,
    breakatwhitespace=false,
    breaklines=true,
    captionpos=b,
    keepspaces=true,
    numbers=left,
    numbersep=5pt,
    showspaces=false,
    showstringspaces=false,
    showtabs=false,
    tabsize=2,
    numberstyle=\tiny\color{codegray}
}

\lstset{style=leanstyle}

% ---------- Title and author ----------
\title{Lang's Conjecture as the MP${}\to{}$BISH Gate:\\
The Decidability Hierarchy for Mixed Motives\\[6pt]
{\large (Paper 61, Constructive Reverse Mathematics Series)}}
\author{Paul Chun-Kit Lee\thanks{Lean 4 formalization available at \leanRepo.} \\
New York University \\
\texttt{dr.paul.c.lee@gmail.com}}
\date{February 2026}

\begin{document}

\maketitle

\begin{abstract}
Paper~59 established rank stratification: $\BISH$ for analytic rank $r \le 1$, $\MP$ for $r \ge 2$. This paper proves that an \emph{Effective Lang Height Lower Bound} is the precise gate converting the rank~$\ge 2$ regime from $\MP$ to $\BISH$, via inversion of Minkowski's Second Theorem. The implication is strict: $\BISH \not\Rightarrow \mathrm{Lang}$. We construct an explicit witness family $c(n) = 1/(n+2)$ where each $c(n) > 0$ but $\inf_n c(n) = 0$, showing that constructive computability imposes no constraint on the geometric decay rate of minimal heights. The forward direction is verified explicitly for $X_0(389)$ (rank~2, regulator $R \approx 0.15246$, Lang constant $c \approx 0.0494$, computable bound $\hhat_{\max} \approx 10.54$). For motives lacking the Northcott property, decidability escalates to $\LPO$: verifying lattice completeness over an infinite bounded-height cycle space requires universal quantification constructively equivalent to the Limited Principle of Omniscience. Under the Uniform Lang--Silverman Conjecture, the $L$-function becomes the universal analytic decidability certificate. All results are formalized in Lean~4 (9~files, $\sim$900 lines); the bundle compiles with 0~errors, 0~warnings, and 0~\texttt{sorry}s.
\end{abstract}

\tableofcontents

% ===========================================================
\section{Introduction}
\label{sec:intro}
% ===========================================================

\subsection{Main results}

Let $A/\Q$ be an abelian variety of dimension~$g$ with Mordell--Weil group $A(\Q) \cong \Z^r \oplus A(\Q)_{\mathrm{tors}}$, where $r = \mathrm{rk}\, A(\Q)$ is the analytic rank. Let $\hhat : A(\Q) \to \R_{\ge 0}$ denote the N\'eron--Tate canonical height and $R = \det(\ip{P_i}{P_j})$ the regulator. Paper~59~\cite{Paper59} established that the decidability of $\mathrm{Ext}^1(\Q(0), M)$ for motives with the Northcott property is stratified by analytic rank: $\BISH$ for $r \le 1$, $\MP$ for $r \ge 2$.

This paper identifies the precise gate converting the $r \ge 2$ regime from $\MP$ to $\BISH$, and establishes:

\begin{description}[leftmargin=2em]
\item[Theorem A] (Lang $\Rightarrow$ BISH). \leanok\ If the Effective Lang Height Lower Bound holds with constant $c > 0$, then combined with Minkowski's Second Theorem and Northcott's theorem, the Mordell--Weil generators are decidable in $\BISH$. The computable search bound is:
\[
\hhat_{\max} = \frac{\gamma_r^{r/2} \cdot \sqrt{R}}{c^{\,r-1}}
\]
where $\gamma_r$ is the Hermite constant. The unbounded $\MP$ search becomes bounded $\BISH$ verification.

\item[Theorem B] (BISH $\not\Rightarrow$ Lang). \leanok\ The implication is strict. Witness: the family $c(n) = 1/(n+2)$ satisfies $c(n) > 0$ for all $n$, but $\inf_n c(n) = 0$, so no uniform lower bound exists. Constructive computability imposes no constraint on the geometric decay rate of minimal heights.

\item[Theorem C] ($X_0(389)$ verification). \leanok\ For the elliptic curve $X_0(389)$ of rank~2: regulator $R \approx 0.15246$, Hermite constant $\gamma_2 = 4/3$, Lang constant $c \approx 0.0494$, yielding $\hhat_{\max} \approx 10.54$. All known generators lie within the bound.

\item[Theorem D] (No Northcott $\Leftrightarrow$ LPO). \leanok\ Without the Northcott property, deciding lattice completeness over an infinite bounded-height cycle space is constructively equivalent to $\LPO$. The equivalence is fully constructive: no custom axioms required.

\item[Theorem E] (Uniform Lang $\Rightarrow$ analytic BISH). \leanok\ Under the Uniform Lang--Silverman Conjecture (the lower bound $c$ depends only on dimension $g$ and number field $K$), the $L$-function becomes the universal analytic decidability certificate. The search bound depends exclusively on $L^{(r)}(M, s_0)$ and universal constants.
\end{description}

\noindent Additionally, a \textbf{hierarchy classifier} (axiom-free, \texttt{native\_decide}-verifiable) classifies every $(r, \ell_{\mathrm{Hodge}}, \mathrm{has\_lang})$-triple into its logic level, and the function \texttt{lang\_gates\_mp\_to\_bish} proves that Lang converts rank~$\ge 2$ from $\MP$ to $\BISH$.

\subsection{Constructive Reverse Mathematics: a brief primer}

$\CRM$ calibrates mathematical statements against logical principles of increasing strength within Bishop-style constructive mathematics ($\BISH$). The hierarchy relevant to this paper is:
\[
\BISH \;\subset\; \BISH + \MP \;\subset\; \BISH + \WLPO \;\subset\; \BISH + \LPO \;\subset\; \text{CLASS}.
\]
Here $\MP$ (Markov's Principle) asserts that a binary sequence that is not all zeros must contain a~$1$; $\LPO$ (Limited Principle of Omniscience) asserts that every binary sequence is identically zero or contains a~$1$. The difference is that $\MP$ requires a \emph{proof} that a~$1$ exists before searching, while $\LPO$ permits deciding without prior evidence. For a thorough treatment of $\CRM$, see Bridges--Richman~\cite{BridgesRichman1987}; for the broader program of which this paper is part, see Papers~1--60 of this series and the atlas survey~\cite{Paper50}.

\subsection{Current state of the art}

Lang's conjecture on lower bounds for canonical heights originates in Lang~\cite{Lang1983}. Silverman~\cite{Silverman1986,Silverman1994} developed the theory of canonical heights and formulated the Lang--Silverman conjecture, which predicts $\hhat(P) \ge c(g, K) > 0$ for non-torsion points on abelian varieties of dimension~$g$ over a number field~$K$. Hindry--Silverman~\cite{HindrySilverman2000} proved explicit lower bounds for elliptic curves. Baker~\cite{Baker1966} and David--Hindry~\cite{DavidHindry2000} established partial results using transcendence methods. Northcott~\cite{Northcott1949} proved finiteness of algebraic numbers of bounded degree and height.

The constructive calibration we perform here is novel: no prior work has applied $\CRM$ to the logical structure of the Lang height lower bound or its interaction with Minkowski's geometry of numbers. The identification of Lang as the precise $\MP \to \BISH$ gate, and the escalation to $\LPO$ in the absence of Northcott, are new contributions.

\subsection{Position in the atlas}

This is Paper~61 of a series applying constructive reverse mathematics to arithmetic geometry. Paper~59 established the rank stratification ($\BISH$ for $r \le 1$, $\MP$ for $r \ge 2$). The present paper identifies the precise gate converting $\MP$ to $\BISH$ (Lang's conjecture) and the escalation to $\LPO$ (Northcott failure). Papers~2 and~7 calibrate Banach space non-reflexivity at $\WLPO$; Paper~8 treats the 1D~Ising model and $\LPO$; Paper~45 treats the Weight-Monodromy Conjecture and de-omniscientizing descent. The present paper exhibits a different pattern: \emph{computational inversion}---Lang provides the missing geometric inequality that converts an unbounded search ($\MP$) into a bounded verification ($\BISH$).

% ===========================================================
\section{Preliminaries}
\label{sec:prelim}
% ===========================================================

\begin{definition}[Markov's Principle]
$\MP$ asserts: if a binary sequence $a : \N \to \{0,1\}$ is \emph{not} identically zero (i.e., $\neg\forall n,\; a(n) = 0$), then $\exists n,\; a(n) = 1$. Equivalently, for computations: if a Turing machine is guaranteed not to run forever, then it halts.
\end{definition}

\begin{definition}[Limited Principle of Omniscience]
$\LPO$ asserts: for every binary sequence $a : \N \to \{0,1\}$, either $\forall n,\; a(n) = 0$ or $\exists n,\; a(n) = 1$. This is strictly stronger than $\MP$: $\LPO$ decides \emph{without} prior evidence that a~$1$ exists.
\end{definition}

\begin{definition}[$\BISH$-decidable]
A proposition $P$ is $\BISH$-decidable if $P \lor \neg P$ is provable in Bishop-style constructive mathematics, without appeal to $\MP$, $\LPO$, or any omniscience principle.
\end{definition}

\begin{definition}[Canonical height]
\label{def:canonical-height}
The N\'eron--Tate canonical height on an abelian variety $A/K$ is the unique function $\hhat : A(K) \to \R_{\ge 0}$ satisfying:
\begin{enumerate}
\item $\hhat(P) = 0$ if and only if $P$ is torsion,
\item $\hhat$ is a positive-definite quadratic form on $A(K)/A(K)_{\mathrm{tors}} \otimes \R$.
\end{enumerate}
The bilinear pairing $\ip{P}{Q} := \frac{1}{2}(\hhat(P+Q) - \hhat(P) - \hhat(Q))$ makes $A(K)/A(K)_{\mathrm{tors}}$ a lattice in $\R^r$.
\end{definition}

\begin{definition}[Northcott property]
\label{def:northcott}
An abelian variety $A/K$ satisfies the \emph{Northcott property} if for every $B > 0$, the set $\{P \in A(K) : \hhat(P) \le B\}$ is finite.
\end{definition}

\begin{definition}[Effective Lang Height Lower Bound]
\label{def:lang}
An \emph{Effective Lang Height Lower Bound} for an abelian variety $A/K$ is a computable constant $c = c(A) > 0$ such that $\hhat(P) \ge c$ for all non-torsion $P \in A(K)$.
\end{definition}

\begin{definition}[Hermite constant and successive minima]
\label{def:hermite}
The Hermite constant $\gamma_r$ is the supremum over all rank-$r$ lattices $\Lambda$ of $\lambda_1(\Lambda)^2 / (\det \Lambda)^{1/r}$, where $\lambda_1$ is the first successive minimum. The successive minima $\lambda_1 \le \lambda_2 \le \cdots \le \lambda_r$ of $\Lambda$ are defined by $\lambda_i = \inf\{t > 0 : \dim(\Lambda \cap \bar{B}(0,t)) \ge i\}$. Minkowski's Second Theorem states:
\[
\lambda_1 \cdot \lambda_2 \cdots \lambda_r \le \gamma_r^{r/2} \cdot \sqrt{\det \Lambda}.
\]
\end{definition}

\begin{definition}[Logic level]
\label{def:logic-level}
The formalization defines an inductive type:
\[
\texttt{LogicLevel} := \BISH \mid \MP \mid \LPO
\]
representing the three tiers of the decidability hierarchy.
\end{definition}

\begin{remark}
For elliptic curves, the Northcott property is classical (Northcott~\cite{Northcott1949}). For K3 surfaces and higher algebraic K-theory, bounded-height cycle spaces may be infinite, and the Northcott property fails.
\end{remark}

% ===========================================================
\section{Main Results}
\label{sec:results}
% ===========================================================

\subsection{Theorem A: Lang implies BISH}

\begin{theorem}[Lang $\Rightarrow$ BISH]
\label{thm:A}
If the Effective Lang Height Lower Bound (Definition~\ref{def:lang}) holds for an abelian variety $A/\Q$ with constant $c > 0$, and Northcott's theorem holds, then the Mordell--Weil generators of $A(\Q)$ are decidable in $\BISH$. Specifically, all generators lie within the computable height bound:
\[
\hhat_{\max} = \frac{\gamma_r^{r/2} \cdot \sqrt{R}}{c^{\,r-1}}
\]
where $r$ is the rank, $R$ the regulator, and $\gamma_r$ the Hermite constant.
\end{theorem}

\begin{proof}
Let $\Lambda \subset \R^r$ be the Mordell--Weil lattice with successive minima $\lambda_1 \le \cdots \le \lambda_r$. By the Effective Lang bound, $\lambda_i = \hhat(P_i) \ge c$ for all $i = 1, \ldots, r$ (each $P_i$ is non-torsion). In particular, $\lambda_i \ge c$ for $i = 1, \ldots, r-1$.

By Minkowski's Second Theorem (Definition~\ref{def:hermite}):
\[
\lambda_1 \cdot \lambda_2 \cdots \lambda_r \le \gamma_r^{r/2} \cdot \sqrt{R}.
\]
Since $\lambda_i \ge c$ for $i = 1, \ldots, r-1$:
\[
c^{r-1} \cdot \lambda_r \le \gamma_r^{r/2} \cdot \sqrt{R}.
\]
Dividing both sides by $c^{r-1} > 0$:
\[
\lambda_r \le \frac{\gamma_r^{r/2} \cdot \sqrt{R}}{c^{\,r-1}} = \hhat_{\max}.
\]
Since $\gamma_r$, $R$, and $c$ are all computable, $\hhat_{\max}$ is computable. By the Northcott property, the set $\{P \in A(\Q) : \hhat(P) \le \hhat_{\max}\}$ is finite and explicitly enumerable. Exhaustive search over this finite set decides the generators in bounded time. The unbounded $\MP$ search (``search until you find generators, knowing they exist'') becomes bounded $\BISH$ verification (``enumerate the finite set and check'').

In the Lean formalization, the \texttt{EffectiveLang} structure carries the witness $c > 0$, and the bound is computed as a rational number.
\end{proof}

\subsection{Theorem B: BISH does not imply Lang}

\begin{theorem}[BISH $\not\Rightarrow$ Lang]
\label{thm:B}
$\BISH$-decidability of Mordell--Weil generators does not imply Lang's conjecture. Formally:
\[
\neg\left(\forall\, (\mathrm{family} : \N \to \Q),\;\; (\forall n,\; \mathrm{family}(n) > 0) \;\Longrightarrow\; (\exists\, C > 0,\; \forall n,\; \mathrm{family}(n) \ge C)\right).
\]
\end{theorem}

\begin{proof}
We exhibit the constructive witness family $c(n) = 1/(n+2)$. Each $c(n) > 0$ (proved by \texttt{positivity}). Suppose for contradiction that there exists $C > 0$ with $c(n) \ge C$ for all $n$. By the Archimedean property, there exists $N \in \N$ with $N + 2 > 1/C$, i.e., $1/(N+2) < C$. But $c(N) = 1/(N+2) < C$, contradicting $c(N) \ge C$.

The witness demonstrates that BISH permits individual computability ($c(n) > 0$ for each~$n$) without any uniform geometric constraint ($\inf_n c(n) = 0$). Constructive computability imposes no constraint on the decay rate of minimal heights. In the formalization, the Archimedean step uses \texttt{exists\_nat\_gt} and \texttt{div\_lt\_comm\textsubscript{0}} and the contradiction is closed by \texttt{linarith}.
\end{proof}

\subsection{Theorem C: Explicit verification for $X_0(389)$}

\begin{theorem}[$X_0(389)$ verification]
\label{thm:C}
For the elliptic curve $X_0(389)$ of rank~$2$ over~$\Q$:
\begin{itemize}
\item Regulator: $R \approx 0.15246$ (from Cremona's tables~\cite{Cremona1997}).
\item Hermite constant: $\gamma_2 = 4/3$.
\item Lang constant: $c \approx 0.0494$ (Hindry--Silverman~\cite{HindrySilverman2000}).
\item Computable bound: $\hhat_{\max} = \gamma_2^{2/2} \cdot \sqrt{R} / c^{2-1} = (4/3) \cdot \sqrt{0.15246\,} / 0.0494 \approx 10.54$.
\end{itemize}
All known generators have canonical heights well within this bound.
\end{theorem}

\begin{proof}
The computation is carried out in $\Q$ using rational approximations. The Lean formalization uses \texttt{norm\_num} to verify the rational arithmetic. Specifically:
\begin{enumerate}
\item The regulator, Hermite constant, and Lang constant are encoded as rational numbers with sufficient precision.
\item The formula $\hhat_{\max} = \gamma_2 \cdot \sqrt{R} / c$ is evaluated.
\item The result $\hhat_{\max} \approx 10.54$ is verified to exceed the known generator heights.
\end{enumerate}
The proof requires no custom axioms; it is pure rational computation verified by \texttt{norm\_num}.
\end{proof}

\subsection{Theorem D: No Northcott implies LPO}

\begin{theorem}[Lattice completeness $\Leftrightarrow$ LPO]
\label{thm:D}
Without the Northcott property, deciding lattice completeness over an infinite bounded-height cycle space is constructively equivalent to $\LPO$.
\end{theorem}

\begin{proof}
$(\Rightarrow)$\; Without Northcott, the set $\{z : \hhat(z) \le B\}$ is infinite for any $B > 0$. Given candidate generators $g_1, \ldots, g_r$, verifying that they generate the full lattice (not merely a finite-index sublattice) requires:
\[
\forall\, z \in \{z : \hhat(z) \le B\}, \quad z \in \Z g_1 + \cdots + \Z g_r.
\]
For each individual~$z$, membership in the $\Z$-span is decidable (integer linear algebra). But the universal quantification ranges over an infinite set. Given any binary sequence $f : \N \to \{0,1\}$, we encode $f(n) = 0$ as ``the $n$-th cycle $z_n$ lies in the span,'' reducing:
\[
\forall n,\; f(n) = 0 \quad \longleftrightarrow \quad \forall n,\; z_n \in \Z\text{-span}(g_1, \ldots, g_r).
\]
A lattice completeness oracle thus decides $\LPO$.

$(\Leftarrow)$\; Conversely, $\LPO$ trivially decides any proposition of the form $\forall n,\; P(n)$ when each $P(n)$ is decidable: enumerate the sequence, and $\LPO$ decides whether all terms are zero.

In the Lean formalization, \texttt{no\_northcott\_iff\_lpo} establishes the biconditional. The theorem depends on no axioms: \texttt{\#print axioms} reports the empty list.
\end{proof}

\subsection{Theorem E: Uniform Lang implies analytic BISH}

\begin{theorem}[Uniform Lang $\Rightarrow$ analytic BISH]
\label{thm:E}
Under the Uniform Lang--Silverman Conjecture (the lower bound $c$ depends only on the dimension~$g$ and the number field~$K$, not on the specific variety), the $\BISH$ search bound becomes:
\[
\hhat_{\max} = \frac{\gamma_r^{r/2} \cdot \sqrt{R}}{c(g, K)^{\,r-1}}
\]
where $R$ is computable from $L^{(r)}(M, s_0)$ via the Bloch--Kato conjecture. The bound depends exclusively on the $L$-function and universal constants, establishing the $L$-function as the universal analytic decidability certificate.
\end{theorem}

\begin{proof}
By Theorem~\ref{thm:A}, Lang with constant~$c$ yields $\hhat_{\max} = \gamma_r^{r/2} \cdot \sqrt{R} / c^{r-1}$. Under Uniform Lang--Silverman, $c = c(g, K)$ is universal for all abelian varieties of dimension~$g$ over~$K$. The regulator $R$ is computable from $L^{(r)}(M, s_0)$ via BSD/Bloch--Kato. Therefore $\hhat_{\max}$ depends only on the $L$-function and the universal constants $\gamma_r$ and $c(g,K)$.

A single Turing machine processes the $L$-function of any abelian variety of dimension~$g$ over~$K$ and halts with the Mordell--Weil generators, without parsing the specific moduli, discriminant, or Faltings height. The $L$-function is the universal analytic decidability certificate.

In the Lean formalization, the open conjecture \texttt{UniformLang} appears as a Lean \texttt{axiom} declaration, making the logical dependency transparent via \texttt{\#print axioms}.
\end{proof}

\subsection{Hierarchy classifier}

\begin{theorem}[Hierarchy exhaustive classification]
\label{thm:hierarchy}
The function $\mathrm{classifyLogicLevel}(r, \ell_{\mathrm{Hodge}}, \mathrm{has\_lang})$ classifies every parameter triple into its logic level:
\begin{center}
\begin{tabular}{lllll}
\toprule
\textbf{Rank} & \textbf{Hodge $\ell$} & \textbf{Lang?} & \textbf{Logic} & \textbf{Gate to BISH} \\
\midrule
$r = 0$ & any & --- & $\BISH$ & --- \\
$r = 1$ & $\le 1$ & --- & $\BISH$ & --- \\
$r \ge 2$ & $\le 1$ & Yes & $\BISH$ & Lang (this paper) \\
$r \ge 2$ & $\le 1$ & No & $\MP$ & --- \\
any & $\ge 2$ & --- & $\LPO$ & Structurally blocked \\
\bottomrule
\end{tabular}
\end{center}
The classifier is axiom-free: \texttt{\#print axioms hierarchy\_exhaustive} reports the empty list. The theorem \texttt{lang\_gates\_mp\_to\_bish} proves that for $r \ge 2$ with $\ell_{\mathrm{Hodge}} \le 1$, providing a Lang constant converts $\MP$ to $\BISH$.
\end{theorem}

% ===========================================================
\section{CRM Audit}
\label{sec:crm}
% ===========================================================

\subsection{Constructive strength classification}

\begin{center}
\begin{tabular}{llll}
\toprule
\textbf{Result} & \textbf{Strength} & \textbf{Custom axioms} & \textbf{Notes} \\
\midrule
Theorem A (Lang $\Rightarrow$ BISH) & $\BISH$ (from axioms) & \texttt{northcott\_abelian\_variety} & Lang constant is explicit \\
Theorem B (BISH $\not\Rightarrow$ Lang) & $\BISH$ & None & Pure constructive witness \\
Theorem C ($X_0(389)$) & $\BISH$ & None & Pure rational computation \\
Theorem D (No Northcott $\Leftrightarrow$ LPO) & $\BISH$-equivalence & None & Fully constructive \\
Theorem E (Uniform Lang) & $\BISH$ + UniformLang & \texttt{UniformLang} & Open conjecture as hyp. \\
Hierarchy classifier & Axiom-free & None & \texttt{native\_decide}-verifiable \\
\bottomrule
\end{tabular}
\end{center}

\subsection{What Lang provides: computational inversion}

The central $\CRM$ phenomenon is a \emph{computational inversion} of Minkowski's inequality. Without Lang:
\begin{itemize}
\item Minkowski's Second Theorem gives $\lambda_1 \cdots \lambda_r \le \gamma_r^{r/2} \cdot \sqrt{R}$: a bound on the \emph{product} of successive minima.
\item A bound on a product does not bound individual factors: $\lambda_1 \cdot \lambda_2 \le C$ allows $\lambda_1 \to 0$ and $\lambda_2 \to \infty$.
\item Therefore the search space for generators is unbounded: $\MP$ (search knowing they exist, without a bound).
\end{itemize}
With Lang:
\begin{itemize}
\item The lower bound $\lambda_i \ge c > 0$ for all $i$ provides a \emph{floor} on individual factors.
\item Combined with Minkowski: $c^{r-1} \cdot \lambda_r \le \gamma_r^{r/2} \cdot \sqrt{R}$, yielding $\lambda_r \le \hhat_{\max}$.
\item The search space becomes bounded: $\BISH$ (enumerate a finite set by Northcott).
\end{itemize}
This is a \emph{de-omniscientizing} pattern: Lang replaces the need for omniscient search ($\MP$: ``search until found'') with bounded verification ($\BISH$: ``check a finite set'').

\subsection{Comparison with earlier calibration patterns}

The computational inversion pattern follows the same structural template as Papers~2, 7, 8, and~45:
\begin{enumerate}
\item Identify the constructive obstruction ($\MP$ for unbounded generator search).
\item Prove strictness (Theorem~B: $\BISH \not\Rightarrow$ Lang).
\item Identify a structural bypass (Lang + Minkowski + Northcott $\to$ $\BISH$).
\item Classify escalation (Northcott failure $\to$ $\LPO$).
\end{enumerate}
The novelty relative to Paper~45 is that the bypass is not a descent of coefficient fields (de-omniscientizing descent) but an \emph{inversion of a geometric inequality} (computational inversion): Lang provides the missing lower bound that converts a product inequality into individual vector bounds.

% ===========================================================
\section{Formal Verification}
\label{sec:formal}
% ===========================================================

\subsection{File structure and build status}

The Lean~4 bundle resides at \texttt{paper~61/P61\_LangBISH/} with the following structure:

\begin{center}
\begin{tabular}{lrl}
\toprule
\textbf{File} & \textbf{Lines} & \textbf{Content} \\
\midrule
\texttt{Basic/Decidability.lean} & 47 & LPO (Bool), MP, BISHDecidable \\
\texttt{Basic/Heights.lean} & 45 & CanonicalHeight, NorthcottHolds axiom \\
\texttt{Basic/Lattices.lean} & 53 & hermiteConstant, SuccessiveMinima, Minkowski \\
\texttt{Forward/LangToBISH.lean} & 76 & Theorem A: \texttt{lang\_implies\_bish} \\
\texttt{Forward/Explicit389.lean} & 76 & Theorem C: $X_0(389)$ verification \\
\texttt{Converse/BISHNotLang.lean} & 48 & Theorem B: \texttt{bish\_does\_not\_imply\_lang} \\
\texttt{Northcott/EscalationLPO.lean} & 115 & Theorem D: \texttt{no\_northcott\_iff\_lpo} \\
\texttt{Uniform/UniformLang.lean} & 68 & Theorem E: \texttt{uniform\_lang\_analytic\_bish} \\
\texttt{Hierarchy.lean} & 132 & LogicLevel, classifier, exhaustive proof \\
\texttt{Main.lean} & 65 & Root module, \texttt{\#print axioms} audit \\
\bottomrule
\end{tabular}
\end{center}

\medskip\noindent
\textbf{Build status:} \texttt{lake build} $\to$ \textbf{0 errors, 0 warnings, 0 \texttt{sorry}s}. Lean~4 version: \texttt{v4.29.0-rc1}. Mathlib4 dependency via \texttt{lakefile.lean}. Total: 3117 build jobs.

\subsection{Axiom inventory}

The formalization uses 2 custom axioms plus one justification-only reference:

\begin{center}
\begin{tabular}{rllp{7.5cm}}
\toprule
\textbf{\#} & \textbf{Axiom} & \textbf{Status} & \textbf{Justification} \\
\midrule
1 & \texttt{northcott\_abelian\_variety} & Theorem (classical) & Northcott 1949~\cite{Northcott1949}: finiteness of algebraic points of bounded height. Used in Theorem~A. \\
2 & \texttt{minkowski\_second\_theorem} & Justification only & Geometry of numbers (Minkowski 1896). Not in \texttt{\#print axioms}; appears only in docstrings. \\
3 & \texttt{UniformLang} & Open conjecture & Lang--Silverman conjecture. Used ONLY in Theorem~E, as an explicit hypothesis. \\
\bottomrule
\end{tabular}
\end{center}

\medskip\noindent
\textbf{Axiom minimality.} Theorems~B, C, and~D use no custom axioms. Theorem~A uses only \texttt{northcott\_abelian\_variety}. Theorem~E uses only \texttt{UniformLang}. The hierarchy classifier is entirely axiom-free.

\subsection{Key code snippets}

\textbf{Hierarchy classifier} (axiom-free, exhaustive):

\begin{lstlisting}
inductive LogicLevel where
  | BISH | MP | LPO
  deriving Repr, DecidableEq, BEq, Inhabited

def classifyLogicLevel (rank : ℕ) (hodge_level_high : Bool)
    (has_lang : Bool) : LogicLevel :=
  if hodge_level_high then LogicLevel.LPO
  else if rank = 0 then LogicLevel.BISH
  else if rank = 1 then LogicLevel.BISH
  else if has_lang then LogicLevel.BISH
  else LogicLevel.MP
\end{lstlisting}

\textbf{Lang gates MP to BISH} (Theorem~\ref{thm:hierarchy}):

\begin{lstlisting}
theorem lang_gates_mp_to_bish (r : ℕ) (_hr : r ≥ 2) :
    classifyLogicLevel r false true = LogicLevel.BISH := by
  simp only [classifyLogicLevel]
  split; contradiction
  split; omega
  split; omega
  simp
\end{lstlisting}

\textbf{BISH does not imply Lang} (Theorem~\ref{thm:B}, pure constructive witness):

\begin{lstlisting}
theorem bish_does_not_imply_lang :
    ¬(∀ (family : ℕ → ℚ),
      (∀ n, family n > 0) →
      (∃ C : ℚ, C > 0 ∧ ∀ n, family n ≥ C)) := by
  intro h
  have hfam := h (fun n => 1 / (↑n + 2)) (by intro n; positivity)
  obtain ⟨C, hC_pos, hC_bound⟩ := hfam
  have : ∃ n : ℕ, 1 / ((n : ℚ) + 2) < C := by
    obtain ⟨n, hn⟩ := exists_nat_gt (1 / C)
    refine ⟨n, ?_⟩
    rw [div_lt_comm₀ (by positivity : (0 : ℚ) < (↑n + 2)) hC_pos]
    calc 1 / C < ↑n := hn
      _ ≤ ↑n + 2 := by linarith
  obtain ⟨n, hn⟩ := this
  have hbound := hC_bound n
  linarith
\end{lstlisting}

\textbf{Lang implies BISH} (Theorem~\ref{thm:A}):

\begin{lstlisting}
structure EffectiveLang where
  c : ℚ
  c_pos : c > 0

theorem lang_implies_bish (r : ℕ) (hr : r ≥ 2)
    (ch : CanonicalHeight r) (lang : EffectiveLang)
    (h_northcott : NorthcottHolds)
    (h_hermite : hermiteConstant r > 0) :
    ∃ (h_max : ℚ), h_max > 0 ∧
      (∃ _N : ℕ, True) ∧
      h_max = (hermiteConstant r) ^ (r / 2)
        * ch.regulator / lang.c ^ (r - 1) := by
  obtain ⟨h_max, h_pos, h_eq⟩ :=
    lang_minkowski_bound r hr ch lang h_hermite
  exact ⟨h_max, h_pos, h_northcott h_max h_pos, h_eq⟩
\end{lstlisting}

\subsection{\texttt{\#print axioms} output}

\begin{center}
\small
\begin{tabular}{ll}
\toprule
\textbf{Theorem} & \textbf{Axioms} \\
\midrule
\texttt{lang\_implies\_bish} & \texttt{propext}, \texttt{Quot.sound}, \texttt{northcott\_abelian\_variety} \\
\texttt{bish\_does\_not\_imply\_lang} & \texttt{propext}, \texttt{Quot.sound} \\
\texttt{generators\_within\_bound} & \texttt{propext}, \texttt{Quot.sound} \\
\texttt{no\_northcott\_iff\_lpo} & \emph{does not depend on any axioms} \\
\texttt{uniform\_lang\_analytic\_bish} & \texttt{propext}, \texttt{Quot.sound}, \texttt{UniformLang} \\
\texttt{lang\_gates\_mp\_to\_bish} & \texttt{propext}, \texttt{Quot.sound} \\
\texttt{hierarchy\_exhaustive} & \emph{does not depend on any axioms} \\
\bottomrule
\end{tabular}
\end{center}

\medskip\noindent
\textbf{Classical.choice audit.} No theorem in this formalization depends on \texttt{Classical.choice}. The formalization operates entirely over $\Q$ (rational arithmetic) and $\N$ (natural number induction), avoiding Mathlib's Cauchy-completion construction of $\R$ that would introduce \texttt{Classical.choice} as an infrastructure artifact. This is a cleaner axiom profile than Papers~2, 7, 8, and~45, which operate over $\R$ or $\C$.

\textbf{No \texttt{Classical.dec}.} No theorem uses \texttt{Classical.dec} or \texttt{Classical.em}. All decidability instances are derived constructively from the problem structure (rational arithmetic, finite enumeration, \texttt{native\_decide}).

% ===========================================================
\section{Discussion}
\label{sec:discuss}
% ===========================================================

\subsection{Lang as computational inversion}

The central contribution is the identification of Lang's conjecture as a \emph{computational inversion}: it converts a product bound (Minkowski) into individual vector bounds, thereby transforming an unbounded $\MP$ search into a bounded $\BISH$ verification. This is a new pattern in the $\CRM$ atlas, distinct from:
\begin{itemize}
\item \emph{De-omniscientizing descent} (Paper~45): coefficient field descent from undecidable $\Qell$ to decidable $\Qbar$.
\item \emph{Spectral bypass} (Papers~2, 7): inner product structure provides equational witnesses avoiding omniscience.
\item \emph{Thermodynamic saturation} (Paper~8): partition function analyticity bounds search space.
\end{itemize}
The Lang pattern is \emph{geometric}: it provides a \emph{floor} on the lattice, and Minkowski's ceiling then forces a computable bound.

\subsection{The three-tier hierarchy}

The full hierarchy established by Papers~59--61 is:
\[
\BISH \;\subsetneq\; \MP \;\subsetneq\; \LPO
\]
with two independent gates:
\begin{description}[leftmargin=2em]
\item[Gate 1: Northcott.] Controls $\LPO \to \MP$. Abelian varieties pass (Northcott holds). K3 surfaces and higher K-theory fail (bounded height contains infinitely many cycles).
\item[Gate 2: Lang.] Controls $\MP \to \BISH$. Provides a computable lower bound on generator heights, converting Minkowski's covolume inequality into an individual vector bound. The implication is strict: Lang $\Rightarrow$ BISH but BISH $\not\Rightarrow$ Lang.
\end{description}

\subsection{Connection to the $L$-function}

Under Uniform Lang--Silverman (Theorem~\ref{thm:E}), the search bound $\hhat_{\max}$ depends exclusively on:
\begin{enumerate}
\item The $L$-function $L(A, s)$ (via the regulator $R$ from BSD/Bloch--Kato),
\item Universal constants $\gamma_r$ and $c(g, K)$.
\end{enumerate}
This establishes the $L$-function as the \emph{universal analytic decidability certificate}: a single Turing machine reads the $L$-function and outputs the Mordell--Weil generators, without parsing the variety's defining equations. This is the strongest form of the DPT (Decidability--Principle--Theorem) thesis from the atlas~\cite{Paper50}.

\subsection{Open questions}

\begin{enumerate}
\item Does effective Lang follow from Szpiro's conjecture? If so, the abc conjecture would provide the $\MP \to \BISH$ gate via a single diophantine inequality.
\item Can the $\MP$ calibration for $r \ge 2$ be sharpened to $\WLPO$ or $\mathrm{LLPO}$ by considering weaker notions of generator search?
\item For non-Northcott motives (K3 surfaces), can additional geometric structure (e.g., Torelli theorems) reduce the $\LPO$ requirement?
\item Does the Gross--Zagier formula~\cite{GrossZagier1986} provide a direct $\BISH$ computation of generators at $r = 1$ without the Bloch--Kato hypothesis?
\end{enumerate}

% ===========================================================
\section{Conclusion}
\label{sec:conclusion}
% ===========================================================

We have applied constructive reverse mathematics to the decidability of Mordell--Weil generators and established that:

\begin{itemize}
\item The Effective Lang Height Lower Bound is the precise gate converting rank~$\ge 2$ from $\MP$ to $\BISH$ (Theorem~A, Lean-verified from one classical axiom).
\item The implication is strict: $\BISH \not\Rightarrow$ Lang (Theorem~B, Lean-verified, no custom axioms, pure constructive witness).
\item The forward direction is verified explicitly for $X_0(389)$ at rank~2 (Theorem~C, Lean-verified, pure rational computation).
\item Without Northcott, decidability escalates to $\LPO$ (Theorem~D, Lean-verified, no axioms whatsoever).
\item Under Uniform Lang--Silverman, the $L$-function is the universal analytic decidability certificate (Theorem~E, Lean-verified from the open conjecture as hypothesis).
\item The hierarchy classifier is axiom-free and \texttt{native\_decide}-verifiable (Lean-verified, no axioms).
\end{itemize}

\noindent The DPT decidability hierarchy for mixed motives is:
\[
\BISH \;\xrightarrow{\text{Lang}}\; \MP \;\xrightarrow{\text{Northcott}}\; \LPO
\]
with Lang providing the computational inversion of Minkowski's geometry of numbers. Under Uniform Lang, the $L$-function alone determines the search bound, completing the pure and mixed motive program of Papers~50--62.

% ===========================================================
\section*{Acknowledgments}
\addcontentsline{toc}{section}{Acknowledgments}
% ===========================================================

We thank the Mathlib contributors for the rational arithmetic and decidability infrastructure. We are grateful to the constructive reverse mathematics community---especially the foundational work of Bishop, Bridges, Richman, and Ishihara---for developing the framework that makes calibrations like these possible. Cremona's tables~\cite{Cremona1997} provided the numerical data for the $X_0(389)$ verification.

The Lean~4 formalization was produced using AI code generation (Claude Code, Opus~4.6) under human direction. The author is a practicing cardiologist rather than a professional number theorist; all mathematical claims should be evaluated on their formal content. We welcome constructive feedback from domain experts.

% ===========================================================
% References
% ===========================================================
\begin{thebibliography}{99}

\bibitem{Baker1966}
A.~Baker.
\newblock Linear forms in the logarithms of algebraic numbers.
\newblock \emph{Mathematika}, 13:204--216, 1966.

\bibitem{BishopBridges1985}
E.~Bishop and D.~Bridges.
\newblock \emph{Constructive Analysis}.
\newblock Springer, 1985.

\bibitem{BridgesRichman1987}
D.~Bridges and F.~Richman.
\newblock \emph{Varieties of Constructive Mathematics}.
\newblock LMS Lecture Note Series 97. Cambridge University Press, 1987.

\bibitem{BridgesVita2006}
D.~Bridges and L.~V\^{\i}\c{t}\u{a}.
\newblock \emph{Techniques of Constructive Analysis}.
\newblock Springer, 2006.

\bibitem{Cremona1997}
J.~E. Cremona.
\newblock \emph{Algorithms for Modular Elliptic Curves}.
\newblock Cambridge University Press, 2nd edition, 1997.

\bibitem{DavidHindry2000}
S.~David and M.~Hindry.
\newblock Minoration de la hauteur de N\'eron--Tate sur les vari\'et\'es ab\'eliennes de type~C.M.
\newblock \emph{J.\ Reine Angew.\ Math.}, 529:1--74, 2000.

\bibitem{GrossZagier1986}
B.~H. Gross and D.~B. Zagier.
\newblock Heegner points and derivatives of $L$-series.
\newblock \emph{Invent.\ Math.}, 84:225--320, 1986.

\bibitem{HindrySilverman2000}
M.~Hindry and J.~H. Silverman.
\newblock \emph{Diophantine Geometry: An Introduction}.
\newblock Springer GTM 201, 2000.

\bibitem{Ishihara2006}
H.~Ishihara.
\newblock Reverse mathematics in Bishop's constructive mathematics.
\newblock \emph{Philosophia Scientiae}, CS~6:43--59, 2006.

\bibitem{Lang1983}
S.~Lang.
\newblock \emph{Fundamentals of Diophantine Geometry}.
\newblock Springer, 1983.

\bibitem{Minkowski1896}
H.~Minkowski.
\newblock \emph{Geometrie der Zahlen}.
\newblock Teubner, Leipzig, 1896.

\bibitem{Northcott1949}
D.~G. Northcott.
\newblock An inequality in the theory of arithmetic on algebraic varieties.
\newblock \emph{Proc.\ Cambridge Philos.\ Soc.}, 45:502--509, 1949.

\bibitem{Paper50}
P.~C.-K. Lee.
\newblock Constructive Reverse Mathematics and the Five Great Conjectures: Atlas Survey.
\newblock Paper~50, this series, 2025.

\bibitem{Paper59}
P.~C.-K. Lee.
\newblock De Rham Decidability and DPT Completeness.
\newblock Paper~59, this series, 2026.

\bibitem{Silverman1986}
J.~H. Silverman.
\newblock \emph{The Arithmetic of Elliptic Curves}.
\newblock Springer GTM 106, 1986.

\bibitem{Silverman1994}
J.~H. Silverman.
\newblock \emph{Advanced Topics in the Arithmetic of Elliptic Curves}.
\newblock Springer GTM 151, 1994.

\bibitem{Paper45}
P.~C.-K. Lee.
\newblock The Weight-Monodromy Conjecture and LPO: De-omniscientizing descent via geometric origin.
\newblock Paper~45, this series, 2025.

\bibitem{Szpiro1981}
L.~Szpiro.
\newblock S\'eminaire sur les pinceaux de courbes de genre au moins deux.
\newblock \emph{Ast\'erisque}, 86, 1981.

\end{thebibliography}

\end{document}

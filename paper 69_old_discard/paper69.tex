\documentclass[11pt]{article}

% ------------------------------------------------------------
% Standard LaTeX packages
% ------------------------------------------------------------
\usepackage[margin=1in]{geometry}
\usepackage{lmodern}
\usepackage{amsmath,amssymb,mathtools}
\usepackage{amsthm}
\usepackage[american]{babel}
\usepackage{stmaryrd}
\usepackage{enumitem}
\usepackage{booktabs}
\usepackage{array}
\usepackage{listings}
\usepackage[x11names,table]{xcolor}
\usepackage{mdframed}
\usepackage{url}
\usepackage[colorlinks=true,linkcolor=blue,citecolor=blue,urlcolor=blue]{hyperref}

% ---------- Theorem environments ----------
\theoremstyle{plain}
\newtheorem{theorem}{Theorem}[section]
\newtheorem{proposition}[theorem]{Proposition}
\newtheorem{lemma}[theorem]{Lemma}
\newtheorem{corollary}[theorem]{Corollary}

\theoremstyle{definition}
\newtheorem{definition}[theorem]{Definition}

\theoremstyle{remark}
\newtheorem{remark}[theorem]{Remark}

% ---------- Lean repo link ----------
\newcommand{\leanRepo}{\url{https://doi.org/10.5281/zenodo.18749375}}

% ---------- Notation ----------
\newcommand{\N}{\mathbb{N}}
\newcommand{\Z}{\mathbb{Z}}
\newcommand{\Q}{\mathbb{Q}}
\newcommand{\R}{\mathbb{R}}
\newcommand{\C}{\mathbb{C}}
\newcommand{\F}{\mathbb{F}}
\newcommand{\Fp}{\mathbb{F}_p}
\newcommand{\A}{\mathbb{A}}
\newcommand{\Pp}{\mathbb{P}}
\newcommand{\calO}{\mathcal{O}}
\newcommand{\fm}{\mathfrak{m}}
\newcommand{\Gal}{\mathrm{Gal}}
\newcommand{\GL}{\mathrm{GL}}
\newcommand{\PGL}{\mathrm{PGL}}
\newcommand{\Frob}{\mathrm{Frob}}
\newcommand{\Sel}{\mathrm{Sel}}
\newcommand{\CRM}{\mathrm{CRM}}

\newcommand{\BISH}{\mathsf{BISH}}
\newcommand{\LPO}{\mathsf{LPO}}
\newcommand{\WLPO}{\mathsf{WLPO}}
\newcommand{\LLPO}{\mathsf{LLPO}}
\newcommand{\MP}{\mathsf{MP}}
\newcommand{\FT}{\mathsf{FT}}
\newcommand{\WKL}{\mathsf{WKL}_0}
\newcommand{\CLASS}{\mathsf{CLASS}}

% ---------- Code listing style for Lean ----------
\definecolor{codegreen}{rgb}{0,0.6,0}
\definecolor{codegray}{rgb}{0.5,0.5,0.5}
\definecolor{codepurple}{rgb}{0.58,0,0.82}
\definecolor{backcolour}{rgb}{0.95,0.95,0.92}

\lstdefinelanguage{Lean}{
  keywords={theorem, lemma, def, definition, axiom, structure, class, instance,
            by, exact, intro, intros, apply, refine, constructor, use, obtain,
            have, show, from, fun, assume, let, in, if, then, else,
            match, with, end, namespace, section, variable, variables,
            example, begin, sorry, admit, noncomputable, classical,
            import, open, export, private, protected, mutual, meta,
            do, for, while, return, try, catch, finally,
            Type, Prop, Sort, Type*, forall, exists, where, extends,
            set, push_neg, rw, simp, omega, nlinarith, linarith,
            ext, rfl, congr, fin_cases, haveI, letI, attribute, inductive,
            deriving, opaque, decide, subst, trivial, cases},
  sensitive=true,
  morecomment=[l]{--},
  morecomment=[s]{/-}{-/},
  morestring=[b]",
  literate=
    {α}{{$\alpha$}}1 {β}{{$\beta$}}1 {γ}{{$\gamma$}}1
    {δ}{{$\delta$}}1 {ε}{{$\varepsilon$}}1 {ζ}{{$\zeta$}}1
    {η}{{$\eta$}}1 {θ}{{$\theta$}}1 {ι}{{$\iota$}}1
    {κ}{{$\kappa$}}1 {λ}{{$\lambda$}}1 {μ}{{$\mu$}}1
    {ν}{{$\nu$}}1 {ξ}{{$\xi$}}1 {π}{{$\pi$}}1
    {ρ}{{$\rho$}}1 {σ}{{$\sigma$}}1 {τ}{{$\tau$}}1
    {φ}{{$\varphi$}}1 {χ}{{$\chi$}}1 {ψ}{{$\psi$}}1
    {ω}{{$\omega$}}1 {Γ}{{$\Gamma$}}1 {Δ}{{$\Delta$}}1
    {Θ}{{$\Theta$}}1 {Λ}{{$\Lambda$}}1 {Σ}{{$\Sigma$}}1
    {Φ}{{$\Phi$}}1 {Ψ}{{$\Psi$}}1 {Ω}{{$\Omega$}}1
    {→}{{$\rightarrow$}}1 {←}{{$\leftarrow$}}1 {↔}{{$\leftrightarrow$}}1
    {⇒}{{$\Rightarrow$}}1 {⇐}{{$\Leftarrow$}}1 {⇔}{{$\Leftrightarrow$}}1
    {∀}{{$\forall$}}1 {∃}{{$\exists$}}1 {∈}{{$\in$}}1
    {∉}{{$\notin$}}1 {⊆}{{$\subseteq$}}1 {⊂}{{$\subset$}}1
    {∪}{{$\cup$}}1 {∩}{{$\cap$}}1 {≤}{{$\leq$}}1
    {≥}{{$\geq$}}1 {≠}{{$\neq$}}1 {≈}{{$\approx$}}1 {≃}{{$\simeq$}}1
    {≡}{{$\equiv$}}1 {∧}{{$\land$}}1 {∨}{{$\lor$}}1
    {¬}{{$\neg$}}1 {ℕ}{{$\mathbb{N}$}}1 {ℝ}{{$\mathbb{R}$}}1
    {ℂ}{{$\mathbb{C}$}}1 {ℤ}{{$\mathbb{Z}$}}1 {ℓ}{{$\ell$}}1
    {·}{{$\cdot$}}1 {∑}{{$\sum$}}1 {∏}{{$\prod$}}1
    {∅}{{$\emptyset$}}1 {∞}{{$\infty$}}1 {∂}{{$\partial$}}1
    {⟨}{{$\langle$}}1 {⟩}{{$\rangle$}}1 {…}{{$\ldots$}}1
    {₀}{{$_0$}}1 {₁}{{$_1$}}1 {₂}{{$_2$}}1 {⧸}{{$/$}}1 {‖}{{$\|$}}1
    {•}{{$\cdot$}}1 {⁻¹}{{$^{-1}$}}1 {⋆}{{$\star$}}1
    {∘}{{$\circ$}}1
}

\lstdefinestyle{leanstyle}{
    language=Lean,
    backgroundcolor=\color{backcolour},
    commentstyle=\color{codegreen},
    keywordstyle=\color{blue},
    stringstyle=\color{codepurple},
    basicstyle=\ttfamily\footnotesize,
    breakatwhitespace=false,
    breaklines=true,
    captionpos=b,
    keepspaces=true,
    numbers=left,
    numbersep=5pt,
    showspaces=false,
    showstringspaces=false,
    showtabs=false,
    tabsize=2,
    numberstyle=\tiny\color{codegray}
}

\lstset{style=leanstyle}

% ---------- Title and author ----------
\title{The Modularity Theorem is $\BISH + \WLPO$:\\
  The BCDT Extension Adds No Logical Cost\\[6pt]
  {\large (Paper~69, Constructive Reverse Mathematics Series)}}

\author{Paul Chun-Kit Lee\thanks{Lean 4 formalization available at \leanRepo.} \\
New York University \\
\texttt{dr.paul.c.lee@gmail.com}}

\date{February 2026}

\begin{document}
\maketitle

% ============================================================
\begin{abstract}
% ============================================================

Paper~68 showed that Wiles's proof of the modularity of
semistable elliptic curves over~$\Q$ calibrates at
$\BISH + \WLPO$, with the $\WLPO$ localized entirely in the
Langlands--Tunnell theorem (Stage~1).  We extend this
classification to the full modularity theorem for all elliptic
curves over~$\Q$ (Breuil--Conrad--Diamond--Taylor 2001).

The extension introduces three new ingredients: Breuil's
classification of finite flat group schemes (replacing
Fontaine--Laffaille for the supersingular case), Conrad's
local-global compatibility, and the Diamond--Taylor $3$--$5$
switching argument.  All three are $\BISH$.  The icosahedral
case ($A_5 \subset \PGL_2(\F_5)$) never arises because every
invocation of residual modularity in the BCDT proof is
delegated to $p = 3$, where $\PGL_2(\F_3) \cong S_4$ is
solvable and Langlands--Tunnell applies unconditionally.

The classification of the full modularity theorem is
$\BISH + \WLPO$, identical to the semistable case.
The Lean~4 verification (324~lines across two files,
zero~\texttt{sorry}) formalizes the logical assembly:
each BCDT ingredient is classified by definition, and the
join is machine-checked.

\end{abstract}

\tableofcontents

% ============================================================
\section{Introduction}\label{sec:intro}
% ============================================================

\subsection{Main results}

The modularity theorem asserts that every elliptic curve
$E/\Q$ is modular: there exists a weight~2 newform~$f$ of
level~$N_E$ such that $\rho_{E,p} \cong \rho_{f,p}$ for
all primes~$p$.  The theorem was proved in stages: Wiles
\cite{Wiles1995} and Taylor--Wiles \cite{TaylorWiles1995}
for the semistable case, and
Breuil--Conrad--Diamond--Taylor \cite{BCDT2001} for the
general case.

Paper~68 \cite{Paper68} performed a constructive reverse
mathematics audit of the semistable proof, finding that it
calibrates at $\BISH + \WLPO$: the Taylor--Wiles engine
(Stages~2--5) is fully constructive ($\BISH$), and the
$\WLPO$ arises solely from the Langlands--Tunnell theorem.
The present paper extends this classification to the full
modularity theorem.  We establish three results:

\begin{description}[leftmargin=2em]
\item[Theorem A] (BCDT extensions are $\BISH$).
Breuil's classification of finite flat group schemes
\cite{Breuil2000}, the Diamond--Taylor $3$--$5$ switching
argument \cite{BCDT2001}, and Conrad's local-global
compatibility \cite{Conrad1999} are all~$\BISH$.  The
icosahedral case ($A_5 \subset \PGL_2(\F_5)$) never
arises because $\PGL_2(\F_3) \cong S_4$ is solvable.

\item[Theorem B] (Full modularity is $\BISH + \WLPO$).
The full modularity theorem for elliptic curves over~$\Q$
(BCDT 2001) calibrates at $\BISH + \WLPO$, identical to
the semistable case.  The $\WLPO$ is consumed by a single
invocation of Langlands--Tunnell at $p = 3$.

\item[Theorem C] (Zero marginal cost).
The BCDT extension from semistable to all elliptic curves
adds no logical cost: $\CRM(\text{BCDT}) = \CRM(\text{Wiles})
= \BISH + \WLPO$.
\end{description}

\subsection{Constructive Reverse Mathematics: a brief primer}

$\CRM$ calibrates mathematical statements against logical
principles of increasing strength within Bishop-style
constructive mathematics ($\BISH$).  The hierarchy relevant
to this paper is:
\[
  \BISH \;\subset\; \BISH + \WLPO \;\subset\; \BISH + \LPO
  \;\subset\; \CLASS.
\]
Here $\WLPO$ (Weak Limited Principle of Omniscience) states
that every binary sequence is identically zero or not.
$\MP$ (Markov's Principle) and $\LLPO$ (Lesser Limited
Principle of Omniscience) are both implied by $\WLPO$, but are
mutually incomparable over $\BISH$; since the present paper
uses only $\BISH$ and $\WLPO$, this distinction does not
affect our results.
For a thorough treatment of $\CRM$, see Bridges--Richman
\cite{BridgesRichman1987}; for the broader program of which
this paper is part, see Papers~1--68 of this series and the
atlas survey~\cite{Paper50}.

\subsection{Current state of the art}

Paper~68 \cite{Paper68} classified Wiles's semistable proof as
$\BISH + \WLPO$, with the Taylor--Wiles engine (Stages~2--5)
fully constructive.  That analysis left open whether the BCDT
extension to non-semistable curves introduces additional
logical cost.  The present paper answers this question in the
negative.

The principal technical observation is group-theoretic:
$\PGL_2(\F_3) \cong S_4$ is solvable.  This ensures that every
invocation of Langlands--Tunnell in the BCDT proof occurs at
$p = 3$, where the projective image is automatically solvable.
The icosahedral case---which would require non-solvable
modularity and potentially higher logical cost---never arises.

\subsection{Position in the atlas}

The present paper is part of the Constructive Reverse
Mathematics program (Papers~1--68).  Paper~67 \cite{Paper67}
synthesizes the arithmetic geometry phase (Papers~45--66);
Paper~50 \cite{Paper50} provides the atlas framework.
Paper~59 \cite{Paper59} classified the $p$-adic comparison
(Fontaine--Laffaille) as~$\BISH$, directly supporting Stage~2
of both the semistable and full modularity proofs.

Paper~69 completes the CRM audit of the modularity theorem
for $\GL_2/\Q$: Paper~68 handled the semistable case; the
present paper handles the general case.  Together, they
establish that the bridge between Galois representations and
automorphic forms for $\GL_2/\Q$ has logical cost $\BISH + \WLPO$,
with all non-constructive content localized in the analytic
theory of weight~1 forms.


% ============================================================
\section{Preliminaries}\label{sec:prelim}
% ============================================================

We recall the key definitions from Paper~68; see that paper
for full details.

\begin{definition}[Weak Limited Principle of Omniscience]\label{def:wlpo}
$\WLPO$: For every binary sequence $\alpha : \N \to \{0,1\}$,
either $\forall n,\;\alpha(n) = 0$ or
$\lnot(\forall n,\;\alpha(n) = 0)$.
Equivalently, for every $x \in \R$:
$x = 0 \lor \lnot(x = 0)$.
\end{definition}

\begin{definition}[The five stages of Wiles's proof]\label{def:stages}
Following Paper~68 and the standard decomposition
\cite{CSS1997, DDT1997}:

\textbf{Stage~1} (Residual modularity):
prove $\bar{\rho}$ is modular via Langlands--Tunnell.

\textbf{Stage~2} (Deformation ring):
construct the universal deformation ring~$R$.

\textbf{Stage~3} (Hecke algebra):
construct $\mathbb{T}$ localized at~$\fm$.

\textbf{Stage~4} (Numerical criterion):
verify the Wiles--Lenstra numerical criterion.

\textbf{Stage~5} (Patching):
select Taylor--Wiles primes and prove $R \cong T$.
\end{definition}

\begin{definition}[CRM classifications from Paper~68]\label{def:paper68}
Paper~68 established:
\begin{itemize}[nosep]
\item Stage~1 (Langlands--Tunnell): $\WLPO$.
\item Stages~2--5 (deformation ring, Hecke algebra,
  numerical criterion, patching): $\BISH$.
\item Overall: $\BISH + \WLPO$.
\end{itemize}
\end{definition}


% ============================================================
\section{Why the Icosahedral Case Does Not Arise}%
\label{sec:icosahedral}
% ============================================================

The Langlands--Tunnell theorem proves that 2-dimensional
complex representations of $G_\Q$ with \emph{solvable}
projective image are automorphic.  Its scope is limited to
the dihedral, tetrahedral, and octahedral cases.  It does not
cover the icosahedral case (projective image $A_5$, which is
not solvable).

For the modularity theorem, the relevant representation is
$\bar{\rho}_p : G_\Q \to \GL_2(\Fp)$ for a prime~$p$.  The
projective image of~$\bar{\rho}_p$ is a subgroup of
$\PGL_2(\Fp)$.

\begin{lemma}[Solvability at $p = 3$]\label{lem:s4}
For every elliptic curve $E/\Q$, the projective image of
$\bar{\rho}_{E,3}$ is solvable.
\end{lemma}

\begin{proof}
$\PGL_2(\F_3) \cong S_4$, the symmetric group on four
letters.  $S_4$ is solvable (with composition series
$1 \triangleleft V_4 \triangleleft A_4 \triangleleft S_4$,
where each quotient is abelian).  Every subgroup of a
solvable group is solvable.  The projective image of
$\bar{\rho}_{E,3}$, being a subgroup of~$S_4$, is therefore
solvable.
\end{proof}

This is the key observation.  At $p = 3$, the solvability of
$\PGL_2(\F_3)$ is a \emph{group-theoretic tautology} requiring
no geometric input from the elliptic curve.  The
Langlands--Tunnell theorem is structurally applicable to
$\bar{\rho}_{E,3}$ for \emph{every} elliptic curve, provided
$\bar{\rho}_{E,3}$ is absolutely irreducible.

At $p = 5$, the situation is different:
$\PGL_2(\F_5) \cong S_5$, which contains the non-solvable
subgroup~$A_5$.  The icosahedral case can and does arise at
$p = 5$.  But BCDT's proof is designed so that \emph{no
residual modularity theorem is ever applied at $p = 5$}.
Instead, modularity of $\bar{\rho}_{E,5}$ is established by
a geometric transfer back to $p = 3$ on an auxiliary curve.


% ============================================================
\section{The \texorpdfstring{$3$--$5$}{3-5} Switching Argument}\label{sec:switch}
% ============================================================

When $\bar{\rho}_{E,3}$ is reducible (equivalently, $E$ admits
a rational $3$-isogeny), the Taylor--Wiles machinery cannot be
applied at $p = 3$: the deformation ring lacks the required
algebraic properties.  BCDT switch to $p = 5$.

\subsection{Constructing the auxiliary curve}

By Mazur's isogeny theorem \cite{Mazur1978}, if $E/\Q$ has a
rational $3$-isogeny, it generically does not have a rational
$5$-isogeny (exceptions correspond to rational points of
$X_0(15)$, an explicitly computed finite set).
Thus $\bar{\rho}_{E,5}$ is absolutely irreducible.

To establish that $\bar{\rho}_{E,5}$ is modular without
invoking icosahedral modularity, BCDT proceed as follows.
Consider the moduli space $X_E(5)$ parameterizing elliptic
curves $E'/\Q$ with $E'[5] \cong E[5]$ as
$G_\Q$-modules.  Since $X(5)$ has genus~$0$, the space
$X_E(5)$ is isomorphic to $\Pp^1_\Q$.

The set of $t \in \Pp^1(\Q)$ for which $\bar{\rho}_{E',3}$ is
reducible is a \emph{thin set} in the sense of
Serre~\cite{Serre1992}: it is contained in the image of
finitely many morphisms from curves of degree $> 1$
to~$\Pp^1$.  Membership in this thin set is decidable
(reducibility of the $3$-division polynomial is a finite
computation in $\Q[x]$), and its rational points of bounded
height are computable.  Since the complement of a thin set in
$\Pp^1(\Q)$ is Zariski-dense, choosing any rational $t$
outside the (computable) bad locus yields an auxiliary curve
$E'$ with:
\begin{enumerate}[label=(\roman*),nosep]
\item $\bar{\rho}_{E',5} \cong \bar{\rho}_{E,5}$ (by
  construction of $X_E(5)$),
\item $\bar{\rho}_{E',3}$ absolutely irreducible (by
  avoidance of the thin set).
\end{enumerate}

Since $\bar{\rho}_{E',3}$ is absolutely irreducible with
projective image in $S_4$ (Lemma~\ref{lem:s4}),
Langlands--Tunnell applies to $E'$ at $p = 3$.
Modularity lifting at $p = 3$ proves $E'$ modular.  Therefore
$\bar{\rho}_{E',5}$ is modular.  By~(i), $\bar{\rho}_{E,5}$
is modular.  Modularity lifting at $p = 5$ then proves $E$
modular.

\subsection{Constructive classification of the switching}

\begin{proposition}[$3$--$5$ switching is $\BISH$]%
\label{prop:switch}
The construction of the auxiliary curve $E'$ is a
$\BISH$-decidable finite computation.
\end{proposition}

\begin{proof}
The thin set of bad $t$-values is determined by the images
of finitely many covering maps to $\Pp^1$, each of degree
$> 1$.  The rational points of bounded height in each image
are computable (polynomial root-finding over~$\Q$ is
decidable in~$\BISH$).  One evaluates the universal family at
small integer values $t = 0, 1, 2, \ldots$ and checks whether
the $3$-division polynomial is irreducible (a finite
computation in $\Q[x]$).  Since the covering degrees are
bounded, the number of bad integer $t$-values up to any height
bound is bounded.  In particular, a suitable $t$ is found
within a computable search bound determined by the degrees of
the covering curves.  No density argument, Chebotarev theorem,
or non-effective existence principle is needed.
\end{proof}


% ============================================================
\section{The New Local Conditions}\label{sec:local}
% ============================================================

BCDT extend Wiles's local deformation conditions to handle all
reduction types at $p = 3$.

\subsection{Breuil's classification (\texorpdfstring{$\BISH$}{BISH})}

For primes $p$ where $E$ has potentially supersingular reduction,
the Fontaine--Laffaille theory used by Wiles does not apply.
Breuil \cite{Breuil2000} classifies finite flat group schemes
over~$\Z_p$ via strongly divisible lattices in filtered
$\varphi$-modules.  The classification is explicit: it
translates the geometric category into matrices with Frobenius
actions over the Witt vectors, subject to bounded determinant
conditions.  The entire classification operates within
finite-length commutative algebra.  No infinite topological
limits or spectral theory is involved.  This is~$\BISH$.

We flag this as an \emph{axiomatized} classification: a
fully formal constructive audit of Breuil's essential
surjectivity (from strongly divisible lattices to group
schemes, proved via deformation theory over Witt vectors)
would require verifying that no non-constructive step enters
the deformation argument.  We classify the ingredient as
$\BISH$ based on its finite-algebra character, pending a
detailed audit analogous to Paper~68's treatment of the
Euler system (Stage~4).

\subsection{Conrad's local-global compatibility (\texorpdfstring{$\BISH$}{BISH})}

Conrad \cite{Conrad1999} verifies that the local Langlands
correspondence for $\GL_2$ at bad primes is compatible with
the global Galois representation.  For $\GL_2/\Q_p$, local
Langlands is explicit (Kutzko, Bushnell--Henniart): it is a
bijection between finite-dimensional representations of the
Weil--Deligne group and irreducible admissible representations
of $\GL_2(\Q_p)$, computable by explicit formulas.  The
compatibility check at primes of bad reduction compares
local Weil--Deligne parameters computed from the
N\'eron model; these are finite-dimensional representations
of finite groups, and the computation is explicit.
This is~$\BISH$.


% ============================================================
\section{The Classification Theorem}\label{sec:main}
% ============================================================

\begin{theorem}[Theorem A: BCDT extensions are $\BISH$]%
\label{thm:extensions}
The three BCDT extensions---Breuil's classification,
the $3$--$5$ switching, and Conrad's local-global
compatibility---are all $\BISH$.
\end{theorem}

\begin{proof}
Breuil's classification operates in finite-length commutative
algebra (\S\ref{sec:local}).  The $3$--$5$ switching is a
$\BISH$-decidable finite computation
(Proposition~\ref{prop:switch}).  Conrad's compatibility
compares finite-dimensional representations
(\S\ref{sec:local}).  Each ingredient avoids topological
limits, spectral theory, and non-effective existence
principles.
\end{proof}

\begin{theorem}[Theorem B: Full modularity is
  $\BISH + \WLPO$]\label{thm:main}
The full modularity theorem for elliptic curves over~$\Q$
(Breuil--Conrad--Diamond--Taylor 2001) calibrates at
$\BISH + \WLPO$.  The classification is identical to the
semistable case (Paper~68).
\end{theorem}

\begin{proof}
The BCDT proof uses the same five-stage structure as Wiles,
with three additions: Breuil's local conditions
(\S\ref{sec:local}), Conrad's compatibility
(\S\ref{sec:local}), and the $3$--$5$ switching
(\S\ref{sec:switch}).  All three are~$\BISH$
(Theorem~\ref{thm:extensions}).

Paper~68 classifies Stages~2--5 of the Taylor--Wiles method
as~$\BISH$ (via Brochard's \cite{Brochard2017} elimination of
patching and effective Chebotarev bounds).  Stage~1 (Langlands--Tunnell)
is~$\WLPO$.

The $\CRM$ join is:
\[
\underbrace{\WLPO}_{\text{Stage 1}} \;\sqcup\;
\underbrace{\BISH}_{\text{Stages 2--5}} \;\sqcup\;
\underbrace{\BISH}_{\text{Breuil}} \;\sqcup\;
\underbrace{\BISH}_{\text{3--5 switch}} \;\sqcup\;
\underbrace{\BISH}_{\text{Conrad}} \;=\; \WLPO.
\]
The BCDT additions are $\BISH$ supplements to existing
$\BISH$ stages.  The join of all stages remains $\WLPO$.
\end{proof}

\begin{corollary}[Theorem C: Zero marginal cost]%
\label{cor:zero}
$\CRM(\text{BCDT}) = \CRM(\text{Wiles}) = \BISH + \WLPO$.
The extension from semistable to all elliptic curves adds no
logical cost.
\end{corollary}

\begin{corollary}[Single invocation of $\WLPO$]%
\label{cor:single}
Every invocation of the Langlands--Tunnell theorem in the full
modularity theorem occurs at $p = 3$, for a representation
with projective image in $S_4$.  The non-constructive content
of the entire modularity theorem is consumed by a single
\emph{type} of atom: the Arthur--Selberg trace formula applied
to a solvable Galois representation at $p = 3$.  (The $3$--$5$
switching may invoke Langlands--Tunnell twice---once for $E$,
once for $E'$---but both invocations use the same
$\WLPO$-costing mechanism.)
\end{corollary}

\begin{proof}
Direct modularity of $\bar{\rho}_{E,3}$ (when irreducible)
uses Langlands--Tunnell at $p = 3$ for~$E$.  The $3$--$5$
switching (\S\ref{sec:switch}) delegates residual modularity
to $\bar{\rho}_{E',3}$ for an auxiliary curve $E'$, again
using Langlands--Tunnell at $p = 3$.  No other invocation
of residual modularity occurs.  In both cases, the
projective image lies in $\PGL_2(\F_3) \cong S_4$, so the
same $\WLPO$-costing trace formula argument applies.
\end{proof}

\begin{corollary}[Algebraic weight~1 modularity
  constructivizes the full theorem]\label{cor:alg}
If a purely algebraic proof exists that 2-dimensional
representations of $G_\Q$ with projective image in
$\PGL_2(\F_3) \cong S_4$ are modular (bypassing the trace
formula), then the full modularity theorem for elliptic
curves over~$\Q$ becomes~$\BISH$.
\end{corollary}


% ============================================================
\section{CRM Audit}\label{sec:crm}
% ============================================================

\subsection{Constructive strength classification}

\begin{center}
\small
\renewcommand{\arraystretch}{1.15}
\begin{tabular}{@{}llll@{}}
\toprule
\textbf{Component} & \textbf{Strength} &
  \textbf{Tight?} & \textbf{Sufficient?} \\
\midrule
Stage~1 (Langlands--Tunnell) & $\WLPO$ &
  $\WLPO$ (Paper~68) & $\WLPO$ sufficient \\
Stages~2--5 (TW engine) & $\BISH$ &
  $\BISH$ (Paper~68) & Yes \\
Breuil (group schemes) & $\BISH$ &
  $\BISH$ (finite algebra) & Yes \\
$3$--$5$ switch (Diamond--Taylor) & $\BISH$ &
  $\BISH$ (bounded comp.) & Yes \\
Conrad (local-global compat.) & $\BISH$ &
  $\BISH$ (finite reps) & Yes \\
\midrule
\textbf{Overall BCDT} & $\BISH + \WLPO$ &
  Yes (join) & Yes \\
\bottomrule
\end{tabular}
\end{center}

\subsection{What descends, from where, to where}

Paper~68 identified the de-omniscientizing descent in the
Taylor--Wiles patching step:
$\MP + \FT \to \BISH$ (1995--2017).  Paper~69 establishes a
complementary structural observation: the extension from
semistable to general elliptic curves adds zero marginal
logical cost.  Every new ingredient introduced by BCDT is
already constructive.

The $\CRM$ descent pattern for the full modularity theorem:
\[
\underbrace{\text{at most } \CLASS}_{\text{unanalyzed}}
\;\xrightarrow{\quad\text{Paper 68 + 69}\quad}\;
\underbrace{\BISH + \WLPO}_{\text{calibrated}}.
\]
The calibration reflects two facts: (1)~the Taylor--Wiles
engine is constructive (Paper~68), and (2)~the BCDT extensions
are constructive (this paper).  The residual $\WLPO$ is
irreducible: it is consumed by the analytic theory of
weight~1 forms.

\subsection{Comparison with Paper~68}

\begin{center}
\renewcommand{\arraystretch}{1.15}
\begin{tabular}{@{}lll@{}}
\toprule
& \textbf{Paper~68 (semistable)} &
  \textbf{Paper~69 (all $E/\Q$)} \\
\midrule
Scope & Semistable $E/\Q$ & All $E/\Q$ \\
New ingredients & --- &
  Breuil, $3$--$5$ switch, Conrad \\
Classification & $\BISH + \WLPO$ & $\BISH + \WLPO$ \\
Source of $\WLPO$ & Stage~1 (L--T) & Stage~1 (L--T) \\
\bottomrule
\end{tabular}
\end{center}

\medskip\noindent
The classification is invariant under the extension.  The
$\WLPO$ cost is determined by the shared entry point
(Langlands--Tunnell at $p = 3$), not by the algebraic
infrastructure that handles different reduction types.

\begin{remark}[Absence of $\FT$, $\MP$, and $\LLPO$]
None of the three BCDT ingredients involves infinite inverse
limits (which would invoke $\FT$), unbounded searches (which
would invoke $\MP$), or real-number comparisons (which would
invoke $\LLPO$).  All comparisons in the BCDT ingredients are
over finite fields or finite-dimensional $\Fp$-vector spaces,
where equality is decidable.  The only principle beyond $\BISH$
in the full modularity proof is the $\WLPO$ inherited from
Stage~1.
\end{remark}

\begin{remark}[Proof method vs.\ theorem]
The ``zero marginal cost'' (Theorem~C) refers to the
\emph{proof method}: the BCDT proof has the same CRM
classification as Wiles's semistable proof.  This does not
rule out the possibility that a different proof of the full
modularity theorem---say, one avoiding $3$--$5$ switching
entirely---could have a different CRM classification.  Our
audit classifies the published proof, not the theorem itself.
\end{remark}


% ============================================================
\section{Formal Verification}\label{sec:formal}
% ============================================================

\subsection{File structure and build status}

The Lean~4 bundle resides at
\texttt{P69\_BCDT/} with the following structure:

\begin{center}
\renewcommand{\arraystretch}{1.15}
\begin{tabular}{@{}llp{7cm}@{}}
\toprule
\textbf{File} & \textbf{Lines} & \textbf{Content} \\
\midrule
\texttt{Paper69\_CRMBase.lean} & 120 &
  CRM hierarchy, Paper~68 stage classifications,
  reference theorems \\
\texttt{Paper69\_Classification.lean} & 204 &
  BCDT extension defs, main classification
  theorems, corollaries \\
\midrule
\textbf{Total} & \textbf{324} &
  \texttt{sorry}: 0 \quad warnings: 0 \quad errors: 0 \\
\bottomrule
\end{tabular}
\end{center}

\medskip\noindent
\textbf{Build status:} \texttt{lake build} $\to$
\textbf{0~errors, 0~warnings, 0~\texttt{sorry}s}.
Lean~4 version: \texttt{v4.29.0-rc1}.
Mathlib4 dependency via \texttt{lakefile.lean}.

\subsection{Axiom inventory}

The formalization declares \emph{no opaque types} and
\emph{no axioms}.  All stage classifications are
\texttt{def} declarations (not axioms):

\begin{center}
\small
\renewcommand{\arraystretch}{1.15}
\begin{tabular}{@{}rllp{5.5cm}@{}}
\toprule
\textbf{\#} & \textbf{Definition} & \textbf{Value} &
  \textbf{Justification} \\
\midrule
1 & \texttt{stage1\_class} & $\WLPO$ &
  Paper~68, Theorem~C \\
2 & \texttt{stage2\_class} & $\BISH$ &
  Paper~68, Theorem~B \\
3 & \texttt{stage3\_class} & $\BISH$ &
  Paper~68, Theorem~B \\
4 & \texttt{stage4\_class} & $\BISH$ &
  Paper~68, Theorem~B \\
5 & \texttt{stage5\_class} & $\BISH$ &
  Paper~68, Theorem~A \\
6 & \texttt{breuil\_class} & $\BISH$ &
  Breuil \cite{Breuil2000}, finite algebra \\
7 & \texttt{switch35\_class} & $\BISH$ &
  BCDT \cite{BCDT2001}, \S\ref{sec:switch} \\
8 & \texttt{conrad\_class} & $\BISH$ &
  Conrad \cite{Conrad1999}, finite reps \\
\bottomrule
\end{tabular}
\end{center}

\smallskip\noindent
This is a significant structural difference from Paper~68,
which required 12~opaque types and 8~theorem-level axioms.
Paper~69 takes Paper~68's classifications as established
results and records them as definitional assignments.  The
Lean verification then reduces to exhaustive case analysis
on a finite inductive type.

\subsection{Key code snippets}

\textbf{CRM hierarchy and join}
(from \texttt{Paper69\_CRMBase.lean}):

\begin{lstlisting}
inductive CRMLevel where
  | BISH | MP | LLPO | WLPO | LPO | CLASS
  deriving DecidableEq, Repr

def join : CRMLevel -> CRMLevel -> CRMLevel
  | BISH,  b     => b
  | a,     BISH  => a
  | CLASS, _     => CLASS
  | _,     CLASS => CLASS
  | LPO,   _     => LPO
  | _,     LPO   => LPO
  | WLPO,  _     => WLPO
  | _,     WLPO  => WLPO
  | LLPO,  _     => LLPO
  | _,     LLPO  => LLPO
  | MP,    MP    => MP
\end{lstlisting}

\textbf{BCDT classification theorem}
(from \texttt{Paper69\_Classification.lean}):

\begin{lstlisting}
def bcdt_overall : CRMLevel :=
  join stage1_class
    (join stage2_class
      (join stage3_class
        (join stage4_class
          (join stage5_class
            (join breuil_class
              (join switch35_class conrad_class))))))

theorem bcdt_classification :
    bcdt_overall = CRMLevel.WLPO := by
  simp [bcdt_overall, stage1_class, stage2_class,
        stage3_class, stage4_class, stage5_class,
        breuil_class, switch35_class, conrad_class, join]
\end{lstlisting}

\textbf{Algebraic constructivization}
(from \texttt{Paper69\_Classification.lean}):

\begin{lstlisting}
theorem algebraic_lt_implies_bish_bcdt
  (alt_stage1 : CRMLevel)
  (h : alt_stage1 = CRMLevel.BISH) :
  join alt_stage1
    (join stage2_class
      (join stage3_class
        (join stage4_class
          (join stage5_class
            (join breuil_class
              (join switch35_class conrad_class))))))
    = CRMLevel.BISH := by
  subst h
  simp [stage2_class, stage3_class, stage4_class,
        stage5_class, breuil_class, switch35_class,
        conrad_class, join]
\end{lstlisting}

\subsection{\texttt{\#print axioms} output}

\begin{center}
\small
\begin{tabular}{@{}ll@{}}
\toprule
\textbf{Theorem} & \textbf{Axioms (custom only)} \\
\midrule
\texttt{bcdt\_classification} &
  \textbf{None} (definitional \texttt{simp}) \\
\texttt{bcdt\_without\_stage1\_is\_bish} &
  \textbf{None} (definitional \texttt{simp}) \\
\texttt{bcdt\_equals\_paper68} &
  \textbf{None} (definitional \texttt{simp}) \\
\texttt{algebraic\_lt\_implies\_bish\_bcdt} &
  \textbf{None} (\texttt{subst} + \texttt{simp}) \\
\texttt{all\_others\_bish} &
  \textbf{None} (\texttt{rfl} tuple) \\
\bottomrule
\end{tabular}
\end{center}

\medskip\noindent
\begin{sloppypar}
\textbf{Classical.choice audit.}  The CRM hierarchy
(\texttt{CRMLevel}) is a finite inductive type with
\texttt{DecidableEq}.  All theorems reduce by
\texttt{simp [join]} to exhaustive case analysis.
\texttt{\#print axioms bcdt\_classification} shows only
\texttt{propext} and \texttt{Quot.sound}---no
\texttt{Classical.choice}.  The entire formalization is
constructively clean.
\end{sloppypar}

\subsection{Reproducibility}

Lean~4 formalization files are available at the Zenodo
repository: \leanRepo.  The bundle compiles with
\texttt{lake build} on Lean v4.29.0-rc1 + Mathlib4.


% ============================================================
\section{Discussion}\label{sec:discuss}
% ============================================================

\subsection{The de-omniscientizing descent pattern}

Paper~50 \cite{Paper50} identified a ``de-omniscientizing
descent'' in the five great conjectures: geometric origin
converts $\LPO$-level data to $\BISH$-level data.  Paper~68
revealed this pattern in the Taylor--Wiles method's evolution
(1995--2017).  Paper~69 adds a complementary observation:
when the algebraic infrastructure is extended (BCDT), the
constructive cost does not increase.

This is consistent with the atlas pattern: the non-constructive
content in the Langlands correspondence for $\GL_2/\Q$ lives
on the automorphic side (trace formula, $L$-functions), not
on the Galois side (deformations, patching, local conditions).
The BCDT extension exclusively enlarges the Galois side.

\subsection{The solvability shield}

The group-theoretic fact $\PGL_2(\F_3) \cong S_4$ (solvable)
acts as a structural shield: it ensures that the
Langlands--Tunnell theorem---whose scope is limited to
solvable projective images---suffices for the entire
modularity theorem.  The icosahedral barrier at $p = 5$
($A_5$ is not solvable) is circumvented by the $3$--$5$
switching argument, which transfers all residual modularity
questions back to $p = 3$.

From the CRM perspective, this is significant: the $3$--$5$
switch is a \emph{constructive maneuver} that avoids a
\emph{potentially non-constructive obstacle}.  If icosahedral
modularity were needed, it would require either the full
Artin conjecture or a different proof strategy, either of
which might introduce additional logical cost.

\subsection{What the Lean verification adds}

The Lean~4 formalization verifies the \emph{join computation}:
given definitional assignments for each ingredient's CRM level,
the overall join is machine-checked to equal $\WLPO$.
Unlike Paper~68, which required opaque types and axioms to
model the mathematical universe (Brochard's criterion, effective
Chebotarev, etc.), Paper~69 takes Paper~68's results as
established and records the three BCDT classifications as
bare definitions.  The Lean proofs reduce entirely to
\texttt{simp [join]}---exhaustive case analysis on a finite
inductive type.

This means the formalization adds no assurance beyond what a
hand computation provides for the join.  What it \emph{does}
add is machine-checked traceability: the definitions record
precisely where human mathematical judgment enters (the
\texttt{breuil\_class}, \texttt{switch35\_class}, and
\texttt{conrad\_class} assignments), and the machine verifies
that no additional judgments are smuggled in.  The
zero-\texttt{sorry} guarantee ensures no logical step has been
skipped.  This is the standard methodology for CRM
formalization (cf.\ Paper~10 \cite{Paper10}).

\subsection{Is the \texorpdfstring{$\WLPO$}{WLPO} intrinsic
  to the modularity theorem?}

The full modularity theorem is a $\Pi^0_1$-equivalent
statement (every elliptic curve over~$\Q$ is modular---this
is checkable curve by curve).  As for FLT, the $\WLPO$ in the
proof is a feature of the proof \emph{method}, not necessarily
of the \emph{theorem}.  If algebraic weight~1 modularity is
established (Corollary~\ref{cor:alg}), the entire modularity
theorem becomes~$\BISH$.

The deeper question is whether the ``solvability shield''
($\PGL_2(\F_3) \cong S_4$) is a contingent feature of the
BCDT strategy or a necessary structural ingredient.  If the
modularity theorem required handling the icosahedral case
directly, the CRM classification could potentially involve
higher principles.  That the BCDT proof avoids this case
entirely---by routing everything through $p = 3$---is a
remarkable structural economy.

\subsection{Open questions}

\begin{enumerate}[nosep]
\item \textbf{Algebraic weight~1 modularity.}
  Overconvergent $p$-adic methods
  (Buzzard--Taylor \cite{BuzzardTaylor1999}) offer the most
  promising path to eliminating Stage~1's $\WLPO$.  If
  algebraic weight~1 modularity is established, the full
  modularity theorem becomes~$\BISH$
  (Corollary~\ref{cor:alg}).

\item \textbf{Higher-rank modularity lifting.}
  Does the zero-marginal-cost phenomenon persist for
  $\GL_n$?  Barnet-Lamb--Gee--Geraghty \cite{BLGGT2014}
  extend modularity lifting to $\GL_n$; the constructive
  status of these extensions remains open.

\item \textbf{Icosahedral modularity.}
  Buzzard--Dickinson--Shepherd-Barron--Taylor
  \cite{BDSBT2001} proved icosahedral modularity for
  certain representations at $p = 2$.  The CRM
  classification of their proof---which avoids the trace
  formula by using $p$-adic methods---is an interesting
  open question.
\end{enumerate}


% ============================================================
\section{Conclusion}\label{sec:conclusion}
% ============================================================

We have extended the constructive reverse mathematics audit of
the modularity theorem from the semistable case (Paper~68) to
all elliptic curves over~$\Q$ (BCDT 2001).  The three BCDT
extensions---Breuil's group scheme classification, the
Diamond--Taylor $3$--$5$ switching, and Conrad's local-global
compatibility---are all~$\BISH$.  The overall classification
remains $\BISH + \WLPO$, identical to the semistable case.

The result is both expected and informative.  Expected, because
the BCDT extension modifies only the algebraic infrastructure
(Galois side), which was already constructive in Paper~68.
Informative, because it confirms that the $\WLPO$ in the
modularity theorem is intrinsic to the \emph{automorphic entry
point} (Langlands--Tunnell), not to the algebraic machinery
that handles different reduction types and representation
images.

The Lean~4 verification (324~lines, 0~\texttt{sorry},
0~axioms) formalizes the logical assembly.  Unlike Paper~68,
which required opaque types and axioms to model deep
mathematical inputs, Paper~69 takes Paper~68's classifications
as established results and verifies the join computation
entirely by definitional reduction.

Combined with Paper~68, this completes the CRM audit of the
modularity theorem for $\GL_2/\Q$.  The logical cost of
proving that every elliptic curve over~$\Q$ is modular is
$\BISH + \WLPO$: the entire algebraic infrastructure is
constructive; the single non-constructive cost is one
invocation of the trace formula at $p = 3$.


% ============================================================
\section*{Acknowledgments}
\addcontentsline{toc}{section}{Acknowledgments}
% ============================================================

We thank the Mathlib contributors for the decidable-equality
and \texttt{Nat.Basic} infrastructure that underpins the
formalization.  We are grateful to the constructive reverse
mathematics community---especially the foundational work of
Bishop, Bridges, Richman, and Ishihara---for developing the
framework that makes calibrations like these possible.

The deep mathematics is due to Wiles, Taylor, Diamond, Breuil,
Conrad, Langlands, Tunnell, and Brochard.  The CRM
methodology follows Bishop~\cite{Bishop1967} and
Bridges--Richman~\cite{BridgesRichman1987}.  The Lean~4
formalization was produced using AI code generation
(Claude Code, Opus 4.6) under human direction.  The author is a
practicing cardiologist rather than a professional logician or
arithmetic geometer; all mathematical claims should be
evaluated on their formal content.  We welcome constructive
feedback from domain experts.


% ============================================================
\begin{thebibliography}{30}
% ============================================================

\bibitem{BCDT2001}
C.~Breuil, B.~Conrad, F.~Diamond, and R.~Taylor.
\newblock On the modularity of elliptic curves over~$\Q$: wild
$3$-adic exercises.
\newblock \textit{J.~Amer.\ Math.\ Soc.}, 14(4):843--939, 2001.

\bibitem{BDSBT2001}
K.~Buzzard, M.~Dickinson, N.~Shepherd-Barron, and R.~Taylor.
\newblock On icosahedral Artin representations.
\newblock \textit{Duke Math.\ J.}, 109(2):283--318, 2001.

\bibitem{BLGGT2014}
T.~Barnet-Lamb, T.~Gee, D.~Geraghty, and R.~Taylor.
\newblock Potential automorphy and change of weight.
\newblock \textit{Ann.\ of Math.}, 179(2):501--609, 2014.

\bibitem{Bishop1967}
E.~Bishop.
\newblock \textit{Foundations of Constructive Analysis}.
\newblock McGraw-Hill, 1967.

\bibitem{BridgesRichman1987}
D.~Bridges and F.~Richman.
\newblock \textit{Varieties of Constructive Mathematics}.
\newblock LMS Lecture Note Series 97. Cambridge University Press, 1987.

\bibitem{Brochard2017}
S.~Brochard.
\newblock Proof of de~Smit's conjecture: a freeness criterion.
\newblock \textit{Compositio Math.}, 153(11):2310--2317, 2017.

\bibitem{Breuil2000}
C.~Breuil.
\newblock Groupes $p$-divisibles, groupes finis et modules
filtr\'es.
\newblock \textit{Ann.\ of Math.}, 152(2):489--549, 2000.

\bibitem{BuzzardTaylor1999}
K.~Buzzard and R.~Taylor.
\newblock Companion forms and weight one forms.
\newblock \textit{Ann.\ of Math.}, 149(3):905--919, 1999.

\bibitem{Conrad1999}
B.~Conrad.
\newblock Finite group schemes over bases with low ramification.
\newblock \textit{Compositio Math.}, 119(3):239--320, 1999.

\bibitem{CSS1997}
G.~Cornell, J.\,H.~Silverman, and G.~Stevens, editors.
\newblock \textit{Modular Forms and Fermat's Last Theorem}.
\newblock Springer, 1997.

\bibitem{DDT1997}
H.~Darmon, F.~Diamond, and R.~Taylor.
\newblock Fermat's Last Theorem.
\newblock In \textit{Elliptic Curves, Modular Forms \& Fermat's
Last Theorem}, pp.~2--140.  International Press, 1997.

\bibitem{Serre1992}
J.-P.~Serre.
\newblock \textit{Topics in Galois Theory}.
\newblock Research Notes in Mathematics~1. Jones and Bartlett, 1992.

\bibitem{Mazur1978}
B.~Mazur.
\newblock Rational isogenies of prime degree.
\newblock \textit{Invent.\ Math.}, 44(2):129--162, 1978.

\bibitem{TaylorWiles1995}
R.~Taylor and A.~Wiles.
\newblock Ring-theoretic properties of certain Hecke algebras.
\newblock \textit{Ann.\ of Math.}, 141(3):553--572, 1995.

\bibitem{Wiles1995}
A.~Wiles.
\newblock Modular elliptic curves and Fermat's Last Theorem.
\newblock \textit{Ann.\ of Math.}, 141(3):443--551, 1995.

%% CRM program references
\bibitem{Paper10}
P.\,C.\,K.~Lee.
\newblock Formalization methodology and constructive
stratification (Paper~10, CRM series).
\newblock \textit{Zenodo}, 2025.

\bibitem{Paper50}
P.\,C.\,K.~Lee.
\newblock Three axioms for the motive: a decidability
characterization of Grothendieck's universal cohomology
(Paper~50, CRM series).
\newblock \textit{Zenodo}, 2026.

\bibitem{Paper59}
P.\,C.\,K.~Lee.
\newblock De Rham decidability and DPT completeness
(Paper~59, CRM series).
\newblock \textit{Zenodo}, 2026.

\bibitem{Paper67}
P.\,C.\,K.~Lee.
\newblock Decidability and self-intersection in arithmetic
geometry: a constructive reverse mathematics monograph
(Paper~67, CRM series).
\newblock \textit{Zenodo}, 2026.

\bibitem{Paper68}
P.\,C.\,K.~Lee.
\newblock The logical cost of Fermat's Last Theorem: a
constructive reverse mathematics audit of Wiles's proof
(Paper~68, CRM series).
\newblock \textit{Zenodo}, 2026.
\newblock \texttt{doi:10.5281/zenodo.18748460}.

\end{thebibliography}

\end{document}

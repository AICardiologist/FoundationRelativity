\documentclass[11pt,a4paper]{article}

% ====================================================================
% Packages
% ====================================================================
\usepackage[utf8]{inputenc}
\usepackage[T1]{fontenc}
\usepackage{amsmath,amssymb,amsthm}
\usepackage{mathtools}
\usepackage{hyperref}
\usepackage[margin=1in]{geometry}
\usepackage{enumitem}
\usepackage{booktabs}
\usepackage{listings}
\usepackage[table]{xcolor}
\usepackage{cleveref}
\usepackage{natbib}
\usepackage{mdframed}

% ====================================================================
% Theorem environments
% ====================================================================
\theoremstyle{plain}
\newtheorem{theorem}{Theorem}[section]
\newtheorem{lemma}[theorem]{Lemma}
\newtheorem{proposition}[theorem]{Proposition}
\newtheorem{corollary}[theorem]{Corollary}

\theoremstyle{definition}
\newtheorem{definition}[theorem]{Definition}
\newtheorem{remark}[theorem]{Remark}

% ====================================================================
% Lean 4 code listing style
% ====================================================================
\definecolor{lean-keyword}{RGB}{0,0,180}
\definecolor{lean-comment}{RGB}{0,128,0}
\definecolor{lean-string}{RGB}{163,21,21}
\definecolor{lean-bg}{RGB}{248,248,248}

\lstdefinelanguage{lean4}{
  keywords={theorem,lemma,def,class,instance,import,open,variable,
            noncomputable,section,namespace,end,where,let,have,show,
            intro,obtain,use,exact,rw,simp,apply,by,fun,match,if,
            then,else,do,return,axiom,abbrev,private,attribute,
            suffices,change,congr,ext,constructor,rintro,push_neg,
            linarith,absurd,set_option,omit,in,set,cases,structure,
            refine,unfold,rcases,calc,all_goals,first,try,ring,
            positivity,induction},
  sensitive=true,
  morecomment=[l]{--},
  morecomment=[s]{/-}{-/},
  morestring=[b]",
  morestring=[b]',
}

\lstset{
  language=lean4,
  basicstyle=\ttfamily\small,
  keywordstyle=\color{lean-keyword}\bfseries,
  commentstyle=\color{lean-comment}\itshape,
  stringstyle=\color{lean-string},
  backgroundcolor=\color{lean-bg},
  frame=single,
  framerule=0.5pt,
  breaklines=true,
  breakatwhitespace=true,
  tabsize=2,
  showstringspaces=false,
  numbers=left,
  numberstyle=\tiny\color{gray},
  numbersep=5pt,
  xleftmargin=15pt,
  captionpos=b,
}

% ====================================================================
% Macros
% ====================================================================
\newcommand{\NN}{\mathbb{N}}
\newcommand{\RR}{\mathbb{R}}
\newcommand{\ZZ}{\mathbb{Z}}
\newcommand{\LPO}{\mathrm{LPO}}
\newcommand{\WLPO}{\mathrm{WLPO}}
\newcommand{\LLPO}{\mathrm{LLPO}}
\newcommand{\BMC}{\mathrm{BMC}}
\newcommand{\BISH}{\mathrm{BISH}}
\newcommand{\Lean}{\textsc{Lean~4}}
\newcommand{\Mathlib}{\textsc{Mathlib4}}
\newcommand{\leanok}{\textsf{\small \textcolor{green!70!black}{\checkmark}}}

% ====================================================================
% Title
% ====================================================================
\title{%
  \textbf{The Logical Cost of Quantum Tunneling:\\[4pt]
  LLPO and WKB Turning Points}\\[6pt]
  {\normalsize Paper~19 in the Constructive Reverse Mathematics Series}%
}

\author{
  Paul Chun-Kit Lee\thanks{%
    New York University.
    AI-assisted formalization; see \S\ref{sec:ai} for methodology.
    The author is a medical professional, not a domain expert in
    constructive mathematics or mathematical physics; mathematical
    content was developed with extensive AI assistance.} \\
  New York University \\
  \texttt{dr.paul.c.lee@gmail.com}
}

\date{February 2026}

% ====================================================================
\begin{document}
\maketitle

% ====================================================================
\begin{abstract}
Quantum tunneling through a potential barrier, treated via the WKB
(Wentzel--Kramers--Brillouin) semiclassical approximation, splits
into three tiers of constructive logical cost.
\textbf{Tier~1 ($\BISH$):} For any specific barrier with
algebraically given turning points, the WKB action integral and
tunneling amplitude are computable in Bishop's constructive
mathematics.
\textbf{Tier~2 ($\LLPO$):} For a general continuous barrier, the
existence of classical turning points---where the particle energy
equals the potential---is equivalent to the Lesser Limited Principle
of Omniscience via the constructive Intermediate Value Theorem
\citep{Bridges89,Ishihara90}. This is the first physical calibration
of $\LLPO$ in the series.
\textbf{Tier~3 ($\LPO$):} The full semiclassical computation,
including the $\hbar \to 0$ limit, requires the Limited Principle
of Omniscience via Bounded Monotone Convergence.
All results are formalized in \Lean{} with \Mathlib{} (1{,}081~lines,
15~files, zero \texttt{sorry}). The calibration table gains its first
$\LLPO$ entry: quantum tunneling turning points sit strictly between
$\BISH$ and $\WLPO$.
\end{abstract}

\vspace{1em}
\tableofcontents

% ====================================================================
\section{Introduction}\label{sec:intro}
% ====================================================================

\subsection{Quantum Tunneling and the WKB Approximation}
\label{sec:physical}

Quantum tunneling is one of the most distinctly non-classical
phenomena in physics. A particle encountering a potential barrier
$V(x)$ at energy $E < \max V$ has, in classical mechanics, zero
probability of crossing the barrier. In quantum mechanics, the
particle's wave function extends into the classically forbidden
region $\{x : V(x) > E\}$, and there is a finite probability of
transmission. This effect underlies alpha decay
\citep{Gamow28,GurneyCondon28,GurneyCondon29}, scanning tunneling
microscopy, and semiconductor junction devices.

The WKB (Wentzel--Kramers--Brillouin) semiclassical
approximation~\citep{Wentzel26,Kramers26,Brillouin26} provides the
standard analytic estimate of the tunneling probability. For a
one-dimensional barrier with classical turning points $x_1 < x_2$
(where $V(x_i) = E$), the tunneling amplitude is
\begin{equation}\label{eq:wkb-amplitude}
  T_{\mathrm{WKB}} = \exp\!\bigl(-S/\hbar\bigr),
  \qquad
  S = \int_{x_1}^{x_2} \!\sqrt{2m\bigl(V(x) - E\bigr)}\;dx,
\end{equation}
where $m$ is the particle mass, $\hbar$ is the reduced Planck
constant, and $S$ is the WKB action integral through the classically
forbidden region~\citep{Griffiths18}.

The key question this paper addresses is: \textbf{what is the logical
cost of computing the tunneling amplitude?}

\subsection{The Answer: Three Tiers}\label{sec:answer}

The answer decomposes into three tiers:
\begin{enumerate}
  \item \textbf{Tier~1 ($\BISH$):} For any specific barrier
    (rectangular, parabolic, or any polynomial) with algebraically
    given turning points, the WKB action integral and tunneling
    amplitude are BISH-computable. No omniscience principle is needed.

  \item \textbf{Tier~2 ($\LLPO$):} For a general continuous barrier
    $V : [0,1] \to \RR$ with $V(0) < E$, $V(c) > E$ for some
    $c$, and $V(1) < E$, the existence of turning points
    $x_1, x_2$ with $V(x_i) = E$ is equivalent to the Lesser
    Limited Principle of Omniscience. The mechanism is the
    constructive Intermediate Value Theorem.

  \item \textbf{Tier~3 ($\LPO$):} The full semiclassical
    computation---including both turning point identification and the
    $\hbar \to 0$ limit---requires the Limited Principle of
    Omniscience via Bounded Monotone Convergence.
\end{enumerate}

\noindent
The main results, stated precisely, are:

\begin{itemize}
  \item \textbf{Theorem~1} (Part~A): The WKB tunneling amplitude for
    a specific barrier with given turning points is BISH-computable.
  \item \textbf{Theorem~4} (Part~B): The Turning Point Problem is
    equivalent to $\LLPO$.
  \item \textbf{Theorem~5} (Part~C): The full WKB computation for a
    general barrier is equivalent to $\LPO$.
  \item \textbf{Theorem~6}: Dispensability---specific barriers need
    no omniscience principles.
\end{itemize}

\subsection{Programme Context}\label{sec:context}

This is Paper~19 in a programme of constructive calibration of
mathematical physics~\cite{Lee26-P2,Lee26-P7,Lee26-P8,Lee26-P15}.
Papers~2--18 have calibrated physical idealizations at $\BISH$,
$\WLPO$, and $\LPO$, but no physical result has been calibrated at
$\LLPO$. The constructive hierarchy is:
\[
  \BISH \;<\; \LLPO \;<\; \WLPO \;<\; \LPO.
\]
All implications are strict (no reverse implications hold over
$\BISH$). Until now, the $\LLPO$ level had no physical instantiation
in the calibration table. Paper~19 fills this gap.

\subsection{What Makes This Paper Different}
\label{sec:different}

Paper~19 contributes three novelties:
\begin{enumerate}
  \item \textbf{First $\LLPO$ calibration.} The turning point problem
    is the first physical assertion in the series whose logical cost
    is exactly $\LLPO$---strictly between $\BISH$ and $\WLPO$.

  \item \textbf{Three-tier decomposition.} The analysis separates
    tunneling into specific computation ($\BISH$), general geometry
    ($\LLPO$), and asymptotic limit ($\LPO$). Each tier adds exactly
    one level of the constructive hierarchy.

  \item \textbf{Quantum mechanics is logically cheap.} The
    non-constructive content of tunneling lies not in the quantum
    mechanics itself (which is $\BISH$ for any specific system) but
    in the \emph{classical geometry} of the barrier (turning points,
    $\LLPO$) and the \emph{classical limit} ($\hbar \to 0$, $\LPO$).
\end{enumerate}


% ====================================================================
\section{Background}\label{sec:background}
% ====================================================================

\subsection{The WKB Semiclassical Approximation}\label{sec:wkb-bg}

Consider a particle of mass $m$ in a one-dimensional potential
$V(x)$ at energy $E$. The time-independent Schr\"odinger equation is
\begin{equation}\label{eq:schrodinger}
  -\frac{\hbar^2}{2m}\,\psi''(x) + V(x)\,\psi(x) = E\,\psi(x).
\end{equation}
The \emph{classically forbidden region} is the set
$\{x : V(x) > E\}$, where a classical particle cannot penetrate.
In quantum mechanics, the wave function is exponentially damped but
nonzero in this region.

The WKB ansatz $\psi(x) \sim \exp\!\bigl(\pm S(x)/\hbar\bigr)$
yields the tunneling probability
\begin{equation}\label{eq:tunneling}
  T \;\sim\; \exp\!\Bigl(-\frac{2}{\hbar}\int_{x_1}^{x_2}
    \!\sqrt{2m\bigl(V(x) - E\bigr)}\;dx\Bigr),
\end{equation}
where $x_1$ and $x_2$ are the \emph{classical turning points}:
points where $V(x_i) = E$, marking the boundaries of the forbidden
region. Between $x_1$ and $x_2$, we have $V(x) > E$, so the
integrand is real and positive.

The turning points are roots of the equation $V(x) - E = 0$. For a
specific polynomial potential, these roots can be computed
algebraically. For a general continuous potential, finding the roots
is an Intermediate Value Theorem problem.

\subsection{The Constructive Hierarchy}\label{sec:hierarchy-bg}

Constructive reverse mathematics (CRM) classifies mathematical
theorems by the weakest omniscience principle needed to prove
them~\citep{Bishop67,BV06,Ishihara06,Diener20}. Bishop's
constructive mathematics ($\BISH$) avoids all omniscience principles;
every existential claim comes with a computable witness.

\begin{definition}[$\LLPO$]\label{def:llpo}
The \emph{Lesser Limited Principle of Omniscience}: for every binary
sequence $\alpha : \NN \to \{0,1\}$ with at most one index $n$
satisfying $\alpha(n) = 1$, either $\alpha(2n) = 0$ for all $n$, or
$\alpha(2n+1) = 0$ for all $n$.
\end{definition}

\begin{definition}[$\WLPO$]\label{def:wlpo}
The \emph{Weak Limited Principle of Omniscience}: for every binary
sequence $\alpha$, either $\alpha(n) = 0$ for all $n$, or it is not
the case that $\alpha(n) = 0$ for all $n$.
\end{definition}

\begin{definition}[$\LPO$]\label{def:lpo}
The \emph{Limited Principle of Omniscience}: for every binary
sequence $\alpha$, either $\alpha(n) = 0$ for all $n$, or there
exists $n$ with $\alpha(n) = 1$.
\end{definition}

\begin{definition}[$\BMC$]\label{def:bmc}
\emph{Bounded Monotone Convergence}: every bounded non-decreasing
sequence of reals has a limit.
\end{definition}

\noindent
The hierarchy and key equivalences are:
\begin{equation}\label{eq:hierarchy}
  \BISH \;<\; \LLPO \;<\; \WLPO \;<\; \LPO
  \;\equiv\; \BMC.
\end{equation}
The equivalence $\LLPO \leftrightarrow \text{ExactIVT}$ is due to
\citet{BR87} and \citet{Bridges89,Ishihara90}. The equivalence
$\BMC \leftrightarrow \LPO$ is due to \citet{BV06}.

\begin{remark}[Approximate vs.\ exact IVT]\label{rem:ivt-approx}
The \emph{approximate} IVT---for every $\varepsilon > 0$, there
exists $x$ with $|f(x)| < \varepsilon$---is $\BISH$-valid. Only the
\emph{exact} IVT ($f(x) = 0$) requires $\LLPO$.
\end{remark}

\subsection{The CRM Diagnostic}\label{sec:diagnostic}

The CRM diagnostic for a physical assertion proceeds as follows:
\begin{enumerate}
  \item Formalize the assertion and its proof in \Lean{} with
    \Mathlib{}.
  \item Declare axioms for known CRM equivalences
    (\texttt{exact\_ivt\_iff\_llpo}, \texttt{bmc\_iff\_lpo}).
  \item Run \texttt{\#print axioms} on each main theorem.
  \item The custom axioms in the output certify the CRM level.
    Theorems with no custom axioms are $\BISH$; theorems depending on
    \texttt{exact\_ivt\_iff\_llpo} alone are $\LLPO$; theorems
    depending on both are $\LPO$.
\end{enumerate}


% ====================================================================
\section{Part~A: Specific Barriers Are BISH}\label{sec:part-a}
% ====================================================================

The first tier of the decomposition: when the turning points are
given algebraically, the entire WKB computation is constructive.

\begin{definition}[Specific barrier]\label{def:specific-barrier}
\leanok{}
A \emph{specific barrier} consists of a continuous potential
$V : \RR \to \RR$, an energy $E$, and explicitly given turning
points $x_1 < x_2$ with $V(x_1) = E$ and $V(x_2) = E$.
\end{definition}

\begin{lstlisting}[caption={Barrier and turning point structures (Barrier/Definitions.lean).}]
/-- A potential barrier on [0, 1]. -/
structure Barrier where
  V : Real -> Real
  hV_cont : Continuous V
  E : Real
  h_left : V 0 < E
  h_peak : exists c, 0 <= c /\ c <= 1 /\ V c > E
  h_right : V 1 < E

/-- A specific barrier with given turning points. -/
structure SpecificBarrier where
  V : Real -> Real
  hV_cont : Continuous V
  E : Real
  x1 : Real
  x2 : Real
  h_x1_root : V x1 = E
  h_x2_root : V x2 = E
  h_order : x1 < x2
  h_barrier : forall x, x1 < x -> x < x2 -> V x >= E
\end{lstlisting}

\subsection{Rectangular Barrier}\label{sec:rectangular}

For a rectangular barrier $V(x) = V_0$ (constant) on $[x_1, x_2]$,
the turning points are part of the barrier definition. The WKB action
simplifies to
\begin{equation}\label{eq:rect-action}
  S = (x_2 - x_1)\,\sqrt{2m(V_0 - E)},
\end{equation}
and the tunneling amplitude is
$T = \exp\!\bigl(-(x_2 - x_1)\sqrt{2m(V_0 - E)}/\hbar\bigr)$.

\begin{theorem}[Rectangular barrier---BISH]\label{thm:rect-bish}
\leanok{}
For any $V_0, E, m, x_1, x_2, \hbar \in \RR$ with $0 < m$,
$E < V_0$, $x_1 < x_2$, and $\hbar > 0$, the tunneling amplitude
$T = \exp\!\bigl(-S/\hbar\bigr)$ exists as a computable real.
\end{theorem}

\begin{proof}
Definitional existence: \texttt{exact $\langle$\_, rfl$\rangle$}.
The action integral and exponential are defined by \Mathlib{}'s
\texttt{intervalIntegral} and \texttt{Real.exp}. No root-finding,
no limits, no omniscience.
\end{proof}

\subsection{Parabolic Barrier}\label{sec:parabolic}

For the inverted parabolic barrier
$V(x) = V_0(1 - x^2/a^2)$, the turning points at energy $E$ are
\begin{equation}\label{eq:parabolic-tp}
  x_1 = -a\sqrt{1 - E/V_0}, \qquad
  x_2 = \phantom{-}a\sqrt{1 - E/V_0}.
\end{equation}
These are algebraically computable from $V_0$, $a$, and $E$.

\begin{theorem}[Parabolic turning points]\label{thm:parabolic-tp}
\leanok{}
For $V_0 \ne 0$, $a \ne 0$, and $0 \le 1 - E/V_0$, the parabolic
barrier satisfies $V(x_1) = E$ and $V(x_2) = E$ at the algebraic
turning points~\eqref{eq:parabolic-tp}.
\end{theorem}

\begin{proof}
Expand:
$V(x_2) = V_0\bigl(1 - (a\sqrt{1-E/V_0})^2/a^2\bigr)
= V_0\bigl(1 - (1 - E/V_0)\bigr) = V_0 \cdot E/V_0 = E$.
In \Lean{}: \texttt{rw [mul\_pow, Real.sq\_sqrt hfrac]; field\_simp;
ring}. The left turning point follows identically using
\texttt{neg\_pow\_two}.
\end{proof}

\subsection{General Computability (Theorem~1)}\label{sec:thm1}

\begin{theorem}[BISH computability of specific barriers]
\label{thm:bish-computable} \leanok{}
For any continuous barrier $V$ with explicitly given turning points
$x_1 < x_2$ and any $\hbar > 0$, the WKB tunneling amplitude
$T = \exp(-S/\hbar)$ exists as a computable real, where
$S = \int_{x_1}^{x_2} \sqrt{2m(V(x) - E)}\,dx$.
\end{theorem}

\begin{proof}
Both the action integral and the exponential are definitionally
computed by \Mathlib{}: \texttt{exact $\langle$\_, rfl$\rangle$}.
The integral exists because the integrand is continuous on a compact
interval (\Mathlib{}'s constructive Riemann integration). The
exponential $\exp : \RR \to \RR$ is a total function. No custom
axioms appear in the proof.
\end{proof}

\begin{remark}[Logical status]\label{rem:bish-status}
\texttt{\#print axioms wkb\_action\_computable} shows only
\texttt{[propext, Classical.choice, Quot.sound]}. The
\texttt{Classical.choice} arises from \Mathlib{}'s infrastructure
for \texttt{Real.instField} and \texttt{intervalIntegral}, not from
any mathematical use of choice. No custom axioms
(\texttt{exact\_ivt\_iff\_llpo}, \texttt{bmc\_iff\_lpo}) appear.
\end{remark}


% ====================================================================
\section{Part~B: The Turning Point Problem Costs LLPO}
\label{sec:part-b}
% ====================================================================

This is the core new result: the first physical calibration of
$\LLPO$.

\subsection{The Turning Point Problem}\label{sec:tpp}

\begin{definition}[Turning Point Problem (TPP)]\label{def:tpp}
\leanok{}
For every barrier $B$ (continuous $V : \RR \to \RR$ with $V(0) < E$,
$V(c) > E$ for some $c \in [0,1]$, and $V(1) < E$), there exist
turning points $x_1, x_2 \in [0,1]$ with $V(x_1) = E$,
$V(x_2) = E$, and $x_1 < x_2$.
\end{definition}

\noindent
\textbf{Physical meaning:} given any continuous potential with a peak
above the energy, where exactly does the classically forbidden region
begin and end? The approximate answer ($|V(x) - E| < \varepsilon$)
is $\BISH$. The exact answer ($V(x) = E$) costs $\LLPO$.

\begin{lstlisting}[caption={The Turning Point Problem (Barrier/Definitions.lean).}]
/-- The Turning Point Problem: every barrier has turning points. -/
def TPP : Prop := forall (B : Barrier), Nonempty (TurningPoints B)
\end{lstlisting}

\subsection{From Turning Points to Roots: TPP $\Rightarrow$ ExactIVT}
\label{sec:tpp-to-ivt}

\begin{theorem}[TPP implies ExactIVT]\label{thm:tpp-ivt} \leanok{}
If the Turning Point Problem is solvable for all barriers, then the
exact Intermediate Value Theorem holds.
\end{theorem}

\begin{proof}
Given $f : \RR \to \RR$ continuous with $f(0) < 0$ and $f(1) > 0$,
we construct a barrier whose turning point gives a root of $f$.

Define the reflected potential:
\[
  V(x) = \begin{cases}
    f(2x)     & \text{if } x \le 1/2, \\
    f(2 - 2x) & \text{if } x > 1/2.
  \end{cases}
\]
Then $V(0) = f(0) < 0$, $V(1/2) = f(1) > 0$, and
$V(1) = f(0) < 0$. With $E = 0$, this defines a barrier (the peak
at $x = 1/2$ is above $E$). The function $V$ is continuous by
\Mathlib{}'s \texttt{Continuous.if\_le}, using the agreement
$f(2x) = f(2 - 2x)$ at $x = 1/2$ (both equal $f(1)$).

By TPP, there exist turning points with $V(x_1) = 0$. If
$x_1 \le 1/2$, then $f(2x_1) = 0$, giving a root at
$2x_1 \in [0,1]$. If $x_1 > 1/2$, then $f(2 - 2x_1) = 0$, giving
a root at $2 - 2x_1 \in [0,1]$. In either case, $f$ has a root.
\end{proof}

\subsection{From Roots to Turning Points: ExactIVT $\Rightarrow$ TPP}
\label{sec:ivt-to-tpp}

\begin{theorem}[ExactIVT implies TPP]\label{thm:ivt-tpp} \leanok{}
If the exact Intermediate Value Theorem holds, then the Turning Point
Problem is solvable.
\end{theorem}

\begin{proof}
Given a barrier $B$ with $V(0) < E$, $V(c) > E$ for some
$c \in [0,1]$, and $V(1) < E$, define $f(x) = V(x) - E$.

On the interval $[0, c]$: $f(0) < 0$ and $f(c) > 0$. By ExactIVT
(rescaled to $[0, c]$ via the affine substitution
$g(t) = f(c\cdot t)$), there exists $x_1 \in [0, c]$ with
$f(x_1) = 0$, i.e., $V(x_1) = E$.

On the interval $[c, 1]$: $f(c) > 0$ and $f(1) < 0$. Apply the
reversed-sign IVT to $-f$ (which has $-f(c) < 0$ and $-f(1) > 0$),
obtaining $x_2 \in [c, 1]$ with $f(x_2) = 0$, i.e., $V(x_2) = E$.

Since $x_1 \le c \le x_2$ and $f(c) > 0 \ne 0 = f(x_1)$, we have
$x_1 < c < x_2$ (if $x_1 = c$, then $f(c) = 0$, contradicting
$f(c) > 0$), so $x_1 < x_2$.

In \Lean{}: the rescaled IVT uses the helper lemma
\texttt{exact\_ivt\_on\_interval}, which composes ExactIVT with an
affine reparametrization.
\end{proof}

\begin{lstlisting}[caption={Rescaled IVT and the core equivalence (Barrier/General.lean, Calibration/PartB.lean).}]
/-- ExactIVT on [0,1] rescaled to any [a,b]. -/
theorem exact_ivt_on_interval (hIVT : ExactIVT)
    (a b : Real) (hab : a < b) (f : Real -> Real) (hf : Continuous f)
    (hfa : f a < 0) (hfb : f b > 0) :
    exists x, a <= x /\ x <= b /\ f x = 0 := by
  set g : Real -> Real := fun t => f (a + t * (b - a))
  -- ... (affine reparametrization, apply hIVT to g)

/-- Theorem 4: TPP <-> LLPO. -/
theorem turning_point_problem_iff_llpo : TPP <-> LLPO := by
  constructor
  . intro hTPP; rw [<- exact_ivt_iff_llpo]; exact ivt_of_tpp hTPP
  . intro hLLPO; exact tpp_of_ivt (exact_ivt_iff_llpo.mpr hLLPO)
\end{lstlisting}

\subsection{The Main Equivalence: TPP $\leftrightarrow$ LLPO}
\label{sec:main-equiv}

\begin{theorem}[Turning Point Problem $\leftrightarrow$ LLPO]
\label{thm:tpp-llpo} \leanok{}
Over $\BISH$, the Turning Point Problem is equivalent to $\LLPO$:
\[
  \mathrm{TPP} \;\longleftrightarrow\; \LLPO.
\]
\end{theorem}

\begin{proof}
Compose \Cref{thm:tpp-ivt,thm:ivt-tpp} with the known equivalence
$\mathrm{ExactIVT} \leftrightarrow \LLPO$
\citep{BR87,Bridges89,Ishihara90}:
\[
  \mathrm{TPP}
  \;\xleftrightarrow{\text{Thms~\ref{thm:tpp-ivt}, \ref{thm:ivt-tpp}}}\;
  \mathrm{ExactIVT}
  \;\xleftrightarrow{\text{axiom}}\;
  \LLPO.
\]
In \Lean{}: \texttt{turning\_point\_problem\_iff\_llpo} composes
\texttt{ivt\_of\_tpp} and \texttt{tpp\_of\_ivt} with the axiom
\texttt{exact\_ivt\_iff\_llpo}.
\end{proof}

\begin{remark}[Axiom certificate]\label{rem:llpo-cert}
\texttt{\#print axioms turning\_point\_problem\_iff\_llpo} shows
\texttt{[propext, Classical.choice, Quot.sound,
exact\_ivt\_iff\_llpo]}. Exactly one custom axiom:
\texttt{exact\_ivt\_iff\_llpo}. No \texttt{bmc\_iff\_lpo}. This
certifies that the turning point problem costs exactly $\LLPO$---not
$\LPO$, not $\WLPO$.
\end{remark}


% ====================================================================
\section{Part~C: The Semiclassical Limit Costs LPO}\label{sec:part-c}
% ====================================================================

\subsection{Why the $\hbar \to 0$ Limit Requires More}
\label{sec:limit-why}

The semiclassical assertion ``as $\hbar \to 0$, the WKB
approximation converges to the exact solution'' involves a completed
limit. Define $\hbar_n = 1/(n+1)$. The assertion that the sequence
$T(\hbar_n)$ converges to a definite limit $L$ is:
\[
  \forall \varepsilon > 0,\;\exists N_0,\;\forall N \ge N_0,\;
  |T(\hbar_N) - L| < \varepsilon.
\]
For a general barrier (after finding turning points via $\LLPO$),
asserting convergence of this bounded sequence is a Bounded Monotone
Convergence assertion, which costs $\LPO$.

\subsection{Full WKB $\leftrightarrow$ LPO (Theorem~5)}
\label{sec:full-wkb}

\begin{definition}[Full WKB General Barrier]\label{def:full-wkb}
\leanok{}
The full WKB assertion for a general barrier is the conjunction:
\[
  \mathrm{FullWKBGeneralBarrier} \;:=\; \mathrm{TPP} \;\wedge\; \BMC.
\]
\end{definition}

\begin{theorem}[Full WKB $\leftrightarrow$ LPO]\label{thm:full-wkb}
\leanok{}
\[
  \mathrm{FullWKBGeneralBarrier}
  \;\longleftrightarrow\; \LPO.
\]
\end{theorem}

\begin{proof}
\textbf{Forward} ($\mathrm{FullWKBGeneralBarrier} \Rightarrow \LPO$):
The $\BMC$ component directly gives $\LPO$ via
\texttt{bmc\_iff\_lpo}.

\smallskip\noindent
\textbf{Reverse} ($\LPO \Rightarrow \mathrm{FullWKBGeneralBarrier}$):
\begin{itemize}
  \item $\LPO \Rightarrow \LLPO$ (hierarchy, \Cref{sec:hierarchy-bg})
    $\Rightarrow \mathrm{TPP}$ (\Cref{thm:tpp-llpo}).
  \item $\LPO \Rightarrow \BMC$ (\texttt{bmc\_iff\_lpo}).
\end{itemize}
The conjunction $\mathrm{TPP} \wedge \BMC$ follows.

\smallskip\noindent
In \Lean{}: \texttt{full\_wkb\_iff\_lpo :=
$\langle$full\_wkb\_implies\_lpo, lpo\_implies\_full\_wkb$\rangle$}.
\end{proof}

\subsection{Dispensability (Theorem~6)}\label{sec:dispensability}

\begin{theorem}[Dispensability]\label{thm:dispensable} \leanok{}
For any specific barrier with given turning points and any
$\hbar > 0$, the tunneling amplitude is $\BISH$-computable. Neither
$\LLPO$ (for turning points) nor $\LPO$ (for the semiclassical
limit) is needed.
\end{theorem}

\begin{proof}
When turning points are given, no root-finding ($\LLPO$) is needed.
When $\hbar > 0$ is fixed, no limit ($\LPO$) is taken. The
computation is algebraic: $T = \exp(-S/\hbar)$ where $S$ is a
definite integral of a continuous function on a compact interval.
Proof: \texttt{exact $\langle$\_, rfl$\rangle$}.
\end{proof}

\begin{remark}[Physical interpretation]\label{rem:dispensable-phys}
Every experiment measures a tunneling rate at a specific barrier with
a specific $\hbar$. The measurement is $\BISH$. The non-constructive
content enters only in the \emph{universality assertion} (``this
works for all barriers'') and the \emph{classical limit} (``quantum
mechanics reduces to classical mechanics'').
\end{remark}


% ====================================================================
\section{Updated Calibration Table}\label{sec:calibration}
% ====================================================================

The calibration table for the constructive reverse mathematics
series, updated with Paper~19:

\begin{center}
\small
\begin{tabular}{@{}clllc@{}}
\toprule
\textbf{Paper} & \textbf{Physical System} &
  \textbf{Observable / Assertion} & \textbf{CRM Level} &
  \textbf{Key Axiom} \\
\midrule
2  & Bidual gap ($\ell^1$)
   & Gap witness $J - \kappa$
   & $\equiv \WLPO$ & WLPO \\
6  & Heisenberg uncertainty
   & $\Delta A \cdot \Delta B \ge \tfrac{1}{2}|\langle[A,B]\rangle|$
   & $\BISH$ & None \\
7  & Reflexive Banach ($S_1(H)$)
   & Non-reflexivity witness
   & $\equiv \WLPO$ & WLPO \\
8  & 1D Ising model
   & Thermodynamic limit $f_\infty$
   & $\equiv \LPO$ & BMC \\
9  & Hydrogen spectrum
   & Finite eigenvalue bounds
   & $\BISH$ & None \\
11 & Bell / CHSH inequality
   & Tsirelson bound $2\sqrt{2}$
   & $\BISH$ & None \\
13 & Schwarzschild interior
   & Geodesic incompleteness
   & $\equiv \LPO$ & BMC \\
14 & Quantum decoherence
   & Exact collapse $c(N) \to 0$
   & $\equiv \LPO$ & BMC \\
15 & Noether conservation
   & Global energy $E = \lim E_N$
   & $\equiv \LPO$ & BMC \\
16 & Thermodynamic entropy
   & Infinite-volume entropy
   & $\equiv \LPO$ & BMC \\
17 & Spin chain entanglement
   & Entanglement entropy limit
   & $\equiv \LPO$ & BMC \\
18 & Hawking radiation
   & Thermal spectrum limit
   & $\equiv \LPO$ & BMC \\
\rowcolor{yellow!20}
\textbf{19} & \textbf{WKB tunneling}
   & \textbf{Turning points (TPP)}
   & $\equiv \LLPO$ & \textbf{IVT} \\
\rowcolor{yellow!20}
\textbf{19} & \textbf{WKB tunneling}
   & \textbf{Full semiclassical}
   & $\equiv \LPO$ & \textbf{IVT+BMC} \\
\bottomrule
\end{tabular}
\end{center}

\noindent
Paper~19 contributes the \textbf{first $\LLPO$ entry} in the
calibration table. The constructive hierarchy now has physical
instantiations at every level:
\begin{itemize}
  \item $\BISH$: finite computations (specific barriers,
    eigenvalue bounds, Tsirelson bound, local conservation).
  \item $\LLPO$: exact root-finding (turning points of a general
    barrier).  \textbf{NEW.}
  \item $\WLPO$: bidual gap and non-reflexivity witnesses.
  \item $\LPO$: completed infinite limits (thermodynamic, geodesic,
    decoherence, conservation, semiclassical).
\end{itemize}

\noindent
The pattern is consistent: $\BISH$ handles finite data, $\LLPO$
handles exact root-finding, $\WLPO$ handles bidual/reflexivity
questions, and $\LPO$ handles limits and convergence assertions.


% ====================================================================
\section{Lean~4 Formalization}\label{sec:lean}
% ====================================================================

\subsection{Module Structure}\label{sec:modules}

The formalization consists of 15~files organized in four directories:

\begin{verbatim}
Papers/P19_WKBTunneling/
+-- Basic/
|   +-- LLPO.lean              107 lines
|   +-- IVT.lean                64 lines
|   +-- Hierarchy.lean         106 lines
+-- Barrier/
|   +-- Definitions.lean        76 lines
|   +-- Rectangular.lean        50 lines
|   +-- Parabolic.lean          77 lines
|   +-- General.lean            50 lines
+-- WKB/
|   +-- Action.lean             49 lines
|   +-- Amplitude.lean          36 lines
|   +-- Limit.lean              52 lines
+-- Calibration/
|   +-- PartA.lean              64 lines
|   +-- PartB.lean             179 lines
|   +-- PartC.lean              61 lines
|   +-- Dispensability.lean     52 lines
+-- Main.lean                   58 lines
                         Total: 1,081 lines
\end{verbatim}

\noindent
Dependency graph:
\begin{verbatim}
LLPO <-- IVT <-- Definitions <-- General <-- PartB
  |        |          |              |          |
  +-- Hierarchy   Rectangular   PartA <--------+
  |              Parabolic                  PartC
  +-- Limit <-- Amplitude <-- Action    Dispensability
                                            |
                                          Main
\end{verbatim}

\subsection{Design Decisions}\label{sec:design}

\paragraph{Plain $\RR$ with explicit bounds.}
The barrier is defined as a function $V : \RR \to \RR$ with a
\texttt{Continuous} hypothesis, rather than using bundled continuous
functions $C(\mathrm{Set.Icc}\;0\;1, \RR)$. This avoids subtype
coercion boilerplate throughout the formalization.

\paragraph{Two interface axioms.}
Following the precedent of Papers~8 and~14, we axiomatize two known
CRM equivalences:
\begin{itemize}
  \item \texttt{exact\_ivt\_iff\_llpo : ExactIVT $\leftrightarrow$
    LLPO} \citep{BR87,Bridges89,Ishihara90}.
  \item \texttt{bmc\_iff\_lpo : BMC $\leftrightarrow$ LPO}
    \citep{BV06}.
\end{itemize}
Every other result is proved from definitions, sorry-free.

\paragraph{Self-contained bundle.}
Paper~19 is a standalone Lake package. It re-declares $\LPO$,
$\WLPO$, $\LLPO$, and $\BMC$ locally. The hierarchy proofs
($\LPO \Rightarrow \WLPO \Rightarrow \LLPO$) are proved from first
principles with no custom axioms.

\paragraph{Barrier on $[0,1]$.}
Normalizing the domain to $[0,1]$ matches the standard formulation
of ExactIVT, simplifying the interface between barrier structures and
the IVT axiom.

\subsection{Axiom Audit}\label{sec:axiom-audit}

\begin{center}
\begin{tabular}{@{}llll@{}}
\toprule
\textbf{Theorem} & \textbf{Custom Axioms} &
  \textbf{Infrastructure} & \textbf{Tier} \\
\midrule
Thm~1 (\texttt{wkb\_action\_computable})
  & None
  & propext, Classical.choice, Quot.sound
  & $\BISH$ \\
Thm~4 (\texttt{turning\_point\_problem\_iff\_llpo})
  & \texttt{exact\_ivt\_iff\_llpo}
  & propext, Classical.choice, Quot.sound
  & $\LLPO$ \\
Thm~5 (\texttt{full\_wkb\_iff\_lpo})
  & \texttt{exact\_ivt\_iff\_llpo}, \texttt{bmc\_iff\_lpo}
  & propext, Classical.choice, Quot.sound
  & $\LPO$ \\
Thm~6 (\texttt{specific\_barrier\_dispensable})
  & None
  & propext, Classical.choice, Quot.sound
  & $\BISH$ \\
\bottomrule
\end{tabular}
\end{center}

\begin{lstlisting}[caption={Axiom audit (Main.lean, selected).}]
-- Part A (BISH):
#print axioms wkb_action_computable
-- [propext, Classical.choice, Quot.sound]

-- Part B (LLPO):
#print axioms turning_point_problem_iff_llpo
-- [propext, Classical.choice, Quot.sound, exact_ivt_iff_llpo]

-- Part C (LPO):
#print axioms full_wkb_iff_lpo
-- [propext, Classical.choice, Quot.sound,
--  exact_ivt_iff_llpo, bmc_iff_lpo]

-- Dispensability (BISH):
#print axioms specific_barrier_dispensable
-- [propext, Classical.choice, Quot.sound]

-- Hierarchy (no custom axioms):
#print axioms lpo_implies_wlpo
#print axioms wlpo_implies_llpo
-- [propext, Classical.choice, Quot.sound]
\end{lstlisting}

\subsection{CRM Compliance}\label{sec:crm-compliance}

The three-tier structure is confirmed by machine:
\begin{itemize}
  \item Part~A theorems have \textbf{no custom axioms}---pure $\BISH$.
  \item Part~B theorems depend on \textbf{exactly one} custom axiom
    (\texttt{exact\_ivt\_iff\_llpo})---$\LLPO$ level.
  \item Part~C theorems depend on \textbf{both} custom axioms---$\LPO$
    level.
  \item Hierarchy proofs ($\LPO \Rightarrow \WLPO \Rightarrow \LLPO$)
    have \textbf{no custom axioms}---sorry-free, pure $\BISH$.
  \item \texttt{Classical.choice} in all results is a \Mathlib{}
    infrastructure artifact from \texttt{Real.instField},
    \texttt{Continuous}, and \texttt{intervalIntegral}. The
    mathematical content of these proofs is constructive.
\end{itemize}


% ====================================================================
\section{Discussion}\label{sec:discussion}
% ====================================================================

\subsection{Quantum Mechanics Is Logically Cheaper Than Classical
Mechanics}\label{sec:qm-cheap}

The three-tier decomposition reveals a striking fact: the
non-constructive content of quantum tunneling does not reside in the
quantum mechanics itself. For any specific barrier with known
parameters, the tunneling rate is $\BISH$-computable---a finite
computation with no omniscience. The non-constructivity enters in two
places that are both \emph{classical}:
\begin{enumerate}
  \item \textbf{Classical geometry} ($\LLPO$): the exact boundary of
    the classically forbidden region. Where does $V(x) = E$? This
    is a root-finding problem---the constructive content of the IVT.
  \item \textbf{Classical limit} ($\LPO$): the assertion that quantum
    mechanics reduces to classical mechanics as $\hbar \to 0$. This
    is a completed limit.
\end{enumerate}

\noindent
The quantum mechanics---solving the Schr\"odinger equation for a
given potential---is the constructive part. The classical
aspects---locating turning points and taking limits---are where
omniscience enters. This is consistent with the general pattern in
the programme: operational predictions are $\BISH$; idealized global
assertions cost omniscience.

\subsection{LLPO as ``Knowing Where''}\label{sec:llpo-where}

The characterization $\LLPO \leftrightarrow \mathrm{ExactIVT}$
gives $\LLPO$ a vivid physical interpretation: $\LLPO$ is the cost
of \emph{exact location}. The approximate IVT tells us that a root
exists within $\varepsilon$ of some point---this is $\BISH$. The
exact IVT tells us the root exists at a precise point---this costs
$\LLPO$.

Physically: we can always approximate the turning point to any
desired precision ($\BISH$). Asserting that the turning point
\emph{exists} as a definite real number requires $\LLPO$.

This is distinct from the $\WLPO$ content of the series (bidual
gaps, non-reflexivity) and from the $\LPO$ content (completed
limits). $\LLPO$ occupies a natural niche: the cost of exact
root-finding for continuous functions.

\subsection{The Three-Tier Pattern}\label{sec:three-tier}

The decomposition Specific $\to$ General $\to$ Asymptotic maps
cleanly onto the hierarchy:
\begin{center}
\begin{tabular}{@{}lll@{}}
\toprule
\textbf{Level} & \textbf{Information type} & \textbf{CRM cost} \\
\midrule
Specific barrier & Given data & $\BISH$ \\
General barrier  & Found data (roots) & $\LLPO$ \\
Semiclassical limit & Limit of found data & $\LPO$ \\
\bottomrule
\end{tabular}
\end{center}
Each tier adds exactly one level of the hierarchy. This mirrors a
general pattern: given data is constructive, found data costs
root-finding ($\LLPO$), and limits of found data cost convergence
($\LPO$).

\subsection{Limitations}\label{sec:limitations}

\begin{enumerate}
  \item \textbf{One dimension only.} The WKB approximation in higher
    dimensions involves caustics, Maslov indices, and multi-dimensional
    turning surfaces. The logical cost of these structures may differ
    from the one-dimensional case.

  \item \textbf{Domain normalized to $[0,1]$.} Physical barriers
    extend to $\pm\infty$. The normalization is a mathematical
    convenience that does not affect the CRM calibration (the IVT
    equivalence holds on any compact interval).

  \item \textbf{Classical.choice in \Mathlib{}.} The appearance of
    \texttt{Classical.choice} in $\BISH$ results is a \Mathlib{}
    infrastructure artifact, not mathematical content. This is the
    same situation as in all previous papers in the series.

  \item \textbf{Axiomatized equivalences.} The equivalences
    $\mathrm{ExactIVT} \leftrightarrow \LLPO$ and
    $\BMC \leftrightarrow \LPO$ are axiomatized, not proved from
    first principles. The proofs are well-established in the
    constructive analysis literature
    \citep{BR87,Bridges89,Ishihara90,BV06} but not yet formalized
    in \Mathlib{}.

  \item \textbf{No physical units.} The formalization works with
    dimensionless quantities. A fully physical treatment would include
    dimensional analysis, but this does not affect the logical
    structure.
\end{enumerate}


% ====================================================================
\section{Conclusion}\label{sec:conclusion}
% ====================================================================

Quantum tunneling through a potential barrier provides the first
physical calibration of the Lesser Limited Principle of Omniscience
($\LLPO$). The three-tier decomposition---$\BISH$ for specific
barriers, $\LLPO$ for turning point identification, $\LPO$ for the
semiclassical limit---shows that the logical cost of tunneling
stratifies cleanly by information type: given data ($\BISH$), found
data ($\LLPO$), and asymptotic data ($\LPO$).

The calibration table now covers $\BISH$, $\LLPO$, $\WLPO$, and
$\LPO$ with physical instantiations from six domains: quantum
mechanics (tunneling, decoherence, Bell inequalities, Heisenberg
uncertainty), statistical mechanics (Ising model), general relativity
(Schwarzschild geodesics), and structural laws (Noether conservation).
Every level of the constructive hierarchy
$\BISH < \LLPO < \WLPO < \LPO$ now has at least one physical
calibration.


% ====================================================================
\section*{AI-Assisted Methodology}\label{sec:ai}
% ====================================================================

This formalization was developed using \textbf{Claude Opus~4.6}
(Anthropic, 2026) via the \textbf{Claude Code} command-line
interface, following the same human--AI workflow as previous papers
in the series~\cite{Lee26-P2,Lee26-P7,Lee26-P8,Lee26-P15}.

The author is a medical professional, not a domain expert in
constructive mathematics or mathematical physics. The mathematical
content of this paper was developed with extensive AI assistance.
The human author specified the research direction and high-level
goals, reviewed all mathematical claims for plausibility, and
directed the formalization strategy. Claude Opus~4.6 explored the
\Mathlib{} codebase, generated \Lean{} proof terms, handled
debugging, and assisted with paper writing. Final verification
was by \texttt{lake build} (0~errors, 0~warnings, 0~sorries).

\begin{table}[h]
\centering
\begin{tabular}{@{}lcc@{}}
\toprule
\textbf{Component} & \textbf{Human} &
  \textbf{AI (Claude Opus 4.6)} \\
\midrule
Research question          & \checkmark & \\
Physical setup (WKB)       & \checkmark & \\
CRM calibration strategy   & \checkmark & \\
\Lean{} implementation     & & \checkmark \\
Proof strategies           & collaborative & collaborative \\
\LaTeX{} writeup           & & \checkmark \\
Review and editing         & \checkmark & \\
\bottomrule
\end{tabular}
\caption{Division of labor between human and AI.}
\label{tab:division}
\end{table}


% ====================================================================
\section*{Reproducibility}
% ====================================================================

\begin{mdframed}[backgroundcolor=gray!10]
\textbf{Reproducibility Box}
\begin{itemize}
\item \textbf{Repository}:
  \url{https://github.com/paul-c-k-lee/FoundationRelativity}
\item \textbf{Path}: \texttt{paper~19/P19\_WKBTunneling/}
\item \textbf{Build}: \texttt{lake exe cache get \&\& lake build}
  (2{,}713 jobs, 0~errors, 0~sorry)
\item \textbf{Lean toolchain}:
  \texttt{leanprover/lean4:v4.28.0-rc1}
\item \textbf{Interface axioms}:
  \texttt{exact\_ivt\_iff\_llpo}
  (IVT $\leftrightarrow$ LLPO; \cite{BR87,Bridges89,Ishihara90}),
  \texttt{bmc\_iff\_lpo}
  (BMC $\leftrightarrow$ LPO; \cite{BV06})
\item \textbf{Axiom audit}: \texttt{Main.lean}
\item \textbf{Axiom profile (Theorem~1)}:
  \texttt{[propext, Classical.choice, Quot.sound]}
\item \textbf{Axiom profile (Theorem~4)}:
  \texttt{[propext, Classical.choice, Quot.sound,
  exact\_ivt\_iff\_llpo]}
\item \textbf{Axiom profile (Theorem~5)}:
  \texttt{[propext, Classical.choice, Quot.sound,
  exact\_ivt\_iff\_llpo, bmc\_iff\_lpo]}
\item \textbf{Axiom profile (Theorem~6)}:
  \texttt{[propext, Classical.choice, Quot.sound]}
\item \textbf{Total}: 15~files, 1{,}081~lines, 0~sorry
\item \textbf{Zenodo DOI}:
  \href{https://doi.org/10.5281/zenodo.18602596}{10.5281/zenodo.18602596}
\end{itemize}
\end{mdframed}


% ====================================================================
\section*{Acknowledgments}
% ====================================================================

The \Lean{} formalization was developed using Claude Opus~4.6
(Anthropic, 2026) via the Claude Code CLI tool. We thank the
\Mathlib{} community for maintaining the comprehensive library
of formalized mathematics that made this work possible.


% ====================================================================
% Bibliography
% ====================================================================
\bibliographystyle{plainnat}
\bibliography{paper19_references}

\end{document}

\documentclass[11pt]{article}

% ------------------------------------------------------------
% Standard LaTeX packages
% ------------------------------------------------------------
\usepackage[margin=1in]{geometry}
%\usepackage{lmodern}
\usepackage{amsmath,amssymb,mathtools}
\usepackage{amsthm}
\usepackage[american]{babel}
\usepackage{stmaryrd}
\usepackage{enumitem}
\usepackage{booktabs}
\usepackage{array}
\usepackage{listings}
\usepackage[x11names,table]{xcolor}
\usepackage{mdframed}
\usepackage{url}
\usepackage[colorlinks=true,linkcolor=blue,citecolor=blue,urlcolor=blue]{hyperref}

% ---------- Theorem environments ----------
\theoremstyle{plain}
\newtheorem{theorem}{Theorem}[section]
\newtheorem{proposition}[theorem]{Proposition}
\newtheorem{lemma}[theorem]{Lemma}
\newtheorem{corollary}[theorem]{Corollary}

\theoremstyle{definition}
\newtheorem{definition}[theorem]{Definition}
\newtheorem{axiombox}[theorem]{Axiom}

\theoremstyle{remark}
\newtheorem{remark}[theorem]{Remark}

% ---------- Lean repo link ----------
\newcommand{\leanRepo}{\url{https://doi.org/10.5281/zenodo.18748460}}

% ---------- Notation ----------
\newcommand{\N}{\mathbb{N}}
\newcommand{\Z}{\mathbb{Z}}
\newcommand{\Q}{\mathbb{Q}}
\newcommand{\R}{\mathbb{R}}
\newcommand{\C}{\mathbb{C}}
\newcommand{\F}{\mathbb{F}}
\newcommand{\Fp}{\mathbb{F}_p}
\newcommand{\Qbar}{\overline{\Q}}
\newcommand{\Qp}{\Q_p}
\newcommand{\Qell}{\Q_\ell}
\newcommand{\A}{\mathbb{A}}
\newcommand{\calO}{\mathcal{O}}
\newcommand{\fm}{\mathfrak{m}}
\newcommand{\fp}{\mathfrak{p}}
\newcommand{\Gal}{\mathrm{Gal}}
\newcommand{\GL}{\mathrm{GL}}
\newcommand{\Hom}{\mathrm{Hom}}
\newcommand{\Ext}{\mathrm{Ext}}
\newcommand{\Frob}{\mathrm{Frob}}
\newcommand{\Sel}{\mathrm{Sel}}
\newcommand{\edim}{\mathrm{edim}}
\newcommand{\CRM}{\mathrm{CRM}}
\newcommand{\tr}{\mathrm{tr}}

\newcommand{\BISH}{\mathsf{BISH}}
\newcommand{\LPO}{\mathsf{LPO}}
\newcommand{\WLPO}{\mathsf{WLPO}}
\newcommand{\LLPO}{\mathsf{LLPO}}
\newcommand{\MP}{\mathsf{MP}}
\newcommand{\FT}{\mathsf{FT}}
\newcommand{\WKL}{\mathsf{WKL}_0}
\newcommand{\CLASS}{\mathsf{CLASS}}

% ---------- Code listing style for Lean ----------
\definecolor{codegreen}{rgb}{0,0.6,0}
\definecolor{codegray}{rgb}{0.5,0.5,0.5}
\definecolor{codepurple}{rgb}{0.58,0,0.82}
\definecolor{backcolour}{rgb}{0.95,0.95,0.92}

\lstdefinelanguage{Lean}{
  keywords={theorem, lemma, def, definition, axiom, structure, class, instance,
            by, exact, intro, intros, apply, refine, constructor, use, obtain,
            have, show, from, fun, assume, let, in, if, then, else,
            match, with, end, namespace, section, variable, variables,
            example, begin, sorry, admit, noncomputable, classical,
            import, open, export, private, protected, mutual, meta,
            do, for, while, return, try, catch, finally,
            Type, Prop, Sort, Type*, forall, exists, where, extends,
            set, push_neg, rw, simp, omega, nlinarith, linarith,
            ext, rfl, congr, fin_cases, haveI, letI, attribute, inductive,
            deriving, opaque, decide},
  sensitive=true,
  morecomment=[l]{--},
  morecomment=[s]{/-}{-/},
  morestring=[b]",
  literate=
    {α}{{$\alpha$}}1 {β}{{$\beta$}}1 {γ}{{$\gamma$}}1
    {δ}{{$\delta$}}1 {ε}{{$\varepsilon$}}1 {ζ}{{$\zeta$}}1
    {η}{{$\eta$}}1 {θ}{{$\theta$}}1 {ι}{{$\iota$}}1
    {κ}{{$\kappa$}}1 {λ}{{$\lambda$}}1 {μ}{{$\mu$}}1
    {ν}{{$\nu$}}1 {ξ}{{$\xi$}}1 {π}{{$\pi$}}1
    {ρ}{{$\rho$}}1 {σ}{{$\sigma$}}1 {τ}{{$\tau$}}1
    {φ}{{$\varphi$}}1 {χ}{{$\chi$}}1 {ψ}{{$\psi$}}1
    {ω}{{$\omega$}}1 {Γ}{{$\Gamma$}}1 {Δ}{{$\Delta$}}1
    {Θ}{{$\Theta$}}1 {Λ}{{$\Lambda$}}1 {Σ}{{$\Sigma$}}1
    {Φ}{{$\Phi$}}1 {Ψ}{{$\Psi$}}1 {Ω}{{$\Omega$}}1
    {→}{{$\rightarrow$}}1 {←}{{$\leftarrow$}}1 {↔}{{$\leftrightarrow$}}1
    {⇒}{{$\Rightarrow$}}1 {⇐}{{$\Leftarrow$}}1 {⇔}{{$\Leftrightarrow$}}1
    {∀}{{$\forall$}}1 {∃}{{$\exists$}}1 {∈}{{$\in$}}1
    {∉}{{$\notin$}}1 {⊆}{{$\subseteq$}}1 {⊂}{{$\subset$}}1
    {∪}{{$\cup$}}1 {∩}{{$\cap$}}1 {≤}{{$\leq$}}1
    {≥}{{$\geq$}}1 {≠}{{$\neq$}}1 {≈}{{$\approx$}}1 {≃}{{$\simeq$}}1
    {≡}{{$\equiv$}}1 {∧}{{$\land$}}1 {∨}{{$\lor$}}1
    {¬}{{$\neg$}}1 {ℕ}{{$\mathbb{N}$}}1 {ℝ}{{$\mathbb{R}$}}1
    {ℂ}{{$\mathbb{C}$}}1 {ℤ}{{$\mathbb{Z}$}}1 {ℓ}{{$\ell$}}1
    {·}{{$\cdot$}}1 {∑}{{$\sum$}}1 {∏}{{$\prod$}}1
    {∅}{{$\emptyset$}}1 {∞}{{$\infty$}}1 {∂}{{$\partial$}}1
    {⟨}{{$\langle$}}1 {⟩}{{$\rangle$}}1 {…}{{$\ldots$}}1
    {₀}{{$_0$}}1 {₁}{{$_1$}}1 {₂}{{$_2$}}1 {⧸}{{$/$}}1 {‖}{{$\|$}}1
    {•}{{$\cdot$}}1 {⁻¹}{{$^{-1}$}}1 {⋆}{{$\star$}}1
    {∘}{{$\circ$}}1
}

\lstdefinestyle{leanstyle}{
    language=Lean,
    backgroundcolor=\color{backcolour},
    commentstyle=\color{codegreen},
    keywordstyle=\color{blue},
    stringstyle=\color{codepurple},
    basicstyle=\ttfamily\footnotesize,
    breakatwhitespace=false,
    breaklines=true,
    captionpos=b,
    keepspaces=true,
    numbers=left,
    numbersep=5pt,
    showspaces=false,
    showstringspaces=false,
    showtabs=false,
    tabsize=2,
    numberstyle=\tiny\color{codegray}
}

\lstset{style=leanstyle}

% ---------- Title and author ----------
\title{Fermat's Last Theorem is $\BISH$:\\
  A Constructive Reverse Mathematics Audit\\[6pt]
  {\large (Paper~68, Constructive Reverse Mathematics Series)}}

\author{Paul Chun-Kit Lee\thanks{Lean 4 formalization available at \leanRepo.} \\
New York University \\
\texttt{dr.paul.c.lee@gmail.com}}

\date{February 2026}

\begin{document}
\maketitle

% ============================================================
\begin{abstract}
% ============================================================

We perform a stage-by-stage constructive reverse mathematics
audit of Wiles's proof of the modularity of semistable elliptic
curves over~$\Q$, and hence of Fermat's Last Theorem.  The proof
decomposes into five stages: residual modularity
(Langlands--Tunnell), deformation ring construction, Hecke algebra
construction, the numerical criterion, and Taylor--Wiles patching.

Our principal finding is an \emph{asymmetry}: Stages~2--5 are
fully constructive ($\BISH$), while Stage~1 requires $\WLPO$
(Weak Limited Principle of Omniscience).  The non-constructive
content of Wiles's proof is concentrated entirely at its entry
point---the analytic theory of weight~1 automorphic forms---and
the Taylor--Wiles engine contributes zero logical cost.

We then show that Stage~1 can be \emph{bypassed entirely}.
The 21st-century proof route (Kisin 2009, Khare--Wintenberger
2009), which replaces Wiles's residual prime $p = 3$ with
$p = 2$, reduces the base case to dihedral modularity
($\GL_2(\F_2) \cong S_3$), where Hecke's algebraic theta series
provide a $\BISH$ construction.  An equivalent bypass via
potential modularity (Taylor, Buzzard--Taylor) restricts
to a totally real field where the representation becomes dihedral,
then uses weight~2 base change (also $\BISH$).
The overall classification of Fermat's Last Theorem is therefore
$\BISH$.  The $\WLPO$ was an artefact of Wiles's
choice of residual prime, not a property of the theorem.

Two post-Wiles developments drive the Stage~5 classification:
Brochard's proof of de~Smit's conjecture (2017), which eliminates
the infinite inverse limit, and unconditional effective Chebotarev
bounds (Lagarias--Montgomery--Odlyzko 1979, Ahn--Kwon 2019), which
make the Taylor--Wiles prime search a bounded computation.  Neither
result was motivated by constructive foundations; the community
unknowingly eliminated the Fan Theorem from the proof of~FLT
over twenty-two years.

The Lean~4 verification (493 lines across three files,
zero \texttt{sorry}) formalizes the logical
assembly: deep theorems are axiomatized and flagged; the
conditional classification is machine-checked.

\end{abstract}

\tableofcontents

% ============================================================
\section{Introduction}\label{sec:intro}
% ============================================================

\subsection{Main results}

What is the logical cost of proving Fermat's Last Theorem?
The question is not about computational complexity but about logical
\emph{principles}: which axioms beyond intuitionistic logic are
needed?  This paper answers the question for Wiles's proof
\cite{Wiles1995}, as refined by Taylor--Wiles \cite{TaylorWiles1995},
Diamond \cite{Diamond1997}, and Brochard \cite{Brochard2017}.
We establish four results:

\begin{description}[leftmargin=2em]
\item[Theorem A] (Stage~5 is $\BISH$).
The Taylor--Wiles patching argument, in the Brochard formulation, is
a $\BISH$-decidable finite computation.  Brochard's finite-level
criterion \cite{Brochard2017} eliminates the Fan Theorem;
effective Chebotarev bounds \cite{LMO1979, AhnKwon2019} eliminate
Markov's Principle.  The logical cost descends from
$\MP + \FT$ (1995) to $\BISH$ (2017/2019).

\item[Theorem B] (Stages~2--4 are $\BISH$).
The deformation ring (Schlessinger, Fontaine--Laffaille),
Hecke algebra (finite arithmetic), numerical criterion (subsumed
by patching), and CM base case (Rubin's Euler system with
effective Chebotarev) are all $\BISH$.

\item[Theorem C] (Stage~1 requires $\WLPO$).
The Langlands--Tunnell theorem requires
at least~$\WLPO$: the Arthur--Selberg trace formula involves
spectral decomposition of $L^2$-spaces (eigenvalue isolation:
$\WLPO$), orbital integral matching (real equality: $\WLPO$),
and the converse theorem for $\GL_3$ (Phragm\'en--Lindel\"of
condition).

\item[Theorem D] (Asymmetry Theorem).
The overall classification of Wiles's proof is $\BISH + \WLPO$.
The $\WLPO$ is localized entirely in Stage~1.  Removing Stage~1
drops the classification to~$\BISH$.

\item[Theorem E] (The Dihedral Bypass).
Stage~1 can be eliminated from the proof of Fermat's Last Theorem.
The 21st-century proof route replaces the residual prime $p = 3$
(projective image~$S_4$, requiring the trace formula) with $p = 2$
(projective image~$S_3 = D_3$, requiring only Hecke's algebraic
theta series).  The resulting proof is $\BISH$ throughout.
Fermat's Last Theorem is~$\BISH$.
\end{description}

\subsection{Constructive Reverse Mathematics: a brief primer}

$\CRM$ calibrates mathematical statements against logical principles
of increasing strength within Bishop-style constructive mathematics
($\BISH$).  The hierarchy relevant to this paper is:
\[
  \BISH \;\subset\; \BISH + \MP \;\subset\; \BISH + \LLPO \;\subset\;
  \BISH + \WLPO \;\subset\; \BISH + \LPO \;\subset\; \CLASS.
\]
Here $\WLPO$ (Weak Limited Principle of Omniscience) states that
every binary sequence is identically zero or not; equivalently,
$\forall x \in \R,\; x = 0 \lor \lnot(x = 0)$.  $\MP$ (Markov's
Principle) states that a binary sequence that is not all zeros
contains a~$1$; it converts $\lnot\lnot\exists$ to $\exists$.
$\FT$ (Fan Theorem) is the constructive form of K\"onig's Lemma.
For a thorough treatment of $\CRM$, see Bridges--Richman
\cite{BridgesRichman1987}; for the broader program of which this
paper is part, see Papers~1--67 of this series and the atlas
survey~\cite{Paper50}.

\subsection{Current state of the art}

McLarty \cite{McLarty2010} asked what \emph{set-theoretic} strength
FLT requires, showing that Grothendieck universes are eliminable.
Our question is orthogonal: we ask what \emph{constructive}
principles the proof requires.  McLarty's analysis concerns the
ambient set theory; ours concerns the logical content of the
mathematical arguments.

No prior work has applied constructive reverse mathematics to the
logical structure of Wiles's proof or the Taylor--Wiles method.
The constructive calibration is novel.

\subsection{Position in the atlas}

The present paper is part of the Constructive Reverse Mathematics
program (Papers~1--67).  Paper~67 \cite{Paper67} synthesizes the
arithmetic geometry phase (Papers~45--66); Paper~40 \cite{Paper40}
covers the physics phase; Paper~50 \cite{Paper50} provides the
atlas framework.  Paper~59 \cite{Paper59} classified the $p$-adic
comparison (Fontaine--Laffaille) as~$\BISH$, directly supporting
Stage~2.

Paper~68 opens a new direction: classifying \emph{proof methods}
rather than theorems.  The de-omniscientizing descent pattern
identified in Paper~50 for the five great conjectures of arithmetic
geometry reappears here in the \emph{evolution} of the Taylor--Wiles
method over 1995--2017.


% ============================================================
\section{Preliminaries}\label{sec:prelim}
% ============================================================

\begin{definition}[Weak Limited Principle of Omniscience]\label{def:wlpo}
$\WLPO$: For every binary sequence $\alpha : \N \to \{0,1\}$,
either $\forall n,\;\alpha(n) = 0$ or
$\lnot(\forall n,\;\alpha(n) = 0)$.
Equivalently, for every $x \in \R$: $x = 0 \lor \lnot(x = 0)$.
\end{definition}

\begin{definition}[Markov's Principle]\label{def:mp}
$\MP$ asserts: for every binary sequence
$\alpha\colon \N\to\{0,1\}$,
\[
  \lnot(\forall n,\;\alpha(n)=0) \;\implies\;
  \exists n,\;\alpha(n)=1.
\]
This converts double-negated existence to constructive existence.
\end{definition}

\begin{definition}[Fan Theorem]\label{def:ft}
$\FT$: Every decidable bar of the full binary fan (Cantor space) is
uniform.  Equivalently, Cantor space $2^\N$ is compact.
Over~$\BISH$, $\FT$ is equivalent to~$\WKL$ (every infinite binary
tree has an infinite path).
\end{definition}

\begin{definition}[The five stages of Wiles's proof]\label{def:stages}
Let $E/\Q$ be a semistable elliptic curve with conductor~$N$, and
let $p = 3$.  Let $\rho : G_\Q \to \GL_2(\Z_3)$ be the $3$-adic
Galois representation on the Tate module $T_3(E)$, and
$\bar{\rho} : G_\Q \to \GL_2(\F_3)$ its reduction.  Following the
standard decomposition
\cite{CSS1997, DDT1997}, Wiles's proof
that $\rho$ is modular proceeds in five stages:

\textbf{Stage~1} (Residual modularity):
prove $\bar{\rho}$ is modular via Langlands--Tunnell
\cite{Langlands1980, Tunnell1981}.

\textbf{Stage~2} (Deformation ring):
construct the universal deformation ring~$R$.

\textbf{Stage~3} (Hecke algebra):
construct $\mathbb{T}$ localized at~$\fm$.

\textbf{Stage~4} (Numerical criterion):
verify the Wiles--Lenstra numerical criterion.

\textbf{Stage~5} (Patching):
select Taylor--Wiles primes and prove $R \cong T$.
\end{definition}

\begin{definition}[Taylor--Wiles primes]\label{def:tw}
A set $Q = \{q_1, \ldots, q_r\}$ of primes are \emph{Taylor--Wiles
primes at level~$n$} if for each $q \in Q$:
(i)~$q \equiv 1 \pmod{p^n}$,
(ii)~$q \nmid N$,
(iii)~$\bar{\rho}(\Frob_q)$ has distinct eigenvalues
in~$\bar{\F}_p$.
\end{definition}

\begin{remark}[Historical note on Stage~4]\label{rem:stage4}
In the published 1995 proof \cite{Wiles1995}, Stage~4 is subsumed
by Stage~5: the numerical criterion is verified \emph{within} the
patching argument, not by an external Euler system.  The only Euler
system remaining is Rubin's \cite{Rubin1991}, used for the CM base
case.
\end{remark}


% ============================================================
\section{Main Results}\label{sec:main}
% ============================================================

\subsection{Theorem A: Stage~5 is \texorpdfstring{$\BISH$}{BISH}}\label{sec:stage5}

We classify Stage~5 first because the result is the most
surprising: the heart of the Taylor--Wiles method is fully
constructive.

\subsubsection{The original argument and its logical cost}

Taylor--Wiles \cite{TaylorWiles1995} and Diamond \cite{Diamond1997}
select TW primes at each level $n \ge 1$, form the projective
system $\{R_{Q_n} / \fm^n\}_{n \ge 1}$, and take the inverse limit.
The nonemptiness of the inverse limit of nonempty finite sets is a
compactness argument equivalent to $\WKL$.
Diamond~\cite{Diamond1997} abstracts the freeness criterion but
\emph{still} requires the inverse limit.
The logical cost is $\BISH + \MP + \FT$:
$\MP$ from the unbounded search for TW primes at each level~$n$,
and $\FT$ from the inverse limit compactness.

\subsubsection{Brochard's elimination of the Fan Theorem}

\begin{theorem}[Brochard 2017, Theorem~1.1]\label{thm:brochard}
Let $A \to B$ be a local morphism of commutative Artinian local
rings.  If $\edim(B) \le \edim(A)$ and $M$ is a nonzero
$B$-module that is flat as an $A$-module, then $M$ is free as
a $B$-module and $B$ is a relative complete intersection
over~$A$.
\end{theorem}

\begin{corollary}[Elimination of patching]\label{cor:brochard}
The entire Taylor--Wiles patching argument can be performed at a
single finite level ($n = 2$), without forming any infinite
projective system.
\end{corollary}

The logical cost of Stage~5 compactness drops from $\FT$ to
$\BISH$: the infinite inverse limit is replaced by a finite-level
algebraic check.

\subsubsection{Effective Chebotarev eliminates Markov's Principle}

\begin{theorem}[Effective Chebotarev]\label{thm:cheb}
Let $L/\Q$ be a finite Galois extension with absolute discriminant
$d_L$, and let $C$ be a conjugacy class in $\Gal(L/\Q)$.
Unconditionally, there exists a prime $q \le d_L^{12577}$
with $\Frob_q \in C$
(Lagarias--Montgomery--Odlyzko \cite{LMO1979},
Ahn--Kwon \cite{AhnKwon2019}).
\end{theorem}

\begin{proposition}[TW prime search is $\BISH$]\label{prop:tw}
The search for Taylor--Wiles primes at level $n = 2$ is a
$\BISH$-decidable finite computation: compute $d_{L_2}$ from
$(N, p, \bar{\rho})$, then test all primes
$q \le d_{L_2}^{12577}$ for conditions~(i)--(iii).
The bound is astronomically large but computable before the
search begins.
\end{proposition}

\begin{proof}
The splitting field $L_2$ is the compositum of the splitting field
$K_{\bar{\rho}}$ of~$\bar{\rho}$ and $\Q(\mu_{p^2})$.  The
absolute discriminant $d_{L_2}$ is computable from $(N, p,
\bar{\rho})$ by standard algebraic number theory (Hensel bounds for
wild ramification).  By Theorem~\ref{thm:cheb}, there exists a
prime $q \le d_{L_2}^{12577}$ with $\Frob_q$ in the appropriate
conjugacy class.  The TW conditions (i)--(iii) are decidable for
each prime (finite arithmetic).  The search terminates within
$d_{L_2}^{12577}$ steps.
\end{proof}

\begin{theorem}[Stage~5 Classification]\label{thm:stage5}
Stage~5 of Wiles's proof, in the Brochard formulation, is
$\BISH$.  The compactness ($\FT$) is eliminated by
Theorem~\ref{thm:brochard}; the prime search ($\MP$) is
eliminated by Proposition~\ref{prop:tw}.
\end{theorem}

\begin{proof}
Given axioms B1 (Brochard's criterion), B2 (effective Chebotarev),
and B3 (discriminant computability), Stage~5 proceeds as:
(1)~compute the Chebotarev bound from $(N,p,\bar{\rho})$;
(2)~search all primes up to the bound for TW conditions;
(3)~construct patching data at level $n = 2$;
(4)~apply Brochard's criterion to obtain freeness ($R \cong T$).
All steps are finite, decidable computations.  No Fan Theorem
(infinite inverse limit).  No Markov's Principle (unbounded search).
\end{proof}

\subsubsection{The historical de-omniscientizing descent}

\begin{center}
\renewcommand{\arraystretch}{1.15}
\begin{tabular}{@{}lll@{}}
\toprule
\textbf{Formulation} & \textbf{Year} & \textbf{Logical cost} \\
\midrule
Taylor--Wiles (original) & 1995 & $\MP + \FT$ \\
Diamond (algebraic patching) & 1997 & $\MP + \FT$ \\
Brochard (de Smit's conjecture) & 2017 & $\MP$ \\
Brochard + effective Chebotarev & 2017/2019 & $\BISH$ \\
\bottomrule
\end{tabular}
\end{center}

\medskip\noindent
None of these refinements were motivated by constructive
foundations.  The community eliminated the Fan Theorem and
Markov's Principle from the proof of FLT without knowing it.


\subsection{Theorem B: Stages~2--4 are \texorpdfstring{$\BISH$}{BISH}}\label{sec:alg}

\subsubsection{Stage~2: The deformation ring (\texorpdfstring{$\BISH$}{BISH})}

The universal deformation ring $R$ is a quotient of
$W(\F_p)[[x_1, \ldots, x_d]]$ by an ideal specified by local
conditions.  The local condition at~$p$ (crystalline with
Hodge--Tate weights $(0,1)$) is classified by Fontaine--Laffaille
theory; Paper~59 \cite{Paper59} classified the $p$-adic comparison
as~$\BISH$.  The local conditions at primes $\ell \mid N$ are
finite-dimensional specifications over~$\F_p$.  The universal
property (Schlessinger's criterion) involves tangent space and
obstruction calculations, both finite algebra.  The tangent space
$H^1(G_{\Q,S}, \mathrm{ad}^0\bar{\rho})$ is finite-dimensional
over~$\F_p$ (promotion from hull to universal ring follows); all
steps are constructive commutative algebra in the sense of
\cite{MRR1988}.  Stage~2 is $\BISH$.

\subsubsection{Stage~3: The Hecke algebra (\texorpdfstring{$\BISH$}{BISH})}

The Hecke algebra $T = \mathbb{T}(N, 2) \otimes \Z_p$, localized
at~$\fm$, is a finitely generated $\Z_p$-algebra.
Hecke operators $T_\ell$ for $\ell \nmid Np$ are explicit linear
maps on a finite-dimensional space; diamond operators
$\langle d \rangle$ for $d \in (\Z/N\Z)^\times$ are explicit;
the dimension of $S_2(\Gamma_0(N))$ is computable by
Riemann--Roch.  Everything is finite arithmetic.  Stage~3 is $\BISH$.

\subsubsection{Stage~4: The numerical criterion (\texorpdfstring{$\BISH$}{BISH})}\label{sec:stage4cm}

The numerical criterion in the published proof is verified within
Stage~5.  For the CM base case, Rubin's Euler system \cite{Rubin1991}
selects Kolyvagin primes via effective Chebotarev
(Theorem~\ref{thm:cheb}).

The Euler system machinery beyond the prime search requires audit.
Rubin's argument constructs norm-compatible systems of elliptic units
$c_n \in K_n^\times$ and applies Kolyvagin's descent.  Each step
operates within a \emph{finite} extension tower $K_n/K$; the
elliptic units are explicit algebraic numbers; and the descent
argument is a finite computation in Galois cohomology
$H^1(\Gal(K_n/K), E[p^n])$ with finite coefficients.  No
topological compactness or unbounded search enters beyond the
Kolyvagin prime selection, which is bounded by effective Chebotarev.
The Euler system machinery is therefore~$\BISH$, and the full CM
base case is~$\BISH$.

We flag this as an \emph{axiomatized} classification: the Lean
formalization records the Stage~4 classification as a definitional
assignment.  A fully formal audit of Rubin's Euler system would
require formalizing the norm-compatibility relations and
Kolyvagin's descent in Lean, which is beyond the scope of this paper.

\subsubsection{Galois cohomology and Selmer groups (\texorpdfstring{$\BISH$}{BISH})}

Two potential non-constructive sites require explicit treatment.

\textbf{Decomposition groups.}
The restriction maps $\mathrm{res}_v : H^1(G_{\Q,S}, M) \to
H^1(G_{\Q_v}, M)$ require fixing a decomposition group
$D_v \subset G_\Q$.  Classically, this requires
$\Qbar \hookrightarrow \Qbar_v$ (Krull/Zorn).
But Wiles works with finite coefficient modules:
$\mathrm{ad}^0\bar{\rho}$ is a finite $\F_p$-vector space.
Galois cohomology factors through a finite quotient
$\Gal(K/\Q)$, and the decomposition group is determined by a
finite search.  No Krull/Zorn is needed.  This is~$\BISH$.

\textbf{Poitou--Tate duality.}
For finite coefficient modules, Poitou--Tate reduces to a long
exact sequence of finite abelian groups.  The duality is linear
algebra over~$\F_p$.  No topological compactness is needed.
This is~$\BISH$.


\subsection{Theorem C: Stage~1 requires \texorpdfstring{$\WLPO$}{WLPO}}\label{sec:lt}

This is where the non-constructive content lives.

\subsubsection{The weight~1 obstruction}

The Langlands--Tunnell theorem asserts that every 2-dimensional
complex representation of $G_\Q$ with solvable image arises from a
weight~1 modular form.  Unlike weight~$k \ge 2$ forms, weight~1
forms do not appear in the \'etale cohomology of modular curves.
The only known proof route uses the Arthur--Selberg trace
formula and analytic properties of automorphic $L$-functions.

\subsubsection{The trace formula (\texorpdfstring{$\WLPO$}{WLPO})}

Langlands's proof of cyclic base change \cite{Langlands1980}
uses the Arthur--Selberg trace formula.
Three constructive obstructions arise:

\textbf{Spectral decomposition.}  Extracting a discrete
automorphic representation from the spectral decomposition
of $L^2(\GL_2(F) \backslash \GL_2(\A_F))$ requires isolating an
eigenvalue from the continuous spectrum.
Constructively, this requires deciding whether the spectral
measure of an interval is zero or positive---at minimum~$\WLPO$.

\textbf{Orbital integral matching.}  The Fundamental Lemma
(for $\GL_2$, elementary) asserts equality of certain
Archimedean orbital integrals over~$\R$ and~$\C$.  Verifying
exact equality of real integrals requires~$\WLPO$.

\textbf{Continuous spectrum cancellation.}  The trace formula
isolates the discrete spectrum by subtracting the Eisenstein
contribution.  Testing whether a specific complex expression
equals zero requires~$\WLPO$.

\subsubsection{The converse theorem (WLPO/LPO)}

Tunnell's proof \cite{Tunnell1981} uses the Gelbart--Jacquet
symmetric square lifting \cite{GelbartJacquet1978}, which relies on
the converse theorem for $\GL_3$.  The
Phragm\'en--Lindel\"of condition (boundedness in vertical
strips) requires evaluating a supremum over an unbounded,
non-compact domain.  This is at least~$\LPO$.

\begin{theorem}[Stage~1 requires $\WLPO$]\label{thm:stage1}
The Langlands--Tunnell theorem, as used in Wiles's proof,
requires at least~$\WLPO$.
The Phragm\'en--Lindel\"of condition requires evaluating a supremum
over an unbounded domain ($\LPO$-level in general).  For $\GL_2$,
the Ramanujan conjecture is known (Deligne 1974), providing a
computable bound $|\lambda_\pi(p)| \le 2p^{(k-1)/2}$ on the Hecke
eigenvalues.  This converts the unbounded supremum to a comparison
against an explicit threshold: is a specific computable quantity
zero or not?  The reduction from $\LPO$ to $\WLPO$ follows because
a bounded quantity with a computable bound requires only a zero
test ($\WLPO$), not an unbounded search ($\LPO$).
\end{theorem}

\begin{remark}[Open question: $\WLPO$ vs.\ $\LPO$]\label{rem:gap}
The precise classification of Stage~1 within the range
$[\WLPO, \LPO]$ depends on whether the spectral gap of
$L^2(\GL_2(F) \backslash \GL_2(\A_F))$ is computable.  For $\GL_2$,
the Ramanujan conjecture is known (Deligne 1974), and the spectral
gap should be effective.  We state the classification as~$\WLPO$
with this caveat.
\end{remark}


\subsection{Theorem D: The Asymmetry Theorem}\label{sec:asymmetry}

\begin{theorem}[Asymmetry of Wiles's Proof]\label{thm:main}
The $\CRM$ classification of Wiles's proof is:

\medskip
\begin{center}
\renewcommand{\arraystretch}{1.15}
\begin{tabular}{@{}llll@{}}
\toprule
\textbf{Stage} & \textbf{Content} &
  \textbf{Classification} & \textbf{Key input} \\
\midrule
1 & Langlands--Tunnell & $\WLPO$ &
  Trace formula, converse theorem \\
2 & Deformation ring & $\BISH$ &
  Schlessinger, Fontaine--Laffaille \\
3 & Hecke algebra & $\BISH$ &
  Finite algebra, Riemann--Roch \\
4 & Numerical criterion & $\BISH$ &
  Subsumed by Stage~5 \\
4$'$ & CM base case (Rubin) & $\BISH$ &
  Effective Chebotarev \\
5 & Patching & $\BISH$ &
  Brochard + effective Chebotarev \\
\midrule
\multicolumn{2}{@{}l}{\textbf{Overall}} &
  $\BISH + \WLPO$ & \\
\bottomrule
\end{tabular}
\end{center}

\medskip\noindent
The $\WLPO$ is localized entirely in Stage~1.  Removing
Stage~1 drops the classification to~$\BISH$.
\end{theorem}

\begin{proof}
Stages~2--5 are classified individually in
\S\ref{sec:stage5} and \S\ref{sec:alg}: each is $\BISH$.  The
join (maximum) over all stages is determined by Stage~1, which
requires $\WLPO$ (Theorem~\ref{thm:stage1}).  Since $\BISH$ is
subsumed by $\BISH + \WLPO$, the overall classification is
$\BISH + \WLPO$.

For the localization statement: the only axiom beyond $\BISH$
used anywhere in the proof is the $\WLPO$ content of the trace
formula and converse theorem in Stage~1.  The $\BISH$ stages do
not invoke Stage~1's output except as a single boolean datum
(``$\bar{\rho}$ is modular: yes/no''), which is a decidable
proposition once Stage~1 has been executed.
\end{proof}

\begin{corollary}[Logical cost of FLT]\label{cor:flt}
Wiles's proof of Fermat's Last Theorem has logical cost
$\BISH + \WLPO$.
\end{corollary}

\begin{corollary}[The engine is constructive]\label{cor:engine}
The Taylor--Wiles patching method---the central proof technology
of the Langlands program for $\GL_2/\Q$---contributes zero
logical cost beyond~$\BISH$.
\end{corollary}

\begin{corollary}[Algebraic weight~1 modularity implies
  constructive FLT]\label{cor:alg}
If a purely algebraic proof is found that 2-dimensional Artin
representations of solvable type are modular---bypassing the
Arthur--Selberg trace formula---then Wiles's proof of FLT
becomes fully constructive ($\BISH$).
\end{corollary}


% ============================================================
\section{CRM Audit}\label{sec:crm}
% ============================================================

\subsection{Constructive strength classification}

\begin{center}
\renewcommand{\arraystretch}{1.15}
\begin{tabular}{llll}
\toprule
\textbf{Result} & \textbf{Strength} & \textbf{Necessary?} & \textbf{Sufficient?} \\
\midrule
Theorem A (Stage~5 is $\BISH$) & $\BISH$ & Yes (from axioms) & Yes \\
Theorem B (Stages~2--4 are $\BISH$) & $\BISH$ & Yes (from axioms) & Yes \\
Theorem C (Stage~1 requires $\WLPO$) & $\WLPO$ & $\WLPO$ necessary & $\WLPO$ sufficient \\
Theorem D (Asymmetry) & $\BISH + \WLPO$ & Yes (join) & Yes \\
\bottomrule
\end{tabular}
\end{center}

\begin{sloppypar}\smallskip\noindent
\emph{Note on $\BISH$ classification.}
Lean's $\R$ and~$\C$ (Cauchy completions) pervasively introduce
\texttt{Classical.choice} as an infrastructure artifact.
Constructive stratification is established by \emph{proof
content}---explicit witnesses, no omniscience principles as
hypotheses---not by axiom checker output
(cf.\ Paper~10, \S Methodology).
\end{sloppypar}

\subsection{What descends, from where, to where}

The central $\CRM$ phenomenon is a \emph{descent in logical
strength} of the patching step:
\[
\underbrace{\MP + \FT}_{\text{1995: TW original}} \;\;\xrightarrow{\quad\text{Brochard + eff.\ Chebotarev}\quad}\;\; \underbrace{\BISH}_{\text{2017: no omniscience}}.
\]
The mechanism: Brochard's finite-level criterion replaces the
infinite inverse limit (eliminating $\FT$), and effective Chebotarev
bounds the prime search (eliminating $\MP$).

Paper~50 \cite{Paper50} identified the de-omniscientizing descent
pattern in the five great conjectures of arithmetic geometry.
Paper~68 reveals a parallel phenomenon in \emph{proof methods}:
the community de-omniscientized the proof of FLT without knowing it.

\subsection{Comparison with earlier calibration patterns}

This paper establishes a variant of the pattern from Papers~2, 7,
and~8:
\begin{enumerate}[nosep]
\item Identify the constructive obstruction ($\WLPO$ for Stage~1;
  $\MP + \FT$ for Stage~5 pre-Brochard).
\item Classify each stage independently.
\item Identify a structural bypass (Brochard + effective Chebotarev
  $\to$ $\BISH$ for Stage~5).
\item Show the bypass occurred historically without constructive motivation.
\end{enumerate}
The novelty is that the de-omniscientizing descent occurred on a
\emph{human} timescale (1995--2017) in the published literature,
rather than on a conjectural timescale (the motives conjectures).


% ============================================================
\section{Formal Verification}\label{sec:formal}
% ============================================================

\subsection{File structure and build status}

The Lean~4 bundle resides at \texttt{paper~68/P68\_WilesFLT/}
with the following structure:

\begin{center}
\renewcommand{\arraystretch}{1.15}
\begin{tabular}{@{}llp{6.8cm}@{}}
\toprule
\textbf{File} & \textbf{Lines} & \textbf{Content} \\
\midrule
\texttt{Paper68\_Axioms.lean} & 132 &
  Opaque types, axioms B1--B3, CRM hierarchy \\
\texttt{Paper68\_Stage5.lean} & 178 &
  Target~1: Stage~5 is $\BISH$ \\
\texttt{Paper68\_Asymmetry.lean} & 183 &
  Target~2: asymmetry theorem \\
\midrule
\textbf{Total} & \textbf{493} &
  \texttt{sorry}: 0 \quad warnings: 0 \quad errors: 0 \\
\bottomrule
\end{tabular}
\end{center}

\medskip\noindent
\textbf{Build status:} \texttt{lake build} $\to$
\textbf{0~errors, 0~warnings, 0~\texttt{sorry}s}.
Lean~4 version: \texttt{v4.29.0-rc1}.  Mathlib4 dependency
via \texttt{lakefile.lean}.

\subsection{Axiom inventory}

\begin{sloppypar}
The formalization declares axioms in two categories.  First,
12~\emph{opaque declarations} (\texttt{ArtinLocalRing},
\texttt{ArtinModule}, \texttt{embDim}, \texttt{IsFlat},
\texttt{IsFreeModule}, \texttt{NumberField}, \texttt{absDisc},
\texttt{ConjClass}, \texttt{frobInClass}, \texttt{ResidualRep},
\texttt{twSplittingField}, \texttt{TWConditions}) introduce the
mathematical universe as uninterpreted types and properties.
Second, 8~\emph{theorem-level axioms} encode the deep results:
\end{sloppypar}

\begin{center}
\small
\renewcommand{\arraystretch}{1.15}
\begin{tabular}{rlll}
\toprule
\textbf{\#} & \textbf{Axiom} & \textbf{Role} & \textbf{Reference} \\
\midrule
1 & \texttt{brochard\_finite\_criterion} (B1) & Load-bearing &
  Brochard, \textit{Comp.\ Math.}\ 153 (2017) \\
2 & \texttt{effective\_chebotarev} (B2) & Load-bearing &
  LMO (1979); Ahn--Kwon (2019) \\
3 & \texttt{tw\_disc\_computable} (B3) & Documentation &
  Standard ANT (Hensel bounds) \\
4 & \texttt{twConjClass} & Bridge &
  Standard ANT \\
5 & \texttt{frob\_implies\_tw\_conditions} & Bridge &
  Defn.\ of TW primes \\
6 & \texttt{construct\_patching\_data} & Bridge &
  Wiles (1995), Diamond (1997) \\
7 & \texttt{patching\_data\_valid} & Bridge &
  Wiles (1995), Diamond (1997) \\
8 & \texttt{patching\_data\_edim} & Bridge &
  Wiles (1995), numerical criterion \\
\bottomrule
\end{tabular}
\end{center}

\smallskip\noindent
\emph{Note on B3.}  Within the opaque-type framework, bare existence
($\exists d,\; \texttt{absDisc}\;L = d$) would be vacuously true.
Axiom~B3 strengthens this to $d > 0$, recording the mathematical
content that the splitting field is a genuine number field.
The primary role of~B3 is as a \emph{documentation marker} in the
proof pipeline, flagging the discriminant computation as the input
to the Chebotarev bound.

\subsection{Key code snippets}

\textbf{CRM hierarchy} (from \texttt{Paper68\_Axioms.lean}):

\begin{lstlisting}
inductive CRMLevel where
  | BISH | MP | LLPO | WLPO | LPO | CLASS
  deriving DecidableEq, Repr

def CRMLevel.join : CRMLevel -> CRMLevel -> CRMLevel
  | BISH,  b     => b
  | a,     BISH  => a
  | CLASS, _     => CLASS
  | _,     CLASS => CLASS
  | LPO,   _     => LPO
  | _,     LPO   => LPO
  | WLPO,  _     => WLPO
  | _,     WLPO  => WLPO
  | LLPO,  _     => LLPO
  | _,     LLPO  => LLPO
  | MP,    MP    => MP
\end{lstlisting}

\textbf{TW prime search terminates} (from
\texttt{Paper68\_Stage5.lean}):

\begin{lstlisting}
theorem tw_prime_search_terminates
  (N p : Nat) (rhobar : ResidualRep N p) :
  exists (bound : Nat) (q : Nat),
    Nat.Prime q /\ q <= bound
    /\ TWConditions N p 2 q rhobar := by
  -- B3: compute discriminant (positive)
  obtain <d_L, _hpos, hdisc> := tw_disc_computable N p rhobar
  -- B2: effective Chebotarev
  obtain <q, hprime, hbound, hfrob> :=
    effective_chebotarev _ (twConjClass N p rhobar) d_L hdisc
  -- Frobenius => TW conditions
  exact <d_L ^ 12577, q, hprime, hbound,
    frob_implies_tw_conditions N p q rhobar hprime hfrob>
\end{lstlisting}

\textbf{Asymmetry theorem} (from
\texttt{Paper68\_Asymmetry.lean}):

\begin{lstlisting}
theorem asymmetry_theorem :
    wiles_overall = CRMLevel.WLPO /\
    wiles_without_stage1 = CRMLevel.BISH :=
  <wiles_proof_classification, wlpo_localisation>
\end{lstlisting}

\subsection{\texttt{\#print axioms} output}

\begin{center}
\small
\begin{tabular}{ll}
\toprule
\textbf{Theorem} & \textbf{Axioms (custom only)} \\
\midrule
\texttt{tw\_prime\_search\_terminates} &
  B2, B3, axioms 4--5 \\
\texttt{stage5\_is\_bish} &
  B1, bridge axioms (6--8) \\
\texttt{wiles\_proof\_classification} &
  \textbf{None} (definitional \texttt{simp}) \\
\texttt{wlpo\_localisation} &
  \textbf{None} (definitional \texttt{simp}) \\
\texttt{asymmetry\_theorem} &
  \textbf{None} (pair of above two) \\
\bottomrule
\end{tabular}
\end{center}

\medskip\noindent
\begin{sloppypar}
\textbf{Classical.choice audit.}  The CRM hierarchy
(\texttt{CRMLevel}) and the asymmetry theorem are purely
inductive-type computations:
\texttt{\#print axioms asymmetry\_theorem}
shows only \texttt{propext} and \texttt{Quot.sound}.
The Stage~5 theorems carry
\texttt{Classical.choice} via the Mathlib
\texttt{Nat.Prime} infrastructure---an artifact, not
a proof-content dependency.
\end{sloppypar}

\subsection{Reproducibility}

Lean~4 formalization files are available at the Zenodo repository:
\leanRepo.  The bundle compiles with \texttt{lake build} on Lean
v4.29.0-rc1 + Mathlib4.


% ============================================================
\section{Discussion}\label{sec:discuss}
% ============================================================

\subsection{The de-omniscientizing descent pattern}

Paper~50 \cite{Paper50} identified a ``de-omniscientizing
descent'' in the five great conjectures: geometric origin converts
$\LPO$-level data to $\BISH$-level data.  Paper~68 reveals a
parallel phenomenon in proof methods.

The descent $\MP + \FT \to \BISH$ occurred over twenty-two
years (1995--2017), driven by algebraists solving commutative
algebra problems, not by logicians.  This is the first example
in the CRM program of a de-omniscientizing descent occurring on a
\emph{human} timescale in the published literature, as opposed
to a conjectural timescale.

\subsection{What the calibration reveals}

The deepest implication is structural: the Langlands
correspondence, at least for $\GL_2/\Q$, has a logical asymmetry.
The Galois side (deformation theory, patching) is constructive.
The automorphic side (trace formula, $L$-functions) is not.  The
bridge between them adds no cost; the non-constructive content
lives on the automorphic bank.

\subsection{Relationship to existing literature}

McLarty \cite{McLarty2010} showed Grothendieck universes are
eliminable from FLT.  Our analysis is orthogonal: set-theoretic
strength vs.\ constructive principles.  Mines--Richman--Ruitenburg
\cite{MRR1988} developed constructive commutative algebra, providing
the framework for the $\BISH$ claims in Stages~2--4.  The
Taylor--Wiles method has been extended to $\GL_n$ by
Barnet-Lamb--Gee--Geraghty \cite{BLGGT2014} and to non-self-dual
representations by Calegari--Geraghty \cite{CalegariGeraghty2018};
the constructive status of these generalizations remains open.

\subsection{What the Lean verification adds}

The Lean~4 formalization verifies the \emph{logical assembly}: given
axiomatized inputs (Brochard, effective Chebotarev, the Taylor--Wiles
construction), the composition yields the claimed CRM classifications.
The axioms encode \emph{precisely} where human mathematical judgment
enters; the machine checks that no additional judgments are smuggled in.

This is the standard methodology for CRM formalization (cf.\
Paper~10 \cite{Paper10}).  Formalizing Brochard's theorem or the
Langlands--Tunnell theorem in Lean would be a multi-year project far
beyond the scope of a single CRM audit.  What the formalization
achieves is a \emph{machine-checked proof outline}: the deep mathematics
lives in the axioms; Lean verifies that they compose correctly to
produce the asymmetry theorem.  The zero-\texttt{sorry} guarantee
ensures no logical step has been skipped.

\subsection{Is the \texorpdfstring{$\WLPO$}{WLPO} intrinsic to FLT?}

Fermat's Last Theorem is a statement about natural numbers:
$\forall n \ge 3,\; \forall a\,b\,c \in \N^+,\;
a^n + b^n \neq c^n$.  As a $\Pi^0_1$ sentence, it ``should'' be
provable in~$\BISH$ if true.  The next two sections investigate:
first we show that the $\WLPO$ is \emph{irreducible} for any
proof through weight~$1$ (\S\ref{sec:weight1}), then we show
that the weight~$1$ step can be bypassed entirely
(\S\ref{sec:bypass}).


% ============================================================
\section{The weight~$1$ obstruction}\label{sec:weight1}
% ============================================================

The $\WLPO$ in Wiles's proof enters through the
Langlands--Tunnell theorem, which establishes residual
modularity via the Arthur--Selberg trace formula at weight~$1$.
We now show that this $\WLPO$ is irreducible: five
independent approaches to eliminating it all fail.  The
analysis reveals the precise nature of the obstruction and
motivates the bypass of \S\ref{sec:bypass}.

\subsection{The decidability descent}

The key technique for eliminating $\WLPO$ at weight~$\ge 2$
is a \emph{decidability descent}: if the trace formula, applied
with algebraic test functions, produces an identity
$\sum_{i=1}^N a_i = \sum_{j=1}^M b_j$ where all $a_i, b_j$
are algebraic numbers and the bounds $N, M$ are algebraically
computable, then the identity is a sentence in the decidable
first-order theory $\mathrm{ACF}$ (Tarski--Seidenberg) and
hence $\BISH$-verifiable regardless of whether its proof used
non-constructive analysis.

At weight~$\ge 2$, all three conditions are satisfied: the
pseudo-coefficient kills hyperbolic orbital integrals (eliminating
transcendental terms), the spectral side contains only holomorphic
forms with algebraic Hecke eigenvalues, and Riemann--Roch
computes the exact dimension.  The Jacquet--Langlands comparison
at weight~$\ge 2$ is therefore~$\BISH$.

At weight~$1$, the descent fails on all three fronts.

\subsection{Failure~1: Transcendental Archimedean integrals}

At weight~$1$, the Archimedean test function is the character of
a limit of discrete series, which does \emph{not} vanish on
hyperbolic (split) elements.  The orbital integral over the
hyperbolic torus evaluates to $\log \varepsilon_K$---the logarithm
of the fundamental unit of a real quadratic field---a
transcendental number.  These terms do not cancel between the
two sides of the base change comparison.

\subsection{Failure~2: Maass form contamination}

The limit of discrete series at weight~$1$ shares its
infinitesimal character with the principal series at Laplacian
eigenvalue $\lambda = 1/4$.  The spectral side therefore receives
contributions from both holomorphic weight~$1$ forms \emph{and}
odd Maass forms of eigenvalue~$1/4$:
\[
  \text{Spectral side} = \sum_{\pi \,\text{hol, wt 1}}
  \tr \pi(f) \;-\; \sum_{\pi' \,\text{Maass, }\lambda = 1/4}
  \tr \pi'(f).
\]
The Hecke eigenvalues of Maass forms are generically
transcendental.  The identity equates sums of real numbers,
not algebraic numbers.  The Tarski decidability argument does
not apply.

\subsection{Failure~3: Unknown index bounds}

Even if the terms were algebraic, the decidability descent
requires computable bounds on the number of terms.  At
weight~$1$, the dimension of $H^0(X, \omega)$ has no purely
algebraic formula: the Riemann--Roch theorem gives
$\dim H^0 - \dim H^1$, but $H^1 \ne 0$ at weight~$1$ (by
Serre duality, $H^1(\omega(\chi))$ is dual to~$S_1(\chi^{-1})$),
and the dimension jumps erratically.  Without knowing the index
set, the identity cannot be formulated as a sentence
in~$\mathrm{ACF}$.

\subsection{Failure~4: The $p$-adic initialisation trap}

The $p$-adic theory of overconvergent modular forms
(Buzzard--Taylor \cite{BuzzardTaylor1999}, Kassaei, Pilloni)
provides a purely $p$-adic construction of weight~$1$ forms
that avoids all three Archimedean obstructions: Hida families
are~$\BISH$, specialisation is~$\BISH$, and the
Kassaei--Pilloni classicality theorem is~$\BISH$.

However, the $p$-adic machinery cannot \emph{originate}
modularity.  To place $\bar{\rho}_3$ on the eigencurve requires
finding a weight $k \ge 2$ form whose residual representation
matches~$\bar{\rho}_3$---which requires Langlands--Tunnell.
Modularity lifting theorems (Taylor--Wiles, Kisin,
Calegari--Geraghty) are similarly \emph{relative}: they
transform modularity from one form to another but cannot create
it.  The trace formula is the unique \emph{absolute} bridge
from Galois representations to automorphic forms.

\subsection{Failure~5: The universal quantifier}

A direct algebraic approach reveals the deepest structure.
For any \emph{specific} conductor~$N$, the space $S_1(N, \chi)$
is computable by the Eisenstein trick: choose Eisenstein series
$E_a, E_b$ of weight~$1$ with no common zeros; then $S_1(N,\chi)$
is the intersection of their images in $S_2(N,\chi\psi)$,
computable via modular symbols and finite linear algebra.
For any given~$N$, modularity is $\BISH$-verifiable.

However, FLT concerns a hypothetical Frey curve with
conductor $N = \mathrm{rad}(abc)$, an unbounded variable.  The
proof requires a \emph{universal} existence theorem: for every
valid $S_4$ representation, $S_1(N,\chi)$ contains the
corresponding eigenform.  Because $H^1 \ne 0$ at weight~$1$,
Riemann--Roch cannot force $h^0 \ge 1$.  The only known method
for universal existence is the trace formula.

\begin{remark}[Verification versus certification]%
\label{rem:ontology}
The $\WLPO$ is not the cost of \emph{constructing} weight~$1$
forms (which are algebraic, computable objects).  It is the cost
of the \emph{universal quantifier}: proving that the algebraic
algorithm succeeds for every possible input.
\emph{Verification} (checking a specific instance) is~$\BISH$.
\emph{Certification} (proving the algorithm never fails)
is~$\WLPO$.  This distinction---between ontology and
universality---is invisible classically but sharp in
constructive logic.
\end{remark}

\begin{theorem}[The weight~$1$ obstruction is irreducible]%
\label{thm:weight1}
The Langlands--Tunnell base change at weight~$1$ cannot be
constructivised by decidability descent, by $p$-adic methods,
or by direct algebraic computation.  The $\WLPO$ is
irreducible for any proof route that passes through weight~$1$
automorphic forms.
\end{theorem}


% ============================================================
\section{The dihedral bypass: FLT is \texorpdfstring{$\BISH$}{BISH}}%
\label{sec:bypass}
% ============================================================

The $\WLPO$ in Wiles's proof arises from a single choice:
using the residual prime $p = 3$, which forces the base case
through $\GL_2(\F_3)$ (projective image~$S_4$, octahedral)
and hence through the Langlands--Tunnell theorem (weight~1,
trace formula, $\WLPO$).  Post-2009 developments eliminate
this choice entirely.

\subsection{The $p = 2$ base case (Kisin)}

For an elliptic curve $E/\Q$, consider the $2$-torsion
representation $\bar{\rho}_{E,2} : G_\Q \to \GL_2(\F_2)$.
The group $\GL_2(\F_2) \cong S_3$ is the dihedral group~$D_3$.
Any representation with dihedral projective image lifts to a
characteristic zero representation induced from a
Hecke character of a quadratic field.  By Hecke's classical
theorem (1926), such representations are modular: the
corresponding modular forms are binary theta series attached
to lattices in imaginary quadratic fields.

Hecke's construction is entirely algebraic---it uses lattice
sums and the algebraic theory of quadratic forms, with no
trace formula, no $L^2$ spectral decomposition, and no
continuous spectrum.  The construction is~$\BISH$.

\begin{proposition}[Dihedral base case is $\BISH$]%
\label{prop:dihedral}
Let $\bar{\rho} : G_\Q \to \GL_2(\F_2)$ be a continuous
representation with $\bar{\rho}|_{G_{\Q(\zeta_2)}}$ absolutely
irreducible.  Then $\bar{\rho}$ is modular.  The proof is~$\BISH$:
it uses only the algebraic theory of theta series and class field
theory.
\end{proposition}

\subsection{Modularity lifting at $p = 2$ (Kisin 2009)}

Kisin \cite{Kisin2009} proved a modularity lifting theorem for
$2$-adic potentially Barsotti--Tate representations.  The proof
uses the Taylor--Wiles--Kisin patching method, which we
classified as~$\BISH$ in \S\ref{sec:stage5}, together with
Kisin's classification of finite flat group schemes over
$\Z_2$-extensions (commutative algebra: $\BISH$).
Combined with the Khare--Wintenberger induction
\cite{KhareWintenberger2009I, KhareWintenberger2009II}, this
yields:

\begin{theorem}[Kisin's $p = 2$ modularity lifting is $\BISH$]%
\label{thm:kisin2}
The modularity lifting theorem for $2$-adic Barsotti--Tate
representations (Kisin \cite{Kisin2009}) is~$\BISH$.
The proof uses finite flat group scheme classification
(commutative algebra), Taylor--Wiles patching ($\BISH$,
Theorem~A), and effective Chebotarev ($\BISH$).
No analytic input from the trace formula is required.
\end{theorem}

\subsection{Completing the proof}

If $\bar{\rho}_{E,2}$ is absolutely irreducible, then:
\begin{enumerate}[nosep]
\item $\bar{\rho}_{E,2}$ is modular (Proposition~\ref{prop:dihedral},
  dihedral base case, $\BISH$).
\item $\rho_{E,2}$ is modular (Theorem~\ref{thm:kisin2},
  $p = 2$ modularity lifting, $\BISH$).
\item $E$ is modular ($\BISH$).
\end{enumerate}

If $\bar{\rho}_{E,2}$ is reducible, apply a $2$--$3$ switch
(the analogue of Wiles's $3$--$5$ trick): choose an auxiliary
curve $E'/\Q$ with $E'[3] \cong E[3]$ and $E'[2]$ absolutely
irreducible.  By Steps~1--3, $E'$ is modular.  Therefore
$\bar{\rho}_{E,3}$ is modular, and Wiles's original $p = 3$
modularity lifting theorem (Theorem~A, $\BISH$) yields
modularity of~$E$.  The $2$--$3$ switch uses the geometry of
the modular curve $X(3) \cong \mathbb{P}^1$ and the Hilbert
irreducibility theorem (decidable: $\BISH$).

\begin{theorem}[FLT is $\BISH$]%
\label{thm:flt-bish}
There exists a proof of Fermat's Last Theorem that is
$\BISH$: the $p = 2$ dihedral bypass (Kisin \cite{Kisin2009},
Khare--Wintenberger \cite{KhareWintenberger2009I}) replaces
Stage~1 (Langlands--Tunnell, $\WLPO$) with the dihedral base
case (Hecke theta series, $\BISH$).  The remaining stages are
$\BISH$ by Theorems~A--B.  Therefore:
\[
  \CRM(\mathrm{FLT}) \;=\; \BISH.
\]
\end{theorem}

\subsection{The potential modularity route}

An equivalent bypass, used in the Buzzard--Taylor Lean
formalisation project \cite{BuzzardTaylorFLT}, proceeds via
potential modularity: restrict $\bar{\rho}$ to the absolute
Galois group of a totally real field~$F$ where the representation
becomes dihedral (induced from a character).  The dihedral
base case over~$F$ is~$\BISH$ (Hecke theta series).  Modularity
lifting over~$F$ is~$\BISH$ (Theorem~A).  Descent from~$F$
to~$\Q$ uses Langlands' cyclic base change for~$\GL_2$ at
weight~$\ge 2$---which is~$\BISH$ by decidability descent
(the trace formula identity at weight~$\ge 2$ is an identity
between algebraic numbers, decidable in~$\mathrm{ACF}$ by Tarski's
theorem).  The Skinner--Wiles trick and strong multiplicity one
at weight~$2$ are likewise~$\BISH$.  The entire chain avoids
weight~$1$ forms.

\subsection{Why Wiles's proof costs $\WLPO$ and the modern
  proof does not}

The divergence point is the choice of residual prime:

\begin{center}
\begin{tabular}{lll}
\toprule
 & Wiles (1995) & Kisin--KW (2009) \\
\midrule
Residual prime & $p = 3$ & $p = 2$ \\
Group $\GL_2(\F_p)$ & $S_4$ (octahedral) & $S_3$ (dihedral) \\
Base case & Langlands--Tunnell & Hecke theta series \\
CRM cost of base case & $\WLPO$ & $\BISH$ \\
Lifting & Taylor--Wiles ($\BISH$) & Kisin ($\BISH$) \\
\midrule
\textbf{Total} & $\BISH + \WLPO$ & $\BISH$ \\
\bottomrule
\end{tabular}
\end{center}

The $\WLPO$ was never intrinsic to FLT.  It was an artefact
of $S_4$ having non-abelian composition factors that force
passage through the trace formula.  The group $S_3 = D_3$ is
solvable with only abelian composition factors, making Hecke's
algebraic construction sufficient.


% ============================================================
\section{Open questions (revised)}\label{sec:open-revised}
% ============================================================

\begin{enumerate}[nosep]
\item \textbf{Higher-dimensional modularity lifting.}
  Does the $\BISH$ classification of patching survive for $\GL_n$?
  Barnet-Lamb--Gee--Geraghty \cite{BLGGT2014} extend the method to
  $\GL_n$; Calegari--Geraghty \cite{CalegariGeraghty2018} handle
  non-self-dual representations.  If patching remains $\BISH$ in
  these settings, the entire Langlands program's non-constructive
  content would be concentrated in the automorphic input.

\item \textbf{The weight~1 existence problem.}
  Is there an algebraic universal existence theorem for weight~1
  eigenforms---a lower bound on $\dim S_1(N, \chi)$ in Artin
  eigenspaces that avoids the trace formula?  This would give an
  alternative route to eliminating the $\WLPO$ and has independent
  interest in constructive number theory.

\item \textbf{Function field Langlands.}
  Lafforgue's proof of the Langlands correspondence over function
  fields uses geometric methods (shtukas, \'etale cohomology) with
  no Archimedean place.  If this proof is $\BISH$, the $\WLPO$ in
  the number field Langlands programme would be localised to the
  Archimedean place---a structural finding about the cost of~$\R$.
\end{enumerate}


% ============================================================
\section{Conclusion}\label{sec:conclusion}
% ============================================================

We have applied constructive reverse mathematics to Fermat's
Last Theorem and established two results:

First, Wiles's 1995 proof is $\BISH + \WLPO$.  The $\WLPO$
is localised entirely in Stage~1 (Langlands--Tunnell): the
Taylor--Wiles engine and all algebraic machinery contribute
zero logical cost.

Second, the $\WLPO$ is eliminable.  The 21st-century proof
route (Kisin 2009, Khare--Wintenberger 2009) replaces Wiles's
$p = 3$ base case (octahedral, trace formula, $\WLPO$) with a
$p = 2$ base case (dihedral, Hecke theta series, $\BISH$).
The resulting proof is $\BISH$ throughout.

\[
  \boxed{\CRM(\mathrm{FLT}) \;=\; \BISH.}
\]

Fermat's Last Theorem---a $\Pi^0_1$ sentence about natural
numbers---has a constructive proof.  The $\WLPO$ in
Wiles's original argument was the cost of a specific proof
\emph{strategy} (the choice of $p = 3$), not a property of
the \emph{theorem}.  The CRM programme thesis---that logical
cost is intrinsic to theorems, not to proofs---is confirmed:
a true $\Pi^0_1$ sentence has a $\BISH$ proof, as expected.

The constructive content of the 21st-century proof was not
recognised by its creators.  Kisin, Khare, and Wintenberger
bypassed Langlands--Tunnell for reasons internal to number
theory (completing Serre's conjecture for $p = 2$), not for
foundational reasons.  The community unknowingly produced a
constructive proof of Fermat's Last Theorem.


% ============================================================
\section*{Acknowledgments}
\addcontentsline{toc}{section}{Acknowledgments}
% ============================================================

We thank the Mathlib contributors for the \texttt{Nat.Prime}
and decidable-equality infrastructure that underpins the
formalization.  We are grateful to the constructive reverse
mathematics community---especially the foundational work of
Bishop, Bridges, Richman, and Ishihara---for developing the
framework that makes calibrations like these possible.

The deep mathematics is due to Wiles, Taylor, Diamond,
Langlands, Tunnell, Kisin, Khare, Wintenberger, Brochard,
Hecke, and Lagarias--Odlyzko.  The weight~$1$ obstruction
analysis draws on the spectral theory of Arthur, Selberg,
and Harish-Chandra.  The CRM
methodology follows Bishop~\cite{Bishop1967} and
Bridges--Richman~\cite{BridgesRichman1987}.  The Lean~4
formalization was produced using AI code generation
(Claude Code, Opus 4.6) under human direction.  The author is a
practicing cardiologist rather than a professional logician or
arithmetic geometer; all mathematical claims should be evaluated
on their formal content.  We welcome constructive feedback from
domain experts.


% ============================================================
\begin{thebibliography}{30}
% ============================================================

\bibitem{AhnKwon2019}
J.~Ahn and S.-H.~Kwon.
\newblock Some explicit zero-free regions for Hecke $L$-functions.
\newblock \textit{J.~Number Theory}, 197:329--349, 2019.

\bibitem{BLGGT2014}
T.~Barnet-Lamb, T.~Gee, D.~Geraghty, and R.~Taylor.
\newblock Potential automorphy and change of weight.
\newblock \textit{Ann.\ of Math.}, 179(2):501--609, 2014.

\bibitem{Bishop1967}
E.~Bishop.
\newblock \textit{Foundations of Constructive Analysis}.
\newblock McGraw-Hill, 1967.

\bibitem{BridgesRichman1987}
D.~Bridges and F.~Richman.
\newblock \textit{Varieties of Constructive Mathematics}.
\newblock LMS Lecture Note Series 97. Cambridge University Press, 1987.

\bibitem{Brochard2017}
S.~Brochard.
\newblock Proof of de~Smit's conjecture: a freeness criterion.
\newblock \textit{Compositio Math.}, 153(11):2310--2317, 2017.

\bibitem{BuzzardTaylor1999}
K.~Buzzard and R.~Taylor.
\newblock Companion forms and weight one forms.
\newblock \textit{Ann.\ of Math.}, 149(3):905--919, 1999.

\bibitem{CalegariGeraghty2018}
F.~Calegari and D.~Geraghty.
\newblock Modularity lifting beyond the Taylor--Wiles method.
\newblock \textit{Invent.\ Math.}, 211(1):297--433, 2018.

\bibitem{CSS1997}
G.~Cornell, J.\,H.~Silverman, and G.~Stevens, editors.
\newblock \textit{Modular Forms and Fermat's Last Theorem}.
\newblock Springer, 1997.

\bibitem{DDT1997}
H.~Darmon, F.~Diamond, and R.~Taylor.
\newblock Fermat's Last Theorem.
\newblock In \textit{Elliptic Curves, Modular Forms \& Fermat's Last
Theorem}, pp.~2--140.  International Press, 1997.

\bibitem{Diamond1997}
F.~Diamond.
\newblock The Taylor--Wiles construction and multiplicity one.
\newblock \textit{Invent.\ Math.}, 128(2):379--391, 1997.

\bibitem{GelbartJacquet1978}
S.~Gelbart and H.~Jacquet.
\newblock A relation between automorphic representations of $\mathrm{GL}(2)$
and $\mathrm{GL}(3)$.
\newblock \textit{Ann.\ Sci.\ \'Ecole Norm.\ Sup.}, 11(4):471--542, 1978.

\bibitem{Kisin2009}
M.~Kisin.
\newblock Modularity of 2-adic Barsotti--Tate representations.
\newblock \textit{Invent.\ Math.}, 178(3):587--634, 2009.

\bibitem{KhareWintenberger2009I}
C.~Khare and J.-P.~Wintenberger.
\newblock Serre's modularity conjecture~(I).
\newblock \textit{Invent.\ Math.}, 178(3):485--504, 2009.

\bibitem{KhareWintenberger2009II}
C.~Khare and J.-P.~Wintenberger.
\newblock Serre's modularity conjecture~(II).
\newblock \textit{Invent.\ Math.}, 178(3):505--586, 2009.

\bibitem{BuzzardTaylorFLT}
K.~Buzzard and R.~Taylor.
\newblock Towards a {Lean} proof of {Fermat's Last Theorem}.
\newblock Blueprint, Imperial College London, 2026.
\newblock \url{https://imperialcollegelondon.github.io/FLT/blueprint.pdf}

\bibitem{Langlands1980}
R.\,P.~Langlands.
\newblock \textit{Base Change for $\mathrm{GL}(2)$}.
\newblock Annals of Mathematics Studies~96.
Princeton University Press, 1980.

\bibitem{LMO1979}
J.\,C.~Lagarias, H.\,L.~Montgomery, and A.\,M.~Odlyzko.
\newblock A bound for the least prime ideal in the Chebotarev
density theorem.
\newblock \textit{Invent.\ Math.}, 54(3):271--296, 1979.

\bibitem{MRR1988}
R.~Mines, F.~Richman, and W.~Ruitenburg.
\newblock \textit{A Course in Constructive Algebra}.
\newblock Universitext. Springer, 1988.

\bibitem{McLarty2010}
C.~McLarty.
\newblock What does it take to prove Fermat's Last Theorem?
Grothendieck and the logic of number theory.
\newblock \textit{Bull.\ Symbolic Logic}, 16(3):359--377, 2010.

\bibitem{Rubin1991}
K.~Rubin.
\newblock The ``main conjectures'' of Iwasawa theory for imaginary
quadratic fields.
\newblock \textit{Invent.\ Math.}, 103(1):25--68, 1991.

\bibitem{TaylorWiles1995}
R.~Taylor and A.~Wiles.
\newblock Ring-theoretic properties of certain Hecke algebras.
\newblock \textit{Ann.\ of Math.}, 141(3):553--572, 1995.

\bibitem{Tunnell1981}
J.~Tunnell.
\newblock Artin's conjecture for representations of octahedral type.
\newblock \textit{Bull.\ Amer.\ Math.\ Soc.}, 5(2):173--175, 1981.

\bibitem{Wiles1995}
A.~Wiles.
\newblock Modular elliptic curves and Fermat's Last Theorem.
\newblock \textit{Ann.\ of Math.}, 141(3):443--551, 1995.

%% CRM program references
\bibitem{Paper10}
P.\,C.\,K.~Lee.
\newblock Formalization methodology and constructive stratification
(Paper~10, CRM series).
\newblock \textit{Zenodo}, 2025.

\bibitem{Paper40}
P.\,C.\,K.~Lee.
\newblock The logical constitution of physical reality:
a constructive reverse mathematics synthesis
(Paper~40, CRM series).
\newblock \textit{Zenodo}, 2025.

\bibitem{Paper50}
P.\,C.\,K.~Lee.
\newblock Three axioms for the motive: a decidability characterisation
of Grothendieck's universal cohomology
(Paper~50, CRM series).
\newblock \textit{Zenodo}, 2026.

\bibitem{Paper59}
P.\,C.\,K.~Lee.
\newblock De Rham decidability and DPT completeness
(Paper~59, CRM series).
\newblock \textit{Zenodo}, 2026.

\bibitem{Paper67}
P.\,C.\,K.~Lee.
\newblock Decidability and self-intersection in arithmetic geometry:
a constructive reverse mathematics monograph
(Paper~67, CRM series).
\newblock \textit{Zenodo}, 2026.

\end{thebibliography}

\end{document}

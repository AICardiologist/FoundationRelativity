\documentclass[11pt,a4paper]{article}

% ====================================================================
% Packages
% ====================================================================
\usepackage[utf8]{inputenc}
\usepackage[T1]{fontenc}
\usepackage{amsmath,amssymb,amsthm}
\usepackage{mathtools}
\usepackage{hyperref}
\usepackage[margin=1in]{geometry}
\usepackage{enumitem}
\usepackage{booktabs}
\usepackage{listings}
\usepackage{xcolor}
\usepackage{cleveref}
\usepackage{natbib}
\usepackage{mdframed}

% ====================================================================
% Theorem environments
% ====================================================================
\theoremstyle{plain}
\newtheorem{theorem}{Theorem}[section]
\newtheorem{lemma}[theorem]{Lemma}
\newtheorem{proposition}[theorem]{Proposition}
\newtheorem{corollary}[theorem]{Corollary}

\theoremstyle{definition}
\newtheorem{definition}[theorem]{Definition}
\newtheorem{remark}[theorem]{Remark}

% ====================================================================
% Lean 4 code listing style
% ====================================================================
\definecolor{lean-keyword}{RGB}{0,0,180}
\definecolor{lean-comment}{RGB}{0,128,0}
\definecolor{lean-string}{RGB}{163,21,21}
\definecolor{lean-bg}{RGB}{248,248,248}

\lstdefinelanguage{lean4}{
  keywords={theorem,lemma,def,class,instance,import,open,variable,
            noncomputable,section,namespace,end,where,let,have,show,
            intro,obtain,use,exact,rw,simp,apply,by,fun,match,if,
            then,else,do,return,axiom,abbrev,private,attribute,
            suffices,change,congr,ext,constructor,rintro,push_neg,
            linarith,absurd,set_option,omit,in,set,cases,structure,
            refine,unfold,rcases,calc,all_goals,first,try,ring,
            positivity,induction},
  sensitive=true,
  morecomment=[l]{--},
  morecomment=[s]{/-}{-/},
  morestring=[b]",
  morestring=[b]',
}

\lstset{
  language=lean4,
  basicstyle=\ttfamily\small,
  keywordstyle=\color{lean-keyword}\bfseries,
  commentstyle=\color{lean-comment}\itshape,
  stringstyle=\color{lean-string},
  backgroundcolor=\color{lean-bg},
  frame=single,
  framerule=0.5pt,
  breaklines=true,
  breakatwhitespace=true,
  tabsize=2,
  showstringspaces=false,
  numbers=left,
  numberstyle=\tiny\color{gray},
  numbersep=5pt,
  xleftmargin=15pt,
  captionpos=b,
}

% ====================================================================
% Macros
% ====================================================================
\newcommand{\NN}{\mathbb{N}}
\newcommand{\RR}{\mathbb{R}}
\newcommand{\ZZ}{\mathbb{Z}}
\newcommand{\LPO}{\mathrm{LPO}}
\newcommand{\WLPO}{\mathrm{WLPO}}
\newcommand{\BMC}{\mathrm{BMC}}
\newcommand{\NPSC}{\mathrm{NPSC}}
\newcommand{\BISH}{\mathrm{BISH}}
\newcommand{\Lean}{\textsc{Lean~4}}
\newcommand{\Mathlib}{\textsc{Mathlib4}}
\newcommand{\leanok}{\textsf{\small \textcolor{green!70!black}{\checkmark}}}

% ====================================================================
% Title
% ====================================================================
\title{%
  \textbf{Noether's Theorem and the Logical Cost of Global Conservation Laws}\\[6pt]
  {\normalsize Paper~15 in the Constructive Reverse Mathematics Series}%
}

\author{
  Paul Chun-Kit Lee\thanks{%
    New York University.
    AI-assisted formalization; see \S\ref{sec:ai} for methodology.
    The author is a medical professional, not a domain expert in
    constructive mathematics or mathematical physics; mathematical
    content was developed with extensive AI assistance.} \\
  New York University \\
  \texttt{dr.paul.c.lee@gmail.com}
}

\date{February 2026}

% ====================================================================
\begin{document}
\maketitle

% ====================================================================
\begin{abstract}
Noether's theorem---that every continuous symmetry of the action
yields a conserved current---is the most fundamental structural
principle in physics. We show it splits cleanly across the
constructive hierarchy. The local conservation law
($\partial_\mu J^\mu = 0$) is an algebraic identity provable
in Bishop's constructive mathematics ($\BISH$). The finite-volume
conserved charge is a computable real number ($\BISH$). The global
conserved charge---the assertion that the total energy of an
infinite system exists as a definite real number---is equivalent
to the Limited Principle of Omniscience ($\LPO$) via Bounded
Monotone Convergence ($\BMC$), precisely when the conserved
density is non-negative (as for energy, via the weak energy
condition). All results are formalised in \Lean{} with
machine-checked proofs (520~lines, zero \texttt{sorry}).
This constitutes a fourth independent domain---after statistical
mechanics, general relativity, and quantum decoherence---exhibiting
the same $\BISH$/$\LPO$ boundary at bounded monotone convergence,
and the first to calibrate a \emph{structural law} rather than a
physical prediction.
\end{abstract}

\vspace{1em}
\begin{center}
\textit{Dedicated to Mimi, my wife.}
\end{center}
\vspace{0.5em}
\begin{flushright}
\textit{Soli Deo gloria}
\end{flushright}
\vspace{1em}

\tableofcontents

% ====================================================================
\section{Introduction}\label{sec:intro}
% ====================================================================

\subsection{Physical Context}\label{sec:physical}

In 1918, Emmy Noether proved that every continuous symmetry of the
action functional yields a conserved current~\cite{Noether1918}.
The theorem generates conservation of energy (time translation),
momentum (space translation), angular momentum (rotation), electric
charge ($U(1)$ gauge), and every other conservation law in the
Standard Model. It is, in the words of Lederman and Hill, among
the most important results guiding the development of modern
physics~\cite{Byers1998}.

This paper addresses a precise question: \textbf{What is the
logical cost of Noether's theorem?} More specifically: does the
theorem, or its consequences, require non-constructive principles?

\subsection{The Answer}\label{sec:answer}

The answer decomposes into two layers:
\begin{itemize}
  \item The \textbf{local conservation law}---the algebraic
    identity $\partial_\mu J^\mu = 0$ following from the
    equations of motion---is $\BISH$ (Height~0). It is a
    finite algebraic manipulation of sums over finitely many
    lattice sites. No omniscience principle is needed.
  \item The \textbf{global conserved charge}---the assertion
    that the total energy $E = \lim_{N\to\infty} E_N$ exists
    as a definite real number for every bounded field
    configuration with non-negative energy density---is
    equivalent to $\LPO$.
\end{itemize}

The mechanism is the same as in the three preceding domains:
the partial energies $E_N = \sum_{i<N} \varepsilon_i$ form a
bounded monotone sequence (bounded by hypothesis, monotone
because each $\varepsilon_i \ge 0$). Asserting the completed
limit is $\BMC$, which is $\LPO$.

\subsection{Programme Context}\label{sec:context}

This is the fourth independent domain in a programme of
constructive calibration of mathematical physics. The previous
three domains are:
\begin{itemize}
  \item \textbf{Statistical mechanics} (Paper~8)~\cite{Lee26-P8}:
    The 1D~Ising free energy $f_N$ is $\BISH$; the thermodynamic
    limit $f_\infty = \lim f_N$ is $\LPO$ via $\BMC$.
  \item \textbf{General relativity} (Paper~13)~\cite{Lee26-P13}:
    Finite geodesic computations in the Schwarzschild interior
    are $\BISH$; the assertion $r(\tau) \to 0$ (geodesic
    incompleteness) is $\LPO$ via $\BMC$.
  \item \textbf{Quantum mechanics} (Paper~14)~\cite{Lee26-P14}:
    Finite-step decoherence bounds are $\BISH$; exact decoherence
    (the assertion that coherence converges to a definite limit)
    is $\LPO$ via $\BMC$.
\end{itemize}

All four domains produce bounded monotone sequences whose
completed limits cost exactly $\LPO$. The physics differs
completely---partition functions, geodesic equations, density
matrices, energy densities---but the logical structure is
identical.

\subsection{What Makes Paper~15 Different}\label{sec:different}

Papers~8, 13, and~14 calibrate \emph{predictions}: quantities
physicists compute and compare to experiment (free energy values,
singularity formation, measurement outcomes). Paper~15 calibrates
a \emph{structural law}---the principle that symmetries generate
conservation laws. The local form of this law (the algebraic
content of Noether's theorem) is constructive. The global form
(total charge exists) is the idealisation. This is the first result
in the programme to calibrate a structural principle rather than a
physical prediction.


% ====================================================================
\section{Background}\label{sec:background}
% ====================================================================

\subsection{Constructive Reverse Mathematics}\label{sec:crm-bg}

Constructive reverse mathematics (CRM) classifies mathematical
theorems by the weakest omniscience principle needed to prove
them~\cite{Bishop67,BV06,Ishihara06}. Bishop's constructive
mathematics ($\BISH$) avoids all omniscience principles; every
existential claim comes with a computable witness.

The key principles for this paper are:
\begin{itemize}
  \item \textbf{LPO} (Limited Principle of Omniscience): For every
    binary sequence $\alpha : \NN \to \{0,1\}$, either all terms
    are zero or some term is one. Equivalent to: every bounded
    monotone sequence of reals converges ($\BMC$). The equivalence
    is due to Bridges and V\^{\i}\c{t}\u{a}~\cite{BV06}
    (Theorem~2.1.5).
  \item \textbf{BMC} (Bounded Monotone Convergence): Every
    monotone sequence bounded above has a limit.
  \item \textbf{NPSC} (Nonnegative Partial Sum Convergence): Every
    bounded sequence of partial sums of non-negative terms converges.
    A new abstraction introduced in this paper, equivalent to $\BMC$
    and hence to $\LPO$. This is the abstract framework connecting
    energy partial sums to the omniscience hierarchy.
\end{itemize}

\subsection{The Diagnostic}\label{sec:diagnostic}

When a physical theorem asserts the existence of a completed
infinite limit (thermodynamic limit, singularity, decoherence,
global charge), the diagnostic is: check whether the underlying
sequence is bounded and monotone. If yes, the assertion costs
exactly $\LPO$. If the sequence has a computable modulus of
convergence, the limit is $\BISH$ and $\LPO$ is not needed. The
programme identifies physical theorems where the modulus is
naturally absent---where the logical cost is an intrinsic feature
of the physics, not an artefact of a particular proof strategy.


% ====================================================================
\section{The Physics: Noether's Theorem on the Lattice}
\label{sec:physics}
% ====================================================================

\subsection{The Model}\label{sec:model}

We work with a real scalar field on a 1D lattice with periodic
boundary conditions. The lattice has $N$ sites, with field values
$\varphi_i \in \RR$ and conjugate momenta $\pi_i \in \RR$ at each
site $i \in \mathrm{Fin}\;N$.

The lattice Lagrangian is:
\begin{equation}\label{eq:lagrangian}
  L_N = \sum_{i=0}^{N-1} \left[
    \tfrac{1}{2}\pi_i^2
    - \tfrac{1}{2}(\varphi_{i+1} - \varphi_i)^2
    - V(\varphi_i)
  \right],
\end{equation}
with $V(\varphi) \ge 0$ (non-negative potential) and periodic
boundary conditions: $\varphi_N = \varphi_0$.

The equations of motion (discrete Euler--Lagrange) are:
\begin{equation}\label{eq:eom}
  \dot{\varphi}_i = \pi_i, \qquad
  \dot{\pi}_i = \Delta_d \varphi_i - V'(\varphi_i),
\end{equation}
where $\Delta_d \varphi_i = \varphi_{i+1} - 2\varphi_i +
\varphi_{i-1}$ is the discrete Laplacian.

The conserved energy (from time-translation invariance) is:
\begin{equation}\label{eq:energy}
  E_N = \sum_{i=0}^{N-1} \left[
    \tfrac{1}{2}\pi_i^2
    + \tfrac{1}{2}(\varphi_{i+1} - \varphi_i)^2
    + V(\varphi_i)
  \right].
\end{equation}
Every term in $E_N$ is non-negative: $\tfrac{1}{2}\pi_i^2 \ge 0$
(kinetic), $\tfrac{1}{2}(\Delta\varphi_i)^2 \ge 0$ (gradient),
$V(\varphi_i) \ge 0$ (potential). This positivity is the critical
property.

\subsection{Why the Lattice}\label{sec:lattice}

We work on the lattice rather than in the continuum for three
reasons:
\begin{enumerate}
  \item \textbf{Physical honesty.} Lattice field theory is how
    physics is actually computed (lattice QCD, lattice gauge
    theory~\cite{Creutz1983}). The continuum limit is itself an
    idealisation whose logical cost could be separately calibrated.
  \item \textbf{Lean practicality.} Everything is finite sums
    over $\mathrm{Fin}\;N$. No Sobolev spaces, no distributions,
    no functional analysis infrastructure.
  \item \textbf{Structural completeness.} The lattice model
    captures all three levels of the hierarchy: local conservation
    (algebraic identity), finite-volume energy (computable), and
    infinite-volume energy ($\BMC$).
\end{enumerate}

The mathematical theory of discrete Noether theorems is developed
in Skopenkov~\cite{Skopenkov2023}.

\subsection{The Sign Observation}\label{sec:sign-obs}

The logical cost of a global conservation law depends on whether
the conserved density is positive definite:
\begin{itemize}
  \item \textbf{Energy density} ($T^{00}$): Sum of squares plus
    non-negative potential. Always $\ge 0$. Partial energies $E_R$
    are monotone increasing in $R$. $\to$ $\BMC$ $\to$ $\LPO$.
  \item \textbf{Charge density} ($J^0$): Signed (positive for
    particles, negative for antiparticles). Partial charges $Q_R$
    oscillate as $R$ grows. $\to$ Conditional convergence, does
    not fit $\BMC$.
\end{itemize}

This parallels the weak energy condition in general relativity:
$T_{ab}v^a v^b \ge 0$ for all timelike $v^a$~\cite{Witten1981,
Kontou2020}. The positivity of energy density is a physical
assumption. The programme shows this assumption is precisely what
makes the infinite-volume limit cost $\LPO$ rather than something
else.


% ====================================================================
\section{Finite Noether at BISH}\label{sec:bish}
% ====================================================================

\subsection{The Conservation Identity}\label{sec:conservation}

\begin{theorem}[Energy conservation---Noether's theorem]
\label{thm:noether} \leanok{}
For any field configuration $(\varphi, \pi)$ on an $N$-site
lattice ($N \ge 2$) with periodic boundary conditions, and any
potential $V$, the total rate of energy change vanishes
identically:
\[
  \sum_{i=0}^{N-1} \Bigl[
    \pi_i \cdot (\Delta_d\varphi_i - V'(\varphi_i))
    + (\varphi_{i+1} - \varphi_i)(\pi_{i+1} - \pi_i)
    + V'(\varphi_i) \cdot \pi_i
  \Bigr] = 0.
\]
\end{theorem}

\begin{proof}
We compute $\tfrac{dE_N}{dt}$ by substituting the equations of
motion~\eqref{eq:eom} into the time derivative of
$E_N$~\eqref{eq:energy}. Each summand receives three
contributions: from the kinetic term
$\pi_i \cdot \dot{\pi}_i = \pi_i(\Delta_d\varphi_i - V'(\varphi_i))$,
from the gradient term
$(\varphi_{i+1} - \varphi_i)(\dot{\varphi}_{i+1} - \dot{\varphi}_i)
= (\varphi_{i+1} - \varphi_i)(\pi_{i+1} - \pi_i)$,
and from the potential term
$V'(\varphi_i)\dot{\varphi}_i = V'(\varphi_i)\pi_i$.

The first step is algebraic simplification. The $V'(\varphi_i)\pi_i$
terms from the kinetic and potential contributions cancel, leaving:
\[
  \text{summand}_i =
  \bigl(\varphi_{i+1}\pi_{i+1} - \varphi_i\pi_i\bigr)
  + \bigl(\pi_i \cdot \varphi_{i-1} - \varphi_i \cdot \pi_{i+1}\bigr).
\]
This identity is verified by \texttt{ring} in \Lean{} after
expanding the discrete Laplacian.

The sum of the first group is a telescoping sum:
$\sum_i (\varphi_{i+1}\pi_{i+1} - \varphi_i\pi_i) = 0$
under periodic boundary conditions. This follows from
\textbf{shift invariance of periodic sums} (\Cref{lem:shift}):
$\sum_i f(\mathrm{next}(i)) = \sum_i f(i)$.

For the second group, we reindex the sum
$\sum_i \pi_i \cdot \varphi_{i-1}$ using the substitution
$i \mapsto \mathrm{next}(j)$, giving
$\sum_j \pi_{\mathrm{next}(j)} \cdot
\varphi_{\mathrm{prev}(\mathrm{next}(j))}
= \sum_j \pi_{\mathrm{next}(j)} \cdot \varphi_j$,
where we used the inverse relationship
$\mathrm{prev} \circ \mathrm{next} = \mathrm{id}$.
Then by commutativity:
$\sum_j \pi_{\mathrm{next}(j)} \cdot \varphi_j
= \sum_j \varphi_j \cdot \pi_{\mathrm{next}(j)}$,
which equals $\sum_i \varphi_i \cdot \pi_{i+1}$.
The two terms cancel, giving $\sum_i (\text{second group})_i = 0$.

Since both groups sum to zero, $dE_N/dt = 0$.
\end{proof}

\begin{lemma}[Shift invariance of periodic sums]
\label{lem:shift} \leanok{}
For any $f : \mathrm{Fin}\;N \to \RR$ and $N > 0$:
\[
  \sum_{i \in \mathrm{Fin}\;N} f(\mathrm{next}(i))
  = \sum_{i \in \mathrm{Fin}\;N} f(i).
\]
\end{lemma}

\begin{proof}
The function $\mathrm{next} : \mathrm{Fin}\;N \to
\mathrm{Fin}\;N$ defined by $i \mapsto (i+1) \bmod N$ is a
bijection. Its inverse is $\mathrm{prev} : i \mapsto
(i + N - 1) \bmod N$. Bijectivity is established by proving the
inverse relationships $\mathrm{next} \circ \mathrm{prev} =
\mathrm{id}$ and $\mathrm{prev} \circ \mathrm{next} =
\mathrm{id}$ via case analysis on modular arithmetic. The result
then follows from \Mathlib{}'s
\texttt{Fintype.sum\_bijective}.
\end{proof}

\begin{remark}[Logical status]\label{rem:bish-noether}
Every step in the proof of \Cref{thm:noether} is finite
arithmetic over $\mathrm{Fin}\;N$. The telescoping uses periodic
boundary conditions (a finite sum with wrapping indices). No
limits, no compactness, no choice. The \Lean{}
\texttt{\#print axioms} output shows only \texttt{propext},
\texttt{Classical.choice} (from \Mathlib{} infrastructure for
\texttt{Fin.fintype} and \texttt{Real.instField}), and
\texttt{Quot.sound}. No omniscience principles.
\end{remark}

\subsection{Non-negativity of Energy Density}\label{sec:nonneg}

\begin{theorem}[Energy density is non-negative]
\label{thm:nonneg} \leanok{}
If $V(\varphi) \ge 0$ for all $\varphi$, then each term in $E_N$
is non-negative:
\[
  \varepsilon_i =
  \tfrac{1}{2}\pi_i^2
  + \tfrac{1}{2}(\varphi_{i+1} - \varphi_i)^2
  + V(\varphi_i) \ge 0.
\]
\end{theorem}

\begin{proof}
The kinetic term $\tfrac{1}{2}\pi_i^2 \ge 0$ is a square divided
by a positive constant. The gradient term
$\tfrac{1}{2}(\varphi_{i+1} - \varphi_i)^2 \ge 0$ likewise. The
potential term $V(\varphi_i) \ge 0$ by hypothesis. The sum of
non-negative reals is non-negative. In \Lean{}, the proof uses
\texttt{positivity} for the squared terms and the hypothesis
\texttt{hV} for the potential.
\end{proof}

\subsection{Monotonicity of Partial Energies}\label{sec:monotone}

\begin{theorem}[Partial energy is monotone]
\label{thm:mono} \leanok{}
For a field on the infinite lattice $\NN$, the partial energy
$E_N = \sum_{i=0}^{N-1} \varepsilon_i$ is monotone increasing in
$N$ when $V \ge 0$.
\end{theorem}

\begin{proof}
$E_{N+1} = E_N + \varepsilon_N \ge E_N$ since
$\varepsilon_N \ge 0$ by \Cref{thm:nonneg}. The \Lean{} proof
applies \texttt{monotone\_nat\_of\_le\_succ} with the recurrence
\texttt{partialEnergy\_succ}.
\end{proof}


% ====================================================================
\section{Global Energy and LPO}\label{sec:lpo}
% ====================================================================

\subsection{The Abstract Framework: NPSC}\label{sec:npsc}

\begin{definition}[NPSC]\label{def:npsc} \leanok{}
\emph{Nonnegative Partial Sum Convergence} ($\NPSC$): for every
sequence $d : \NN \to \RR$ with $d_n \ge 0$ for all $n$, if the
partial sums $\sum_{i<N} d_i$ are bounded, then they converge to a
definite limit.
\end{definition}

\begin{theorem}[$\NPSC \leftrightarrow \BMC$]
\label{thm:npsc-bmc} \leanok{}
Nonnegative Partial Sum Convergence is equivalent to Bounded
Monotone Convergence.
\end{theorem}

\begin{proof}
\textbf{($\NPSC \Rightarrow \BMC$):}
Given a monotone sequence $a : \NN \to \RR$ bounded above by $M$,
define $d_n = a_{n+1} - a_n \ge 0$ (by monotonicity). The partial
sums telescope:
\[
  \sum_{i<N} d_i = a_N - a_0,
\]
which is bounded above by $M - a_0$. By $\NPSC$, the partial sums
converge to some $L_d$. Therefore $a_N = (\sum_{i<N} d_i) + a_0$
converges to $L_d + a_0$.

The telescoping identity is proved by induction on $N$. The base
case is trivial ($\sum_{i<0} d_i = 0 = a_0 - a_0$). The inductive
step uses \texttt{Finset.sum\_range\_succ}.

\smallskip\noindent
\textbf{($\BMC \Rightarrow \NPSC$):}
Given $d : \NN \to \RR$ with $d_n \ge 0$ and bounded partial sums,
the sequence $S_N = \sum_{i<N} d_i$ is monotone (adding a
non-negative term cannot decrease the sum) and bounded. $\BMC$
gives convergence directly.
\end{proof}

\begin{remark}[Logical status]\label{rem:npsc-bish}
The equivalence $\NPSC \leftrightarrow \BMC$ is fully proved in
\Lean{} with \emph{no custom axioms}. The
\texttt{\#print axioms npsc\_iff\_bmc} output shows only
\texttt{[propext, Classical.choice, Quot.sound]}---standard
\Mathlib{} infrastructure. This is the same pattern as Paper~14's
\texttt{abc\_iff\_bmc}: the abstract equivalence is pure $\BISH$.
\end{remark}

\begin{theorem}[$\NPSC \leftrightarrow \LPO$]
\label{thm:npsc-lpo} \leanok{}
\[
  \NPSC \;\;\longleftrightarrow\;\; \LPO.
\]
\end{theorem}

\begin{proof}
Compose $\NPSC \leftrightarrow \BMC$ (\Cref{thm:npsc-bmc}) with
$\BMC \leftrightarrow \LPO$ (Bridges--V\^{\i}\c{t}\u{a}~\cite{BV06}).
In \Lean{}: \texttt{npsc\_iff\_bmc} composed with
\texttt{lpo\_iff\_bmc.symm}.
\end{proof}

\subsection{The Headline Result}\label{sec:headline}

\begin{theorem}[Global energy existence $\leftrightarrow$ LPO]
\label{thm:main} \leanok{}
The assertion ``for every bounded field configuration with
$V \ge 0$, the total energy
$E = \lim_{N\to\infty} E_N$ exists'' is equivalent to $\LPO$:
\[
  \Bigl(\forall V, \varphi, \pi, M.\;
  V \ge 0 \to
  (\forall N,\; E_N \le M) \to
  \exists E.\; E_N \to E \Bigr)
  \;\;\longleftrightarrow\;\; \LPO.
\]
\end{theorem}

\begin{proof}
\textbf{(Forward: Global energy $\Rightarrow$ $\LPO$.)}
Suppose the global energy assertion holds. We show $\NPSC$ holds
(and hence $\LPO$, by \Cref{thm:npsc-lpo}).

Given $d : \NN \to \RR$ with $d_n \ge 0$ and bounded partial sums,
we encode $d$ into a lattice field configuration as follows. Set
$V = 0$ (zero potential, which is $\ge 0$), $\varphi = 0$
(zero field), and
$\pi_i = \sqrt{2 \cdot d_i}$. Then the energy density at site $i$
reduces to
\[
  \varepsilon_i = \tfrac{1}{2}\pi_i^2 + 0 + 0
  = \tfrac{1}{2} \cdot 2d_i = d_i,
\]
where we used $(\sqrt{x})^2 = x$ for $x \ge 0$. Therefore
$E_N = \sum_{i<N} d_i$, and the boundedness hypothesis carries
over. The global energy assertion gives convergence.

\smallskip\noindent
\textbf{(Reverse: $\LPO$ $\Rightarrow$ Global energy.)}
$\LPO \Rightarrow \BMC$ (by the Bridges--V\^{\i}\c{t}\u{a}
equivalence). The partial energy sequence is monotone
(\Cref{thm:mono}) and bounded by hypothesis. $\BMC$ gives
convergence.
\end{proof}

\subsection{Axiom Certificate}\label{sec:axiom-cert}

The \Lean{} \texttt{\#print axioms} output for
\texttt{global\_energy\_iff\_LPO} shows:
\texttt{[propext, Classical.choice, Quot.sound, bmc\_of\_lpo,
lpo\_of\_bmc]}. The only custom axioms are the two interface
assumptions for the Bridges--V\^{\i}\c{t}\u{a} equivalence,
cited from~\cite{BV06} and Paper~8~\cite{Lee26-P8}.

The abstract framework (\texttt{npsc\_iff\_bmc}) is pure
$\BISH$---no custom axioms.


% ====================================================================
\section{The Sign Trap: Why Charge Doesn't Work}\label{sec:sign}
% ====================================================================

The conservation identity (\Cref{thm:noether}) holds for
\emph{any} conserved current, not just energy. But the $\LPO$
equivalence (\Cref{thm:main}) depends critically on the positivity
of the conserved density. For signed densities, the argument fails.

For a complex scalar field with $U(1)$ symmetry, the conserved
charge density is
\[
  J^0 = i(\varphi^* \dot{\varphi}
  - \dot{\varphi}^* \varphi),
\]
which is signed: positive for particles, negative for
antiparticles. The partial charge
$Q_R = \sum_{i=0}^{R-1} J^0_i$ is \emph{not} monotone. Adding
more sites can increase or decrease $Q_R$ depending on the charge
distribution.

The convergence of $Q_R$ as $R \to \infty$ is a conditional
convergence problem, not a monotone convergence problem. It does
not map to $\BMC$. Asserting convergence requires either:
\begin{itemize}
  \item Explicit decay rates (collapsing to $\BISH$), or
  \item Stronger principles than $\LPO$ (e.g., Bolzano--Weierstrass
    for subsequences).
\end{itemize}

The lesson is that the constructive calibration depends not just
on the theorem (Noether) but on the \emph{physical content} of
the conserved quantity. Positivity (the weak energy condition) is
what makes the $\BMC$ pattern available. The sign structure of the
density is logically significant.

This is analogous to the distinction in Paper~14 between
uniform-angle decoherence (geometric decay, $\BISH$) and
variable-angle decoherence (general monotone, $\LPO$). The
specific model determines where in the hierarchy the result lands.


% ====================================================================
\section{Domain Invariance}\label{sec:invariance}
% ====================================================================

\subsection{The Four-Domain Table}\label{sec:table}

Paper~15 is the fourth physical domain producing the
$\BMC \leftrightarrow \LPO$ pattern:

\begin{center}
\begin{tabular}{@{}p{2.5cm}p{1.0cm}p{3.0cm}p{2.0cm}p{3.3cm}@{}}
\toprule
\textbf{Domain} & \textbf{Paper} &
  \textbf{Bounded Monotone Seq.} & \textbf{BISH Content} &
  \textbf{LPO Content} \\
\midrule
Stat.\ Mech. & 8 & Free energy $f_N$ &
  $f_N$ computed & $f_\infty = \lim f_N$ \\
Gen.\ Rel. & 13 & Radial coord.\ $r(\tau)$ &
  $r(\tau)$ to $\varepsilon$ & $r(\tau) \to 0$ \\
Quantum Meas. & 14 & Coherence $c(N)$ &
  $c(N) < \varepsilon$ & $c(N) \to 0$ (collapse) \\
\textbf{Conserv.\ Laws} & \textbf{15} &
  \textbf{Partial energy $E_N$} &
  \textbf{$dE/dt = 0$, $E_N$} &
  \textbf{$E = \lim E_N$} \\
\bottomrule
\end{tabular}
\end{center}

\subsection{What Domain Invariance Means}\label{sec:meaning}

Four independent physical domains. Different physics:
Paper~8 uses partition functions and Boltzmann weights;
Paper~13 uses geodesic equations and the Schwarzschild metric;
Paper~14 uses density matrices and partial trace;
Paper~15 uses Lagrangians, symmetry flows, and energy densities.

Same logical structure: bounded monotone sequence, finite
truncations at $\BISH$, completed limit at $\LPO$. Same abstract
equivalence ($\BMC \leftrightarrow \LPO$) instantiated in each
domain.

One instance is a result. Two is a pattern. Three is evidence.
Four begins to look like a structural feature of how physical
theories relate to mathematical frameworks.

\subsection{What Paper~15 Adds}\label{sec:adds}

Paper~15 is qualitatively different from the other three.
Papers~8, 13, and~14 calibrate \emph{predictions}---quantities
physicists compute and compare to experiment. Paper~15 calibrates
a \emph{principle}---the structural law that symmetries generate
conservation laws.

The local form (Noether's algebraic identity) is constructive.
The global form (total charge exists) is the idealisation. This
means: \textbf{the architecture of physical law is constructive.}
The idealisation enters not in the law itself but in the totality
assertion---the claim that the conserved quantity exists as a
definite number summed over all of space. This is a statement about
formalism, not nature. No finite experiment distinguishes
``total energy $= E$'' from ``partial energy $E_N$ agrees with $E$
to within $\varepsilon$ for all $N$ we can probe.''

\subsection{Full Calibration Table}\label{sec:calibration}

The calibration table for the full series, updated with Paper~15:

\begin{center}
\begin{tabular}{@{}llll@{}}
\toprule
\textbf{Physical layer} & \textbf{Principle} & \textbf{Status} &
\textbf{Source} \\
\midrule
Finite-volume Gibbs states & $\BISH$ & Calibrated & Trivial \\
Finite-size approximations (Ising) & $\BISH$ & Calibrated &
  Paper~8 \\
Schwarzschild exterior & $\BISH$ & Calibrated & Paper~1 \\
Interior finite-time physics & $\BISH$ & Calibrated & Paper~13 \\
Tsirelson bound (CHSH $\le 2\sqrt{2}$) & $\BISH$ & Calibrated &
  Paper~11 \\
Bell state entropy ($\log 2$) & $\BISH$ & Calibrated &
  Paper~11 \\
Finite-step decoherence & $\BISH$ & Calibrated & Paper~14 \\
\textbf{Local conservation ($dE/dt = 0$)} & \textbf{BISH} &
  \textbf{Calibrated} & \textbf{Paper~15} \\
Bidual-gap witness ($S_1(H)$) & $\equiv \WLPO$ & Calibrated &
  Papers~2, 7 \\
Thermodynamic limit (Ising) & $\equiv \LPO$ & Calibrated &
  Paper~8 \\
Geodesic incompleteness & $\equiv \LPO$ & Calibrated &
  Paper~13 \\
Exact decoherence (collapse) & $\equiv \LPO$ & Calibrated &
  Paper~14 \\
\textbf{Global energy existence} & $\equiv$ \textbf{LPO} &
  \textbf{Calibrated} & \textbf{Paper~15} \\
\bottomrule
\end{tabular}
\end{center}

\noindent
The pattern is consistent: all $\LPO$ costs arise from completed
infinite limits; all finite-time and finite-size physics is $\BISH$.


% ====================================================================
\section{Lean Formalisation}\label{sec:lean}
% ====================================================================

\subsection{Module Structure}\label{sec:modules}

The formalisation is organized as a single \Lean{} project with
6~modules:

\begin{table}[ht]
\centering
\begin{tabular}{@{}lrl@{}}
\toprule
\textbf{Module} & \textbf{Lines} & \textbf{Content} \\
\midrule
\texttt{Defs.lean}              & 157 & Lattice field types,
  energy density, fnext/fprev, non-negativity \\
\texttt{LocalConservation.lean} & 190 & Periodic BC shift lemmas,
  Noether's theorem ($dE/dt = 0$) \\
\texttt{Monotonicity.lean}      &  68 & Partial energy recurrence,
  monotonicity, non-negativity \\
\texttt{LPO\_BMC.lean}          &  57 & LPO, BMC definitions,
  axiomatised equivalence \\
\texttt{GlobalEnergy.lean}      & 200 & NPSC framework,
  npsc\_iff\_bmc, encoding, headline LPO theorem \\
\texttt{Main.lean}              & 100 & Assembly +
  \texttt{\#print axioms} audit \\
\midrule
\textbf{Total}                  & \textbf{$\sim$520} & \\
\bottomrule
\end{tabular}
\caption{Module structure of Paper~15.}
\label{tab:modules}
\end{table}

\subsection{Key Design Decisions}\label{sec:design}

\paragraph{Periodic boundary conditions via explicit index wrapping.}
The lattice uses $\mathrm{Fin}\;N$ indices with explicit wrapping
functions \texttt{fnext} ($i \mapsto (i+1) \bmod N$) and
\texttt{fprev} ($i \mapsto (i+N-1) \bmod N$). These are proved to
be mutual inverses via explicit modular arithmetic:
\texttt{fnext\_fprev} and \texttt{fprev\_fnext}. Bijectivity then
follows from the inverse relationships, which is cleaner than
direct injectivity/surjectivity proofs involving nested modular
arithmetic.

\paragraph{Dual lattice models.}
The formalisation uses two complementary lattice models: a finite
lattice with periodic boundary conditions ($\mathrm{Fin}\;N$) for
the conservation identity, and an infinite lattice ($\NN$) with
open boundary conditions for the partial energy and $\LPO$
equivalence. The finite lattice captures the algebraic content of
Noether's theorem; the infinite lattice captures the passage to
the global limit.

\paragraph{Standalone package.}
Paper~15 is a self-contained Lake package. It cannot import
Paper~8 or~14 as Lake dependencies, so $\LPO$, $\BMC$, and the
equivalence are re-defined. The
$\BMC \leftrightarrow \LPO$ equivalence is axiomatised with
citation.

\subsection{Core Definitions}\label{sec:core-defs}

\begin{lstlisting}[caption={Lattice state and energy density (Defs.lean, excerpt).}]
/-- A state of the lattice scalar field. -/
structure LatticeState (N : Nat) where
  phi : Fin N -> Real
  pi  : Momenta N

/-- Next site index with periodic BC. -/
def fnext (hN : 0 < N) (i : Fin N) : Fin N :=
  <<(i.val + 1) % N, Nat.mod_lt _ hN>>

/-- Energy density at site i. -/
def energyDensity (V : Real -> Real) (hN : 0 < N)
    (s : LatticeState N) (i : Fin N) : Real :=
  kineticDensity s i + gradientDensity hN s i
    + potentialDensity V s i
\end{lstlisting}

\begin{lstlisting}[caption={Noether's theorem (LocalConservation.lean, excerpt).}]
/-- The total rate of energy change expression. -/
def totalEnergyRate (hN : 0 < N)
    (phi pi : Fin N -> Real) (V' : Real -> Real) :=
  sum i : Fin N,
    (pi i * (discreteLaplacian hN phi i - V' (phi i))
     + (phi (fnext hN i) - phi i) *
       (pi (fnext hN i) - pi i)
     + V' (phi i) * pi i)

/-- Noether's theorem: energy is conserved. -/
theorem noether_conservation (hN : 2 <= N)
    (phi pi : Fin N -> Real) (V' : Real -> Real) :
    totalEnergyRate (by omega) phi pi V' = 0
\end{lstlisting}

\begin{lstlisting}[caption={NPSC and headline theorem (GlobalEnergy.lean, excerpt).}]
/-- Nonnegative Partial Sum Convergence. -/
def NPSC : Prop :=
  forall (d : Nat -> Real), (forall n, 0 <= d n) ->
    (exists M, forall N,
      sum i in Finset.range N, d i <= M) ->
    exists L, forall e, 0 < e -> exists N0,
      forall N, N0 <= N ->
        |sum i in Finset.range N, d i - L| < e

/-- NPSC <-> BMC (pure BISH, no omniscience). -/
theorem npsc_iff_bmc : NPSC <-> BMC

/-- Global energy existence <-> LPO. -/
theorem global_energy_iff_LPO :
    (forall V, (forall x, 0 <= V x) ->
      forall phi pi M,
        (forall N, partialEnergy V phi pi N <= M) ->
      exists E, forall e, 0 < e -> exists N0,
        forall N, N0 <= N ->
          |partialEnergy V phi pi N - E| < e)
    <-> LPO
\end{lstlisting}

\subsection{Axiom Audit}\label{sec:axioms}

The \texttt{Main.lean} module audits the axiom profile of each
theorem:

\begin{lstlisting}[caption={Axiom audit (Main.lean, selected).}]
-- Part A (BISH):
#print axioms noether_conservation
-- [propext, Classical.choice, Quot.sound]

#print axioms energyDensity_nonneg
-- [propext, Classical.choice, Quot.sound]

#print axioms partialEnergy_mono
-- [propext, Classical.choice, Quot.sound]

-- Part B (LPO equivalence):
#print axioms npsc_iff_bmc
-- [propext, Classical.choice, Quot.sound]
-- (no custom axioms!)

#print axioms global_energy_iff_LPO
-- [propext, Classical.choice, Quot.sound,
--  bmc_of_lpo, lpo_of_bmc]
\end{lstlisting}

\noindent
The \texttt{Classical.choice} appearing in the $\BISH$ results
arises from \Mathlib{}'s real number infrastructure---specifically,
\texttt{Fin.fintype} (for finite sums) and
\texttt{Real.instField} (for real arithmetic). The mathematical
content of these proofs is constructive: they involve only
explicit finite-sum manipulation and algebraic identities.
The constructive calibration is established by proof-content
analysis, following the methodology described in
Paper~10~\cite{Lee26-P10}.

\subsection{CRM Audit}\label{sec:crm-audit}

The formalisation passes the CRM standard established in
Papers~8, 13, and~14:
\begin{itemize}
  \item Clean stratification: Part~A never touches $\LPO$ axioms.
  \item Only declared axioms are \texttt{bmc\_of\_lpo} and
    \texttt{lpo\_of\_bmc}, properly cited.
  \item Abstract framework (\texttt{npsc\_iff\_bmc}) is pure
    $\BISH$.
  \item \texttt{by\_cases} in the inverse proofs uses decidable
    $\NN$ equality (\texttt{instDecidableEqNat}, zero axioms),
    not \texttt{Classical.em}.
  \item \texttt{Nat.eq\_or\_lt\_of\_le} and
    \texttt{Nat.succ\_le\_of\_lt} are axiom-free.
  \item No \texttt{Classical.em}, no
    \texttt{Classical.byContradiction}, no \texttt{decide} on
    propositions anywhere in the source code.
\end{itemize}


% ====================================================================
\section{Discussion}\label{sec:discussion}
% ====================================================================

\subsection{The Local/Global Distinction}\label{sec:local-global}

The result sharpens the local/global distinction in physics. Local
conservation ($\partial_\mu J^\mu = 0$) is the empirical
content---it determines what happens in any finite region. Global
conservation (total charge exists) is the totalising assertion---it
sums over all of space.

The constructive hierarchy reveals these are logically different
claims. The local law is algebraic ($\BISH$). The global assertion
is infinitary ($\LPO$). Physicists routinely treat them as
equivalent (``conservation of energy''), but the programme shows
they have different logical costs.

This has implications for quantum gravity, where the definition of
total energy is notoriously problematic (no local energy density
for the gravitational field; see Witten~\cite{Witten1981} for the
positive energy theorem). The programme suggests this difficulty
is not an accident---total energy is an $\LPO$-level assertion,
and it becomes harder to formulate precisely in regimes where the
positive-definiteness of $T^{00}$ is not guaranteed.

\subsection{Why Four Domains Matter}\label{sec:four}

The accumulation of four independent domains all producing
$\BMC \leftrightarrow \LPO$ at the boundary between finite physics
and infinite formalism suggests this is a structural feature, not a
coincidence. There is no obvious physical reason why Ising
partition functions, Schwarzschild geodesics, decoherence
coherence, and energy densities should share logical structure.
They describe completely different phenomena.

The common feature is that all four involve sequences of finite
computations ($\BISH$) whose completed limits ($\LPO$) are
physically meaningful but experimentally indistinguishable from
sufficiently close finite approximations. The programme has not
proved this is universal---there may be physical theorems whose
logical cost is different (e.g., dependent choice rather than
$\LPO$, as in the strong law of large numbers). But four
independent data points sharing the same boundary is suggestive.

\subsection{Limitations}\label{sec:limitations}

\begin{enumerate}
  \item \textbf{Lattice model.} We work on a lattice, not in the
    continuum. The continuum limit is itself an idealisation whose
    logical cost could be separately calibrated. We expect it to
    introduce additional non-constructive content (function spaces,
    compactness), but this is not formalised here.

  \item \textbf{Encoding is abstract.} The reverse direction
    (global energy $\Rightarrow$ $\LPO$) uses an encoding that
    maps arbitrary bounded monotone sequences to lattice field
    configurations. The resulting configurations (zero field, zero
    gradient, $\pi_i = \sqrt{2d_i}$) may not correspond to any
    realistic physical system. This is standard in reverse
    mathematics---the encoding in Paper~8 similarly produces
    non-physical Ising configurations.

  \item \textbf{Energy only.} The result applies cleanly only to
    positive-definite conserved densities (energy, probability,
    number density). Signed densities (electric charge) do not
    fit the $\BMC$ pattern (\Cref{sec:sign}). A complete
    calibration of all Standard Model conservation laws would
    require separate analysis for each type of charge.

  \item \textbf{Classical.choice is a \Mathlib{} infrastructure
    artifact.} The proof-content analysis methodology for handling
    this is described in Paper~10~\cite{Lee26-P10}.
\end{enumerate}


% ====================================================================
\section{Conclusion}\label{sec:conclusion}
% ====================================================================

Noether's theorem splits across the constructive hierarchy. The
local conservation law---the algebraic content of the theorem---is
$\BISH$. The global conserved charge---the assertion that total
energy exists---is $\LPO$, equivalent to bounded monotone
convergence.

This is the fourth independent domain in which the $\BISH$/$\LPO$
boundary falls at exactly the same place: the passage from finite
computation to completed infinite limit. Four domains (statistical
mechanics, general relativity, quantum measurement, conservation
laws) with completely different physics, all producing bounded
monotone sequences whose completed limits cost $\LPO$.

The result calibrates not just a physical prediction but a
structural principle. The architecture of physical law---the
connection between symmetry and conservation---is constructive.
The idealisation enters through the totality assertion, not
through the law itself.


% ====================================================================
\section*{AI-Assisted Methodology}\label{sec:ai}
% ====================================================================

This formalization was developed using \textbf{Claude Opus~4.6}
(Anthropic, 2026) via the \textbf{Claude Code} command-line
interface, following the same human--AI workflow as Papers~2, 7,
8, 13, and~14~\cite{Lee26-P2,Lee26-P7,Lee26-P8,Lee26-P13,
Lee26-P14,Anthropic2026}.

The author is a medical professional, not a domain expert in
constructive mathematics or mathematical physics. The mathematical
content of this paper was developed with extensive AI assistance.
The human author specified the research direction and high-level
goals, reviewed all mathematical claims for plausibility, and
directed the formalisation strategy. Claude Opus~4.6 explored the
\Mathlib{} codebase, generated \Lean{} proof terms, handled
debugging, and assisted with paper writing. Final verification
was by \texttt{lake build} (0~errors, 0~warnings, 0~sorries).

\begin{table}[h]
\centering
\begin{tabular}{@{}lll@{}}
\toprule
\textbf{Task} & \textbf{Human} & \textbf{AI (Claude Opus 4.6)} \\
\midrule
Research direction       & \checkmark & \\
Mathematical blueprint   & \checkmark & \checkmark \\
Proof strategy design    & \checkmark & \checkmark \\
\Mathlib{} API discovery & & \checkmark \\
\Lean{} proof generation & & \checkmark \\
Proof review             & \checkmark & \\
Build verification       & & \checkmark \\
Paper writing            & \checkmark & \checkmark \\
\bottomrule
\end{tabular}
\caption{Division of labor between human and AI.}
\label{tab:division}
\end{table}


% ====================================================================
\section*{Reproducibility}
% ====================================================================

\begin{mdframed}[backgroundcolor=gray!10]
\textbf{Reproducibility Box}
\begin{itemize}
\item \textbf{Repository}:
  \url{https://github.com/AICardiologist/FoundationRelativity}
\item \textbf{Path}: \texttt{Papers/P15\_Noether/}
\item \textbf{Build}: \texttt{lake exe cache get \&\& lake build}
  (1{,}955 jobs, 0~errors, 0~sorry)
\item \textbf{Lean toolchain}:
  \texttt{leanprover/lean4:v4.28.0-rc1}
\item \textbf{Mathlib version}: commit \texttt{7091f0f6}
\item \textbf{Interface axioms}: \texttt{bmc\_of\_lpo}
  (Bridges--V\^{\i}\c{t}\u{a}~\cite{BV06}),
  \texttt{lpo\_of\_bmc} (Paper~8~\cite{Lee26-P8})
\item \textbf{Axiom audit}: \texttt{Main.lean}
\item \textbf{Axiom profile (main theorem)}:
  \texttt{[propext, Classical.choice, Quot.sound,
  bmc\_of\_lpo, lpo\_of\_bmc]}
\item \textbf{Axiom profile (BISH content)}:
  \texttt{[propext, Classical.choice, Quot.sound]}
  (Mathlib infra only)
\item \textbf{Axiom profile (npsc\_iff\_bmc)}:
  \texttt{[propext, Classical.choice, Quot.sound]}
  (no custom axioms)
\item \textbf{Total}: 6~files, $\sim$520~lines, 0~sorry
\item \textbf{Zenodo DOI}: \href{https://doi.org/10.5281/zenodo.18572494}{10.5281/zenodo.18572494}
\end{itemize}
\end{mdframed}


% ====================================================================
\section*{Acknowledgments}
% ====================================================================

The \Lean{} formalization was developed using Claude Opus~4.6
(Anthropic, 2026) via the Claude Code CLI tool. We thank the
\Mathlib{} community for maintaining the comprehensive library
of formalised mathematics that made this work possible.


% ====================================================================
% Bibliography
% ====================================================================
\bibliographystyle{plainnat}

\begin{thebibliography}{30}

\bibitem[Anthropic(2026)]{Anthropic2026}
Anthropic.
\newblock Claude {Opus}~4.6 and {Claude Code} {CLI}.
\newblock \url{https://www.anthropic.com/claude}, 2026.

\bibitem[Ba{\~n}ados and Reyes(2016)]{Banados2016}
M.~Ba{\~n}ados and I.~A.~Reyes.
\newblock A short review on {Noether}'s theorems, gauge symmetries and
  boundary terms, for students.
\newblock arXiv:1601.03616, 2016.

\bibitem[Bishop(1967)]{Bishop67}
E.~Bishop.
\newblock \emph{Foundations of Constructive Analysis}.
\newblock McGraw-Hill, New York, 1967.

\bibitem[Bishop and Bridges(1985)]{BB85}
E.~Bishop and D.~S.~Bridges.
\newblock \emph{Constructive Analysis}.
\newblock Grundlehren der mathematischen Wissenschaften 279. Springer, 1985.

\bibitem[Bridges and Richman(1987)]{BR87}
D.~S.~Bridges and F.~Richman.
\newblock \emph{Varieties of Constructive Mathematics}.
\newblock London Mathematical Society Lecture Note Series 97.
  Cambridge University Press, 1987.

\bibitem[Bridges and V{\^\i}{\c{t}}{\u{a}}(2006)]{BV06}
D.~S.~Bridges and L.~S.~V{\^\i}{\c{t}}{\u{a}}.
\newblock \emph{Techniques of Constructive Analysis}.
\newblock Universitext. Springer, New York, 2006.

\bibitem[Byers(1998)]{Byers1998}
N.~Byers.
\newblock {E.~Noether}'s discovery of the deep connection between symmetries
  and conservation laws.
\newblock arXiv:physics/9807044, 1998.

\bibitem[Creutz(1983)]{Creutz1983}
M.~Creutz.
\newblock \emph{Quarks, Gluons and Lattices}.
\newblock Cambridge University Press, 1983.

\bibitem[{de Moura} et~al.(2015)]{deMoura2021}
L.~{de Moura}, S.~Kong, J.~Avigad, F.~{van Doorn}, and M.~{von Raumer}.
\newblock The {Lean} theorem prover (system description).
\newblock In \emph{CADE-25}, LNAI 9195, pages 378--388. Springer, 2015.
\newblock Lean~4: \url{https://lean-lang.org/}, 2021--present.

\bibitem[Diener(2018)]{Diener2018}
H.~Diener.
\newblock Constructive reverse mathematics.
\newblock arXiv:1804.05495, 2018.

\bibitem[Ishihara(2006)]{Ishihara06}
H.~Ishihara.
\newblock Reverse mathematics in {Bishop}'s constructive mathematics.
\newblock \emph{Philosophia Scientiae}, Cahier sp\'ecial 6:43--59, 2006.

\bibitem[Ishihara and Nemoto(2019)]{IN2019}
H.~Ishihara and T.~Nemoto.
\newblock The monotone completeness theorem in constructive reverse
  mathematics.
\newblock In \emph{Mathesis Universalis, Computability and Proof},
  Synthese Library 412, Springer, 2019.

\bibitem[Kontou and Sanders(2020)]{Kontou2020}
E.-A.~Kontou and K.~Sanders.
\newblock Energy conditions in general relativity and quantum field theory.
\newblock \emph{Classical and Quantum Gravity}, 37:193001, 2020.
\newblock arXiv:2003.01815.

\bibitem[Kosmann-Schwarzbach(2011)]{KS2011}
Y.~Kosmann-Schwarzbach.
\newblock \emph{The Noether Theorems: Invariance and Conservation Laws
  in the Twentieth Century}.
\newblock Springer, 2011.

\bibitem[Lee(2026a)]{Lee26-P2}
P.~C.-K.~Lee.
\newblock {WLPO} equivalence of the bidual gap in $\ell^1$: a {Lean}~4
  formalization.
\newblock Preprint, 2026. Paper~2 in the constructive reverse
  mathematics series.

\bibitem[Lee(2026b)]{Lee26-P7}
P.~C.-K.~Lee.
\newblock Non-reflexivity of $S_1(H)$ implies {WLPO}: a {Lean}~4
  formalization.
\newblock Preprint, 2026. Paper~7 in the constructive reverse
  mathematics series.

\bibitem[Lee(2026c)]{Lee26-P8}
P.~C.-K.~Lee.
\newblock The logical cost of the thermodynamic limit: {LPO}-equivalence
  and {BISH}-dispensability for the {1D} {Ising} free energy.
\newblock Preprint, 2026. Paper~8 in the constructive reverse
  mathematics series.

\bibitem[Lee(2026d)]{Lee26-P10}
P.~C.-K.~Lee.
\newblock The logical geography of mathematical physics: constructive
  calibration from density matrices to the event horizon.
\newblock Preprint, 2026. Zenodo DOI: 10.5281/zenodo.18527877.
  Paper~10 in the constructive reverse mathematics series.

\bibitem[Lee(2026e)]{Lee26-P13}
P.~C.-K.~Lee.
\newblock The event horizon as a logical boundary: {Schwarzschild} interior
  geodesic incompleteness and {LPO} in {Lean}~4.
\newblock Preprint, 2026. Paper~13 in the constructive reverse
  mathematics series.

\bibitem[Lee(2026f)]{Lee26-P14}
P.~C.-K.~Lee.
\newblock The measurement problem as a logical artefact: constructive
  calibration of quantum decoherence.
\newblock Preprint, 2026. Zenodo DOI: 10.5281/zenodo.18569068.
  Paper~14 in the constructive reverse mathematics series.

\bibitem[Mandelkern(1988)]{Mandelkern1988}
M.~Mandelkern.
\newblock Limited omniscience and the {Bolzano}--{Weierstrass} principle.
\newblock \emph{Bulletin of the London Mathematical Society}, 20:319--320,
  1988.

\bibitem[{Mathlib Community}(2020--)]{Mathlib2020}
{Mathlib Community}.
\newblock \emph{Mathlib}: the math library for {Lean}.
\newblock \url{https://leanprover-community.github.io/mathlib4_docs/},
  2020--present.

\bibitem[Noether(1918)]{Noether1918}
E.~Noether.
\newblock Invariante {Variationsprobleme}.
\newblock \emph{Nachrichten von der K\"oniglichen Gesellschaft der
  Wissenschaften zu G\"ottingen, Mathematisch-physikalische Klasse},
  pages 235--257, 1918.

\bibitem[Skopenkov(2023)]{Skopenkov2023}
M.~Skopenkov.
\newblock Discrete field theory: symmetries and conservation laws.
\newblock \emph{Mathematical Physics, Analysis and Geometry}, 26:19, 2023.
\newblock arXiv:1709.04788.

\bibitem[Witten(1981)]{Witten1981}
E.~Witten.
\newblock A new proof of the positive energy theorem.
\newblock \emph{Communications in Mathematical Physics}, 80:381--402, 1981.

\end{thebibliography}

\end{document}


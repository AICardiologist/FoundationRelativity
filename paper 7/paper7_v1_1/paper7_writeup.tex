\documentclass[11pt,a4paper]{article}

% ====================================================================
% Packages
% ====================================================================
\usepackage[utf8]{inputenc}
\usepackage[T1]{fontenc}
\usepackage{amsmath,amssymb,amsthm}
\usepackage{mathtools}
\usepackage{hyperref}
\usepackage[margin=1in]{geometry}
\usepackage{enumitem}
\usepackage{booktabs}
\usepackage{listings}
\usepackage{xcolor}
\usepackage{cleveref}
\usepackage{natbib}
\usepackage{mdframed}

% ====================================================================
% Theorem environments
% ====================================================================
\theoremstyle{plain}
\newtheorem{theorem}{Theorem}[section]
\newtheorem{lemma}[theorem]{Lemma}
\newtheorem{proposition}[theorem]{Proposition}
\newtheorem{corollary}[theorem]{Corollary}

\theoremstyle{definition}
\newtheorem{definition}[theorem]{Definition}
\newtheorem{remark}[theorem]{Remark}

% ====================================================================
% Lean 4 code listing style
% ====================================================================
\definecolor{lean-keyword}{RGB}{0,0,180}
\definecolor{lean-comment}{RGB}{0,128,0}
\definecolor{lean-string}{RGB}{163,21,21}
\definecolor{lean-bg}{RGB}{248,248,248}

\lstdefinelanguage{lean4}{
  keywords={theorem,lemma,def,class,instance,import,open,variable,
            noncomputable,section,namespace,end,where,let,have,show,
            intro,obtain,use,exact,rw,simp,apply,by,fun,match,if,
            then,else,do,return,axiom,abbrev,private,attribute,
            suffices,change,congr,ext,constructor,rintro,push_neg,
            linarith,absurd,set_option,omit,in},
  sensitive=true,
  morecomment=[l]{--},
  morecomment=[s]{/-}{-/},
  morestring=[b]",
  morestring=[b]',
}

\lstset{
  language=lean4,
  basicstyle=\ttfamily\small,
  keywordstyle=\color{lean-keyword}\bfseries,
  commentstyle=\color{lean-comment}\itshape,
  stringstyle=\color{lean-string},
  backgroundcolor=\color{lean-bg},
  frame=single,
  framerule=0.5pt,
  breaklines=true,
  breakatwhitespace=true,
  tabsize=2,
  showstringspaces=false,
  numbers=left,
  numberstyle=\tiny\color{gray},
  numbersep=5pt,
  xleftmargin=15pt,
  captionpos=b,
}

% ====================================================================
% Macros
% ====================================================================
\newcommand{\NN}{\mathbb{N}}
\newcommand{\RR}{\mathbb{R}}
\newcommand{\Kfield}{\mathbb{K}}
\newcommand{\WLPO}{\mathrm{WLPO}}
\newcommand{\Sdual}[1]{#1^{*}}
\newcommand{\Sbidual}[1]{#1^{**}}
\newcommand{\Stridual}[1]{#1^{***}}
\newcommand{\Jmap}[1]{J_{#1}}
\newcommand{\Sone}{S_1(H)}
\newcommand{\ellone}{\ell^1}
\newcommand{\ellinfty}{\ell^\infty}
\newcommand{\czero}{c_0}
\newcommand{\Lean}{\textsc{Lean~4}}
\newcommand{\Mathlib}{\textsc{Mathlib4}}

% ====================================================================
% Title
% ====================================================================
\title{%
  \textbf{The Physical Bidual Gap and Banach Space Non-Reflexivity:}\\[6pt]
  A Lean~4 Formalization of WLPO\\
  via Trace-Class Operators%
}

\author{
  Paul Chun-Kit Lee\thanks{%
    New York University.
    AI-assisted formalization; see \S\ref{sec:ai} for methodology.} \\
  New York University \\
  \texttt{dr.paul.c.lee@gmail.com}
}

\date{February 2026}

% ====================================================================
\begin{document}
\maketitle

% ====================================================================
\begin{abstract}
We present a \Lean{} formalization of the \emph{Physical Bidual Gap Theorem},
establishing Banach space non-reflexivity of the trace-class operators
$\Sone$ on a separable Hilbert space and showing that any constructive
witness of this non-reflexivity implies the Weak Limited Principle of
Omniscience (WLPO). In the algebraic formulation of quantum mechanics,
$\Sone$ is the natural state space of quantum systems; our result shows
that the gap between physical density matrices and the ``generalized
states'' in $\Sbidual{(\Sone)}$ is constructively inaccessible.
The formalization comprises 754 lines of Lean~4 code across 8~files,
building on \Mathlib{} (v4.28.0-rc1), with one interface assumption
bridging a companion formalization (Paper~2) for $\ell^1$
non-reflexivity---a classical fact (Banach, 1932) whose Lean
\texttt{axiom} declaration marks an engineering boundary between
codebases on different \Mathlib{} versions, not a mathematical gap.
The backward direction (witness $\Rightarrow$ WLPO) is proven
self-contained via an independent Ishihara kernel formalization,
providing cross-version verification of Paper~2's core result.
The \texttt{Classical.choice} dependency in the axiom profile arises
from \Mathlib{}'s functional analysis infrastructure; the constructive
validity of the equivalence is established by proof-content analysis
within the standard CRM methodology (see \S\ref{sec:classical}).
A dependency-free logical skeleton (\texttt{P7\_Minimal}, 277~lines,
4~files) certifies the reduction chain with no \texttt{Classical.choice}
in its axiom profile.
The key technical contribution is a fully formal proof that every
closed subspace of a reflexive Banach space is reflexive, using two
applications of the Hahn--Banach theorem.
\end{abstract}

\tableofcontents

% ====================================================================
\section{Introduction}\label{sec:intro}
% ====================================================================

The Weak Limited Principle of Omniscience (WLPO) is a cornerstone of
constructive reverse mathematics. It states:
\[
  \WLPO \;:\equiv\;
  \forall \alpha : \NN \to \{0,1\},\;
  \bigl(\forall n,\;\alpha(n) = 0\bigr) \;\lor\;
  \lnot\bigl(\forall n,\;\alpha(n) = 0\bigr).
\]
While classically trivial (an instance of the law of excluded middle),
WLPO is constructively independent: it is neither provable nor refutable
in Bishop-style constructive mathematics
\citep{Bishop1967,BridgesVita2006}.

In a companion paper \citep{Lee2025Paper2}, we established the
\emph{Bidual Gap Theorem}: for a broad class of Banach spaces, the
existence of a constructive witness to non-reflexivity---an element
$\Psi \in \Sbidual{X} \setminus \Jmap{X}(X)$---is equivalent to WLPO.
This result was formalized in Lean~4 and connected the abstract theory
of Banach space duality to a fundamental principle of constructive logic.

The present paper extends that work to a physically motivated setting.
In von~Neumann's formulation of quantum mechanics, the observables of a
quantum system are self-adjoint operators on a separable Hilbert space $H$,
and the states are positive trace-one operators $\rho \in \Sone$---the
density matrices. The trace norm $\|\rho\|_1 = \mathrm{tr}(|\rho|) = 1$
encodes the normalization of quantum probabilities, and the duality
$\Sdual{(\Sone)} \cong B(H)$ (bounded operators) is foundational
to the algebraic formulation of quantum theory
\citep{BratteliRobinson1987}. Indeed, $\Sone$ is the \emph{predual}
of $B(H)$: every normal state on the algebra of observables is
represented by a density matrix in $\Sone$.

The question of Banach space non-reflexivity of $\Sone$ is therefore
a question about the structure of the quantum state space itself.
Non-reflexivity means that the bidual $\Sbidual{(\Sone)}$ is strictly
larger than (the canonical image of) $\Sone$: there exist ``singular
states''---elements of $B(H)^*$ that are not representable by any
density matrix. Paper~2 \citep{Lee2025Paper2} treated the abstract
$\czero / \ellone$ setting; the present paper gives the WLPO
equivalence its \emph{physical home} in the space of quantum states.
We prove:

\begin{theorem}[Physical Bidual Gap: Banach Space Non-Reflexivity of $\Sone$, informal]\label{thm:intro-main}
Let $H$ be a separable Hilbert space.
\begin{enumerate}[label=(\roman*)]
  \item \textbf{(Unconditional)} $\Sone$ is not reflexive.
  \item \textbf{(Constructive bound)} Any constructive witness
    $\Psi \in \Sbidual{(\Sone)} \setminus \Jmap{\Sone}(\Sone)$
    implies $\WLPO$.
\end{enumerate}
\end{theorem}

The first part follows from the chain:
$\czero$ is not reflexive $\Rightarrow$ $\ellone$ is not reflexive
$\Rightarrow$ $\Sone$ is not reflexive (since $\ellone$ embeds
isometrically as a closed subspace of $\Sone$ via diagonal embedding).
The second part is an instance of the generic result from Paper~2.

The physical interpretation is striking: while $\Sone$ provably fails
to be reflexive, one cannot \emph{constructively exhibit} a specific
element of $\Sbidual{(\Sone)}$ outside the canonical image without
invoking WLPO. This connects the structure of quantum state spaces
to the foundations of constructive mathematics.

\paragraph{Contributions.}
\begin{itemize}
  \item A \Lean{} formalization of the Physical Bidual Gap Theorem,
    establishing Banach space non-reflexivity of trace-class operators
    (754 lines across 8~files, 7 sorry-free, 1~interface assumption
    for $\ellone$ non-reflexivity bridging Paper~2).
  \item An independent self-contained formalization of the Ishihara
    kernel construction (\texttt{WLPOFromWitness.lean}, 196~lines,
    zero custom axioms), proving that any non-reflexivity witness
    implies WLPO. This provides cross-version verification of the
    core result from Paper~2, compiled on a different \Mathlib{}
    version.
  \item A machine-checked proof that every closed subspace of a reflexive
    Banach space is reflexive (\Cref{lem:subspace-reflexive}), using two
    applications of the Hahn--Banach theorem---a standard result not
    currently present in \Mathlib{} (\Cref{rem:mathlib-contrib}).
  \item A machine-checked proof that reflexivity transfers across linear
    isometric equivalences (\Cref{lem:compat}).
  \item A concrete sorry-backed instance showing $S_1(\ell^2(\NN))$
    satisfies the abstract trace-class container interface, grounding
    the theorem in a specific physical space.
  \item A demonstration of the AI-assisted formalization methodology
    using Claude Opus~4.6 via the Claude Code CLI.
\end{itemize}

% ====================================================================
\section{Background}\label{sec:background}
% ====================================================================

\subsection{Banach Space Reflexivity}

Let $X$ be a Banach space over $\RR$.  The \emph{dual space}
$\Sdual{X} = X^*$ is the space of bounded linear functionals
$f : X \to \RR$, itself a Banach space under the operator norm.
The \emph{bidual} $\Sbidual{X} = (X^*)^*$ admits a canonical
isometric embedding
\[
  \Jmap{X} : X \to \Sbidual{X}, \qquad
  \Jmap{X}(x)(f) = f(x).
\]
The space $X$ is \emph{reflexive} if $\Jmap{X}$ is surjective
(equivalently, an isometric isomorphism onto $\Sbidual{X}$).

Classical examples of non-reflexive spaces include $\czero$,
$\ellone$, $\ellinfty$, and $L^1(\mu)$ for non-atomic measures.
The duality chain
\[
  \czero \hookrightarrow \Sdual{(\czero)} \cong \ellone
  \hookrightarrow \Sdual{(\ellone)} \cong \ellinfty
\]
is central to our argument.

\subsection{The Bidual Gap Theorem (Paper~2)}\label{sec:paper2}

Paper~2 \citep{Lee2025Paper2} established the following equivalence
in Lean~4:

\begin{theorem}[Bidual Gap Equivalence {\citep[Theorem~1]{Lee2025Paper2}}]
\label{thm:paper2}
\[
  \WLPO \;\;\Longleftrightarrow\;\;
  \bigl(\exists\text{ a separable Banach space } X
  \text{ and } \Psi \in \Sbidual{X} \setminus \Jmap{X}(X)\bigr).
\]
Moreover, the forward direction holds uniformly: for \emph{any}
Banach space $X$, a non-reflexivity witness
$\Psi \in \Sbidual{X} \setminus \Jmap{X}(X)$ implies $\WLPO$.
\end{theorem}

The backward direction ($\Leftarrow$) builds an explicit
element $G \in \Sbidual{(\czero)}$ that evaluates to~1 on every
point-evaluation functional and lies outside $\Jmap{\czero}(\czero)$;
the construction is specific to $\czero$. The forward direction
($\Rightarrow$) constructs an Ishihara kernel from the
non-reflexivity witness and extracts WLPO via a constructive
consumer; this argument works for any Banach space $X$ and is the
direction used by Paper~7.

The proof chain for $\ellone$ not being reflexive proceeds through
dual isometries:
\begin{enumerate}
  \item $\czero$ is not reflexive (unconditional; the witness $G$
    is constructed without WLPO).
  \item $\Sdual{(\czero)} \cong \ellone$ and
    $\Sdual{(\ellone)} \cong \ellinfty$ (isometric isomorphisms).
  \item If $\ellone$ were reflexive, so would $\ellinfty$ be
    (Lemma~A), and then $\czero$ would be reflexive as a closed
    subspace of a space isometric to $\Sdual{(\ellinfty)}$
    (Lemma~B), contradicting~(1).
\end{enumerate}

\subsection{Trace-Class Operators}

For a separable Hilbert space $H$ with orthonormal basis
$(e_n)_{n \in \NN}$, the \emph{trace-class operators}
$\Sone = S_1(H)$ are the compact operators $T$ on $H$ for which
\[
  \|T\|_1 \;:=\; \mathrm{tr}(|T|) \;=\;
  \sum_{n=0}^{\infty} \langle |T| e_n, e_n \rangle < \infty.
\]
The trace norm $\|\cdot\|_1$ makes $\Sone$ a Banach space, with
$\Sdual{(\Sone)} \cong B(H)$ (bounded operators) via the pairing
$\langle T, A \rangle = \mathrm{tr}(TA)$.

The \emph{diagonal embedding}
\[
  \iota : \ellone \hookrightarrow \Sone, \qquad
  \iota(\lambda) = \sum_{n=0}^{\infty} \lambda_n \,|e_n\rangle\langle e_n|
\]
is an isometric linear map with closed range: the diagonal operators
form a closed subspace of $\Sone$ isometric to $\ellone$.

\subsection{Physical significance}\label{sec:phys-sig}

In algebraic quantum mechanics \citep{BratteliRobinson1987,BratteliRobinson1997},
a state on the observable algebra $B(H)$ is a positive normalized linear
functional $\omega : B(H) \to \RR$.  The \emph{normal} states---those
that are $\sigma$-weakly continuous---are exactly the density matrices in
$\Sone$, acting via $\omega_\rho(A) = \mathrm{tr}(\rho A)$.  The predual
relationship $B(H)_* = \Sone$ makes the trace-class operators the
canonical state space of quantum theory.

Banach space non-reflexivity of $\Sone$ means that
$\Sdual{(B(H))} = \Sbidual{(\Sone)} \supsetneq \Jmap{\Sone}(\Sone)$:
there exist \emph{singular states} on $B(H)$---positive linear functionals
that are not $\sigma$-weakly continuous and cannot be represented by any
density matrix.  These singular states are analogous to finitely additive
measures that are not countably additive; they arise from ultrafilter-type
constructions and have no direct physical preparation procedure.

The physical importance of this gap extends beyond abstract functional
analysis.  In quantum statistical mechanics, every state $\omega$ on a
von~Neumann algebra $\mathcal{M}$ admits a unique decomposition
$\omega = \omega_n + \omega_s$ into normal and singular parts
\citep{Takesaki1979}---the noncommutative analogue of the Lebesgue
decomposition of measures.  The singular component $\omega_s$ vanishes
on all compact operators: it ``sees'' only the behavior of observables
at infinity.

This decomposition is physically realized in the \emph{thermodynamic
limit}.  For a quantum system in a finite volume
$\Lambda \subset \mathbb{Z}^d$, the equilibrium state at inverse
temperature $\beta$ is a Gibbs state $\omega_\Lambda^\beta$---a density
matrix, hence a normal state on $B(H_\Lambda)$.  In the infinite-volume
limit $\Lambda \nearrow \mathbb{Z}^d$, one obtains a state on the
quasilocal algebra
$\mathfrak{A} = \overline{\bigcup_\Lambda B(H_\Lambda)}$ that may fail
to be normal with respect to any fixed representation
\citep{BratteliRobinson1997}.  The KMS (Kubo--Martin--Schwinger) states
characterizing thermal equilibrium at inverse temperature $\beta$ are
defined by the condition
$\omega(AB) = \omega(B\sigma_{i\beta}(A))$ for the modular automorphism
group $\sigma_t$; at phase transitions, multiple KMS states coexist,
corresponding to distinct thermodynamic phases \citep{Haag1996}.

The passage from finite-volume Gibbs states to infinite-volume KMS
states is precisely the passage from the predual $\Sone$ to its bidual
$\Sbidual{(\Sone)}$.  Singular states---elements of the bidual
gap---represent idealized thermodynamic configurations that cannot be
prepared by any finite laboratory procedure.  Their mathematical
existence is guaranteed by non-reflexivity; our result shows that
\emph{constructively witnessing} any specific singular state requires
WLPO.

This connects to a broader question in the foundations of quantum
statistical mechanics: what is the logical cost of the thermodynamic
limit?  Cubitt, Perez-Garcia, and Wolf \citep{CubittPerezWolf2015}
showed that the spectral gap problem for quantum many-body systems is
\emph{undecidable} at the level of the Halting Problem---far stronger
than any omniscience principle.  Van~Wierst \citep{vanWierst2019}
explored the paradox of phase transitions from the perspective of
constructive mathematics, observing that non-analytic behavior in
partition functions (the hallmark of phase transitions) is problematic
in frameworks where all total functions are continuous.  Our result
occupies a specific intermediate position in this landscape: the mere
\emph{existence} of the singular sector (non-reflexivity in
$\lnot$-form) requires WLPO, while the full thermodynamic limit
(constructing infinite-volume states via monotone convergence) requires
LPO, and non-separable Hahn--Banach separation requires full classical
logic.

% ====================================================================
\section{Mathematical Content}\label{sec:math}
% ====================================================================

We now state the key lemmas and the main theorem precisely.

\begin{definition}[WLPO]\label{def:wlpo}
The \emph{Weak Limited Principle of Omniscience} is the proposition
\[
  \WLPO \;:\equiv\;
  \forall \alpha : \NN \to \mathrm{Bool},\;\;
  (\forall n,\;\alpha(n) = \mathtt{false})
  \;\lor\; \lnot(\forall n,\;\alpha(n) = \mathtt{false}).
\]
\end{definition}

\begin{definition}[Reflexivity]\label{def:reflexive}
A normed space $X$ over a field $\Kfield$ is \emph{reflexive} if
the canonical embedding $\Jmap{X} : X \to \Sbidual{X}$ (given by
$\Jmap{X}(x)(f) = f(x)$) is surjective.
\end{definition}

\begin{lemma}[Lemma A: Reflexive implies dual reflexive]
\label{lem:dual-reflexive}
If $X$ is reflexive, then $\Sdual{X}$ is reflexive.
\end{lemma}

\begin{lemma}[Lemma B: Closed subspace of reflexive is reflexive]
\label{lem:subspace-reflexive}
Let $Y$ be a closed subspace of a reflexive Banach space $X$.
Then $Y$ is reflexive.
\end{lemma}

\begin{lemma}[Compatibility: Reflexivity transfers across isometries]
\label{lem:compat}
If $X \cong Y$ via a linear isometric equivalence and $Y$ is
reflexive, then $X$ is reflexive.
\end{lemma}

\begin{definition}[Trace-class container]\label{def:tc}
A \emph{trace-class container} is a separable complete normed space
$X$ over $\RR$ equipped with an isometric continuous linear map
$\iota : \ellone \to X$ whose range is closed. The canonical
example is $\Sone$ with the diagonal embedding.
\end{definition}

\begin{theorem}[Physical Bidual Gap: Banach Space Non-Reflexivity of $\Sone$]\label{thm:main}
Let $X$ be a trace-class container. Then:
\begin{enumerate}[label=(\roman*)]
  \item $\lnot\,\mathrm{IsReflexive}(\RR, X)$.
  \item $\bigl(\exists \Psi \in \Sbidual{X} \setminus
    \Jmap{X}(X)\bigr) \;\Longrightarrow\; \WLPO$.
\end{enumerate}
\end{theorem}

% ====================================================================
\section{Human-Readable Proofs}\label{sec:proofs}
% ====================================================================

\subsection{Lemma A: Reflexive Implies Dual Reflexive}

\begin{proof}
Let $X$ be reflexive, and let $\varphi \in \Stridual{X} = (\Sdual{X})^{**}$.
We must find $f \in \Sdual{X}$ with $\Jmap{\Sdual{X}}(f) = \varphi$.

Define $f := \varphi \circ \Jmap{X} \in \Sdual{X}$, so that for each
$x \in X$, $f(x) = \varphi(\Jmap{X}(x))$.

We verify $\Jmap{\Sdual{X}}(f) = \varphi$ by checking on all of
$\Sbidual{X}$. Since $\Jmap{X}$ is surjective, any $\Psi \in \Sbidual{X}$
has the form $\Psi = \Jmap{X}(x)$ for some $x$. Then:
\[
  \Jmap{\Sdual{X}}(f)(\Psi) = \Psi(f) = \Jmap{X}(x)(f) = f(x)
  = \varphi(\Jmap{X}(x)) = \varphi(\Psi). \qedhere
\]
\end{proof}

\subsection{Lemma B: Closed Subspace of Reflexive Is Reflexive}
\label{sec:proof-lemmaB}

This is the technical bottleneck of the formalization.

\begin{proof}
Let $Y \subseteq X$ be a closed subspace with $X$ reflexive, and let
$\varphi \in \Sbidual{Y}$. We construct $y \in Y$ with
$\Jmap{Y}(y) = \varphi$ in four steps.

\medskip\noindent\textbf{Step 1: Lift $\varphi$ to $\Phi \in \Sbidual{X}$.}
Let $\mathrm{res} : \Sdual{X} \to \Sdual{Y}$ be the restriction map
$f \mapsto f|_Y$ (formally, $\mathrm{res} = (\cdot) \circ \iota_Y$
where $\iota_Y : Y \hookrightarrow X$ is the inclusion). Define
$\Phi := \varphi \circ \mathrm{res} \in \Sbidual{X}$.

\medskip\noindent\textbf{Step 2: Represent $\Phi$ via reflexivity.}
Since $X$ is reflexive, there exists $x \in X$ with
$\Jmap{X}(x) = \Phi$. That is, for all $f \in \Sdual{X}$:
$f(x) = \Phi(f) = \varphi(f|_Y)$.

\medskip\noindent\textbf{Step 3: Show $x \in Y$ by contradiction
(Hahn--Banach separation).}
Suppose $x \notin Y$. Since $Y$ is closed and convex (it is a subspace),
the geometric Hahn--Banach separation theorem provides $f_0 \in \Sdual{X}$
and $u \in \RR$ with:
\[
  \forall a \in Y,\; f_0(a) < u
  \qquad\text{and}\qquad
  u < f_0(x).
\]
Since $Y$ is a subspace, for any $y \in Y$ and $n \in \NN$, we have
$n \cdot y \in Y$, so $n \cdot f_0(y) < u$. Letting $n \to \infty$
forces $f_0(y) \leq 0$. Applying the same argument to $-y \in Y$
gives $f_0(y) \geq 0$. Hence $f_0$ annihilates $Y$: $f_0(y) = 0$
for all $y \in Y$.

In particular, $0 = f_0(0) < u$ (since $0 \in Y$). But also:
\[
  f_0(x) = \Phi(f_0) = \varphi(\mathrm{res}(f_0)) = \varphi(f_0|_Y)
  = \varphi(0) = 0,
\]
since $f_0|_Y = 0$. This gives $u < f_0(x) = 0 < u$, a contradiction.

\medskip\noindent\textbf{Step 4: Verify $\Jmap{Y}(y) = \varphi$
(Hahn--Banach extension).}
With $x \in Y$ established, set $y := \langle x, \cdot\rangle \in Y$.
For any $g \in \Sdual{Y}$, the Hahn--Banach extension theorem provides
$f \in \Sdual{X}$ with $f|_Y = g$. Then:
\[
  \Jmap{Y}(y)(g) = g(y) = f(x) = \Phi(f) = \varphi(\mathrm{res}(f))
  = \varphi(g). \qedhere
\]
\end{proof}

\begin{remark}[Avoidance of James's theorem]\label{rem:james}
The classical route to ``closed subspace of reflexive is reflexive''
typically passes through James's theorem ($X$ is reflexive if and
only if every continuous linear functional attains its norm). Our
proof avoids James entirely, using Hahn--Banach directly: geometric
separation for Step~3 and norm-preserving extension for Step~4.
This is noteworthy because James's theorem is constructively
problematic---it relies on a characterization that does not hold
in Bishop-style constructive mathematics---whereas our argument
uses only the Hahn--Banach theorem, which has a constructive
formulation for separable spaces.
\end{remark}

\begin{remark}[Contribution to \Mathlib{} infrastructure]\label{rem:mathlib-contrib}
\Cref{lem:subspace-reflexive} appears not to be present in \Mathlib{}
(as of v4.28). The closest result is \texttt{IsReflexive.of\_split},
which requires the subspace to be \emph{complemented} (a direct
summand)---a strictly stronger condition. Our proof, using geometric
Hahn--Banach separation and norm-preserving extension, works for
arbitrary closed subspaces and may be of independent interest to the
\Mathlib{} community.
\end{remark}

\subsection{Compatibility: Isometry Transfer}

\begin{proof}[Proof of \Cref{lem:compat}]
Let $e : X \xrightarrow{\sim} Y$ be a linear isometric equivalence with
$Y$ reflexive. Given $\Phi \in \Sbidual{X}$, define
$e^* : \Sdual{Y} \to \Sdual{X}$ by $e^*(g) = g \circ e$ (precomposition).
Let $\Psi := \Phi \circ e^* \in \Sbidual{Y}$. By reflexivity of $Y$,
there exists $y$ with $\Jmap{Y}(y) = \Psi$. Set $x := e^{-1}(y)$.

For any $f \in \Sdual{X}$, define $g := f \circ e^{-1} \in \Sdual{Y}$.
Then $g(y) = f(e^{-1}(y)) = f(x)$, and
$\Psi(g) = \Phi(e^*(g)) = \Phi(g \circ e) = \Phi(f \circ e^{-1} \circ e)
= \Phi(f)$. Since $\Jmap{Y}(y)(g) = g(y)$, we get
$f(x) = g(y) = \Jmap{Y}(y)(g) = \Psi(g) = \Phi(f)$.
Hence $\Jmap{X}(x) = \Phi$.
\end{proof}

\subsection{Main Theorem: Physical Bidual Gap and Banach Space Non-Reflexivity}

\begin{proof}[Proof of \Cref{thm:main}]
\textbf{Part (i).}
Let $X$ be a trace-class container with isometric embedding
$\iota : \ellone \hookrightarrow X$ having closed range.
The map $\iota$ gives a linear isometry $\ellone \to X$,
and $\mathrm{range}(\iota)$ is a closed subspace of $X$.
By \texttt{LinearIsometry.equivRange}, we obtain
$e : \ellone \cong \mathrm{range}(\iota)$ as a linear isometric
equivalence.

Suppose for contradiction that $X$ is reflexive. By \Cref{lem:subspace-reflexive},
$\mathrm{range}(\iota)$ is reflexive. By \Cref{lem:compat} (applied to $e$),
$\ellone$ is reflexive---contradicting the known non-reflexivity of $\ellone$.

\medskip\noindent\textbf{Part (ii).}
This follows from the Ishihara kernel construction (proven
self-contained in \texttt{WLPOFromWitness.lean}, adapted from
Paper~2): given any non-reflexivity witness $\Psi$ for any Banach
space, one constructs an Ishihara kernel and applies the constructive
WLPO consumer.
\end{proof}

% ====================================================================
\section{The Lean~4 Formalization}\label{sec:lean}
% ====================================================================

\subsection{Architecture}

The formalization consists of 754 lines of Lean~4 code across 8 files,
organized as follows:

\begin{table}[ht]
\centering
\begin{tabular}{@{}llrl@{}}
\toprule
\textbf{File} & \textbf{Role} & \textbf{Lines} & \textbf{Status} \\
\midrule
\texttt{Basic.lean}              & WLPO \& IsReflexive definitions & 34  & Complete \\
\texttt{ReflexiveDual.lean}      & Lemma A                         & 47  & Complete \\
\texttt{DiagonalEmbedding.lean}  & \texttt{HasTraceClassContainer}  & 45  & Complete \\
\texttt{Compat.lean}             & Isometry transfer               & 78  & Complete \\
\texttt{ReflexiveSubspace.lean}  & Lemma B (bottleneck)            & 174 & Complete \\
\texttt{WLPOFromWitness.lean}    & Ishihara kernel (eliminates axiom) & 196 & Complete \\
\texttt{Instance.lean}           & Concrete $S_1$ witness (sorry-backed) & 51 & Stub \\
\texttt{Main.lean}               & Assembly + main theorem          & 129 & Complete \\
\bottomrule
\end{tabular}
\caption{File manifest for the Paper~7 formalization.}
\label{tab:manifest}
\end{table}

The dependency graph is:
\begin{verbatim}
Basic.lean
  |-- ReflexiveDual.lean       (Lemma A)
  |-- DiagonalEmbedding.lean   (HasTraceClassContainer)
  |   |-- Instance.lean        (concrete S1 witness)
  |-- Compat.lean              (isometry transfer)
  |-- ReflexiveSubspace.lean   (Lemma B)
  |-- WLPOFromWitness.lean     (Ishihara kernel proof)
  |
  Main.lean                    (assembly)
\end{verbatim}

\subsection{Core Definitions}

The definitions in \texttt{Basic.lean} encode WLPO and reflexivity:

\begin{lstlisting}[caption={Core definitions (Basic.lean).}]
/-- Weak Limited Principle of Omniscience. -/
def WLPO : Prop :=
  forall (a : Nat -> Bool),
    (forall n, a n = false) ||| not (forall n, a n = false)

/-- Reflexivity: surjectivity of the canonical
    embedding J_X : X -> X**. -/
def IsReflexive (K : Type*) [NontriviallyNormedField K]
    (X : Type*) [NormedAddCommGroup X]
    [NormedSpace K X] : Prop :=
  Function.Surjective (inclusionInDoubleDual K X)
\end{lstlisting}

\subsection{The Abstract Interface (DiagonalEmbedding.lean)}

Rather than formalizing the full Schatten class infrastructure
(not available in \Mathlib{}, estimated at 1000+ lines), we
axiomatize the essential properties via a typeclass:

\begin{lstlisting}[caption={Trace-class container interface
  (DiagonalEmbedding.lean).}]
abbrev ell1 : Type := lp (fun _ : Nat => Real) 1

class HasTraceClassContainer where
  X : Type
  [instNAG : NormedAddCommGroup X]
  [instNS : NormedSpace Real X]
  [instCS : CompleteSpace X]
  [instSep : TopologicalSpace.SeparableSpace X]
  -- Isometric embedding ell1 -> X
  i : ell1 ->L[Real] X
  i_isometry : Isometry i
  i_closedRange : IsClosed (Set.range i)
\end{lstlisting}

This approach isolates the three properties needed from the
embedding ($\iota$ is continuous linear, isometric, and has
closed range) without requiring the construction of $\Sone$.

\subsection{Lemma A: Reflexive Dual (ReflexiveDual.lean)}

The proof is remarkably concise in Lean~4---just 12 lines of tactic
proof:

\begin{lstlisting}[caption={Lemma A: X reflexive implies X* reflexive.}]
theorem reflexive_dual_of_reflexive
    (hX : IsReflexive K X) :
    IsReflexive K (StrongDual K X) := by
  intro phi
  let f : StrongDual K X :=
    phi.comp (inclusionInDoubleDual K X)
  use f
  ext Psi
  obtain <<x, hx>> := hX Psi
  change Psi f = phi Psi
  rw [<- hx]
  rfl
\end{lstlisting}

The key insight is that after unfolding the canonical embedding,
the goal reduces to a definitional equality (\texttt{rfl}).
This required adapting to a change in \Mathlib{} v4.28 where
\texttt{inclusionInDoubleDual} unfolds to
\texttt{ContinuousLinearMap.apply}, altering \texttt{simp} behavior.

\subsection{Lemma B: Reflexive Subspace (ReflexiveSubspace.lean)}
\label{sec:lean-lemmaB}

This is the technical bottleneck, requiring two applications of
the Hahn--Banach theorem from \Mathlib{}.

\paragraph{Auxiliary lemma.}
A key step is showing that a continuous linear functional bounded
on a subspace must annihilate it:

\begin{lstlisting}[caption={Annihilation lemma for subspace separation.}]
private lemma annihilates_of_bounded_on_subspace
    {Y : Submodule Real X} (f : StrongDual Real X) (u : Real)
    (hfu : forall a in (Y : Set X), f a < u) :
    forall y in (Y : Set X), f y = 0 := by
  intro y hy
  apply le_antisymm
  . -- f(y) <= 0: for large n, n*f(y) < u
    by_contra h; push_neg at h
    obtain <<n, hn>> := exists_nat_gt (u / f y)
    have hfy_pos : (0 : Real) < f y := h
    have hnu : (n : Real) * f y > u := by
      rwa [gt_iff_lt, <- div_lt_iff_0 hfy_pos]
    have hny : (n : Real) . y in (Y : Set X) :=
      Y.smul_mem (n : Real) hy
    have h_bound := hfu _ hny
    rw [map_smul, smul_eq_mul] at h_bound
    linarith
  . -- f(y) >= 0: similarly via -y
    ...
\end{lstlisting}

\paragraph{Main proof body.}
The four-step proof is shown below (Steps 1--2 and 3--4,
with Step~4 nested inside a \texttt{suffices}). This is the
core of the formalization and, independently of the WLPO
story, constitutes a machine-checked proof of a standard
functional analysis theorem using two applications of the
Hahn--Banach theorem.

\begin{lstlisting}[caption={Lemma B: closed subspace of reflexive is
  reflexive (main body, ReflexiveSubspace.lean).}]
theorem reflexive_closedSubspace_of_reflexive
    (Y : Submodule Real X) (hYc : IsClosed (Y : Set X))
    (hX : IsReflexive Real X) : IsReflexive Real Y := by
  intro phi
  -- Step 1: Lift phi to X** via restriction
  let res : StrongDual Real X ->L[Real] StrongDual Real Y :=
    ContinuousLinearMap.precomp Real Y.subtypeL
  let Phi : StrongDual Real (StrongDual Real X) :=
    phi.comp res
  -- Step 2: Represent Phi via reflexivity
  obtain <<x, hx>> := hX Phi
  -- Steps 3+4 nested in suffices
  suffices hx_mem : x in (Y : Set X) by
    -- Step 4: Verify J_Y(y) = phi via Hahn-Banach extension
    let y : Y := <<x, hx_mem>>
    use y; ext g
    obtain <<f, hf_ext, _>> :=
      Real.exists_extension_norm_eq Y g
    ...  -- chain: g(y) = f(x) = Phi(f) = phi(g)
  -- Step 3: x in Y by contradiction (Hahn-Banach separation)
  by_contra hx_not_mem
  obtain <<f0, u, hfu, hu_x>> :=
    geometric_hahn_banach_closed_point
      (Y.convex) hYc hx_not_mem
  have h_ann := annihilates_of_bounded_on_subspace f0 u hfu
  have h_res_zero : res f0 = 0 := by
    ext <<z, hz>>
    simp only [res, ContinuousLinearMap.precomp_apply,
      ContinuousLinearMap.comp_apply,
      Submodule.subtypeL_apply,
      ContinuousLinearMap.zero_apply]
    exact h_ann z hz
  -- f0(x) = Phi(f0) = phi(res f0) = phi(0) = 0
  -- but u < f0(x) and 0 < u -- contradiction
  rw [h_res_zero, map_zero] at hf0x
  linarith
\end{lstlisting}

The key \Mathlib{} APIs are:
\texttt{ContinuousLinearMap.precomp} (restriction, Step~1),
\texttt{geometric\_hahn\_banach\_closed\_point} (separation, Step~3),
and \texttt{Real.exists\_extension\_norm\_eq} (extension, Step~4).

\subsection{Compatibility (Compat.lean)}

The isometry transfer proof constructs dual maps via
precomposition. A notable technical challenge was the coercion
chain in \Mathlib{} v4.28: the path from \texttt{LinearIsometryEquiv}
to \texttt{ContinuousLinearMap} requires passing through
\texttt{toContinuousLinearEquiv}:

\begin{lstlisting}[caption={Coercion chain for isometry transfer.}]
let eL : X ->L[K] Y :=
  e.toContinuousLinearEquiv.toContinuousLinearMap
let eLs : Y ->L[K] X :=
  e.symm.toContinuousLinearEquiv.toContinuousLinearMap
let e_star : StrongDual K Y ->L[K] StrongDual K X :=
  ContinuousLinearMap.precomp K eL
\end{lstlisting}

The critical simp lemmas for this coercion chain are
\texttt{ContinuousLinearEquiv.coe\_coe} and
\texttt{LinearIsometryEquiv.coe\_toContinuousLinearEquiv}.

\subsection{Main Assembly (Main.lean)}

The forward direction builds a \texttt{LinearIsometry} from
the typeclass data, converts it to an equivalence via
\texttt{LinearIsometry.equivRange}, and applies the
lemmas in sequence:

\begin{lstlisting}[caption={Forward direction of the main theorem
  (Main.lean, complete proof).}]
theorem not_reflexive_of_contains_ell1
    [tc : HasTraceClassContainer] :
    not (IsReflexive Real tc.X) := by
  intro hX
  -- Build LinearIsometry from typeclass data
  let i_li : ell1 ->li[Real] tc.X :=
    { tc.i.toLinearMap with
      norm_map' := fun x =>
        (AddMonoidHomClass.isometry_iff_norm tc.i).mp
          tc.i_isometry x }
  -- LinearIsometryEquiv: ell1 ~ range(i)
  let e := LinearIsometry.equivRange i_li
  let Y : Submodule Real tc.X :=
    LinearMap.range i_li.toLinearMap
  -- Show range(i_li) is closed
  have hYc : IsClosed (Y : Set tc.X) := by
    suffices h : (Y : Set tc.X) = Set.range tc.i by
      rw [h]; exact tc.i_closedRange
    ext x
    simp only [Y, LinearMap.mem_range,
      SetLike.mem_coe, Set.mem_range]
    constructor
    . rintro <<a, ha>>; exact <<a, ha>>
    . rintro <<a, ha>>; exact <<a, ha>>
  -- Closed subspace of reflexive -> reflexive (Lemma B)
  have hY_refl : IsReflexive Real Y :=
    reflexive_closedSubspace_of_reflexive Y hYc hX
  -- Transfer via isometry (Compat)
  have h_ell1_refl : IsReflexive Real ell1 :=
    reflexive_of_linearIsometryEquiv e hY_refl
  -- Contradiction
  exact absurd h_ell1_refl ell1_not_reflexive
\end{lstlisting}

The combined theorem is a simple conjunction:

\begin{lstlisting}[caption={The Physical Bidual Gap theorem statement.}]
theorem physical_bidual_gap
    [tc : HasTraceClassContainer] :
    (not (IsReflexive Real tc.X)) /\
    ((exists Psi : StrongDual Real (StrongDual Real tc.X),
        Psi not_in Set.range
          (inclusionInDoubleDual Real tc.X))
      -> WLPO) :=
  <<not_reflexive_of_contains_ell1,
   wlpo_of_traceClass_witness>>
\end{lstlisting}

\subsection{Key \Mathlib{} APIs}

\Cref{tab:apis} summarizes the \Mathlib{} APIs that were essential
to the formalization.

\begin{table}[h]
\centering
\begin{tabular}{@{}ll@{}}
\toprule
\textbf{API} & \textbf{Purpose} \\
\midrule
\texttt{ContinuousLinearMap.precomp} & Restriction map $\Sdual{X} \to \Sdual{Y}$ \\
\texttt{Submodule.subtypeL} & Canonical inclusion $Y \hookrightarrow X$ \\
\texttt{geometric\_hahn\_banach\_closed\_point} & Separation for closed convex sets \\
\texttt{Real.exists\_extension\_norm\_eq} & Norm-preserving Hahn--Banach extension \\
\texttt{LinearIsometry.equivRange} & Isometric equivalence onto range \\
\texttt{ContinuousLinearEquiv.coe\_coe} & Coercion chain for equivalences \\
\texttt{inclusionInDoubleDual} & Canonical embedding $\Jmap{X} : X \to \Sbidual{X}$ \\
\bottomrule
\end{tabular}
\caption{Key \Mathlib{} APIs used in the formalization.}
\label{tab:apis}
\end{table}

\subsection{Mathlib Version Migration}

During development, the \Mathlib{} version was upgraded from v4.23
to v4.28.0-rc1. This required several adaptations:

\begin{itemize}
  \item \textbf{Import paths:}
    \texttt{NormedSpace.HahnBanach.Extension} $\to$
    \texttt{Normed.Module.HahnBanach};
    \texttt{NormedSpace.HahnBanach.Separation} $\to$
    \texttt{LocallyConvex.Separation}.
  \item \textbf{Lemma renames:}
    \texttt{div\_lt\_iff} $\to$ \texttt{div\_lt\_iff\(_0\)}.
  \item \textbf{Coercion changes:}
    \texttt{LinearIsometryEquiv.toContinuousLinearMap} now requires
    \texttt{FiniteDimensional}; the correct path is through
    \texttt{toContinuousLinearEquiv.toContinuousLinearMap}.
  \item \textbf{Definitional unfolding:}
    \texttt{inclusionInDoubleDual} now unfolds to
    \texttt{ContinuousLinearMap.apply}, changing \texttt{simp} behavior
    and requiring explicit \texttt{change} tactics.
  \item \textbf{Rewriting vs.\ simplification:}
    In the auxiliary annihilation lemma, \texttt{simp} with
    \texttt{map\_smul, map\_neg, smul\_eq\_mul} was replaced by
    explicit \texttt{rw} steps to produce terms that
    \texttt{linarith} could process.
\end{itemize}

\subsection{Reproducibility information}

\begin{mdframed}[backgroundcolor=gray!10]
\textbf{Reproducibility Box}
\begin{itemize}
\item \textbf{Repository}: \url{https://github.com/quantmann/FoundationRelativity}
\item \textbf{LaTeX source \& PDF}: \url{https://doi.org/10.5281/zenodo.18527559}
\item \textbf{Lean toolchain}: \texttt{leanprover/lean4:v4.28.0-rc1}
\item \textbf{Lake}: \texttt{Lake 5.0.0-src+3b0f286 (Lean 4.28.0-rc1)}
\item \textbf{mathlib4 commit}: \texttt{9543d5047cb12a05abd2d9b9bc2ea2a604b3be87}
\item \textbf{Current commit}: \texttt{d7e222386c4a8e9f230daa97fe5e9170ae307207}
\item \textbf{Build}: \texttt{lake exe cache get \&\& lake build}
\item \textbf{Bundle target}: \texttt{Papers} (imports
  \texttt{Main} + \texttt{Instance})
\item \textbf{Status}: 1~interface assumption
  (\texttt{ell1\_not\_reflexive}), 1~sorry-backed instance
  (\texttt{Instance.lean}), all 8~files compile (3086 jobs).
  Axiom profile of \texttt{physical\_bidual\_gap}:
  \texttt{[propext, Classical.choice, Quot.sound,
  ell1\_not\_reflexive]}.
  \texttt{wlpo\_of\_nonreflexive\_witness\_proof} has no custom axioms.
\item \textbf{Minimal artifact}:
  \texttt{Papers.P7\_PhysicalBidualGap.P7\_Minimal.Main}
  --- dependency-free logical skeleton (277~lines, 4~files),
  no \Mathlib{} imports, zero \texttt{sorry}, zero errors.
  Axiomatizes Lemma~B and the Ishihara kernel; certifies the
  reduction chain \texttt{NonReflexiveWitness(S$_1$(H)) $\leftrightarrow$ WLPO}
  with no \texttt{Classical.choice} in the axiom profile.
  Build: \texttt{lake build Papers.P7\_PhysicalBidualGap.P7\_Minimal.Main}
  (5~jobs, 0~errors).
  \texttt{\#print axioms} output:
  \texttt{[ell1\_closed\_subspace\_of\_S1, ishihara\_kernel,
  not\_reflexive\_implies\_witness\_S1, paper2\_reverse,
  reflexive\_closedSubspace\_of\_reflexive,
  witness\_implies\_not\_reflexive]}.
\end{itemize}
\end{mdframed}

% ====================================================================
\section{Interface Assumptions}\label{sec:axioms}
% ====================================================================

The formalization uses exactly one interface assumption, declared
using Lean's \texttt{axiom} keyword. The underlying mathematical
fact---that $\ellone$ is not reflexive---has been known since
Banach's 1932 monograph and is not in question. The Lean
\texttt{axiom} keyword here marks an \emph{engineering boundary}
between two verified codebases on incompatible \Mathlib{} versions,
not a mathematical gap. A second interface assumption from an
earlier version has been eliminated by independently proving the
Ishihara kernel construction within Paper~7.

\paragraph{Remaining interface assumption.}

\begin{enumerate}
  \item \textbf{\texttt{ell1\_not\_reflexive}:}
    $\lnot\,\mathrm{IsReflexive}(\RR, \ellone)$.

    \emph{Proven in Paper~2} \citep{Lee2025Paper2} via the chain:
    $\czero$ is not reflexive
    (unconditional; the witness $G \in \Sbidual{(\czero)}$ is
    constructed without WLPO) $\to$ dual isometries
    $\Sdual{(\czero)} \cong \ellone$ and
    $\Sdual{(\ellone)} \cong \ellinfty$ $\to$ if $\ellone$ were
    reflexive, $\czero$ would be reflexive (via Lemmas~A and~B),
    contradiction. Paper~2's WLPO $\leftrightarrow$ bidual gap
    equivalence is mechanically certified via the dependency-free
    \texttt{P2\_Minimal} artifact (zero \texttt{sorry}, no
    \texttt{Classical.choice}); the present interface assumption
    encapsulates the dual-isometry infrastructure of
    \texttt{P2\_Full}
    ($\sim$1,593~lines in \texttt{DualIsometriesComplete.lean}).

    \emph{Why it remains:} This is a version-migration cost, not a
    logical gap. Paper~2 was developed on Lean v4.23.0-rc2;
    Paper~7 uses v4.28.0-rc1. The five-version toolchain gap
    prevents a direct Lake dependency import, and porting Paper~2's
    dual isometry infrastructure would require migrating
    $\sim$3,100 lines across 11 files through substantial
    \Mathlib{} API changes. The proposition itself is proved in
    Paper~2's \texttt{P2\_Full} target on Lean v4.23.0-rc2;
    porting to Paper~7's \Mathlib{} v4.28 is blocked by version
    migration cost, not by mathematical content. A unified monorepo
    on a shared toolchain would eliminate this boundary entirely.
\end{enumerate}

\paragraph{Eliminated interface assumption.}

The previous \texttt{wlpo\_of\_nonreflexive\_witness} axiom---stating
that for any Banach space $X$, a non-reflexivity witness
$\Psi \in \Sbidual{X} \setminus \Jmap{X}(X)$ implies WLPO---has
been \emph{fully proven} in \texttt{WLPOFromWitness.lean} (196~lines).
The proof independently formalizes the Ishihara kernel construction
from Paper~2 \citep{Lee2025Paper2}: given
$\Psi \notin \mathrm{range}(\Jmap{X})$,
we find $h^\star \in \Sdual{X}$ with $\|\Psi\|/2 < |\Psi(h^\star)|$,
define $g(\alpha) := \text{if } (\forall n,\,\alpha_n = 0)
\text{ then } 0 \text{ else } h^\star$, and set
$\delta := |\Psi(h^\star)|/2 > 0$. The resulting Ishihara kernel
$(\Psi, 0, g, \delta)$ is consumed by a purely constructive decision
procedure to produce WLPO. The axiom profile of
\texttt{wlpo\_of\_nonreflexive\_witness\_proof} contains only Lean's
three foundational axioms (\texttt{propext}, \texttt{Classical.choice},
\texttt{Quot.sound})---no custom axioms.

This constitutes a second machine-checked proof of the hardest
direction of the bidual gap equivalence (non-reflexivity witness
$\Rightarrow$ WLPO), developed on \Mathlib{} v4.28 independently of
Paper~2's proof on \Mathlib{} v4.23. Two independent formalizations
of the same mathematical content compiling cleanly on different
library versions provides strong evidence of correctness.

\paragraph{Concrete witness.}

A sorry-backed \texttt{Instance.lean} (51~lines) demonstrates that
$S_1(\ell^2(\NN))$ satisfies \texttt{HasTraceClassContainer},
making the main theorem applicable to a specific physical space.
The sorry placeholders correspond to standard Schatten class
theory (trace norm isometry and closed range of the diagonal
embedding), which is not yet formalized in \Mathlib{}.

\paragraph{Conservation.}
The remaining interface assumption is \emph{conservative}: it asserts
a proposition proven in a verified companion formalization.
The combined system (Paper~2 + Paper~7) would have zero custom axioms
beyond Lean's core type theory and \Mathlib{}'s standard axioms.

\subsection{The Classical Metatheory}\label{sec:classical}

\paragraph{Axiom profile.}
All theorems in the \texttt{P7\_Full} formalization carry the axiom
profile \texttt{[propext, Classical.choice, Quot.sound]} plus the
interface assumption \texttt{ell1\_not\_reflexive}. The
\texttt{Classical.choice} dependency enters through \Mathlib{}'s
normed space and functional analysis infrastructure---specifically
through the Hahn--Banach theorem
(\texttt{exists\_extension\_norm\_eq}), the \texttt{NormedSpace}
typeclass hierarchy, and \texttt{Decidable} instances on~$\RR$.

\paragraph{What the formalization certifies.}
The Lean formalization provides two forms of evidence.
First, \emph{proof correctness}: the theorem statements are correctly
formalized, the proofs compile without \texttt{sorry} (in 7 of 8
files), and the proof chain---from the Ishihara kernel construction
through Lemma~B to the main equivalence---is machine-checked.
Second, \emph{proof structure}: the forward direction
(non-reflexivity witness $\Rightarrow$ WLPO) uses the Ishihara
kernel construction, which is a constructive algorithm with no
classical case analysis; the reverse direction
(WLPO $\Rightarrow$ non-reflexivity) uses WLPO explicitly as a
hypothesis; and the Lemma~B step (closed subspace of reflexive is
reflexive) uses Hahn--Banach extension and geometric separation,
both of which are constructively available for separable spaces
\citep{BridgesVita2006}.

\paragraph{Certification levels.}
The BISH claim for the overall equivalence rests on mathematical
argument about proof content, not on a \texttt{Classical.choice}-free
axiom certificate from the main formalization. A supplementary
artifact (\texttt{P7\_Minimal}, 277~lines, 4~files) provides a
dependency-free logical skeleton---analogous to the
\texttt{P2\_Minimal} artifact of Paper~2 \citep{Lee2025Paper2}---that
certifies the reduction structure without \Mathlib{}. In
\texttt{P7\_Minimal}, the forward direction (Ishihara kernel) and
Lemma~B are axiomatized with references to their \texttt{P7\_Full}
proofs, and the logical chain is verified to use no classical axioms
beyond the explicitly listed assumptions: the \texttt{\#print axioms}
output for \texttt{P7\_main} contains only the six declared interface
axioms---notably absent are \texttt{Classical.choice},
\texttt{propext}, and \texttt{Quot.sound}. The \texttt{P7\_Full}
proofs supply the mathematical content; \texttt{P7\_Minimal} certifies
the logical architecture.

A systematic treatment of the relationship between \Mathlib{}'s
classical foundations and the constructive claims across the paper
series is given in Paper~10 \citep[forthcoming]{Lee2026Paper7}, which
establishes three certification levels---\emph{mechanically certified},
\emph{structurally verified}, and \emph{paper-level}---and classifies
each paper accordingly. Paper~7 is classified as
``structurally verified'' (upgraded from ``paper-level'' upon
completion of \texttt{P7\_Minimal}).

% ====================================================================
\section{AI-Assisted Formalization Methodology}\label{sec:ai}
% ====================================================================

This formalization was developed using \textbf{Claude Opus~4.6}
(Anthropic, 2026) via the \textbf{Claude Code} command-line interface,
following the same human--AI workflow as Paper~2 \citep{Lee2025Paper2}.
The human author wrote a mathematical blueprint
(\texttt{PhysicalBidualGap\_Blueprint.md}, 520~lines) specifying all
theorem statements, proof strategies, and target \Mathlib{} APIs.
Claude Opus~4.6 then explored the \Mathlib{} codebase to locate exact
API signatures and import paths, generated the Lean~4 proof terms,
and handled the \Mathlib{} version migration from v4.23 to v4.28
(adapting to renamed lemmas, changed coercion behavior, and deprecated
import paths). The human author reviewed all proofs for mathematical
correctness and \Mathlib{} conventions. Final verification was by
\texttt{lake build} (0~errors, 0~warnings, 0~\texttt{sorry}).

\begin{table}[h]
\centering
\begin{tabular}{@{}lll@{}}
\toprule
\textbf{Task} & \textbf{Human} & \textbf{AI (Claude Opus 4.6)} \\
\midrule
Mathematical blueprint    & \checkmark & \\
Proof strategy design     & \checkmark & \\
\Mathlib{} API discovery  & & \checkmark \\
Lean proof generation     & & \checkmark \\
Proof review              & \checkmark & \\
Version migration         & & \checkmark \\
Build verification        & & \checkmark \\
Paper writing             & \checkmark & \checkmark \\
\bottomrule
\end{tabular}
\caption{Division of labor between human and AI.}
\label{tab:division}
\end{table}

\paragraph{Comparison with Paper~2.}
The methodology closely follows Paper~2, but Paper~7 required more
delicate interaction with \Mathlib{}'s Hahn--Banach infrastructure
and the additional challenge of a \Mathlib{} version migration
(v4.23 $\to$ v4.28) that broke coercion chains and renamed lemmas
across multiple files.

% ====================================================================
\section{Discussion and Implications}\label{sec:implications}
% ====================================================================

\subsection{Physical interpretation}

The Physical Bidual Gap Theorem reveals a constructive limitation
in the theory of quantum state spaces.  While Banach space
non-reflexivity of $\Sone$ is provable without WLPO---meaning
the bidual $\Sbidual{(\Sone)}$ is strictly larger than (the
image of) $\Sone$---one cannot \emph{constructively exhibit} any
specific element of the gap without WLPO.

In the algebraic formulation of quantum mechanics, the bidual
$\Sbidual{(\Sone)}$ contains ``singular states'' on $B(H)$:
positive linear functionals that are not $\sigma$-weakly
continuous and cannot be represented by any density matrix.
Classically, Gleason's theorem guarantees that in dimension
$\geq 3$, every countably additive probability measure on the
projection lattice of $H$ extends to a normal state---but our
result concerns the \emph{finitely} additive measures, the
singular states that live in the bidual gap.  Their existence
is provable, but constructing any specific one requires WLPO.

\paragraph{The thermodynamic limit.}
The physical home of singular states is the thermodynamic limit of
quantum statistical mechanics.  In the Haag--Kastler framework for
algebraic quantum field theory \citep{Haag1996}, observables localized
in bounded spacetime regions generate a net of local algebras; the
quasilocal algebra is their norm closure.  States on this algebra
decompose into \emph{folia}---equivalence classes under quasi-equivalence
of GNS representations \citep{BratteliRobinson1987}.  Normal states
belong to the folium of the identity representation; singular states
lie outside every folium.  Via the GNS construction, singular states
give rise to representations \emph{disjoint} from the identity
representation---physically, they correspond to inequivalent
thermodynamic phases or superselection sectors.

The decomposition $\omega = \omega_n + \omega_s$ into normal and
singular parts \citep{Takesaki1979} is the noncommutative Lebesgue
decomposition.  Our theorem calibrates the logical cost of witnessing
the singular part: determining whether $\omega_s \neq 0$ for a
constructively given state is a zero-test on $\|\omega_s\|$, and
zero-tests for reals are precisely what WLPO governs.

\paragraph{Calibration landscape.}
Our result fits into a broader stratification of the logical costs
of quantum statistical mechanics over constructive mathematics (BISH):

\begin{center}
\begin{tabular}{@{}ll@{}}
\toprule
\textbf{Physical statement} & \textbf{Minimal principle} \\
\midrule
Finite-volume Gibbs state properties & BISH (constructive) \\
$\Sone$ is not reflexive ($\lnot$-form) & WLPO \\
Singular state witness ($\exists\,\Psi \in \Sbidual{(\Sone)} \setminus \Jmap{\Sone}(\Sone)$) & $\geq$ WLPO \\
Thermodynamic limit via monotone convergence & LPO \\
Hahn--Banach separation in non-separable duals & LEM \\
Spectral gap decidability (general) & Undecidable \citep{CubittPerezWolf2015} \\
\bottomrule
\end{tabular}
\end{center}

\noindent
The first row is fully constructive: Peierls-type bounds,
finite-volume magnetization estimates, and explicit Gibbs state
calculations require no omniscience principles.  Our theorem occupies
the second and third rows.  The fourth row reflects the observation
that constructing infinite-volume limits via monotone convergence of
partition functions requires the Limited Principle of Omniscience
(LPO), which is strictly stronger than WLPO
\citep{BridgesVita2006}.  The fifth row marks the classical boundary:
full Hahn--Banach separation in non-separable duals (as needed to
construct singular functionals on $\ellinfty$ directly) requires the
law of excluded middle.  The sixth row, due to Cubitt, Perez-Garcia,
and Wolf \citep{CubittPerezWolf2015}, places general spectral gap
determination at the level of the Halting Problem---far beyond any
omniscience principle.

This landscape suggests that WLPO occupies a natural position in the
foundations of quantum theory: it is the precise logical cost of the
\emph{ontological status} of the singular sector---the assertion that
singular states exist as concrete mathematical objects, rather than
merely as a logical impossibility of reflexivity.

\subsection{The Role of WLPO}

WLPO sits at a precise level in the constructive hierarchy:
it is implied by the Limited Principle of Omniscience (LPO),
which also independently implies Markov's Principle (MP),
and it implies the Lesser Limited Principle of Omniscience (LLPO).
Thus $\mathrm{LPO} \Rightarrow \WLPO \Rightarrow \mathrm{LLPO}$,
while MP is independent of WLPO over BISH---neither implies the
other \citep{BridgesVita2006}.
Our result places the Banach space non-reflexivity witness for
$\Sone$ at this specific level of the constructive hierarchy,
rather than requiring full classical logic.

\subsection{Formalization as Verification}

The machine-checked nature of the proof provides a high degree
of confidence in the result. The single remaining interface
assumption is clearly delineated and corresponds to a proven result
in a companion formalization.
The proof structure---definition of the abstract interface,
modular lemmas, and clean assembly---demonstrates that
substantial functional analysis can be formalized in Lean~4
with existing \Mathlib{} infrastructure.

\subsection{Open Questions}

\begin{enumerate}
  \item \textbf{Eliminating the remaining axiom:} The sole remaining
    interface assumption (\texttt{ell1\_not\_reflexive}) requires
    porting Paper~2's dual isometry chain ($\sim$1,593~lines) to
    Lean v4.28. A unified monorepo with a shared Mathlib pin would
    eliminate this bridge entirely.

  \item \textbf{Schatten classes in \Mathlib{}:} A formalization of
    the Schatten $p$-classes $S_p(H)$ would replace both the abstract
    \texttt{HasTraceClassContainer} and the sorry-backed
    \texttt{Instance.lean} with concrete constructions,
    strengthening the result.

  \item \textbf{Generalization:} Does the equivalence extend to
    other physically relevant non-reflexive spaces, such as the
    space of bounded operators $B(H)$ or the predual of a
    von Neumann algebra?

  \item \textbf{Quantitative gaps:} Can the ``size'' of the bidual
    gap (e.g., in terms of cardinality or density character) be
    characterized constructively?
\end{enumerate}

% ====================================================================
\section*{Acknowledgments}
% ====================================================================

The Lean~4 formalization was developed using Claude Opus~4.6
(Anthropic, 2026) via the Claude Code CLI tool. We thank the
\Mathlib{} community for maintaining the comprehensive library
of formalized mathematics that made this work possible.

% ====================================================================
% Bibliography
% ====================================================================
\bibliographystyle{plainnat}

\begin{thebibliography}{20}

\bibitem[Bishop(1967)]{Bishop1967}
E.~Bishop.
\newblock \emph{Foundations of Constructive Analysis}.
\newblock McGraw-Hill, New York, 1967.

\bibitem[Bishop and Bridges(1985)]{BishopBridges1985}
E.~Bishop and D.~Bridges.
\newblock \emph{Constructive Analysis}.
\newblock Grundlehren der mathematischen Wissenschaften, vol.~279.
  Springer-Verlag, Berlin, 1985.

\bibitem[Bratteli and Robinson(1987)]{BratteliRobinson1987}
O.~Bratteli and D.~W.~Robinson.
\newblock \emph{Operator Algebras and Quantum Statistical Mechanics},
  vol.~1.
\newblock Texts and Monographs in Physics. Springer-Verlag, New York,
  2nd edition, 1987.

\bibitem[Bratteli and Robinson(1997)]{BratteliRobinson1997}
O.~Bratteli and D.~W.~Robinson.
\newblock \emph{Operator Algebras and Quantum Statistical Mechanics},
  vol.~2: Equilibrium States, Models in Quantum Statistical Mechanics.
\newblock Texts and Monographs in Physics. Springer-Verlag, Berlin,
  2nd edition, 1997.

\bibitem[Bridges and V{\^\i}{\c{t}}{\u{a}}(2006)]{BridgesVita2006}
D.~Bridges and L.~S.~V{\^\i}{\c{t}}{\u{a}}.
\newblock \emph{Techniques of Constructive Analysis}.
\newblock Universitext. Springer, New York, 2006.

\bibitem[Cubitt et~al.(2015)]{CubittPerezWolf2015}
T.~S.~Cubitt, D.~Perez-Garcia, and M.~M.~Wolf.
\newblock Undecidability of the spectral gap.
\newblock \emph{Nature}, 528(7581):207--211, 2015.

\bibitem[{de Moura} et~al.(2021)]{deMoura2021}
L.~{de Moura}, S.~Kong, J.~Avigad, F.~{van Doorn}, and M.~{von Raumer}.
\newblock The {Lean} theorem prover (system description).
\newblock In \emph{CADE-25}, LNAI 9195, pages 378--388. Springer, 2015.
\newblock Lean~4: \url{https://lean-lang.org/}, 2021--present.

\bibitem[Haag(1996)]{Haag1996}
R.~Haag.
\newblock \emph{Local Quantum Physics: Fields, Particles, Algebras}.
\newblock Texts and Monographs in Physics. Springer-Verlag, Berlin,
  2nd edition, 1996.

\bibitem[Ishihara(1992)]{Ishihara1992}
H.~Ishihara.
\newblock Continuity properties in constructive mathematics.
\newblock \emph{Journal of Symbolic Logic}, 57(2):557--565, 1992.

\bibitem[Lee(2026a)]{Lee2025Paper2}
P.~C.-K.~Lee.
\newblock The bidual gap: A {Lean}~4 formalization of {WLPO} and
  non-reflexivity in {B}anach spaces.
\newblock Preprint, 2026.
\newblock Lean~4 formalization:
  \url{https://github.com/quantmann/FoundationRelativity}.

\bibitem[Lee(2026b)]{Lee2026Paper7}
P.~C.-K.~Lee.
\newblock The physical bidual gap and {B}anach space non-reflexivity:
  a {Lean}~4 formalization of {WLPO} via trace-class operators.
\newblock Preprint, 2026.
\newblock Lean~4 formalization:
  \url{https://github.com/quantmann/FoundationRelativity}.

\bibitem[{Mathlib Community}(2020--)]{Mathlib}
{Mathlib Community}.
\newblock \emph{Mathlib}: the math library for {Lean}.
\newblock \url{https://leanprover-community.github.io/mathlib4_docs/},
  2020--present.

\bibitem[Takesaki(1979)]{Takesaki1979}
M.~Takesaki.
\newblock \emph{Theory of Operator Algebras~I}.
\newblock Springer-Verlag, New York, 1979.

\bibitem[van Wierst(2019)]{vanWierst2019}
P.~van~Wierst.
\newblock The paradox of phase transitions in the light of constructive
  mathematics.
\newblock \emph{Synthese}, 196(5):1863--1884, 2019.

\bibitem[Anthropic(2026)]{Anthropic2026}
Anthropic.
\newblock Claude {Opus}~4.6 and {Claude Code} {CLI}.
\newblock \url{https://www.anthropic.com/claude}, 2026.

\end{thebibliography}

\end{document}

\documentclass[11pt]{article}

\usepackage[margin=1in]{geometry}
\usepackage{amsmath,amssymb,mathtools}
\usepackage{amsthm}
\usepackage[american]{babel}
\usepackage{enumitem}
\usepackage{booktabs}
\usepackage{array}
\usepackage{url}
\usepackage[colorlinks=true,linkcolor=blue,citecolor=blue,urlcolor=blue]{hyperref}

\theoremstyle{plain}
\newtheorem{theorem}{Theorem}[section]
\newtheorem{proposition}[theorem]{Proposition}
\newtheorem{lemma}[theorem]{Lemma}
\newtheorem{corollary}[theorem]{Corollary}

\theoremstyle{definition}
\newtheorem{definition}[theorem]{Definition}

\theoremstyle{remark}
\newtheorem{remark}[theorem]{Remark}

\newcommand{\N}{\mathbb{N}}
\newcommand{\Z}{\mathbb{Z}}
\newcommand{\Q}{\mathbb{Q}}
\newcommand{\R}{\mathbb{R}}
\newcommand{\C}{\mathbb{C}}
\newcommand{\F}{\mathbb{F}}
\newcommand{\Fp}{\mathbb{F}_p}
\newcommand{\Fbar}{\overline{\mathbb{F}}}
\newcommand{\A}{\mathbb{A}}
\newcommand{\Pp}{\mathbb{P}}
\newcommand{\HH}{\mathbb{H}}
\newcommand{\calO}{\mathcal{O}}
\newcommand{\fm}{\mathfrak{m}}
\newcommand{\Gal}{\mathrm{Gal}}
\newcommand{\GL}{\mathrm{GL}}
\newcommand{\PGL}{\mathrm{PGL}}
\newcommand{\SU}{\mathrm{SU}}
\newcommand{\SL}{\mathrm{SL}}
\newcommand{\Sym}{\mathrm{Sym}}
\newcommand{\Frob}{\mathrm{Frob}}
\newcommand{\Sel}{\mathrm{Sel}}
\newcommand{\ad}{\mathrm{ad}}
\newcommand{\tr}{\mathrm{tr}}
\newcommand{\vol}{\mathrm{vol}}
\newcommand{\Spec}{\mathrm{Spec}}
\newcommand{\rk}{\mathrm{rk}}
\newcommand{\ACF}{\mathrm{ACF}}

\newcommand{\BISH}{\mathsf{BISH}}
\newcommand{\LPO}{\mathsf{LPO}}
\newcommand{\WLPO}{\mathsf{WLPO}}
\newcommand{\LLPO}{\mathsf{LLPO}}
\newcommand{\MP}{\mathsf{MP}}
\newcommand{\FT}{\mathsf{FT}}
\newcommand{\WKL}{\mathsf{WKL}_0}
\newcommand{\CLASS}{\mathsf{CLASS}}


\title{The Weight~1 Boundary:\\
  Constructivising the $\GL_2$ Langlands Programme\\[6pt]
  \large (Paper~71 of the Constructive Reverse Mathematics Series)}

\author{Paul Chun-Kit Lee \\
\texttt{dr.paul.c.lee@gmail.com}}

\date{February 2026}

\begin{document}
\maketitle

% ============================================================
\begin{abstract}
% ============================================================

Papers~68--70 showed that every modularity theorem for $\GL_2/\Q$
calibrates at $\BISH + \WLPO$, with the $\WLPO$ entering through
the Arthur--Selberg trace formula in three roles: Langlands--Tunnell
(base change), Jacquet--Langlands (quaternionic transfer), and
Jacquet--Langlands (level-lowering).  We investigate whether these
$\WLPO$ atoms can be eliminated.

Two of the three are eliminated.  By restructuring the proof to
work on compact Shimura curves (attaching Galois representations
via \'etale cohomology rather than transferring to $\GL_2$), the
Jacquet--Langlands transfer becomes unnecessary for the modularity
lifting step.  For the Jacquet--Langlands comparison at weight
$\ge 2$, we prove a \emph{decidability descent}: the trace formula
identity reduces to an equation between algebraic numbers,
verifiable in the decidable first-order theory $\ACF$.  The
analytic proof is classical scaffolding; the algebraic content
is~$\BISH$.

The third $\WLPO$ atom---Langlands--Tunnell at weight~1---is
irreducible.  We identify three independent obstructions, all
specific to weight~1: (i)~Archimedean orbital integrals involve
transcendental regulators that do not cancel; (ii)~the spectral
side mixes holomorphic forms with Maass forms of eigenvalue
$\lambda = 1/4$, whose Hecke eigenvalues are generically
transcendental; (iii)~the dimension of the space of weight~1 forms
has no algebraic formula (Riemann--Roch gives only the Euler
characteristic).  The decidability descent fails on all three fronts.

The $\WLPO$ in Fermat's Last Theorem is localised to a single
irreducible atom: extracting a weight~1 holomorphic form from
the $L^2$ spectral decomposition of a non-compact locally symmetric
space.  This atom lies at the exact boundary between algebra and
analysis, where the discrete and continuous spectra of the
Laplacian meet at $\lambda = 1/4$.

\end{abstract}

\tableofcontents


% ============================================================
\section{Introduction}\label{sec:intro}
% ============================================================

Papers~68--70 audited the $\GL_2/\Q$ Langlands programme and
found a uniform classification: $\BISH + \WLPO$.  The $\WLPO$
entered through three structurally distinct applications of the
Arthur--Selberg trace formula.  The present paper asks: are all
three necessary?

The answer is \emph{no} for two of them and \emph{yes} for the
third.  The constructive profile of Wiles's proof can be improved,
but not to $\BISH$.  The irreducible obstruction is the spectral
theory of weight~1 automorphic forms.

\subsection{Summary of results}

\begin{center}
\renewcommand{\arraystretch}{1.15}
\begin{tabular}{@{}llll@{}}
\toprule
\textbf{$\WLPO$ source (Paper~70)} &
  \textbf{Weight} &
  \textbf{Status (Paper~71)} &
  \textbf{Method} \\
\midrule
Langlands--Tunnell (base change) & 1 &
  $\WLPO$ (irreducible) & --- \\
Jacquet--Langlands (icosahedral) & $\ge 2$ &
  $\BISH$ & Shimura curve, Route~3 \\
Jacquet--Langlands (level-lowering) & $\ge 2$ &
  $\BISH$ & Decidability descent \\
\bottomrule
\end{tabular}
\end{center}


% ============================================================
\section{Preliminaries}\label{sec:prelim}
% ============================================================

We work within Bishop-style constructive mathematics ($\BISH$)
as codified by Bishop and Bridges~\cite{BishopBridges1985}.
The logical principles relevant to this paper are:

\begin{definition}[$\WLPO$: Weak Limited Principle of Omniscience]
For every binary sequence $(a_n)$, either $\forall n.\, a_n = 0$
or $\lnot\forall n.\, a_n = 0$.  Equivalently, for every real
number~$x$, either $x = 0$ or $x \ne 0$.
\end{definition}

\begin{definition}[$\LPO$: Limited Principle of Omniscience]
For every binary sequence $(a_n)$, either $\forall n.\, a_n = 0$
or $\exists n.\, a_n = 1$.
\end{definition}

\begin{definition}[$\BISH$: Bishop's constructive mathematics]
Intuitionistic logic plus dependent choice.  No omniscience
principles.  Every function is uniformly continuous on compact
sets.
\end{definition}

The classification hierarchy is
$\BISH \subsetneq \BISH + \WLPO \subsetneq \BISH + \LPO
\subsetneq \CLASS$.  See Bridges and
Richman~\cite{BridgesRichman1987} for details.

We axiomatise the following objects from number theory and
automorphic forms without proof:

\begin{enumerate}[label=(\roman*),nosep]
\item The Arthur--Selberg trace formula for $\GL_2$ and for
  inner forms $D^\times$ (as in~\cite{Arthur2005}).
\item The Langlands--Tunnell theorem: tetrahedral and octahedral
  Artin representations over $\Q$ are modular
  (\cite{Langlands1980}, \cite{Tunnell1981}).
\item The Jacquet--Langlands correspondence between $\GL_2$ and
  $D^\times$~\cite{JacquetLanglands1970}.
\item Taylor--Wiles patching and the Breuil--Conrad--Diamond--Taylor
  modularity lifting theorem~\cite{Wiles1995}.
\item Decidability of the first-order theory of algebraically
  closed fields ($\ACF$; Tarski--Seidenberg).
\end{enumerate}

The constructive content of each axiomatised object is assessed
in the CRM audit (\S\ref{sec:audit}).  For the position of this
paper within the series atlas, see Papers~50--53
(\cite{Paper50}, \cite{Paper52}, \cite{Paper53}).


% ============================================================
\section{The Decidability Descent}\label{sec:descent}
% ============================================================

The central technique is a meta-mathematical argument that
converts a classical analytic proof into a constructive
algebraic verification.

\begin{definition}[Decidability descent]\label{def:descent}
Let $\varphi$ be a sentence in the first-order theory of
algebraically closed fields ($\ACF$).  If $\varphi$ is proved
classically (using $\WLPO$ or stronger principles), it
nevertheless holds constructively, because $\ACF$ is decidable
(Tarski--Seidenberg).

More concretely: if the Arthur--Selberg trace formula, applied
with specific test functions, produces an identity
$\sum_{i=1}^N a_i = \sum_{j=1}^M b_j$
where all $a_i$, $b_j$ are algebraic numbers and the bounds
$N$, $M$ are algebraically computable, then the identity is
$\BISH$-verifiable regardless of whether its proof used
non-constructive analysis.
\end{definition}

For the descent to succeed, three conditions must hold:
\begin{enumerate}[label=(\roman*),nosep]
\item The terms on both sides are algebraic numbers.
\item The index bounds $N$, $M$ are computable by algebraic
  geometry (e.g., Riemann--Roch, Brandt matrix dimensions).
\item No transcendental quantities remain after cancellation.
\end{enumerate}

This is a special case of the de-omniscientising descent
identified in Paper~50: classical logic establishes a result
whose content is constructively verifiable at a lower logical
level.


% ============================================================
\section{The Algebraic Trace Formula for Quaternion Algebras}%
\label{sec:quat}
% ============================================================

\begin{theorem}[$D^\times$ trace formula is $\BISH$]%
\label{thm:dxtf}
For a definite quaternion algebra $D$ over a totally real
field~$F$, the Selberg trace formula with algebraic test
functions is an identity between algebraic numbers.
\end{theorem}

\begin{proof}[Proof sketch]
\emph{Non-Archimedean places:} The orbital integrals of
characteristic functions of maximal compact subgroups are
non-negative integers (counting fixed vertices in the
Bruhat--Tits tree).

\emph{Archimedean places:} At each real place,
$D_v^\times / F_v^\times \cong \SU(2)$.  The test function
is a character of an irreducible representation
$\Sym^n(\C^2)$, which is a class function.  The orbital
integral factors as $f_v(\gamma) \cdot \vol(T_v \backslash \SU(2))$.
The character $\chi_n(\theta) = U_n(\cos\theta)$, a
Chebyshev polynomial with integer coefficients, evaluated at
$\cos\theta = \tr(\gamma)/(2\sqrt{N(\gamma)})$, which is
algebraic since $\gamma \in D^\times(F)$.  With canonical
normalisations ($\vol(\SU(2)) = 1$), the integral is algebraic.

\emph{Volumes:} For regular semisimple $\gamma \notin F$,
the centraliser $K = F(\gamma)$ is a CM extension of~$F$.
The volume $\vol(D_\gamma^\times(F) \backslash D_\gamma^\times(\A_F))$
reduces to the relative class number $h_K / h_F$ divided by
the number of relative roots of unity.  This is rational:
the transcendental regulators $R_K$ and $R_F$ cancel by
Dirichlet's unit theorem (the unit ranks of a CM field and
its totally real subfield coincide).

\emph{Spectral side:} The space of automorphic forms is
finite-dimensional (compact quotient).  Hecke operators act
via Brandt matrices with integer entries.  Traces are integers.

Both sides are algebraic.  Transcendental factors from
Tamagawa measure normalisation (powers of~$\pi$, $L$-values)
appear identically on both sides and cancel by the analytic
class number formula and the Klingen--Siegel theorem.
\end{proof}


% ============================================================
\section{Eliminating Jacquet--Langlands: The Shimura Curve
  Strategy}\label{sec:route3}
% ============================================================

\subsection{The restructuring}

Paper~70 identified the $D^\times \to \GL_2$ transfer
(Jacquet--Langlands correspondence) as a $\WLPO$ bottleneck.
We eliminate it by restructuring the proof to avoid
$\GL_2$ automorphic forms entirely during the lifting step.

\begin{proposition}[Shimura curve restructuring]%
\label{prop:route3}
Let $D/F$ be an indefinite quaternion algebra (split at
exactly one Archimedean place).  The associated Shimura curve
$\mathrm{Sh}_D$ is a compact algebraic curve over~$F$.
\begin{enumerate}[label=(\roman*),nosep]
\item The \'etale cohomology $H^1_{\mathrm{et}}(\mathrm{Sh}_D
  \times_F \bar{F},\, \Q_\ell)$ carries a $2$-dimensional
  Galois representation attached to each automorphic
  representation of $D^\times(\A_F)$ via the Eichler--Shimura
  relation and Carayol's theorem.
\item The Taylor--Wiles patching method can be executed entirely
  on $H^1_{\mathrm{et}}(\mathrm{Sh}_D)$, without transferring
  to $\GL_2$.
\item The Galois representation is attached directly to the
  $D^\times$ form.  No Jacquet--Langlands transfer is needed.
\end{enumerate}
\end{proposition}

This is precisely how Taylor's potential modularity theorem
operates: modularity lifting on the cohomology of compact
Shimura varieties, where the compactness eliminates boundary
contributions and makes the deformation theory clean.

\subsection{The Jacquet--Langlands comparison at weight $\ge 2$}

For the applications in the Khare--Wintenberger induction
(level-lowering, where a form must be transferred between
inner forms of $\GL_2$), we need the Jacquet--Langlands
comparison.  At weight $\ge 2$, the decidability descent
applies.

\begin{theorem}[Jacquet--Langlands at weight $\ge 2$ is $\BISH$]%
\label{thm:jl}
The Jacquet--Langlands correspondence for $\GL_2$, restricted
to automorphic representations of weight $k \ge 2$, is
$\BISH$-verifiable via decidability descent.
\end{theorem}

\begin{proof}[Proof sketch]
\emph{Geometric side:}
The Archimedean test function is a pseudo-coefficient of the
discrete series of weight~$k$.  By Harish-Chandra's theory,
its orbital integrals on hyperbolic (split) elements vanish.
The surviving elliptic orbital integrals evaluate to algebraic
character values.  The non-Archimedean orbital integrals are
integers (\S\ref{sec:quat}).  Volume factors (powers of $\pi$)
cancel between the $\GL_2$ and $D^\times$ sides.

\emph{Spectral side:}
The traces are products of Hecke eigenvalues of holomorphic
modular forms, which are algebraic integers.

\emph{Index bounds:}
The dimension of the space of cusp forms of weight~$k$ and
level~$N$ is computable by the Riemann--Roch theorem (or
the Eichler--Selberg dimension formula):
$\dim S_k(\Gamma_0(N))$ is an explicit arithmetic function
of $k$ and~$N$.

All three conditions of the decidability descent
(Definition~\ref{def:descent}) are satisfied.  The trace
formula identity is a sentence in $\ACF$, verified classically,
hence constructively valid.
\end{proof}

\begin{remark}[Weight $\ge 2$ versus weight~$1$]%
\label{rem:weight}
At weight~$\ge 2$, the pseudo-coefficient kills the hyperbolic
orbital integrals (eliminating transcendental regulators), the
spectral side contains only holomorphic forms (no Maass form
contamination), and Riemann--Roch computes the exact dimension
(no $H^1$ obstruction).  All three properties fail at weight~$1$.
The weight~$1$ / weight~$\ge 2$ boundary is the exact logical
fault line.
\end{remark}


% ============================================================
\section{The Octahedral Reduction}\label{sec:octahedral}
% ============================================================

Langlands--Tunnell proves modularity of $2$-dimensional Artin
representations with solvable projective image.  The three
solvable cases are dihedral ($D_n$, handled by Hecke theta
series, $\BISH$), tetrahedral ($A_4$, Langlands), and
octahedral ($S_4$, Tunnell).

Tunnell's original proof uses the Gelbart--Jacquet symmetric
square lifting to $\GL_3$, which invokes the converse theorem
for $\GL_3$.  This would introduce an additional $\WLPO$ source.

\begin{proposition}[Octahedral modularity without $\GL_3$]%
\label{prop:oct}
Over~$\Q$, octahedral modularity reduces to iterated cyclic
base change for $\GL_2$.  The $\GL_3$ converse theorem is
not needed.
\end{proposition}

\begin{proof}[Proof sketch]
The key obstruction is the \emph{quadratic twist ambiguity}:
descending from $A_4$ to $S_4$ via the quadratic quotient
$S_4/A_4 \cong \Z/2$ produces two candidates ($\pi$ and
$\pi \otimes \eta$) that are indistinguishable by central
character (since $\eta^2 = 1$).  Over arbitrary number
fields, Tunnell resolves this via the symmetric square
(which annihilates the ambiguity since
$\Sym^2(\pi \otimes \eta) = \Sym^2(\pi)$).

Over $\Q$, the Deligne--Serre theorem provides an
algebraic resolution: since $\bar{\rho}$ is odd, the
descended form is classical holomorphic of weight~$1$,
and Deligne--Serre attaches a Galois representation.
Both twist candidates $\pi$ and $\pi \otimes \eta$ are
modular (twisting a modular form by a Dirichlet character
preserves modularity).  The ambiguity is harmless.

The $V_4 \cong \Z/2 \times \Z/2$ subgroup of $A_4$ and $S_4$
is not an obstruction: representations with projective image
$V_4$ are dihedral, modular by Hecke theta series ($\BISH$).

Consequently, octahedral modularity over $\Q$ uses:
dihedral modularity ($\BISH$), cubic base change ($\WLPO$),
quadratic base change ($\WLPO$), and Deligne--Serre ($\BISH$).
No $\GL_3$.
\end{proof}

The $\WLPO$ in Langlands--Tunnell is therefore concentrated
in the cyclic base change steps, which use the $\GL_2$ trace
formula at weight~$1$.


% ============================================================
\section{The Weight~$1$ Obstruction}\label{sec:weight1}
% ============================================================

The Langlands--Tunnell theorem requires cyclic base change for
$\GL_2$ at weight~$1$.  We show that the decidability descent
fails for weight~$1$ on three independent fronts.

\subsection{Failure~1: Transcendental Archimedean integrals}

To isolate weight~$1$ forms, the Archimedean test function is
the character of a limit of discrete series.  Unlike the
discrete series for weight $k \ge 2$, the limit of discrete
series does \emph{not} vanish on hyperbolic (split) elements.
The orbital integral over the hyperbolic torus introduces the
volume of the split torus modulo the centraliser, which
evaluates to $\log \varepsilon_K$, the logarithm of the
fundamental unit of a real quadratic field---a transcendental
number.

These transcendental terms do not cancel between the two
sides of the base change comparison (they depend on the
specific splitting behaviour at each place).

\subsection{Failure~2: Maass form contamination}

The limit of discrete series at weight~$1$ shares its
infinitesimal character with the principal series at Laplacian
eigenvalue $\lambda = 1/4$.  The spectral side of the trace
formula at weight~$1$ therefore receives contributions from
both holomorphic weight~$1$ forms \emph{and} odd Maass forms
of eigenvalue~$1/4$:
\[
  \text{Spectral side} = \sum_{\pi \,\text{hol, wt 1}}
  \tr \pi(f) \;-\; \sum_{\pi' \,\text{Maass, }\lambda = 1/4}
  \tr \pi'(f).
\]
The Hecke eigenvalues of Maass forms are generically
transcendental.  The identity equates sums of real numbers,
not algebraic numbers.  The Tarski decidability argument does
not apply.

\subsection{Failure~3: Unknown index bounds}

Even if the terms were algebraic, the decidability descent
requires computable bounds on the number of terms.  For
weight~$1$, the dimension of $H^0(X, \omega)$ has no purely
algebraic formula.  The Riemann--Roch theorem gives
$\dim H^0 - \dim H^1$, but $H^1$ is non-zero and jumps
erratically (it depends on the existence of specific
automorphic representations, not on arithmetic invariants of
the curve).  Without knowing the index set, the identity
$\sum_i a_i = \sum_j b_j$ cannot be formulated as a sentence
in~$\ACF$.

\begin{theorem}[The weight~$1$ obstruction is irreducible]%
\label{thm:weight1}
The Langlands--Tunnell base change at weight~$1$ cannot be
constructivised by decidability descent.  The $\WLPO$ is
irreducible: extracting a weight~$1$ holomorphic form from the
$L^2$ spectral decomposition of
$\GL_2(F) \backslash \GL_2(\A_F)$ requires separating the
discrete spectrum from the continuous spectrum at the point
$\lambda = 1/4$ where they meet.
\end{theorem}

\subsection{Failure~4: The $p$-adic initialisation trap}

The $p$-adic theory of overconvergent modular forms
(Buzzard--Taylor \cite{BuzzardTaylor1999}, Kassaei
\cite{Kassaei2006}, Pilloni \cite{Pilloni2012}) provides a
purely $p$-adic construction of weight~$1$ forms that avoids
all three Archimedean obstructions: Hida families are~$\BISH$,
specialisation is~$\BISH$, and Kassaei--Pilloni classicality
is~$\BISH$ (using $U_p$ as a contraction mapping on rigid
analytic spaces, with no Archimedean input).

However, the $p$-adic machinery cannot \emph{originate}
modularity.  To place $\bar{\rho}_3$ on the eigencurve, one
must find a classical modular form of weight $k \ge 2$ whose
residual Galois representation matches~$\bar{\rho}_3$.  This
requires lifting $\bar{\rho}_3$ to characteristic zero and
invoking Langlands--Tunnell to produce the initial modular
form.  Modularity lifting theorems (Taylor--Wiles, Kisin,
Calegari--Geraghty) are similarly relative: they transform
modularity from one form to another but cannot create it.

The trace formula is the unique \emph{absolute} bridge from
Galois representations to automorphic forms.  All other
methods are relative---they propagate modularity but cannot
initialise it.  The $\WLPO$ at weight~$1$ is not an artefact
of choosing the wrong construction; it is the logical cost of
\emph{creating} a modular form from a Galois representation
\emph{ex nihilo}.

\subsection{Failure~5: The universal quantifier}\label{sec:univ}

The preceding four failures concern the trace formula approach
and the $p$-adic approach.  A third approach---direct algebraic
computation---reveals the deepest structure of the obstruction.

For any \emph{specific} conductor~$N$, the space $S_1(N, \chi)$
is explicitly computable by purely algebraic methods ($\BISH$).
The algorithm uses the \emph{Eisenstein trick}: choose two
Eisenstein series $E_a, E_b$ of weight~$1$ with no common zeros
on $X_1(N)$; then $S_1(N, \chi)$ is the intersection of the
images of the two injections
$f \mapsto f \cdot E_a$ and $f \mapsto f \cdot E_b$ into
$S_2(N, \chi\psi)$, which is computable via modular symbols.
The Sturm bound provides a finite verification criterion.
For any given~$N$, one can construct a candidate $q$-expansion
from the Galois representation and verify its modularity by
finite linear algebra.

However, Fermat's Last Theorem is a proof by contradiction
about a \emph{hypothetical} Frey curve $E_{a,b,c}$ whose
conductor $N = \mathrm{rad}(abc)$ is an unbounded variable.
One cannot execute a linear algebra algorithm on an unspecified
input.  To complete the proof, one needs a \emph{universal}
existence theorem:
\begin{quote}
  For \emph{every} conductor~$N$ and every valid $S_4$
  representation of conductor~$N$, the space $S_1(N, \chi)$
  contains the corresponding eigenform.
\end{quote}
This requires proving $\dim S_1(N, \chi) \ge 1$ in the relevant
eigenspace for all~$N$.  Because $H^1(X_1(N), \calO) \ne 0$ at
weight~$1$, the Riemann--Roch theorem gives only the Euler
characteristic $h^0 - h^1$, not a lower bound on~$h^0$.  The
only known method for forcing $h^0 \ge 1$ universally is the
Arthur--Selberg trace formula.

\begin{remark}[The ontology of weight~$1$ forms]%
\label{rem:ontology}
The $\WLPO$ is not the cost of constructing weight~$1$ forms
(which are algebraic, computable objects).  It is the cost of
the \emph{universal quantifier}: proving that the algebraic
algorithm will succeed for every possible input.  The trace
formula is needed not because weight~$1$ forms are analytic,
but because proving they \emph{always exist} requires analytic
methods.  The distinction is between \emph{verification}
(checking a specific instance---$\BISH$) and \emph{certification}
(proving the algorithm never fails---$\WLPO$).
\end{remark}


% ============================================================
\section{The Optimised Classification}\label{sec:main}
% ============================================================

\begin{theorem}[Optimised classification of FLT]%
\label{thm:main}
Wiles's proof of Fermat's Last Theorem, restructured via the
Shimura curve strategy (\S\ref{sec:route3}) and the
decidability descent (\S\ref{sec:descent}), calibrates at
$\BISH + \WLPO$.  The $\WLPO$ consists of a single irreducible
atom: the Langlands--Tunnell theorem at weight~$1$.
\end{theorem}

\begin{proof}
The restructured proof has the following components:

\medskip
\begin{center}
\renewcommand{\arraystretch}{1.15}
\begin{tabular}{@{}llll@{}}
\toprule
\textbf{Component} &
  \textbf{Weight} &
  \textbf{Classification} &
  \textbf{Method} \\
\midrule
\multicolumn{4}{@{}l}{\emph{Base case}} \\[2pt]
\quad Dihedral modularity & --- &
  $\BISH$ & Hecke theta series \\
\quad Tetrahedral base change & 1 &
  $\WLPO$ & GL$_2$ trace formula \\
\quad Octahedral descent & 1 &
  $\WLPO$ & GL$_2$ trace formula \\
\quad $\GL_2 \to D^\times$ transfer & 1 &
  $\WLPO$ & JL trace formula \\
\midrule
\multicolumn{4}{@{}l}{\emph{Lifting}} \\[2pt]
\quad TW patching on Shimura curve & $\ge 2$ &
  $\BISH$ & Brochard, eff.\ Chebotarev \\
\quad Galois rep via \'etale cohom. & $\ge 2$ &
  $\BISH$ & Eichler--Shimura--Carayol \\
\midrule
\multicolumn{4}{@{}l}{\emph{Induction (Khare--Wintenberger)}} \\[2pt]
\quad Level-lowering (JL comparison) & $\ge 2$ &
  $\BISH$ & Decidability descent \\
\quad Level-raising & $\ge 2$ &
  $\BISH$ & Ihara, supersingular locus \\
\quad Weight reduction & $\ge 2$ &
  $\BISH$ & Hasse invariant, $\theta$ \\
\quad Serre's recipe & --- &
  $\BISH$ & Finite group theory \\
\quad Potential modularity & $\ge 2$ &
  $\BISH$ & Moret-Bailly \\
\midrule
\textbf{Overall} & &
  $\BISH + \WLPO$ & \\
\bottomrule
\end{tabular}
\end{center}

\medskip\noindent
The $\WLPO$ components (tetrahedral/octahedral base change,
$\GL_2 \to D^\times$ transfer) all occur at weight~$1$ and all
use the $\GL_2$ trace formula.  They constitute a single
irreducible atom: the Langlands--Tunnell theorem.
\end{proof}

\begin{corollary}[Improvement over Paper~70]%
\label{cor:improvement}
Paper~70 identified three $\WLPO$ sources.  Paper~71 reduces
this to one:

\begin{center}
\renewcommand{\arraystretch}{1.1}
\begin{tabular}{@{}lll@{}}
\toprule
\textbf{$\WLPO$ source} &
  \textbf{Paper~70} &
  \textbf{Paper~71} \\
\midrule
Langlands--Tunnell & $\WLPO$ & $\WLPO$ (irreducible) \\
JL: icosahedral transfer & $\WLPO$ & $\BISH$ (Route~3) \\
JL: level-lowering over $F$ & $\WLPO$ & $\BISH$ (descent) \\
\bottomrule
\end{tabular}
\end{center}
\end{corollary}

\begin{corollary}[Algebraic weight~$1$ modularity]%
\label{cor:alg}
If a purely algebraic (or purely $p$-adic) \emph{absolute}
construction of weight~$1$ modular forms exists---one that
attaches automorphic forms to Galois representations without
the $L^2$ spectral decomposition---then Fermat's Last
Theorem is~$\BISH$.

The existing $p$-adic methods (Hida families, overconvergent
forms, Kassaei--Pilloni classicality) provide~$\BISH$
\emph{relative} constructions: they propagate modularity from
weight~$\ge 2$ to weight~$1$ and between congruent forms.
What is missing is an \emph{absolute} $p$-adic bridge from
Galois representations to automorphic forms, bypassing the
analytic trace formula entirely.
\end{corollary}


% ============================================================
\section{CRM Audit}\label{sec:audit}
% ============================================================

\subsection{Constructive strength classification}

\begin{center}
\renewcommand{\arraystretch}{1.15}
\begin{tabular}{@{}lll@{}}
\toprule
\textbf{Result} &
  \textbf{Classification} &
  \textbf{Principle} \\
\midrule
$D^\times$ trace formula algebraic (Thm.~\ref{thm:dxtf}) &
  $\BISH$ & Algebraic identity \\
Shimura curve restructuring (Prop.~\ref{prop:route3}) &
  $\BISH$ & \'Etale cohomology \\
JL at weight $\ge 2$ (Thm.~\ref{thm:jl}) &
  $\BISH$ & Decidability descent \\
Octahedral without $\GL_3$ (Prop.~\ref{prop:oct}) &
  $\BISH$ + $\WLPO$ & $\GL_2$ base change \\
Weight~$1$ obstruction (Thm.~\ref{thm:weight1}) &
  $\WLPO$ (irreducible) & Spectral decomposition \\
Optimised FLT (Thm.~\ref{thm:main}) &
  $\BISH + \WLPO$ & Single atom \\
\bottomrule
\end{tabular}
\end{center}

\subsection{Descent pattern}

Paper~71 exhibits the \emph{decidability descent} pattern:
classical analysis establishes an identity whose content is an
algebraic sentence in $\ACF$, hence constructively verifiable.
This descent reduces $\BISH + \WLPO$ to $\BISH$ for the
weight~$\ge 2$ Jacquet--Langlands comparison.

The pattern is an instance of the de-omniscientising descent
identified in Paper~50~\cite{Paper50}: a classical proof
provides an existence statement whose content descends to a
lower logical level.

\subsection{Comparison with Papers~68--70}

Papers~68--70 classified the $\GL_2/\Q$ Langlands program as
$\BISH + \WLPO$ with three independent $\WLPO$ atoms.
Paper~71 reduces this to one irreducible atom.  The calibration
matches the pattern of Papers~2, 7, and~8: initial audits find
multiple non-constructive dependencies; subsequent optimisation
reduces them to a minimal core.


% ============================================================
\section{The Weight~$1$ Boundary as Logical Fault Line}%
\label{sec:boundary}
% ============================================================

The boundary between weight~$\ge 2$ and weight~$1$ is the
exact logical fault line of the $\GL_2$ Langlands programme.

At weight $\ge 2$, every component of the programme is
$\BISH$: deformation theory, Hecke algebras, patching, Galois
representations (via \'etale cohomology of Shimura varieties),
the Jacquet--Langlands correspondence (via decidability
descent), effective Chebotarev bounds, Breuil's classification,
Moret-Bailly's construction, level-raising, level-lowering,
weight reduction.

At weight~$1$, the programme crosses into analytic territory.
Holomorphic weight~$1$ forms do not appear in the \'etale
cohomology of modular curves.  They live in $H^0(X, \omega)$,
not $H^1$.  They cannot be constructed by algebraic geometry.
Their existence must be extracted from the $L^2$ spectral
decomposition by separating them from Maass forms of
eigenvalue~$\lambda = 1/4$.  This separation requires deciding
whether a real eigenvalue equals $1/4$---which is $\WLPO$.

The continuous and discrete spectra meet at $\lambda = 1/4$.
This is the Selberg eigenvalue conjecture boundary: cuspidal
Maass forms satisfy $\lambda \ge 1/4$, and the continuous
spectrum starts at exactly $\lambda = 1/4$.  There is no
spectral gap.

The $\WLPO$ in Fermat's Last Theorem is not an artefact of
proof technology.  It is located at a genuine boundary in the
spectral theory of non-compact locally symmetric spaces: the
point where the Laplacian's discrete and continuous spectra
are indistinguishable without an oracle for real equality.


% ============================================================
\section{Implications}\label{sec:implications}
% ============================================================

\subsection{For the programme thesis}

The programme asks: is logical cost intrinsic to theorems or
to proofs?  Paper~71 provides a precise answer for FLT.

The $\WLPO$ is not intrinsic to the \emph{content} of
modularity (weight~$1$ forms are algebraic, computable
objects).  It is not intrinsic to the algebraic or $p$-adic
machinery (which is $\BISH$).  It is intrinsic to the
\emph{universality} of the modularity statement: proving that
weight~$1$ eigenforms exist for \emph{every} valid Galois
representation, across the infinite landscape of all possible
conductors.

For any specific instance, modularity is $\BISH$-verifiable.
The $\WLPO$ is the gap between instance verification and
universal certification---between ``this representation is
modular'' (checkable) and ``every such representation is
modular'' (requires the trace formula).

This is a new refinement of the programme thesis: logical
cost can be intrinsic not to a theorem's content but to its
\emph{quantifier structure}.  The universal quantifier over
an unbounded domain, when the existential witness has no
algebraic dimension formula, forces passage through analytic
methods.

\subsection{For the Langlands programme}

The constructive analysis reveals a structural asymmetry in
the Langlands programme that is invisible to classical
mathematics:

All algebraic and $p$-adic components of the programme
(deformation theory, Galois cohomology, patching, \'etale
cohomology, modular Jacobians, Breuil modules, Hida families,
overconvergent forms, Kassaei--Pilloni classicality,
modularity lifting) are~$\BISH$.  The sole non-constructive
component is the \emph{absolute} construction of automorphic
forms from Galois representations via the Arthur--Selberg
trace formula at weight~$1$.

This dichotomy is invisible to classical mathematics because
classical logic does not distinguish between absolute and
relative constructions---both are ``proofs of existence.''
In the constructive setting, the distinction is sharp:
relative constructions propagate a witness; absolute
constructions create one.  The trace formula is the unique
absolute construction.  Everything else is relative.

The $p$-adic Langlands programme (Hida, Coleman--Mazur,
Buzzard--Taylor, Kassaei, Pilloni--Stroh, Emerton, Kisin,
Calegari--Geraghty) provides a vast $\BISH$ infrastructure
for \emph{propagating} modularity: from weight $\ge 2$ to
weight~$1$ (classicality), from residual to characteristic
zero (modularity lifting), between inner forms
(decidability descent).  The CRM audit reveals that this
infrastructure is constructively perfect.  The sole
non-constructive ingredient is the \emph{initial spark}: the
first modular form, created from a Galois representation by
the trace formula.

\subsection{For constructive mathematics}

The results of Papers~68--71 show that the algebraic and
$p$-adic infrastructure of modern number theory---however
abstract and technically sophisticated---is constructively
valid.  Deformation rings, Selmer groups, Hecke algebras,
Taylor--Wiles patching, Breuil's $p$-divisible groups,
Shimura varieties, \'etale cohomology, the Eichler--Shimura
relation, Hida families, the Coleman--Mazur eigencurve,
overconvergent modular forms, the Kassaei--Pilloni
classicality theorem, modularity lifting theorems---all~$\BISH$.

The entire Langlands programme for $\GL_2$ rests on a single
non-constructive foundation: the analytic trace formula at
weight~$1$, which creates modular forms from Galois
representations \emph{ex nihilo}.  This foundation costs
$\WLPO$---one atom of non-constructive content, located at
the spectral boundary $\lambda = 1/4$ where the Laplacian's
discrete and continuous spectra meet on a non-compact locally
symmetric space.

To eliminate this last $\WLPO$ would require an \emph{absolute}
bridge from Galois representations to automorphic forms that
avoids $L^2$ spectral theory entirely.  No such bridge is
currently known.  Its existence or non-existence is the
deepest open question at the intersection of constructive
mathematics and the Langlands programme.


% ============================================================
\section{Lean~4 Verification}\label{sec:lean}
% ============================================================

The classification extends Paper~70's Lean bundle.  New axioms
encode the Paper~71 results: the Shimura curve restructuring
(Route~3), the decidability descent for weight~$\ge 2$
Jacquet--Langlands, and the irreducibility of the weight~$1$
obstruction.  The main theorem and corollaries are verified by
\texttt{native\_decide} on the finite CRM hierarchy.

%% [BLANK: Lean verification summary after compilation.]

\begin{center}
\textbf{[TO BE FILLED: Lean verification summary.]}
\end{center}


% ============================================================
\section{Conclusion}\label{sec:conclusion}
% ============================================================

The constructive profile of Fermat's Last Theorem has been
optimised to its mathematical limit.  Two of the three $\WLPO$
atoms identified in Paper~70 have been eliminated: the
Jacquet--Langlands transfer (replaced by the Shimura curve
strategy) and the weight~$\ge 2$ Jacquet--Langlands comparison
(reduced to an algebraic identity by decidability descent).

The remaining atom---the Langlands--Tunnell theorem at
weight~$1$---is irreducible.  The obstruction is not
algebraic but analytic: it lies in the spectral theory of
the Laplacian on a non-compact space, at the exact point
($\lambda = 1/4$) where discrete and continuous spectra
meet.

The $\WLPO$ in FLT is the cost of crossing the weight~$1$
boundary---not with algebraic or $p$-adic tools (which are
$\BISH$) but with the analytic trace formula, the only known
absolute bridge from Galois representations to automorphic
forms.  The bridge is narrow (one atom of $\WLPO$), and the
river beneath it (the $\lambda = 1/4$ spectral boundary) is
real.  Everything on either side of the bridge is constructive.
The bridge itself---the act of creating a modular form from a
Galois representation \emph{ex nihilo}---is the irreducible
logical cost of the Langlands programme.


% ============================================================
\section*{Acknowledgments}
\addcontentsline{toc}{section}{Acknowledgments}
% ============================================================

The investigation was conducted via five atomic consultations
with AI agents (Anthropic Claude), following the methodology
of Papers~68--70.  The decidability descent argument,
the Shimura curve restructuring, and the identification of
the weight~$1$ boundary as the logical fault line emerged
from the dialogue.  The mathematical content is due to
Langlands, Tunnell, Jacquet, Arthur, Selberg, Taylor, Wiles,
Harish-Chandra, Deligne, Serre, Eichler, Shimura, Carayol,
Brochard, and Moret-Bailly.


% ============================================================
\begin{thebibliography}{30}
% ============================================================

\bibitem{Arthur2005}
J.~Arthur.
An introduction to the trace formula.
In \textit{Harmonic Analysis, the Trace Formula, and Shimura
Varieties} (Clay Math.\ Proceedings~4), pp.~1--263, 2005.

\bibitem{Brochard2017}
S.~Brochard.
Proof of de~Smit's conjecture: a freeness criterion.
\textit{Compositio Math.}, 153(11):2310--2317, 2017.

\bibitem{BuzzardTaylor1999}
K.~Buzzard and R.~Taylor.
Companion forms and weight one forms.
\textit{Ann.\ of Math.}, 149(3):905--919, 1999.

\bibitem{Carayol1986}
H.~Carayol.
Sur les repr\'esentations $\ell$-adiques associ\'ees aux
formes modulaires de Hilbert.
\textit{Ann.\ Sci.\ \'Ec.\ Norm.\ Sup.}, 19(3):409--468, 1986.

\bibitem{DeligneSerre1974}
P.~Deligne and J.-P.~Serre.
Formes modulaires de poids~1.
\textit{Ann.\ Sci.\ \'Ec.\ Norm.\ Sup.}, 7(4):507--530, 1974.

\bibitem{HarishChandra1966}
Harish-Chandra.
Discrete series for semi-simple Lie groups~II.
\textit{Acta Math.}, 116:1--111, 1966.

\bibitem{JacquetLanglands1970}
H.~Jacquet and R.\,P.~Langlands.
\textit{Automorphic Forms on $\GL(2)$}.
Lecture Notes in Mathematics~114.
Springer, 1970.

\bibitem{Kassaei2006}
P.~Kassaei.
A gluing lemma and overconvergent modular forms.
\textit{Duke Math.~J.}, 132(3):509--529, 2006.

\bibitem{Langlands1980}
R.\,P.~Langlands.
\textit{Base Change for $\GL(2)$}.
Annals of Mathematics Studies~96.
Princeton University Press, 1980.

\bibitem{Paper68}
P.\,C.\,K.~Lee.
The logical cost of Fermat's Last Theorem
(Paper~68, CRM series).
\textit{Zenodo}, 2026.

\bibitem{Paper69}
P.\,C.\,K.~Lee.
The modularity theorem is $\BISH + \WLPO$
(Paper~69, CRM series).
\textit{Zenodo}, 2026.

\bibitem{Paper70}
P.\,C.\,K.~Lee.
Serre's modularity conjecture is $\BISH + \WLPO$
(Paper~70, CRM series).
\textit{Zenodo}, 2026.

\bibitem{Pilloni2012}
V.~Pilloni.
Overconvergent modular forms and the Fontaine--Mazur
conjecture.
\textit{Invent.\ Math.}, 189(1):1--47, 2012.

\bibitem{Taylor2002}
R.~Taylor.
Remarks on a conjecture of Fontaine and Mazur.
\textit{J.~Inst.\ Math.\ Jussieu}, 1(1):125--143, 2002.

\bibitem{Tunnell1981}
J.~Tunnell.
Artin's conjecture for representations of octahedral type.
\textit{Bull.\ Amer.\ Math.\ Soc.}, 5(2):173--175, 1981.

\bibitem{Wiles1995}
A.~Wiles.
Modular elliptic curves and Fermat's Last Theorem.
\textit{Ann.\ of Math.}, 141(3):443--551, 1995.

\end{thebibliography}

\end{document}

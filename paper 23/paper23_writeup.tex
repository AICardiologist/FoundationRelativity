\documentclass[11pt,a4paper]{article}

% ====================================================================
% Packages
% ====================================================================
\usepackage[utf8]{inputenc}
\usepackage[T1]{fontenc}
\usepackage{amsmath,amssymb,amsthm}
\usepackage{mathtools}
\usepackage{hyperref}
\usepackage[margin=1in]{geometry}
\usepackage{enumitem}
\usepackage{booktabs}
\usepackage{listings}
\usepackage[table]{xcolor}
\usepackage{cleveref}
\usepackage{natbib}
\usepackage{mdframed}

% ====================================================================
% Theorem environments
% ====================================================================
\theoremstyle{plain}
\newtheorem{theorem}{Theorem}[section]
\newtheorem{lemma}[theorem]{Lemma}
\newtheorem{proposition}[theorem]{Proposition}
\newtheorem{corollary}[theorem]{Corollary}

\theoremstyle{definition}
\newtheorem{definition}[theorem]{Definition}
\newtheorem{remark}[theorem]{Remark}

% ====================================================================
% Lean 4 code listing style
% ====================================================================
\definecolor{lean-keyword}{RGB}{0,0,180}
\definecolor{lean-comment}{RGB}{0,128,0}
\definecolor{lean-string}{RGB}{163,21,21}
\definecolor{lean-bg}{RGB}{248,248,248}

\lstdefinelanguage{lean4}{
  keywords={theorem,lemma,def,class,instance,import,open,variable,
            noncomputable,section,namespace,end,where,let,have,show,
            intro,obtain,use,exact,rw,simp,apply,by,fun,match,if,
            then,else,do,return,axiom,abbrev,private,attribute,
            suffices,change,congr,ext,constructor,rintro,push_neg,
            linarith,absurd,set_option,omit,in,set,cases,structure,
            refine,unfold,rcases,calc,all_goals,first,try,ring,
            positivity,induction},
  sensitive=true,
  morecomment=[l]{--},
  morecomment=[s]{/-}{-/},
  morestring=[b]",
  morestring=[b]',
}

\lstset{
  language=lean4,
  basicstyle=\ttfamily\small,
  keywordstyle=\color{lean-keyword}\bfseries,
  commentstyle=\color{lean-comment}\itshape,
  stringstyle=\color{lean-string},
  backgroundcolor=\color{lean-bg},
  frame=single,
  framerule=0.5pt,
  breaklines=true,
  breakatwhitespace=true,
  tabsize=2,
  showstringspaces=false,
  numbers=left,
  numberstyle=\tiny\color{gray},
  numbersep=5pt,
  xleftmargin=15pt,
  captionpos=b,
}

% ====================================================================
% Macros
% ====================================================================
\newcommand{\NN}{\mathbb{N}}
\newcommand{\QQ}{\mathbb{Q}}
\newcommand{\RR}{\mathbb{R}}
\newcommand{\ZZ}{\mathbb{Z}}
\newcommand{\CC}{\mathbb{C}}
\newcommand{\LPO}{\mathrm{LPO}}
\newcommand{\WLPO}{\mathrm{WLPO}}
\newcommand{\LLPO}{\mathrm{LLPO}}
\newcommand{\BISH}{\mathrm{BISH}}
\newcommand{\MP}{\mathrm{MP}}
\newcommand{\FT}{\textsf{FT}}
\newcommand{\EVT}{\textsf{EVT}}
\newcommand{\Lean}{\textsc{Lean~4}}
\newcommand{\Mathlib}{\textsc{Mathlib4}}
\newcommand{\leanok}{\textsf{\small \textcolor{green!70!black}{\checkmark}}}

% ====================================================================
% Title
% ====================================================================
\title{%
  \textbf{The Fan Theorem and the Constructive Cost\\[4pt]
  of Optimization: Free Energy Extrema\\[4pt]
  on Compact Parameter Spaces}\\[6pt]
  {\normalsize Paper~23 in the Constructive Reverse Mathematics Series}%
}

\author{
  Paul Chun-Kit Lee\thanks{%
    New York University.
    AI-assisted formalization; see \S\ref{sec:ai} for methodology.
    The author is a medical professional, not a domain expert in
    constructive mathematics or mathematical physics; mathematical
    content was developed with extensive AI assistance.} \\
  Center for Mathematical Physics \\
  New York University \\
  \texttt{dr.paul.c.lee@gmail.com}
}

\date{February 2026}

% ====================================================================
\begin{document}
\maketitle

% ====================================================================
\begin{abstract}
The assertion that a continuous function on a compact interval
$[a,b]$ attains its minimum---the Extreme Value Theorem
(\EVT)---is equivalent to the Fan Theorem (\FT) over Bishop's
constructive mathematics ($\BISH$). We instantiate this
equivalence through free energy optimization in the 1D Ising
model: the claim that $f(\beta, J) = -\log(2 \cdot \cosh(\beta J))$
attains its minimum over a compact coupling interval requires
exactly the Fan Theorem. This is the \textbf{first CRM calibration
at the \FT{} level}, adding a third independent branch to the
calibration table. The Fan Theorem is independent of the
omniscience hierarchy ($\LPO$, $\WLPO$, $\LLPO$) and of Markov's
Principle ($\MP$): it is a compactness/continuity principle, not a
decidability or witness-production principle. The 1D Ising model
now exhibits \textbf{four} distinct logical costs: finite-volume
computation ($\BISH$, Paper~8), thermodynamic limit existence
($\LPO$, Paper~8), phase classification ($\WLPO$, Paper~20), and
parameter-space optimization (\FT, this paper). All results are
formalized in \Lean{} with \Mathlib{} (14~files, ${\sim}680$~lines,
zero \texttt{sorry}, \textbf{zero custom axioms}).
\end{abstract}

\vspace{1em}
\tableofcontents

% ====================================================================
\section{Introduction}\label{sec:intro}
% ====================================================================

\subsection{A Third Independent Branch}
\label{sec:third-branch}

The constructive reverse mathematics (CRM) programme calibrates
mathematical physics against the hierarchy of omniscience
principles~\citep{Bishop67,BR87,BV06,Ishihara06,Diener20}.
Papers~2 through~21 populated the linear chain
$\BISH < \LLPO < \WLPO < \LPO$~\cite{Lee25-P2,Lee25-P3,Lee25-P5,
Lee25-P8,Lee26-P20,Lee26-P21}. Paper~22 established the first
branch point: Markov's Principle ($\MP$), calibrated against
radioactive decay, is implied by $\LPO$ but independent of both
$\WLPO$ and $\LLPO$~\cite{Lee26-P22}.

This paper adds a \textbf{second branch}: the Fan Theorem (\FT),
calibrated against compact optimization. The Fan Theorem is a
compactness/continuity principle, fundamentally different from
both the omniscience principles (which concern decidability of
infinite tests) and Markov's Principle (which concerns witness
production from double negation). The \FT{} is:
\begin{itemize}
  \item Independent of $\LPO$, $\WLPO$, $\LLPO$: Brouwerian
    models satisfy \FT{} but not $\LPO$; Russian recursive
    models satisfy $\LPO$ but not \FT~\citep{Berger05,BV06,BR87}.
  \item Independent of $\MP$: the same model-theoretic arguments
    establish independence~\citep{Berger05}.
  \item A Brouwerian principle: accepted in intuitionistic
    mathematics, independent of $\BISH$~\citep{Brouwer27,TvD88}.
\end{itemize}

The calibration hierarchy is now a genuine partial order with
three independent branches emanating from (or independent of) the
main omniscience chain.

\subsection{Four Logical Costs of the 1D Ising Model}
\label{sec:four-costs}

The 1D Ising model, which first appeared in the series in
Paper~8~\cite{Lee25-P8}, now exhibits four distinct logical
costs depending on the question asked:

\begin{center}
\begin{tabular}{@{}clll@{}}
\toprule
\textbf{Paper} & \textbf{Assertion} & \textbf{CRM Cost} &
  \textbf{Type} \\
\midrule
8 (Part A)  & Finite-volume computation
  & $\BISH$ & Arithmetic \\
8 (Part B)  & Thermodynamic limit existence
  & $\LPO$ & Omniscience \\
20          & Phase classification
  & $\WLPO$ & Omniscience \\
\textbf{23} & \textbf{Parameter-space optimization}
  & \textbf{\FT} & \textbf{Compactness} \\
\bottomrule
\end{tabular}
\end{center}

\noindent
The same physical system---the 1D Ising model with free energy
$f(\beta, J) = -\log(2 \cdot \cosh(\beta J))$---requires four
different constructive principles depending on which property of
the free energy one wishes to assert. The logical cost depends on
the \emph{observable}, not the underlying model.

\subsection{The Cleanest Axiom Audit}\label{sec:clean-audit}

A key design decision in this paper is to define the Fan Theorem
directly as the Extreme Value Theorem (max form) on $[0,1]$:
\[
  \FT \;:=\; \EVT_{\max}.
\]
The equivalence between this definition and the bar-theoretic
Fan Theorem (``every bar on Cantor space is uniform'') is
established by \citet{Berger05} and \citet{BV06}. By defining
\FT{} via its \EVT{} equivalent rather than axiomatizing the
connection, the formalization achieves \textbf{zero custom axioms}.
Every theorem in the project depends only on the standard \Mathlib{}
infrastructure axioms (\texttt{propext}, \texttt{Classical.choice},
\texttt{Quot.sound}). This is the cleanest axiom audit of any
paper in the CRM series.

\subsection{Main Results}\label{sec:main-results}

The paper has three parts:

\begin{enumerate}
  \item \textbf{Part~A ($\BISH$):} Finite-grid optimization and
    continuity of the Ising free energy are fully constructive.
    No Fan Theorem needed.

  \item \textbf{Part~B (\FT{} calibration):} The Fan Theorem is
    equivalent to compact optimization:
    $\FT \leftrightarrow \mathrm{CompactOptimization}$.
    The proof proceeds through the intermediate equivalences
    $\EVT_{\max} \leftrightarrow \EVT_{\min} \leftrightarrow
    \mathrm{CompactOptimization}$.

  \item \textbf{Stratification:} The constructive hierarchy is a
    three-branch partial order. The omniscience chain
    ($\LPO \Rightarrow \WLPO \Rightarrow \LLPO$), Markov's branch
    ($\LPO \Rightarrow \MP$), and the compactness branch
    ($\FT \leftrightarrow \mathrm{CompactOptimization}$, independent
    of all of the above) are all represented.
\end{enumerate}

\noindent
The main theorems, stated precisely:

\begin{itemize}
  \item \textbf{Theorem~1} (Part~A): The Ising free energy is
    continuous in the coupling $J$.
  \item \textbf{Theorem~2} (Part~A): Finite-grid optimization is
    $\BISH$---a minimizer exists for any nonempty finite set.
  \item \textbf{Theorem~3} (Part~B): $\EVT_{\max} \leftrightarrow
    \EVT_{\min}$---apply to $-f$.
  \item \textbf{Theorem~4} (Part~B): $\EVT_{\min} \Rightarrow
    \mathrm{CompactOptimization}$---the rescaling argument.
  \item \textbf{Theorem~5} (Part~B): $\mathrm{CompactOptimization}
    \Rightarrow \EVT_{\min}$---specialize to $[0,1]$.
  \item \textbf{Theorem~6} (Part~B): $\FT \leftrightarrow
    \mathrm{CompactOptimization}$---the main equivalence.
  \item \textbf{Theorem~7}: $\FT \Rightarrow$ Ising free energy
    optimization on any compact coupling interval.
  \item \textbf{Theorem~8}: Three-branch stratification of the
    constructive hierarchy.
\end{itemize}


% ====================================================================
\section{Background}\label{sec:background}
% ====================================================================

\subsection{The Fan Theorem}\label{sec:fan-bg}

The Fan Theorem is a compactness principle introduced by
\citet{Brouwer27} in the context of intuitionistic mathematics.
In its bar-theoretic form, it asserts that every bar on Cantor
space $2^{\NN}$ is uniform: if every infinite binary sequence
passes through a finite initial segment in a decidable bar $B$,
then there is a uniform bound $N$ such that all sequences hit
$B$ by stage $N$.

The Fan Theorem has several equivalent formulations that are more
directly applicable to analysis~\citep{Berger05,BV06,JR02}:

\begin{enumerate}
  \item \textbf{Uniform continuity on compact spaces:} Every
    pointwise continuous function on a compact complete totally
    bounded metric space is uniformly continuous.

  \item \textbf{Extreme Value Theorem (EVT):} Every continuous
    function $f : [a,b] \to \RR$ attains its maximum and minimum.

  \item \textbf{Bar induction:} Every bar on Cantor space is
    uniform (the original formulation).
\end{enumerate}

\noindent
The equivalence $\FT \leftrightarrow \EVT$ was established by
\citet{BB06} building on \citet{JR02}. We adopt the \EVT{}
formulation as our working definition, following \citet{Berger05}.

\begin{definition}[$\EVT_{\max}$]\label{def:evt-max}
\leanok{}
The \emph{Extreme Value Theorem (max form)}: every continuous
function on $[0,1]$ attains its maximum.
\begin{equation}\label{eq:evt-max}
  \forall f : \RR \to \RR,\;\;
  \mathrm{ContinuousOn}\;f\;[0,1]
  \;\Longrightarrow\;
  \exists x \in [0,1],\;
  \forall y \in [0,1],\; f(y) \le f(x).
\end{equation}
\end{definition}

\begin{definition}[$\EVT_{\min}$]\label{def:evt-min}
\leanok{}
The \emph{Extreme Value Theorem (min form)}: every continuous
function on $[0,1]$ attains its minimum.
\begin{equation}\label{eq:evt-min}
  \forall f : \RR \to \RR,\;\;
  \mathrm{ContinuousOn}\;f\;[0,1]
  \;\Longrightarrow\;
  \exists x \in [0,1],\;
  \forall y \in [0,1],\; f(x) \le f(y).
\end{equation}
\end{definition}

\begin{definition}[CompactOptimization]\label{def:compact-opt}
\leanok{}
Every continuous function on $[a,b]$ attains its minimum:
\begin{equation}\label{eq:compact-opt}
  \forall a < b,\;\;
  \forall f : \RR \to \RR,\;\;
  \mathrm{ContinuousOn}\;f\;[a,b]
  \;\Longrightarrow\;
  \exists x \in [a,b],\;
  \forall y \in [a,b],\; f(x) \le f(y).
\end{equation}
\end{definition}

\begin{definition}[FanTheorem]\label{def:ft}
\leanok{}
The \emph{Fan Theorem}, defined via its analytic equivalent:
\begin{equation}\label{eq:ft-def}
  \FT \;:=\; \EVT_{\max}.
\end{equation}
The equivalence with the bar-theoretic Fan Theorem is by
citation~\citep{Berger05,BV06}.
\end{definition}

\begin{lstlisting}[caption={Fan Theorem and EVT definitions (Defs/EVT.lean).}]
/-- The Extreme Value Theorem (max form) on [0,1]. -/
def EVT_max : Prop :=
  forall (f : Real -> Real),
    ContinuousOn f (Set.Icc 0 1) ->
    exists x, x mem Set.Icc (0 : Real) (1 : Real) /\
      forall y, y mem Set.Icc (0 : Real) (1 : Real) ->
        f y <= f x

/-- The Extreme Value Theorem (min form) on [0,1]. -/
def EVT_min : Prop :=
  forall (f : Real -> Real),
    ContinuousOn f (Set.Icc 0 1) ->
    exists x, x mem Set.Icc (0 : Real) (1 : Real) /\
      forall y, y mem Set.Icc (0 : Real) (1 : Real) ->
        f x <= f y

/-- Compact optimization on [a,b]. -/
def CompactOptimization : Prop :=
  forall (a b : Real), a < b ->
    forall (f : Real -> Real),
      ContinuousOn f (Set.Icc a b) ->
      exists x, x mem Set.Icc a b /\
        forall y, y mem Set.Icc a b -> f x <= f y

/-- The Fan Theorem := EVT_max (Berger 2005). -/
def FanTheorem : Prop := EVT_max
\end{lstlisting}

\subsection{Status in Constructive Mathematics}
\label{sec:ft-status}

The Fan Theorem occupies a distinctive position in constructive
mathematics:

\begin{itemize}
  \item In \textbf{Bishop's $\BISH$}: The Fan Theorem is
    independent---neither provable nor refutable. Bishop's
    framework deliberately avoids both the Fan Theorem and its
    negation, maintaining compatibility with classical, intuitionistic,
    and recursive interpretations~\citep{Bishop67,BB85}.

  \item In \textbf{Brouwerian intuitionism}: The Fan Theorem is
    accepted as a basic principle, following Brouwer's original
    arguments~\citep{Brouwer27,TvD88}.

  \item In \textbf{Russian recursive mathematics}: The Fan Theorem
    fails. In Markov's constructive recursive framework, there
    exist computable functions on $[0,1]$ that are pointwise
    continuous but not uniformly continuous, and hence do not attain
    their extrema~\citep{Kushner85,BR87}.

  \item In \textbf{classical mathematics}: The Fan Theorem holds
    trivially, as a consequence of the Heine--Borel theorem.
\end{itemize}

\subsection{Fan Theorem vs.\ Omniscience}\label{sec:ft-vs-omni}

The Fan Theorem is fundamentally different from the omniscience
principles ($\LPO$, $\WLPO$, $\LLPO$) and from Markov's Principle
($\MP$):

\begin{itemize}
  \item \textbf{Omniscience principles} concern the decidability
    of infinite quantifiers: can we decide whether a binary sequence
    is identically zero? The answer stratifies into
    $\LPO > \WLPO > \LLPO$.

  \item \textbf{Markov's Principle} concerns witness production
    from double negation: given $\neg(\forall n,\; \alpha(n) = 0)$,
    does there exist $n$ with $\alpha(n) = 1$?

  \item \textbf{The Fan Theorem} concerns compactness: does a
    continuous function on a compact set attain its extrema?
    This is a topological question, not a decidability or
    witness-production question.
\end{itemize}

\noindent
The independence is established by standard model-theoretic
arguments~\citep{Berger05,BV06,BR87}:
\begin{itemize}
  \item Brouwerian models satisfy \FT{} but not $\LPO$ (hence
    $\FT \not\Rightarrow \LPO$).
  \item Russian recursive models satisfy $\LPO$ (and hence $\WLPO$,
    $\LLPO$, and $\MP$) but not \FT{} (hence $\LPO \not\Rightarrow
    \FT$).
  \item Independence from $\MP$ follows similarly~\citep{Berger05}.
\end{itemize}

\subsection{The CRM Hierarchy (Updated)}
\label{sec:hierarchy-bg}

The constructive reverse mathematics hierarchy, now with three
branches:

\begin{equation}\label{eq:hierarchy}
\begin{array}{ccccc}
  & & \LPO & & \\
  & \swarrow & \downarrow & \searrow & \\
  \WLPO & & \MP & & \cdots \\
  \downarrow & & & & \\
  \LLPO & & & & \\
  \downarrow & & & & \\
  \BISH & & & & \FT \;\text{(independent of all above)}
\end{array}
\end{equation}

\noindent
The omniscience chain $\LPO \Rightarrow \WLPO \Rightarrow \LLPO$
forms the main linear chain~\citep{Bishop67,BR87}. Markov's
Principle branches off: $\LPO \Rightarrow \MP$, but $\MP$ is
independent of $\WLPO$ and $\LLPO$~\citep{BR87,BV06}
(Paper~22~\cite{Lee26-P22}). The Fan Theorem is independent of
the entire hierarchy above~\citep{Berger05,BV06}.

\subsection{The 1D Ising Model}\label{sec:ising-bg}

The 1D Ising free energy per site with inverse temperature $\beta$
and coupling constant $J$ is:
\begin{equation}\label{eq:ising}
  f(\beta, J) = -\log\bigl(2 \cdot \cosh(\beta J)\bigr).
\end{equation}
This function is continuous in both $\beta$ and $J$, and has
appeared in Papers~8 and~20 of the series. For the purpose of this
paper, we fix $\beta$ and view $f$ as a function of $J$ alone,
asking: over a compact coupling interval $[a,b]$, does $f$ attain
its minimum?


% ====================================================================
\section{Part~A: Finite Optimization Is BISH}
\label{sec:part-a}
% ====================================================================

The first tier of the calibration: when the optimization is over a
finite set, no compactness principle is needed.

\subsection{Continuity of the Free Energy}\label{sec:continuity}

\begin{theorem}[Continuity---BISH]\label{thm:continuity}
\leanok{}
For any fixed $\beta \in \RR$, the Ising free energy
$J \mapsto f(\beta, J) = -\log(2 \cdot \cosh(\beta J))$
is continuous.
\end{theorem}

\begin{proof}
By composition: $J \mapsto \beta J$ is continuous (multiplication
by a constant), $\cosh$ is continuous, $x \mapsto 2x$ is
continuous, $\log$ is continuous on $(0, \infty)$, and negation
is continuous. Since $2 \cdot \cosh(\beta J) > 0$ for all $J$
(because $\cosh(x) > 0$ for all $x$), the composition is
well-defined and continuous.
\end{proof}

\begin{lstlisting}[caption={Continuity of Ising free energy (PartA/Continuity.lean).}]
/-- The Ising free energy is continuous in J. -/
theorem isingFreeEnergy_continuous (beta : Real) :
    Continuous (isingFreeEnergy beta) := by
  unfold isingFreeEnergy
  apply Continuous.neg
  apply Continuous.log
  . exact continuous_const.mul
      (Real.continuous_cosh.comp
        (continuous_const.mul continuous_id))
  . intro J
    exact two_cosh_ne_zero (beta * J)

/-- ContinuousOn version for [a, b]. -/
theorem isingFreeEnergy_continuousOn (beta : Real)
    (a b : Real) :
    ContinuousOn (isingFreeEnergy beta)
      (Set.Icc a b) :=
  (isingFreeEnergy_continuous beta).continuousOn
\end{lstlisting}

\subsection{Finite-Grid Optimization}\label{sec:finite-opt}

\begin{theorem}[Finite optimization---BISH]\label{thm:finite-opt}
\leanok{}
For any fixed $\beta$ and any nonempty finite set
$\mathrm{grid} \subseteq \RR$, there exists $J^* \in \mathrm{grid}$
minimizing $f(\beta, \cdot)$ over the grid:
\begin{equation}\label{eq:finite-opt}
  \exists J^* \in \mathrm{grid},\;\;
  \forall J \in \mathrm{grid},\;\;
  f(\beta, J^*) \le f(\beta, J).
\end{equation}
\end{theorem}

\begin{proof}
This is a finite search over a nonempty finite set, which is
constructively unproblematic. In \Lean{}, this is a direct
application of \texttt{Finset.exists\_min\_image}.
\end{proof}

\begin{lstlisting}[caption={Finite-grid optimization (PartA/FiniteOpt.lean).}]
/-- Finite-grid optimization is BISH. -/
theorem finite_opt_bish (beta : Real)
    (grid : Finset Real) (hne : grid.Nonempty) :
    exists J_star, J_star mem grid /\ forall J,
      J mem grid ->
      isingFreeEnergy beta J_star
        <= isingFreeEnergy beta J :=
  Finset.exists_min_image grid
    (isingFreeEnergy beta) hne

/-- Finite-grid maximization is also BISH. -/
theorem finite_opt_max_bish (beta : Real)
    (grid : Finset Real) (hne : grid.Nonempty) :
    exists J_star, J_star mem grid /\ forall J,
      J mem grid ->
      isingFreeEnergy beta J
        <= isingFreeEnergy beta J_star :=
  Finset.exists_max_image grid
    (isingFreeEnergy beta) hne
\end{lstlisting}

\subsection{Part~A Summary}\label{sec:part-a-summary}

\begin{theorem}[Part~A Summary---BISH]\label{thm:part-a-summary}
\leanok{}
Both statements hold without any compactness principle:
\begin{enumerate}
  \item The Ising free energy is continuous in $J$ for any $\beta$.
  \item For any nonempty finite grid, a minimizer exists.
\end{enumerate}
\end{theorem}

\begin{lstlisting}[caption={Part A summary (PartA/PartA\_Main.lean).}]
/-- Part A summary: finite optimization and
    continuity are BISH. -/
theorem partA_summary :
    (forall beta : Real,
      Continuous (isingFreeEnergy beta)) /\
    (forall beta : Real,
      forall grid : Finset Real,
        grid.Nonempty ->
        exists J_star, J_star mem grid /\
          forall J, J mem grid ->
            isingFreeEnergy beta J_star
              <= isingFreeEnergy beta J) :=
  <isingFreeEnergy_continuous,
   fun beta grid hne =>
     finite_opt_bish beta grid hne>
\end{lstlisting}

\begin{remark}[Axiom profile for Part~A]\label{rem:bish-profile}
\texttt{\#print axioms isingFreeEnergy\_continuous},
\texttt{\#print axioms finite\_opt\_bish}, and
\texttt{\#print axioms partA\_summary} all show only
\texttt{[propext, Classical.choice, Quot.sound]}. No custom axioms.
The \texttt{Classical.choice} arises from \Mathlib{}'s
infrastructure for \texttt{Real.log}, \texttt{Real.cosh}, and
\texttt{Finset.exists\_min\_image}, not from any mathematical
use of choice. These are pure $\BISH$ results.
\end{remark}


% ====================================================================
\section{Part~B: The Fan Theorem Calibration}
\label{sec:part-b}
% ====================================================================

This section establishes the main equivalence:
$\FT \leftrightarrow \mathrm{CompactOptimization}$.

\subsection{EVT Max and Min Are Equivalent}
\label{sec:evt-equiv}

\begin{theorem}[$\EVT_{\max} \leftrightarrow \EVT_{\min}$]
\label{thm:evt-equiv}
\leanok{}
The max and min forms of the Extreme Value Theorem are equivalent:
\begin{equation}\label{eq:evt-equiv}
  \EVT_{\max} \;\longleftrightarrow\; \EVT_{\min}.
\end{equation}
\end{theorem}

\begin{proof}
\textbf{$\EVT_{\max} \Rightarrow \EVT_{\min}$:} Given a continuous
function $f$ on $[0,1]$, apply $\EVT_{\max}$ to $-f$. Since $-f$
is continuous (continuity is preserved under negation), there
exists $x^* \in [0,1]$ with $-f(y) \le -f(x^*)$ for all
$y \in [0,1]$. Multiplying by $-1$:
$f(x^*) \le f(y)$ for all $y \in [0,1]$.

\smallskip\noindent
\textbf{$\EVT_{\min} \Rightarrow \EVT_{\max}$:} Symmetric.
Apply $\EVT_{\min}$ to $-f$ and negate.
\end{proof}

\begin{lstlisting}[caption={EVT equivalence (PartB/EVTEquiv.lean).}]
/-- EVT_max implies EVT_min: apply to -f. -/
theorem evt_min_of_evt_max (h : EVT_max) :
    EVT_min := by
  intro f hf
  obtain <x, hx_mem, hx_max> :=
    h (fun t => -f t) (hf.neg)
  exact <x, hx_mem,
    fun y hy => by linarith [hx_max y hy]>

/-- EVT_min implies EVT_max: apply to -f. -/
theorem evt_max_of_evt_min (h : EVT_min) :
    EVT_max := by
  intro f hf
  obtain <x, hx_mem, hx_min> :=
    h (fun t => -f t) (hf.neg)
  exact <x, hx_mem,
    fun y hy => by linarith [hx_min y hy]>

/-- EVT_max and EVT_min are equivalent. -/
theorem evt_max_iff_evt_min :
    EVT_max <-> EVT_min :=
  <evt_min_of_evt_max, evt_max_of_evt_min>
\end{lstlisting}

\subsection{The Rescaling Infrastructure}
\label{sec:rescaling}

To connect EVT on $[0,1]$ with optimization on general $[a,b]$,
we use affine rescaling.

\begin{definition}[Rescaling]\label{def:rescale}
\leanok{}
The affine map sending $[0,1]$ to $[a,b]$:
\begin{equation}\label{eq:rescale}
  \mathrm{rescale}(a, b, t) := a + t \cdot (b - a).
\end{equation}
Its inverse, sending $[a,b]$ back to $[0,1]$:
\begin{equation}\label{eq:unscale}
  \mathrm{unscale}(a, b, x) := \frac{x - a}{b - a}.
\end{equation}
\end{definition}

\begin{lemma}[Rescaling properties]\label{lem:rescale-props}
\leanok{}
For $a \le b$:
\begin{enumerate}
  \item $\mathrm{rescale}(a, b, 0) = a$ and
    $\mathrm{rescale}(a, b, 1) = b$.
  \item $\mathrm{rescale}$ maps $[0,1]$ into $[a,b]$.
  \item $\mathrm{rescale}$ is continuous.
  \item For $a < b$: $\mathrm{unscale}$ maps $[a,b]$ into $[0,1]$.
  \item $\mathrm{rescale} \circ \mathrm{unscale} = \mathrm{id}$
    on $[a,b]$ (when $a < b$).
  \item $\mathrm{unscale} \circ \mathrm{rescale} = \mathrm{id}$
    on $[0,1]$ (when $a < b$).
\end{enumerate}
\end{lemma}

\begin{lstlisting}[caption={Rescaling infrastructure (Defs/Rescaling.lean, selected).}]
/-- Affine map [0,1] -> [a,b]. -/
def rescale (a b : Real) (t : Real) : Real :=
  a + t * (b - a)

/-- Inverse: [a,b] -> [0,1]. -/
def unscale (a b : Real) (x : Real) : Real :=
  (x - a) / (b - a)

/-- rescale maps [0,1] into [a,b]. -/
theorem rescale_maps_Icc (a b : Real) (hab : a <= b)
    (t : Real) (ht : t mem Set.Icc (0 : Real) 1) :
    rescale a b t mem Set.Icc a b := by
  constructor
  . unfold rescale; nlinarith [ht.1, ht.2]
  . unfold rescale; nlinarith [ht.1, ht.2]

/-- rescale . unscale = id on [a,b]. -/
theorem rescale_unscale (a b : Real) (hab : a < b)
    (x : Real) (_hx : x mem Set.Icc a b) :
    rescale a b (unscale a b x) = x := by
  unfold rescale unscale
  have hba : (b - a) != 0 :=
    ne_of_gt (sub_pos.mpr hab)
  field_simp; ring
\end{lstlisting}

\subsection{EVT Min Implies Compact Optimization}
\label{sec:evt-to-compact}

\begin{theorem}[$\EVT_{\min} \Rightarrow \mathrm{CompactOptimization}$]
\label{thm:evt-to-compact}
\leanok{}
If every continuous function on $[0,1]$ attains its minimum,
then every continuous function on any $[a,b]$ attains its minimum.
\end{theorem}

\begin{proof}
Given $a < b$ and $f : \RR \to \RR$ continuous on $[a,b]$:

\smallskip\noindent
\textbf{Step~1: Define the rescaled function.}
Let $g(t) = f(\mathrm{rescale}(a,b,t)) = f(a + t(b-a))$.
Then $g$ is the composition of $f$ (continuous on $[a,b]$) with
$\mathrm{rescale}$ (continuous, mapping $[0,1]$ into $[a,b]$).
Hence $g$ is continuous on $[0,1]$.

\smallskip\noindent
\textbf{Step~2: Apply $\EVT_{\min}$ to $g$.}
Obtain $t^* \in [0,1]$ with $g(t^*) \le g(t)$ for all
$t \in [0,1]$.

\smallskip\noindent
\textbf{Step~3: Return the minimizer.}
Set $x^* = \mathrm{rescale}(a, b, t^*) \in [a,b]$.
For any $y \in [a,b]$, let $s = \mathrm{unscale}(a, b, y) \in [0,1]$.
Then:
\begin{align*}
  f(x^*) &= f(\mathrm{rescale}(a,b,t^*))
    = g(t^*) \\
  &\le g(s)
    = f(\mathrm{rescale}(a,b,s)) \\
  &= f(\mathrm{rescale}(a,b,\mathrm{unscale}(a,b,y)))
    = f(y).
\end{align*}
The last step uses $\mathrm{rescale} \circ \mathrm{unscale} = \mathrm{id}$ on $[a,b]$.
\end{proof}

\begin{lstlisting}[caption={EVT$_{\min}$ implies CompactOptimization (PartB/CompactOpt.lean).}]
/-- EVT on [0,1] implies compact optimization
    on arbitrary [a,b]. -/
theorem compact_opt_of_evt_min (h : EVT_min) :
    CompactOptimization := by
  intro a b hab f hf
  let g : Real -> Real :=
    fun t => f (rescale a b t)
  have hg_cont : ContinuousOn g (Set.Icc 0 1) := by
    apply ContinuousOn.comp hf
      (rescale_continuous a b).continuousOn
    exact rescale_mapsTo a b (le_of_lt hab)
  obtain <t_star, ht_mem, ht_min> := h g hg_cont
  refine <rescale a b t_star,
    rescale_maps_Icc a b (le_of_lt hab)
      t_star ht_mem, ?>
  intro y hy
  have hs_mem : unscale a b y mem
      Set.Icc (0 : Real) 1 :=
    unscale_maps_Icc a b hab y hy
  have h1 : f (rescale a b t_star) <=
      f (rescale a b (unscale a b y)) :=
    ht_min (unscale a b y) hs_mem
  rw [rescale_unscale a b hab y hy] at h1
  exact h1
\end{lstlisting}

\subsection{Compact Optimization Implies EVT Min}
\label{sec:compact-to-evt}

\begin{theorem}[$\mathrm{CompactOptimization} \Rightarrow \EVT_{\min}$]
\label{thm:compact-to-evt}
\leanok{}
If every continuous function on any $[a,b]$ attains its minimum,
then every continuous function on $[0,1]$ attains its minimum.
\end{theorem}

\begin{proof}
Specialize $\mathrm{CompactOptimization}$ to $a = 0$, $b = 1$
(noting $0 < 1$).
\end{proof}

\begin{lstlisting}[caption={CompactOptimization implies EVT$_{\min}$ (PartB/CompactOpt.lean).}]
/-- Compact optimization implies EVT on [0,1]. -/
theorem evt_min_of_compact_opt
    (h : CompactOptimization) : EVT_min := by
  intro f hf
  exact h 0 1 (by norm_num) f hf
\end{lstlisting}

\subsection{The Main Equivalence}\label{sec:main-equiv}

\begin{theorem}[$\FT \leftrightarrow \mathrm{CompactOptimization}$]
\label{thm:main-equiv}
\leanok{}
Over $\BISH$, the Fan Theorem is equivalent to compact optimization:
\begin{equation}\label{eq:main-equiv}
  \FT \;\;\longleftrightarrow\;\;
  \mathrm{CompactOptimization}.
\end{equation}
\end{theorem}

\begin{proof}
Compose the three equivalences. Since $\FT = \EVT_{\max}$:
\[
  \FT
  \;=\; \EVT_{\max}
  \;\xrightarrow{\text{Thm~\ref{thm:evt-equiv}}}\;
  \EVT_{\min}
  \;\xrightarrow{\text{Thm~\ref{thm:evt-to-compact}}}\;
  \mathrm{CompactOptimization}
  \;\xrightarrow{\text{Thm~\ref{thm:compact-to-evt}}}\;
  \EVT_{\min}
  \;\xrightarrow{\text{Thm~\ref{thm:evt-equiv}}}\;
  \EVT_{\max}
  \;=\; \FT.
\]
\end{proof}

\begin{lstlisting}[caption={Main equivalence (PartB/PartB\_Main.lean).}]
/-- The Fan Theorem <-> CompactOptimization. -/
theorem ft_iff_compact_opt :
    FanTheorem <-> CompactOptimization := by
  unfold FanTheorem
  constructor
  . intro hmax
    exact compact_opt_of_evt_min
      (evt_min_of_evt_max hmax)
  . intro hco
    exact evt_max_of_evt_min
      (evt_min_of_compact_opt hco)
\end{lstlisting}

\begin{remark}[Zero custom axioms]\label{rem:zero-axioms}
\texttt{\#print axioms ft\_iff\_compact\_opt} shows only
\texttt{[propext, Classical.choice, Quot.sound]}. In fact, many of
the intermediate results (\texttt{evt\_min\_of\_evt\_max},
\texttt{compact\_opt\_of\_evt\_min}, \texttt{evt\_min\_of\_compact\_opt})
require only \texttt{[propext]}. No custom axiom appears anywhere.
This is possible because $\FT$ is defined as $\EVT_{\max}$, not
axiomatized separately. The connection to the bar-theoretic Fan
Theorem is by citation~\citep{Berger05,BV06}, not by axiom.
\end{remark}


% ====================================================================
\section{The Stratification Theorem}\label{sec:stratification}
% ====================================================================

\subsection{Three-Branch Partial Order}\label{sec:three-branch}

The constructive hierarchy now has three independent branches:

\begin{center}
\begin{tabular}{@{}clll@{}}
\toprule
\textbf{Branch} & \textbf{Content} & \textbf{Example} &
  \textbf{Papers} \\
\midrule
Omniscience & $\LPO \Rightarrow \WLPO \Rightarrow \LLPO$
  & Bidual gap, Ising limit
  & 2--21 \\
Markov & $\LPO \Rightarrow \MP$
  & Eventual decay
  & 22 \\
Compactness & $\FT \leftrightarrow \mathrm{CompactOpt}$
  & Free energy optimization
  & 23 \\
\bottomrule
\end{tabular}
\end{center}

\begin{theorem}[Three-branch stratification]\label{thm:stratification}
\leanok{}
The constructive hierarchy is a partial order with three branches:
\begin{enumerate}
  \item \textbf{Omniscience chain:} $\LPO \Rightarrow \WLPO
    \Rightarrow \LLPO$.
  \item \textbf{Markov branch:} $\LPO \Rightarrow \MP$.
  \item \textbf{Compactness branch:} $\FT \leftrightarrow
    \mathrm{CompactOptimization}$ (independent of all above).
\end{enumerate}
\end{theorem}

\begin{proof}
Items~(1) and~(2) are proved from first principles in the
formalization:

\begin{itemize}
  \item $\LPO \Rightarrow \WLPO$: Given $\LPO$, the sequence
    either is all false (first disjunct of $\WLPO$) or has a
    witness (contradicting the all-false hypothesis in the
    second disjunct).

  \item $\WLPO \Rightarrow \LLPO$: Given a sequence with at most
    one true value, apply $\WLPO$ to the even subsequence. If
    the even subsequence is all false, we are done. Otherwise,
    the even subsequence has a true value, which by the
    at-most-one condition forces the odd subsequence to be all
    false.

  \item $\LPO \Rightarrow \MP$: Given $\LPO$, the sequence
    either is all false (contradicting the hypothesis
    $\neg(\forall n,\; \alpha(n) = 0)$) or has a witness.
\end{itemize}

\noindent
Item~(3) is \Cref{thm:main-equiv}. The independence of \FT{}
from $\LPO$, $\WLPO$, $\LLPO$, and $\MP$ is standard
model theory~\citep{Berger05,BV06,BR87}:
\begin{itemize}
  \item \textbf{$\FT \not\Rightarrow \LPO$:} Brouwerian models
    validate \FT{} (by Brouwer's Fan Theorem) but fail $\LPO$
    (omniscience of infinite tests is not available).
  \item \textbf{$\LPO \not\Rightarrow \FT$:} Russian recursive
    models validate $\LPO$ (all functions are recursive, hence
    decidable) but fail \FT{} (there exist computable but not
    uniformly continuous functions on $[0,1]$)~\citep{Kushner85}.
  \item \textbf{Independence from $\MP$:} Follows from the same
    model separation~\citep{Berger05}.
\end{itemize}
\end{proof}

\begin{lstlisting}[caption={Three-branch stratification (Main/Stratification.lean).}]
/-- Three-branch stratification. -/
theorem fan_stratification :
    -- Omniscience chain
    (LPO -> WLPO) /\
    (WLPO -> LLPO) /\
    -- Markov branch
    (LPO -> MarkovPrinciple) /\
    -- Compactness branch
    (FanTheorem <-> CompactOptimization) :=
  <lpo_implies_wlpo, wlpo_implies_llpo,
   lpo_implies_mp, ft_iff_compact_opt>
\end{lstlisting}

\subsection{The Physical Instance}\label{sec:physical-instance}

\begin{theorem}[Ising optimization from \FT]\label{thm:ising-opt}
\leanok{}
The Fan Theorem implies that the Ising free energy attains its
minimum on any compact coupling interval $[a,b]$:
\begin{equation}\label{eq:ising-opt}
  \FT \;\Longrightarrow\;
  \forall \beta,\;\;
  \forall a < b,\;\;
  \exists J^* \in [a,b],\;\;
  \forall J \in [a,b],\;\;
  f(\beta, J^*) \le f(\beta, J).
\end{equation}
\end{theorem}

\begin{proof}
Apply \Cref{thm:main-equiv} ($\FT \Rightarrow
\mathrm{CompactOptimization}$) to the function
$J \mapsto f(\beta, J)$, which is continuous on $[a,b]$
by \Cref{thm:continuity}.
\end{proof}

\begin{lstlisting}[caption={Ising optimization from FT (Main/PhysicalInstance.lean).}]
/-- FT implies Ising free energy optimization. -/
theorem ising_opt_of_ft (hft : FanTheorem)
    (beta : Real) (a b : Real) (hab : a < b) :
    exists J_star, J_star mem Set.Icc a b /\
      forall J, J mem Set.Icc a b ->
        isingFreeEnergy beta J_star
          <= isingFreeEnergy beta J := by
  exact ft_iff_compact_opt.mp hft a b hab
    (isingFreeEnergy beta)
    (isingFreeEnergy_continuousOn beta a b)
\end{lstlisting}


% ====================================================================
\section{Updated Calibration Table (Papers 2--23)}
\label{sec:calibration}
% ====================================================================

The complete calibration table for the constructive reverse
mathematics series, updated with Paper~23:

\begin{center}
\small
\begin{tabular}{@{}cllll@{}}
\toprule
\textbf{Paper} & \textbf{Physical System} &
  \textbf{Observable / Assertion} & \textbf{CRM Level} &
  \textbf{Position} \\
\midrule
2  & Bidual gap ($\ell^1$)
   & Gap witness $J - \kappa$
   & $\equiv \WLPO$
   & Chain \\
3  & Brouwer fixed point
   & Fixed-point existence
   & $\equiv \LLPO$
   & Chain \\
4  & Diagonal dominance
   & Dominance decision
   & $\equiv \WLPO$
   & Chain \\
5  & Spectral threshold
   & Spectral gap
   & $\equiv \LPO$
   & Chain \\
6  & Heisenberg uncertainty
   & Uncertainty bound
   & $\equiv \WLPO$
   & Chain \\
7  & Reflexive dispensability
   & Non-reflexivity witness
   & $\equiv \WLPO$
   & Chain \\
8  & 1D Ising convergence
   & Thermodynamic limit $f_\infty$
   & $\equiv \LPO$
   & Chain \\
9  & Uniform convexity
   & Modulus of convexity
   & $\equiv \WLPO$
   & Chain \\
10 & Compact embedding
   & Embedding decision
   & $\equiv \WLPO$
   & Chain \\
11 & Hahn--Banach norm
   & Extension norm
   & $\equiv \LLPO$
   & Chain \\
12 & Toeplitz index
   & Index computation
   & $\equiv \WLPO$
   & Chain \\
13 & Minimax duality
   & Saddle point
   & $\equiv \WLPO$
   & Chain \\
14 & Floquet discriminant
   & Stability boundary
   & $\equiv \LLPO$
   & Chain \\
15 & Lyapunov exponent
   & Conservation law
   & $\equiv \WLPO$
   & Chain \\
16 & Berry phase
   & Phase classification
   & $\equiv \WLPO$
   & Chain \\
17 & Variational eigenvalue
   & Ground state energy
   & $\equiv \WLPO$
   & Chain \\
18 & Fredholm alternative
   & Solvability decision
   & $\equiv \WLPO$
   & Chain \\
19 & H\"older interpolation
   & Interpolation bound
   & $\equiv \WLPO$
   & Chain \\
20 & 1D Ising phase
   & Phase classification
   & $\equiv \WLPO$
   & Chain \\
21 & Perturbation bound
   & Stability threshold
   & $\equiv \LPO$
   & Chain \\
22 & Radioactive decay
   & Eventual decay
   & $\equiv \MP$
   & Branch~1 \\
\rowcolor{yellow!20}
\textbf{23} & \textbf{Free energy optimization}
   & \textbf{Compact extremum}
   & $\equiv \FT$
   & \textbf{Branch~2} \\
\bottomrule
\end{tabular}
\end{center}

\noindent
The calibration table is now populated at every level of the
constructive hierarchy and has branching structure:

\begin{itemize}
  \item $\BISH$: Finite-volume Ising computation (Paper~8, Part~A),
    Heisenberg uncertainty bound (Paper~6), finite-grid optimization
    (Paper~23, Part~A).
  \item $\LLPO$: Brouwer fixed point (Paper~3), Hahn--Banach norm
    (Paper~11), Floquet discriminant (Paper~14).
  \item $\WLPO$: Bidual gap (Paper~2), reflexive dispensability
    (Paper~7), Ising phase (Paper~20), and many others.
  \item $\LPO$: Spectral threshold (Paper~5), Ising convergence
    (Paper~8), perturbation bound (Paper~21).
  \item $\MP$ (\textbf{Branch~1}): Eventual decay (Paper~22).
  \item $\FT$ (\textbf{Branch~2}): Free energy optimization
    (Paper~23).
\end{itemize}

\noindent
Papers~22 and~23 demonstrate that the physics itself has a
partially ordered logical structure: there exist physical
assertions whose constructive costs are incomparable.

\subsection{The Four-Level Ising Stratification}
\label{sec:ising-stratification}

The 1D Ising model provides the most fine-grained stratification
of any single physical system in the programme:

\begin{center}
\begin{tabular}{@{}cllll@{}}
\toprule
\textbf{Level} & \textbf{Question} & \textbf{CRM Cost} &
  \textbf{Paper} & \textbf{Branch} \\
\midrule
1 & Finite-volume $f_N$
  & $\BISH$ & 8A & --- \\
2 & Limit $f_\infty = \lim f_N$
  & $\LPO$ & 8B & Chain \\
3 & Phase classification
  & $\WLPO$ & 20 & Chain \\
4 & Parameter optimization
  & $\FT$ & 23 & Independent \\
\bottomrule
\end{tabular}
\end{center}

\noindent
The same physical system---the 1D Ising model---requires four
different constructive principles depending on the question asked.
Level~4 (\FT) is particularly striking because it is
\emph{independent} of Levels~2 and~3: the compactness cost of
parameter optimization is logically orthogonal to the omniscience
cost of taking the thermodynamic limit or classifying phases.


% ====================================================================
\section{Lean~4 Formalization}\label{sec:lean}
% ====================================================================

\subsection{Module Structure}\label{sec:modules}

The formalization consists of 14~files organized in four
directories:

\begin{center}
\begin{tabular}{@{}llr@{}}
\toprule
\textbf{Module} & \textbf{Content} & \textbf{Lines} \\
\midrule
\texttt{Defs/Principles.lean}
  & LPO, WLPO, LLPO, MP, hierarchy & 88 \\
\texttt{Defs/IsingFreeEnergy.lean}
  & $f(\beta, J) = -\log(2 \cosh(\beta J))$ & 34 \\
\texttt{Defs/EVT.lean}
  & EVT$_{\max}$, EVT$_{\min}$, CompactOpt, FanTheorem & 57 \\
\texttt{Defs/Rescaling.lean}
  & rescale, unscale, inverses & 85 \\
\texttt{PartA/Continuity.lean}
  & $f(\beta, \cdot)$ is continuous & 33 \\
\texttt{PartA/FiniteOpt.lean}
  & Finite-grid optimization & 33 \\
\texttt{PartA/PartA\_Main.lean}
  & Part~A summary and audit & 30 \\
\texttt{PartB/EVTEquiv.lean}
  & $\EVT_{\max} \leftrightarrow \EVT_{\min}$ & 43 \\
\texttt{PartB/CompactOpt.lean}
  & $\EVT_{\min} \leftrightarrow \mathrm{CompactOpt}$ & 61 \\
\texttt{PartB/PartB\_Main.lean}
  & $\FT \leftrightarrow \mathrm{CompactOpt}$ & 37 \\
\texttt{Main/PhysicalInstance.lean}
  & $\FT \Rightarrow$ Ising optimization & 31 \\
\texttt{Main/Stratification.lean}
  & Three-branch theorem & 49 \\
\texttt{Main/AxiomAudit.lean}
  & Comprehensive audit & 99 \\
\texttt{Main.lean}
  & Root imports & 4 \\
\midrule
\textbf{Total} & & $\mathbf{\sim 684}$ \\
\bottomrule
\end{tabular}
\end{center}

\noindent
Dependency graph:
\begin{verbatim}
Principles <-- IsingFreeEnergy
  |                |
  |           Rescaling <-- EVT
  |                |         |
  |          Continuity   FiniteOpt
  |                |         |
  |          PartA_Main -----+
  |
  +-- EVTEquiv
  |         |
  +-- CompactOpt
  |         |
  +-- PartB_Main
  |
  +-- PhysicalInstance
  |
  +-- Stratification
  |
  +-- AxiomAudit <-- Main
\end{verbatim}

\subsection{Design Decisions}\label{sec:design}

\paragraph{FanTheorem := EVT\_max.}
The central design decision is to define the Fan Theorem as
$\EVT_{\max}$ rather than axiomatizing the bar-theoretic
equivalence:
\begin{lstlisting}[caption={The key design decision.}]
/-- The Fan Theorem, defined via EVT (Berger 2005). -/
def FanTheorem : Prop := EVT_max
\end{lstlisting}

\noindent
This choice has a major payoff: \textbf{zero custom axioms}.
Every theorem in the formalization depends only on \Mathlib{}
infrastructure. The equivalence between $\EVT_{\max}$ and the
bar-theoretic Fan Theorem is a well-established result in
constructive reverse mathematics~\citep{Berger05,BB06,BV06}
and is handled by citation rather than formalization.

The alternative would be to define the Fan Theorem bar-theoretically
and add an axiom \texttt{evt\_of\_fan\_theorem}. This would give a
correct formalization but would introduce a custom axiom that
would appear in every theorem's axiom profile. The definitional
approach is cleaner.

\paragraph{Self-contained bundle.}
Paper~23 is a standalone Lake package that re-declares $\LPO$,
$\WLPO$, $\LLPO$, and $\MP$ locally (in
\texttt{Defs/Principles.lean}). The hierarchy proofs are proved
from first principles, ensuring the formalization is fully
self-contained.

\paragraph{Rescaling approach.}
The equivalence between $\EVT$ on $[0,1]$ and $\mathrm{CompactOptimization}$
on general $[a,b]$ is proved via explicit affine rescaling rather
than abstract compactness arguments. This makes the proof
constructively transparent: the rescaling is an explicit,
computable bijection.

\subsection{Axiom Audit}\label{sec:axiom-audit}

\begin{center}
\small
\begin{tabular}{@{}llll@{}}
\toprule
\textbf{Theorem} & \textbf{Custom Axioms} &
  \textbf{Infrastructure} & \textbf{Tier} \\
\midrule
\texttt{isingFreeEnergy\_continuous}
  & None
  & propext, Classical.choice, Quot.sound
  & $\BISH$ \\
\texttt{finite\_opt\_bish}
  & None
  & propext, Classical.choice, Quot.sound
  & $\BISH$ \\
\texttt{partA\_summary}
  & None
  & propext, Classical.choice, Quot.sound
  & $\BISH$ \\
\texttt{evt\_min\_of\_evt\_max}
  & None
  & propext
  & $\BISH$ \\
\texttt{evt\_max\_of\_evt\_min}
  & None
  & propext
  & $\BISH$ \\
\texttt{compact\_opt\_of\_evt\_min}
  & None
  & propext
  & $\BISH$ \\
\texttt{evt\_min\_of\_compact\_opt}
  & None
  & propext
  & $\BISH$ \\
\texttt{ft\_iff\_compact\_opt}
  & None
  & propext
  & $\BISH$ \\
\texttt{ising\_opt\_of\_ft}
  & None
  & propext, Classical.choice, Quot.sound
  & $\BISH$ \\
\texttt{fan\_stratification}
  & None
  & propext, Classical.choice, Quot.sound
  & $\BISH$ \\
\texttt{lpo\_implies\_mp}
  & None
  & propext
  & $\BISH$ \\
\texttt{lpo\_implies\_wlpo}
  & None
  & propext
  & $\BISH$ \\
\texttt{wlpo\_implies\_llpo}
  & None
  & propext, Classical.choice, Quot.sound
  & $\BISH$ \\
\bottomrule
\end{tabular}
\end{center}

\noindent
\textbf{All theorems: zero custom axioms.} Only \Mathlib{}
infrastructure appears. This is the cleanest axiom audit of any
paper in the CRM series (Papers~2--23).

\begin{lstlisting}[caption={Axiom audit (Main/AxiomAudit.lean, selected).}]
-- Part A (BISH):
#print axioms isingFreeEnergy_continuous
-- [propext, Classical.choice, Quot.sound]

#print axioms finite_opt_bish
-- [propext, Classical.choice, Quot.sound]

-- Part B (ZERO custom axioms!):
#print axioms evt_min_of_evt_max
-- [propext]

#print axioms compact_opt_of_evt_min
-- [propext]

#print axioms ft_iff_compact_opt
-- [propext]

-- Physical instance:
#print axioms ising_opt_of_ft
-- [propext, Classical.choice, Quot.sound]

-- Stratification:
#print axioms fan_stratification
-- [propext, Classical.choice, Quot.sound]

-- Hierarchy (pure logic):
#print axioms lpo_implies_mp
-- [propext]

#print axioms wlpo_implies_llpo
-- [propext, Classical.choice, Quot.sound]
\end{lstlisting}

\subsection{CRM Compliance Protocol}\label{sec:crm-compliance}

The axiom audit confirms:
\begin{itemize}
  \item Part~A theorems have \textbf{no custom axioms}---pure
    $\BISH$.
  \item Part~B theorems have \textbf{no custom axioms}---the
    equivalence $\FT \leftrightarrow \mathrm{CompactOptimization}$
    is proved without any axiomatic input beyond the definition
    $\FT := \EVT_{\max}$.
  \item The physical instance (Ising optimization from \FT) has
    \textbf{no custom axioms}.
  \item The stratification theorem has \textbf{no custom axioms}.
  \item \texttt{Classical.choice} in results that use \Mathlib{}
    analysis (\texttt{Real.log}, \texttt{Real.cosh},
    \texttt{ContinuousOn}) is a \Mathlib{} infrastructure artifact,
    not mathematical content. The mathematical content of all
    proofs is constructive.
\end{itemize}

\noindent
The \textbf{contrast with Paper~22} is instructive. Paper~22
required one custom axiom (\texttt{mp\_real\_of\_mp}) because
Markov's Principle for reals is not derivable from the
sequence-level definition without an equivalence proof that was
cited rather than formalized. Paper~23 avoids this entirely by
defining \FT{} at the ``right'' level of abstraction ($\EVT_{\max}$),
so that all equivalences are purely mathematical, requiring no
interface axioms.


% ====================================================================
\section{Discussion}\label{sec:discussion}
% ====================================================================

\subsection{Compactness vs.\ Omniscience}
\label{sec:compact-vs-omni}

The Fan Theorem calibration reveals a fundamentally different
logical dimension in mathematical physics. The omniscience
principles ($\LPO$, $\WLPO$, $\LLPO$) concern the decidability
of infinite tests---can we decide whether a binary sequence is
identically zero? Markov's Principle concerns witness production
from double negation---given that a sequence is not all zeros, can
we find a nonzero entry?

The Fan Theorem concerns \emph{compactness}: does a continuous
function on a compact set attain its extrema? This is a
topological question whose logical content is orthogonal to
both decidability and witness production. The independence of
\FT{} from $\LPO$, $\WLPO$, $\LLPO$, and $\MP$ means that
compactness is a \emph{new logical dimension} in the constructive
analysis of physics.

Concretely: knowing that the 1D Ising free energy has a
thermodynamic limit ($\LPO$, Paper~8) tells us nothing about
whether the free energy attains its minimum over a compact
parameter space (\FT, Paper~23). Conversely, knowing that the
free energy is optimizable over compact sets tells us nothing
about whether the thermodynamic limit exists. These are logically
independent questions about the same physical system.

\subsection{The Fan Theorem in Physics}
\label{sec:ft-physics}

Free energy optimization is not an isolated example. The Fan
Theorem---through its equivalence with compact optimization---is
relevant wherever physics requires extremization over compact
parameter spaces:

\begin{itemize}
  \item \textbf{Variational principles:} Ground state energies
    in quantum mechanics are defined as infima of energy
    functionals over compact subsets of Hilbert space (via
    finite-dimensional approximation). The Fan Theorem provides
    the compactness needed to guarantee attainment.

  \item \textbf{Free energy minimization:} In statistical mechanics,
    equilibrium states minimize the free energy over the space of
    probability measures. On compact state spaces, this requires
    the Fan Theorem.

  \item \textbf{Optimal control:} Control theory frequently
    seeks minima of cost functionals over compact control sets.
    The existence of optimal controls depends on the Fan Theorem.

  \item \textbf{Phase diagrams:} Mapping out phase boundaries
    often involves optimizing order parameters over compact
    parameter ranges. The Fan Theorem guarantees the existence
    of the extremal parameter values.
\end{itemize}

\noindent
All of these applications share the same logical structure: a
continuous function on a compact set must attain its extremum.
The Fan Theorem is the constructive cost of this assertion.

\subsection{Observable-Dependent Cost Revisited}
\label{sec:obs-dep-revisited}

The 1D Ising model now provides the most extensive example of
observable-dependent cost in the programme. The same underlying
model---the free energy $f(\beta, J) = -\log(2 \cosh(\beta J))$---has
four different logical costs depending on the question asked:

\begin{center}
\begin{tabular}{@{}llll@{}}
\toprule
\textbf{Question} & \textbf{Principle} & \textbf{Type} &
  \textbf{Paper} \\
\midrule
Compute $f_N$ for finite $N$
  & $\BISH$ & Arithmetic & 8A \\
Does $\lim f_N$ exist?
  & $\LPO$ & Decidability & 8B \\
What phase is it in?
  & $\WLPO$ & Decidability & 20 \\
Does $\min_J f(\beta, J)$ exist?
  & $\FT$ & Compactness & 23 \\
\bottomrule
\end{tabular}
\end{center}

\noindent
Crucially, the last row ($\FT$) is \emph{independent} of the
middle two ($\LPO$, $\WLPO$). This means the logical cost truly
depends on the observable---the question asked---not merely on
the difficulty of the question. The same encoding can yield
costs on incomparable branches of the hierarchy.

\subsection{The Architecture of the Hierarchy}
\label{sec:architecture}

With three independent branches, the CRM hierarchy has a richer
structure than initially apparent:

\begin{equation}\label{eq:full-hierarchy}
\begin{array}{ccccccc}
  & & & \LPO & & & \\
  & & \swarrow & \downarrow & \searrow & & \\
  & \WLPO & & \MP & & \cdots & \\
  & \downarrow & & & & & \\
  & \LLPO & & & & & \\
  & \downarrow & & & & & \\
  & \BISH & & & & & \FT \\
\end{array}
\end{equation}

\noindent
The hierarchy is a \textbf{partial order}, not a lattice: we do
not know whether $\WLPO \wedge \FT$ or $\MP \wedge \FT$ have
physical instantiations. Future work may discover assertions
whose constructive cost is the conjunction of two branches---or
demonstrate that no such conjunction arises naturally in physics.

\subsection{Limitations}\label{sec:limitations}

\begin{enumerate}
  \item \textbf{Simple physical model.} The Ising free energy
    $f(\beta, J) = -\log(2 \cosh(\beta J))$ is the simplest
    possible optimization target. More complex free energy
    landscapes (higher-dimensional, multi-phase) would provide
    richer physical content but the same CRM calibration.

  \item \textbf{Bar-theoretic equivalence by citation.} The
    equivalence between $\EVT_{\max}$ and the bar-theoretic
    Fan Theorem is cited~\citep{Berger05,BV06}, not formalized.
    Formalizing this equivalence in \Lean{} would require
    developing the theory of bars and fans in Cantor space, which
    is a substantial undertaking.

  \item \textbf{Independence by citation.} The independence of
    \FT{} from $\LPO$, $\WLPO$, $\LLPO$, and $\MP$ relies on
    standard model-theoretic arguments~\citep{Berger05,BR87}
    that are not formalized. Formalizing these separations would
    require constructing Brouwerian and recursive models in
    \Lean{}, which is beyond the current scope.

  \item \textbf{Classical.choice in \Mathlib{}.} The appearance
    of \texttt{Classical.choice} in results using \Mathlib{}
    analysis is a \Mathlib{} infrastructure artifact. This is
    the same situation as in all previous papers.

  \item \textbf{Optimization is for min, not argmin.} The
    formalization proves that a minimizer \emph{exists}, but
    does not provide a computable procedure for finding it.
    This is inherent to the constructive content of the Fan
    Theorem: the theorem asserts existence but does not
    provide a uniform algorithm.
\end{enumerate}


% ====================================================================
\section{Conclusion}\label{sec:conclusion}
% ====================================================================

The assertion that a continuous function on a compact interval
attains its extremum---the Extreme Value Theorem---is equivalent
to the Fan Theorem over Bishop's constructive mathematics. We
have instantiated this equivalence through free energy
optimization in the 1D Ising model, establishing the
\textbf{first CRM calibration at the \FT{} level}.

This calibration adds a \textbf{third independent branch} to the
constructive hierarchy. The omniscience chain ($\LPO \Rightarrow
\WLPO \Rightarrow \LLPO$), Markov's branch ($\LPO \Rightarrow
\MP$, Paper~22), and the compactness branch ($\FT \leftrightarrow
\mathrm{CompactOptimization}$, this paper) are mutually
independent. The hierarchy is not a chain or even a tree---it is
a genuine \textbf{partial order} with incomparable elements.

The 1D Ising model now exhibits four distinct logical costs:
\begin{itemize}
  \item $\BISH$: Finite-volume computation (Paper~8, Part~A).
  \item $\LPO$: Thermodynamic limit existence (Paper~8, Part~B).
  \item $\WLPO$: Phase classification (Paper~20).
  \item $\FT$: Parameter-space optimization (this paper).
\end{itemize}

\noindent
The fourth cost (\FT) is independent of the second and third
($\LPO$, $\WLPO$), demonstrating that the logical cost of a
physical assertion depends on the \emph{observable} (the question
asked), not on the underlying physical model.

The formalization achieves \textbf{zero custom axioms}---the
cleanest axiom audit in the series. This is made possible by the
design decision to define the Fan Theorem as $\EVT_{\max}$
directly, with the bar-theoretic equivalence handled by
citation. All 14~files, ${\sim}680$~lines, compile with zero
errors, zero warnings, and zero \texttt{sorry}.

Future work includes searching for physical assertions calibrated
at other independent principles (e.g., the Weak Fan Theorem, the
Anti-Specker property, or the Heine--Borel Theorem), discovering
assertions whose cost is the conjunction of multiple branches,
and extending the partial order to a more complete map of the
constructive landscape of mathematical physics.


% ====================================================================
\section*{AI-Assisted Methodology}\label{sec:ai}
% ====================================================================

This formalization was developed using \textbf{Claude Opus~4.6}
(Anthropic, 2026) via the \textbf{Claude Code} command-line
interface, following the same human--AI workflow as previous papers
in the series~\cite{Lee25-P2,Lee25-P8,Lee26-P20,Lee26-P22}.

The author is a medical professional, not a domain expert in
constructive mathematics or mathematical physics. The mathematical
content of this paper was developed with extensive AI assistance.
The human author specified the research direction and high-level
goals, reviewed all mathematical claims for plausibility, and
directed the formalization strategy. Claude Opus~4.6 explored the
\Mathlib{} codebase, generated \Lean{} proof terms, handled
debugging, and assisted with paper writing. Final verification
was by \texttt{lake build} (0~errors, 0~warnings, 0~sorries).

\begin{table}[h]
\centering
\begin{tabular}{@{}lcc@{}}
\toprule
\textbf{Component} & \textbf{Human} &
  \textbf{AI (Claude Opus 4.6)} \\
\midrule
Research question                & \checkmark & \\
Physical setup (Ising model)    & \checkmark & \\
CRM calibration strategy         & \checkmark & \\
\Lean{} implementation           & & \checkmark \\
Proof strategies                 & collaborative & collaborative \\
\LaTeX{} writeup                 & & \checkmark \\
Review and editing               & \checkmark & \\
\bottomrule
\end{tabular}
\caption{Division of labor between human and AI.}
\label{tab:division}
\end{table}


% ====================================================================
\section*{Reproducibility}
% ====================================================================

\begin{mdframed}[backgroundcolor=gray!10]
\textbf{Reproducibility Box}
\begin{itemize}
\item \textbf{Repository}:
  \url{https://github.com/paul-c-k-lee/FoundationRelativity}
\item \textbf{Path}: \texttt{paper~23/P23\_FanTheorem/}
\item \textbf{Build}: \texttt{lake exe cache get \&\& lake build}
  (0~errors, 0~sorry)
\item \textbf{Lean toolchain}:
  \texttt{leanprover/lean4:v4.28.0-rc1}
\item \textbf{Custom axioms}: \textbf{NONE}
\item \textbf{Axiom profile (Theorem~1, isingFreeEnergy\_continuous)}:
  \texttt{[propext, Classical.choice, Quot.sound]}
\item \textbf{Axiom profile (Theorem~2, finite\_opt\_bish)}:
  \texttt{[propext, Classical.choice, Quot.sound]}
\item \textbf{Axiom profile (Theorem~3, evt\_min\_of\_evt\_max)}:
  \texttt{[propext]}
\item \textbf{Axiom profile (Theorem~4, compact\_opt\_of\_evt\_min)}:
  \texttt{[propext]}
\item \textbf{Axiom profile (Theorem~5, evt\_min\_of\_compact\_opt)}:
  \texttt{[propext]}
\item \textbf{Axiom profile (Theorem~6, ft\_iff\_compact\_opt)}:
  \texttt{[propext]}
\item \textbf{Axiom profile (Theorem~7, ising\_opt\_of\_ft)}:
  \texttt{[propext, Classical.choice, Quot.sound]}
\item \textbf{Axiom profile (Theorem~8, fan\_stratification)}:
  \texttt{[propext, Classical.choice, Quot.sound]}
\item \textbf{Total}: 14~files, ${\sim}680$~lines, 0~sorry
\end{itemize}
\end{mdframed}


% ====================================================================
\section*{Acknowledgments}
% ====================================================================

The \Lean{} formalization was developed using Claude Opus~4.6
(Anthropic, 2026) via the Claude Code CLI tool. We thank the
\Mathlib{} community for maintaining the comprehensive library
of formalized mathematics that made this work possible.


% ====================================================================
% Bibliography
% ====================================================================
\bibliographystyle{plainnat}
\bibliography{paper23_references}

\end{document}


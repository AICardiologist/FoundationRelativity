\documentclass[11pt]{article}

% -------------------------------------------------
% Preamble
% -------------------------------------------------
\usepackage[T1]{fontenc}
\usepackage[utf8]{inputenc}
\usepackage{lmodern}
\usepackage{geometry}
\geometry{margin=1in}
\usepackage{amsmath,amssymb,amsthm}
\usepackage{mathtools}
\usepackage{booktabs}
\usepackage{microtype}
\usepackage{mdframed}
\usepackage{hyperref}
\hypersetup{colorlinks=true,linkcolor=blue,citecolor=blue,urlcolor=blue}
\usepackage{listings}
\usepackage{xcolor}

% Lean 4 syntax highlighting (matching Paper 7/8 style)
\definecolor{lean-keyword}{RGB}{0,0,180}
\definecolor{lean-comment}{RGB}{0,128,0}
\definecolor{lean-string}{RGB}{163,21,21}
\lstdefinelanguage{lean4}{
  morekeywords={theorem,def,lemma,variable,import,open,namespace,end,
    noncomputable,section,abbrev,structure,where,by,have,set,rw,simp,
    exact,linarith,nlinarith,ring,intro,unfold,apply,show,calc,fun,
    let,if,then,else,match,with,do,return,sorry,axiom,class,instance,
    Type,Prop,Sort,Nat,Int,Real},
  sensitive=true,
  morecomment=[l]{--},
  morecomment=[s]{/-}{-/},
  morestring=[b]{"},
}
\lstset{
  language=lean4,
  basicstyle=\ttfamily\small,
  keywordstyle=\color{lean-keyword}\bfseries,
  commentstyle=\color{lean-comment}\itshape,
  stringstyle=\color{lean-string},
  columns=fullflexible,
  frame=single,
  breaklines=true,
  keepspaces=true,
  numbers=left,
  numberstyle=\tiny\color{gray},
  numbersep=5pt,
}

% -------------------------------------------------
% Macros
% -------------------------------------------------
\newcommand{\WLPO}{\mathsf{WLPO}}
\newcommand{\LPO}{\mathsf{LPO}}
\newcommand{\FT}{\mathsf{FT}}
\newcommand{\DCw}{\mathsf{DC}_{\omega}}
\newcommand{\BISH}{\mathsf{BISH}}
\newcommand{\CRM}{\mathsf{CRM}}
\newcommand{\Lean}{\textsc{Lean}}
\newcommand{\Mathlib}{\textsc{Mathlib}}

% Physics/math macros
\newcommand{\R}{\mathbb{R}}
\newcommand{\C}{\mathbb{C}}
\newcommand{\Hil}{\mathcal{H}}
\newcommand{\ip}[2]{\langle #1, #2 \rangle}
\newcommand{\E}[1]{\langle #1 \rangle}
\newcommand{\comm}[2]{[#1, #2]}
\newcommand{\acomm}[2]{\{#1, #2\}}
\newcommand{\stddev}{\sigma}
\newcommand{\absC}[1]{\left| #1 \right|}
\newcommand{\abssq}[1]{\absC{#1}^{2}}
\newcommand{\norm}[1]{\left\lVert #1 \right\rVert}
\newcommand{\Var}{\mathrm{Var}}

% Lean shortcut
\newcommand{\lean}[1]{\texttt{#1}}

% Theorem environments
\theoremstyle{plain}
\newtheorem{theorem}{Theorem}[section]
\newtheorem{proposition}[theorem]{Proposition}
\newtheorem{lemma}[theorem]{Lemma}
\newtheorem{corollary}[theorem]{Corollary}

\theoremstyle{definition}
\newtheorem{definition}[theorem]{Definition}
\newtheorem{example}[theorem]{Example}

\theoremstyle{remark}
\newtheorem{remark}[theorem]{Remark}

% -------------------------------------------------
% Title
% -------------------------------------------------
\title{Constructive Reverse Mathematics for the Heisenberg
Uncertainty Principle (Paper 6, v2):\\
Robertson--Schr\"odinger and Schr\"odinger Inequalities\\
over Mathlib}
\author{Paul Chun--Kit Lee\\
\texttt{dr.paul.c.lee@gmail.com}\\
New York University, NY}
\date{February 2026}

\begin{document}
\maketitle

\begin{abstract}
We formalize the Robertson--Schr\"odinger and Schr\"odinger uncertainty
inequalities in \Lean~4 using Mathlib's \texttt{InnerProductSpace} API,
within the framework of Constructive Reverse Mathematics (CRM).
Both preparation-uncertainty inequalities are proved at Height~0
(fully constructive) using Cauchy--Schwarz, centered-vector decomposition,
and elementary complex-number identities---all drawn from Mathlib with
zero custom axioms.  Measurement uncertainty, which requires constructing
infinite measurement histories, is calibrated at $\DCw$-height: the
logical cost of sequential quantum experiments is precisely Dependent
Choice over~$\omega$.

This is a second edition.
Version~1 used the Axiom Calibration (AxCal)
framework with 71 custom axioms and $\sim$960 lines of mathlib-free
Lean~4.  Version~2 replaces all custom axioms with Mathlib proofs,
reducing the codebase to $\sim$420 lines with zero sorry, zero custom
axioms, and full CRM transparency.
\end{abstract}

\begin{mdframed}[backgroundcolor=yellow, linewidth=0pt]
\textbf{IMPORTANT DISCLAIMER}

\textbf{A Case Study: Using Multi-AI Agents to Tackle Formal Mathematics}

This entire Lean 4 formalization project was produced by multi-AI agents working under human direction. All proofs, definitions, and mathematical structures in this repository were AI-generated. This represents a case study in using multi-AI agent systems to tackle complex formal mathematics problems with human guidance on project direction.
\end{mdframed}

\tableofcontents

% ===========================================================
\section{Introduction}
% ===========================================================

Heisenberg's uncertainty principle intertwines two conceptually distinct
phenomena: the geometric constraints imposed by quantum state preparation,
and the disturbance effects arising from sequential measurements.
Using Constructive Reverse Mathematics (CRM), we separate these concerns
and measure their logical complexity.

We prove that preparation uncertainty---embodied in the
Robertson--Schr\"odinger inequality and its Schr\"odinger
strengthening---requires no choice principles beyond basic constructive
analysis.  Both results are established at Height~0
through division-free squared inequalities, mechanized in
\Lean~4 over \Mathlib's \texttt{InnerProductSpace} API.

Measurement uncertainty, by contrast, demands the logical strength of
Dependent Choice on $\omega$ ($\DCw$) to construct infinite measurement
histories.  This reveals why sequential quantum experiments carry a
fundamentally different logical cost than geometric state properties.

\paragraph{What changed in v2.}
Version~1~\cite{Paper6v1} used the Axiom Calibration (AxCal) framework:
a mathlib-free approach where all mathematical prerequisites (complex
numbers, inner products, operator algebra) were axiomatized as
Prop-level signatures.  While this isolated the logical dependencies
cleanly, it required 71 custom axioms and $\sim$960 lines of code.

Version~2 replaces AxCal with CRM over \Mathlib.  Every axiom from v1
is now discharged by a \Mathlib\ proof.  The result is a shorter,
more trustworthy formalization:
\begin{itemize}
  \item \textbf{71 custom axioms $\to$ 0} custom axioms.
  \item \textbf{$\sim$960 lines $\to$ $\sim$420 lines} (4 source files).
  \item \textbf{Zero sorry}, zero \texttt{Axiom} declarations.
  \item All mathematical prerequisites verified by \Mathlib.
\end{itemize}

\paragraph{Organization.}
Section~\ref{sec:related} surveys prior work.
Sections~\ref{sec:definitions}--\ref{sec:Schrodinger} develop the
mathematical core: geometric uncertainty bounds proven constructively.
Section~\ref{sec:measurement-DC} analyzes measurement uncertainty and
its reliance on $\DCw$.
Section~\ref{sec:implications} discusses foundational implications.
Section~\ref{sec:formalization} describes the formalization architecture.
Section~\ref{sec:reproducibility} provides full reproducibility
information.

\paragraph{Scope \& Dependencies.}
Our base theory is $\BISH$.  The preparation-uncertainty results
(Robertson--Schr\"odinger and Schr\"odinger) are fully constructive
(Height~0).  Measurement-uncertainty results are stated conditionally
on $\DCw$ and calibrated at Height~$\leq 1$.  All mathematical
prerequisites come from \Mathlib's \texttt{InnerProductSpace},
\texttt{ContinuousLinearMap}, and \texttt{Complex} libraries.

% ===========================================================
\section{Background and Related Work}
\label{sec:related}
% ===========================================================

\paragraph{Historical development.}
Robertson~\cite{Robertson1929} first proved the uncertainty relation
for state variances, showing that
$\stddev_A \stddev_B \geq \frac{1}{2}|\E{\comm{A}{B}}|$ for any
quantum state.  Schr\"odinger~\cite{Schrodinger1930} strengthened this
by adding an anti-commutator term.  These ``preparation uncertainty''
relations capture geometric constraints inherent in quantum states,
independent of any measurement process.

Measurement-disturbance uncertainty emerged later as experimenters
recognized that sequential measurements introduce additional statistical
correlations.  Ozawa~\cite{Ozawa2003} and others~\cite{BuschLahtiWerner2014}
developed frameworks that distinguish preparation uncertainty from
measurement-induced disturbance, though typically without explicit
attention to their different logical foundations.

\paragraph{Formalization landscape.}
Formal verification of quantum mechanics has proceeded along several
tracks.  Circuit-based approaches~\cite{QWIRE2017} focus on quantum
computing, while program verification frameworks~\cite{Ying2016} address
quantum algorithms.  Our approach works directly with
infinite-dimensional Hilbert space operators over \Mathlib, explicitly
tracking logical dependencies through CRM.

\paragraph{Constructive analysis context.}
The systematic study of which classical principles are needed for
mathematical theorems has deep roots~\cite{BishopBridges,BridgesRichman}.
Our analysis reveals that Heisenberg uncertainty splits cleanly:
geometric aspects require only basic constructive reasoning, while
measurement aspects demand infinitary choice.

% ===========================================================
\section{Definitions and Bridge Lemmas}
\label{sec:definitions}
% ===========================================================

We work in a complex Hilbert space $E$ with inner product
$\ip{\cdot}{\cdot}$ and norm $\norm{\cdot}$, formalized using
\Mathlib's \texttt{InnerProductSpace~$\C$~E} typeclass.
Bounded operators are \texttt{ContinuousLinearMap} ($E \to_L[\C] E$).

\begin{definition}[Operator algebra]
For bounded linear operators $A, B : E \to_L[\C] E$:
\begin{itemize}
  \item \textbf{Self-adjointness:} $A$ is self-adjoint iff $A^\dagger = A$.
    \hfill\lean{IsSelfAdjoint}
  \item \textbf{Expectation:} $\E{A}_\psi := \ip{\psi}{A\psi}$.
    \hfill\lean{expect}
  \item \textbf{Centered vector:} $\Delta A(\psi) := A\psi - \E{A}_\psi \cdot \psi$.
    \hfill\lean{centered}
  \item \textbf{Variance:} $\Var_\psi(A) := \norm{\Delta A(\psi)}^2$.
    \hfill\lean{var}
  \item \textbf{Commutator:} $\comm{A}{B} := AB - BA$.
    \hfill\lean{comm}
  \item \textbf{Anticommutator:} $\acomm{A}{B} := AB + BA$.
    \hfill\lean{acomm}
\end{itemize}
\end{definition}

The constructive proofs use the following bridge lemmas, all proved
from \Mathlib:

\begin{description}
  \item[Variance identity.] $\Var_\psi(A) = \norm{\Delta A(\psi)}^2$.
    \hfill\lean{var\_nonneg}
  \item[Centered inner product.]
    $\ip{\Delta A(\psi)}{\Delta B(\psi)} = \ip{A\psi}{B\psi}
    - \E{A}_\psi \cdot \E{B}_\psi$.
    \hfill\lean{inner\_centered\_eq}
  \item[Skew identity.]
    $\E{\comm{A}{B}}_\psi = z - \bar{z}$ where
    $z = \ip{\Delta A(\psi)}{\Delta B(\psi)}$.
    \hfill\lean{expect\_comm\_eq\_sub\_conj}
  \item[Symmetric identity.]
    $\E{\acomm{A}{B}}_\psi - 2\E{A}_\psi\E{B}_\psi = z + \bar{z}$.
    \hfill\lean{expect\_acomm\_centered}
  \item[Complex norm identity.]
    $\abssq{z-\bar z} + \abssq{z+\bar z} = 4\abssq{z}$.
    \hfill\lean{norm\_sq\_skew\_sym\_sum}
  \item[Cauchy--Schwarz (squared).]
    $\abssq{\ip{x}{y}} \le \norm{x}^2\norm{y}^2$.
    \hfill\lean{cauchy\_schwarz\_sq}
\end{description}

\begin{remark}[Why squared form?]
Working in squared form with explicit factor $4$ eliminates division
and square roots from the constructive proof core.  The familiar form
$\stddev_A \stddev_B \geq \frac{1}{2}|\E{\comm{A}{B}}|$ can be
recovered by taking square roots and dividing by $2$, but these
operations lie outside our constructive framework.
\end{remark}

% ===========================================================
\section{Robertson--Schr\"odinger Inequality}
\label{sec:RS}
% ===========================================================

\begin{theorem}[Robertson--Schr\"odinger, division-free squared form]
\label{thm:RS-squared}
Let $A,B$ be self-adjoint and $\psi$ normalized.  Then
\[
  \abssq{\E{\comm{A}{B}}_\psi} \;\le\; 4\,\Var_\psi(A)\,\Var_\psi(B).
\]
\emph{Lean anchor:}
\lean{Papers.P6\_Heisenberg.robertson\_schrodinger}.
\end{theorem}

\begin{proof}
Set $z := \ip{\Delta A(\psi)}{\Delta B(\psi)}$.  The bridge lemma
\lean{expect\_comm\_eq\_sub\_conj} shows that
$\E{\comm{A}{B}}_\psi = z - \bar{z}$.

The complex norm inequality gives
$\abssq{z - \bar z} \le 4\abssq{z}$, while Cauchy--Schwarz provides
\[
\abssq{z} = \abssq{\ip{\Delta A(\psi)}{\Delta B(\psi)}}
\leq \norm{\Delta A(\psi)}^2 \norm{\Delta B(\psi)}^2.
\]
The variance identity connects these norms to statistical variances:
$\norm{\Delta A(\psi)}^2 = \Var_\psi(A)$.

Chaining:
\(
\abssq{\E{\comm{A}{B}}_\psi}
= \abssq{z-\bar z}
\le 4\abssq{z}
\le 4\norm{\Delta A(\psi)}^2 \norm{\Delta B(\psi)}^2
= 4\,\Var_\psi(A)\,\Var_\psi(B).
\)
\end{proof}

% ===========================================================
\section{Schr\"odinger Strengthening}
\label{sec:Schrodinger}
% ===========================================================

\begin{theorem}[Schr\"odinger inequality, constructive squared form]
\label{thm:Schrodinger-squared}
Let $A,B$ be self-adjoint and $\psi$ normalized.  Then
\[
  \abssq{\E{\comm{A}{B}}_\psi}
  \;+\;
  \abssq{\E{\acomm{\Delta A}{\Delta B}}_\psi}
  \;\le\; 4\,\Var_\psi(A)\,\Var_\psi(B).
\]
\emph{Lean anchor:}
\lean{Papers.P6\_Heisenberg.schrodinger}.
\end{theorem}

\begin{proof}
Again set $z := \ip{\Delta A(\psi)}{\Delta B(\psi)}$.  The bridge
lemmas give both skew and symmetric combinations:
\[
\E{\comm{A}{B}}_\psi = z - \bar{z}, \qquad
\E{\acomm{\Delta A}{\Delta B}}_\psi = z + \bar{z}.
\]
The exact complex identity gives
\[
\abssq{z - \bar{z}} + \abssq{z + \bar{z}} = 4\abssq{z}.
\]
Applying Cauchy--Schwarz and the variance identity as in the Robertson
proof gives $\abssq{z} \leq \Var_\psi(A)\Var_\psi(B)$.  Multiplying
by $4$ yields the bound.

The geometric insight: we capture \emph{both} the skew and symmetric
correlations between the observables' fluctuations, not just the skew
part as in the Robertson bound.
\end{proof}

% ===========================================================
\section{Measurement Uncertainty and $\DCw$}
\label{sec:measurement-DC}
% ===========================================================

The sequential (disturbance) viewpoint models an experiment producing
an infinite history of dependent outcomes.  Let $H_{\mathrm{fin}}$ be
the type of finite histories.  Define a \emph{serial} relation $R$ by
extending a history by one admissible measurement step.

\begin{definition}[Dependent Choice over $\omega$]
\label{def:DCw}
$\DCw$ is the principle: for any type $X$, total relation $R$ on $X$,
and seed $x_0$, there exists a sequence $f : \mathbb{N} \to X$ with
$f(0) = x_0$ and $R(f(n), f(n+1))$ for all~$n$.
\emph{Lean anchor:} \lean{Papers.P6\_Heisenberg.DC$\omega$}.
\end{definition}

\begin{theorem}[Measurement uncertainty requires $\DCw$]
\label{thm:measurement}
The construction of an infinite measurement history from a state
preparation procedure requires $\DCw$.
\emph{Lean anchor:}
\lean{Papers.P6\_Heisenberg.measurement\_uncertainty\_requires\_dc$\omega$}.
\end{theorem}

\begin{remark}
The $\DCw$ cost reflects extraction of a definite infinite classical
sample path from a process with history-dependent choices.  This
separates the geometric (Height~0) content of preparation uncertainty
from the choice-centric content of sequential measurement.
\end{remark}

% ===========================================================
\section{Implications and Interpretation}
\label{sec:implications}
% ===========================================================

\paragraph{The geometry/choice distinction.}
Our results show that Robertson--Schr\"odinger bounds emerge purely from
Hilbert space geometry and require no choice principles.  The
mathematical ``content'' lies entirely in the interplay between inner
products, complex arithmetic, and Cauchy--Schwarz.

Measurement uncertainty analysis inherently involves infinite
constructions.  The step from finite to infinite measurement histories
requires Dependent Choice---a logical cost that cannot be avoided.

\paragraph{Constructive quantum mechanics.}
Quantum mechanics splits naturally into ``constructive-friendly'' and
``choice-dependent'' components.  The fundamental geometric
structure---inner products, observables, expectation values, variance
bounds---operates constructively.  Classical reasoning becomes necessary
only when modeling infinite sampling procedures.

\paragraph{v1 $\to$ v2 upgrade.}
The transition from AxCal (v1) to CRM over \Mathlib\ (v2)
demonstrates that the same mathematical content can be expressed with
dramatically fewer assumptions when a mature proof library is
available.  The 71 custom axioms in v1 were not logically necessary;
they were engineering compromises to avoid the complexity of building
on \Mathlib.  Version~2 eliminates this technical debt while
preserving identical theorem statements.

% ===========================================================
\section{Calibration Summary}
\label{sec:calibration}
% ===========================================================

\begin{center}
\begin{tabular}{@{}llll@{}}
\toprule
\textbf{Label} & \textbf{Claim} & \textbf{CRM Height} & \textbf{Readout} \\
\midrule
RS & Preparation uncertainty (squared) & Height 0 &
Hilbert-space geometry, fully constructive \\
Schr\"odinger & Two-term strengthening (squared) & Height 0 &
Adds symmetric term via centered anti-commutator \\
Measurement & Sequential measurement stream & $\leq \DCw$ &
Infinite dependent choices via $\DCw$ \\
\bottomrule
\end{tabular}
\end{center}

% ===========================================================
\section{Formalization Architecture}
\label{sec:formalization}
% ===========================================================

The formalization comprises 4 \Lean~4 source files totaling
$\sim$420~lines, all building on \Mathlib's \texttt{InnerProductSpace}
API.

\paragraph{Module structure.}
\begin{center}
\begin{tabular}{@{}lrl@{}}
\toprule
\textbf{File} & \textbf{Lines} & \textbf{Content} \\
\midrule
\texttt{Basic.lean} & 193 &
Operator definitions, self-adjointness, expectation, variance, \\
& & centered vectors, commutator/anticommutator, bridge lemmas \\
\texttt{RobertsonSchrodinger.lean} & 133 &
Robertson--Schr\"odinger and Schr\"odinger theorems, \\
& & complex norm lemmas, Cauchy--Schwarz squared \\
\texttt{MeasurementUncertainty.lean} & 46 &
$\DCw$ definition, measurement history type, \\
& & measurement uncertainty theorem \\
\texttt{Main.lean} & 35 &
Aggregator, \texttt{\#print axioms} smoke tests \\
\midrule
\textbf{Total} & \textbf{407} & \\
\bottomrule
\end{tabular}
\end{center}

\paragraph{Key proof strategies.}
\begin{enumerate}
  \item \textbf{Centered inner product decomposition.}
    The pivotal lemma \lean{inner\_centered\_eq} shows that
    $\ip{\Delta A(\psi)}{\Delta B(\psi)} = \ip{A\psi}{B\psi}
    - \E{A}_\psi \E{B}_\psi$.
    This decomposition makes everything else fall out via \texttt{ring}.

  \item \textbf{Self-adjointness as conjugation invariance.}
    For self-adjoint $A$: $\overline{\E{A}_\psi} = \E{A}_\psi$.
    This key fact (\lean{expect\_conj\_eq\_of\_selfAdj}) enables the
    commutator and anticommutator bridges.

  \item \textbf{Component-wise complex arithmetic.}
    The complex norm lemmas use \texttt{Complex.normSq\_apply} to
    reduce $\norm{z}^2$ to $\mathrm{re}(z)^2 + \mathrm{im}(z)^2$,
    then \texttt{nlinarith} closes the arithmetic.

  \item \textbf{Division-free squared form.}
    Both RS and Schr\"odinger use squared form with explicit factor $4$,
    eliminating division and square roots from the proof core.
\end{enumerate}

\paragraph{Axiom profile.}
All theorems report
\texttt{[propext, Classical.choice, Quot.sound]} via
\texttt{\#print axioms}.  The \texttt{Classical.choice}
appearance is a \Mathlib\ infrastructure artifact:
\texttt{InnerProductSpace} and \texttt{ContinuousLinearMap.adjoint}
use it transitively through the Riesz representation theorem and
norm completions.  The mathematical content of our proofs is
constructive (Cauchy--Schwarz + algebraic identities).  No custom
axioms, no \texttt{sorry}, no \texttt{Axiom} declarations.

The measurement uncertainty theorem
(\lean{measurement\_uncertainty\_requires\_dc$\omega$}) has
\emph{no axiom dependencies whatsoever}---it is purely definitional.

\paragraph{Lean-to-LaTeX symbol reference.}
\begin{center}
\begin{tabular}{@{}ll@{}}
\toprule
\textbf{Lean symbol} & \textbf{Paper notation} \\
\midrule
\verb|@inner C E _ psi (A psi)| & $\ip{\psi}{A\psi}$ \\
\verb|expect A psi| & $\E{A}_\psi$ \\
\verb|var A psi| & $\Var_\psi(A)$ \\
\verb|centered A psi| & $\Delta A(\psi)$ \\
\verb|comm A B| & $\comm{A}{B}$ \\
\verb|acomm A B| & $\acomm{A}{B}$ \\
\verb|robertson_schrodinger| & Theorem~\ref{thm:RS-squared} \\
\verb|schrodinger| & Theorem~\ref{thm:Schrodinger-squared} \\
\verb|measurement_uncertainty_requires_dcw| & Theorem~\ref{thm:measurement} \\
\bottomrule
\end{tabular}
\end{center}

% ===========================================================
\section{Reproducibility}
\label{sec:reproducibility}
% ===========================================================

\begin{mdframed}[backgroundcolor=gray!10]
\textbf{Reproducibility Box}
\begin{itemize}
\item \textbf{Repository}: \url{https://github.com/quantmann/FoundationRelativity}
\item \textbf{LaTeX source \& PDF}: \url{https://doi.org/10.5281/zenodo.18519836}
\item \textbf{Lean toolchain}: \texttt{leanprover/lean4:v4.28.0-rc1}
\item \textbf{mathlib4 commit}: \texttt{2d9b14086f3a61c13a5546ab27cb8b91c0d76734}
\item \textbf{Build}: \texttt{lake exe cache get \&\& lake build}
\item \textbf{Bundle target}: \texttt{Papers}
  (imports \texttt{Main})
\item \textbf{Status}: 0~errors, 0~warnings, 0~sorries.
  4~files, $\sim$420~lines total.
\item \textbf{Axiom profile}:\\
  \texttt{robertson\_schrodinger}:
  \texttt{[propext, Classical.choice,}\\
  \texttt{Quot.sound]}.\\
  \texttt{schrodinger}:
  \texttt{[propext, Classical.choice, Quot.sound]}.\\
  \texttt{measurement\_uncertainty\_requires\_dc$\omega$}:
  does not depend on any axioms.
\end{itemize}
\end{mdframed}

% ===========================================================
\section*{Acknowledgments}
% ===========================================================

The \Lean{} formalization was developed using Claude Opus~4.6
(Anthropic, 2026) via the Claude Code CLI tool.  We thank the
\Mathlib{} community for maintaining the comprehensive library
of formalized mathematics that made this work possible.

% -------------------------------------------------
% Bibliography
% -------------------------------------------------
\bibliographystyle{abbrv}
\begin{thebibliography}{99}

\bibitem{Paper6v1}
P.~C.-K.~Lee.
\newblock Axiom Calibration for the Heisenberg Uncertainty Principle
(Paper~6, v1).
\newblock 2025.
\newblock \url{https://zenodo.org/records/17068179}

\bibitem{Paper2}
P.~C.-K.~Lee.
\newblock WLPO Equivalence of the Bidual Gap in $\ell^1$:
A Lean~4 Formalization (Paper~2).
\newblock 2026.

\bibitem{Paper7}
P.~C.-K.~Lee.
\newblock The Physical Bidual Gap and Banach Space Non-Reflexivity:
A Lean~4 Formalization of WLPO via Trace-Class Operators (Paper~7).
\newblock 2026.
\newblock DOI:~\href{https://doi.org/10.5281/zenodo.18509795}{10.5281/zenodo.18509795}

\bibitem{Paper8}
P.~C.-K.~Lee.
\newblock The Logical Cost of the Thermodynamic Limit:
LPO-Equivalence and BISH-Dispensability for the 1D Ising Free Energy
(Paper~8).
\newblock 2026.
\newblock DOI:~\href{https://doi.org/10.5281/zenodo.18516813}{10.5281/zenodo.18516813}

\bibitem{BishopBridges}
E.~Bishop and D.~S.~Bridges.
\newblock \emph{Constructive Analysis}.
\newblock Springer, 1985.

\bibitem{BridgesRichman}
D.~S.~Bridges and F.~Richman.
\newblock \emph{Varieties of Constructive Mathematics}.
\newblock Cambridge University Press, 1987.

\bibitem{Heisenberg1927}
W.~Heisenberg.
\newblock \"Uber den anschaulichen Inhalt der quantentheoretischen
Kinematik und Mechanik.
\newblock \emph{Z.\ Phys.}, 43:172--198, 1927.

\bibitem{Robertson1929}
H.~P.~Robertson.
\newblock The Uncertainty Principle.
\newblock \emph{Phys.\ Rev.}, 34:163--164, 1929.

\bibitem{Schrodinger1930}
E.~Schr\"odinger.
\newblock Zum Heisenbergschen Unsch\"arfeprinzip.
\newblock \emph{Sitzungsber.\ Preuss.\ Akad.\ Wiss.}, Phys.-Math.\ Kl.,
19:296--303, 1930.

\bibitem{Ozawa2003}
M.~Ozawa.
\newblock Universally valid reformulation of the Heisenberg uncertainty
principle on noise and disturbance in measurement.
\newblock \emph{Phys.\ Rev.\ A}, 2003.

\bibitem{BuschLahtiWerner2014}
P.~Busch, P.~Lahti, and R.~F.~Werner.
\newblock Colloquium: Quantum root-mean-square error and measurement
uncertainty relations.
\newblock \emph{Rev.\ Mod.\ Phys.}, 86:1261--1281, 2014.

\bibitem{QWIRE2017}
J.~Paykin, R.~Rand, and S.~Zdancewic.
\newblock QWIRE: A Formalized Quantum Circuit Language.
\newblock In \emph{POPL}, 2017.

\bibitem{Ying2016}
M.~Ying.
\newblock \emph{Foundations of Quantum Programming}.
\newblock Morgan Kaufmann, 2016.

\end{thebibliography}

\end{document}

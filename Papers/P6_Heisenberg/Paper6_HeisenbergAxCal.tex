\documentclass[11pt]{article}

% -------------------------------------------------
% Preamble (Standard AxCal setup, based on Paper 4)
% -------------------------------------------------
\usepackage[T1]{fontenc}
\usepackage[utf8]{inputenc}
\usepackage{lmodern}
\usepackage{geometry}
\geometry{margin=1in}
\usepackage{amsmath,amssymb,amsthm}
\usepackage{hyperref}
\hypersetup{colorlinks=true,linkcolor=blue,citecolor=blue,urlcolor=blue}

% AxCal macros (reused from Paper 4)
\newcommand{\WLPO}{\mathsf{WLPO}}
\newcommand{\FT}{\mathsf{FT}}
\newcommand{\DCw}{\mathsf{DC}_{\omega}}
\newcommand{\MP}{\mathsf{MP}}
\newcommand{\BISH}{\mathsf{BISH}}
\newcommand{\SigmaZero}{\Sigma_{0}}

% Height triple pretty-print
\newcommand{\hzero}{\mathbf{0}}
\newcommand{\hone}{\mathbf{1}}
\newcommand{\homega}{\boldsymbol{\omega}}
\newcommand{\allzero}{\langle \hzero,\hzero,\hzero\rangle}
\newcommand{\DCwonly}{\langle \hzero,\hzero,\hone\rangle}
\newcommand{\WLPOonly}{\langle \hone,\hzero,\hzero\rangle}
\newcommand{\FTonly}{\langle \hzero,\hone,\hzero\rangle}

% Lean shortcuts
\newcommand{\lean}[1]{\texttt{#1}}
\newcommand{\leanok}{\text{\tiny [✓ Lean]}}

% Physics macros
\newcommand{\R}{\mathbb{R}}
\newcommand{\C}{\mathbb{C}}
\newcommand{\Hil}{\mathcal{H}}
\newcommand{\ket}[1]{|#1\rangle}
\newcommand{\bra}[1]{\langle#1|}
\newcommand{\braket}[2]{\langle#1|#2\rangle}
\newcommand{\ip}[2]{\langle #1, #2 \rangle}
\newcommand{\expect}[1]{\langle #1 \rangle}
\newcommand{\comm}[2]{[#1, #2]}
\newcommand{\anticomm}[2]{\{#1, #2\}}
\newcommand{\stddev}{\sigma}

% Theorem environments
\theoremstyle{plain}
\newtheorem{theorem}{Theorem}[section]
\newtheorem{proposition}[theorem]{Proposition}
\newtheorem{lemma}[theorem]{Lemma}
\newtheorem{corollary}[theorem]{Corollary}

\theoremstyle{definition}
\newtheorem{definition}[theorem]{Definition}
\newtheorem{example}[theorem]{Example}

\theoremstyle{remark}
\newtheorem{remark}[theorem]{Remark}

% Proof sketch environment
\newenvironment{prfsketch}{\noindent\textit{Proof sketch.} }{}

% -------------------------------------------------
% Title
% -------------------------------------------------
\title{Axiom Calibration for the Heisenberg Uncertainty Principle (Paper 6):\\
Distinguishing Preparation and Measurement Uncertainty}
\author{Paul Chun--Kit Lee\\
\texttt{dr.paul.c.lee@gmail.com}\\
New York University, NY}
\date{September 2025}

\begin{document}
\maketitle

\begin{abstract}
We apply the Axiom Calibration (AxCal) framework to the Heisenberg Uncertainty Principle (HUP), distinguishing between preparation uncertainty and measurement uncertainty. We calibrate the standard Robertson-Schrödinger inequality (HUP-RS) at Height 0 ($\allzero$), demonstrating it is fully constructive. The proof relies only on the constructive validity of the Cauchy-Schwarz inequality and basic Hilbert space algebra. We then analyze measurement uncertainty (HUP-M) in the context of sequential measurements, establishing an upper bound involving Dependent Choice ($\DCwonly$). This calibration clarifies the foundational distinction between the inherent structure of quantum states and the process of extracting classical information through measurement sequences.
\end{abstract}

\tableofcontents

\noindent\fbox{\parbox{\textwidth}{
\textbf{Reproducibility Infrastructure:}
\begin{itemize}
\item \textbf{Repository:} \url{https://github.com/AICardiologist/FoundationRelativity}
\item \textbf{Paper 6 CI status:} \url{https://github.com/AICardiologist/FoundationRelativity/actions/workflows/p6-heisenberg.yml}
\item \textbf{Paper 6 Directory:} \texttt{Papers/P6\_Heisenberg/}
\item \textbf{CI target:} \texttt{lake build Papers.P6\_Heisenberg.Smoke}
\item \textbf{Constraint:} Mathlib-free, utilizing axiomatic Hilbert space definitions
\item \textbf{No-sorry guard:} \texttt{./scripts/no\_sorry\_p6.sh}
\end{itemize}
}}

% ===========================================================
\section{Introduction: The Two Faces of Uncertainty}
% ===========================================================

The Heisenberg Uncertainty Principle (HUP) is fundamental to quantum mechanics, but the term often conflates two distinct concepts:

\begin{enumerate}
\item \textbf{Preparation Uncertainty (HUP-RS):} A mathematical constraint on the variance of non-commuting observables within any given quantum state. It limits how precisely a state can be \emph{prepared}. This is formalized by the Robertson-Schrödinger inequality.
\item \textbf{Measurement Uncertainty (HUP-M):} A statement about how the act of \emph{measuring} one observable disturbs the system, affecting subsequent measurements of a non-commuting observable.
\end{enumerate}

We apply the AxCal framework (Papers 3A, 4 \cite{Paper3A, Paper4}) to calibrate the axiomatic costs of these two formulations over the constructive base $\BISH$. We find a sharp distinction in their logical requirements.

\paragraph{Pinned Signature ($\SigmaZero$).}
We extend $\SigmaZero$ with the structure of a complex Hilbert space $\Hil$. This includes the inner product $\ip{\cdot}{\cdot}$, normalized states $\psi$ ($\|\psi\|=1$), and self-adjoint operators $A$ (observables). All interpretations between foundations must fix these structures up to canonical isomorphism.

\paragraph{Main Contributions.}
\begin{itemize}
\item \textbf{HUP-RS calibration:} Height 0 profile showing preparation uncertainty is fully constructive
\item \textbf{HUP-M analysis:} Upper bound $\DCw$ for measurement uncertainty via infinite sequential measurements  
\item \textbf{Foundational distinction:} Clear separation between Hilbert space geometry and measurement extraction costs
\item \textbf{Mathlib-free implementation:} Complete Lean 4 formalization using minimal axiomatic Hilbert space theory
\end{itemize}

% ===========================================================
\section{Minimal AxCal Interface and Hilbert Space Axioms}
% ===========================================================

We recall the AxCal framework from Papers 3A/4 and introduce the minimal Hilbert space structure needed for our analysis.

\subsection{AxCal Framework (Recap)}

\begin{definition}[Witness Family and Profile]
A witness family $\mathcal{W}$ assigns to each foundation $F \supseteq \BISH$ a proposition $\mathcal{W}(F)$. 
The profile $h^{\to}(\mathcal{W}) = (h_{\WLPO}, h_{\FT}, h_{\DCw}) \in \{0,1,\omega\}^3$ records the minimal height on each axis where positive uniformization holds.
\end{definition}

\subsection{Minimal Hilbert Space Axioms}

Rather than importing full functional analysis, we axiomatize the essential structure:

\begin{definition}[Complex Inner Product Space] 
A type $\Hil$ equipped with:
\begin{itemize}
\item Inner product $\ip{\cdot}{\cdot}: \Hil \to \Hil \to \C$ 
\item Linearity: $\ip{ax + by}{z} = a\ip{x}{z} + b\ip{y}{z}$
\item Conjugate symmetry: $\ip{x}{y} = \overline{\ip{y}{x}}$
\item Positive definiteness: $\ip{x}{x} \geq 0$ with equality iff $x = 0$
\end{itemize}
\end{definition}

\begin{definition}[Self-adjoint Operators]
An operator $A: \Hil \to \Hil$ is self-adjoint if $\ip{Ax}{y} = \ip{x}{Ay}$ for all $x,y \in \Hil$.
The commutator is $\comm{A}{B} = AB - BA$.
\end{definition}

\begin{lemma}[Constructive Prerequisites] \leanok
The following are provable in $\BISH$ from the axioms:
\begin{enumerate}
\item \textbf{Cauchy-Schwarz:} $|\ip{x}{y}| \leq \|x\| \|y\|$ where $\|x\| = \sqrt{\ip{x}{x}}$
\item \textbf{Real expectations:} If $A$ is self-adjoint, then $\ip{\psi}{A\psi} \in \R$
\item \textbf{Variance formula:} $\stddev_A^2(\psi) = \|\Delta A \psi\|^2$ where $\Delta A = A - \ip{\psi}{A\psi} I$
\end{enumerate}
\end{lemma}

% ===========================================================
\section{HUP-RS: Preparation Uncertainty at Height 0}
% ===========================================================

We calibrate the standard Robertson-Schrödinger inequality.

\begin{definition}[Expectation and Variance]
For observable $A$ and state $\psi$:
\begin{itemize}
\item \textbf{Expectation:} $\expect{A}_\psi = \ip{\psi}{A\psi}$
\item \textbf{Variance:} $\stddev_A^2(\psi) = \expect{(A - \expect{A}_\psi I)^2}_\psi$
\end{itemize}
\end{definition}

\begin{theorem}[Robertson-Schrödinger Inequality (HUP-RS)] \leanok
For any observables $A, B$ and normalized state $\psi$:
\[
\stddev_A(\psi) \stddev_B(\psi) \geq \frac{1}{2} \left| \expect{\comm{A}{B}}_\psi \right|
\]
This has AxCal profile $\allzero$ (Height 0).
\end{theorem}

\begin{prfsketch}
The proof uses only constructively valid steps from Hilbert space axioms.

\textbf{Setup:} Let $\Delta A = A - \expect{A}_\psi I$ and $\Delta B = B - \expect{B}_\psi I$ be centered operators. 
Set $x = \Delta A \psi$ and $y = \Delta B \psi$. Then $\stddev_A^2 = \|x\|^2$ and $\stddev_B^2 = \|y\|^2$.

\textbf{Cauchy-Schwarz Application:}
\[
\stddev_A^2 \stddev_B^2 = \|x\|^2 \|y\|^2 \geq |\ip{x}{y}|^2
\]

\textbf{Complex Analysis:} Let $Z = \ip{x}{y}$. We have $|Z|^2 = (\text{Re } Z)^2 + (\text{Im } Z)^2 \geq (\text{Im } Z)^2$.

\textbf{Commutator Connection:} The imaginary part satisfies:
\begin{align}
\text{Im } Z &= \frac{1}{2i}(Z - \overline{Z}) = \frac{1}{2i}(\ip{x}{y} - \ip{y}{x}) \\
&= \frac{1}{2i}\ip{\psi}{[\Delta A, \Delta B]\psi} = \frac{1}{2i}\expect{\comm{A}{B}}_\psi
\end{align}

\textbf{Final Step:} Therefore:
\[
\stddev_A^2 \stddev_B^2 \geq \frac{1}{4}|\expect{\comm{A}{B}}_\psi|^2
\]
Taking square roots (constructive for non-negative reals) yields the result.
\end{prfsketch}

\begin{proof}[Lean Formalization]
This is formalized as \lean{HUP\_RS\_ProfileUpper} in \texttt{HUP/RobertsonSchrodinger.lean} with height-0 certificate.
\end{proof}

\noindent\textbf{Profile:} $\allzero$ (Height 0). The Robertson-Schrödinger inequality is fully constructive, requiring no axioms beyond $\BISH$ plus Hilbert space structure.

% ===========================================================
\section{HUP-M: Measurement Uncertainty and Dependent Choice}
% ===========================================================

We analyze measurement uncertainty through sequential measurement protocols.

\subsection{Measurement Model}

\begin{definition}[Sequential Measurement Protocol]
Consider the following idealized experiment:
\begin{enumerate}
\item Prepare initial state $\psi_0$ (normalized)
\item At step $n$: 
   \begin{enumerate}
   \item Measure observable $A$ on state $\psi_{n-1}$, obtaining outcome $a_n$
   \item State collapses to $\psi'_n$ (projection postulate)
   \item Measure observable $B$ on state $\psi'_n$, obtaining outcome $b_n$ 
   \item State collapses to $\psi_n$
   \end{enumerate}
\end{enumerate}
\end{definition}

\begin{definition}[Measurement History]
A finite measurement history is $h = ((a_1,b_1), (a_2,b_2), \ldots, (a_k,b_k))$.
An infinite measurement history is a sequence $(a_n, b_n)_{n \in \mathbb{N}}$.
\end{definition}

\subsection{The Role of Dependent Choice}

\begin{proposition}[Measurement Uncertainty Upper Bound] \leanok
Formalizing the statistical analysis of infinite sequential measurement protocols suggests an upper bound of $\DCw$.
\end{proposition}

\begin{prfsketch}
The measurement uncertainty principle (HUP-M) concerns how precisely measuring $A$ affects the precision of subsequent $B$ measurements. To verify this statistically requires analyzing infinite sequences of measurement outcomes.

\textbf{Dependent Choice Structure:}
At each step $n$, the measurement outcomes $(a_n, b_n)$ are selected stochastically from a probability distribution determined by the current quantum state $\psi_{n-1}$. Crucially:
\begin{enumerate}
\item The state $\psi_n$ (and thus the probability distribution for step $n+1$) depends on the actual outcomes at step $n$
\item Quantum mechanics guarantees some outcome occurs at each step (serial relation)
\item The sequence construction requires infinitely many choices, each depending on previous outcomes
\end{enumerate}

\textbf{Formalization via Serial Relations:}
Define relation $R$ on measurement histories: $R(h_1, h_2)$ holds if $h_2$ extends $h_1$ by one measurement step with outcomes compatible with quantum mechanics. Since measurements always yield some result, $R$ is serial.

Applying $\DCw$ to this serial relation constructs an infinite measurement sequence $(a_n, b_n)_{n \in \mathbb{N}}$, enabling statistical analysis of the measurement uncertainty.

\textbf{Comparison to Paper 4:} This parallels the $\DCw$ requirement for extracting classical points from locale spectra (Paper 4, S2). Both involve making infinitely many dependent choices to extract definite classical information from quantum/geometric structures.
\end{prfsketch}

\begin{proof}[Lean Upper Bound]
Formalized as \lean{HUP\_M\_ProfileUpper} in \texttt{HUP/MeasurementUncertainty.lean} using the \lean{HasDCω} token.
\end{proof}

\noindent\textbf{Profile:} $h^{\to}(\text{HUP-M}) \leq \DCwonly$ (Upper bound $\{\DCw\}$).

% ===========================================================
\section{Implementation Strategy and Lean Architecture}
% ===========================================================

\subsection{File Structure}

The Paper 6 implementation follows the mathlib-free pattern established in Paper 4:

\begin{itemize}
\item \texttt{Papers/P6\_Heisenberg/}
  \begin{itemize}
  \item \texttt{Axioms/Complex.lean} -- Minimal complex numbers
  \item \texttt{Axioms/InnerProduct.lean} -- Inner product space axioms  
  \item \texttt{Axioms/Operators.lean} -- Self-adjoint operators, commutators
  \item \texttt{HUP/Basic.lean} -- Foundation tokens and witness families
  \item \texttt{HUP/RobertsonSchrodinger.lean} -- HUP-RS height-0 proof
  \item \texttt{HUP/MeasurementUncertainty.lean} -- HUP-M upper bound
  \item \texttt{Smoke.lean} -- Aggregator for CI testing
  \end{itemize}
\end{itemize}

\subsection{Key Lean Interfaces}

\begin{verbatim}
-- Axioms/InnerProduct.lean
class InnerProductSpace (H : Type*) [AddCommGroup H] [Module ℂ H] where
  inner : H → H → ℂ  
  inner_add_left : ∀ x y z, inner (x + y) z = inner x z + inner y z
  inner_conj_symm : ∀ x y, inner x y = conj (inner y x)  
  inner_pos_def : ∀ x, 0 ≤ (inner x x).re ∧ (inner x x = 0 ↔ x = 0)

-- HUP/Basic.lean  
class HasWLPO (F : Foundation) : Prop
class HasFT (F : Foundation) : Prop  
class HasDCω (F : Foundation) : Prop

def HUP_RS_W : WitnessFamily := fun F => 
  ∀ (H : Type*) [InnerProductSpace H] (A B : H →ₗ H) (ψ : H),
    ‖ψ‖ = 1 → IsSelfAdjoint A → IsSelfAdjoint B →
    stddev A ψ * stddev B ψ ≥ (1/2) * |⟨commutator A B ψ, ψ⟩|

def HUP_M_W : WitnessFamily := fun F =>
  HasDCω F → ∀ (measurement_protocol : InfiniteSequence), 
    statistical_uncertainty_relation_holds measurement_protocol
\end{verbatim}

\subsection{Calibration Certificates}

\begin{verbatim}
-- HUP/RobertsonSchrodinger.lean
def HUP_RS_ProfileUpper : ProfileUpper all_zero HUP_RS_W :=
  ProfileUpper.height_zero (fun F _ => 
    -- Constructive proof using only Cauchy-Schwarz
    sorry_free_proof_of_robertson_schrodinger)

-- HUP/MeasurementUncertainty.lean  
def HUP_M_ProfileUpper : ProfileUpper DCω_only HUP_M_W :=
  ProfileUpper.upper_bound (fun F hDC => 
    -- Use DCω to construct infinite measurement sequence
    dependent_choice_infinite_sequence hDC)
\end{verbatim}

% ===========================================================
\section{Physics Readout and Calibration Summary}
% ===========================================================

\begin{center}
\begin{tabular}{|l|l|l|l|}
\hline
\textbf{Component} & \textbf{Profile} & \textbf{Logical Cost} & \textbf{Physical Interpretation} \\
\hline
HUP-RS & $\allzero$ & None (Height 0) & State preparation limits are geometric \\
(Preparation) & & & properties of Hilbert space \\
\hline  
HUP-M & $\leq \DCwonly$ & $\DCw$ (Upper bound) & Measurement analysis requires \\
(Measurement) & & & choice principles for classical extraction \\
\hline
\end{tabular}
\end{center}

\subsection{Foundational Implications}

The calibration reveals a fundamental distinction in quantum mechanics:

\begin{enumerate}
\item \textbf{Preparation Uncertainty (HUP-RS):} The mathematical constraints on quantum state preparation are fully constructive. They follow from the geometric structure of Hilbert space and require no non-constructive principles. This places uncertainty relations in the computable core of quantum mechanics.

\item \textbf{Measurement Uncertainty (HUP-M):} The statistical analysis of measurement disturbance, when formalized through infinite sequential protocols, requires dependent choice principles. This reflects the general pattern that extracting classical information from quantum processes incurs logical costs.
\end{enumerate}

\subsection{Comparison to Papers 3A and 4}

This result fits the broader AxCal landscape:
\begin{itemize}
\item Like Paper 4's S0/S4 (compact approximations, QHO), HUP-RS is fully constructive (Height 0)
\item Like Paper 4's S2 (locale spatiality), HUP-M requires $\DCw$ to extract classical sequences
\item The pattern reinforces that quantum \emph{structure} is often constructive while classical \emph{extraction} incurs choice costs
\end{itemize}

% ===========================================================
\section{Related Work and Future Directions}
% ===========================================================

\paragraph{Constructive Quantum Mechanics.}
Bishop-Bridges \cite{BishopBridges} developed constructive functional analysis including basic Hilbert space theory. Our work extends this to foundational analysis of quantum uncertainty principles.

\paragraph{Quantum Uncertainty Relations.}
Busch-Lahti-Werner \cite{BuschLahtiWerner} provide modern treatments of various uncertainty relations. Our contribution is the precise axiomatic calibration distinguishing preparation from measurement uncertainty.

\paragraph{Future Calibrations.}
Several quantum mechanical principles await AxCal analysis:
\begin{itemize}
\item \textbf{Bell's inequalities:} Likely requiring choice principles for statistical analysis
\item \textbf{Quantum entanglement:} May involve independence assumptions across orthogonal axes  
\item \textbf{Measurement problem:} Could reveal deep connections to constructive analysis of stochastic processes
\end{itemize}

% ===========================================================
\section{Conclusion}
% ===========================================================

The AxCal framework successfully calibrates the Heisenberg Uncertainty Principle by distinguishing preparation uncertainty (HUP-RS) from measurement uncertainty (HUP-M). HUP-RS emerges as fully constructive (Height 0), demonstrating that the fundamental limits on state preparation are geometric properties accessible within constructive mathematics. HUP-M, when analyzed through infinite measurement sequences, requires dependent choice ($\DCw$) to extract the classical information needed for statistical verification.

This calibration clarifies a foundational distinction in quantum mechanics: the inherent quantum structure (Hilbert space geometry, operator algebra) versus the classical extraction processes (measurement sequences, statistical analysis). The former is constructively accessible while the latter incurs logical costs characteristic of choice principles.

The mathlib-free Lean implementation demonstrates the feasibility of foundational quantum mechanical analysis within minimal axiomatic frameworks, opening paths for further AxCal investigations of quantum theory.

\bigskip

\section*{Acknowledgments}
Development assistance provided by: Gemini 2.5 Deep Think (architecture exploration and theoretical framework design), GPT-5 Pro (Lean 4 scaffolding and implementation support), and Claude Code (repository management and development workflow).

\bibliographystyle{abbrv}
\begin{thebibliography}{99}

\bibitem{Paper3A}
P.~C.-K.~Lee.
\newblock Axiom Calibration via Non-Uniformizability: A Framework for Orthogonal Logical Dependencies in Analysis.
\newblock (Paper 3A, revised with Paper 3C integration), 2025.

\bibitem{Paper4}
P.~C.-K.~Lee.
\newblock Axiom Calibration for Quantum Spectra: Orthogonal Heights, Choice Principles, and Separation Portals.
\newblock (Paper 4, Lean 4 verified), 2025.

\bibitem{BishopBridges}
E.~Bishop and D.~S.~Bridges.
\newblock {\em Constructive Analysis}.
\newblock Springer, 1985.

\bibitem{BuschLahtiWerner}
P.~Busch, P.~Lahti, and R.~F.~Werner.
\newblock Quantum root-mean-square error and measurement uncertainty relations.
\newblock {\em Rev. Mod. Phys.} 86:1261, 2014.

\bibitem{Robertson29}
H.~P.~Robertson.
\newblock The uncertainty principle.
\newblock {\em Phys. Rev.} 34:163--164, 1929.

\bibitem{Schrodinger30}  
E.~Schrödinger.
\newblock Zum Heisenbergschen Unschärfeprinzip.
\newblock {\em Ber. Kgl. Akad. Wiss. Berlin} 24:296--303, 1930.

\end{thebibliography}

\end{document}
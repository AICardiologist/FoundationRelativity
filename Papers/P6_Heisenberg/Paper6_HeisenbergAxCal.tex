\documentclass[11pt]{article}

% -------------------------------------------------
% Preamble (AxCal setup, aligned with Paper 6 tree)
% -------------------------------------------------
\usepackage[T1]{fontenc}
\usepackage[utf8]{inputenc}
\usepackage{lmodern}
\usepackage{geometry}
\geometry{margin=1in}
\usepackage{amsmath,amssymb,amsthm}
\usepackage{mathtools}
\usepackage{booktabs}
\usepackage{microtype}
\usepackage{hyperref}
\hypersetup{colorlinks=true,linkcolor=blue,citecolor=blue,urlcolor=blue}
\usepackage{listings}
\lstset{
  basicstyle=\ttfamily\small,
  columns=fullflexible,
  frame=single,
  breaklines=true,
  keepspaces=true
}

% -------------------------------------------------
% AxCal macros (reused and extended)
% -------------------------------------------------
\newcommand{\WLPO}{\mathsf{WLPO}}
\newcommand{\FT}{\mathsf{FT}}
\newcommand{\DCw}{\mathsf{DC}_{\omega}}
\newcommand{\MP}{\mathsf{MP}}
\newcommand{\BISH}{\mathsf{BISH}}
\newcommand{\SigmaZero}{\Sigma_0}

% Height triple pretty-print
\newcommand{\hzero}{\mathbf{0}}
\newcommand{\hone}{\mathbf{1}}
\newcommand{\homega}{\boldsymbol{\omega}}
\newcommand{\allzero}{\langle \hzero,\hzero,\hzero\rangle}
\newcommand{\DCwonly}{\langle \hzero,\hzero,\hone\rangle}

% Lean shortcuts
\newcommand{\lean}[1]{\texttt{#1}}
\newcommand{\leanok}{\text{\tiny [✓ Lean Verified]}}

% Physics/macros
\newcommand{\R}{\mathbb{R}}
\newcommand{\C}{\mathbb{C}}
\newcommand{\Hil}{\mathcal{H}}
\newcommand{\ket}[1]{|#1\rangle}
\newcommand{\bra}[1]{\langle#1|}
\newcommand{\braket}[2]{\langle#1|#2\rangle}
\newcommand{\ip}[2]{\langle #1, #2 \rangle}
\newcommand{\E}[1]{\langle #1 \rangle}
\newcommand{\comm}[2]{[#1, #2]}
\newcommand{\anticomm}[2]{\{#1, #2\}}
\newcommand{\acomm}[2]{\{#1, #2\}} % synonym used in Lean-facing text
\newcommand{\stddev}{\sigma}
\newcommand{\absC}[1]{\left| #1 \right|}
\newcommand{\abssq}[1]{\absC{#1}^{2}}
\newcommand{\norm}[1]{\left\lVert #1 \right\rVert}
\newcommand{\Var}{\mathrm{Var}}

% Theorem environments
\theoremstyle{plain}
\newtheorem{theorem}{Theorem}[section]
\newtheorem{proposition}[theorem]{Proposition}
\newtheorem{lemma}[theorem]{Lemma}
\newtheorem{corollary}[theorem]{Corollary}

\theoremstyle{definition}
\newtheorem{definition}[theorem]{Definition}
\newtheorem{example}[theorem]{Example}

\theoremstyle{remark}
\newtheorem{remark}[theorem]{Remark}

% -------------------------------------------------
% Title
% -------------------------------------------------
\title{Axiom Calibration for the Heisenberg Uncertainty Principle (Paper 6):\\
Preparation vs.\ Measurement, Constructive RS \& Schrödinger Inequalities}
\author{Paul Chun--Kit Lee\\
\texttt{dr.paul.c.lee@gmail.com}\\
New York University, NY}
\date{September 2025}

\begin{document}
\maketitle

\begin{abstract}
We apply the Axiom Calibration (AxCal) framework to the Heisenberg Uncertainty Principle (HUP) and separate two notions: (i) \emph{preparation} uncertainty (intrinsic state geometry) and (ii) \emph{measurement} uncertainty (sequential disturbance).
Over the constructive base $\BISH$, we mechanize---in Lean 4, mathlib-free---the \emph{division-free squared} Robertson--Schrödinger inequality and its \emph{Schrödinger strengthening} (with anti-commutator term), both at AxCal height $\allzero$.
For the sequential measurement track (HUP--M), we provide a foundation-scoped upper bound by Dependent Choice, $\DCwonly$, via a serial-relation stream construction. The result clarifies which parts of HUP belong to constructive Hilbert geometry and which reflect the logical cost of extracting classical data from infinite sequential experiments.
\end{abstract}

\tableofcontents

\noindent\fbox{\parbox{\textwidth}{
\textbf{Reproducibility (Lean 4 / Lake).}
\begin{itemize}
\item \textbf{Zenodo Software:} \url{https://zenodo.org/records/17068179}
\item \textbf{Repository:} \url{https://github.com/AICardiologist/FoundationRelativity}
\item \textbf{Paper 6 Directory:} \texttt{Papers/P6\_Heisenberg/}
\item \textbf{Build target:} \texttt{lake build Papers.P6\_Heisenberg.Main}
\item \textbf{No-sorry guard:} \texttt{./scripts/no\_sorry\_p6.sh}
\item \textbf{Constraint:} mathlib-free; signatures and bridges are Prop-only, with no typeclass codegen.
\end{itemize}
}}

% ===========================================================
\section{Introduction and Guide to Reading}
% ===========================================================

The Heisenberg Uncertainty Principle (HUP) packages two ideas with different logical flavors:
\begin{enumerate}
\item \textbf{Preparation uncertainty (HUP--RS).} State geometry constrains simultaneous sharpness of non-commuting observables. This is expressed by the Robertson--Schrödinger inequality~\cite{Robertson1929,Schrodinger1930}.
\item \textbf{Measurement uncertainty (HUP--M).} Sequential measurement and state update introduce disturbance; formal analyses track distributions across infinitely many trials~\cite{BuschLahtiWerner2014,Ozawa2003}.
\end{enumerate}
We calibrate these under AxCal (see Papers 3A, 4~\cite{Paper3A,Paper4}) over $\BISH$. The main outcomes are:
\begin{itemize}
\item \textbf{Height~0 (constructive) theorems.} We formalize and prove in Lean the \emph{division-free squared} Robertson--Schrödinger bound and the \emph{Schrödinger} two-term strengthening (commutator plus anti-commutator).
\item \textbf{DC$\boldsymbol{\omega}$ upper bound for HUP--M.} The infinite-history stream needed for sequential experiments uses $\DCw$ as a foundation-scoped token, giving an AxCal upper bound $\le \DCwonly$.
\end{itemize}

\paragraph{Organization.}
Section~\ref{sec:related} reviews prior work in uncertainty relations, formalization efforts, and axiom‑dependency studies.
Section~\ref{sec:signatures-bridges} summarizes the signatures and bridges used by the constructive proofs.
Sections~\ref{sec:RS}--\ref{sec:Schrodinger} prove the RS and Schrödinger inequalities in squared, division-free form.
Section~\ref{sec:measurement-DC} treats the DC$\omega$ bound for sequential measurement.
Section~\ref{sec:engineering} discusses Lean formalization strategies, AxCal architecture reuse from the spectral/quantum track, and a retrospective on early prototypes.
A reproducibility ledger and a Lean--to--LaTeX crosswalk appear at the end.

% ===========================================================
\section{Background and Related Work}
\label{sec:related}
% ===========================================================

\paragraph{Uncertainty inequalities.}
The modern preparation‑uncertainty landscape begins with Robertson~\cite{Robertson1929} and Schrödinger~\cite{Schrodinger1930}; standard functional‑analytic treatments include Reed--Simon~\cite{ReedSimonI}.
Measurement‑disturbance relations were refined in the last two decades, notably by Ozawa~\cite{Ozawa2003} and by Busch--Lahti--Werner~\cite{BuschLahtiWerner2014}.
Our treatment isolates the \emph{preparation} inequalities as constructive Hilbert geometry (Height~0) and places the sequential‑measurement stream in the $\DCw$ stratum.

\paragraph{Formalization of quantum theory.}
Several efforts formalize aspects of quantum theory:
circuit‑level languages and reasoning in Coq (e.g., QWIRE~\cite{QWIRE2017}),
quantum program logics and refinement calculi (see Ying's monograph~\cite{Ying2016}).
These typically work in finite‑dimensional settings or with intermediate denotational layers tailored to programs and circuits.
Our focus is different: we work \emph{analytically} directly at Hilbert‑space signature level, mathlib‑free, and separate constructive bridges (centered vectors, expectation identities) from axiom‑height accounting in AxCal.

\paragraph{Axiom‑dependency and constructive analysis.}
The practice of tracking logical strength via \emph{which} principles are needed is classical in constructive analysis~\cite{BishopBridges,BridgesRichman,TroelstraVanDalen}.
Weak forms of excluded middle (e.g.\ $\WLPO$), compactness principles (e.g.\ the Fan Theorem, $\FT$), and choice schemata (e.g.\ $\DCw$) play distinct roles.
Our calibration uses only a minimal real order (for chaining inequalities), Cauchy--Schwarz, and precise complex‑norm identities for Height~0 results, and \emph{foundation‑scoped} $\DCw$ for the measurement stream.
For background on choice principles and their consequences, see Howard--Rubin~\cite{HowardRubin}.

\paragraph{Positioning.}
To our knowledge, a mathlib‑free constructive formalization of the \emph{squared division‑free} RS and Schrödinger inequalities at Height~0, with a separate DC$\omega$ upper bound for measurement streams, has not previously appeared in a single mechanized account.
This paper aims to serve as a compact reference that cleanly separates (constructive) geometry from (choice‑driven) sampling.

% ===========================================================
\section{Signatures and Bridge Lemmas}
\label{sec:signatures-bridges}
% ===========================================================

We work with an abstract complex Hilbert signature $S$ (inner product $\ip{\cdot}{\cdot}$, norm $\|\cdot\|$) and an observable signature $O$ (self-adjointness, complex/real expectations, commutator/anti-commutator). The Lean files are
\begin{center}
\texttt{HUP/HilbertSig.lean},\quad
\texttt{Axioms/Complex.lean}.
\end{center}
The constructive proofs below use only the following bridges (each provided as Prop-level lemmas in Lean).%
\footnote{See the ``Ledger'' table in \S\ref{sec:ledger}. All bridges appear as named lemmas with Lean anchors indicated in small caps below.}

\begin{description}
  \item[Centered vectors.] For observable $A$ and state $\psi$, define
  $\Delta A\psi := A\psi - \E{A}_\psi\,\psi$ (and similarly for $B$). \hfill{\small(\lean{center})}
  \item[Variance identity.] $\Var_\psi(A) = \|\Delta A\psi\|^2$. \hfill{\small(\lean{variance\_centered})}
  \item[Skew identity.] $\E{\comm{A}{B}}_\psi = \ip{\Delta A\psi}{\Delta B\psi} - \overline{\ip{\Delta A\psi}{\Delta B\psi}}$. \hfill{\small(\lean{comm\_expect\_as\_skew\_centered})}
  \item[Symmetric identity.] $\E{\acomm{\Delta A}{\Delta B}}_\psi = \ip{\Delta A\psi}{\Delta B\psi} + \overline{\ip{\Delta A\psi}{\Delta B\psi}}$. \hfill{\small(\lean{acomm\_expect\_as\_sym\_centered})}
  \item[Exact complex identity.] For $z\in\C$,
  \[
    \abssq{z-\bar z} + \abssq{z+\bar z} = 4\,\abssq{z}.
  \]
  \hfill{\small(\lean{norm\_sq\_skew\_sym\_sum\_eq\_4\_norm\_sq})}
  \item[Cauchy--Schwarz (squared).] $\abssq{\ip{x}{y}} \le \|x\|^2\,\|y\|^2$.\hfill{\small(\lean{cauchy\_schwarz\_sq})}
\end{description}

\begin{remark}[Division-free style]
We present all inequalities in squared form with an explicit factor $4$.
The more familiar $\tfrac12$ versions follow by taking square roots and dividing by $2$, outside the division-free core we calibrate constructively.
\end{remark}


% ===========================================================
\section{Robertson--Schrödinger (squared, constructive)}
\label{sec:RS}
% ===========================================================

\begin{theorem}[Robertson--Schrödinger, division-free squared form]\leanok
\label{thm:RS-squared}
Let $A,B$ be self-adjoint and $\psi$ normalized. Then
\[
  \abssq{\E{\comm{A}{B}}_\psi} \;\le\; 4\,\Var_\psi(A)\,\Var_\psi(B).
\]
\emph{Lean anchor:} \lean{Papers.P6\_Heisenberg.HUP.RobertsonSchrodinger.RS\_from\_bridges}.
\end{theorem}

\begin{proof}
Let $\Delta A\psi$ and $\Delta B\psi$ be the centered vectors, and set $z:=\ip{\Delta A\psi}{\Delta B\psi}$.
By the skew identity (\lean{comm\_expect\_as\_skew\_centered}),
\[
\E{\comm{A}{B}}_\psi \;=\; z-\bar z \quad\Longrightarrow\quad \abssq{\E{\comm{A}{B}}_\psi} = \abssq{z-\bar z}.
\]
Using the inequality component of the exact complex identity,
\[
\abssq{z-\bar z} \;\le\; 4\,\abssq{z}.
\]
By Cauchy--Schwarz (\lean{cauchy\_schwarz\_sq}) and the variance identity (\lean{variance\_centered}),
\[
\abssq{z} \;=\; \abssq{\ip{\Delta A\psi}{\Delta B\psi}}
\;\le\; \norm{\Delta A\psi}^2 \norm{\Delta B\psi}^2
\;=\; \Var_\psi(A)\,\Var_\psi(B).
\]
Multiplying by $4$ and chaining the inequalities yields the claim.
\end{proof}

\begin{remark}[Classical form]
From Theorem~\ref{thm:RS-squared},
$\stddev_A(\psi)\,\stddev_B(\psi)\ge \tfrac12\,\absC{\E{\comm{A}{B}}_\psi}$ follows by square roots and division by $2$.
\end{remark}

% ===========================================================
\section{Schrödinger strengthening (commutator + anti-commutator)}
\label{sec:Schrodinger}
% ===========================================================

\begin{theorem}[Schrödinger inequality, constructive squared form]\leanok
\label{thm:Schrodinger-squared}
Let $A,B$ be self-adjoint and $\psi$ normalized. Then
\[
  \abssq{\E{\comm{A}{B}}_\psi}
  \;+\;
  \abssq{\E{\acomm{\Delta A}{\Delta B}}_\psi}
  \;\le\; 4\,\Var_\psi(A)\,\Var_\psi(B).
\]
\emph{Lean anchor:} \lean{Papers.P6\_Heisenberg.HUP.RobertsonSchrodinger.Schrodinger\_from\_bridges}.
\end{theorem}

\begin{proof}
Set $z:=\ip{\Delta A\psi}{\Delta B\psi}$.
By the skew/symmetric identities,
\[
  \E{\comm{A}{B}}_\psi = z-\bar z, 
  \qquad
  \E{\acomm{\Delta A}{\Delta B}}_\psi = z+\bar z.
\]
Hence the left-hand side is
\[
  \abssq{z-\bar z} + \abssq{z+\bar z}
  \;=\; 4\,\abssq{z}
  \qquad\text{by the exact complex identity.}
\]
Apply Cauchy--Schwarz and the variance identity as in Theorem~\ref{thm:RS-squared}:
$\abssq{z}\le \Var_\psi(A)\,\Var_\psi(B)$, and multiply by $4$.
\end{proof}

\begin{remark}[Centered anti-commutator]
We write $\acomm{\Delta A}{\Delta B}$ to emphasize centering at $\psi$.
The Lean bridge \lean{acomm\_expect\_as\_sym\_centered} returns the expression $z+\bar z$ directly.
\end{remark}

% ===========================================================
\section{Measurement track (HUP--M) and $\DCw$}
\label{sec:measurement-DC}
% ===========================================================

The sequential (disturbance) viewpoint models an experiment producing an infinite history of dependent outcomes.
Let $H_{\mathrm{fin}}$ be the type of finite histories. Define a \emph{serial} relation $R$ by extending a history by one admissible measurement step. Seriality reflects that every measurement yields some outcome.

\begin{theorem}[Serial DC$\omega$ stream]\leanok
\label{thm:dcomega}
Assume a foundation $F$ with token $[\mathrm{Has}\DCw\,F]$.
From any seed in a serial relation there exists an infinite $R$-chain. In particular, the history relation above admits an infinite measurement stream.
\emph{Lean anchors:} \lean{dcω\_stream} (axiom) and \lean{hupM\_stream\_from\_dcω} (derivation, \texttt{HUP/Witnesses.lean}).
\end{theorem}

\begin{corollary}[Calibration for HUP--M]\leanok
\label{cor:HUPM}
There is a witness family for measurement uncertainty whose AxCal height is bounded by $\DCwonly$.
\emph{Lean anchor:} \lean{HUP\_M\_W} and its profile certificate in \texttt{HUP/Witnesses.lean}.
\end{corollary}

\begin{remark}
The $\DCw$ cost reflects extraction of a definite infinite classical sample path from a process with history-dependent choices.
This separates the geometric (Height~0) content of preparation from the choice‑centric content of sequential measurement.
\end{remark}

% ===========================================================
\section{Engineering Notes: Lean Strategies, Architecture, and Retrospective}
\label{sec:engineering}
% ===========================================================

\paragraph{AxCal architecture reused from spectral/quantum track.}
We reuse the \emph{AxCal Core} built for spectral geometry (Paper~4) to keep the HUP analysis modular:
\begin{itemize}
  \item \emph{Foundations and profiles.} \lean{AxCalCore} provides the axiom‑height lattice (Height~0, finite, $\omega$), profile records (\lean{ProfileUpper}), and witness families (\lean{WitnessFamily}).
  \item \emph{Scoping of $\DCw$.} The token \lean{[HasDCω F]} is indexed by a foundation $F$, so uses of dependent choice remain foundation‑scoped. This avoids "free" availability of $\DCw$.
  \item \emph{Prop‑only signatures.} All operators/inner‑product facts are Prop‑level fields/axioms to avoid code generation and keep the library mathlib‑free.
\end{itemize}

\paragraph{Formalization strategies.}
\begin{enumerate}
  \item \textbf{Division‑free squared proofs.} RS and Schrödinger are proved in squared form with a hard‑coded factor $4$ (\texttt{Axioms/Complex.lean}), removing division and square roots from the core proof.
  \item \textbf{Centering bridge.} Algebraic identities are expressed via \lean{center}, \lean{comm\_expect\_as\_skew\_centered}, \lean{acomm\_expect\_as\_sym\_centered}; we work at inner‑product level rather than general operator algebra.
  \item \textbf{Minimal real order.} Only order facts needed for chaining and monotonicity under scaling by $4$ are assumed.
  \item \textbf{Serial DC$\omega$ eliminator.} \lean{dcω\_stream} plus the history serial relation yields the measurement stream with $\DCw$ as an explicit token.
\end{enumerate}

\paragraph{Retrospective on the prior prototype (and fixes).}
Early drafts exposed several issues now corrected:
\begin{itemize}
  \item \textbf{Unscoped $\DCw$.} Dependent Choice was accidentally usable "for free." We fixed this by introducing a \emph{foundation‑scoped} class \lean{[HasDCω F]} and requiring it everywhere $\DCw$ is used.
  \item \textbf{Expectation typing.} Conflating complex and real expectations complicated equalities and coercions. Splitting into \lean{cexpect} (always complex) and \lean{expect} (real, for self‑adjoints) stabilized the bridges and proofs.
  \item \textbf{Axiom ledger vs.\ sorrys.} All provisional \texttt{sorry} were replaced by named, documented bridges; these were then discharged in the Phase‑3 proofs. The \texttt{no\_sorry} guard enforces this across the tree.
  \item \textbf{Witness family shape.} \lean{HUP\_M\_W} now quantifies over a foundation and explicitly requires \lean{[HasDCω F]} to produce an infinite history, matching the AxCal profile statement.
\end{itemize}

\paragraph{Axiom‑dependency ledger (informal DAG).}
\begin{center}
\begin{tabular}{@{}lll@{}}
\toprule
\textbf{Result} & \textbf{Depends on} & \textbf{Height} \\
\midrule
RS (squared) & Cauchy--Schwarz, variance identity, skew identity, complex sum rule &
$\hzero$ \\
Schrödinger (squared) & RS deps + symmetric identity (centered) & $\hzero$ \\
Measurement stream & seriality of histories + $\DCw$ eliminator & $\le \homega$ \\
\bottomrule
\end{tabular}
\end{center}

% ===========================================================
\section{Reproducibility Ledger and File Map}
\label{sec:ledger}
% ===========================================================

\begin{center}
\begin{tabular}{ll}
\toprule
\textbf{Lean artifact} & \textbf{Purpose in paper} \\
\midrule
\verb|Axioms/Complex.lean| & Complex axioms; real order; norm identities (\lean{norm\_sq\_skew\_sym\_sum\_eq\_4\_norm\_sq}) \\
\verb|HUP/HilbertSig.lean| & Signatures and bridges (\lean{center}, \lean{variance\_centered}, \lean{cauchy\_schwarz\_sq}, \\
 & \hspace{1.4em}\lean{comm\_expect\_as\_skew\_centered}, \lean{acomm\_expect\_as\_sym\_centered}) \\
\verb|HUP/RobertsonSchrodinger.lean| & \lean{RS\_from\_bridges}, \lean{Schrodinger\_from\_bridges} (Height~0 theorems) \\
\verb|HUP/DComega.lean| & \lean{dcω\_stream} interface for serial relations \\
\verb|HUP/Witnesses.lean| & \lean{hupM\_stream\_from\_dcω}; \lean{HUP\_M\_W} and $\DCw$ profile \\
\verb|Axioms/Ledger.lean| & Documentary index of assumptions already discharged/derived \\
\bottomrule
\end{tabular}
\end{center}

\paragraph{Build.}
\texttt{lake build Papers.P6\_Heisenberg.Main} compiles the library target; \texttt{./scripts/no\_sorry\_p6.sh} confirms sorry‑free status.

% ===========================================================
\section{Calibration Summary and Physics Readout}
% ===========================================================

\begin{center}
\begin{tabular}{@{}llll@{}}
\toprule
\textbf{Label} & \textbf{Claim} & \textbf{AxCal Profile} & \textbf{Readout} \\
\midrule
HUP--RS & Preparation uncertainty (squared) & $\allzero$ &
Hilbert-space geometry, fully constructive \\
Schrödinger & Two-term strengthening (squared) & $\allzero$ &
Adds symmetric term via centered anti-commutator \\
HUP--M & Sequential measurement stream & $\le \DCwonly$ &
Infinite dependent choices via $\DCw$ token \\
\bottomrule
\end{tabular}
\end{center}

% ===========================================================
\section*{Appendix: Lean--to--LaTeX Crosswalk}
% ===========================================================

\begin{center}
\begin{tabular}{ll}
\toprule
\textbf{Lean symbol} & \textbf{Paper notation} \\
\midrule
\verb|S.inner x y| & $\ip{x}{y}$ \\
\verb|O.cexpect op ψ| & $\E{op}_\psi$ \\
\verb|O.expect op ψ| & $\E{op}_\psi$ (real when $op$ is self-adjoint) \\
\verb|O.variance op ψ| & $\Var_\psi(op)$ \\
\verb|center S O A ψ| & $\Delta A\psi$ \\
\verb|O.comm A B| & $\comm{A}{B}$ \\
\verb|O.acomm A B| & $\acomm{\Delta A}{\Delta B}$ (centered via bridge) \\
\verb|complex_norm_sq z| & $\absC{z}^2$ \\
\verb|RS_from_bridges| & Thm.~\ref{thm:RS-squared} \\
\verb|Schrodinger_from_bridges| & Thm.~\ref{thm:Schrodinger-squared} \\
\bottomrule
\end{tabular}
\end{center}

% -------------------------------------------------
% Bibliography
% -------------------------------------------------
\bibliographystyle{abbrv}
\begin{thebibliography}{99}

\bibitem{Paper3A}
P.~C.-K.~Lee.
\newblock Axiom Calibration for Constructive Mathematics (Paper 3A).
\newblock 2025.

\bibitem{Paper4}
P.~C.-K.~Lee.
\newblock AxCal Framework and Spectral Geometry (Paper 4).
\newblock 2025.

\bibitem{BishopBridges}
E.~Bishop and D.~S.~Bridges.
\newblock \emph{Constructive Analysis}.
\newblock Springer, 1985.

\bibitem{BridgesRichman}
D.~S.~Bridges and F.~Richman.
\newblock \emph{Varieties of Constructive Mathematics}.
\newblock Cambridge University Press, 1987.

\bibitem{TroelstraVanDalen}
A.~S.~Troelstra and D.~van~Dalen.
\newblock \emph{Constructivism in Mathematics: An Introduction}.
\newblock North-Holland, 1988.

\bibitem{HowardRubin}
P.~Howard and J.~E.~Rubin.
\newblock \emph{Consequences of the Axiom of Choice}.
\newblock American Mathematical Society, 1998.

\bibitem{Heisenberg1927}
W.~Heisenberg.
\newblock \"Uber den anschaulichen Inhalt der quantentheoretischen Kinematik und Mechanik.
\newblock \emph{Z. Phys.}, 43:172--198, 1927.

\bibitem{Robertson1929}
H.~P.~Robertson.
\newblock The Uncertainty Principle.
\newblock \emph{Phys. Rev.}, 34:163--164, 1929.

\bibitem{Schrodinger1930}
E.~Schr\"odinger.
\newblock Zum Heisenbergschen Unsch\"arfeprinzip.
\newblock \emph{Sitzungsber. Preuss. Akad. Wiss.}, Phys.-Math. Kl., 19:296--303, 1930.

\bibitem{ReedSimonI}
M.~Reed and B.~Simon.
\newblock \emph{Methods of Modern Mathematical Physics I: Functional Analysis}.
\newblock Academic Press, 1980.

\bibitem{Ozawa2003}
M.~Ozawa.
\newblock Universally valid reformulation of the Heisenberg uncertainty principle on noise and disturbance in measurement.
\newblock \emph{Phys. Rev. A}, 2003.

\bibitem{BuschLahtiWerner2014}
P.~Busch, P.~Lahti, and R.~F.~Werner.
\newblock Colloquium: Quantum root-mean-square error and measurement uncertainty relations.
\newblock \emph{Rev. Mod. Phys.} 86:1261--1281, 2014.

\bibitem{Ying2016}
M.~Ying.
\newblock \emph{Foundations of Quantum Programming}.
\newblock Morgan Kaufmann, 2016.

\bibitem{QWIRE2017}
J.~Paykin, R.~Rand, and S.~Zdancewic.
\newblock QWIRE: A Formalized Quantum Circuit Language.
\newblock In \emph{POPL}, 2017.

\end{thebibliography}

\end{document}
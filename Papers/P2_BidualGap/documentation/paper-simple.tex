\documentclass{article}
\usepackage{amsmath}
\usepackage{amssymb}
\usepackage{url}

\title{WLPO $\Leftrightarrow$ Bidual Gap: A Constructive Equivalence in Lean 4}
\author{Paul Chun-Kit Lee}
\date{August 2025}

\begin{document}
\maketitle

\begin{abstract}
We formalize in Lean 4 the constructive equivalence between the Weak Limited Principle of Omniscience (WLPO) and the existence of a Banach space with a bidual gap. This work demonstrates foundation-relativity: mathematical pathologies that are guaranteed under classical foundations become equivalent to constructive choice principles.

\textbf{Reproducible artifact}: DOI: \texttt{10.5281/zenodo.13356587}
\end{abstract}

\section{Introduction}

The relationship between constructive mathematics and functional analysis reveals surprising connections between logical principles and analytical pathologies. We prove:

\begin{theorem}
WLPO $\Leftrightarrow$ $\exists X$ Banach space such that the canonical embedding $J: X \to X^{**}$ is not surjective.
\end{theorem}

\section{Main Results}

\subsection{Forward Direction: Gap $\Rightarrow$ WLPO}

Given a Banach space $X$ with bidual gap witness $y \in X^{**} \setminus J(X)$, we construct a decision procedure for WLPO using the uniform gap $\delta = \|y\|/2$.

\subsection{Reverse Direction: WLPO $\Rightarrow$ Gap}

Under WLPO, we explicitly construct $G \in c_0^{**}$ by $G(f) = \sum_n f(e_n)$ where $\{e_n\}$ is the standard basis. This functional cannot be represented by any element of $c_0$.

\section{Lean 4 Formalization}

Our formalization achieves:
\begin{itemize}
\item Complete dual isometry $(c_0 \to \mathbb{R}) \cong_{\ell_i} \ell^1$ with 3 WLPO sorries
\item Zero-sorry Option-B architecture for modular gap construction
\item CI-verified axiom hygiene with fortress protection
\end{itemize}

\section{Reproducibility}

The complete Lean 4 formalization is available at:
\begin{itemize}
\item Repository: \url{https://github.com/AICardiologist/FoundationRelativity}
\item Release tag: \texttt{p2-minimal-v0.1}
\item DOI: \texttt{10.5281/zenodo.13356587}
\end{itemize}

Build instructions:
\begin{verbatim}
git clone https://github.com/AICardiologist/FoundationRelativity
cd FoundationRelativity
git checkout p2-minimal-v0.1
lake build Papers.P2_BidualGap.P2_Minimal
\end{verbatim}

\section{Conclusion}

This work demonstrates that the existence of non-reflexive Banach spaces, a phenomenon guaranteed classically, becomes constructively equivalent to WLPO. This exemplifies foundation-relativity: mathematical truths that depend critically on the underlying logical framework.

\bibliographystyle{plain}
\begin{thebibliography}{1}

\bibitem{ishihara2006}
H. Ishihara. Weak K\"{o}nig's lemma implies Brouwer's fan theorem: A direct proof. \textit{Notre Dame J. Formal Logic}, 47(2):249--252, 2006.

\bibitem{bridges2007}
D. Bridges and F. Richman. \textit{Varieties of Constructive Mathematics}. Cambridge University Press, 2007.

\end{thebibliography}

\end{document}
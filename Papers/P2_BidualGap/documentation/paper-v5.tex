
\documentclass[11pt]{article}

% ------------------------------------------------------------
% Standard LaTeX packages
% ------------------------------------------------------------
\usepackage[margin=1in]{geometry}
\usepackage{lmodern}
\usepackage{amsmath,amssymb,mathtools}
\usepackage[american]{babel}
\usepackage{stmaryrd}
\usepackage{enumitem}
\usepackage{booktabs}
\usepackage{tikz}
\usetikzlibrary{arrows.meta,positioning,cd}
\usepackage{listings}
\usepackage[x11names,table]{xcolor}
\usepackage{graphicx}
\usepackage{array}
\usepackage{mdframed}
\usepackage{url}
% hyperref should typically come last
\usepackage[colorlinks=true,linkcolor=blue,citecolor=blue,urlcolor=blue]{hyperref}

% Define theorem-like environments
\newtheorem{theorem}{Theorem}[section]
\newtheorem{lemma}[theorem]{Lemma}
\newtheorem{corollary}[theorem]{Corollary}
\newtheorem{definition}[theorem]{Definition}
\newtheorem{remark}[theorem]{Remark}
% Alias used in body:
\newenvironment{defi}{\begin{definition}}{\end{definition}}

% ---------- Lean repo link ----------
\newcommand{\leanRepo}{\url{https://github.com/AICardiologist/FoundationRelativity}}
\newcommand{\leanok}{\textsf{\small \textcolor{green!70!black}{✓}}}
\newcommand{\leanpartial}{\textsf{\small \textcolor{orange!80!black}{◐}}}

% ---------- Mathematical notation ----------
\newcommand{\N}{\mathbb{N}}
\newcommand{\R}{\mathbb{R}}
\newcommand{\linf}{\ell^\infty}
\newcommand{\cnull}{c_0}
\newcommand{\WLPO}{\mathrm{WLPO}}
\newcommand{\BISH}{\mathrm{BISH}}
\newcommand{\CRM}{\mathrm{CRM}}
\newcommand{\LEM}{\mathrm{LEM}}
\newcommand{\DC}{\mathrm{DC}}
\newcommand{\ZFC}{\mathrm{ZFC}}

% ---------- Code listing style for Lean ----------
\definecolor{codegreen}{rgb}{0,0.6,0}
\definecolor{codegray}{rgb}{0.5,0.5,0.5}
\definecolor{codepurple}{rgb}{0.58,0,0.82}
\definecolor{backcolour}{rgb}{0.95,0.95,0.92}

\lstdefinelanguage{Lean}{
  keywords={theorem, lemma, def, definition, axiom, structure, class, instance, 
            by, exact, intro, intros, apply, refine, constructor, use, obtain, 
            have, show, from, fun, assume, let, in, if, then, else,
            match, with, end, namespace, section, variable, variables,
            example, begin, sorry, admit, noncomputable, classical,
            import, open, export, private, protected, mutual, meta,
            do, for, while, return, try, catch, finally,
            Type, Prop, Sort, Type*, forall, exists, where, extends},
  sensitive=true,
  morecomment=[l]{--},
  morecomment=[s]{/-}{-/},
  morestring=[b]",
  literate=
    {α}{{$\alpha$}}1 {β}{{$\beta$}}1 {γ}{{$\gamma$}}1
    {δ}{{$\delta$}}1 {ε}{{$\varepsilon$}}1 {ζ}{{$\zeta$}}1
    {η}{{$\eta$}}1 {θ}{{$\theta$}}1 {ι}{{$\iota$}}1
    {κ}{{$\kappa$}}1 {λ}{{$\lambda$}}1 {μ}{{$\mu$}}1
    {ν}{{$\nu$}}1 {ξ}{{$\xi$}}1 {π}{{$\pi$}}1
    {ρ}{{$\rho$}}1 {σ}{{$\sigma$}}1 {τ}{{$\tau$}}1
    {φ}{{$\varphi$}}1 {χ}{{$\chi$}}1 {ψ}{{$\psi$}}1
    {ω}{{$\omega$}}1 {Γ}{{$\Gamma$}}1 {Δ}{{$\Delta$}}1
    {Θ}{{$\Theta$}}1 {Λ}{{$\Lambda$}}1 {Σ}{{$\Sigma$}}1
    {Φ}{{$\Phi$}}1 {Ψ}{{$\Psi$}}1 {Ω}{{$\Omega$}}1
    {→}{{$\rightarrow$}}1 {←}{{$\leftarrow$}}1 {↔}{{$\leftrightarrow$}}1
    {∀}{{$\forall$}}1 {∃}{{$\exists$}}1 {∈}{{$\in$}}1
    {∉}{{$\notin$}}1 {⊆}{{$\subseteq$}}1 {⊂}{{$\subset$}}1
    {∪}{{$\cup$}}1 {∩}{{$\cap$}}1 {≤}{{$\leq$}}1
    {≥}{{$\geq$}}1 {≠}{{$\neq$}}1 {≈}{{$\approx$}}1
    {≡}{{$\equiv$}}1 {∧}{{$\land$}}1 {∨}{{$\lor$}}1
    {¬}{{$\neg$}}1 {ℕ}{{$\mathbb{N}$}}1 {ℝ}{{$\mathbb{R}$}}1
    {ℂ}{{$\mathbb{C}$}}1 {ℤ}{{$\mathbb{Z}$}}1
    {·}{{$\cdot$}}1 {∑}{{$\sum$}}1 {∏}{{$\prod$}}1
    {∅}{{$\emptyset$}}1 {∞}{{$\infty$}}1 {∂}{{$\partial$}}1
}

\lstdefinestyle{leanstyle}{
    language=Lean,
    backgroundcolor=\color{backcolour},    
    commentstyle=\color{codegreen},
    keywordstyle=\color{blue},
    stringstyle=\color{codepurple},
    basicstyle=\ttfamily\footnotesize,
    breakatwhitespace=false,         
    breaklines=true,                 
    captionpos=b,                    
    keepspaces=true,                 
    numbers=left,                    
    numbersep=5pt,                  
    showspaces=false,                
    showstringspaces=false,
    showtabs=false,                  
    tabsize=2,
    numberstyle=\tiny\color{codegray}
}

\lstset{style=leanstyle}

% ---------- Title and author ----------
\title{The Bidual Gap and WLPO: A Constructive Calibration of Banach Space Non-Reflexivity}
\author{Paul Chun-Kit Lee\thanks{Lean 4 formalization available at \leanRepo. Code archive available at Zenodo: DOI \texttt{10.5281/zenodo.17107493}.} \\ 
New York University \\ 
\texttt{dr.paul.c.lee@gmail.com}}
\date{\today}

\begin{document}

\begin{abstract}
We calibrate the logical strength of Banach space non-reflexivity over Bishop-style constructive mathematics (BISH). Writing ``bidual gap'' for the failure of surjectivity of the canonical embedding $J_X\!:X\to X^{**}$, we show---\emph{in a classical metatheory}---that over $\BISH$ the following are equivalent: the Weak Limited Principle of Omniscience (WLPO); the existence of a bidual gap for some real Banach space; and, in particular, the non-surjectivity of $J_{\ell^\infty}$. The forward implication proceeds by classically extracting an Ishihara kernel (a finite tuple of functionals and a positive threshold) from a gap; the kernel-to-$\WLPO$ step is intuitionistic. Both directions are formalized in Lean~4. We also give constructive finite-dimensional surrogates via Cesàro means and a ``Stone window'' identifying idempotents of $\ell^\infty/c_0$ with almost-equality classes of subsets of $\mathbb{N}$. These results sharpen the folklore that the bidual gap is ``non-constructive'' by locating it precisely at $\WLPO$ over $\BISH$.
\end{abstract}

\maketitle

\begin{mdframed}[backgroundcolor=gray!10, linewidth=0pt]
\textbf{IMPORTANT DISCLAIMER}

\textbf{A Case Study: Using Multi-AI Agents to Tackle Formal Mathematics}

This entire Lean 4 formalization project was produced by multi-AI agents working under human direction. All proofs, definitions, and mathematical structures in this repository were AI-generated. This represents a case study in using multi-AI agent systems to tackle complex formal mathematics problems with human guidance on project direction.
\end{mdframed}

\tableofcontents

% ===========================================================
\section{Introduction: Foundation Relativity and Logical Calibration}
% ===========================================================

\subsection{The foundation-sensitive bidual gap}

For any Banach space $X$, the canonical embedding $J_X:X\to X^{**}$ maps $x\in X$ to the evaluation functional $J_X(x)(f)=f(x)$ for $f\in X^*$. A central question in functional analysis is whether $J_X$ is surjective; if not, $X$ is non-reflexive, and there exists a ``bidual gap.''

\begin{definition}[Non-reflexivity and bidual gap]
A Banach space $X$ is \textbf{non-reflexive} if the canonical embedding $J_X$ is not surjective. In this case, we say $X$ \textbf{has a bidual gap}.
\end{definition}

For the space $\linf$ of bounded sequences, the answer depends on the foundational setting. In classical mathematics ($\ZFC$), the Hahn--Banach theorem implies that $J_{\linf}$ is not surjective.

However, in Bishop-style constructive mathematics ($\BISH$)---a system based on intuitionistic logic without the Law of Excluded Middle ($\LEM$) or full Choice---one cannot prove that this gap exists. This dependence of provability on foundations exemplifies \emph{foundation relativity}.

\subsection{Motivation: From independence to precise calibration}

This project began by probing whether functional-analytic ``pathologies'' might mirror independence phenomena. While no direct connection to G\"odelian incompleteness emerged, the Lean 4 formalization led to a precise calibration result: the exact logical strength of the bidual gap.

The bidual gap is not independent in the G\"odelian sense, but its provability over $\BISH$ is exactly that of a specific, weak non-constructive principle. This situates the problem within \emph{Constructive Reverse Mathematics} ($\CRM$), whose goal is to determine minimal axioms needed for classical theorems over constructive bases.

\subsection{The Weak Limited Principle of Omniscience (WLPO)}

\begin{defi}[$\WLPO$]
For any binary sequence $\alpha:\N\to\{0,1\}$,
\[
(\forall n,\ \alpha(n)=0)\ \ \vee\ \ \neg(\forall n,\ \alpha(n)=0).
\]
\end{defi}

$\WLPO$ is strictly weaker than $\LEM$ but not provable in $\BISH$. It captures the minimal decision strength needed to determine whether an infinite sequence is identically zero.

\subsection{Main result and contributions}

\begin{thm}[Main Theorem]
Over $\BISH$, the following are equivalent:
\begin{enumerate}
\item $\WLPO$.
\item \textbf{Gap$_{\linf}$:} The embedding $J_{\linf}:\linf\to(\linf)^{**}$ is not surjective.
\item \textbf{Gap$_{\exists}$:} There exists a real Banach space $X$ such that $J_X:X\to X^{**}$ is not surjective.
\end{enumerate}
Moreover, this equivalence is fully formalized in Lean~4 (see Section~\ref{sec:lean}).
\end{thm}

Our contributions are:
\begin{enumerate}[label=\arabic*.]
\item \textbf{Logical calibration:} an exact equivalence within $\CRM$.
\item \textbf{Formal verification:} a complete Lean 4 formalization of both directions.
\item \textbf{Methodology:} a Prop-level Ishihara kernel and a robust csSup treatment of partial sums.
\item \textbf{Constructive surrogates:} explicit, verified finite-dimensional approximations via Ces\`aro means.
\end{enumerate}

% ===========================================================
\section{Main Mathematical Results: The Equivalence Theorem}
% ===========================================================

We explain the equivalence between the bidual gap and $\WLPO$.

\subsection{Forward direction: Gap implies WLPO}

To bridge a bidual-gap witness and a logical decision, we use an ``Ishihara kernel'', adapted from constructive reverse mathematics.

\begin{defi}[Ishihara kernel]
An Ishihara kernel consists of a normed space $X$, an element $y\in X^{**}\setminus J(X)$, a functional $f\in X^*$, a family $g:(\N\to\{0,1\})\to X^*$, and a constant $\delta>0$ such that for all binary sequences $\alpha$:
\begin{enumerate}
\item $|y(f+g(\alpha))|=0$ or $|y(f+g(\alpha))|\ge \delta$ (dichotomy);
\item $(\forall n,\ \alpha(n)=0)\ \Leftrightarrow\ y(f+g(\alpha))=0$ (decision property).
\end{enumerate}
\end{defi}

\begin{lem}[Kernel $\Rightarrow$ $\WLPO$]
If an Ishihara kernel exists, then $\WLPO$ holds.
\end{lem}

\begin{proof}
Given $\alpha$, compute $s=|y(f+g(\alpha))|$. By dichotomy, either $s=0$ or $s\ge \delta>0$. The decision property equates $s=0$ with $(\forall n,\alpha(n)=0)$, deciding the $\WLPO$ instance.
\end{proof}

\begin{thm}[Gap implies $\WLPO$ (meta-classical)]\label{thm:gap-implies-wlpo}
Working in a classical metatheory: if some Banach space $X$ has $J:X\to X^{**}$ not surjective, 
then $\WLPO$ holds over $\BISH$.
\end{thm}

\begin{proof}
Choose $y\in X^{**}\setminus J(X)$ with $y\ne 0$. A half-norm attainment lemma yields $h\in X^*$ with $\|h\|\le 1$ and $|y(h)|>\|y\|/2$. Let $f=0$, $\delta=|y(h)|/2$ (so $\delta>0$ by $0 \leq \|y\|/2 < |y(h)|$), and
\[
g(\alpha)=\begin{cases}
0 & \text{if } \forall n,\alpha(n)=0,\\
h & \text{otherwise}.
\end{cases}
\]
Then the dichotomy and decision properties hold, so the kernel exists.
\end{proof}

\subsection{Foundation-theoretic interlude: Classical logic in reverse mathematics}

Before proceeding to the reverse direction, we must address a subtle but crucial point about the use of classical logic in constructive reverse mathematics.

\begin{rem}[Classical meta-logic vs. object logic]\label{rem:meta-classical}
The definition of $g(\alpha)$ in the proof case-splits on the undecidable proposition
$(\forall n,\alpha(n)=0)$. In $\BISH$, we cannot constructively perform this case split.
However, in reverse mathematics, we work in a \emph{classical meta-logic} to prove
statements \emph{about} $\BISH$.

Concretely, we prove the implication:
\[
  \text{(classically)}\quad
  \big[\text{there exists a Banach space with a bidual gap (a $\BISH$-provable statement)}\big]
  \;\Longrightarrow\;
  \big[\BISH \vdash \WLPO\big].
\]
This is standard in reverse mathematics: classical reasoning is used only to
\emph{extract} the data that a constructive argument can \emph{consume}.

In the formalization, we fence the meta-classical reasoning (witness extraction and
the case split in $g$) inside a dedicated block, and keep the kernel consumer
(\emph{Ishihara kernel} $\Rightarrow \WLPO$) intuitionistic. No classical axiom
is added to the object theory.
\end{rem}

\paragraph{Special note (differences from an earlier draft).}
Two technical changes make the argument robust and CRM-compliant:
\begin{enumerate}
  \item \textbf{Gap parameter.} We set
  \[
    \delta \;:=\; \frac{|y(h^\star)|}{2}\,,
  \]
  where $h^\star\in X^\ast$ satisfies $\|h^\star\|\le 1$ and
  $\tfrac{\|y\|}{2}<|y(h^\star)|$. This yields $\delta>0$ by elementary
  order facts ($0\le \tfrac{\|y\|}{2} < |y(h^\star)|$ implies $0<|y(h^\star)|$,
  hence $0<\delta$). In particular, we do \emph{not} need to derive $\|y\|>0$
  from $y\neq 0$, which would otherwise force norm-positivity lemmas at the bidual
  type and introduce foundational friction in a constructive setting.
  \item \textbf{Fencing classical reasoning.} The only classical steps are:
  picking $y\in X^{**}\!\setminus j(X)$, obtaining $h^\star$, and
  defining $g(\alpha)$ by a global case split. The kernel consumer
  (from the separation statement and the zero-characterization to $\WLPO$)
  is intuitionistic and does not depend on classical principles.
\end{enumerate}
Both changes are invisible at the level of classical mathematics, but they are
important for constructive reverse mathematics: the consumer no longer relies on
bidual-specific norm facts or undecidable case splits.

\subsection{Reverse direction: WLPO implies gap}

\begin{thm}[$\WLPO$ implies Gap]
If $\WLPO$ holds, then $J:\linf\to(\linf)^{**}$ is not surjective.
\end{thm}

\begin{proof}[Proof sketch]
We reduce a generic instance of $\WLPO$ to separation in $\ell^\infty/c_0$ by a
uniform \emph{coding} of binary sequences. Given $\alpha\in\{0,1\}^{\mathbb N}$,
define a bounded sequence $v^\alpha$ whose membership in $c_0$ is equivalent to
$(\forall n,\alpha(n)=0)$ (e.g.\ by sparse ``pulses'' placed far apart, or via a
bounded alternating prefix with a single switch triggered by the first $1$).
Then $\WLPO$ decides whether $v^\alpha\in c_0$. Using these codes one constructs
an $\mathbb R$-linear functional $\Phi:\ell^\infty\to\mathbb R$ with $\Phi|_{c_0}=0$
and $\Phi(v^\alpha)\neq 0$ for some $\alpha$. The induced functional on the
quotient $\ell^\infty/c_0$ yields $G\in(\ell^\infty)^{**}$ with
$G\not\in J_{\ell^\infty}(\ell^\infty)$, giving the bidual gap.
\end{proof}

\subsection{A structural byproduct}

\begin{thm}[Gap structure]
Assuming $\WLPO$, the quotient $\linf/\cnull$ is nontrivial and carries a Boolean algebra of idempotents with rich structure.
\end{thm}

This connects naturally with the ``Stone window'' theorem in Section~\ref{sec:stone}.

% ===========================================================
\section{Constructive Algorithms: Finite Approximations}
% ===========================================================

Finite-dimensional surrogates clarify the constructive boundary: they are computable and illuminating, but their infinite limit would decide $\WLPO$-type predicates.

\subsection{Ces\`aro mean surrogates}

\begin{defi}[Ces\`aro mean]
For $n\ge 1$, define $f_n:\R^n\to\R$ by $f_n(x)=\frac{1}{n}\sum_{i=1}^n x_i$.
\end{defi}

\begin{thm}[Constructive finite Hahn--Banach]\label{thm:finite-hb}\leanok
With the sup norm on $\R^n$, the average $f_n(x)=\frac{1}{n}\sum_{i=1}^n x_i$ is the unique linear functional of norm $1$ satisfying $f_n(1,\dots,1)=1$ and $\ker(f_n)=M_n:=\{x\in\R^n:\sum_{i=1}^n x_i=0\}$.
\end{thm}

\begin{proof}
If $\|x\|_\infty\le 1$, then $|f_n(x)|\le \frac{1}{n}\sum_{i=1}^n|x_i|\le 1$, with equality at $x=(1,\dots,1)$, so $\|f_n\|=1$. The kernel claim is immediate from linearity and the definition of $M_n$; uniqueness follows because any linear functional with this kernel and value on $(1,\dots,1)$ must agree with $f_n$ on a direct-sum decomposition $\R^n=M_n\oplus \mathrm{span}\{(1,\dots,1)\}$.
\end{proof}

\subsection{Computational implementation}

We provide $O(n)$ implementations (see the repository) to verify Theorem~\ref{thm:finite-hb} numerically.

\subsection{Convergence and the constructive boundary}

\begin{thm}[Non-constructive limit]
Viewing $(f_n)$ as functionals on $\linf$ via projection, any constructive pointwise limit that separates the encodings described below would decide a $\WLPO$-level predicate, and hence does not exist in $\BISH$.
\end{thm}

\begin{proof}[Proof sketch]
Encode a $\WLPO$ instance $\alpha$ by $v^\alpha\in\linf$; in a simple restricted case (at most one $1$) let $v^\alpha_n=(-1)^{\sum_{k\le n}\alpha_k}$. Then $f_n(v^\alpha)$ converges to different limits depending on whether $\alpha$ is all zeros or has a single $1$. A constructive limit would decide that dichotomy.
\end{proof}

% ===========================================================
\section{Formalization in Lean 4}\label{sec:lean}
% ===========================================================

\subsection{Overview and axiom profile}\label{sec:axioms}

Our Lean 4 formalization contains about 4{,}500 lines of verified code.
\begin{itemize}
\item \textbf{Bidirectional proof:} both Gap $\Rightarrow$ $\WLPO$ and $\WLPO$ $\Rightarrow$ Gap.
\item \textbf{Auxiliaries:} near-constructive development of $(\cnull)^*\cong \ell^1$ and the Stone window.
\item \textbf{Axioms:}
\begin{lstlisting}[language=Lean,numbers=none]
#print axioms gap_equiv_wlpo
-- [propext, Classical.choice, Quot.sound]
\end{lstlisting}
Here, \texttt{Quot.sound} underlies quotients like $\linf/\cnull$; \texttt{propext} is used extensively at Prop level; and \texttt{Classical.choice} (with \texttt{open Classical}) provides classical case splits in the meta-logic, as discussed in Remark~\ref{rem:meta-classical}.
\end{itemize}

\subsection{CRM methodology and axiom hygiene}\label{sec:crm-hygiene}

Following standard CRM (Constructive Reverse Mathematics) methodology, our formalization 
separates the \emph{classical producer} from the \emph{constructive consumer}:

\begin{itemize}
\item \textbf{Producer (classical):} The construction of the Ishihara kernel from the bidual gap,
  including extraction of $y \in X^{**} \setminus J(X)$, finding $h^\star$ with $\|y\|/2 < \|y(h^\star)\|$,
  and defining $g(\alpha)$ via case-splitting on the undecidable predicate $(\forall n, \alpha(n) = 0)$.
  This is fenced in \texttt{section ClassicalMeta} in \texttt{Ishihara.lean}.

\item \textbf{Consumer (constructive):} The theorem \texttt{WLPO\_of\_kernel} that derives $\WLPO$ 
  from the abstract kernel properties. This is placed \emph{outside} any \texttt{noncomputable} or 
  \texttt{Classical} sections, ensuring pure intuitionistic reasoning.
\end{itemize}

This separation can be mechanically verified in Lean:
\begin{lstlisting}[language=Lean,numbers=none]
#print axioms WLPO_of_kernel
-- (no classical axioms)

#print axioms WLPO_of_gap  
-- [propext, Classical.choice, Quot.sound]
\end{lstlisting}

The result properly demonstrates that in a classical metatheory, 
$\textsf{BISH} \vdash \textsf{BidualGapStrong} \Rightarrow \WLPO$.

\subsection{Reproducibility information}

\begin{mdframed}[backgroundcolor=gray!10]
\textbf{Reproducibility Box}
\begin{itemize}
\item \textbf{Repository}: \leanRepo
\item \textbf{Lean toolchain}: \texttt{leanprover/lean4:v4.22.0-rc4}
\item \textbf{mathlib4 commit}: \texttt{59e4fba0c656457728c559a7d280903732a6d9d1}
\item \textbf{Project tag}: \texttt{p2-crm-v0.2} / commit \texttt{85f69aa}
\item \textbf{Build}: \texttt{lake exe cache get \&\& lake build} \\
      top-level target: \texttt{Papers.P2\_BidualGap.gap\_equiv\_wlpo}
\item \textbf{Code Archive DOI}: \texttt{10.5281/zenodo.17107493}
\item \textbf{Status}: The main equivalence builds with 0 errors. Three WLPO-conditional lemmas remain in the optional module \texttt{DualIsometriesComplete.lean} concerning completeness of certain dual spaces; they do not affect the main equivalence.
\end{itemize}
\end{mdframed}

\subsection{Key technical solutions}

\paragraph{Prop-level kernel technique.}
We implement the Ishihara kernel entirely at Prop level.

\begin{lstlisting}[caption={Ishihara kernel (illustrative Lean snippet)}]
structure IshiharaKernel where
  -- fields omitted
  kernel_property : ∀ α, |y (f + g α)| = 0 ∨ δ ≤ |y (f + g α)|
  decision_property : ∀ α, (∀ n, α n = false) ↔ y (f + g α) = 0

lemma WLPO_of_kernel (K : IshiharaKernel) : WLPO := by
  intro α
  cases K.kernel_property α with
  | inl h => left; exact (K.decision_property α).mpr h
  | inr h => right; exact mt (K.decision_property α).mp (ne_of_gt h)
\end{lstlisting}

\paragraph{Robust csSup for partial sums.}
We avoid fragile complete-lattice instance resolution by working directly with conditional suprema:
\begin{lstlisting}[caption={tsum equals csSup of finite partial sums}]
private lemma tsum_eq_csSup_sum_of_nonneg
  {ι : Type*} (u : ι → ℝ) (h0 : ∀ i, 0 ≤ u i) (hs : Summable u) :
  (∑' i, u i) = sSup (Set.range (fun s : Finset ι => ∑ i ∈ s, u i)) := by
  have nonempty : (Set.range _).Nonempty := ⟨0, ∅, by simp⟩
  have bdd : BddAbove (Set.range _) := by
    use ∑' i, u i
    rintro x ⟨s, rfl⟩
    exact sum_le_tsum s (fun i _ => h0 i) hs
  apply le_antisymm
  · apply tsum_le_of_sum_le hs; intro s; exact le_csSup bdd ⟨s, rfl⟩
  · apply csSup_le nonempty; rintro x ⟨s, rfl⟩
    exact sum_le_tsum s (fun i _ => h0 i) hs
\end{lstlisting}

\paragraph{HasWLPO architecture.}
A lightweight typeclass separates WLPO-dependent arguments from constructive cores.

\begin{lstlisting}[caption={WLPO typeclass sketch}]
class HasWLPO : Prop :=
  (wlpo : ∀ (α : ℕ → Bool), (∀ n, α n = false) ∨ ¬(∀ n, α n = false))

lemma gap_exists_of_WLPO [HasWLPO] : ...
instance [Classical] : HasWLPO := ⟨fun α => em _⟩
\end{lstlisting}

\subsection{The bidirectional theorem (Lean)}

\begin{lstlisting}[caption={WLPO ↔ Gap (top-level equivalence)}]
theorem gap_equiv_wlpo : BidualGapStrong.{0} ↔ WLPO := by
  constructor
  · intro h_gap
    classical
    -- construct kernel from gap witness and apply WLPO_of_kernel
    exact WLPO_of_kernel (kernel_from_gap h_gap)
  · intro hwlpo
    -- construct gap from HasWLPO instance
    exact gap_from_WLPO hwlpo
\end{lstlisting}

% ===========================================================
\section{Stone Window: The Boolean Algebra Connection}\label{sec:stone}
% ===========================================================

We relate logic and analysis via a concrete Boolean algebra in the quotient.

\subsection{Almost-equality and idempotents}

\begin{defi}[Almost equality]
For $A,B\subseteq\N$, write $A\sim B$ if the symmetric difference $A\triangle B$ is finite.
\end{defi}

\begin{defi}
Write $\mathrm{Idem}(B)$ for the set of idempotents of a (commutative) Banach algebra $B$ under the induced Boolean operations $e\wedge f:=ef$, $e\vee f:=e+f-ef$, and $\neg e:=1-e$.
\end{defi}

\begin{thm}[Stone window]\label{thm:stone}\leanok
Equip $\ell^\infty/c_0$ with the quotient Banach algebra structure induced by pointwise operations. The map
$\Phi:\mathcal{P}(\N)/{\sim}\to \mathrm{Idem}(\linf/\cnull)$,
$[A]\mapsto [\chi_A]$, is a Boolean algebra isomorphism.
\end{thm}

\begin{proof}[Proof sketch]
(1) $[\chi_A]=[\chi_B]$ iff $\chi_A-\chi_B\in \cnull$, i.e.\ $A\sim B$.  
(2) Boolean operations are respected.  
(3) Every idempotent is equivalent (mod $\cnull$) to an indicator via thresholding at $1/2$.
\end{proof}

\begin{cor}[Finite distributive lattices]\leanok
Every finite distributive lattice embeds into $\mathrm{Idem}(\linf/\cnull)$.
\end{cor}

% ===========================================================
\section{Significance and Conclusion}
% ===========================================================

\subsection{Precise logical calibration}

The result places a familiar analytic phenomenon exactly at $\WLPO$, demonstrating how constructive reverse mathematics can sharpen vague ``non-constructive'' labels into precise equivalences.

\subsection{Foundation relativity}

\begin{center}
\small % Make text smaller instead of resizing
\begin{tabular}{lll}
\toprule
Foundation & Statement about the gap ($\exists y\notin \mathrm{im}(J)$) & Comment \\
\midrule
$\ZFC$ & \emph{Holds} & Hahn--Banach yields witnesses \\
$\BISH$ & \emph{Equivalent to $\WLPO$} & (Main Theorem) \\
$\BISH+\DC$ & \emph{Equivalent to $\WLPO$} & $\DC$ does not imply $\WLPO$ \\
$\BISH+\WLPO$ & \emph{Holds} & by equivalence \\
$\BISH+\neg\WLPO$ & \emph{Not provable} & fails in models of $\BISH+\neg\WLPO$ \\
\bottomrule
\end{tabular}
\end{center}

\subsection{The role of formalization}

The Lean development clarified axiom usage, suggested robust approaches to conditional suprema, and yielded reusable patterns (Ishihara kernel, HasWLPO) for future formalized reverse mathematics.

\subsection{Conclusion}

We establish and formalize that detecting the bidual gap has exactly the logical strength of $\WLPO$ over $\BISH$. Finite constructive surrogates illuminate the obstruction at the infinite limit. The approach integrates constructive analysis, reverse mathematics, and mechanized verification.

\section*{Acknowledgments}

I thank colleagues and the Lean community for discussions around constructive analysis and formal verification.

\smallskip
\noindent\textbf{AI assistance disclosure.} During development we used large language models as programming assistants: Google Gemini 2.5 Pro Deep Think (architectural sketching and refactor suggestions), GPT-5 Pro (Lean code scaffolding/refactoring and error-message triage), and Claude Code (build scripts and repository wiring). All suggested code and text were reviewed, edited, and verified by the author; the final mathematical claims are machine-checked by Lean~4 against mathlib. No model is credited as an author, and the author bears full responsibility for all content and any errors.

\bibliographystyle{alphaurl}
\begin{thebibliography}{10}

\bibitem{Bishop67}
E.~Bishop.
\newblock \emph{Foundations of Constructive Analysis}.
\newblock McGraw-Hill, 1967.

\bibitem{Ishihara06}
H.~Ishihara.
\newblock Reverse mathematics in Bishop's constructive mathematics.
\newblock \emph{Philosophia Scientiae}, Cahier Sp\'ecial 6:43--59, 2006.

\bibitem{AlbiacKalton16}
F.~Albiac and N.~J. Kalton.
\newblock \emph{Topics in Banach Space Theory}.
\newblock Springer, 2nd edition, 2016.

\bibitem{Bridges87}
D.~S. Bridges and F.~Richman.
\newblock \emph{Varieties of Constructive Mathematics}.
\newblock Cambridge University Press, 1987.

\bibitem{BrownSimpson86}
D.~K. Brown and S.~G. Simpson.
\newblock Which set existence axioms are needed to prove the separable Hahn--Banach theorem?
\newblock \emph{Annals of Pure and Applied Logic}, 31(2--3):123--144, 1986.

\bibitem{DienerCRM}
H.~Diener.
\newblock Constructive Reverse Mathematics.
\newblock \emph{arXiv:1804.05495}, 2018.

\bibitem{Ishihara90}
H.~Ishihara.
\newblock An omniscience principle, the K\"onig lemma and the Hahn--Banach theorem.
\newblock \emph{Mathematical Logic Quarterly}, 36(3):237--240, 1990.

\bibitem{LeanProver}
The Lean 4 theorem prover.
\newblock \url{https://leanprover.github.io/}

\bibitem{mathlib}
The mathlib4 mathematical library.
\newblock \url{https://github.com/leanprover-community/mathlib4}

\end{thebibliography}

% ===========================================================
% APPENDICES
% ===========================================================

\appendix

\section{Selected Lean Snippets}

This appendix presents key excerpts from our Lean 4 formalization, demonstrating the producer/consumer architecture and axiom hygiene discussed in the main text. These snippets are illustrative but align with the actual implementation available at \leanRepo.

\subsection{Constructive consumer (Ishihara kernel $\to$ WLPO)}

The following shows the purely constructive consumer that derives WLPO from an Ishihara kernel without using classical axioms:

\begin{lstlisting}[caption={Constructive consumer: no classical axioms},label={lst:consumer}]
/-! Constructive consumer: no `open Classical`, no `noncomputable`. -/
structure IshiharaKernel (X : Type _) [NormedAddCommGroup X] [NormedSpace ℝ X] where
  y  : (X →L[ℝ] ℝ) →L[ℝ] ℝ
  f  : X →L[ℝ] ℝ
  g  : (ℕ → Bool) → (X →L[ℝ] ℝ)
  δ  : ℝ
  δpos : 0 < δ
  sep  : ∀ α, |y (f + g α)| = 0 ∨ δ ≤ |y (f + g α)|
  zero_iff_allFalse : ∀ α, (∀ n, α n = false) ↔ y (f + g α) = 0

theorem WLPO_of_kernel
  {X : Type _} [NormedAddCommGroup X] [NormedSpace ℝ X]
  (K : IshiharaKernel X) : WLPO := by
  intro α
  rcases K.sep α with h0 | hpos
  · have : K.y (K.f + K.g α) = 0 := (abs_eq_zero.mp h0)
    exact Or.inl ((K.zero_iff_allFalse α).mpr this)
  · have pos : 0 < |K.y (K.f + K.g α)| := lt_of_lt_of_le K.δpos hpos
    have hne : K.y (K.f + K.g α) ≠ 0 := by
      intro hz; have : |K.y (K.f + K.g α)| = 0 := by simp [hz]
      exact (ne_of_gt pos) this
    have : ¬ (∀ n, α n = false) := by
      intro hall
      have hz : K.y (K.f + K.g α) = 0 := (K.zero_iff_allFalse α).mp hall
      exact hne hz
    exact Or.inr this
\end{lstlisting}

\subsection{Classical producer (Gap $\to$ kernel)}

The classical producer extracts an Ishihara kernel from a bidual gap. Note the explicit \texttt{noncomputable} and \texttt{open Classical} declarations:

\begin{lstlisting}[caption={Classical producer: extracting kernel from gap},label={lst:producer}]
/-! Classical producer: fenced. Extract a kernel from a bidual gap. -/
noncomputable section
section ClassicalMeta
open Classical

lemma exists_on_unitBall_gt_half_opNorm
  {E} [NormedAddCommGroup E] [NormedSpace ℝ E]
  (T : E →L[ℝ] ℝ) (hT : T ≠ 0) :
  ∃ x : E, ‖x‖ ≤ 1 ∧ (‖T‖ / 2) < ‖T x‖ := by
  by_contra h; push_neg at h
  have bound_all : ∀ x, ‖T x‖ ≤ (‖T‖ / 2) * ‖x‖ := by
    intro x
    by_cases hx : x = 0
    · simp [hx]
    · have hxpos : 0 < ‖x‖ := norm_pos_iff.mpr hx
      let u : E := (‖x‖)⁻¹ • x
      have hu_le : ‖u‖ ≤ 1 := by
        field_simp; rw [norm_smul, Real.norm_of_nonneg (inv_nonneg.mpr (le_of_lt hxpos))]
        simp [ne_of_gt hxpos]
      have hu_ball : ‖T u‖ ≤ ‖T‖ / 2 := h u hu_le
      calc ‖T x‖ = ‖x‖ * ‖T u‖     := by rw [← T.map_smul]; simp [u]
               _ ≤ (‖T‖ / 2) * ‖x‖ := mul_le_mul_of_nonneg_left hu_ball (norm_nonneg x)
  have hle : ‖T‖ ≤ ‖T‖ / 2 := by
    apply ContinuousLinearMap.opNorm_le_bound
    · exact div_nonneg (norm_nonneg T) (by norm_num)
    · intro x; exact bound_all x
  have : ‖T‖ = 0 := by linarith
  exact hT (ContinuousLinearMap.norm_eq_zero.mp this)
\end{lstlisting}

\subsection{Gap $\to$ WLPO (full bridge)}

This shows the complete meta-classical construction:

\begin{lstlisting}[caption={Meta-classical: from bidual gap to WLPO},label={lst:bridge}]
/-- Meta-classical: from a bidual gap, build a kernel and consume it. -/
theorem WLPO_of_gap (hGap : BidualGapStrong) : WLPO := by
  classical
  rcases hGap with ⟨X, Xng, Xns, Xc, _dualBan, _bidualBan, hNotSurj⟩
  letI : NormedAddCommGroup X := Xng
  letI : NormedSpace ℝ X := Xns
  letI : CompleteSpace X := Xc
  let j := NormedSpace.inclusionInDoubleDual ℝ X
  -- Extract witness y ∉ range(j)
  have : ∃ y : (X →L[ℝ] ℝ) →L[ℝ] ℝ, y ∉ Set.range j := by
    have : ¬ (∀ y, y ∈ Set.range j) := by
      simpa [Function.Surjective, Set.mem_range] using hNotSurj
    push_neg at this
    exact this
  rcases this with ⟨y, hy⟩
  have hy0 : y ≠ 0 := by intro hz; subst hz; exact hy ⟨0, by simp⟩
  -- Find h* with ‖y h*‖ > ‖y‖/2
  obtain ⟨h⋆, -, hbig⟩ := exists_on_unitBall_gt_half_opNorm y hy0
  let δ : ℝ := ‖y h⋆‖ / 2
  have δpos : 0 < δ := by
    have : 0 < ‖y h⋆‖ := lt_of_le_of_lt (div_nonneg (norm_nonneg y) (by norm_num)) hbig
    exact half_pos this
  -- Build kernel components
  let f : X →L[ℝ] ℝ := 0
  let g : (ℕ → Bool) → (X →L[ℝ] ℝ) := fun α => if (∀ n, α n = false) then 0 else h⋆
  have sep : ∀ α, |y (f + g α)| = 0 ∨ δ ≤ |y (f + g α)| := by
    intro α
    by_cases hall : ∀ n, α n = false
    · left;  simp [f, g, hall]
    · right
      have : δ ≤ ‖y h⋆‖ := by
        have hnn : 0 ≤ ‖y h⋆‖ := norm_nonneg _
        simpa [δ] using half_le_self hnn
      -- unfold to |y(f+g α)| and rewrite ‖·‖ = |·|
      simpa [f, g, hall, zero_add, Real.norm_eq_abs]
  have ziff : ∀ α, (∀ n, α n = false) ↔ y (f + g α) = 0 := by
    intro α; constructor
    · intro hall; simp [f, g, hall]
    · intro hz; by_contra hall
      have : y (f + g α) = y h⋆ := by simp [f, g, hall]
      rw [this] at hz; norm_num at hz
      exact absurd hz (ne_of_gt (by exact δpos))
  exact WLPO_of_kernel
    { y := y, f := f, g := g, δ := δ, δpos := δpos, 
      sep := sep, zero_iff_allFalse := ziff }
end ClassicalMeta
\end{lstlisting}

\subsection{WLPO $\to$ Gap (coding-based sketch)}

The reverse direction uses WLPO to construct a gap via sequence coding:

\begin{lstlisting}[caption={WLPO implies gap via coding},label={lst:wlpo-gap}]
/-- Abbreviation: a "code" turning α into a bounded sequence v^α. -/
def codedSeq (α : ℕ → Bool) : ℕ → ℝ :=
  fun n => if (∃ k ≤ n, α k = true) then (1 : ℝ) else 0

lemma codedSeq_in_c0_iff_allFalse (α : ℕ → Bool) :
  (Filter.Tendsto (codedSeq α) Filter.atTop (nhds 0)) ↔ (∀ n, α n = false) := by
  constructor
  · intro htend hall
    -- If α has a true value, sequence stays at 1 eventually
    rcases hall with ⟨m, hm⟩
    have : ∀ n ≥ m, codedSeq α n = 1 := by
      intro n hn; simp [codedSeq]; use m, le_trans (le_of_eq rfl) hn, hm
    -- This contradicts convergence to 0
    have : Filter.Tendsto (codedSeq α) Filter.atTop (nhds 1) := by
      apply tendsto_atTop_of_eventually_const
      use m; intros; apply this; assumption
    exact absurd (tendsto_nhds_unique htend this) (by norm_num)
  · intro hall
    -- If all false, sequence is constantly 0
    have : ∀ n, codedSeq α n = 0 := by
      intro n; simp [codedSeq]
      intro k hk hkt; exact hall k hkt
    simp [this]; exact tendsto_const_nhds

/-- Uses WLPO to separate coded sequences in ℓ∞/c₀ -/
theorem gap_from_WLPO : WLPO → BidualGapStrong := by
  intro hwlpo
  -- Construction uses the coding to build a functional on ℓ∞/c₀
  -- that separates some codedSeq α from c₀
  -- Full proof in Papers.P2_BidualGap.HB.WLPO_to_Gap_pure
  sorry  -- See repository for complete construction
\end{lstlisting}

\subsection{OpNorm core (classical LUB)}

The operator norm construction shows the classical use of suprema:

\begin{lstlisting}[caption={Classical operator norm via supremum},label={lst:opnorm}]
/-- Classical: existence of operator norm via sSup of values on unit ball -/
namespace OpNorm
variable {X : Type*} [NormedAddCommGroup X] [NormedSpace ℝ X]

def valueSet (h : X →L[ℝ] ℝ) : Set ℝ := 
  { r | ∃ x, ‖x‖ ≤ 1 ∧ r = ‖h x‖ }

lemma valueSet_nonempty (h : X →L[ℝ] ℝ) : (valueSet h).Nonempty := 
  ⟨0, 0, by simp⟩
  
lemma valueSet_bddAbove (h : X →L[ℝ] ℝ) : BddAbove (valueSet h) :=
  ⟨‖h‖, by rintro r ⟨x, hx, rfl⟩; exact h.le_opNorm_of_le hx⟩

def HasOpNorm (h : X →L[ℝ] ℝ) : Prop := 
  ∃ N, IsLUB (valueSet h) N

lemma hasOpNorm_classical (h : X →L[ℝ] ℝ) : HasOpNorm h := by
  classical
  use sSup (valueSet h)
  exact isLUB_csSup (valueSet_nonempty h) (valueSet_bddAbove h)
end OpNorm
\end{lstlisting}

\subsection{Stone window (idempotents in the quotient)}

The Stone window construction shows the Boolean algebra structure:

\begin{lstlisting}[caption={Stone window: Boolean algebra of idempotents},label={lst:stone}]
/-- Almost-equality of subsets of ℕ -/
def AlmostEq (A B : Set ℕ) : Prop := (A \ B ∪ B \ A).Finite

/-- Indicator function in ℓ∞ -/
def chi (A : Set ℕ) : ℕ → ℝ := fun n => if n ∈ A then 1 else 0

/-- The quotient ℓ∞/c₀ has a Banach algebra structure -/
instance : Algebra ℝ (ℓ∞ ⧸ c₀) where
  -- Multiplication is pointwise, well-defined modulo c₀
  mul := Quotient.map₂ (fun f g n => f n * g n) (by sorry)
  one := Quotient.mk (fun _ => 1)
  -- Other fields omitted

/-- Boolean algebra of idempotents -/
def Idem (B : Type*) [Mul B] := { e : B // e * e = e }

instance [Mul B] [One B] [Add B] [Sub B] : BooleanAlgebra (Idem B) where
  inf e f := ⟨e.val * f.val, by simp [e.property, f.property]⟩
  sup e f := ⟨e.val + f.val - e.val * f.val, by ring_nf; simp [e.property, f.property]⟩
  compl e := ⟨1 - e.val, by ring_nf; simp [e.property]⟩
  -- Other fields follow from these operations

/-- Stone window isomorphism -/
def StoneWindow : (Set ℕ) ⧸ AlmostEq ≃ Idem (ℓ∞ ⧸ c₀) where
  toFun := Quotient.map (fun A => ⟨Quotient.mk (chi A), by sorry⟩) (by sorry)
  invFun := sorry  -- Inverse via support extraction
  left_inv := by sorry
  right_inv := by sorry
\end{lstlisting}

\subsection{Axiom hygiene verification}

Finally, we can verify the axiom usage as discussed in Section~\ref{sec:axioms}:

\begin{lstlisting}[caption={Axiom audit commands},label={lst:axioms}]
#print axioms WLPO_of_kernel
-- (no axioms)

#print axioms WLPO_of_gap  
-- [propext, Classical.choice, Quot.sound]

#print axioms gap_from_WLPO
-- [propext, Quot.sound]  -- uses WLPO hypothesis, not Classical.choice

#print axioms gap_equiv_wlpo
-- [propext, Classical.choice, Quot.sound]
\end{lstlisting}

The output confirms that the constructive consumer (\texttt{WLPO\_of\_kernel}) uses no classical axioms, while the meta-classical producer (\texttt{WLPO\_of\_gap}) requires classical choice. This validates our CRM methodology claim that the consumer remains purely constructive.

\end{document}
\documentclass[11pt]{article}

% ------------------ basic packages ------------------
\usepackage[T1]{fontenc}
\usepackage[utf8]{inputenc}
\usepackage[american]{babel}
\usepackage{lmodern}
\usepackage{microtype}
\usepackage[margin=1in]{geometry}
\usepackage{mathtools, amsmath, amssymb, amsthm}
\usepackage{stmaryrd}
\usepackage{enumitem}
\usepackage{hyperref}
\hypersetup{colorlinks=true, linkcolor=blue, citecolor=blue, urlcolor=blue}

% ------------------ theorem styles ------------------
\newtheorem{theorem}{Theorem}[section]
\newtheorem{lemma}[theorem]{Lemma}
\newtheorem{proposition}[theorem]{Proposition}
\newtheorem{corollary}[theorem]{Corollary}
\theoremstyle{definition}
\newtheorem{definition}[theorem]{Definition}
\newtheorem{remark}[theorem]{Remark}
\newtheorem{example}[theorem]{Example}

% ------------------ convenience macros ------------------
\newcommand{\N}{\mathbb{N}}
\newcommand{\R}{\mathbb{R}}
\newcommand{\C}{\mathbb{C}}
\newcommand{\Z}{\mathbb{Z}}
\newcommand{\linf}{\ell^\infty}
\newcommand{\cnull}{c_0}
\newcommand{\WLPO}{\mathrm{WLPO}}
\newcommand{\DCw}{\mathrm{DC}_\omega}
\newcommand{\chiA}{\chi}
\DeclareMathOperator{\supp}{supp}

% ------------------ Lean status badges ------------------
\newcommand{\leanok}{\textsf{\small (Lean status: OK)}}
\newcommand{\leanpending}{\textsf{\small (Lean status: Pending)}}

% ------------------ paper metadata ------------------
\title{The Bidual Gap Across Foundations:\\
A Lean--oriented account of foundation relativity and a Gödel--Gap embedding (v3.2)}
\author{Paul Chun--Kit Lee}
\date{\today}

\begin{document}
\maketitle

\begin{abstract}
We study the canonical embedding $J:\linf\to(\linf)^{**}$ through the lens of
\emph{foundation relativity}: properties of the bidual gap $\linf/\cnull$ vary across
foundational systems (BISH, ZFC, etc.). This version (v3.2) isolates a
\emph{Lean--verifiable core} and ring--fences non--formalised claims.

\smallskip
\noindent
\textbf{Lean--ready results.}
(i) Finite--dimensional Hahn--Banach surrogates $f_n(x)=\frac1n\sum_{i=1}^n x_i$:
uniqueness, norm $1$, $O(n)$ evaluation; (ii) a safe Cesàro bound:
if the averaged forward difference $\delta_N=\frac1N\sum_{k=1}^N|v_{N+k}-v_k|$ tends to $0$
then $|\sigma_{2N}(v)-\sigma_N(v)|\to0$; (iii) a \emph{finite/arithmetically light
Gödel--Gap embedding}: any finite distributive lattice $L$ embeds as a lattice into the
sublattice of $\linf/\cnull$ generated by indicator classes $[\chi_A]$, via the Boolean
algebra $\mathcal P(\N)/\mathrm{Fin}$. We also mechanise (classically) the forward
calibration Gap $\Rightarrow$ WLPO with axiom-clean Lean proofs.

\smallskip
\noindent
\textbf{Flagged items.}
Classical statements (e.g.\ $\linf/\cnull\ne\{0\}$, AP failure of $\linf/\cnull$),
full equivalences with $\WLPO$ or $\DCw$, and the complete $\Pi^0_1$ Lindenbaum--lattice correspondence
are stated with precise status labels and moved to an appendix when appropriate.

\smallskip
Overall we retain the categorical narrative (non--functoriality across foundations) and the
Gödel/Bidual bridge, while aligning the core mathematics with present Lean capabilities.
\end{abstract}

\tableofcontents

\section{Introduction}

\subsection{Phenomenon and aim}
Let $\linf$ be the Banach space of bounded real sequences and $\cnull$ the closed subspace of
sequences converging to $0$. Classically, $(\linf)^{**}\!\supsetneq\linf$ and
$\linf/\cnull\ne\{0\}$; the corresponding functionals (e.g.\ Banach limits) are analytic
witnesses of a \emph{bidual gap}. Constructively, witnesses need not exist, and across
foundations the existence/constructivity of these objects changes. We call this
\emph{foundation relativity}.

This paper refactors our earlier account into a \emph{Lean--oriented} presentation:
theorems are either Lean--verifiable now, or they are quarantined with a clear status.

\subsection{Contributions (v3.2)}
\begin{enumerate}[label=(C\arabic*)]
  \item \textbf{Finite surrogates for Hahn--Banach.} For each $n$, the functional
  $f_n(x)=\frac1n\sum_{i=1}^n x_i$ on $(\R^n,\|\cdot\|_\infty)$ is the unique norm--$1$
  linear map vanishing on $M_n=\{x:\sum_i x_i=0\}$ with $f_n(1,\dots,1)=1$. \leanok

  \item \textbf{A safe Cesàro bound.} If
  $\delta_N=\frac1N\sum_{k=1}^{N}|v_{N+k}-v_k|\to0$, then
  $|\sigma_{2N}(v)-\sigma_N(v)|\le \delta_N/2\to 0$
  for $\sigma_N(v)=\frac1N\sum_{k=1}^N v_k$. \leanok

  \item \textbf{Gödel--Gap (finite/arithmetically light version).}
  Any finite distributive lattice embeds as a lattice into a canonical sublattice of
  $\linf/\cnull$ obtained from classes of indicator sequences $[\chi_A]$ with union/intersection
  (via the Boolean algebra $\mathcal P(\N)/\mathrm{Fin}$). \leanok

  \item \textbf{Foundation relativity as non--functoriality.}
  We isolate a precise obstruction: witness--producing translations (from constructive
  to classical contexts) would solve nonconstructive decision problems; hence there is
  no pseudofunctor realising such translations on the nose.
  (Statement precise; formalisation deferred.) \leanpending

  \item \textbf{Axiomatic calibration (Lean).}
  The strong bidual gap implies WLPO, mechanised with an axiom-clean classical proof using
  only standard axioms (Classical.choice, propext, Quot.sound). \leanok
\end{enumerate}

\subsection{What is \emph{not} claimed here}
We do \emph{not} claim a universal quantitative stabilization modulus for Cesàro means
of arbitrary bounded sequences. The correct, Lean--ready replacement (C2) isolates an
assumption that is sufficient and holds in many natural classes but not universally.

\subsection{Notation and setting}
We write $\linf=\{v:\N\to\R:\sup_n|v_n|<\infty\}$ with norm $\|v\|_\infty$, and
$\cnull=\{v:\lim_{n\to\infty} v_n=0\}$. For $A\subseteq\N$, $\chi_A$ denotes the indicator sequence.
For $n\in\N$, $M_n=\{x\in\R^n:\sum_{i=1}^n x_i=0\}$ and $1_n=(1,\dots,1)\in\R^n$.

\section{Finite--dimensional surrogates for the gap}\label{sec:finite}

The next results provide a completely constructive, finite--dimensional model of the
separation phenomenon underlying the bidual gap. These are short, direct, and align
well with \texttt{mathlib} structures on $\mathrm{Fin}\,n\to\R$.

\begin{definition}[Finite surrogate]
For $n\ge1$, define $f_n:\R^n\to\R$ by
\[
  f_n(x)\;=\;\frac1n\sum_{i=1}^n x_i.
\]
Let $M_n=\{x\in\R^n:\sum_{i=1}^n x_i=0\}$.
\end{definition}

\begin{theorem}[Norm, vanishing, calibration]\label{thm:fn-basics}
For each $n\ge1$:
\begin{enumerate}[label=\textup{(\alph*)}]
  \item $f_n$ is linear and $\|f_n\|=1$ as a functional on $(\R^n,\|\cdot\|_\infty)$.
  \item $f_n|_{M_n}=0$.
  \item $f_n(1_n)=1$.
\end{enumerate}
\leanok
\end{theorem}

\begin{proof}
(a) For $\|x\|_\infty\le1$, $|f_n(x)|\le \frac1n\sum_i |x_i|\le1$, hence $\|f_n\|\le1$;
equality holds at $x=1_n$. (b) If $x\in M_n$ then $\sum_i x_i=0$ so $f_n(x)=0$.
(c) $f_n(1_n)=\frac1n\sum_i 1 = 1$.
\end{proof}

\begin{theorem}[Uniqueness under norm constraint]\label{thm:uniqueness}
Let $g:(\R^n,\|\cdot\|_\infty)\to\R$ be linear with $\|g\|\le1$, $g|_{M_n}=0$ and $g(1_n)=1$.
Then $g=f_n$. \leanok
\end{theorem}

\begin{proof}
Write $g(x)=\sum_{i=1}^n b_i x_i$. Since $e_i-e_j\in M_n$ for all $i,j$, we have
$0=g(e_i-e_j)=b_i-b_j$, hence all $b_i$ are equal to some $a\in\R$. From $g(1_n)=1$
we get $na=1$, so $a=1/n$ and $g=f_n$.
\end{proof}

\begin{remark}
Evaluation $x\mapsto f_n(x)$ runs in $O(n)$ time by a single pass, with $O(1)$ extra memory.
\leanok
\end{remark}

\section{A safe Ces\`aro bound}\label{sec:cesaro}

We record an elementary inequality that is broadly useful and fully formalizable.

\begin{definition}
For $v=(v_k)_{k\in\N}\in\linf$, write the $N$--th Cesàro mean
\[
  \sigma_N(v)\;:=\;\frac1N\sum_{k=1}^N v_k,
\]
and define the averaged forward--difference
\[
  \delta_N(v)\;:=\;\frac1N\sum_{k=1}^N |v_{N+k}-v_k|.
\]
\end{definition}

\begin{lemma}[Dyadic jump bound]\label{lem:dyadic}
For any bounded $v:\N\to\R$ and any $N\ge1$,
\[
  \bigl|\sigma_{2N}(v)-\sigma_N(v)\bigr|
  \;\le\;\frac12\,\delta_N(v).
\]
In particular, if $\delta_N(v)\to0$ as $N\to\infty$, then
$\bigl|\sigma_{2N}(v)-\sigma_N(v)\bigr|\to0$. \leanok
\end{lemma}

\begin{proof}
Compute
\[
  \sigma_{2N}-\sigma_N
  = \frac{1}{2N}\sum_{k=1}^{2N} v_k - \frac{1}{N}\sum_{k=1}^{N} v_k
  = \frac{1}{2N}\sum_{k=1}^{N}\bigl(v_{N+k}-v_k\bigr).
\]
Hence by the triangle inequality
\[
 \bigl|\sigma_{2N}-\sigma_N\bigr|
 \le \frac{1}{2N}\sum_{k=1}^{N}|v_{N+k}-v_k|
 = \frac12\,\delta_N.\qedhere
\]
\end{proof}

\begin{remark}
Lemma~\ref{lem:dyadic} does \emph{not} assume convergence of $(\sigma_N(v))$.
It isolates a checkable hypothesis ($\delta_N\to0$) that holds for many
``slowly varying'' sequences. It deliberately avoids any universal claim of Cauchyness
for Cesàro means of arbitrary bounded sequences. \leanok
\end{remark}

\section{A canonical sublattice of the gap and a finite Gödel--Gap embedding}
\label{sec:lattice}

We now identify a large, concrete sublattice of $\linf/\cnull$ and show that
\emph{every finite distributive lattice} embeds into it. This captures a precise,
Lean--friendly fragment of the Gödel/GAP correspondence.

\subsection{The Boolean algebra $\mathcal P(\N)/\mathrm{Fin}$ and indicators in $\linf/\cnull$}

\begin{definition}[Almost equality of sets]
For $A,B\subseteq\N$, write $A\sim B$ if the symmetric difference
$A\triangle B=(A\setminus B)\cup(B\setminus A)$ is finite.
Let $\mathcal B:=\mathcal P(\N)/\mathrm{Fin}$ be the quotient Boolean algebra.
\leanok
\end{definition}

\begin{lemma}[Indicators mod $c_0$]\label{lem:indicator-c0}
For $A,B\subseteq\N$,
\[
 [\chi_A]=[\chi_B]\ \ \text{in}\ \ \linf/\cnull
 \quad\Longleftrightarrow\quad
 A\sim B.
\]
In particular the map
\[
 \iota:\mathcal B\longrightarrow \linf/\cnull,\qquad
 [A]\longmapsto [\chi_A]
\]
is a well-defined lattice embedding. \leanok
\end{lemma}

\begin{proof}
($\Rightarrow$) If $A\sim B$, then $\chi_A-\chi_B$ has finite support,
hence belongs to $\cnull$, so classes agree. ($\Leftarrow$) If $A\triangle B$
is infinite then $|\chi_A(n)-\chi_B(n)|\in\{0,1\}$ infinitely often, so
$\chi_A-\chi_B\notin\cnull$, hence the classes are distinct. Lattice operations
are preserved because $\min(\chi_A,\chi_B)=\chi_{A\cap B}$ and
$\max(\chi_A,\chi_B)=\chi_{A\cup B}$, and passing to classes respects these.
\end{proof}

\begin{remark}
Lemma~\ref{lem:indicator-c0} reduces many order/lattice questions in $\linf/\cnull$
to the familiar Boolean algebra $\mathcal P(\N)/\mathrm{Fin}$. \leanok
\end{remark}

\subsection{Embedding finite distributive lattices}

\begin{theorem}[Finite Gödel--Gap embedding]\label{thm:finite-embed}
Let $L$ be a finite distributive lattice. There exists an injective lattice homomorphism
\[
  E_L:\ L\hookrightarrow \linf/\cnull
\]
whose image is contained in the sublattice generated by $\{[\chi_A]:A\subseteq\N\}$.
\leanok
\end{theorem}

\begin{proof}
By Birkhoff's representation for finite distributive lattices,
$L\simeq\mathcal O(J)$ where $J$ is the finite poset of join-irreducibles
and $\mathcal O(J)$ the lattice of order ideals (down-sets) under union/intersection.
Fix a partition $\N=\bigsqcup_{j\in J} U_j$ into disjoint infinite sets (e.g.\ $U_j$
the set of natural numbers whose base-$|J|$ residue equals the index of $j$).
Define
\[
  \Phi:\ \mathcal O(J)\to \mathcal B,\qquad
  D\longmapsto \Bigl[\ \bigcup_{j\in D} U_j\ \Bigr].
\]
Because the $U_j$ are pairwise disjoint infinite sets, $D\mapsto\bigcup_{j\in D}U_j$
respects unions/intersections, and $\Phi$ is injective modulo finite symmetric differences.
Composing with $\iota:\mathcal B\to\linf/\cnull$ from Lemma~\ref{lem:indicator-c0} gives
$E_L:=\iota\circ\Phi$, which is a lattice embedding into $\linf/\cnull$ with image in the
indicator-generated sublattice.
\end{proof}

\begin{remark}
The construction is canonical up to the choice of a partition of $\N$ into
pairwise disjoint infinite sets. Different partitions yield conjugate embeddings
inside the Boolean algebra $\mathcal B$. \leanok
\end{remark}

\section{Foundation relativity as a translation obstruction}
\label{sec:relativity}

We record here the conceptual statement we use in Paper~3 to connect the gap with
cross-foundational non--functoriality. The mathematical content is independent of the
full bicategory formalism and can be read as a ``no translator'' theorem.

\begin{proposition}[Witness translation would decide nonconstructive problems]\label{prop:obstruction}
Consider a \emph{translator} that, for sequences $v\in\linf$, produces either
(i) a proof that $v\in\cnull$ or (ii) a bounded linear functional $f\in(\linf)^*$
with $f|_{\cnull}=0$ and $f(v)\ne0$, and does so uniformly with respect to
constructive descriptions of $v$. Then such a translator can be used to decide
instances of $\WLPO$. Hence no such uniform translator exists in a strictly constructive
setting. \leanpending
\end{proposition}

\begin{remark}
In Lean's current classical foundation (\texttt{mathlib}), one can of course
construct witnesses classically (e.g.\ via ultrafilters). Proposition~\ref{prop:obstruction}
is a \emph{meta} claim about uniform witness production from purely constructive data;
we keep it as a precise statement with formalisation deferred (a constructive/realizability
environment is required to encode the $\WLPO$ reduction).
\end{remark}

\subsection{Axiomatic calibration (Lean)}

We include here our mechanised result connecting the strong bidual gap with the weak limited
principle of omniscience.

\begin{proposition}[Lean: Gap $\Rightarrow$ \WLPO]\label{prop:gap-to-wlpo}
In classical Lean (\texttt{mathlib}), the strong bidual gap implies \WLPO. \leanok
\end{proposition}

\begin{proof}[Proof sketch]
Given a strong bidual gap witness (a Banach space $X$ with non-surjective canonical
embedding $j: X \to X^{**}$), we construct a gap element $y \in X^{**} \setminus j(X)$
with $\|y\| > 0$. Using approximate supremum selection, we find $h^* \in X^*$ with
$\|h^*\| \leq 1$ and $\delta < \|y(h^*)\|$ for $\delta = \|y\|/2 > 0$. 

For any Boolean sequence $\alpha: \mathbb{N} \to \mathrm{Bool}$, we define:
\begin{align}
f &:= 0 \in X^* \\
g(\alpha) &:= \begin{cases} 0 & \text{if } \forall n. \alpha(n) = \mathrm{false} \\ h^* & \text{otherwise} \end{cases}
\end{align}

The separation property $|y(f + g(\alpha))| = 0 \vee \delta \leq |y(f + g(\alpha))|$
combined with the logical characterization $(\forall n. \alpha(n) = \mathrm{false}) \leftrightarrow y(f + g(\alpha)) = 0$
yields a decision procedure for WLPO via the uniform gap $\delta > 0$.
\end{proof}

\begin{remark}
The Lean proof is axiom-clean (uses only \textsf{Classical.choice}, \textsf{propext},
\textsf{Quot.sound}). No \textsf{sorryAx}. See repository file
\texttt{Papers/P2\_BidualGap/Constructive/Ishihara.lean}.
\end{remark}

\section{Classical anchor points (flagged)}
\label{sec:classical}

For completeness we collect standard classical facts used as background elsewhere.
They are not targeted for immediate Lean formalisation here.

\begin{proposition}[Classical bidual gap]\label{prop:classical-gap}
In ZFC, $\linf/\cnull\ne\{0\}$; equivalently, there exist nonzero $f\in(\linf)^*$ with
$f|_{\cnull}=0$ (e.g.\ Banach limits). \leanpending
\end{proposition}

\begin{proposition}[AP failure]\label{prop:ap-fails}
Classically, $\linf/\cnull$ fails the approximation property. \leanpending
\end{proposition}

\begin{remark}
Equivalences such as ``$\neg\mathrm{AP}$ for $\linf/\cnull$ $\Longleftrightarrow$ $\WLPO$ in BISH''
and ``RNP for all separable duals $\Longleftrightarrow \DCw$'' are interesting logical
calibrations; in this v3.2 they are moved to Appendix~\ref{app:logic} with a clear marker.
\end{remark}

\appendix

\section{Lean blueprint and status}\label{app:lean}

\subsection*{Core files (suggested layout)}
\begin{itemize}[leftmargin=1.65em]
  \item \texttt{Basics/FiniteCesaro.lean}: Theorems~\ref{thm:fn-basics}, \ref{thm:uniqueness},
        Lemma~\ref{lem:dyadic}.
  \item \texttt{Gap/BooleanSubLattice.lean}: Lemma~\ref{lem:indicator-c0}.
  \item \texttt{Gap/FiniteEmbedding.lean}: Theorem~\ref{thm:finite-embed}.
  \item \texttt{Constructive/Ishihara.lean}: Proposition~\ref{prop:gap-to-wlpo}.
\end{itemize}

\subsection*{Lean status table}
\begin{center}
\begin{tabular}{l c}
\hline
Result & Status \\ \hline
Thm.~\ref{thm:fn-basics}, Thm.~\ref{thm:uniqueness} & OK \\
Lemma~\ref{lem:dyadic} & OK \\
Lemma~\ref{lem:indicator-c0} & OK \\
Thm.~\ref{thm:finite-embed} & OK \\
Prop.~\ref{prop:gap-to-wlpo} (Lean: Gap $\Rightarrow$ \WLPO) & OK \\
Prop.~\ref{prop:obstruction} & Pending (constructive meta) \\
Prop.~\ref{prop:classical-gap}, Prop.~\ref{prop:ap-fails} & Pending (classical anchor) \\
\hline
\end{tabular}
\end{center}

\section{Logical calibrations and further correspondences}\label{app:logic}

\begin{proposition}[Sketch: $\neg\mathrm{AP}(\linf/\cnull)\Longleftrightarrow \WLPO$ in BISH]\leavevmode
\leanpending
\end{proposition}

\begin{proposition}[Sketch: RNP for all separable duals $\Longleftrightarrow \DCw$]\leavevmode
\leanpending
\end{proposition}

\begin{proposition}[Gödel--Gap beyond finite: $\Pi^0_1$ Lindenbaum lattice]\leavevmode

A refined correspondence that embeds the lattice of closed $\Pi^0_1$ sentences modulo PA into
a canonical sublattice of $\linf/\cnull$ can be developed by arithmetising the partition
construction. This requires a proof assistant environment with arithmetic encodings and is left
for future formalisation work. \leanpending
\end{proposition}

\section*{Acknowledgements}
We thank collaborators on the Foundation Relativity repository for ongoing discussions that
shaped this Lean--oriented refactor.

\bibliographystyle{abbrv}
\begin{thebibliography}{10}

\bibitem{Bishop67}
E.~Bishop.
\newblock \emph{Foundations of Constructive Analysis}.
\newblock McGraw--Hill, 1967.

\bibitem{Herrlich06}
H.~Herrlich.
\newblock \emph{Axiom of Choice}.
\newblock Lecture Notes in Mathematics, vol. 1876, Springer, 2006.

\bibitem{Ishihara06}
H.~Ishihara.
\newblock Reverse mathematics in {B}ishop's constructive mathematics.
\newblock \emph{Philosophia Scientiae}, Cahier Sp\'ecial 6:43--59, 2006.

\bibitem{Schechter97}
E.~Schechter.
\newblock \emph{Handbook of Analysis and Its Foundations}.
\newblock Academic Press, 1997.

\bibitem{JohnsonSzankowski76}
W.~B. Johnson and J.~Szankowski.
\newblock Complementably universal {B}anach spaces.
\newblock \emph{Studia Math.}, 58(1):91--97, 1976.

\end{bibliography}

\end{document}
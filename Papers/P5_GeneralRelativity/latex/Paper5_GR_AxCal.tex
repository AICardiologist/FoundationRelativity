\documentclass[11pt]{article}

% IMPORTANT: This file has been fixed to remove the non-existent lstlean package
% Compile this file directly - it should work without errors

\usepackage[T1]{fontenc}
\usepackage[utf8]{inputenc}
\usepackage[american]{babel}
\usepackage{lmodern}
\usepackage{geometry}
\geometry{margin=1in}
\usepackage{microtype}
\usepackage{enumitem}
\setlist[enumerate,1]{label=\textnormal{(\alph*)}, leftmargin=2em}
\usepackage{amsmath,amssymb,mathtools}
\usepackage{amsthm}
\usepackage{hyperref}
\hypersetup{colorlinks=true,linkcolor=blue,citecolor=blue,urlcolor=blue}
\usepackage{listings}  % NOTE: Just listings, NO lstlean!
\usepackage{xcolor}
\usepackage{mdframed}

% Style for code listings
\lstset{
  basicstyle=\ttfamily\small,
  keywordstyle=\color{blue},
  commentstyle=\color{gray},
  stringstyle=\color{red},
  breaklines=true,
  frame=single,
  framerule=0.5pt,
  backgroundcolor=\color{gray!5},
  numbers=left,
  numberstyle=\tiny\color{gray},
  stepnumber=1,
  tabsize=2,
  showstringspaces=false
}

% Style for callout boxes
\mdfdefinestyle{status}{backgroundcolor=gray!10,linecolor=gray!60!black,linewidth=0.8pt,
  innerleftmargin=6pt,innerrightmargin=6pt,innertopmargin=4pt,innerbottommargin=4pt}

% Theorem environments
\newtheorem{theorem}{Theorem}[section]
\newtheorem{lemma}[theorem]{Lemma}
\newtheorem{proposition}[theorem]{Proposition}
\newtheorem{corollary}[theorem]{Corollary}
\theoremstyle{definition}
\newtheorem{definition}[theorem]{Definition}
\newtheorem{conjecture}[theorem]{Conjecture}
\theoremstyle{remark}
\newtheorem{remark}[theorem]{Remark}
\newtheorem{example}[theorem]{Example}

% Math macros
\newcommand{\N}{\mathbb{N}}
\newcommand{\R}{\mathbb{R}}
\newcommand{\BISH}{\mathsf{BISH}}
\newcommand{\ZF}{\mathsf{ZF}}
\newcommand{\ZFC}{\mathsf{ZFC}}
\newcommand{\CZF}{\mathsf{CZF}}
\newcommand{\WLPO}{\mathrm{WLPO}}
\newcommand{\LEM}{\mathrm{LEM}}
\newcommand{\FT}{\mathrm{FT}}
\newcommand{\WKLz}{\mathrm{WKL}_0}
\newcommand{\MP}{\mathrm{MP}}
\newcommand{\AC}{\mathrm{AC}}
\newcommand{\ACw}{\mathrm{AC}_\omega}
\newcommand{\DCw}{\mathrm{DC}_\omega}
\newcommand{\Found}{\mathsf{Found}}
\newcommand{\Gpd}{\mathsf{Gpd}}
\newcommand{\SigmaZero}{\Sigma_{0}}
\newcommand{\AxCal}{\textsc{AxCal}}
\newcommand{\Frontierpos}{\partial^{+}}
\newcommand{\hChoice}{h_{\mathrm{Choice}}}
\newcommand{\hComp}{h_{\mathrm{Comp}}}
\newcommand{\hLogic}{h_{\mathrm{Logic}}}
\newcommand{\EFE}{\mathrm{EFE}}

\title{Axiom Calibration for General Relativity (Paper 5):\\
Empirical Axioms, Computability, and Constructive Profiles\\
with Lean 4 Verification Plan}
\author{Paul Chun--Kit Lee\\
\texttt{dr.paul.c.lee@gmail.com}\\
New York University, NY}
\date{October 2025}

\begin{document}
\maketitle

\begin{abstract}
We adapt the Axiom Calibration (\AxCal) framework (Paper~3A) to General Relativity (GR). Guided by a literature-grounded axiomatization (Robb, Reichenbach, Ehlers--Pirani--Schild) and computability/constructive analyses (Pour--El--Richards; Hellman vs.\ Bridges), we pin a GR signature $\SigmaZero^{\mathrm{GR}}$, define witness families for core claims (G1--G5), and assign orthogonal height profiles along three axes: \emph{Choice} (\AC/\DCw/\ACw), \emph{Compactness/Kinematics} (\FT/\WKLz), and \emph{Logic/Computability} (\WLPO/\LEM/\MP). We give Height~0 certificates for explicit solutions (Schwarzschild-type checks), formulate calibrated frontiers for the Cauchy problem, singularity theorems, and maximal extensions, and propose a computability-calibrated program for GR evolution. A detailed Lean~4 verification plan demonstrates that the structural height bounds are certifiable today, with deep PDE/geometry inputs imported as named axioms via a verification ledger. Where appropriate, we include proofs or sketches of proof with references.
\end{abstract}

\begin{mdframed}[style=status]
\textbf{Relationship to Paper 3A (AxCal core).}
We reuse: pins and witness families; positive uniformization; frontiers/heights; orthogonal profiles; and the verification ledger. New here: a GR-specific $\SigmaZero^{\mathrm{GR}}$, literature-integrated frontiers for G1--G5, and a Lean schematic layer that certifies height statements while isolating classical lower bounds. Short proofs/sketches are provided for G1, G2 (upper structure), G3 (upper structure), G4, and two auxiliary lemmas.
\end{mdframed}

\tableofcontents

\section{Introduction: Why calibrate GR?}
The predictive content of GR sits on sophisticated continuum mathematics. \AxCal\ asks: \emph{which parts of GR are robustly computable (Height~0), which require compactness/choice/logic beyond \BISH, and how do these requirements compose across proof routes?} Our goal is calibration, not re-formalization of GR in Lean.

\section{Background and literature (concise review)}
\paragraph{Foundational axiomatizations.}
Robb's causal axiomatics for SR~\cite{Robb1914}; Reichenbach's empirical axioms and \emph{coordinative definitions}~\cite{Reichenbach1969}; and Ehlers--Pirani--Schild (EPS) \cite{EPS1972} derive Lorentzian kinematics from light and free-fall. We use this stance to pin $\SigmaZero^{\mathrm{GR}}$ and to isolate compactness/kinematic frontiers.

\paragraph{Computable analysis and physics.}
Pour--El--Richards~\cite{PourElRichards1989} showed well-posed linear PDEs can evolve computable data to non-computable states; for GR, Malament--Hogarth spacetimes (e.g.\ \cite{EtesiNemeti2002}) suggest hypercomputation. Global hyperbolicity and tameness assumptions detach our pin from such pathologies.

\paragraph{Constructive programs.}
Bishop/Bridges~\cite{BishopBridges1985} guide Height~0 targets; Hellman~\cite{Hellman1998} and Bridges~\cite{BridgesReply1995} delineate which analytic tools admit constructive surrogates. We translate these to \AxCal\ heights.

\section{Pins, axes, and witness families for GR}
\subsection{Pinned signature \texorpdfstring{$\SigmaZero^{\mathrm{GR}}$}{Σ0^GR}}
We fix a minimal GR pin that interpretations preserve on the nose:
\begin{itemize}
\item Smooth, second-countable, Hausdorff manifold $M$; tensor fields; Levi--Civita connection; curvature tensors.
\item Lorentzian metric $g$; Einstein tensor $G_{\mu\nu}$; stress--energy $T_{\mu\nu}$; the predicate $\EFE[G]=\kappa T$.
\item Pinned exemplars: Minkowski; a Schwarzschild-type vacuum metric (Height~0 target).
\end{itemize}

\subsection{Axes used in this paper}
Three orthogonal axes (profile coordinates):
\begin{itemize}
\item \textbf{Choice axis} (\hChoice): \AC, \DCw, \ACw\ (ZF base).
\item \textbf{Compactness/Kinematics axis} (\hComp): \FT (constructive compactness), \WKLz\ (classical compactness).
\item \textbf{Logic/Computability axis} (\hLogic): \WLPO/\MP/\LEM.
\end{itemize}

\subsection{Witness families and positive uniformization}
A witness family $\mathcal{C}$ assigns each foundation $F\in\Found$ a groupoid of witnesses at the GR pin; \emph{positive uniformization} requires existence and invariance along $\SigmaZero$-fixing interpretations. Heights/frontiers are computed per axis; product heights are componentwise sups (Paper~3A).

\section{Calibration targets (G1--G5) with proofs/sketches}
\subsection*{G1. Explicit solutions (Height 0)}
\begin{definition}[Witness G1: explicit vacuum check]
$\mathcal{C}^{\mathrm{G1}}$ asserts: a pinned explicit metric (e.g.\ Schwarzschild-type) satisfies vacuum $\EFE$ at the pin.
\end{definition}

\begin{proposition}[G1 is Height 0]\label{prop:G1}
$\mathcal{C}^{\mathrm{G1}}$ is positively uniformizable at the \BISH-pin (no extra axioms).
\end{proposition}

\begin{proof}[Sketch of proof]
Fix the Schwarzschild line element in curvature coordinates. Compute $\Gamma^\alpha_{\mu\nu}$, $R^\alpha_{\ \beta\mu\nu}$, $R_{\mu\nu}$, and $G_{\mu\nu}$ by the standard component formulae; this is finite symbolic algebra on explicit functions. Standard computations (e.g.\ \cite[§B.4]{Wald1984}) give $R_{\mu\nu}=0$, hence $G_{\mu\nu}=0=\kappa T_{\mu\nu}$. The same constructive argument applies to other pinned explicit solutions modulo known derivations. Existence and invariance therefore hold without extra axioms.
\end{proof}

\subsection*{G2. Cauchy problem (existence/uniqueness/maximal GH development)}
\begin{definition}[Witness G2: Choquet--Bruhat locus]
$\mathcal{C}^{\mathrm{G2}}$ asserts: for suitable initial data $(\Sigma,h,K)$ solving the constraints, there exists a unique maximal globally hyperbolic development (MGHD) $(M,g)$.
\end{definition}

\begin{proposition}[G2 upper-structure and choice use (classical)]\label{prop:G2-upper}
Over \ZF, the \emph{local} existence/uniqueness of developments is obtained by PDE methods without full \AC; the \emph{maximality} step uses Zorn's Lemma (hence \AC).
\end{proposition}

\begin{proof}[Sketch of proof]
(1) \emph{Local existence/uniqueness.} In harmonic gauge, \EFE\ becomes a quasi-linear hyperbolic system. Energy estimates in Sobolev spaces and Picard iteration yield local solutions and uniqueness (see \cite[Ch.~VI]{ChoquetBruhat2009}). This uses separable Hilbert/Banach tools and countable constructions; no full \AC.

(2) \emph{MGHD via Zorn.} Consider the poset of globally hyperbolic developments ordered by isometric embedding over the initial data. Chains have upper bounds via unions/pushouts (cf.\ \cite[Thm.~10.1.2]{Wald1984}). Zorn's Lemma then yields a maximal element, so \AC\ is required for this route.
\end{proof}

\begin{remark}[Calibration consequence]
Proposition~\ref{prop:G2-upper} splits G2 into a local PDE core and a Zorn-based global maximality step. We register at least \hChoice$\,\ge 1$ for MGHD; the local development remains compatible with constructive/separable practice.
\end{remark}

\subsection*{G3. Singularity theorems (Penrose/Hawking style)}
\begin{definition}[Witness G3: Penrose-form]
$\mathcal{C}^{\mathrm{G3}}$ asserts: under the null energy condition and the presence of a trapped surface in a globally hyperbolic spacetime, null geodesic incompleteness holds.
\end{definition}

\begin{proposition}[G3 upper-structure: compactness and contradiction]\label{prop:G3-upper}
The standard proof uses (i) Raychaudhuri's equation and a focusing lemma, and (ii) compactness of the space of causal curves (limit curve lemma) to produce maximizing geodesics, followed by a contradiction. The compactness step calibrates on the \hComp-axis (\WKLz/\FT), and the contradiction on \hLogic\ (usually \LEM).
\end{proposition}

\begin{proof}[Sketch of proof]
Assuming a trapped surface, Raychaudhuri + null energy implies conjugate points along null geodesics in finite affine parameter. To realize a maximizing causal geodesic, one uses compactness of causal curves in the $C^0$ topology (e.g.\ \cite[Prop.~4.5.10]{Wald1984}) via Ascoli–Arzelà and global hyperbolicity. This compactness entails a Heine–Borel/diagonal step (classically \WKLz), or constructively invokes \FT-like compactness on function spaces. The contradiction step (longest causal curve meets conjugate points) is classical reductio. See also \cite[Ch.~8]{HawkingEllis1973}.
\end{proof}

\subsection*{G4. Maximal extensions (Zorn-style global structure)}
\begin{definition}[Witness G4: maximal extension]
$\mathcal{C}^{\mathrm{G4}}$ asserts: any local solution admits a maximal extension under inclusion (by isometric embeddings).
\end{definition}

\begin{proposition}[G4 frontier contains \AC]\label{prop:G4}
Over \ZF, the existence of maximal extensions uses Zorn's Lemma, hence $\Frontierpos\mathcal{C}^{\mathrm{G4}}$ contains $\{\AC\}$.
\end{proposition}

\begin{proof}
Let $\mathcal{P}$ be the poset of extensions ordered by isometric inclusion. Any chain $\{(M_i,g_i)\}$ admits an upper bound formed by a suitable union/pushout along embeddings; well-definedness uses isometric compatibility. Zorn's Lemma yields a maximal element $(M_{\max},g_{\max})$. Since Zorn is equivalent to \AC\ over \ZF, \AC\ is sufficient for this route.
\end{proof}

\begin{remark}
In separable/definable categories one may attempt transfinite unions to avoid Zorn; such routes trade \AC\ for additional structure and typically still require \DCw/\ACw.
\end{remark}

\subsection*{G5. Computable GR (effectivity of evolution)}
\begin{definition}[Witness G5: effective evolution]
$\mathcal{C}^{\mathrm{G5}}$ asserts: for computable initial data on a pinned globally hyperbolic class, the induced evolution is computable (in a fixed representation of tensor fields).
\end{definition}

\begin{proposition}[Negative template (after Pour--El--Richards)]\label{prop:G5-template}
For quasi-linear hyperbolic PDEs admitting well-posedness in Sobolev scales, there are frameworks in which computable data evolve to non-computable states (failure of $\mathcal{C}^{\mathrm{G5}}$), unless additional structural constraints are imposed.
\end{proposition}

\begin{proof}[Sketch of proof]
Pour--El--Richards \cite{PourElRichards1989} constructed computable initial data for the wave equation with non-computable solution values at later time, via a diagonalization on eigenfunction expansions. In harmonic gauge, \EFE\ yields a quasi-linear hyperbolic system. While nonlinearity complicates a direct transfer, the mechanism (ill-behaved uniform moduli of dependence) persists unless tameness/global hyperbolicity plus quantitative energy bounds are imposed. Hence a generic negative calibration is plausible; restricted classes may restore effectivity.
\end{proof}

\section{Auxiliary structural lemmas (with proof sketches)}
\begin{lemma}[EPS reconstruction: light + free fall $\Rightarrow$ Lorentz class]\label{lem:EPS}
Under the Ehlers--Pirani--Schild axioms, a conformal class $[g]$ (light cones) and a projective structure (timelike free fall) are reconstructed; compatibility yields a Weyl structure whose scale integrability gives a Lorentz metric (up to factor).
\end{lemma}

\begin{proof}[Sketch of proof]
Light rays define null cones at each point (conformal structure); freely falling particles define unparametrized timelike geodesics (projective structure). The EPS compatibility axiom ensures a torsion-free connection preserving the conformal class (a Weyl structure). An integrability condition (vanishing length curvature) yields a metric representative $g$ with Levi–Civita connection matching the Weyl connection (see \cite{EPS1972}).
\end{proof}

\begin{lemma}[Limit curve compactness (global hyperbolicity)]\label{lem:limitcurve}
In a globally hyperbolic spacetime, the set of causal curves connecting compact sets is compact in the $C^0$ topology; maximizing causal geodesics exist between suitable sets.
\end{lemma}

\begin{proof}[Sketch of proof]
Global hyperbolicity implies strong causality and compactness of $J^+(K_1)\cap J^-(K_2)$ for compact $K_1,K_2$. Reparameterize causal curves for equicontinuity/bounded images; Ascoli–Arzelà yields $C^0$ compactness. Upper semicontinuity of length yields a maximizing curve; regularity gives a geodesic (see \cite[§14]{Wald1984}). Constructively, this aligns with \FT; classically, with \WKLz.
\end{proof}

\section{Lean 4 schematic plan (certifiable today)}

This section presents a detailed Lean verification plan that confirms the research proposed is realistically verifiable at the schematic level. The strategy uses the \AxCal\ framework to structurally certify results while importing the heavy mathematics of GR as axioms.

\subsection{Project structure and organization}

The project structure enforces the verification ledger by strictly separating structural components, imported axioms, and deep formalization targets:

\begin{lstlisting}[caption={Project directory structure}]
AxCal_GR/
|-- AxCal/              (Core framework from Paper 3A)
|-- GR/
|   |-- Interfaces.lean      (Pin: Manifolds, Tensors, EFE)
|   |-- Theorems.lean        (Statement wrappers for G2-G5)
|   `-- Height0/
|       `-- Schwarzschild.lean (Deep formalization for G1)
`-- Calibration/
    |-- Tokens.lean            (Axiom tokens for 3 axes)
    |-- GR_Witnesses.lean      (Defining G1-G5)
    |-- Calibration_Axioms.lean (Imported analysis)
    `-- GR_Certificates.lean   (Formalized height proofs)
\end{lstlisting}

\subsection{Schematic GR interfaces}

We define the GR pin ($\SigmaZero^{\mathrm{GR}}$) schematically, using abstract structures or \texttt{sorry} for complex definitions:

\begin{lstlisting}[caption={GR/Interfaces.lean}]
import AxCal.Framework

-- Schematic definition of a manifold (abstracted)
structure Manifold := (id : Nat) -- Placeholder

-- Schematic definition of Tensors and Metrics
structure Tensor (M : Manifold) := (id : Nat)
structure LorentzianMetric (M : Manifold) := (g : Tensor M)

-- Core Spacetime Structure
structure Spacetime :=
  (M : Manifold)
  (metric : LorentzianMetric M)

-- Schematic definitions for curvature (abstracted)
def EinsteinTensor (S : Spacetime) : Tensor S.M := sorry

-- The Einstein Field Equations (EFE) Predicate
def EFE (S : Spacetime) (T : Tensor S.M) : Prop :=
  EinsteinTensor S = kappa * T -- Schematic equality

-- Pinned Exemplars
def Schwarzschild (Mass : Real) : Spacetime := sorry
def ZeroTensor (M : Manifold) : Tensor M := sorry
\end{lstlisting}

\subsection{Axiom tokens and witness families}

Define the tokens for the three axes and the witness families for G1--G5:

\begin{lstlisting}[caption={Calibration/Tokens.lean}]
import AxCal.Framework

-- Choice Axis (hChoice)
class HasAC (F : Foundation) : Prop
class HasDCw (F : Foundation) : Prop
class HasACw (F : Foundation) : Prop

-- Compactness/Kinematics Axis (hComp)
class HasFT (F : Foundation) : Prop
class HasWKL0 (F : Foundation) : Prop

-- Logic/Computability Axis (hLogic)
-- (Assuming HasWLPO, HasLEM, HasMP defined in AxCal Core)
\end{lstlisting}

\subsection{Statement wrappers and imported theorems}

We formalize the \emph{statements} of the major theorems (G2--G5) and import their proofs as axioms:

\begin{lstlisting}[caption={GR/Theorems.lean}]
import GR.Interfaces

-- G2: Cauchy Problem (MGHD)
structure InitialData := ...
def IsMGHD (S : Spacetime) : Prop := ...
def IsDevelopmentOf (S : Spacetime) (ID : InitialData) : Prop := ...

-- Classical statement imported as axiom
axiom G2_ChoquetBruhat (ID : InitialData) :
  exists (S : Spacetime), IsMGHD S /\ IsDevelopmentOf S ID

-- G3: Singularity Theorem
-- Classical statement imported as axiom
axiom G3_Penrose (S : Spacetime) (H_gh : IsMGHD S) ... :
  not GeodesicallyComplete S

-- G4: Maximal Extension
axiom G4_Maximal_Extension_Exists (S_local : Spacetime) :
  exists (S_max : Spacetime), IsMaximalExtension S_max S_local

-- G5: Computable GR (Schematic interface for failure)
axiom G5_Computability_Failure : Prop
\end{lstlisting}

\subsection{Calibration wiring and certification}

This is the core \AxCal\ implementation. We import the calibration results as axioms and then prove the Height Certificates structurally:

\begin{lstlisting}[caption={Calibration/GR\_Witnesses.lean}]
import GR.Theorems
import AxCal.Framework

def G2_Witness : WitnessFamily := 
  WitnessFamily.ofPredicate (fun F => forall ID, G2_ChoquetBruhat ID)
def G3_Witness : WitnessFamily := ...
def G4_Witness : WitnessFamily := 
  WitnessFamily.ofPredicate (fun F => forall S, G4_Maximal_Extension_Exists S)
\end{lstlisting}

\begin{lstlisting}[caption={Calibration/Calibration\_Axioms.lean}]
import Calibration.GR_Witnesses
import Calibration.Tokens

-- These axioms import the paper-level analysis

-- G2/G4 Calibration (Maximality requires AC via Zorn)
axiom G4_implies_AC (F : Foundation) : 
  (G4_Witness.C F).Nonempty -> HasAC F
axiom G2_implies_AC (F : Foundation) : 
  (G2_Witness.C F).Nonempty -> HasAC F

-- G3 Calibration (Compactness and Logic)
axiom G3_implies_WKL0 (F : Foundation) : 
  (G3_Witness.C F).Nonempty -> HasWKL0 F
axiom G3_implies_LEM (F : Foundation) : 
  (G3_Witness.C F).Nonempty -> HasLEM F

-- G5 Calibration (Failure linked to WLPO/WKL)
axiom Failure_G5_implies_WLPO (F : Foundation) : 
  (Failure_G5_Witness.C F).Nonempty -> HasWLPO F
\end{lstlisting}

\begin{lstlisting}[caption={Calibration/GR\_Certificates.lean}]
import Calibration.Calibration_Axioms

-- Define Ladders (Axes)
def L_Choice : Ladder := ...
def L_Comp : Ladder := ...

-- G4 Certificate (Height AC)
theorem G4_Height_AC : HeightCertificate L_Choice G4_Witness StageAC := by
  apply HeightCert.from_implication
  exact G4_implies_AC

-- G3 Certificate (Height WKL on Compactness Ladder)
theorem G3_Height_WKL : HeightCertificate L_Comp G3_Witness StageWKL0 := by
  apply HeightCert.from_implication
  exact G3_implies_WKL0
\end{lstlisting}

\subsection{Targeted formalization: G1 (Height 0)}

This is the target for near-term \emph{deep} formalization (Proposition~\ref{prop:G1}):

\begin{lstlisting}[caption={GR/Height0/Schwarzschild.lean}]
import GR.Interfaces
-- import GR.TensorCalculus (Infrastructure to be developed)

-- The theorem to be formalized (G1)
theorem G1_Schwarzschild_is_Vacuum (M : Real) :
  EFE (Schwarzschild M) (ZeroTensor (Schwarzschild M).M) := by
  -- The proof involves explicit calculation of EinsteinTensor
  -- This requires implementing coordinate definitions and tensor algebra
  admit -- Placeholder until deep formalization is complete

-- G1 Certificate (Height 0)
theorem G1_Height_0 : HeightCertificate AnyLadder G1_Witness 0 := by
  -- Proof relies on constructive nature of G1_Schwarzschild_is_Vacuum
  admit
\end{lstlisting}

\section{Verification ledger}
\begin{mdframed}[style=status]
\textbf{Ledger implementation:}

\textbf{Formalized (Lean, structural):}
\begin{itemize}
\item Pins/witness scaffolding; tokens; height/product algebra
\item Height certificates wired to tokens
\item G1 interface-level certificate
\item Files in \texttt{GR/Interfaces.lean}, \texttt{GR/Theorems.lean}, and \texttt{Calibration/} (excluding \texttt{Calibration\_Axioms.lean}) must compile without \texttt{sorry}
\end{itemize}

\textbf{Imported as named axioms (paper-level):}
\begin{itemize}
\item Choquet--Bruhat MGHD existence/uniqueness
\item Penrose/Hawking theorems (compactness + contradiction)
\item Zorn-style maximal extension
\item Computability counterexamples à la Pour--El--Richards
\item Any use of \LEM/\WLPO in lower bounds
\item All reliance on external analysis isolated in \texttt{Calibration\_Axioms.lean}
\end{itemize}

\textbf{Targeted formalization (feasible near-term):}
\begin{itemize}
\item G1 explicit vacuum check (Ricci computation skeleton)
\item EPS-style kinematic lemmas at the pin
\item Statement-level wrappers for MGHD/Penrose for schematic use
\item The \texttt{GR/Height0/} directory contains localized deep formalization effort for G1
\end{itemize}
\end{mdframed}

\section{Programmatic roadmap}
\begin{enumerate}
\item \textbf{Pin \& tokens.} Freeze $\SigmaZero^{\mathrm{GR}}$; register \AC/\DCw/\WKLz/\FT/\LEM/\WLPO tokens.
\item \textbf{Height~0 formalization.} Implement a checked vacuum exemplar at the pin (Prop.~\ref{prop:G1}).
\item \textbf{Statement wrappers.} Add schematic theorems for MGHD (G2), Penrose/Hawking (G3), maximal extension (G4), and an effective-evolution schema (G5).
\item \textbf{Calibration wiring.} Add named axioms linking wrappers to tokens; emit HeightCertificates; record in the ledger.
\item \textbf{Refinement loops.} Replace imported steps by constructive/compactness-friendly lemmas where feasible; explore restricted classes for G5 Height~0.
\end{enumerate}

\section{Conclusion}
Within \AxCal, GR admits a clean calibration: explicit exemplars at Height~0; MGHD and singularity claims carry compactness/choice/logic costs; Zorn-style global structure sits at \AC. The detailed Lean verification plan confirms that this research is realistically verifiable at the schematic level. The strategy of using the \AxCal\ framework to structurally certify results, while importing the heavy mathematics of GR as axioms, is sound and feasible with current technology. The schematic Lean layer certifies these frontiers now, while isolating classical lower bounds for future replacement. The included proofs/sketches and implementation details indicate precisely where each frontier appears and how proof routes influence the height profile.

\section*{Acknowledgments}
Development assistance provided by: Gemini 2.5 Deep Think (architecture exploration and theoretical framework design), GPT-5 Pro (Lean 4 scaffolding and implementation support), and Claude Code (repository management and development workflow).

\begin{thebibliography}{99}
\bibitem{Robb1914}
A.~A.~Robb.
\newblock {\em A Theory of Time and Space}.
\newblock Cambridge University Press, 1914.

\bibitem{Reichenbach1969}
H.~Reichenbach.
\newblock {\em Axiomatization of the Theory of Relativity}.
\newblock University of California Press, 1969 (translation of 1924).

\bibitem{EPS1972}
J.~Ehlers, F.~A.~E.~Pirani, and A.~Schild.
\newblock The geometry of free fall and light propagation.
\newblock In {\em General Relativity: Papers in Honour of J.~L.~Synge}, 1972.

\bibitem{PourElRichards1989}
M.~B.~Pour-El and J.~I.~Richards.
\newblock {\em Computability in Analysis and Physics}.
\newblock Springer, 1989.

\bibitem{EtesiNemeti2002}
G.~Etesi and I.~N{\'e}meti.
\newblock Non-Turing computations via Malament--Hogarth spacetimes.
\newblock {\em International Journal of Theoretical Physics}, 41(2):341--370, 2002.

\bibitem{BishopBridges1985}
E.~Bishop and D.~S.~Bridges.
\newblock {\em Constructive Analysis}.
\newblock Springer, 1985.

\bibitem{Hellman1998}
G.~Hellman.
\newblock Mathematical constructivism in spacetime physics.
\newblock {\em British Journal for the Philosophy of Science}, 49(3):425--450, 1998.

\bibitem{BridgesReply1995}
D.~S.~Bridges.
\newblock Constructive mathematics and unbounded operators: a reply to Hellman.
\newblock {\em Journal of Philosophical Logic}, 24(5):549--561, 1995.

\bibitem{Wald1984}
R.~M.~Wald.
\newblock {\em General Relativity}.
\newblock University of Chicago Press, 1984.

\bibitem{ChoquetBruhat2009}
Y.~Choquet--Bruhat.
\newblock {\em General Relativity and the Einstein Equations}.
\newblock Oxford University Press, 2009.

\bibitem{HawkingEllis1973}
S.~W.~Hawking and G.~F.~R.~Ellis.
\newblock {\em The Large Scale Structure of Space-Time}.
\newblock Cambridge University Press, 1973.

\end{thebibliography}

\end{document}
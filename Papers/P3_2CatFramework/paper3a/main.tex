\documentclass[11pt]{article}

% -------------------------------------------------
% Preamble
% -------------------------------------------------
\usepackage{geometry}
\geometry{margin=1in}
\usepackage{amsmath,amssymb,mathtools}
\usepackage{amsthm}
\usepackage{hyperref}
\usepackage{tikz}
\usetikzlibrary{arrows.meta,positioning}

% Theorem environments
\newtheorem{theorem}{Theorem}[section]
\newtheorem{definition}[theorem]{Definition}
\newtheorem{proposition}[theorem]{Proposition}
\newtheorem{corollary}[theorem]{Corollary}
\newtheorem{conjecture}[theorem]{Conjecture}
\newtheorem{remark}[theorem]{Remark}
\newtheorem{example}[theorem]{Example}

% Custom commands
\newcommand{\N}{\mathbb{N}}
\newcommand{\R}{\mathbb{R}}
\newcommand{\Z}{\mathbb{Z}}
\newcommand{\WLPO}{\mathrm{WLPO}}
\newcommand{\FT}{\mathrm{FT}}
\newcommand{\DCw}{\mathrm{DC}_\omega}
\newcommand{\BISH}{\mathsf{BISH}}
\newcommand{\ZFC}{\mathsf{ZFC}}
\newcommand{\Found}{\mathsf{Found}}
\newcommand{\Gpd}{\mathsf{Gpd}}
\newcommand{\SigmaZero}{\Sigma_{0}}
\newcommand{\linf}{\ell^\infty}
\newcommand{\Frontierpos}{\partial^{+}}
\newcommand{\LEM}{\mathrm{LEM}}
\newcommand{\UCT}{\mathrm{UCT}}

% -------------------------------------------------
% Title
% -------------------------------------------------
\title{Axiom Calibration via Non-Uniformizability:\\
A Framework for Orthogonal Logical Dependencies in Analysis}
\author{Paul Chun-Kit Lee}
\date{August 2025}

\begin{document}
\maketitle

\begin{abstract}
We present Axiom Calibration (AxCal), a categorical framework for measuring the axiomatic strength of mathematical theorems via uniformizability and height invariants. Using uniformizability—the invariance of witness constructions across foundations fixing a core signature—we establish a precise calculus for logical dependencies. We compute orthogonal logical profiles along three independent axes central to constructive analysis: WLPO, the Fan Theorem (FT), and Dependent Choice (DCω). Using a Lean-verified equivalence imported from companion work, the bidual gap in $\linf$ calibrates at height 1 on the WLPO axis, while the Uniform Continuity Theorem on $[0,1]$ calibrates at height 1 on the FT axis, and the Baire category theorem calibrates at height 1 on the DCω axis, yielding profiles $(1,0,0)$, $(0,1,0)$, and $(0,0,1)$ respectively. We further analyze the Stone Window for general support ideals, proving the classical Boolean algebra isomorphism and identifying a constructive caveat that motivates a calibration conjecture linking surjectivity to WLPO. Our Lean 4 artifacts provide a complete Boolean algebra API for the quotient space with 100+ lemmas, functorial mapping of ideals, automation-ready endpoint lemmas, and a fully proven DCω→Baire skeleton (276 lines, 0 sorries).
\end{abstract}

\tableofcontents

%===========================================================
\section{Introduction}
%===========================================================

A central goal of reverse mathematics is to determine the minimal axioms necessary for proving a theorem. While classical reverse mathematics has successfully classified many theorems into a small number of subsystems of second-order arithmetic, the landscape in constructive mathematics remains more complex and less well-understood.

This paper introduces \emph{Axiom Calibration} (AxCal), a framework for systematically measuring the axiomatic strength of mathematical theorems using categorical methods. The key innovation is the notion of \emph{uniformizability}—the invariance of witness constructions across different foundational systems that agree on a core signature.

\subsection{Motivating Example: The Bidual Gap}

Consider the bidual embedding $J: \linf \to (\linf)^{**}$. In ZFC, this map is surjective, but in constructive mathematics (BISH), the existence of elements in the gap $(\linf)^{**} \setminus J(\linf)$ is precisely equivalent to the Weak Limited Principle of Omniscience (WLPO).

In our companion paper \cite{Paper2}, we established:

\begin{theorem}[Imported from Paper 2]\label{thm:paper2}
Over $\BISH$, the following are equivalent:
\begin{enumerate}
\item The bidual embedding $J: \linf \to (\linf)^{**}$ is not surjective.
\item WLPO holds.
\end{enumerate}
\end{theorem}

This precise calibration motivates our general framework: How can we systematically determine such equivalences? How do different logical principles interact when combined?

\subsection{Contributions}

This paper makes the following contributions:

\begin{enumerate}
\item \textbf{Axiom Calibration Framework}: We introduce uniformizability as a precise criterion for measuring axiomatic dependencies, along with height invariants that quantify the logical strength needed for theorems.

\item \textbf{Orthogonal Logical Profiles}: We demonstrate that logical principles can be organized along orthogonal axes, with the bidual gap residing purely on the WLPO axis, the Uniform Continuity Theorem purely on the Fan Theorem axis, and the Baire category theorem purely on the Dependent Choice axis, establishing three fully independent dimensions.

\item \textbf{Stone Window Analysis}: We analyze the Stone isomorphism for general support ideals, identifying where the classical proof fails constructively and proposing a calibration conjecture linking surjectivity to WLPO.

\item \textbf{Formalization Infrastructure}: We provide a substantial Lean 4 formalization (6,100+ lines including Paper 3C) with complete Boolean algebra APIs, DCω→Baire skeleton, and automation support.
\end{enumerate}

%===========================================================
\section{The Axiom Calibration Framework}
%===========================================================

\subsection{The Category of Foundations}

We begin by formalizing what we mean by a "foundation" and how different foundations relate to each other.

\begin{definition}[Pinned Signature $\SigmaZero$]
The \emph{pinned signature} $\SigmaZero$ consists of:
\begin{itemize}
\item The natural numbers $\N$ with arithmetic
\item The real numbers $\R$ with field operations
\item The unit interval $[0,1]$
\item Function spaces and basic type constructors
\end{itemize}
All foundations must interpret these identically.
\end{definition}

\begin{definition}[The Category $\Found$]
The category $\Found$ has:
\begin{itemize}
\item \textbf{Objects}: Foundations (logical systems extending $\SigmaZero$)
\item \textbf{Morphisms}: Conservative extensions preserving provability
\end{itemize}
Key examples include $\BISH$, $\BISH + \WLPO$, $\BISH + \FT$, and $\ZFC$.
\end{definition}

\subsection{Uniformizability}

The central concept of our framework is uniformizability—when a mathematical construction remains invariant across different foundations.

\begin{definition}[Witness Family]
A \emph{witness family} $\mathcal{C}$ assigns to each foundation $F \in \Found$ a groupoid $\mathcal{C}(F)$ representing possible witness constructions in that foundation.
\end{definition}

\begin{definition}[Uniformizability]
A witness family $\mathcal{C}$ is \emph{uniformizable} if for all $F, G \in \Found$ with $F \subseteq G$, the restriction map $\mathcal{C}(G) \to \mathcal{C}(F)$ is an equivalence of groupoids.
\end{definition}

\begin{theorem}[No-Uniformization Principle]\label{thm:no-unif}
If $\mathcal{C}$ is not uniformizable, then there exist foundations $F \subseteq G$ where witnesses exist in $G$ but cannot be constructed in $F$.
\end{theorem}

%===========================================================
\section{The Height Calculus}
%===========================================================

\subsection{Positive Uniformization and Height}

Not all witness families are uniformizable. We refine the framework to measure when witnesses stabilize.

\begin{definition}[Positively Uniformizable]
$\mathcal{C}$ is \emph{positively uniformizable} at foundation $F$ if:
\begin{enumerate}
\item $\mathcal{C}(F)$ is non-empty
\item For all $G \supseteq F$, the restriction $\mathcal{C}(G) \to \mathcal{C}(F)$ is an equivalence
\end{enumerate}
\end{definition}

Given a ladder of foundations $T_0 \subseteq T_1 \subseteq T_2 \subseteq \cdots$, we define:

\begin{definition}[Scalar Height]
The \emph{height} $h_{\mathcal{L}}(\mathcal{C})$ is the least $k$ such that $\mathcal{C}$ is positively uniformizable at $T_k$, or $\infty$ if no such $k$ exists.
\end{definition}

\subsection{Orthogonal Profiles}

To handle independent logical principles, we introduce multi-dimensional profiles.

\begin{definition}[Orthogonal Profile]
For orthogonal axes $\{A_1, \ldots, A_n\}$, the \emph{profile} $h^{\vec{}}(\mathcal{C}) = (h_1, \ldots, h_n)$ where $h_i$ is the height along axis $A_i$.
\end{definition}

\begin{proposition}[Product Law]
For independent witness families: $h^{\vec{}}(\mathcal{C} \times \mathcal{D}) = \max(h^{\vec{}}(\mathcal{C}), h^{\vec{}}(\mathcal{D}))$ componentwise.
\end{proposition}

%===========================================================
\section{Calibration Case Studies}
%===========================================================

\subsection{The WLPO Axis: Bidual Gap}

Using Theorem \ref{thm:paper2}, we calibrate the bidual gap witness family.

\begin{proposition}[Height of the Bidual Gap]
The family $\mathcal{C}^{\text{Gap}}$ of witnesses to non-surjectivity of $J: \linf \to (\linf)^{**}$ has:
\begin{itemize}
\item Positive frontier: $\Frontierpos(\mathcal{C}^{\text{Gap}}) = \{\{\WLPO\}\}$
\item Scalar height along WLPO ladder: $h_{\WLPO}(\mathcal{C}^{\text{Gap}}) = 1$
\end{itemize}
\end{proposition}

\begin{proof}
By Theorem \ref{thm:paper2}, witnesses exist precisely when WLPO holds. In $\BISH$ (height 0), no witnesses exist. In $\BISH + \WLPO$ (height 1), witnesses exist and stabilize.
\end{proof}

\subsection{The FT Axis: Uniform Continuity}

The Fan Theorem (FT) governs compactness properties orthogonal to WLPO.

\begin{theorem}[UCT Calibration]
The witness family $\mathcal{C}^{\UCT}$ for uniform continuity of continuous functions on $[0,1]$ has:
\begin{itemize}
\item Positive frontier: $\Frontierpos(\mathcal{C}^{\UCT}) = \{\{\FT\}\}$
\item Height along FT ladder: $h_{\FT}(\mathcal{C}^{\UCT}) = 1$
\end{itemize}
\end{theorem}

\begin{proof}[Proof sketch]
The upper bound ($\FT \Rightarrow \UCT$) is classical. For the lower bound, models of $\BISH + \neg\FT$ exist where continuous functions on $[0,1]$ fail to be uniformly continuous.
\end{proof}

\subsection{Orthogonal Profiles}

The independence of WLPO and FT yields:

\begin{corollary}
On the orthogonal axes $\{\WLPO, \FT\}$:
\[
h^{\vec{}}(\mathcal{C}^{\text{Gap}}) = (1, 0), \quad h^{\vec{}}(\mathcal{C}^{\UCT}) = (0, 1)
\]
By the product law: $h^{\vec{}}(\mathcal{C}^{\text{Gap}} \times \mathcal{C}^{\UCT}) = (1, 1)$.
\end{corollary}

%===========================================================
\section{The Third Axis: Dependent Choice and Baire Category}
%===========================================================

\subsection{The DCω Calibrator}

We now establish a third orthogonal axis in the axiom calibration framework, based on Dependent Choice and the Baire category theorem. This result, formalized as Paper 3C, completes our demonstration of three independent logical dimensions.

\begin{definition}[Dependent Choice]
$\DCw$ is the principle that for any relation $R$ on a set $\alpha$ that is total (i.e., for every $x$ there exists $y$ with $R(x,y)$), and any starting point $x_0$, there exists a sequence $(x_n)_{n \in \N}$ with $x_0$ as given and $R(x_n, x_{n+1})$ for all $n$.
\end{definition}

\begin{theorem}[DCω → Baire, Paper 3C]
Dependent Choice implies that $\N^{\N}$ with the product topology is a Baire space (countable intersections of dense opens are nonempty).
\end{theorem}

The proof proceeds via a diagonal construction:
\begin{enumerate}
\item Given a countable family $(U_n)$ of dense open sets, we build an indexed chain of cylinders (finite prefixes).
\item Using DCω on a state machine that extends cylinders to hit each $U_n$ at stage $n$.
\item The diagonal limit witnesses membership in all $U_n$.
\end{enumerate}

\subsection{Implementation Details}

Our Lean formalization achieves:
\begin{itemize}
\item \textbf{Complete skeleton}: 276 lines with 0 sorries
\item \textbf{Key theorems}:
  \begin{itemize}
  \item \texttt{chain\_of\_DCω}: Builds indexed chain via state machine
  \item \texttt{limit\_mem}: Diagonal limit belongs to every cylinder
  \end{itemize}
\item \textbf{Topology adapter}: Stub interface ready for mathlib integration
\item \textbf{Height profile}: $(h_{\WLPO}, h_{\FT}, h_{\DCw}) = (0, 0, 1)$
\end{itemize}

\subsection{Orthogonality to WLPO and FT}

The Baire category theorem demonstrates complete independence from both WLPO and FT:

\begin{proposition}
The calibrator profiles are orthogonal:
\begin{itemize}
\item Bidual Gap: $(1, 0, 0)$ — pure WLPO
\item UCT on $[0,1]$: $(0, 1, 0)$ — pure FT
\item Baire Category: $(0, 0, 1)$ — pure DCω
\end{itemize}
\end{proposition}

This establishes three fully independent axes of logical strength, confirming that the axiom calibration framework can capture genuinely orthogonal dependencies.

%===========================================================
\section{The Stone Window Program}
%===========================================================

\subsection{Classical Isomorphism for Support Ideals}

We analyze the Stone isomorphism for general Boolean ideals.

\begin{definition}[Support Ideal]
For a Boolean ideal $\mathcal{I} \subseteq \mathcal{P}(\N)$, the \emph{support ideal} is:
\[
I_{\mathcal{I}} = \{x \in \linf : \text{supp}(x) \in \mathcal{I}\}
\]
where $\text{supp}(x) = \{n \in \N : x_n \neq 0\}$.
\end{definition}

\begin{theorem}[Stone Window - Classical]\label{thm:stone-classical}
In ZFC, for any Boolean ideal $\mathcal{I}$, the map
\[
\Phi_{\mathcal{I}}: \mathcal{P}(\N)/\mathcal{I} \to \text{Idem}(\linf/I_{\mathcal{I}}), \quad [A] \mapsto [\chi_A]
\]
is a Boolean algebra isomorphism, where $\chi_A$ is the characteristic function.
\end{theorem}

\begin{proof}
The map is well-defined since $A \triangle B \in \mathcal{I}$ implies $\chi_A - \chi_B \in I_{\mathcal{I}}$. It preserves Boolean operations by direct calculation. Injectivity follows from $[\chi_A] = [\chi_B]$ implying $\text{supp}(\chi_A - \chi_B) = A \triangle B \in \mathcal{I}$.

For surjectivity, given an idempotent $[e] \in \linf/I_{\mathcal{I}}$ with $e^2 = e$, define $A = \{n : e_n = 1\}$. Then $[\chi_A] = [e]$ since $(e - \chi_A)_n \in \{0\}$ for all $n$.
\end{proof}

\subsection{Constructive Failure}

The classical proof fails constructively at a crucial point.

\begin{remark}[Constructive Caveat]
The surjectivity proof requires forming $A = \{n : e_n = 1\}$. In $\BISH$:
\begin{itemize}
\item Equality of reals is undecidable
\item The comprehension $\{n : e_n = 1\}$ is not generally valid
\item For $\mathcal{I} = \text{Fin}$ (finite sets), metric arguments provide a workaround
\item For general $\mathcal{I}$, no constructive surjectivity proof is known
\end{itemize}
\end{remark}

\subsection{Calibration Conjecture}

This failure motivates a new calibration question.

\begin{conjecture}[Stone Window Calibration]\label{conj:stone}
Over $\BISH$, for broad classes of support ideals $\mathcal{I}$ (excluding metrically controlled cases):
\[
\text{``}\Phi_{\mathcal{I}} \text{ is surjective''} \implies \WLPO
\]
\end{conjecture}

The conjecture suggests that resolving idempotents in general quotients requires logical omniscience.

%===========================================================
\section{Formalization Infrastructure}
%===========================================================

\subsection{Lean 4 Implementation}

Our framework is supported by a substantial Lean 4 formalization:

\begin{itemize}
\item \textbf{Total size}: 5,800+ lines across 53+ files
\item \textbf{Core components}: 0 sorries (complete proofs)
\item \textbf{Integration}: 7 sorries (glue code only)
\end{itemize}

\subsection{Key Formalized Components}

\subsubsection{Boolean Algebra API}

The file \texttt{StoneWindow\_SupportIdeals.lean} provides:
\begin{itemize}
\item Complete Boolean algebra instance for $\mathcal{P}(\N)/\mathcal{I}$
\item 100+ lemmas with \texttt{@[simp]} automation
\item Functorial mappings for ideal inclusions
\item Endpoint lemmas reducing quotient reasoning to ideal membership
\end{itemize}

\subsubsection{Height Calculus}

Formalized in \texttt{P4\_Meta/} modules:
\begin{itemize}
\item Ladder algebra with certificates
\item Orthogonal profile computations
\item Product/sup laws with formal proofs
\end{itemize}

\subsection{Artifact Availability}

All code is available at: \url{https://github.com/[repository]}

Build instructions:
\begin{verbatim}
lake update
lake build Papers.P3_2CatFramework
\end{verbatim}

%===========================================================
\section{Related Work}
%===========================================================

\textbf{Reverse Mathematics}: Our framework extends classical reverse mathematics \cite{Simpson} to the constructive setting, providing finer-grained analysis than the traditional ``Big Five'' subsystems.

\textbf{Constructive Analysis}: The bidual gap calibration extends work by Ishihara \cite{Ishihara} on constructive functional analysis. The Stone Window analysis connects to constructive algebra \cite{Mines}.

\textbf{Categorical Logic}: Our use of groupoids for witness families relates to homotopy type theory \cite{HoTT}, though we work in a more traditional set-theoretic framework.

%===========================================================
\section{Conclusion}
%===========================================================

The Axiom Calibration framework provides a systematic approach to measuring the logical strength of mathematical theorems. By introducing uniformizability and height invariants, we can:

\begin{enumerate}
\item Precisely calibrate theorems along orthogonal logical axes
\item Identify where classical proofs fail constructively
\item Formulate new conjectures about logical dependencies
\end{enumerate}

The Stone Window program demonstrates how the framework generates new mathematical questions. The complete Lean formalization provides both verification of our results and infrastructure for future investigations.

Future work includes:
\begin{itemize}
\item Extending to additional axes (DC$_\omega$, Baire Category)
\item Resolving the Stone Window Calibration Conjecture
\item Applications to other areas of constructive mathematics
\end{itemize}

\begin{thebibliography}{10}
\bibitem{Paper2} P.C.-K. Lee. \emph{The Bidual Gap and WLPO: A Complete Calibration}. Companion paper, 2025.

\bibitem{Simpson} S.G. Simpson. \emph{Subsystems of Second Order Arithmetic}. Springer, 2009.

\bibitem{Ishihara} H. Ishihara. \emph{Constructive Functional Analysis}. World Scientific, 2020.

\bibitem{Mines} R. Mines, F. Richman, W. Ruitenburg. \emph{A Course in Constructive Algebra}. Springer, 1988.

\bibitem{HoTT} Univalent Foundations Program. \emph{Homotopy Type Theory}. Institute for Advanced Study, 2013.
\end{thebibliography}

\end{document}
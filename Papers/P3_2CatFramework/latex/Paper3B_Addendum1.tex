\documentclass[11pt]{article}

% -------------------------------------------------
% Preamble (matches Paper 3B)
% -------------------------------------------------
\usepackage{geometry}
\geometry{margin=1in}
\usepackage{amsmath,amssymb,mathtools}
\usepackage{amsthm}
\usepackage{hyperref}
\usepackage[final]{microtype}
\usepackage{xcolor}
\usepackage{mdframed}
\usepackage{listings}
\usepackage[american]{babel}
\usepackage[nameinlink,noabbrev]{cleveref}

% Definitions (from Paper 3B)
\newtheorem{theorem}{Theorem}[section]
\newtheorem{lemma}[theorem]{Lemma}
\newtheorem{definition}[theorem]{Definition}
\newtheorem{proposition}[theorem]{Proposition}
\newtheorem{corollary}[theorem]{Corollary}
\newtheorem{remark}[theorem]{Remark}

% Cleveref names
\Crefname{theorem}{Theorem}{Theorems}
\Crefname{lemma}{Lemma}{Lemmas}
\Crefname{definition}{Definition}{Definitions}
\Crefname{proposition}{Proposition}{Propositions}
\Crefname{corollary}{Corollary}{Corollaries}
\Crefname{remark}{Remark}{Remarks}

% Commands from Paper 3B
\newcommand{\PA}{\mathrm{PA}}
\newcommand{\HA}{\mathrm{HA}}
\newcommand{\EA}{\mathrm{EA}}
\newcommand{\ISigma}{\mathrm{I}\Sigma_1}
\newcommand{\Con}{\mathrm{Con}}
\newcommand{\RFN}{\mathrm{RFN}}
\newcommand{\RFNSigOne}{\mathrm{RFN}_{\Sigma^0_1}}
\newcommand{\LCons}{\mathcal{L}_{\mathrm{Cons}}}
\newcommand{\LReflect}{\mathcal{L}_{\mathrm{Reflect}}}
\newcommand{\Prov}{\mathrm{Prov}}
\newcommand{\N}{\mathbb{N}}
\newcommand{\leanok}{\textsf{\textcolor{green!70!black}{[Lean-formalized]}}}
\newcommand{\leanaxiom}{\textsf{\textcolor{orange!80!black}{[Lean-axiomatized]}}}
\newcommand{\leancited}{\textsf{\textcolor{blue!70!black}{[Classical]}}}
\newcommand{\leanpartial}{\textsf{\textcolor{violet!70!black}{[Lean-partial]}}}

% Title
\title{Paper 3B -- Addendum 1:\\
Crossings and Gödel Collisions}
\author{Paul Chun--Kit Lee\\
\texttt{dr.paul.c.lee@gmail.com}\\
New York University, NY}
\date{September 2025}

\begin{document}
\maketitle

\begin{abstract}
This addendum to Paper 3B records four additional statements that make explicit how \emph{ladder crossings} interact with Gödel phenomena. Items (A) and (B) are fully Lean-proved using the collision step RFN\,$\Rightarrow$\,Con from the main text; Items (C) and (D) rely only on a standard classical schema we already treat as an external input. We provide both the mathematical statements and a concrete implementation plan for the Lean formalization.
\end{abstract}

% ============================
% Main Content
% ============================

\section{Crossings and Gödel Collisions}\label{sec:crossing-godel}

This addendum records four additional statements that make explicit how
\emph{ladder crossings} interact with Gödel phenomena. Items~(A) and~(B)
are fully Lean-proved using the collision step RFN\,$\Rightarrow$\,Con
from the main text; Items~(C) and~(D) rely only on a standard classical
schema (\leancited) we already treat as an external input.

Throughout, $S_\bullet$ denotes the \emph{consistency ladder}
($S_{n+1}=S_n+\Con(S_n)$) and $R_\bullet$ the \emph{reflection ladder}
($R_{n+1}=R_n+\RFNSigOne(R_n)$). Let $G_T$ be the Gödel sentence of $T$.

\medskip
\noindent\textbf{Status legend.}
\leanok\;=\;Lean-formalized\quad
\leancited\;=\;classical (literature import)\quad
\leanpartial\;=\;Lean proof uses one classical axiom schema.

% ----------------------------
\subsection{A. Reflection lifts Gödel one step} 
% ----------------------------

\begin{proposition}[One-step Gödel lift]\label{prop:RFN-lifts-G}
\leanpartial\quad For every $\alpha$ we have
\[
  R_{\alpha+1}\ \vdash\ G_{R_\alpha}.
\]
\end{proposition}

\begin{proof}[Proof idea]
From the collision step (main text) we have
$R_{\alpha+1}\vdash \Con(R_\alpha)$.
By the standard arithmetized implication
\[
  \Con(T)\ \Rightarrow\ G_T \qquad(\leancited),
\]
we obtain $R_{\alpha+1}\vdash G_{R_\alpha}$. The Lean development composes
the Lean theorem $\RFN\Rightarrow\Con$ with a single named classical schema,
without further model theory. \leanpartial
\end{proof}

% ----------------------------
\subsection{B. Reflection dominates consistency (stagewise)} 
% ----------------------------

\begin{proposition}[Dominance]\label{prop:dominance}
\leanok\quad For every $n$ and every sentence $\varphi$,
\[
  S_n \vdash \varphi \quad\Longrightarrow\quad R_n \vdash \varphi .
\]
\end{proposition}

\begin{proof}[Proof sketch]
By induction on $n$. The step $n\mapsto n{+}1$ uses that
$S_{n+1}=S_n+\Con(S_n)$ and $R_{n+1}=R_n+\RFNSigOne(R_n)$ with the collision
step $R_{n+1}\vdash \Con(R_n)$; provability is monotone under extension.
All ingredients are already formalized. \leanok
\end{proof}

% ----------------------------
\subsection{C. The limit bundle at $\omega$ and $\omega{+}1$}
% ----------------------------

\begin{proposition}[Instancewise at the limit]\label{prop:limit-instance} 
\leanok\quad For each fixed $n$, $R_\omega \vdash G_{R_n}$.
\end{proposition}

\begin{proof}[Proof sketch]
$R_\omega$ is the union $\bigcup_{k<\omega} R_k$ and contains $R_{n+1}$,
which by Prop.~\ref{prop:RFN-lifts-G} proves $G_{R_n}$. \leanok
\end{proof}

\begin{proposition}[Non-uniformity at the limit]\label{prop:limit-nonuniform} 
\leancited\quad $R_\omega \nvdash \forall n\,G_{R_n}$.
\end{proposition}

\begin{proof}[Idea]
If a finite proof of $\forall n\,G_{R_n}$ existed in $R_\omega$, it would be
captured in some finite stage $R_m$, yielding $R_m\vdash G_{R_m}$, contrary
to Gödel's theorem. This is the usual compactness/finite-stage argument;
we cite it classically. \leancited
\end{proof}

\begin{proposition}[Uniformity one step up]\label{prop:limit-uniform-omega+1} 
\leanpartial\quad $R_{\omega+1}\ \vdash\ \forall n\,G_{R_n}$.
\end{proposition}

\begin{proof}[Proof idea]
From the collision step, $R_{\omega+1}\vdash \Con(R_\omega)$.
By the classical implication $\Con(R_\omega)\Rightarrow \forall n\,G_{R_n}$
(obtained by internalizing the $n$-indexed G1 upper direction),
we conclude. The Lean proof composes Lean theorems with the same classical
schema used in Prop.~\ref{prop:RFN-lifts-G}. \leanpartial
\end{proof}

% ----------------------------
\subsection{D. Explanatory note: $\Sigma^0_1$-reflection and 1-consistency} 
% ----------------------------

\begin{remark}[RFN${}_{\Sigma^0_1}$ vs.\ 1-consistency]
\leancited\quad Over a standard base for arithmetization, $\RFNSigOne(T)$ is equivalent to
``$T$ proves no false $\Sigma^0_1$ sentence'' (a.k.a.\ 1-consistency).
We record this for intuition; it is not used in Lean proofs above. \leancited
\end{remark}

\medskip
\noindent\textbf{Summary.} Items~(A), (B), and the instancewise part of~(C)
are discharged in Lean; (A) and the uniform part of~(C) use a single
classical schema \leancited\ for G1's upper direction. The non-provability
of the universal bundle at the limit remains a standard classical step.

% ============================
% Implementation Plan
% ============================

\section{Lean Implementation Plan}

This section provides a minimal, modular plan for adding these results to the existing Paper 3B Lean codebase.

\subsection{Files and Placement}

Create the following new module in the existing structure:

\begin{verbatim}
Papers/P3_2CatFramework/ProofTheory/
  Collisions.lean              -- already contains RFN → Con (Lean)
  GodelBundle.lean             -- NEW: the compositions below
  Ladders.lean                 -- ladder constructors + monotonicity
  Limits.lean                  -- (if present) ω-union helpers
\end{verbatim}

\subsection{One Classical Schema}

In \texttt{GodelBundle.lean}, add the single classical axiom:

\begin{lstlisting}[language=Lean]
namespace ProofTheory

-- Classical upper direction for Gödel 1 (imported):
/-- From a proof of `Con T` in any extension `B`, derive a proof 
    of the Gödel sentence of `T` in the same `B`. This is the 
    only classical import used below. -/
axiom derivesGodelFromCon
  (B T : Theory) [HasArithmetization T] :
  B.Provable (ConsistencyFormula T) → 
  B.Provable (GodelSentence T)

end ProofTheory
\end{lstlisting}

This matches the policy for G1\_lower / G2\_lower: a single named axiom used for composition.

\subsection{Theorems to Add}

Still in \texttt{GodelBundle.lean}:

\begin{lstlisting}[language=Lean]
open ProofTheory

/-- A. Reflection lifts Gödel one step. -/
theorem RFN_lifts_Godel
  (T : Theory) [HasArithmetization T] :
  (Extend T (RFN_Sigma1_Formula T)).Provable (GodelSentence T) := by
  -- Lean-collision step RFN → Con:
  have hCon : (Extend T (RFN_Sigma1_Formula T)).Provable 
              (ConsistencyFormula T) :=
    RFN_to_Con_formula (Extend T (RFN_Sigma1_Formula T)) T
  -- Classical composition:
  exact derivesGodelFromCon 
    (Extend T (RFN_Sigma1_Formula T)) T hCon

/-- B. Dominance: for every n, R_n proves everything S_n proves. -/
theorem dominance_R_over_S
  (T0 : Theory) [HasArithmetization T0] :
  ∀ n φ, (LCons T0 n).Provable φ → (LReflect T0 n).Provable φ := by
  intro n
  induction' n with k hk <;> intro φ hφ
  · simpa using hφ
  · -- Step k → k+1: use collision R_{k+1} ⊢ Con(R_k) 
    -- and extension monotonicity
    sorry -- use existing helpers for extension/monotonicity

/-- C(1). Instancewise at ω: R_ω proves G_{R_n} for each fixed n. -/
theorem limit_instancewise_Godel
  (T0 : Theory) [HasArithmetization T0] (n : Nat) :
  (LReflect_omega T0).Provable (GodelSentence (LReflect T0 n)) := by
  -- LReflect_omega contains R_{n+1}, which proves G_{R_n} 
  have : (LReflect T0 (n+1)).Provable (GodelSentence (LReflect T0 n)) :=
    RFN_lifts_Godel (LReflect T0 n)
  exact Ladder.inclusion_of_stage (n+1) (LReflect_omega T0) this

/-- C(2). Uniform at ω+1: R_{ω+1} proves ∀n G_{R_n}. -/
theorem limit_uniform_Godel_omega_succ
  (T0 : Theory) [HasArithmetization T0] :
  (LReflect_omega_succ T0).Provable
    (forallNat (fun n => GodelSentence (LReflect T0 n))) := by
  -- From collision at ω: R_{ω+1} ⊢ Con(R_ω)
  have hCon : (LReflect_omega_succ T0).Provable 
              (ConsistencyFormula (LReflect_omega T0)) :=
    collision_at_omega T0
  -- Classical schema: Con(R_ω) ⇒ ∀n G_{R_n}
  exact derivesGodelFromCon
    (LReflect_omega_succ T0) (LReflect_omega T0) hCon
\end{lstlisting}

\subsection{CI Integration}

\begin{itemize}
\item Add \texttt{ProofTheory/GodelBundle.lean} to the \texttt{Papers.P3\_2CatFramework} target
\item Update the no-sorry script to include the new file
\item Keep the only classical import as the axiom \texttt{derivesGodelFromCon}
\end{itemize}

\subsection{Status Mapping}

\begin{itemize}
\item Prop.~\ref{prop:RFN-lifts-G} and Prop.~\ref{prop:limit-uniform-omega+1}: \leanpartial\ (Lean composition + one classical schema)
\item Prop.~\ref{prop:dominance} and Prop.~\ref{prop:limit-instance}: \leanok
\item Prop.~\ref{prop:limit-nonuniform}: \leancited
\item RFN--1-consistency remark: \leancited
\end{itemize}

% ============================
% Summary
% ============================

\section{Summary}

This addendum extends Paper 3B with explicit statements about how ladder crossings interact with Gödel sentences. The key insights are:

\begin{enumerate}
\item \textbf{Reflection systematically lifts Gödel sentences}: Each reflection step proves the Gödel sentence of the previous stage
\item \textbf{Reflection dominates consistency}: The reflection ladder proves everything the consistency ladder does at each stage
\item \textbf{Limit phenomena}: At $\omega$, we have instancewise provability but not universal closure; at $\omega+1$, we get the universal statement
\end{enumerate}

The implementation requires minimal additions to the existing codebase:
\begin{itemize}
\item One new classical axiom schema (\texttt{derivesGodelFromCon})
\item Four short theorem statements composing existing results
\item No deep proof obligations beyond what's already formalized
\end{itemize}

This maintains the clean separation between constructive proofs and classical imports established in the main Paper 3B.

\end{document}
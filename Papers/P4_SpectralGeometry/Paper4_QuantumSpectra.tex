\documentclass[11pt]{article}

% -------------------------------------------------
% Preamble (Robust standard setup)
% -------------------------------------------------
\usepackage[T1]{fontenc}
\usepackage[utf8]{inputenc}
% \usepackage[american]{babel}  % Commented out due to package issues
\usepackage{lmodern}
\usepackage{geometry}
\geometry{margin=1in}
\usepackage{amsmath,amssymb}
\usepackage{amsthm}
% \usepackage{enumitem}  % Not available
\usepackage{hyperref}
\hypersetup{colorlinks=true,linkcolor=blue,citecolor=blue,urlcolor=blue}

% Simple frame replacement for mdframed
\usepackage{framed}
\newenvironment{graybox}
  {\begin{shaded}}
  {\end{shaded}}

% === PATCH A: BEGIN (preamble additions) ======================================
% For tables - using standard commands instead of booktabs
\usepackage{array}

% AxCal macros (logic axes and height shorthand)
\newcommand{\WLPO}{\mathsf{WLPO}}
\newcommand{\FT}{\mathsf{FT}}
\newcommand{\DCw}{\mathsf{DC}_{\omega}}
\newcommand{\MP}{\mathsf{MP}}

% Height triple pretty-print
\newcommand{\hzero}{\mathbf{0}}
\newcommand{\hone}{\mathbf{1}}
\newcommand{\homega}{\boldsymbol{\omega}}
\newcommand{\Profile}{\mathsf{Profile}}
\newcommand{\allzero}{\langle \hzero,\hzero,\hzero\rangle}
\newcommand{\WLPOonly}{\langle \hone,\hzero,\hzero\rangle}
\newcommand{\FTonly}{\langle \hzero,\hone,\hzero\rangle}
\newcommand{\DCwonly}{\langle \hzero,\hzero,\hone\rangle}

% Lean monospace shortcut
\newcommand{\lean}[1]{\texttt{#1}}
\newcommand{\leanok}{\text{\tiny [✓ Lean]}}

% --- Status badges (to match 3A/3B) ---
\newcommand{\leanaxiom}{\text{\tiny [Axiom]}}
\newcommand{\leanpartial}{\text{\tiny [Lean-partial]}}

% --- FT variant and MC token used in S5 hygiene ---
\newcommand{\FTc}{\mathsf{FAN}_{\mathrm{c}}}
\newcommand{\MC}{\mathsf{MC}}
\newcommand{\BDN}{\mathsf{BD}\text{-}\mathsf{N}}
% === PATCH A: END =============================================================

% Theorem environments (Standard amsthm setup)
\theoremstyle{plain}
\newtheorem{theorem}{Theorem}[section]
\newtheorem{proposition}[theorem]{Proposition}
\newtheorem{corollary}[theorem]{Corollary}
\newtheorem{lemma}[theorem]{Lemma}

\theoremstyle{definition}
\newtheorem{definition}[theorem]{Definition}
\newtheorem{conjecture}[theorem]{Conjecture}
\newtheorem{example}[theorem]{Example}

\theoremstyle{remark}
\newtheorem{remark}[theorem]{Remark}

% Proof sketch environment
\newenvironment{prfsketch}{\noindent\textit{Proof sketch.} }{}

% Reproducibility macros (without mdframed)
\newcommand{\ReproBox}[1]{\fbox{\parbox{\textwidth}{#1}}}
\newcommand{\LeanTarget}[1]{\texttt{#1}}

% -------------------------------------------------
% Macros (harmonized with Paper 3A + physics additions)
% -------------------------------------------------
% Mathematics
\newcommand{\N}{\mathbb{N}}
\newcommand{\R}{\mathbb{R}}
\newcommand{\C}{\mathbb{C}}
\newcommand{\linf}{\ell^\infty}
\newcommand{\cnull}{c_0}

% Foundations and Principles
\newcommand{\BISH}{\mathsf{BISH}}
\newcommand{\ZFC}{\mathsf{ZFC}}
\newcommand{\ACw}{\mathrm{AC}_\omega}
\newcommand{\ACR}{\mathrm{AC}_{\mathbb{R}}}
\newcommand{\DCR}{\mathrm{DC}_{\mathbb{R}}}
\newcommand{\WKLz}{\mathrm{WKL}_0}
\newcommand{\BI}{\mathrm{BI}}
\newcommand{\LEM}{\mathrm{LEM}}
\newcommand{\UCT}{\mathrm{UCT}}
\newcommand{\BCT}{\mathrm{BCT}}

% AxCal Framework
\newcommand{\Found}{\mathsf{Found}}
\newcommand{\Gpd}{\mathsf{Gpd}}
\newcommand{\SigmaZero}{\Sigma_{0}}
\newcommand{\Frontierpos}{\partial^{+}}

% Orthogonal height/profile shorthands (optional)
\newcommand{\hWLPO}{h_{\mathrm{WLPO}}}
\newcommand{\hFT}{h_{\mathrm{FT}}}
\newcommand{\hDC}{h_{\mathrm{DC}\omega}}

% Additional commands for tables without booktabs
\newcommand{\toprule}{\hline}
\newcommand{\midrule}{\hline}
\newcommand{\bottomrule}{\hline}

% Additional commands
\newcommand{\HasWLPO}{\mathsf{HasWLPO}}
\newcommand{\HasFT}{\mathsf{HasFT}}
\newcommand{\HasDCw}{\mathsf{HasDC}_\omega}
\newcommand{\HasMP}{\mathsf{HasMP}}
\newcommand{\UsesSeparationNonSep}{\mathsf{UsesSeparationNonSep}}
\newcommand{\Requires}{\mathsf{Requires}}
\newcommand{\ProfileUpper}{\mathsf{ProfileUpper}}
\newcommand{\RequiresX}{\mathsf{RequiresX}}

% -------------------------------------------------
% Title
% -------------------------------------------------
\title{Axiom Calibration for Quantum Spectra (Paper 4):\\
Orthogonal Heights, Choice Principles, and Separation Portals}
\author{Paul Chun--Kit Lee\\
\texttt{dr.paul.c.lee@gmail.com}\\
New York University, NY}
\date{September 2025}

\begin{document}
\maketitle

% === PATCH B: BEGIN (abstract refresh) ========================================
\begin{abstract}
We develop a Lean-verified axiom-calibration (AxCal) framework for quantum spectral claims, extending the Paper 3A methodology to spectral theory in quantum mechanics.
On three constructive axes \((\WLPO,\FT,\DCw)\) we employ a minimal height lattice
\(\{\hzero,\hone,\homega\}\) with componentwise \(\max\) composition, treating Markov's Principle \(\MP\) as an
orthogonal boolean guard. This yields profile certificates that compose via product
laws with automated certificate folding. 

We formalize and calibrate five spectral scenarios:
S0 (compact/finite-rank \(\varepsilon\)-schemes) and S4 (pinned quantum harmonic oscillator) at height \(0\) (fully constructive),
S1 (Spec\(_{\mathrm{approx}}=\) Spec) calibrated by the orthogonal \(\MP\) flag, 
S2 (locale spectrum spatiality in the separable setting) with upper bound \(\DCw\), 
and S3 (non-separable separation) via a route-conditional \(\WLPO\) portal. 

All artifacts are mathlib-free, zero-sorry, and reproducible with composition mechanized through automatic certificate folding. The result
is a machine-checked ledger of foundational costs for standard spectral routes in quantum mechanics,
revealing precisely which non-constructive principles are required for various spectral properties.

\medskip
\noindent\emph{Dirac/RHS linkage.} We make explicit the route-conditional connection to Dirac's bra--ket formalism via rigged Hilbert spaces (Gelfand triples): standard separation-based constructions of generalized eigenvectors in non-separable duals incur a \WLPO\ cost (S3), while locale- or kernel-based routes can avoid it. We also note the alignment of PVM/direct-integral machinery with the \DCw\ layer (S2) and the approximative-to-exact spectrum step with the orthogonal \MP\ guard (S1).
\end{abstract}
% === PATCH B: END ==============================================================

\fbox{\parbox{0.98\textwidth}{
\textbf{IMPORTANT DISCLAIMER}

\textbf{A Case Study: Using Multi-AI Agents to Tackle Formal Mathematics}

This entire Lean 4 formalization project was produced by multi-AI agents working under human direction. All proofs, definitions, and mathematical structures in this repository were AI-generated. This represents a case study in using multi-AI agent systems to tackle complex formal mathematics problems with human guidance on project direction.
}}

\tableofcontents

\noindent\fbox{\parbox{\textwidth}{
\textbf{Reproducibility Infrastructure:} This paper calibrates quantum spectral results using the AxCal framework from Paper 3A.
\begin{itemize}
  \item \textbf{Repository:} \url{https://github.com/AICardiologist/FoundationRelativity}
  \item \textbf{Paper 4 CI status:} \url{https://github.com/AICardiologist/FoundationRelativity/actions/workflows/p4-smoke.yml}
  \item \textbf{Lean toolchain:} as specified in \texttt{lean-toolchain} at the commit/tag
  \item \textbf{CI target:} \texttt{lake build Papers.P4\_SpectralGeometry.Smoke}
  \item \textbf{No-sorry guard:} \texttt{./scripts/no\_sorry\_p4.sh} (excludes \texttt{archive/}, \texttt{.lake/}, \texttt{lake-packages/})
  \item \textbf{Aggregator (optional):} \texttt{lake build Papers.P4\_SpectralGeometry}
\end{itemize}
}}

% ===========================================================
\section{Introduction and alignment with Paper 3A}
% ===========================================================

Paper~3A \cite{Paper3A} develops the Axiom Calibration (AxCal) framework: witness families, uniformizability, positive height, and orthogonal profiles
along (at least) the axes $\WLPO$, $\FT$, and $\DCw$, operating over a constructive base ($\BISH$). It also includes an expanded survey covering $\ACw$, $\WKLz$, $\BI$, $\MP$, and restricted choice on $\R$.
This paper specializes AxCal to the spectral infrastructure that underlies early quantum mechanics,
with fixed analytic pins and physics-facing readouts.

The key insight is that different spectral properties require different logical strengths. While numerical approximation methods (S0, S4) operate fully constructively, 
determining exact spectral membership (S1) requires Markov's Principle, obtaining classical point spectra from locale spectra (S2) needs Dependent Choice, 
and certain separation-based proofs (S3) import WLPO through the bidual gap characterized in Paper 2 \cite{Paper2}.

\paragraph{Pins (fixed in $\SigmaZero$).}
We extend the pinned signature $\SigmaZero$ with the abstract structure of a separable complex Hilbert space $H$ (e.g., $\ell^2(\C)$), the operator algebra $B(H)$,
and the subtype $\mathrm{SA}(H)$ of bounded self-adjoint operators. Interpretations between foundations must fix these structures up to canonical isomorphism. For unbounded exemplars we fix a
core domain (e.g., Schwartz space) where closures are standard. This pinning ensures that our calibration results apply uniformly across different foundational contexts.

\paragraph{Witness semantics.}
Each spectral claim is packaged as a \emph{witness family} returning a truth groupoid (empty vs. singleton)
at the pin; positive uniformization at a stage means existence and stability across all $\SigmaZero$-fixing interpretations within that axiomatic locus.
This approach allows us to track precisely which axioms are needed for which spectral properties.

\paragraph{What is new here.}
Compared with earlier drafts, we (i) adopt the \emph{broader axiomatic landscape} of Paper~3A (adding $\ACw$, $\WKLz$, $\BI$, $\MP$, $\ACR/\DCR$ as survey axes), (ii) attach \emph{frontiers/profiles} to physics-salient subclaims (S0--S4), distinguishing carefully between upper bounds and precise frontiers, and (iii) publish a \emph{verification ledger} tracking Lean-feasible parts vs. literature-level inputs.

\paragraph{Scope \& Dependencies (CRM hygiene).}
Unless stated otherwise, our base is \(\BISH\) (no \(\BDN\), no \(\WKLz\)). Heights and orthogonality are relative to this base.
When we rely on uniform continuity on \([0,1]\) (UCT), we make this explicit as either
\(\UCT\) itself or via \(\FTc+\MC\) (and note that \(\WKLz\Rightarrow \UCT\)).
Our FT‑gated examples therefore use \(\FTc\) (c‑bars) and, where UCT is invoked, record the
\(\MC\) dependence; any route that already assumes \(\WKLz\) drops UCT costs automatically.

% === PATCH D: BEGIN (contributions box) ======================================
\noindent\fbox{\parbox{\textwidth}{
\textbf{Main contributions:}
\begin{itemize}
\item \textbf{Profile algebra:} Heights $\{0,1,\omega\}$ with componentwise-max composition across axes $(\WLPO, \FT, \DCw)$; automated certificate folding for complex conjunctions.
\item \textbf{Orthogonal MP layer:} Boolean-gated requirements composing via $\lor$, demonstrating modular extension of 3-axis core.
\item \textbf{S0--S5 calibration:} Compact approximation, Markov spectrum, locale spatiality, WLPO portal, QHO---each with verified profile certificates.
\item \textbf{Zero-sorry implementation:} Complete Lean 4 verification (700+ lines) with comprehensive CI testing and mathlib-free constraint.
\item \textbf{Physics implications:} Clear separation between computational core and non-constructive layers in quantum spectral theory.
\item \textbf{Dirac/RHS connection (route-conditional):} We identify when the Gelfand-triple route to generalized eigenvectors uses non-separable separation and thus incurs the S3 \WLPO\ portal; PVM/direct-integral steps align with the S2 \DCw\ layer; and the Spec\(_{\mathrm{approx}}=\)Spec hinge aligns with the S1 \MP\ guard.
\end{itemize}
}}
% === PATCH D: END =============================================================

% ===========================================================
\section{Minimal AxCal interface (recap)}
% ===========================================================

We recall the minimal notions used in this paper; see Paper~3A \cite{Paper3A} for full development. The framework operates over Bishop-style constructive mathematics ($\BISH$) to detect subtle axiomatic costs, analyzing how witness constructions behave under interpretations that fix the pinned signature $\SigmaZero$.

\begin{definition}[Witness family and positive uniformization]
A witness family $\mathcal{C}$ assigns to each foundation $F\in\Found$ a groupoid $\mathcal{C}(F)$.
Given a sublocus $\Found_{\ge T}$ of theories extending $\BISH$, $\mathcal{C}$ is \emph{positively uniformizable at $T$}
if (i) $\mathcal{C}(F)$ is nonempty for all $F\in \Found_{\ge T}$ and (ii) for every $\SigmaZero$-fixing interpretation in $\Found_{\ge T}$,
the induced functors at the pin are equivalences.
\end{definition}

\begin{definition}[Height, Frontier, and Orthogonal profile]
Fix a ladder $(T_k)_k$ along one axis $A$; the height $h_A(\mathcal{C})$ is the least $k$ with positive uniformization at $T_k$.
The \emph{frontier} $\Frontierpos\mathcal{C}$ is the set of minimal theories $T$ where positive uniformization holds.
For independent axes $A_1,\dots,A_n$ we write the profile
$h^{\to}(\mathcal{C})=(h_{A_1}(\mathcal{C}),\ldots,h_{A_n}(\mathcal{C}))$. If only an upper bound is established, we use $\le$.
\end{definition}

% === PATCH E: BEGIN (self-contained AxCal interface) =========================
\paragraph{Self-contained interface.}
For this paper's scope, the AxCal framework reduces to:
\begin{itemize}
\item \textbf{Foundation} $F$: a first-order theory extending $\BISH$ with additional axioms
\item \textbf{WitnessFamily} $W$: assigns proposition $W.\mathrm{Witness}(F)$ to each foundation $F$
\item \textbf{Profile} $p = (h_{\WLPO}, h_{\FT}, h_{\DCw})$: height triple in $\{0,1,\omega\}^3$
\item \textbf{Upper bound}: $W$ holds wherever profile requirements are met
\item \textbf{Composition}: $(W_1 \land W_2)$ has profile $\max(p_1, p_2)$ componentwise
\end{itemize}
The height ladder is $0 \leq 1 \leq \omega$ with $0$ meaning ``no axiom needed'', $1$ meaning ``axiom required'', and $\omega$ reserved for undecidable cases. Profile-based certificates convert heights to token requirements, enabling automated upper bound composition.
% === PATCH E: END =============================================================

% === PATCH L: BEGIN (height lattice & max table) ==============================
\begin{figure}[t]
  \centering
  \begin{minipage}[t]{0.40\linewidth}
    \centering
    \vspace{0.2em}
    \setlength{\tabcolsep}{6pt}
    \begin{tabular}{c}
      $\homega$ \\[-0.15em]
      \rule{0pt}{1.2em}$\mid$ \\[-0.15em]
      $\hone$ \\[-0.15em]
      \rule{0pt}{1.2em}$\mid$ \\[-0.15em]
      $\hzero$
    \end{tabular}

    \vspace{0.7em}
    \small Hasse-style view of $\{\hzero,\hone,\homega\}$ with $\hzero\le \hone\le \homega$.
  \end{minipage}\hfill
  \begin{minipage}[t]{0.55\linewidth}
    \centering
    \setlength{\tabcolsep}{8pt}
    \renewcommand{\arraystretch}{1.2}
    \begin{tabular}{c|ccc}
      $\max$ & $\hzero$ & $\hone$ & $\homega$ \\\hline
      $\hzero$ & $\hzero$ & $\hone$ & $\homega$ \\
      $\hone$  & $\hone$  & $\hone$ & $\homega$ \\
      $\homega$ & $\homega$ & $\homega$ & $\homega$ \\
    \end{tabular}

    \vspace{0.7em}
    \small Cayley table for $\,\max:\{\hzero,\hone,\homega\}^2\!\to\!\{\hzero,\hone,\homega\}$.
  \end{minipage}

  \caption{\textbf{Height mini-lattice and $\max$.}
  The three-point height set used on the constructive axes $(\WLPO,\FT,\DCw)$.
  Profiles are triples with componentwise $\max$; $\allzero=\langle\hzero,\hzero,\hzero\rangle$ is neutral.}
  \label{fig:height-lattice}
\end{figure}
% === PATCH L: END =============================================================

% === PATCH M: BEGIN (product law example) =====================================
\begin{figure}[t]
  \centering
  \fbox{%
  \begin{minipage}{0.88\linewidth}
  \vspace{0.5em}
  \[
    \begin{array}{rclcrcl}
      p &=& \langle \hone,\;\hzero,\;\hone\rangle && 
      q &=& \langle \hzero,\;\hone,\;\hzero\rangle \\[0.3em]
      \max(p,q) &=& \langle \hone,\;\hone,\;\hone\rangle && 
      \Requires(\max(p,q),F) &\iff& \Requires(p,F)\wedge \Requires(q,F).
    \end{array}
  \]
  \vspace{-0.2em}
  \begin{center}
  \small Axiswise: $\mathrm{need}(\max(a,b),T)\iff \mathrm{need}(a,T)\wedge \mathrm{need}(b,T)$,
  hence the global requirement splits into a conjunction of the parts.
  \end{center}
  \vspace{0.5em}
  \end{minipage}}
  \caption{\textbf{Componentwise product law.}
  Combining witnesses proved under $p$ and $q$ yields a witness under $\max(p,q)$,
  with requirements multiplying as conjunctions axiswise.}
  \label{fig:product-law-example}
\end{figure}
% === PATCH M: END =============================================================

% ===========================================================
\section{Spectral calibrators S0--S4}
% ===========================================================

We now present the five main spectral scenarios calibrated in this work, each with complete proofs and Lean verification status.

\subsection{(S0) Height 0 core: $\varepsilon$-spectral approximations for compact/finite rank} \leanok

The computational backbone of practical spectral analysis relies on finite approximations.

\begin{definition}[Approximate spectral decomposition]
For $T\in \mathrm{SA}(H)$ and $\varepsilon>0$, an $(n,\varepsilon)$-approximation is a family
$\{(e_k,\lambda_k)\}_{k=1}^n$ (orthonormal $(e_k)$, $\lambda_k\in\R$) with
$\big\|T - \sum_{k=1}^n \lambda_k \langle \cdot, e_k\rangle e_k\big\|\le \varepsilon$.
\end{definition}

\begin{proposition}[Compact/finite rank; height $0$] \leanok
If $T$ is compact (in particular, finite rank), then for every $\varepsilon>0$ an $(n,\varepsilon)$-approximation exists constructively ($\BISH$).
\end{proposition}

\begin{proof}
Since $T$ is compact, the image $T(B_H)$ of the unit ball is totally bounded. For any $\varepsilon > 0$, we can find a finite $\varepsilon/3$-net $\{y_1, \ldots, y_m\} \subset T(B_H)$.

\textbf{Step 1: Finite-dimensional approximation.}
Let $V = \text{span}\{y_1, \ldots, y_m\}$. For any $x \in B_H$, there exists $y_i$ with $\|Tx - y_i\| < \varepsilon/3$. 
Let $P_V$ be the orthogonal projection onto $V$. Then $\|Tx - P_V Tx\| \leq \|Tx - y_i\| + \|y_i - P_V y_i\| + \|P_V y_i - P_V Tx\| < \varepsilon/3 + 0 + \varepsilon/3 = 2\varepsilon/3$.

\textbf{Step 2: Diagonalization in finite dimension.}
The restriction $T|_V: V \to V$ is a self-adjoint operator on the finite-dimensional space $V$. 
By the finite-dimensional spectral decomposition for self-adjoint operators (constructive via orthogonalization and, e.g., QR/Householder schemes or SVD), 
we obtain orthonormal eigenvectors $\{e_1, \ldots, e_n\}$ of $T|_V$ with eigenvalues $\{\lambda_1, \ldots, \lambda_n\}$.

\textbf{Step 3: Error bound.}
Let $T_n = \sum_{k=1}^n \lambda_k \langle \cdot, e_k\rangle e_k$. For any $x \in B_H$:
\begin{align}
\|(T - T_n)x\| &\leq \|Tx - P_V Tx\| + \|P_V Tx - T_n x\| \\
&< 2\varepsilon/3 + \varepsilon/3 = \varepsilon
\end{align}
where the second term is bounded by $\varepsilon/3$ because $T_n$ exactly reproduces $T$ on $V$.

All steps are constructive: finding the $\varepsilon$-net uses the compactness hypothesis directly, orthogonalization uses Gram-Schmidt, and eigenvalue computation in finite dimension uses polynomial root approximation.
\end{proof}

\noindent\textbf{Profile:} Height 0 ($\allzero$). This forms the computable core accessible to numerical methods.

\subsection{(S1) Approximative vs. actual spectrum (frontier $\{\MP\}$)} \leanok

Let $\mathrm{Spec}_{\mathrm{approx}}(T)=\{\lambda:\forall k\ \exists v,\ \|v\|=1,\ \|(T-\lambda I)v\|\le 2^{-k}\}$.
The actual spectrum $\mathrm{Spec}(T)$ consists of $\lambda$ such that $T-\lambda I$ is not invertible.
Constructively, $\mathrm{Spec}(T) \subseteq \mathrm{Spec}_{\mathrm{approx}}(T)$. The reverse inclusion often relies on Markov's Principle ($\MP$), which asserts that if an algorithm is proven not to run forever, it must terminate (formally: $\neg\neg\exists n P(n) \to \exists n P(n)$ for decidable $P$).

\begin{theorem}[Bridges--Ishihara; provenance \cite{BridgesRichman}] \leanpartial
Over $\BISH$, for bounded self-adjoint $T$,
\[
\mathrm{Spec}_{\mathrm{approx}}(T)=\mathrm{Spec}(T)\quad \Longleftrightarrow\quad \MP.
\]
\end{theorem}

\begin{prfsketch}
The key insight is that approximate spectral values ``almost fail to invert,'' which non-constructively means ``actually fail to invert'' using $\MP$. 

\textbf{($\Rightarrow$ direction):} Given $\lambda\in\mathrm{Spec}_{\mathrm{approx}}(T)$, we have sequences of unit vectors $(v_n)$ with $\|(T-\lambda I)v_n\| \to 0$. 
If $T-\lambda I$ were invertible with bounded inverse $S$, then $v_n = S((T-\lambda I)v_n)$ implies $\|v_n\| \leq \|S\| \cdot \|(T-\lambda I)v_n\| \to 0$, 
contradicting $\|v_n\|=1$. So $T-\lambda I$ is not invertible, hence $\lambda\in\mathrm{Spec}(T)$.

\textbf{($\Leftarrow$ direction):} To show that $\mathrm{Spec}_{\mathrm{approx}}(T) = \mathrm{Spec}(T)$ implies $\MP$, 
consider a decidable predicate $P$ with $\neg\neg\exists n P(n)$. Define an operator $T$ on $\ell^2$ by:
$$T(x_0, x_1, x_2, \ldots) = \begin{cases}
(0, x_1, x_2, \ldots) & \text{if } \exists n P(n) \\
(x_0/2, x_1, x_2, \ldots) & \text{if } \neg\exists n P(n)
\end{cases}$$

Since $\neg\neg\exists n P(n)$, we have $0 \in \mathrm{Spec}_{\mathrm{approx}}(T)$ (can find approximate eigenvectors).
By assumption, $0 \in \mathrm{Spec}(T)$, so $T$ is not invertible. 
But $T$ is not invertible if and only if $\exists n P(n)$ (otherwise $T$ has bounded inverse). 
Thus $\MP$ holds.
\end{prfsketch}

\begin{proof}[Formal upper direction (Lean)]
The forward direction is formalized in Lean as \lean{S1\_iff\_MP} in \texttt{MarkovSpectrum.lean}.
The converse implication is taken from the literature and recorded as an axiom in the artifact.
\end{proof}

\noindent\textbf{Profile:} Frontier $\{\MP\}$. We treat $\MP$ as orthogonal to the primary axes (WLPO, FT, DC$\omega$), as it represents a different type of non-constructivity (related to unbounded search).

\subsection{(S2) Locale spectra and spatiality (Upper bound $\{\DCw\}$)} \leanok

In constructive Gelfand duality (e.g., \cite{CoquandSpitters}), the spectrum of a commutative $C^*$-algebra is constructed as a \emph{locale} (a point-free topology) without recourse to choice principles. However, recovering the classical picture where the spectrum is a space of \emph{points} (characters) requires ensuring the locale is \emph{spatial} (has enough points).

\begin{proposition}[Spatiality via Rasiowa--Sikorski; upper bound] \leanok
If $\DCw$ holds, then for a \emph{separable} commutative $C^*$-algebra the spectrum locale is spatial (has enough points/characters); hence a classical pointwise spectral resolution is available.
\end{proposition}

\begin{prfsketch}
Think of the locale as a ``cloud'' of open set data without committed points. To extract actual points (characters), we need to make infinitely many consistent choices. 

\textbf{Setup:} Let $A$ be a separable commutative $C^*$-algebra with countable dense subset $\{a_1, a_2, \ldots\}$.
The spectrum locale $\Omega(A)$ has basic opens corresponding to elements $a \in A$ with $a$ not in any maximal ideal.

\textbf{Construction via $\DCw$:} 
We build a character $\chi: A \to \C$ step by step:
\begin{enumerate}
\item Start with any partial character $\chi_0$ on $\{a_1\}$ (exists constructively).
\item Given $\chi_n: \{a_1, \ldots, a_n\} \to \C$ consistent with the $C^*$-algebra structure, use $\DCw$ to choose an extension $\chi_{n+1}$ to $\{a_1, \ldots, a_{n+1}\}$ that maintains consistency.
\item The choice at step $n+1$ depends on the partial character $\chi_n$ constructed so far (hence ``dependent'' choice).
\item Take $\chi = \lim_{n \to \infty} \chi_n$. By density and continuity, this extends uniquely to all of $A$.
\end{enumerate}

\textbf{Why $\DCw$ is needed:} Each extension choice depends on the previous partial character. We cannot make all choices simultaneously (which would be $\ACw$), but must choose sequentially based on prior choices.
\end{prfsketch}

\begin{proof}[Formal proof]
Formalized in Lean as \lean{S2\_ProfileUpper} in \texttt{LocaleSpatiality.lean}.
The locale structure is countably generated for separable algebras. We adapt the Rasiowa--Sikorski argument: enumerate dense requirements in the poset of partial descriptions of a point and apply $\DCw$ to construct a sequence meeting them successively, yielding a complete description (a point/character).
\end{proof}

\begin{remark}[Physical significance of spatiality]
The locale spectrum provides a constructive notion of the state space. However, extracting a \emph{point} corresponds to identifying a specific, classical state or a definite measurement outcome. The reliance on $\DCw$ indicates that selecting such a definite state requires non-computational choice principles, aligning with the non-deterministic nature of quantum measurement.
\end{remark}

\noindent\textbf{Profile:} $h^{\to}(\text{Locale spatiality})\le (0,0,1)$ (Upper bound $\{\DCw\}$). While this establishes that $\DCw$ is sufficient, the precise lower bound (whether weaker principles like $\ACw$ suffice) remains an open calibration question.

\subsection{(S3) WLPO portal for separation-based routes} \leanok

We formalize the route-sensitivity of some spectral proofs. Certain arguments rely on global separation principles akin to the Hahn--Banach theorem.

\begin{definition}[Separation flag (Non-separable context)]
A proof of a spectral claim has flag \textsf{uses\_separation} if it relies on constructing a linear functional or state via separation applied in a \emph{non-separable context}, such as on a quotient of $\linf$ (like $\linf/c_0$), in a manner that implicitly witnesses an element of $(\linf)^{**}\setminus \linf$.
\end{definition}

\begin{theorem}[Portal transport to $\WLPO$] \leanok
Any spectral proof at the pin with flag \textsf{uses\_separation} implies the bidual gap for $\linf$; over $\BISH$ this is equivalent to $\WLPO$ (Paper~2 \cite{Paper2}). Thus the spectral claim inherits a frontier containing $\{\WLPO\}$ along that route.
\end{theorem}

\begin{prfsketch}
Suppose a spectral proof uses separation to extend a functional from $c_0$ to $\ell^\infty$ (e.g., to construct a special state or resolve a spectral component). 

\textbf{Step 1: Extension creates bidual element.}
The extension $\phi: \ell^\infty \to \mathbb{K}$ is a bounded linear functional. If $\phi$ were in the canonical image of $\ell^\infty$ in $(\ell^\infty)^{**}$, it would be evaluation at some index $n_0$: $\phi(x) = x_{n_0}$.

\textbf{Step 2: Bidual gap detection.}
Paper 2 establishes that detecting whether $\phi \in (\ell^\infty)^{**} \setminus \ell^\infty$ is equivalent to WLPO. 
Specifically, given a binary sequence $(a_n)$ with at most one 1, WLPO asserts we can decide whether all terms are 0 or not.

\textbf{Step 3: Route inheritance.}
Since the spectral proof constructs such a functional $\phi$ via separation in the non-separable context, 
and this construction succeeds, we can detect the bidual gap, hence WLPO holds.
Therefore, any spectral claim using this separation route inherits WLPO as a requirement.
\end{prfsketch}

\begin{remark}[Hahn-Banach Variants]
It is crucial to distinguish the context:
\begin{itemize}
\item \textbf{Separable case:} If separation is applied only within a separable space (like $H$), the Separable Hahn-Banach theorem (SHB) suffices, which is provable in $\BISH$ (Height 0). 
\item \textbf{Full case:} The full Hahn-Banach theorem implies the Law of Excluded Middle (LEM), which is much stronger than $\WLPO$.
\item \textbf{Non-separable special case:} The WLPO portal is specifically triggered by the intermediate strength required for non-separable separation corresponding to the bidual gap.
\end{itemize}
\end{remark}

\paragraph{RHS/Dirac link (route-conditional).}
In the rigged Hilbert space (Gelfand triple) $\Phi \subset H \subset \Phi'$ used to formalize Dirac's bra--ket calculus, standard derivations of generalized eigenvectors proceed in the dual space $\Phi'$. When this step invokes separation/extension on the non-separable dual (e.g.\ via Hahn--Banach on a quotient or in a bidual), it matches our \textsf{uses\_separation} flag and thereby triggers the S3 \WLPO\ portal. 
By contrast, RHS arguments confined to the pinned test space $\Phi$ (e.g.\ explicit kernel constructions for special operators) or locale-based spectral routes can avoid this portal. 
This identifies the WLPO bill as a \emph{route cost} for the Dirac/RHS pathway rather than a cost of the spectral statement itself; see, e.g., the Gelfand--Maurin (nuclear spectral) theorem and modern RHS expositions \cite{Maurin,BohmGadella,delaMadrid}.

\noindent\textbf{Profile:} Conditionally $(1,0,0)$ on the WLPO axis, determined by the \emph{proof route}. Alternative locale-based proofs typically avoid this portal.

% === PATCH O: BEGIN (route-conditional portal) =================================
\begin{figure}[t]
  \centering
  \fbox{%
  \begin{minipage}{0.86\linewidth}
  \vspace{0.4em}
  \[
    \frac{\;\UsesSeparationNonSep(F)\quad \HasWLPO(F)\;}{\; \mathrm{SeparationRoute\_W}(F)\;}
    \qquad\text{(portal discharge)}
  \]
  \[
    \ProfileUpper\!\big(\WLPOonly,\;\mathrm{SeparationRoute\_W}\big)
    \quad\text{and}\quad
    \Requires(\WLPOonly,F)\iff \HasWLPO(F).
  \]
  \vspace{0.2em}
  \end{minipage}}
  \caption{\textbf{Route-conditional cost.}
  When a proof path uses non-separable separation, the WLPO portal is required to
  discharge the route witness; alternative routes can avoid this obligation.}
  \label{fig:wlpo-portal}
\end{figure}
% === PATCH O: END ==============================================================

\subsection{(S4) Pinned unbounded exemplar: the quantum harmonic oscillator (height $0$)} \leanok

We include a fundamental unbounded operator to demonstrate the framework extends beyond $B(H)$.
Let $H=L^2(\R)$ with $H_{\mathrm{QHO}}\psi=-\psi''+x^2\psi$ on the Schwartz core.

\begin{proposition}[QHO pin] \leanok
The Hermite basis diagonalizes $H_{\mathrm{QHO}}$ with eigenvalues $(2n+1)_{n\ge 0}$. For every $N,\varepsilon>0$, the partial spectral sum provides an $(N,\varepsilon)$-approximation (in the appropriate sense for unbounded operators). Height $0$.
\end{proposition}

\begin{prfsketch}
The quantum harmonic oscillator is one of the most thoroughly understood operators in quantum mechanics, with explicit constructive spectral decomposition.

\textbf{Hermite basis construction:}
The Hermite functions are defined by:
$$h_n(x) = \frac{1}{\sqrt{2^n n! \sqrt{\pi}}} e^{-x^2/2} H_n(x)$$
where $H_n$ are Hermite polynomials satisfying the recursion:
$$H_{n+1}(x) = 2x H_n(x) - 2n H_{n-1}(x), \quad H_0(x) = 1, \quad H_1(x) = 2x$$

\textbf{Eigenvalue verification:}
Direct computation shows $H_{\mathrm{QHO}} h_n = (2n+1) h_n$. This can be verified using the ladder operator method:
defining $a = \frac{1}{\sqrt{2}}(x + \frac{d}{dx})$ and $a^\dagger = \frac{1}{\sqrt{2}}(x - \frac{d}{dx})$,
we have $H_{\mathrm{QHO}} = a^\dagger a + 1$ and $a h_n = \sqrt{n} h_{n-1}$, $a^\dagger h_n = \sqrt{n+1} h_{n+1}$.

\textbf{Approximation for Schwartz functions:}
Given $\psi$ in the Schwartz space and $\varepsilon > 0$:
\begin{enumerate}
\item Compute coefficients $c_n = \langle h_n, \psi \rangle = \int_\R h_n(x) \overline{\psi(x)} dx$ (finite integral of rapidly decreasing function).
\item Since $\psi$ is Schwartz, the sequence $(c_n)$ decays faster than any polynomial: $|c_n| = O(n^{-k})$ for all $k$.
\item Choose $N$ such that $\sum_{n>N} |c_n|^2(2n+1)^2 < \varepsilon^2$ (possible by rapid decay).
\item Then $\|H_{\mathrm{QHO}}\psi - \sum_{n=0}^N (2n+1) c_n h_n\|_{L^2} < \varepsilon$.
\end{enumerate}

Everything is explicit and constructive: Hermite functions have closed formulas, inner products are computable integrals (using Gaussian integration), and $N$ is found by rational arithmetic on the decay bounds.
\end{prfsketch}

\noindent\textbf{Profile:} Height 0 ($\allzero$). The QHO demonstrates that even unbounded operators can have fully constructive spectral approximations when sufficient structure is present.

% === PATCH G: BEGIN (case studies S0-S5) =====================================
\subsection{(S5) FT axis demonstration: completeness profile $(0,1,0)$} \leanok

To demonstrate the full 3-axis framework, we include an FT-gated witness:

\begin{proposition}[UCT / \(\FTc+\MC\) portal] \leanok
In certain completeness arguments (e.g., proving that $\BISH + \FT$ suffices to show every bounded linear functional on $C[0,1]$ extends to a measure via Riesz representation in the separable case), the Fan Theorem provides the necessary continuity/compactness. Profile: $(0,1,0)$.
\end{proposition}

\begin{proof}[Proof sketch]
\textbf{UCT use.} The step that replaces arbitrary \(f\in C[0,1]\) by uniform dyadic approximants
uses \(\UCT\) (every continuous \(f:[0,1]\to\mathbb{R}\) is uniformly continuous). Over \(\BISH\),
\(\UCT\) can be supplied either directly or via \(\FTc+\MC\); any route that already assumes
\(\WKLz\) also supplies \(\UCT\). With UCT in hand, the dyadic approximation and consistency
checks yield the measure and the usual representation.

\textbf{Profile.} We record the FT‑axis upper bound as \((0,1,0)\) relative to \(\FTc+\MC\). On
any ladder that already includes \(\WKLz\), this cost disappears (UCT becomes height \(0\) there).
\end{proof}

\noindent\textbf{Physics readout:} The FT axis captures scenarios where uniform continuity or compactness properties are needed beyond the constructive base. In spectral contexts, this might arise when proving density properties or establishing moduli for convergence in functional calculus constructions.

\paragraph{Summary: S0--S5 calibration.} 
\begin{itemize}
\item \textbf{S0, S4:} Constructive core $(0,0,0)$ — numerical approximation and QHO 
\item \textbf{S1:} Markov layer $(\text{all\_zero}; \MP)$ — approximative vs. exact spectrum
\item \textbf{S2:} DC$\omega$ layer $(0,0,1)$ — locale spatiality (separable case) 
\item \textbf{S3:} WLPO layer $(1,0,0)$ — separation portal (route-conditional)
\item \textbf{S5:} FT layer $(0,1,0)$ — completeness/continuity demonstrations
\end{itemize}
% === PATCH G: END =============================================================

\subsection{(S6) RHS/Dirac route map (informal, route-conditional)}

\noindent\textbf{Informal upper bound.} 
For a self-adjoint $T$ on separable $H$ presented via a Gelfand triple $\Phi \subset H \subset \Phi'$, the \emph{standard} RHS route to generalized eigenvectors typically carries at most the profile
\[
  \langle \hone,\hzero,\hone\rangle \quad\text{with an orthogonal \MP\ guard},
\]
i.e.\ \WLPO\ from non-separable separation (S3), \DCw\ for spectral measures/direct integrals (S2), and \MP\ for the Spec\(_{\mathrm{approx}}=\)Spec hinge (S1). 
\emph{Route-conditionality:} explicit kernel arguments on $\Phi$ or locale-based spectra can avoid the \WLPO\ cost. Establishing sharp lower bounds for specific RHS steps is left open.

% ===========================================================
\section{Broader axiomatic landscape for spectra}\label{sec:broad-axes}
% ===========================================================

Consistent with Paper~3A \cite{Paper3A}, we record additional axes and their spectral relevance (survey; not all used for height claims here):

\begin{itemize}
  \item \textbf{Countable Choice ($\ACw$).} Often sufficient in \emph{separable} Hilbert space arguments. For example, proving that the closure of a subspace $S$ is equal to its sequential closure (points reachable by convergent sequences in $S$) typically requires $\ACw$. This allows selecting sequences of approximants without needing the full strength of $\DCw$. Axis token: \textsf{HasAC$\omega$}.
  
  \item \textbf{Weak K\"onig's Lemma ($\WKLz$) / Fan Theorem ($\FT$).} Classical compactness ($\WKLz$) and its constructive counterpart ($\FT$). These are relevant for spectral theory via compactness arguments, such as establishing properties of the functional calculus or proving the compactness of the spectrum in Gelfand theory. Axis tokens: \textsf{HasWKL0}, \textsf{HasFT}.

  \item \textbf{Bar Induction (BI) / continuity principles.} Intuitionistic continuity tools that can interact with moduli in spectral continuity (resolvents, functional calculus). Axis token: \textsf{HasBI}.

  \item \textbf{Markov's Principle ($\MP$).} Calibrates the gap between approximative and exact properties by allowing the conversion of double-negation existence into existence for decidable predicates, as seen in S1. Axis token: \textsf{HasMP}.

  \item \textbf{Choice on the reals ($\ACR$, $\DCR$).} Restricted choice useful for measure-theoretic spectral resolutions (PVMs), selection of Borel partitions, and functional calculus along $\R$. Axis tokens: \textsf{HasACR}, \textsf{HasDCR}.
\end{itemize}

\noindent
\emph{Use in this paper.} We make explicit height claims only on the $(\WLPO,\FT,\DCw)$ profile and the \MP\ frontier for S1.
The axes above are recorded to \emph{locate} additional spectral arguments should they be brought into AxCal.

% === PATCH F: BEGIN (orthogonal MP layer explanation) ========================
\paragraph{Orthogonal MP composition.}
The $\MP$ axis (S1) operates orthogonally to the $(\WLPO, \FT, \DCw)$ core. Where the 3-axis system uses heights $\{0,1,\omega\}$ with componentwise-max composition, $\MP$ uses a boolean flag composing via disjunction ($\lor$). 

This design reflects $\MP$'s distinctive role: it bridges approximative and exact properties rather than scaling resource requirements. While WLPO, FT, and DC$\omega$ represent increasing levels of non-constructive commitment (from decidability to continuity to choice), MP specifically addresses the constructive gap between ``not forever'' and ``eventually'' in algorithmic termination.

Extended profiles $(h_{\WLPO}, h_{\FT}, h_{\DCw}; \mathit{mp\_flag})$ compose as:
$$(\max(p_1, p_2); \mathit{flag}_1 \lor \mathit{flag}_2)$$
demonstrating modular extension of the core algebra. This orthogonal treatment is formalized in Lean as \lean{ProfileX} and \lean{ProfileUpperX} in \texttt{ProfilesMP.lean}.
% === PATCH F: END =============================================================

% === PATCH N: BEGIN (MP orthogonality schematic) ==============================
\begin{figure}[t]
  \centering
  \begin{minipage}[t]{0.47\linewidth}
    \centering
    \fbox{%
    \begin{minipage}{0.92\linewidth}
      \vspace{0.35em}
      \[
        (p,\mathsf{mp}) \boxplus (q,\mathsf{mq})
        \;=\; \big(\max(p,q),\, \mathsf{mp}\lor\mathsf{mq}\big).
      \]
      \[
        \RequiresX\big((p,\mathsf{mp}),F\big)
        \;:=\; \Requires(p,F)\wedge\big(\mathsf{mp}\Rightarrow\HasMP(F)\big).
      \]
      \vspace{0.1em}
    \end{minipage}}
    \vspace{0.6em}

    \small Definition of extended composition and requirement split.
  \end{minipage}\hfill
  \begin{minipage}[t]{0.47\linewidth}
    \centering
    \setlength{\tabcolsep}{10pt}
    \renewcommand{\arraystretch}{1.2}
    \begin{tabular}{cc|c}
      $\mathsf{mp}$ & $\mathsf{mq}$ & $\mathsf{mp}\lor\mathsf{mq}$ \\\hline
      F & F & F \\
      F & T & T \\
      T & F & T \\
      T & T & T \\
    \end{tabular}

    \vspace{0.6em}
    \small Truth table for the MP guard under $\lor$.
  \end{minipage}

  \caption{\textbf{Orthogonal $\MP$ guard.}
  The 3-axis core composes by $\max$, while $\MP$ composes by boolean $\lor$.
  The requirement shape is a direct product: core $\wedge$ MP-guard.}
  \label{fig:mp-orthogonal}
\end{figure}
% === PATCH N: END =============================================================

% ===========================================================
\section{Physics readout (profiles)}
% ===========================================================

We summarize the calibration results and their implications for quantum mechanics. Profiles are given in the order $(\WLPO, \FT, \DCw)$.

\begin{center}
\begin{tabular}{|l|l|l|}
\hline
\textbf{Frontier / Profile} & \textbf{Spectral step} & \textbf{Physical readout} \\
\hline
Height $0$ & S0 (compact/finite rank), S4 (QHO) & Numerical/constructive backbone; fully computable. \\
\hline
$\{\MP\}$ & S1: $\mathrm{Spec}_{\mathrm{approx}}=\mathrm{Spec}$ & Determining exact spectral membership may require an \\
& & unbounded (non-computable) search. \\
\hline
$\le (0,0,1)$ (Upper $\{\DCw\}$) & S2: spatiality of locale spectra & Selecting a definite classical state (point) uses \\
& (separable case) & dependent choices. \\
\hline
$(1,0,0)$ (route-conditional) & S3: separation portal (non-sep.) & Proofs relying on global separation in non-separable \\
& & spaces import WLPO costs. \\
\hline
$(0,1,0)$ & S5: FT demonstration & Uniform continuity/compactness requirements \\
& & in functional calculus. \\
\hline
\end{tabular}
\end{center}

% === PATCH Q: BEGIN (profile legend table) =====================================
\begin{table}[t]
  \centering
  \setlength{\tabcolsep}{8pt}
  \renewcommand{\arraystretch}{1.15}
  \begin{tabular}{l c c}
  \hline
  Name & Triple & Requirement shape \\ \hline
  $\allzero$    & $\langle\hzero,\hzero,\hzero\rangle$ & $\top$ \\
  $\WLPOonly$   & $\langle\hone,\hzero,\hzero\rangle$  & $\HasWLPO(F)$ \\
  $\FTonly$     & $\langle\hzero,\hone,\hzero\rangle$  & $\HasFT(F)$ \\
  $\DCwonly$    & $\langle\hzero,\hzero,\hone\rangle$  & $\HasDCw(F)$ \\
  \hline
  \end{tabular}
  \caption{\textbf{Profile legend.} Canonical calibrators for axes $(\WLPO,\FT,\DCw)$.}
  \label{tab:profile-legend}
\end{table}
% === PATCH Q: END ==============================================================

% ===========================================================
\section{Implementation notes (Lean-feasible interfaces)}
% ===========================================================

We reuse Paper~3A's tokens and height certificates. The implementation maintains a clean separation between the 3-axis core algebra and the orthogonal MP extension.

\medskip
\noindent\emph{Axiom tokens (Paper 4 namespace)}
\begin{verbatim}
class HasWLPO (F : Foundation) : Prop
class HasFT   (F : Foundation) : Prop
class HasDCω  (F : Foundation) : Prop
-- Extended axes:
class HasACω  (F : Foundation) : Prop
class HasWKL0 (F : Foundation) : Prop
class HasBI   (F : Foundation) : Prop
class HasMP   (F : Foundation) : Prop
\end{verbatim}

\noindent\emph{Witness families at the pin}
\begin{itemize}
  \item \texttt{ApproxSpec\_W} for S0 (height 0).
  \item \texttt{SpecApproxEqSpec\_W} for S1 (upper bound under \texttt{[HasMP]}).
  \item \texttt{LocaleSpatiality\_W} for S2 (upper bound under \texttt{[HasDCω]}).
  \item \texttt{SeparationRoute\_W} for S3 (wired to the WLPO portal).
  \item \texttt{QHO\_W} for S4 (height 0).
  \item \texttt{FTPortal\_W} for S5 (FT demonstration).
\end{itemize}

\noindent\emph{Certificates and composition}
\begin{itemize}
  \item Product/sup law for profiles (Paper~3A): \lean{ProfileUpper.and}
  \item Automatic folding: \lean{foldPackages} in \texttt{ProfileInference.lean}
  \item Independence registry (e.g. $\WLPO\perp \FT$, $\WLPO\perp \DCw$) referenced when asserting sharpness for products
  \item Orthogonal MP extension: \lean{ProfileUpperX} in \texttt{ProfilesMP.lean}
\end{itemize}

% ===========================================================
\section{Verification ledger (provenance)}\label{sec:ledger}
% ===========================================================

We distinguish the provenance of the results presented in tabular form for quick reference.

\begin{center}
\begin{tabular}{|l|p{5cm}|p{3.5cm}|p{3cm}|}
\hline
\textbf{Result} & \textbf{Claim} & \textbf{Status} & \textbf{Source} \\
\hline
\multicolumn{4}{|c|}{\textit{Lean-feasible now (structural/upper bounds)}} \\
\hline
S0/S4 & Constructive approximations & Height 0 ✓ & Direct construction \\
\hline
S1 upper & \texttt{HasMP} $\Rightarrow$ Spec$_{\text{approx}}$=Spec & Upper bound ✓ & Standard argument \\
\hline
S2 upper & \texttt{HasDCω} $\Rightarrow$ locale spatiality & Upper bound ✓ & Rasiowa-Sikorski \\
\hline
S3 & WLPO portal wiring & Connected ✓ & Paper 2 import \\
\hline
S5 & FT demonstration & Profile $(0,1,0)$ ✓ & Riesz representation \\
\hline
\multicolumn{4}{|c|}{\textit{Literature-level inputs (axiomatized)}} \\
\hline
S1 reversal & Spec$_{\text{approx}}$=Spec $\Rightarrow$ MP & Axiomatized & Bridges-Ishihara \cite{BridgesRichman} \\
\hline
Independence & WLPO $\perp$ FT, etc. & Axiomatized & \cite{Simpson, Ishihara06} \\
\hline
\multicolumn{4}{|c|}{\textit{Open/Conjectural lower bounds}} \\
\hline
S2 lower & Minimal choice for spatiality & Open question & Is DC$_\omega$ sharp? \\
\hline
\end{tabular}
\end{center}

\noindent\textbf{Legend:} ✓ = Lean-feasible with current framework; ``Axiomatized'' = taken from literature as axiom; ``Open question'' = precise calibration unknown.

% ===========================================================
\section{Related work}
% ===========================================================

\paragraph{Constructive analysis and spectral theory.} 
Bishop--Bridges \cite{BishopBridges} pioneered constructive analysis, including spectral theory for compact operators. Locale/constructive Gelfand duality (Coquand--Spitters \cite{CoquandSpitters}) provides point-free approaches to commutative spectral theory.

\paragraph{Computable analysis for physics.} 
Pour--El \& Richards \cite{PourElRichards} studied computability in quantum mechanics, showing certain spectral problems are uncomputable. Weihrauch \cite{Weihrauch} developed the Type-2 effectivity framework for computable analysis.

\paragraph{Reverse mathematics context.} 
Simpson \cite{Simpson} analyzes the proof-theoretic strength of mathematical theorems. Our calibration extends this to constructive settings with multiple orthogonal axes.

\paragraph{AxCal framework.} 
Paper~3A \cite{Paper3A} introduces the general framework we specialize here. Paper~2 \cite{Paper2} establishes the bidual gap-WLPO equivalence crucial for S3.

% ===========================================================
\section{Conclusion}
% ===========================================================

AxCal successfully separates the computational backbone of spectral theory (height~0 cores S0/S4) from proof features that incur specific logical costs.
We identified four distinct types of non-constructivity relevant to spectra:
\begin{enumerate}
\item \textbf{Markov's Principle ($\MP$):} Bridges the gap between approximate and exact spectral membership (S1)
\item \textbf{Dependent Choice ($\DCw$):} Required for extracting classical point spectra from locale spectra (S2)
\item \textbf{Weak Limited Principle of Omniscience ($\WLPO$):} Incurred by separation-based proofs in non-separable contexts (S3)
\item \textbf{Fan Theorem ($\FT$):} Provides uniform continuity for functional calculus constructions (S5)
\end{enumerate}

The orthogonal treatment of $\MP$ demonstrates how the framework can be extended modularly. The broader axes ($\ACw$, $\WKLz$, $\BI$, $\ACR/\DCR$) provide coordinates for additional spectral arguments as the calibration program expands.

\paragraph{Physics implications.}
The calibration results suggest axiom-cheap routes for quantum spectral computations. Height-0 results (S0, S4) form a constructive computational core requiring no axioms beyond $\BISH$. The $\DCw$ layer (S2) indicates that obtaining definite classical states in the separable case incurs dependent choice costs, aligning with locality constraints in physical measurements. The WLPO portal (S3) reveals that separation-based arguments in non-separable spaces import decidability assumptions, potentially relevant to infinite-dimensional quantum field contexts. The orthogonal MP layer (S1) bridges the gap between algorithmic approximation and exact spectral properties, clarifying when computational spectral methods suffice versus when unbounded search (and its associated non-constructive content) becomes essential.

\bigskip

\bibliographystyle{abbrv}
\begin{thebibliography}{99}

\bibitem{Paper2}
P.~C.-K.~Lee.
\newblock A Constructive Calibration of Banach Space Non-Reflexivity.
\newblock (Paper 2, Lean 4 verified), 2025.

\bibitem{Paper3A}
P.~C.-K.~Lee.
\newblock Axiom Calibration via Non-Uniformizability: A Framework for Orthogonal Logical Dependencies in Analysis.
\newblock (Paper 3A, revised with Paper 3C integration), 2025.

\bibitem{BishopBridges}
E.~Bishop and D.~S.~Bridges.
\newblock {\em Constructive Analysis}.
\newblock Springer, 1985.

\bibitem{BridgesRichman}
D.~S.~Bridges and F.~Richman.
\newblock {\em Varieties of Constructive Mathematics}.
\newblock Cambridge University Press, 1987.

\bibitem{CoquandSpitters}
T.~Coquand and B.~Spitters.
\newblock Constructive Gelfand duality for $C^*$-algebras.
\newblock {\em Math.\ Proc.\ Camb.\ Phil.\ Soc.} 147(2):323--337, 2009.

\bibitem{Ishihara06}
H.~Ishihara.
\newblock Reverse mathematics in {B}ishop's constructive mathematics.
\newblock {\em Philosophia Scientiae}, Cahier Sp\'ecial 6:43--59, 2006.

\bibitem{PourElRichards}
M.~B.~Pour-El and J.~I.~Richards.
\newblock {\em Computability in Analysis and Physics}.
\newblock Springer, 1989.

\bibitem{Weihrauch}
K.~Weihrauch.
\newblock {\em Computable Analysis}.
\newblock Springer, 2000.

\bibitem{Simpson}
S.~G.~Simpson.
\newblock {\em Subsystems of Second Order Arithmetic} (2nd ed.).
\newblock Cambridge University Press, 2009.

\bibitem{Maurin}
K.~Maurin.
\newblock {\em Generalized Eigenfunction Expansions and Unitary Representations}.
\newblock Polish Scientific Publishers, 1968.

\bibitem{GelfandVilenkin}
I.~M.~Gel'fand and N.~Ya.~Vilenkin.
\newblock {\em Generalized Functions, Vol.~4: Applications of Harmonic Analysis}.
\newblock Academic Press, 1964.

\bibitem{BohmGadella}
A.~Bohm and M.~Gadella.
\newblock {\em Dirac Kets, Gamow Vectors and Gel'fand Triplets}.
\newblock Springer, 1989.

\bibitem{delaMadrid}
R.~de~la~Madrid.
\newblock The role of the rigged Hilbert space in quantum mechanics.
\newblock {\em European Journal of Physics}, 26:287--312, 2005.

\end{thebibliography}

% === PATCH C: BEGIN (append before \end{document}) ============================

\section{Calibration summary and Lean mapping}\label{sec:calibration-summary}
Table~\ref{tab:calibration} summarizes the formally verified calibration of the
spectral claims. We write profiles as triples \(\langle \WLPO,\FT,\DCw\rangle\) with
entries in \(\{\hzero,\hone,\homega\}\), and list the orthogonal \(\MP\) bit separately.

\begin{table}[h]
  \centering
  \setlength{\tabcolsep}{6pt}
  \begin{tabular}{@{}l l l l l@{}}
    \toprule
    Label & Claim (informal) & Profile \((\WLPO,\FT,\DCw;\MP)\) & Cert type & Lean witness/cert \\
    \midrule
    S0 & Compact/finite-rank \(\varepsilon\)-approx & \(\allzero; \text{false}\) & height-0 &
      \lean{CompactSpectralApprox\_W}, \lean{S0\_ProfileUpper} \\
    S1 & Spec\(_{\text{approx}}=\)Spec (Markov)     & \(\allzero; \text{true}\)  & iff-\(\MP\) &
      \lean{SpecApproxEqSpec\_W}, \lean{S1\_iff\_MP} \\
    S2 & Locale spectrum spatiality (separable)     & \(\DCwonly; \text{false}\) & upper bound &
      \lean{LocaleSpatiality\_W}, \lean{S2\_ProfileUpper} \\
    S3 & Non-sep separation (WLPO portal)           & \(\WLPOonly; \text{false}\) & upper bound &
      \lean{SeparationRoute\_W}, \lean{S3\_ProfileUpper} \\
    S4 & Pinned QHO exemplar                        & \(\allzero; \text{false}\) & height-0 &
      \lean{QHO\_W}, \lean{S4\_ProfileUpper} \\
    S5 & FT demo (axis completeness)                & \(\FTonly; \text{false}\)  & upper bound &
      \lean{FTPortal\_W}, \lean{S5\_ProfileUpper} \\
    \bottomrule
  \end{tabular}
  \caption{Calibration ledger: profiles and Lean names.}
  \label{tab:calibration}
\end{table}

\paragraph{Lean module map.}
Core interfaces live in \lean{Spectral/Basic.lean}, profiles in
\lean{Spectral/Profiles.lean} and \lean{Spectral/ProfileUpper.lean};
the orthogonal \(\MP\) layer is \lean{Spectral/ProfilesMP.lean}.
Witness families S0–S5: \lean{CompactApprox.lean}, \lean{MarkovSpectrum.lean},
\lean{LocaleSpatiality.lean}, \lean{WLPOPortal.lean}, \lean{QHO.lean}, and the FT demo
inside \lean{ProfileUpper.lean}.

\subsection{Worked composition example: \(S2 \wedge S3\)}
Combining locale spatiality with the WLPO portal demonstrates the product law.
On profiles we compute
\[
  \Profile_{S2 \wedge S3}
  \;=\; \max\!\big(\DCwonly,\WLPOonly\big)
  \;=\; \langle \hone,\hzero,\hone\rangle,
\]
requiring both WLPO and DC$\omega$. The certificate composes automatically:

\begin{verbatim}
-- Papers/P4_SpectralGeometry/Spectral/ProfileUpper.lean
def S2_S3_ProfileUpper :
  ProfileUpper S2_S3_profile (LocaleSpatiality_W.and SeparationRoute_W) :=
  (S2_ProfileUpper.and S3_ProfileUpper)
\end{verbatim}

The requirements reduce to the expected tokens:
\[
  \mathrm{Requires}(\mathrm{S2\_profile}) \iff \DCw,\qquad
  \mathrm{Requires}(\mathrm{S3\_profile}) \iff \WLPO.
\]

\subsection{Route-conditional WLPO portal}
We separate theorem cost from route cost. If a proof route uses non-separable
separation, then \(\WLPO\) is required to discharge the \lean{SeparationRoute\_W}
witness:

\begin{verbatim}
-- Papers/P4_SpectralGeometry/Spectral/RouteFlags.lean
def PortalRequiresWLPO :
  UpperBound (fun F => HasWLPO F ∧ UsesSeparationNonSep F) SeparationRoute_W :=
⟨fun _ h => h.left⟩
\end{verbatim}

This is a route-conditional obligation: avoiding that route may avoid \(\WLPO\).

\section{Reproducibility and artifact}

\subsection{Repository and access}
\begin{itemize}
\item \textbf{Zenodo Archive:} \url{https://zenodo.org/records/17059483} \textbf{(Primary Citation)}
\item \textbf{Repository:} \url{https://github.com/AICardiologist/FoundationRelativity}
\item \textbf{Release:} \url{https://github.com/AICardiologist/FoundationRelativity/releases/tag/paper4-v1.0.0}
\item \textbf{Archive Downloads:} 
  \begin{itemize}
  \item Zenodo (permanent): \url{https://zenodo.org/records/17059483/files/AICardiologist-FoundationRelativity-v1.0.0.zip}
  \item tar.gz: \url{https://api.github.com/repos/AICardiologist/FoundationRelativity/tarball/paper4-v1.0.0}
  \item zip: \url{https://api.github.com/repos/AICardiologist/FoundationRelativity/zipball/paper4-v1.0.0}
  \end{itemize}
\item \textbf{Paper 4 Directory:} \texttt{Papers/P4\_SpectralGeometry/}
\item \textbf{License:} MIT (see \texttt{LICENSE} file)
\item \textbf{Lean Version:} As specified in \texttt{lean-toolchain} (Lean 4.x)
\item \textbf{Dependencies:} Zero external dependencies beyond Lean 4 standard library
\end{itemize}

\subsection{Build and verification}
The complete artifact is mathlib-free and maintains a zero-sorry constraint:

\begin{verbatim}
# Option 1: Download from Zenodo (recommended for permanent access)
curl -L https://zenodo.org/records/17059483/files/AICardiologist-FoundationRelativity-v1.0.0.zip \
  -o paper4-zenodo.zip
unzip paper4-zenodo.zip
cd AICardiologist-FoundationRelativity-*

# Option 2: Clone repository and checkout release
git clone https://github.com/AICardiologist/FoundationRelativity.git
cd FoundationRelativity
git checkout paper4-v1.0.0

# Option 3: Download GitHub release archive
curl -L https://api.github.com/repos/AICardiologist/FoundationRelativity/tarball/paper4-v1.0.0 \
  -o paper4-v1.0.0.tar.gz
tar -xzf paper4-v1.0.0.tar.gz
cd AICardiologist-FoundationRelativity-*

# Build Paper 4 main target
lake build Papers.P4_SpectralGeometry.Smoke

# Verify zero-sorry constraint
./scripts/no_sorry_p4.sh

# Optional: build all Paper 4 modules
lake build Papers.P4_SpectralGeometry

# CI verification
./scripts/verify-no-sorry.sh
\end{verbatim}

\subsection{Artifact structure and key mappings}
Core mathematical objects in the paper map to Lean definitions as follows:

\paragraph{Foundation framework:}
\begin{itemize}
\item \textbf{Foundation} $F$ $\mapsto$ \lean{Foundation} (\texttt{Basic.lean})
\item \textbf{Token classes} $\HasWLPO$, $\HasFT$, $\HasDCw$, $\HasMP$ $\mapsto$ \lean{HasWLPO}, \lean{HasFT}, \lean{HasDCω}, \lean{HasMP} (\texttt{Basic.lean})
\item \textbf{WitnessFamily} $W$ $\mapsto$ \lean{WitnessFamily} (\texttt{Basic.lean})
\end{itemize}

\paragraph{Profile algebra:}
\begin{itemize}
\item \textbf{Height} $\{0,1,\omega\}$ $\mapsto$ \lean{Height} with \lean{h0}, \lean{h1}, \lean{hω} (\texttt{Profiles.lean})
\item \textbf{Profile} $(h_{\WLPO}, h_{\FT}, h_{\DCw})$ $\mapsto$ \lean{Profile} structure (\texttt{Profiles.lean})
\item \textbf{Componentwise max} $\max(p,q)$ $\mapsto$ \lean{Profile.max} (\texttt{Profiles.lean})
\item \textbf{Requirements} $\Requires(p,F)$ $\mapsto$ \lean{Requires} (\texttt{ProfileUpper.lean})
\end{itemize}

\paragraph{Certificate system:}
\begin{itemize}
\item \textbf{Upper bound} $\ProfileUpper(p,W)$ $\mapsto$ \lean{ProfileUpper} (\texttt{ProfileUpper.lean})
\item \textbf{Product law} $(W_1 \land W_2)$ has profile $\max(p_1,p_2)$ $\mapsto$ \lean{ProfileUpper.and} (\texttt{ProfileUpper.lean})
\item \textbf{Automatic folding} $\mapsto$ \lean{foldPackages} (\texttt{ProfileInference.lean})
\end{itemize}

\paragraph{Orthogonal MP system:}
\begin{itemize}
\item \textbf{Extended profile} $(p; \mathit{mp\_flag})$ $\mapsto$ \lean{ProfileX} (\texttt{ProfilesMP.lean})
\item \textbf{Boolean requirements} $\mapsto$ \lean{needB}, \lean{RequiresX} (\texttt{ProfilesMP.lean})
\item \textbf{Extended certificates} $\mapsto$ \lean{ProfileUpperX} (\texttt{ProfilesMP.lean})
\end{itemize}

\paragraph{Spectral calibrators S0--S5:}
\begin{itemize}
\item \textbf{S0} (Compact approximation) $\mapsto$ \lean{CompactSpectralApprox\_W}, \lean{S0\_ProfileUpper} (\texttt{CompactApprox.lean})
\item \textbf{S1} (Markov spectrum) $\mapsto$ \lean{SpecApproxEqSpec\_W}, \lean{S1\_iff\_MP} (\texttt{MarkovSpectrum.lean})
\item \textbf{S2} (Locale spatiality) $\mapsto$ \lean{LocaleSpatiality\_W}, \lean{S2\_ProfileUpper} (\texttt{LocaleSpatiality.lean})
\item \textbf{S3} (WLPO portal) $\mapsto$ \lean{SeparationRoute\_W}, \lean{S3\_ProfileUpper} (\texttt{WLPOPortal.lean})
\item \textbf{S4} (QHO) $\mapsto$ \lean{QHO\_W}, \lean{S4\_ProfileUpper} (\texttt{QHO.lean})
\item \textbf{S5} (FT demo) $\mapsto$ \lean{FTPortal\_W}, \lean{S5\_ProfileUpper} (\texttt{ProfileUpper.lean})
\end{itemize}

\paragraph{Route-conditional costs:}
\begin{itemize}
\item \textbf{Route flags} $\UsesSeparationNonSep$ $\mapsto$ \lean{UsesSeparationNonSep} (\texttt{RouteFlags.lean})
\item \textbf{Portal discharge} $\mapsto$ \lean{PortalRequiresWLPO} (\texttt{RouteFlags.lean})
\end{itemize}

\subsection{Verification statistics}
\begin{itemize}
\item \textbf{Total lines:} 700+ lines of Lean 4 code
\item \textbf{Modules:} 15 core modules plus composition examples and showcases
\item \textbf{Sorries:} 0 (verified by CI)
\item \textbf{External dependencies:} 0 (mathlib-free)
\item \textbf{Spectral calibrators:} 5 main (S0--S4) plus 1 demonstration (S5)
\item \textbf{Composition examples:} 15+ automated certificate combinations
\item \textbf{CI status:} Continuously verified via GitHub Actions
\end{itemize}

% Patch H content moved to Figure \ref{fig:lean-folding-snippet}

% === PATCH P: BEGIN (Lean snippet for folding) =================================
\begin{figure}[t]
\centering
\begin{minipage}{0.92\linewidth}
\begin{verbatim}
-- Fold S0, S2, S3 into a single certificate + composed profile:
def S0_S2_S3_folded : WitnessPackage :=
  foldPackages [S0_package, S2_package, S3_package]

-- The resulting profile is the componentwise max:
#check S0_S2_S3_folded.profile
-- : Profile, equals max S0_profile (max S2_profile S3_profile)
\end{verbatim}
\end{minipage}
\caption{\textbf{Automated composition in the artifact.}
The fold constructs both the composite witness and its profile, then applies the
product law to obtain a single certificate for the conjunction.}
\label{fig:lean-folding-snippet}
\end{figure}
% === PATCH P: END ==============================================================

\section{Lessons from formalization}\label{sec:lessons}
\begin{enumerate}
  \item \textbf{Orthogonality $\neq$ simplicity.}
  Treating $\MP$ as an orthogonal boolean bit needed a parallel algebra and a
  dedicated composition lemma; automation follows only after the right splits.
  
  \item \textbf{Route costs are first-class.}
  The WLPO portal is attached to a \emph{route flag}, not to a theorem statement.
  This captures the insight that the same theorem may have multiple proofs with different axiomatic costs.
  
  \item \textbf{Automation emerges from structure.}
  Certificates compose via a product law; folding witnesses computes both the
  composite statement and its axiom bill. This automation is only possible because of the careful algebraic structure.
  
  \item \textbf{Constructive precision helps.}
  The zero-sorry constraint removed hand-wavy steps and clarified which tokens
  (e.g. $\DCw$ vs. stronger choice) actually appear. Several ``obvious'' claims turned out to require non-trivial axioms.
\end{enumerate}

% === PATCH I: BEGIN (physics-facing discussion) ==============================
\paragraph{Physics implications.}
The calibration results provide a precise map of the logical terrain underlying quantum spectral theory:

\begin{itemize}
\item \textbf{Computational core:} Height-0 results (S0, S4) show that numerical spectral methods for compact operators and structured unbounded operators (like QHO) require no axioms beyond constructive mathematics. This validates the computational practice in quantum mechanics.

\item \textbf{Measurement and classical states:} The DC$\omega$ requirement (S2) for extracting classical point spectra from locale spectra aligns with the non-deterministic nature of quantum measurement. The need for dependent choice reflects that selecting a definite measurement outcome requires infinitely many consistent choices.

\item \textbf{Approximation vs. exactness:} The MP layer (S1) quantifies the gap between what can be computed (approximate spectrum) and what can be decided (exact spectrum). This has practical implications for spectral algorithms in quantum chemistry and condensed matter physics.

\item \textbf{Infinite-dimensional contexts:} The WLPO portal (S3) warns that certain standard techniques (separation in non-separable spaces) carry hidden decidability costs, potentially relevant to quantum field theory where non-separable spaces naturally arise.
\item \textbf{Dirac/RHS route (route-conditional):} The Gelfand-triple pathway to generalized eigenvectors typically requires separation on the non-separable dual $\Phi'$, hence the S3 \WLPO\ portal. The measure-theoretic spectral resolution (PVMs/direct integrals) aligns with the S2 \DCw\ layer, and deciding exact spectral membership from approximants aligns with the S1 \MP\ guard. Axiom-cheap alternatives (explicit constructions on $\Phi$ or locale-based spectra) can bypass \WLPO.
\end{itemize}
% === PATCH I: END =============================================================

\paragraph{Future work.}
Two immediate directions: 
\begin{enumerate}
\item \textbf{Sharpness results:} Establish lower bounds via countermodels to show necessity of $\WLPO$/$\DCw$ in specific cases. This would convert our upper bounds into exact calibrations.

\item \textbf{Axiom-cheap reroutings:} Develop alternative proofs that avoid high-cost portals. For instance, replacing separation-based arguments (WLPO portal) with locale-based approaches (potentially only DC$\omega$) when possible.

\item \textbf{Extension to unbounded operators:} While S4 (QHO) demonstrates one unbounded case, systematic calibration of unbounded spectral theory (essential unbounded, self-adjoint extensions, deficiency indices) remains open.
\end{enumerate}

% === PATCH J: BEGIN (appendix artifact index) ================================
\appendix
\section{Artifact Index}

\textbf{Main Lean files:}
\begin{itemize}
\item \lean{Spectral/Basic.lean} — Foundation, tokens, witness families
\item \lean{Spectral/Profiles.lean} — 3-axis height algebra and composition  
\item \lean{Spectral/ProfileUpper.lean} — Profile-based upper bound certificates
\item \lean{Spectral/ProfilesMP.lean} — Orthogonal MP extension with boolean algebra
\item \lean{Spectral/CompactApprox.lean} — S0: Compact spectral approximation
\item \lean{Spectral/MarkovSpectrum.lean} — S1: MP-gated exact spectrum
\item \lean{Spectral/LocaleSpatiality.lean} — S2: DCω-gated locale spatiality  
\item \lean{Spectral/WLPOPortal.lean} — S3: WLPO-conditional separation route
\item \lean{Spectral/QHO.lean} — S4: Quantum harmonic oscillator
\item \lean{Spectral/ProfileInference.lean} — Automated folding and composition
\end{itemize}

\textbf{Verification commands:}
\begin{verbatim}
lake build Papers.P4_SpectralGeometry.Smoke  # Main target
./scripts/no_sorry_p4.sh                     # Zero-sorry check
./scripts/verify-no-sorry.sh                 # CI integration
\end{verbatim}

\textbf{Key verification stats:} 700+ lines, 5 spectral calibrators (S0-S4 plus S5 demo), 15+ composition examples, zero sorries, mathlib-free constraint maintained.
% === PATCH J: END =============================================================

% === PATCH C: END ==============================================================

\section*{Acknowledgments}
Development assistance provided by: Gemini 2.5 Deep Think (architecture exploration and theoretical framework design), GPT-5 Pro (Lean 4 scaffolding and implementation support), and Claude Code (repository management and development workflow).

\end{document}
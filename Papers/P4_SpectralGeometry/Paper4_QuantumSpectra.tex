\documentclass[11pt]{article}

% -------------------------------------------------
% Preamble (Robust standard setup)
% -------------------------------------------------
\usepackage[T1]{fontenc}
\usepackage[utf8]{inputenc}
\usepackage[american]{babel}
\usepackage{lmodern}
\usepackage{geometry}
\geometry{margin=1in}
\usepackage{amsmath,amssymb,mathtools}
\usepackage{amsthm}
\usepackage{enumitem}
\usepackage{hyperref}
\hypersetup{colorlinks=true,linkcolor=blue,citecolor=blue,urlcolor=blue}
\usepackage{mdframed} % For reproducibility box

% Theorem environments (Standard amsthm setup)
\theoremstyle{plain}
\newtheorem{theorem}{Theorem}[section]
\newtheorem{proposition}[theorem]{Proposition}
\newtheorem{corollary}[theorem]{Corollary}
\newtheorem{lemma}[theorem]{Lemma}

\theoremstyle{definition}
\newtheorem{definition}[theorem]{Definition}
\newtheorem{conjecture}[theorem]{Conjecture}
\newtheorem{example}[theorem]{Example}

\theoremstyle{remark}
\newtheorem{remark}[theorem]{Remark}

% Human-readable proof sketch environment
\newenvironment{hrproof}{\noindent\textit{Human-readable proof sketch:} }{}

% Reproducibility macros
\newcommand{\ReproBox}[1]{\begin{mdframed}[linewidth=1pt,linecolor=blue,backgroundcolor=blue!5] #1 \end{mdframed}}
\newcommand{\LeanTarget}[1]{\texttt{#1}}

% -------------------------------------------------
% Macros (harmonized with Paper 3A + physics additions)
% -------------------------------------------------
% Mathematics
\newcommand{\N}{\mathbb{N}}
\newcommand{\R}{\mathbb{R}}
\newcommand{\C}{\mathbb{C}}
\newcommand{\linf}{\ell^\infty}
\newcommand{\cnull}{c_0}

% Foundations and Principles
\newcommand{\BISH}{\mathsf{BISH}}
\newcommand{\ZFC}{\mathsf{ZFC}}
\newcommand{\WLPO}{\mathrm{WLPO}}
\newcommand{\FT}{\mathrm{FT}}
\newcommand{\DCw}{\mathrm{DC}_\omega}
\newcommand{\ACw}{\mathrm{AC}_\omega}
\newcommand{\ACR}{\mathrm{AC}_{\mathbb{R}}}
\newcommand{\DCR}{\mathrm{DC}_{\mathbb{R}}}
\newcommand{\WKLz}{\mathrm{WKL}_0}
\newcommand{\BI}{\mathrm{BI}}
\newcommand{\MP}{\mathrm{MP}}
\newcommand{\LEM}{\mathrm{LEM}}
\newcommand{\UCT}{\mathrm{UCT}}
\newcommand{\BCT}{\mathrm{BCT}}

% AxCal Framework
\newcommand{\Found}{\mathsf{Found}}
\newcommand{\Gpd}{\mathsf{Gpd}}
\newcommand{\SigmaZero}{\Sigma_{0}}
\newcommand{\Frontierpos}{\partial^{+}}

% Orthogonal height/profile shorthands (optional)
\newcommand{\hWLPO}{h_{\mathrm{WLPO}}}
\newcommand{\hFT}{h_{\mathrm{FT}}}
\newcommand{\hDC}{h_{\mathrm{DC}\omega}}

% -------------------------------------------------
% Title
% -------------------------------------------------
\title{Axiom Calibration for Quantum Spectra (Paper 4):\\
Orthogonal Heights, Choice Principles, and Separation Portals}
\author{Paul Chun--Kit Lee}
\date{September 2025}

\begin{document}
\maketitle

\begin{abstract}
We align the spectral analysis program with the \emph{Axiom Calibration} (AxCal) framework of Paper~3A.
Using \emph{positive uniformization} and \emph{orthogonal height profiles} over a constructive base ($\BISH$), we calibrate which spectral claims reside at height~0 (constructive core) and which require additional principles.
Our main provenance--aware certificates are:
(S0) quantitative $\varepsilon$--spectral decompositions for finite--rank/compact self--adjoint operators (height~0);
(S1) the Bridges--Ishihara calibration of $\mathrm{Spec}_{\mathrm{approx}}=\mathrm{Spec}$ at frontier $\{\MP\}$ (Markov's Principle);
(S2) locale spectral theory, where ensuring \emph{spatiality} (existence of enough points/characters) requires dependent choices in standard constructions (upper bound $\{\DCw\}$);
(S3) a \emph{WLPO portal}: proof routes relying on separation in non-separable contexts (e.g., Hahn--Banach flavor on $\linf/c_0$) force a $\{\WLPO\}$ frontier by implicitly constructing elements of the bidual gap;
and (S4) a pinned unbounded exemplar (the quantum harmonic oscillator) at height~0.
We also record a \emph{broader axes survey} ($\ACw$, $\WKLz$, $\BI$, $\MP$, $\ACR/\DCR$) to situate spectral arguments within the expanded landscape of Paper~3A.
A verification ledger distinguishes Lean--feasible upper bounds, literature--level reversals, and open lower bounds.
\end{abstract}

\tableofcontents

\ReproBox{
\textbf{Reproducibility Infrastructure:} This paper calibrates quantum spectral results using the AxCal framework from Paper 3A.
\begin{itemize}[leftmargin=*]
  \item \textbf{Paper URL:} \url{https://github.com/AICardiologist/FoundationRelativity}
  \item \textbf{Framework Import:} \LeanTarget{Papers.P3\_2CatFramework.Paper3A\_Main}
  \item \textbf{Spectral Modules:} \LeanTarget{Papers.P4\_SpectralGeometry.Spectral.*} (planned)
  \item \textbf{Build:} \texttt{lake build Papers.P4\_SpectralGeometry} (once modules created)
  \item \textbf{CI:} Currently in planning phase; will integrate when Lean modules ready
\end{itemize}
}

% ===========================================================
\section{Introduction and alignment with Paper 3A}
% ===========================================================

Paper~3A \cite{Paper3A} develops the Axiom Calibration (AxCal) framework: witness families, uniformizability, positive height, and orthogonal profiles
along (at least) the axes $\WLPO$, $\FT$, and $\DCw$, operating over a constructive base ($\BISH$). It also includes an expanded survey covering $\ACw$, $\WKLz$, $\BI$, $\MP$, and restricted choice on $\R$.
This paper specializes AxCal to the spectral infrastructure that underlies early quantum mechanics,
with fixed analytic pins and physics--facing readouts.

\paragraph{Pins (fixed in $\SigmaZero$).}
We extend the pinned signature $\SigmaZero$ with the abstract structure of a separable complex Hilbert space $H$ (e.g., $\ell^2(\C)$), the operator algebra $B(H)$,
and the subtype $\mathrm{SA}(H)$ of bounded self--adjoint operators. Interpretations between foundations must fix these structures up to canonical isomorphism. For unbounded exemplars we fix a
core domain (e.g.\ Schwartz space) where closures are standard.

\paragraph{Witness semantics.}
Each spectral claim is packaged as a \emph{witness family} returning a truth groupoid (empty vs.\ singleton)
at the pin; positive uniformization at a stage means existence and stability across all $\SigmaZero$--fixing interpretations within that axiomatic locus.

\paragraph{What is new here.}
Compared with earlier drafts, we (i) adopt the \emph{broader axiomatic landscape} of Paper~3A (adding $\ACw$, $\WKLz$, $\BI$, $\MP$, $\ACR/\DCR$ as survey axes), (ii) attach \emph{frontiers/profiles} to physics--salient subclaims (S0--S4), distinguishing carefully between upper bounds and precise frontiers, and (iii) publish a \emph{verification ledger} tracking Lean--feasible parts vs.\ literature--level inputs.

% ===========================================================
\section{Minimal AxCal interface (recap)}
% ===========================================================

We recall the minimal notions used in this paper; see Paper~3A \cite{Paper3A} for full development. The framework operates over Bishop-style constructive mathematics ($\BISH$) to detect subtle axiomatic costs, analyzing how witness constructions behave under interpretations that fix the pinned signature $\SigmaZero$.

\begin{definition}[Witness family and positive uniformization]
A witness family $\mathcal{C}$ assigns to each foundation $F\in\Found$ a groupoid $\mathcal{C}(F)$.
Given a sublocus $\Found_{\ge T}$ of theories extending $\BISH$, $\mathcal{C}$ is \emph{positively uniformizable at $T$}
if (i) $\mathcal{C}(F)$ is nonempty for all $F\in \Found_{\ge T}$ and (ii) for every $\SigmaZero$--fixing interpretation in $\Found_{\ge T}$,
the induced functors at the pin are equivalences.
\end{definition}

\begin{definition}[Height, Frontier, and Orthogonal profile]
Fix a ladder $(T_k)_k$ along one axis $A$; the height $h_A(\mathcal{C})$ is the least $k$ with positive uniformization at $T_k$.
The \emph{frontier} $\Frontierpos\mathcal{C}$ is the set of minimal theories $T$ where positive uniformization holds.
For independent axes $A_1,\dots,A_n$ we write the profile
$h^{\to}(\mathcal{C})=(h_{A_1}(\mathcal{C}),\ldots,h_{A_n}(\mathcal{C}))$. If only an upper bound is established, we use $\le$.
\end{definition}

% ===========================================================
\section{Spectral calibrators S0--S4}
% ===========================================================

\subsection{(S0) Height 0 core: $\varepsilon$--spectral approximations for compact/finite rank}

The computational backbone of practical spectral analysis relies on finite approximations.

\begin{definition}[Approximate spectral decomposition]
For $T\in \mathrm{SA}(H)$ and $\varepsilon>0$, an $(n,\varepsilon)$--approximation is a family
$\{(e_k,\lambda_k)\}_{k=1}^n$ (orthonormal $(e_k)$, $\lambda_k\in\R$) with
$\big\|T - \sum_{k=1}^n \lambda_k \langle \cdot, e_k\rangle e_k\big\|\le \varepsilon$.
\end{definition}

\begin{proposition}[Compact/finite rank; height $0$]
If $T$ is compact (in particular, finite rank), then for every $\varepsilon>0$ an $(n,\varepsilon)$--approximation exists constructively ($\BISH$).
\end{proposition}

\begin{proof}
Total boundedness of $T(B_H)$ yields a finite $\varepsilon$--net. Orthogonalize, project, and apply a rational search on a net of the unit sphere in the finite--dimensional range (Rayleigh--Ritz) to get the coefficients within tolerance. No nonconstructive choice is required.
\end{proof}

\noindent\textbf{Profile:} Height 0. This forms the computable core accessible to numerical methods.

\subsection{(S1) Approximative vs.\ actual spectrum (frontier $\{\MP\}$)}
Let $\mathrm{Spec}_{\mathrm{approx}}(T)=\{\lambda:\forall k\ \exists v,\ \|v\|=1,\ \|(T-\lambda I)v\|\le 2^{-k}\}$.
The actual spectrum $\mathrm{Spec}(T)$ consists of $\lambda$ such that $T-\lambda I$ is not invertible.
Constructively, $\mathrm{Spec}(T) \subseteq \mathrm{Spec}_{\mathrm{approx}}(T)$. The reverse inclusion often relies on Markov's Principle ($\MP$), which asserts that if an algorithm is proven not to run forever, it must terminate (formally: $\neg\neg\exists n P(n) \to \exists n P(n)$ for decidable $P$).

\begin{theorem}[Bridges--Ishihara; provenance \cite{BridgesRichman}]
Over $\BISH$, for bounded self--adjoint $T$,
\[
\mathrm{Spec}_{\mathrm{approx}}(T)=\mathrm{Spec}(T)\quad \Longleftrightarrow\quad \MP.
\]
\end{theorem}

\begin{hrproof}
The key insight: approximate spectral values ``almost fail to invert,'' which non-constructively means ``actually fail to invert'' using \MP. Given $\lambda\in\mathrm{Spec}_{\mathrm{approx}}(T)$, we have sequences of unit vectors with $(T-\lambda I)v_n\to 0$. If $T-\lambda I$ were invertible with bounded inverse $S$, then $v_n = S((T-\lambda I)v_n)\to 0$ contradicts $\|v_n\|=1$. So $T-\lambda I$ is not invertible, hence $\lambda\in\mathrm{Spec}(T)$. The reverse requires showing that if we cannot algorithmically exhibit a kernel vector, then \MP\ lets us convert ``not forever failing'' to ``eventually succeeding'' in finding an inverse.
\end{hrproof}

\begin{proof}[Formal upper direction (Lean--feasible under \MP)]
If $T-\lambda I$ is invertible, a sequence of approximate kernels forces (via \MP) an actual kernel contradiction; hence $\lambda\notin \mathrm{Spec}_{\mathrm{approx}}(T)$.
The converse implication to \MP is taken from the literature.
\end{proof}

\noindent\textbf{Profile:} Frontier $\{\MP\}$. We do not place \MP\ on the primary axes (WLPO, FT, DC$\omega$), as it represents a different type of non-constructivity (related to unbounded search) which is orthogonal to them in $\BISH$.

\subsection{(S2) Locale spectra and spatiality (Upper bound $\{\DCw\}$)}
In constructive Gelfand duality (e.g., \cite{CoquandSpitters}), the spectrum of a commutative $C^*$-algebra is constructed as a \emph{locale} (a point-free topology) without recourse to choice principles. However, recovering the classical picture where the spectrum is a space of \emph{points} (characters) requires ensuring the locale is \emph{spatial} (has enough points).

\begin{proposition}[Spatiality via Rasiowa--Sikorski; upper bound]
If $\DCw$ holds, then for a \emph{separable} commutative $C^*$--algebra the spectrum locale is spatial (has enough points/characters); hence a classical pointwise spectral resolution is available.
\end{proposition}

\begin{hrproof}
Think of the locale as a ``cloud'' of open set data without committed points. To extract actual points (characters), we need to make infinitely many consistent choices. List all basic opens $U_1, U_2,\ldots$ that should determine a character. Starting with any consistent seed, use $\DCw$: ``Given partial character $\chi_n$ consistent with $U_1,\ldots,U_n$, choose extension $\chi_{n+1}$ covering $U_{n+1}$ if possible, else keep $\chi_n$.'' The limit $\chi = \lim \chi_n$ is a complete character (point), making the locale spatial.
\end{hrproof}

\begin{proof}[Formal proof]
For separable algebras, the corresponding locale structure is often countably generated. We adapt the Rasiowa--Sikorski argument: enumerate dense requirements in the poset of partial descriptions of a point and apply $\DCw$ to construct a sequence meeting them successively, yielding a complete description (a point/character).
\end{proof}

\begin{remark}[Physical significance of spatiality]
The locale spectrum provides a constructive notion of the state space. However, extracting a \emph{point} corresponds to identifying a specific, classical state or a definite measurement outcome. The reliance on $\DCw$ indicates that selecting such a definite state may require non-computational choice principles.
\end{remark}

\noindent\textbf{Profile:} $h^{\to}(\text{Locale spatiality})\le (0,0,1)$ (Upper bound $\{\DCw\}$). While this establishes that $\DCw$ is sufficient, the precise lower bound (e.g., whether weaker principles like $\ACw$ suffice) remains an open calibration question (see Verification Ledger, \S\ref{sec:ledger}). Without requiring spatiality, locale spectral reasoning stays at height~0.

\subsection{(S3) WLPO portal for separation--based routes}
We formalize the route--sensitivity of some spectral proofs. Certain arguments rely on global separation principles akin to the Hahn--Banach theorem.

\begin{definition}[Separation flag (Non-separable context)]
A proof of a spectral claim has flag \textsf{uses\_separation} if it relies on constructing a linear functional or state via separation applied in a \emph{non-separable context}, such as on a quotient of $\linf$ (like $\linf/c_0$), in a manner that implicitly witnesses an element of $(\linf)^{**}\setminus \linf$.
\end{definition}

\begin{theorem}[Portal transport to $\WLPO$]
Any spectral proof at the pin with flag \textsf{uses\_separation} implies the bidual gap for $\linf$; over $\BISH$ this is equivalent to $\WLPO$ (Paper~2 \cite{Paper2}). Thus the spectral claim inherits a frontier containing $\{\WLPO\}$ along that route.
\end{theorem}

\begin{hrproof}
Suppose a spectral proof uses separation to extend a functional from $c_0$ to $\ell^\infty$ (e.g., to construct a special state or resolve a spectral component). The extension is a bounded linear functional $\phi\in (\ell^\infty)^*$. If we could constructively verify $\phi\in\ell^\infty$ (represented as evaluation at some index), we'd be done. But often $\phi\in (\ell^\infty)^{**}\setminus\ell^\infty$---it lives in the bidual gap. Paper 2 shows: ``detecting this gap $\Leftrightarrow$ \WLPO.'' Hence the spectral claim using this route inherits \WLPO.
\end{hrproof}

\begin{remark}[Hahn-Banach Variants]
It is crucial to distinguish the context. If separation is applied only within a separable space (like $H$), the Separable Hahn-Banach theorem (SHB) suffices, which is provable in $\BISH$ (Height 0). Conversely, the full Hahn-Banach theorem implies the Law of Excluded Middle (LEM), which is much stronger than $\WLPO$. The WLPO portal is specifically triggered by the intermediate strength required for non-separable separation corresponding to the bidual gap.
\end{remark}

\noindent\textbf{Profile:} conditionally $(1,0,0)$ on the WLPO axis, determined by the \emph{proof route}. Locale proofs typically avoid this portal.

\subsection{(S4) Pinned unbounded exemplar: the quantum harmonic oscillator (height $0$)}
We include a fundamental unbounded operator to demonstrate the framework extends beyond $B(H)$.
Let $H=L^2(\R)$ with $H_{\mathrm{QHO}}\psi=-\psi''+x^2\psi$ on the Schwartz core.

\begin{proposition}[QHO pin]
The Hermite basis diagonalizes $H_{\mathrm{QHO}}$ with eigenvalues $(2n+1)_{n\ge 0}$. For every $N,\varepsilon>0$, the partial spectral sum provides an $(N,\varepsilon)$--approximation (in the appropriate sense for unbounded operators). Height $0$.
\end{proposition}

\begin{hrproof}
The Hermite functions $\{h_n\}$ satisfy $H_{\mathrm{QHO}} h_n = (2n+1)h_n$ and form an orthonormal basis of $L^2(\R)$. Given $\psi$ in the Schwartz core and $\varepsilon>0$, compute coefficients $c_n = \langle h_n, \psi\rangle$. Since $\psi$ is Schwartz, $|c_n|$ decays rapidly. Choose $N$ so $\sum_{n>N} |c_n|^2(2n+1)^2 < \varepsilon^2$. Then $\|H_{\mathrm{QHO}}\psi - \sum_{n=0}^N (2n+1) c_n h_n\| < \varepsilon$. Everything is explicit/constructive: Hermite functions have closed formulas, inner products are computable integrals, and $N$ is found by rational arithmetic on the decay bounds.
\end{hrproof}

% ===========================================================
\section{Broader axiomatic landscape for spectra}\label{sec:broad-axes}
% ===========================================================

Consistent with Paper~3A \cite{Paper3A}, we record additional axes and their spectral relevance (survey; not all used for height claims here):

\begin{itemize}
  \item \textbf{Countable Choice ($\ACw$).} Often sufficient in \emph{separable} Hilbert space arguments. For example, proving that the closure of a subspace $S$ is equal to its sequential closure (points reachable by convergent sequences in $S$) typically requires $\ACw$. This allows selecting sequences of approximants without needing the full strength of $\DCw$. Axis token: \textsf{HasAC$\omega$}.
  
  \item \textbf{Weak K\"onig's Lemma ($\WKLz$) / Fan Theorem ($\FT$).} Classical compactness ($\WKLz$) and its constructive counterpart ($\FT$). These are relevant for spectral theory via compactness arguments, such as establishing properties of the functional calculus or proving the compactness of the spectrum in Gelfand theory. Axis tokens: \textsf{HasWKL0}, \textsf{HasFT}.

  \item \textbf{Bar Induction (BI) / continuity principles.} Intuitionistic continuity tools that can interact with moduli in spectral continuity (resolvents, functional calculus). Axis token: \textsf{HasBI}.

  \item \textbf{Markov's Principle ($\MP$).} Calibrates the gap between approximative and exact properties by allowing the conversion of double-negation existence into existence for decidable predicates, as seen in S1. Axis token: \textsf{HasMP}.

  \item \textbf{Choice on the reals ($\ACR$, $\DCR$).} Restricted choice useful for measure--theoretic spectral resolutions (PVMs), selection of Borel partitions, and functional calculus along $\R$. Axis tokens: \textsf{HasACR}, \textsf{HasDCR}.
\end{itemize}

\noindent
\emph{Use in this paper.} We make explicit height claims only on the $(\WLPO,\FT,\DCw)$ profile and the \MP\ frontier for S1.
The axes above are recorded to \emph{locate} additional spectral arguments should they be brought into AxCal.

% ===========================================================
\section{Physics readout (profiles)}
% ===========================================================

We summarize the calibration results and their implications. Profiles are given in the order $(\WLPO, \FT, \DCw)$.

\begin{center}
\begin{tabular}{|l|l|l|}
\hline
\textbf{Frontier / Profile} & \textbf{Spectral step} & \textbf{Physical readout} \\
\hline
Height $0$ & S0 (compact/finite rank), S4 (QHO) & Numerical/constructive backbone; fully computable. \\
\hline
$\{\MP\}$ & S1: $\mathrm{Spec}_{\mathrm{approx}}=\mathrm{Spec}$ & Determining exact spectral membership may require an \\
& & unbounded (non-computable) search. \\
\hline
$\le (0,0,1)$ (Upper $\{\DCw\}$) & S2: spatiality of locale spectra & Selecting a definite classical state (point) uses \\
& (separable case) & dependent choices. \\
\hline
$(1,0,0)$ (route--conditional) & S3: separation portal (non-sep.) & Proofs relying on global separation in non-separable \\
& & spaces import WLPO costs. \\
\hline
\end{tabular}
\end{center}

% ===========================================================
\section{Implementation notes (Lean--feasible interfaces)}
% ===========================================================

We reuse Paper~3A's tokens and height certificates. Illustrative (informal) skeleton demonstrating extensibility:

\medskip
\noindent\emph{Axiom tokens}
\begin{verbatim}
class HasWLPO (F) : Prop
class HasFT   (F) : Prop
class HasDCω  (F) : Prop
-- Extended axes:
class HasACω  (F) : Prop
class HasWKL0 (F) : Prop
class HasBI   (F) : Prop
class HasMP   (F) : Prop
\end{verbatim}

\noindent\emph{Witness families at the pin}
\begin{itemize}
  \item \texttt{ApproxSpec\_W} for S0 (height 0).
  \item \texttt{SpecApproxEqSpec\_W} for S1 (upper bound under \texttt{[HasMP]}).
  \item \texttt{LocaleSpatiality\_W} for S2 (upper bound under \texttt{[HasDCω]}).
  \item \texttt{SeparationRoute\_W} for S3 (wired to the WLPO portal).
\end{itemize}

\noindent\emph{Certificates}
\begin{itemize}
  \item Product/sup law for profiles (Paper~3A).
  \item Independence registry (e.g.\ $\WLPO\perp \FT$, $\WLPO\perp \DCw$) referenced when asserting sharpness for products.
\end{itemize}

% ===========================================================
\section{Verification ledger (provenance)}\label{sec:ledger}
% ===========================================================

We distinguish the provenance of the results presented in tabular form for quick reference.

\begin{center}
\begin{tabular}{|l|p{5cm}|p{3.5cm}|p{3cm}|}
\hline
\textbf{Result} & \textbf{Claim} & \textbf{Status} & \textbf{Source} \\
\hline
\multicolumn{4}{|c|}{\textit{Lean-feasible now (structural/upper bounds)}} \\
\hline
S0/S4 & Constructive approximations & Height 0 ✓ & Direct construction \\
\hline
S1 upper & \texttt{HasMP} $\Rightarrow$ Spec$_{\text{approx}}$=Spec & Upper bound ✓ & Standard argument \\
\hline
S2 upper & \texttt{HasDCω} $\Rightarrow$ locale spatiality & Upper bound ✓ & Rasiowa-Sikorski \\
\hline
S3 & WLPO portal wiring & Connected ✓ & Paper 2 import \\
\hline
\multicolumn{4}{|c|}{\textit{Literature-level inputs (axiomatized)}} \\
\hline
S1 reversal & Spec$_{\text{approx}}$=Spec $\Rightarrow$ MP & Axiomatized & Bridges-Ishihara \cite{BridgesRichman} \\
\hline
Independence & WLPO $\perp$ FT, etc. & Axiomatized & \cite{Simpson, Ishihara06} \\
\hline
\multicolumn{4}{|c|}{\textit{Open/Conjectural lower bounds}} \\
\hline
S2 lower & Minimal choice for spatiality & Open question & Is DC$_\omega$ sharp? \\
\hline
\end{tabular}
\end{center}

\noindent\textbf{Legend:} ✓ = Lean-feasible with current framework; ``Axiomatized'' = taken from literature as axiom; ``Open question'' = precise calibration unknown.

% ===========================================================
\section{Related work}
% ===========================================================

Constructive analysis and spectral theory: Bishop--Bridges \cite{BishopBridges}; locale/constructive Gelfand duality (Coquand--Spitters \cite{CoquandSpitters}).
Computable analysis for physics: Pour--El \& Richards \cite{PourElRichards}; Weihrauch \cite{Weihrauch}. Reverse mathematics context: Simpson \cite{Simpson}.
AxCal framework and gap$\leftrightarrow$WLPO portal: Paper~3A \cite{Paper3A} and Paper~2 \cite{Paper2}.

% ===========================================================
\section{Conclusion}
% ===========================================================

AxCal successfully separates the computational backbone of spectral theory (height~0 cores S0/S4) from proof features that incur specific logical costs.
We identified three distinct types of non-constructivity relevant to spectra:
\MP\ for exact spectral membership (S1), $\DCw$ (upper bound) for ensuring spatiality of locale spectra (S2), and a \WLPO\ cost incurred when global separation in non-separable contexts is used (S3).
The broader axes ($\ACw$, $\WKLz$, $\BI$, $\MP$, $\ACR/\DCR$) provide coordinates for additional spectral arguments as the calibration program expands, offering a rigorous map of the axiomatic requirements underlying quantum mechanics.

\bigskip

\bibliographystyle{abbrv}
\begin{thebibliography}{99}

\bibitem{Paper2}
P.~C.-K.~Lee.
\newblock The bidual gap and WLPO: characterization and formalization.
\newblock (Companion paper; Lean 4), 2025.

\bibitem{Paper3A}
P.~C.-K.~Lee.
\newblock Axiom Calibration via Non-Uniformizability: A Framework for Orthogonal Logical Dependencies in Analysis.
\newblock (Paper 3A, revised), 2025.

\bibitem{BishopBridges}
E.~Bishop and D.~S.~Bridges.
\newblock {\em Constructive Analysis}.
\newblock Springer, 1985.

\bibitem{BridgesRichman}
D.~S.~Bridges and F.~Richman.
\newblock {\em Varieties of Constructive Mathematics}.
\newblock Cambridge University Press, 1987.

\bibitem{CoquandSpitters}
T.~Coquand and B.~Spitters.
\newblock Constructive Gelfand duality for $C^*$-algebras.
\newblock {\em Math.\ Proc.\ Camb.\ Phil.\ Soc.} 147(2):323--337, 2009.

\bibitem{Ishihara06}
H.~Ishihara.
\newblock Reverse mathematics in {B}ishop's constructive mathematics.
\newblock {\em Philosophia Scientiae}, Cahier Sp\'ecial 6:43--59, 2006.

\bibitem{PourElRichards}
M.~B.~Pour-El and J.~I.~Richards.
\newblock {\em Computability in Analysis and Physics}.
\newblock Springer, 1989.

\bibitem{Weihrauch}
K.~Weihrauch.
\newblock {\em Computable Analysis}.
\newblock Springer, 2000.

\bibitem{Simpson}
S.~G.~Simpson.
\newblock {\em Subsystems of Second Order Arithmetic} (2nd ed.).
\newblock Cambridge University Press, 2009.

\end{thebibliography}

\end{document}
\documentclass[11pt,a4paper]{article}

% ---- Packages ----
\usepackage[utf8]{inputenc}
\usepackage[T1]{fontenc}
\usepackage{amsmath,amssymb,amsthm}
\usepackage{mathrsfs}
\usepackage{enumerate}
\usepackage{hyperref}
\usepackage{cleveref}
\usepackage{booktabs}
\usepackage{graphicx}
\usepackage{float}
\usepackage{caption}
\usepackage[margin=1in]{geometry}
\usepackage{xcolor}

% ---- Theorem environments ----
\newtheorem{theorem}{Theorem}[section]
\newtheorem{lemma}[theorem]{Lemma}
\newtheorem{proposition}[theorem]{Proposition}
\newtheorem{corollary}[theorem]{Corollary}
\theoremstyle{definition}
\newtheorem{definition}[theorem]{Definition}
\newtheorem{example}[theorem]{Example}
\newtheorem{remark}[theorem]{Remark}

% ---- Notation ----
\newcommand{\Q}{\mathbb{Q}}
\newcommand{\Z}{\mathbb{Z}}
\newcommand{\N}{\mathbb{N}}
\newcommand{\R}{\mathbb{R}}
\newcommand{\F}{\mathbb{F}}
\newcommand{\Qp}{\mathbb{Q}_p}
\newcommand{\Zp}{\mathbb{Z}_p}
\newcommand{\vp}{v_p}
\newcommand{\NM}{N_M}
\newcommand{\BISH}{\mathsf{BISH}}
\newcommand{\LPO}{\mathsf{LPO}}
\newcommand{\LLPO}{\mathsf{LLPO}}
\newcommand{\MP}{\mathsf{MP}}
\newcommand{\WLPO}{\mathsf{WLPO}}
\newcommand{\CLASS}{\mathsf{CLASS}}

\begin{document}

% ====================================================================
% TITLE
% ====================================================================

\title{%
  Uniform $p$-Adic Decidability for Elliptic Curves:\\
  Computational Evidence and Proof\\[6pt]
  \large (Paper 64 of the Constructive Reverse Mathematics Series)
}
\author{Paul Chun-Kit Lee}
\date{February 2026}
\maketitle

% ====================================================================
% ABSTRACT
% ====================================================================

\begin{abstract}
Paper~59 of this series established that the crystalline precision bound
$\NM = \vp(\#E(\F_p))$ for an elliptic curve $E/\Q$ at a prime $p$
of good reduction is $\BISH$-computable.
This paper determines the \emph{uniform} bound: we prove
$\NM \le 2$ for all $(E, p)$, with $\NM = 2$ only possible at
$p = 2$, and $\NM \le 1$ for every $p \ge 3$.
For $p \ge 5$, equality $\NM = 1$ holds if and only if $E$ is anomalous
at~$p$ (i.e., $a_p = 1$).
The proof is a short argument from the Hasse bound $|a_p| \le 2\sqrt{p}$.
We verify the theorem computationally on 1{,}812 elliptic curves across
15 primes, covering 23{,}454 $(E, p)$~pairs, and confirm that $\NM$
does not correlate with rank, torsion structure, or CM status.

The principal consequence is that $p$-adic crystalline decidability for
elliptic curves requires at most two digits of $p$-adic precision at
$p = 2$ and at most one digit everywhere else.
The $p$-adic side of the decidability problem is thus \emph{uniformly
trivial}, contrasting sharply with the Archimedean side, where rank
creates genuine stratification (Papers~59, 61).
This asymmetry---$p$-adic uniform, Archimedean stratified---is a
clean structural result with implications for the mixed-motive
decidability program.
\end{abstract}

% ====================================================================
% 1. INTRODUCTION
% ====================================================================

\section{Introduction}

\subsection{Context}

The Constructive Reverse Mathematics (CRM) program~\cite{P1,P50}
calibrates theorems of mainstream mathematics against the logical
hierarchy
\[
  \BISH \;\subset\; \BISH{+}\MP \;\subset\; \BISH{+}\LLPO
  \;\subset\; \BISH{+}\WLPO \;\subset\; \BISH{+}\LPO
  \;\subset\; \CLASS,
\]
where $\BISH$ denotes Bishop's constructive mathematics~\cite{BB85},
and $\LPO$, $\WLPO$, $\LLPO$, $\MP$ are the limited principle of
omniscience, the weak limited principle of omniscience, the lesser
limited principle of omniscience, and Markov's principle,
respectively.

Paper~59~\cite{P59} in this series proved that for an elliptic curve
$E/\Q$ and a prime $p$ of good reduction, the $p$-adic precision
needed to decide whether the associated Galois representation is
crystalline is
\begin{equation}\label{eq:NM-def}
  \NM = \vp(\#E(\F_p)) = \vp(p + 1 - a_p),
\end{equation}
where $a_p$ is the Frobenius trace and $\#E(\F_p) = p + 1 - a_p$
is the number of $\F_p$-rational points on the reduced curve.
Paper~59 showed that $\NM$ is $\BISH$-computable for each individual
$(E, p)$ pair.

\subsection{Main results}

This paper asks: \emph{what is the uniform behavior of $\NM$
across all elliptic curves and primes?}  The answer is surprisingly
clean.

\begin{theorem}[Uniform $p$-adic precision bound]\label{thm:main}
For any elliptic curve $E/\Q$ and any prime $p$ of good reduction:
\begin{enumerate}[\upshape(i)]
  \item $\NM \le 2$, with $\NM = 2$ only possible at $p = 2$.
  \item For $p \ge 3$: $\NM \le 1$.
  \item For $p \ge 5$: $\NM = 1$ if and only if $E$ is anomalous
    at $p$ (i.e., $a_p = 1$, equivalently $\#E(\F_p) = p$).
\end{enumerate}
\end{theorem}

\begin{theorem}[Rank independence]\label{thm:rank-indep}
The bound $\NM \le 2$ is independent of the rank of $E$.
In particular, there is no $p$-adic analogue of the Archimedean
rank obstruction identified in Paper~59: the $p$-adic precision
bound does not increase with rank.
\end{theorem}

\begin{theorem}[Constructive consequence]\label{thm:crm}
Crystalline decidability for elliptic curves over $\Q$ is uniformly
$\BISH$-decidable: at every prime $p$ of good reduction, a
computation in $\Z/p^2\Z$ (or $\Z/p\Z$ for $p \ge 3$) suffices
to determine $\NM$ and hence to decide crystalline equivalence.
No omniscience principle is required.
\end{theorem}

\subsection{Relationship to the series}

This paper belongs to the arithmetic geometry strand of the CRM
program.  Paper~50~\cite{P50} established the DPT (Decidable Polarized
Tannakian) framework; Papers~51--53~\cite{P51,P52,P53}
tested it on BSD, Birch--Swinnerton-Dyer, and Bloch--Kato conjectures;
Paper~59~\cite{P59} proved individual $\BISH$-computability of $\NM$;
Papers~61--62~\cite{P61,P62} address the Archimedean side
(Lang's conjecture, Hodge level boundary).
Paper~64 completes the $p$-adic side by showing that \emph{no
family-level complication arises}: the bound is uniform, and the
$p$-adic decidability problem is trivially solved for all elliptic
curves simultaneously.

% ====================================================================
% 2. PRELIMINARIES
% ====================================================================

\section{Preliminaries}

\begin{definition}[$p$-adic valuation]
For a prime $p$ and a nonzero integer $n$, the \emph{$p$-adic
valuation} $\vp(n)$ is the largest exponent $k \ge 0$ such that
$p^k \mid n$.  We set $\vp(0) = +\infty$.
\end{definition}

\begin{definition}[Frobenius trace and point count]
Let $E/\Q$ be an elliptic curve and $p$ a prime of good reduction.
The \emph{Frobenius trace} $a_p \in \Z$ is defined by
\[
  \#E(\F_p) = p + 1 - a_p.
\]
\end{definition}

\begin{definition}[Hasse bound {\cite{Hasse33,Silverman09}}]
For any elliptic curve $E/\Q$ and any prime $p$ of good reduction,
\begin{equation}\label{eq:hasse}
  |a_p| \le 2\sqrt{p}.
\end{equation}
\end{definition}

\begin{definition}[Crystalline precision bound]
The \emph{crystalline precision bound} for $(E, p)$ is
\[
  \NM = \vp(\#E(\F_p)) = \vp(p + 1 - a_p).
\]
\end{definition}

\begin{definition}[Anomalous prime]
A prime $p$ of good reduction for $E$ is \emph{anomalous} if
$a_p = 1$, equivalently $\#E(\F_p) = p$.  In this case
$\NM = \vp(p) = 1$.
\end{definition}

\begin{definition}[Logical principles {\cite{BB85,BR87}}]
We work in Bishop's constructive mathematics ($\BISH$) unless
otherwise stated.  The relevant omniscience principles are:
\begin{itemize}
  \item $\LPO$: For every binary sequence $(a_n)$, either
    $\exists n\, (a_n = 1)$ or $\forall n\, (a_n = 0)$.
  \item $\MP$: For every binary sequence $(a_n)$, if it is
    impossible that $\forall n\, (a_n = 0)$, then
    $\exists n\, (a_n = 1)$.
  \item $\BISH$: No omniscience principle assumed.
\end{itemize}
\end{definition}

% ====================================================================
% 3. MAIN RESULTS
% ====================================================================

\section{Main Results}

\subsection{The uniform bound}

\begin{proof}[Proof of \Cref{thm:main}]
We establish $\NM \le 2$ with the stated refinements by case
analysis on~$p$.

\medskip\noindent
\textbf{Case $p \ge 5$.}\quad
The Hasse bound~\eqref{eq:hasse} gives $|a_p| \le 2\sqrt{p}$.
For $p \ge 5$, we have $2\sqrt{p} < p$ (since $4p < p^2$ iff $p > 4$),
so $|a_p| < p$.

Suppose for contradiction that $\NM \ge 2$, i.e.,
$p^2 \mid (p + 1 - a_p)$.
Then $a_p \equiv p + 1 \pmod{p^2}$, i.e.,
$a_p \equiv 1 \pmod{p}$.
Since $|a_p| < p$, the only integer satisfying
$a_p \equiv 1 \pmod{p}$ and $|a_p| < p$ is $a_p = 1$.
But then $p + 1 - a_p = p$, and $\vp(p) = 1 < 2$, contradicting
$\NM \ge 2$.

Therefore $\NM \le 1$ for all $p \ge 5$.

Equality $\NM = 1$ requires $p \mid (p + 1 - a_p)$, i.e.,
$a_p \equiv 1 \pmod{p}$.  Again $|a_p| < p$ forces $a_p = 1$,
which is the anomalous condition.

\medskip\noindent
\textbf{Case $p = 3$.}\quad
The Hasse bound gives $|a_3| \le 2\sqrt{3} \approx 3.46$, so
$a_3 \in \{-3, -2, -1, 0, 1, 2, 3\}$.
The point count $\#E(\F_3) = 4 - a_3$ ranges over $\{1, 2, 3, 4, 5, 6, 7\}$.

We compute $\vp[3]$ for each possible point count:
\begin{center}
\begin{tabular}{ccccccc}
\toprule
$\#E(\F_3)$ & 1 & 2 & 3 & 4 & 5 & 6 & 7 \\
\midrule
$v_3$ & 0 & 0 & 1 & 0 & 0 & 1 & 0 \\
\bottomrule
\end{tabular}
\end{center}
The maximum is $\vp[3](\#E(\F_3)) = 1$, achieved at
$\#E(\F_3) \in \{3, 6\}$.
Note that $\#E(\F_3) = 9$ would give $\NM = 2$, but $9 > 7$,
so this is excluded by the Hasse bound.

\medskip\noindent
\textbf{Case $p = 2$.}\quad
The Hasse bound gives $|a_2| \le 2\sqrt{2} \approx 2.83$, so
$a_2 \in \{-2, -1, 0, 1, 2\}$.
The point count $\#E(\F_2) = 3 - a_2$ ranges over $\{1, 2, 3, 4, 5\}$.

\begin{center}
\begin{tabular}{cccccc}
\toprule
$a_2$ & $-2$ & $-1$ & $0$ & $1$ & $2$ \\
\midrule
$\#E(\F_2)$ & 5 & 4 & 3 & 2 & 1 \\
$v_2(\#E(\F_2))$ & 0 & 2 & 0 & 1 & 0 \\
\bottomrule
\end{tabular}
\end{center}
The maximum is $v_2(4) = 2$, achieved when $a_2 = -1$.
Since $8 > 5$, $\NM \ge 3$ is impossible.
\end{proof}

\subsection{Rank independence}

\begin{proof}[Proof of \Cref{thm:rank-indep}]
The bound in \Cref{thm:main} depends only on the Hasse
bound~\eqref{eq:hasse}, which constrains $a_p$ independently
of the rank of~$E$.  The rank of $E$ over $\Q$ affects the
global arithmetic (the $L$-function, the Mordell--Weil group)
but not the local Frobenius trace at any individual prime.

Concretely, the Hasse bound $|a_p| \le 2\sqrt{p}$ holds for
all elliptic curves over $\Q$ at all primes of good reduction,
regardless of rank.  Since our proof uses nothing beyond this
bound, the result is rank-independent.
\end{proof}

\begin{remark}
The rank independence of $\NM$ contrasts sharply with the
Archimedean side.  Paper~59~\cite{P59} showed that the
Archimedean precision needed for decidability is sensitive to rank:
rank~$\ge 2$ curves require qualitatively different treatment
(Markov's principle is needed in the absence of a Lang-type bound).
Paper~61 explores this further, showing that Lang's conjecture
is the precise gate from $\MP$ to $\BISH$ for the rank obstruction.

The $p$-adic side, by contrast, is uniformly trivial.  This
asymmetry---$p$-adic uniform, Archimedean stratified---is a
structural feature of the decidability landscape for elliptic
curves.
\end{remark}

\subsection{Constructive consequence}

\begin{proof}[Proof of \Cref{thm:crm}]
By \Cref{thm:main}, $\NM \le 2$ for all $(E, p)$.
Given a Weierstrass equation for $E$ and a prime $p$ of good
reduction, the computation of $\NM$ proceeds as follows:
\begin{enumerate}[(1)]
  \item Reduce the equation modulo $p$ to obtain $E/\F_p$.
  \item Count $\#E(\F_p) = p + 1 - a_p$ by enumerating all
    $x \in \F_p$ and checking whether $x^3 + \cdots$ is a
    quadratic residue.  This is a finite computation in $\BISH$.
  \item Compute $\NM = \vp(\#E(\F_p))$ by repeated trial
    division by $p$.  This is also a finite computation.
  \item Since $\NM \le 2$, the result is determined by the
    value of $\#E(\F_p) \bmod p^2$ (or just $\bmod p$ for
    $p \ge 3$).  No infinite-precision computation or omniscience
    principle is needed.
\end{enumerate}
All steps are effective and require no appeal to $\LPO$, $\MP$,
or any other omniscience principle.
\end{proof}

% ====================================================================
% 4. COMPUTATIONAL VERIFICATION
% ====================================================================

\section{Computational Verification}

\subsection{Dataset}

We computed $\NM$ for a dataset of 1{,}812 elliptic curves,
comprising:
\begin{itemize}
  \item 136 named curves from Cremona's tables with conductor
    $\le 1{,}000$ (including curves of rank $0, 1, 2,$ and~$3$,
    various torsion groups, and CM discriminants);
  \item 1{,}676 curves obtained by systematic enumeration of short
    Weierstrass forms $y^2 = x^3 + Ax + B$ with
    $|A|, |B| \le 20$ and nonzero discriminant.
\end{itemize}

For each curve, we computed $a_p$ by direct point-counting
modulo $p$ for all 15 primes $p \le 47$ of good reduction,
yielding 23{,}454 $(E, p)$~pairs.

\subsection{Results}

\subsubsection{Global boundedness}

The global maximum of $\NM$ across all 23{,}454 computed pairs
is~$\mathbf{2}$, achieved at $p = 2$ (e.g., curve \texttt{15.a1}
with $a_2 = -1$, $\#E(\F_2) = 4$, $v_2(4) = 2$).

The distribution of $\max_p \NM$ per curve is:
\begin{center}
\begin{tabular}{lrr}
\toprule
$\max_p \NM$ & Curves & Percentage \\
\midrule
0 & 922 & 50.9\% \\
1 & 870 & 48.0\% \\
2 & 20  & 1.1\% \\
\bottomrule
\end{tabular}
\end{center}

\subsubsection{Confirmation of $\NM \le 1$ for $p \ge 5$}

Among all 23{,}454 computed pairs, \textbf{zero} have $\NM \ge 2$
with $p \ge 5$.  Similarly, zero pairs have $\NM \ge 2$ with
$p = 3$.  All 20 instances of $\NM = 2$ occur at $p = 2$,
confirming \Cref{thm:main} empirically.

\subsubsection{Small prime analysis}

\begin{center}
\begin{tabular}{lrrrrr}
\toprule
$p$ & $\NM = 0$ & $\NM = 1$ & $\NM = 2$ & $\NM \ge 3$ & Max $\NM$ \\
\midrule
2 & 41 & 9 & 20 & 0 & 2 \\
3 & 1{,}209 & 36 & 0 & 0 & 1 \\
5 & 1{,}228 & 223 & 0 & 0 & 1 \\
\bottomrule
\end{tabular}
\end{center}

\subsubsection{Rank correlation}

\begin{center}
\begin{tabular}{lrrl}
\toprule
Rank & Curves & Mean $\max_p \NM$ & Max $\max_p \NM$ \\
\midrule
0 & 111 & 0.802 & 2 \\
1 & 18  & 1.222 & 2 \\
2 & 6   & 0.667 & 1 \\
3 & 1   & 1.000 & 1 \\
\bottomrule
\end{tabular}
\end{center}

No systematic dependence of $\NM$ on rank is observed.  The slight
variation in means is attributable to small sample sizes for
rank~$\ge 2$.  The maximum $\NM$ value does not increase with rank.

\subsubsection{CM vs.\ non-CM}

\begin{center}
\begin{tabular}{lrr}
\toprule
Type & Curves & Mean $\max_p \NM$ \\
\midrule
CM & 18 & 0.278 \\
Non-CM & 1{,}794 & 0.504 \\
\bottomrule
\end{tabular}
\end{center}

CM curves show a slightly lower mean $\NM$, consistent with the
equidistribution of $a_p$ on a circle (rather than Sato--Tate)
making $a_p = 1$ less likely.  This does not affect the uniform
bound.

\subsubsection{Anomalous primes}

We found 1{,}119 anomalous $(E, p)$ pairs (where $a_p = 1$),
with a mean of $0.62$ anomalous primes per curve.

\subsection{Hasse bound comparison}

\Cref{fig:hasse} compares the theoretical maximum $\NM$
(computed from the Hasse bound by exhaustive search over allowed
$a_p$ values) with the empirical maximum observed in our dataset.

\begin{figure}[H]
\centering
\includegraphics[width=0.9\textwidth]{p64_hasse_bound_comparison.png}
\caption{Theoretical (Hasse bound) vs.\ empirical max $\NM$ by
prime.  The theoretical bound is $\NM = 2$ at $p = 2$ and
$\NM = 1$ for all $p \ge 3$.  Empirical data matches the
theoretical bound exactly.}
\label{fig:hasse}
\end{figure}

\begin{figure}[H]
\centering
\includegraphics[width=0.85\textwidth]{p64_histogram_max_NM.png}
\caption{Distribution of $\max_p \NM$ across all 1{,}812 curves.
Over half have $\max_p \NM = 0$ (i.e., $p \nmid \#E(\F_p)$ for
all primes tested), and 99\% have $\max_p \NM \le 1$.}
\label{fig:histogram}
\end{figure}

\begin{figure}[H]
\centering
\includegraphics[width=0.85\textwidth]{p64_small_primes.png}
\caption{Distribution of $\NM$ at small primes $p = 2, 3, 5$.
At $p = 2$, $\NM$ reaches 2; at $p = 3$ and $p = 5$,
$\NM \le 1$.}
\label{fig:small}
\end{figure}


% ====================================================================
% 5. CRM AUDIT
% ====================================================================

\section{CRM Audit}

\subsection{Constructive strength classification}

\begin{center}
\begin{tabular}{lll}
\toprule
Result & Strength & Principles used \\
\midrule
\Cref{thm:main} (uniform bound) & $\BISH$ & None (Hasse bound + arithmetic) \\
\Cref{thm:rank-indep} (rank independence) & $\BISH$ & None \\
\Cref{thm:crm} (constructive decidability) & $\BISH$ & None \\
Point counting algorithm & $\BISH$ & Finite enumeration \\
\bottomrule
\end{tabular}
\end{center}

\subsection{Descent pattern}

This paper completes a clean descent on the $p$-adic side of the
decidability problem:

\begin{center}
\begin{tabular}{lll}
\toprule
Paper & Result & Strength \\
\midrule
Paper 59 & $\NM$ is computable per $(E,p)$ & $\BISH$ \\
Paper 64 (this) & $\NM \le 2$ uniformly & $\BISH$ \\
\bottomrule
\end{tabular}
\end{center}

No omniscience principle is needed at any stage.  The $p$-adic
side requires only finite computation in bounded precision,
placing it firmly in $\BISH$.

\subsection{Comparison with Archimedean side}

The Archimedean decidability picture is more complex:

\begin{center}
\begin{tabular}{lll}
\toprule
Side & Uniform bound? & Strength \\
\midrule
$p$-adic (this paper) & Yes ($\NM \le 2$) & $\BISH$ \\
Archimedean, rank 0--1 (Paper 59) & $\BISH$-decidable & $\BISH$ \\
Archimedean, rank $\ge 2$ (Papers 59, 61) & Needs Lang bound & $\BISH{+}\MP$ \\
\bottomrule
\end{tabular}
\end{center}

The asymmetry is structural: the $p$-adic side is uniformly
trivial because the Hasse bound imposes a rigid constraint on
$a_p$, while the Archimedean side involves unbounded quantities
(rational points, heights) that create genuine logical
stratification.

% ====================================================================
% 6. FORMAL VERIFICATION
% ====================================================================

\section{Reproducibility}

This paper is a computational paper; no Lean formalization is
included.  The proof of \Cref{thm:main} is a short combinatorial
argument from the Hasse bound (integer arithmetic only) and does
not require machine verification.

All computational results are fully reproducible from the
following artifacts, deposited on Zenodo:
\begin{itemize}
  \item Python computation script: \texttt{p64\_compute.py}
  \item Full $(E, p, \NM)$ table: \texttt{p64\_N\_M\_table.csv}
    (23{,}454 rows)
  \item Per-curve summary: \texttt{p64\_curve\_summary.csv}
    (1{,}812 rows)
  \item Per-prime summary: \texttt{p64\_prime\_summary.csv}
  \item Hasse bound analysis: \texttt{p64\_hasse\_analysis.csv}
  \item Zenodo DOI: \texttt{10.5281/zenodo.18737090}
\end{itemize}

The script requires only Python~3 with \texttt{matplotlib} and
\texttt{numpy}; no SageMath or external database access is needed.
Point counting is performed by direct enumeration modulo~$p$.

% ====================================================================
% 7. DISCUSSION
% ====================================================================

\section{Discussion}

\subsection{The $p$-adic/Archimedean asymmetry}

The central finding of this paper is that the $p$-adic precision
bound $\NM$ is uniformly bounded by~2 (and by~1 for $p \ge 3$),
independently of the elliptic curve.  This stands in stark contrast
to the Archimedean side, where the decidability problem is sensitive
to rank and requires progressively stronger logical principles for
higher-rank curves.

This asymmetry has a clean explanation.  On the $p$-adic side, the
Hasse bound $|a_p| \le 2\sqrt{p}$ is a \emph{hard constraint} on the
Frobenius trace, and the point count $\#E(\F_p) = p + 1 - a_p$ is
tightly controlled: it lies in the interval
$[p + 1 - 2\sqrt{p}, \; p + 1 + 2\sqrt{p}]$, which for $p \ge 5$
does not contain any multiple of $p^2$.

On the Archimedean side, the relevant quantities (generators of the
Mordell--Weil group, N\'eron--Tate heights, regulators) are
\emph{unbounded} and grow with rank.  No analogue of the Hasse bound
constrains them, and their computation involves potentially unbounded
search---precisely the situation where omniscience principles become
relevant.

\subsection{Implications for the mixed-motive program}

Papers~50--53 established the DPT framework for calibrating conjectures
in arithmetic geometry against the constructive hierarchy.  The uniform
tameness of $\NM$ for elliptic curves suggests that $p$-adic
decidability may be uniformly tame for a broader class of motives.

The natural next question is: does $\NM = \vp(\#\mathcal{M}(\F_p))$
remain bounded for abelian surfaces, threefolds, or higher-dimensional
motives?  For weight-2 motives (which include elliptic curves), the
Hasse--Weil bound $|a_p| \le 2p^{(w-1)/2}$ with $w = 2$ gives
$|a_p| \le 2\sqrt{p}$, and our argument goes through.  For higher
weight, the bound $|a_p| \le 2p^{(w-1)/2}$ grows with $p$, and the
analysis becomes more delicate.

\subsection{Anomalous primes and uniformity}

The only obstruction to $\NM = 0$ for $p \ge 5$ is the anomalous
condition $a_p = 1$.  By Elkies~\cite{Elkies87}, every elliptic
curve over $\Q$ has infinitely many supersingular primes ($a_p = 0$),
and it is expected (but not proved in general) that anomalous primes
are also infinite in number for non-CM curves.

The distribution of anomalous primes---and hence the distribution
of $\NM = 1$ events---is governed by the Sato--Tate distribution
(for non-CM curves) or the equidistribution on a circle
(for CM curves).  In either case, the \emph{density} of anomalous
primes is zero (since $a_p = 1$ is a measure-zero event in the
limiting distribution), but their \emph{count} is expected to be
infinite.

\subsection{Independence from the three governing invariants}

The CRM Programme Roadmap identifies three invariants that govern
the decidability landscape for mixed motives: the rank~$r$, the
Hodge level~$\ell$, and the effective Lang constant~$c(A)$.
On the Archimedean side, these invariants create genuine
stratification---rank determines whether $\MP$ suffices or
$\LPO$ is required (Papers~59, 61), and Hodge level governs the
$\MP$/$\LPO$ boundary (Paper~62).

On the $p$-adic side, $\NM$ is independent of all three.  The
bound $\NM \le 2$ depends only on the Hasse constraint on $a_p$,
which is insensitive to rank, Hodge level, and Lang constants.
This makes the $p$-adic precision bound a \emph{universal constant}
of the decidability program rather than a variable tied to the
motive's arithmetic complexity.

\subsection{Open questions}

\begin{enumerate}[(1)]
  \item \textbf{Higher-dimensional motives:} Is $\NM$ uniformly
    bounded for abelian surfaces?  For $K3$ surfaces?  For
    arbitrary motives of fixed weight?
  \item \textbf{Quantitative anomalous prime counting:} For a
    given elliptic curve, what is the asymptotic density of
    primes with $\NM = 1$?
  \item \textbf{Function field analogue:} Does the uniform bound
    carry over to elliptic curves over function fields $\F_q(t)$?
\end{enumerate}

% ====================================================================
% 8. CONCLUSION
% ====================================================================

\section{Conclusion}

We have proved that the crystalline precision bound $\NM =
\vp(\#E(\F_p))$ satisfies $\NM \le 2$ for all elliptic curves
$E/\Q$ and all primes $p$ of good reduction, with $\NM = 2$
achievable only at $p = 2$.  For $p \ge 5$, $\NM \le 1$ with
equality if and only if the curve is anomalous at~$p$.  The proof
is a short application of the Hasse bound, verified computationally
on 1{,}812 curves across 23{,}454 $(E, p)$~pairs.

The constructive consequence is that $p$-adic crystalline
decidability for elliptic curves is uniformly $\BISH$-decidable,
requiring no omniscience principle.  This completes the $p$-adic
side of the decidability problem initiated in Paper~59, and
highlights a clean structural asymmetry: the $p$-adic side is
uniformly trivial, while the Archimedean side exhibits genuine
logical stratification tied to rank.

\paragraph{Scope of contribution.}
\Cref{thm:main} is a new observation, proved rigorously and
verified computationally.  The CRM calibration (\Cref{thm:crm})
is a direct consequence.  The computation is original.  The Hasse
bound itself is classical (Hasse, 1933).

% ====================================================================
% ACKNOWLEDGMENTS
% ====================================================================

\section*{Acknowledgments}

This paper was prepared with AI assistance (Claude, Anthropic)
for computation, code generation, and drafting.  The author is not
a domain expert in arithmetic geometry; the mathematical content
has been verified computationally and by formal argument, but has
not undergone independent peer review.

The Constructive Reverse Mathematics program builds on the
foundational work of Errett Bishop and the constructive mathematics
community.  We thank the Mathlib contributors for the Lean~4
formalization infrastructure.

% ====================================================================
% REFERENCES
% ====================================================================

\begin{thebibliography}{99}

\bibitem{BB85}
E.~Bishop and D.~Bridges,
\emph{Constructive Analysis},
Springer, 1985.

\bibitem{BR87}
D.~Bridges and F.~Richman,
\emph{Varieties of Constructive Mathematics},
London Math.\ Soc.\ Lecture Notes 97, Cambridge Univ.\ Press, 1987.

\bibitem{Cremona97}
J.~E.~Cremona,
\emph{Algorithms for Modular Elliptic Curves},
2nd ed., Cambridge Univ.\ Press, 1997.

\bibitem{Elkies87}
N.~D.~Elkies,
``The existence of infinitely many supersingular primes for every
elliptic curve over $\mathbb{Q}$,''
\emph{Invent.\ Math.}\ \textbf{89} (1987), 561--567.

\bibitem{Hasse33}
H.~Hasse,
``Beweis des Analogons der Riemannschen Vermutung f\"ur die
Artinschen und F.K.\ Schmidtschen Kongruenzzetafunktionen in
gewissen elliptischen F\"allen,''
\emph{Nachr.\ Ges.\ Wiss.\ G\"ottingen} (1933), 253--262.

\bibitem{LMFDB}
The LMFDB Collaboration,
\emph{The L-functions and Modular Forms DataBase},
\url{https://www.lmfdb.org}, 2024.

\bibitem{Silverman09}
J.~H.~Silverman,
\emph{The Arithmetic of Elliptic Curves},
2nd ed., Springer GTM 106, 2009.

\bibitem{P1}
P.~C.-K.~Lee,
``Paper~1: Constructive Reverse Mathematics for Physics---Overview,''
CRM Series, 2025.

\bibitem{P50}
P.~C.-K.~Lee,
``Paper~50: The DPT Framework---Decidable Polarized Tannakian
Categories,''
CRM Series, 2025.

\bibitem{P51}
P.~C.-K.~Lee,
``Paper~51: BSD and the DPT Framework,''
CRM Series, 2025.

\bibitem{P52}
P.~C.-K.~Lee,
``Paper~52: Bloch--Kato Selmer Groups under DPT,''
CRM Series, 2025.

\bibitem{P53}
P.~C.-K.~Lee,
``Paper~53: QCD Confinement and the DPT Lens,''
CRM Series, 2026.

\bibitem{P59}
P.~C.-K.~Lee,
``Paper~59: De Rham Decidability---$p$-Adic Precision for
Crystalline Equivalence,''
CRM Series, 2026.

\bibitem{P61}
P.~C.-K.~Lee,
``Paper~61: Lang's Conjecture as the MP$\to$BISH Gate,''
CRM Series, 2026.

\bibitem{P62}
P.~C.-K.~Lee,
``Paper~62: The Hodge Level Boundary,''
CRM Series, 2026.

\end{thebibliography}

\end{document}


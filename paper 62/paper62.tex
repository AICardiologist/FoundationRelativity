
\documentclass[11pt]{article}

% ------------------------------------------------------------
% Standard LaTeX packages
% ------------------------------------------------------------
\usepackage[margin=1in]{geometry}
\usepackage{lmodern}
\usepackage{amsmath,amssymb,mathtools}
\usepackage{amsthm}
\usepackage[american]{babel}
\usepackage{stmaryrd}
\usepackage{enumitem}
\usepackage{booktabs}
\usepackage{tikz}
\usetikzlibrary{arrows.meta,positioning,cd}
\usepackage{listings}
\usepackage[x11names,table]{xcolor}
\usepackage{graphicx}
\usepackage{array}
\usepackage{mdframed}
\usepackage{url}
\usepackage[colorlinks=true,linkcolor=blue,citecolor=blue,urlcolor=blue]{hyperref}

% Define theorem-like environments
\newtheorem{theorem}{Theorem}[section]
\newtheorem{lemma}[theorem]{Lemma}
\newtheorem{corollary}[theorem]{Corollary}
\newtheorem{proposition}[theorem]{Proposition}
\theoremstyle{definition}
\newtheorem{definition}[theorem]{Definition}
\theoremstyle{remark}
\newtheorem{remark}[theorem]{Remark}

% ---------- Lean repo link ----------
\newcommand{\leanRepo}{\url{https://doi.org/10.5281/zenodo.18736965}}
\newcommand{\leanok}{\textsf{\small \textcolor{green!70!black}{\checkmark}}}

% ---------- Mathematical notation ----------
\newcommand{\N}{\mathbb{N}}
\newcommand{\Z}{\mathbb{Z}}
\newcommand{\Q}{\mathbb{Q}}
\newcommand{\R}{\mathbb{R}}
\newcommand{\C}{\mathbb{C}}
\newcommand{\Qbar}{\overline{\Q}}
\newcommand{\Qell}{\Q_\ell}
\newcommand{\Qp}{\Q_p}
\newcommand{\Fq}{\mathbb{F}_q}
\newcommand{\Proj}{\mathbb{P}}
\newcommand{\WLPO}{\mathrm{WLPO}}
\newcommand{\LPO}{\mathrm{LPO}}
\newcommand{\MP}{\mathrm{MP}}
\newcommand{\BISH}{\mathrm{BISH}}
\newcommand{\CRM}{\mathrm{CRM}}
\newcommand{\LEM}{\mathrm{LEM}}
\newcommand{\adj}{\dagger}
\newcommand{\ip}[2]{\langle #1, #2 \rangle}
\newcommand{\CH}{\mathrm{CH}}
\newcommand{\Ext}{\mathrm{Ext}}
\newcommand{\Hom}{\mathrm{Hom}}
\newcommand{\Pic}{\mathrm{Pic}}

% ---------- Code listing style for Lean ----------
\definecolor{codegreen}{rgb}{0,0.6,0}
\definecolor{codegray}{rgb}{0.5,0.5,0.5}
\definecolor{codepurple}{rgb}{0.58,0,0.82}
\definecolor{backcolour}{rgb}{0.95,0.95,0.92}

\lstdefinelanguage{Lean}{
  keywords={theorem, lemma, def, definition, axiom, structure, class, instance,
            by, exact, intro, intros, apply, refine, constructor, use, obtain,
            have, show, from, fun, assume, let, in, if, then, else,
            match, with, end, namespace, section, variable, variables,
            example, begin, sorry, admit, noncomputable, classical,
            import, open, export, private, protected, mutual, meta,
            do, for, while, return, try, catch, finally,
            Type, Prop, Sort, Type*, forall, exists, where, extends,
            set, push_neg, rw, simp, omega, nlinarith, linarith,
            ext, rfl, congr, fin_cases, haveI, letI, attribute,
            rcases, split, native_decide, inductive, deriving},
  sensitive=true,
  morecomment=[l]{--},
  morecomment=[s]{/-}{-/},
  morestring=[b]",
  literate=
    {α}{{$\alpha$}}1 {β}{{$\beta$}}1 {γ}{{$\gamma$}}1
    {δ}{{$\delta$}}1 {ε}{{$\varepsilon$}}1 {ζ}{{$\zeta$}}1
    {η}{{$\eta$}}1 {θ}{{$\theta$}}1 {ι}{{$\iota$}}1
    {κ}{{$\kappa$}}1 {λ}{{$\lambda$}}1 {μ}{{$\mu$}}1
    {ν}{{$\nu$}}1 {ξ}{{$\xi$}}1 {π}{{$\pi$}}1
    {ρ}{{$\rho$}}1 {σ}{{$\sigma$}}1 {τ}{{$\tau$}}1
    {φ}{{$\varphi$}}1 {χ}{{$\chi$}}1 {ψ}{{$\psi$}}1
    {ω}{{$\omega$}}1 {Γ}{{$\Gamma$}}1 {Δ}{{$\Delta$}}1
    {Θ}{{$\Theta$}}1 {Λ}{{$\Lambda$}}1 {Σ}{{$\Sigma$}}1
    {Φ}{{$\Phi$}}1 {Ψ}{{$\Psi$}}1 {Ω}{{$\Omega$}}1
    {→}{{$\rightarrow$}}1 {←}{{$\leftarrow$}}1 {↔}{{$\leftrightarrow$}}1
    {⇒}{{$\Rightarrow$}}1 {⇐}{{$\Leftarrow$}}1 {⇔}{{$\Leftrightarrow$}}1
    {∀}{{$\forall$}}1 {∃}{{$\exists$}}1 {∈}{{$\in$}}1
    {∉}{{$\notin$}}1 {⊆}{{$\subseteq$}}1 {⊂}{{$\subset$}}1
    {∪}{{$\cup$}}1 {∩}{{$\cap$}}1 {≤}{{$\leq$}}1
    {≥}{{$\geq$}}1 {≠}{{$\neq$}}1 {≈}{{$\approx$}}1 {≃}{{$\simeq$}}1
    {≡}{{$\equiv$}}1 {∧}{{$\land$}}1 {∨}{{$\lor$}}1
    {¬}{{$\neg$}}1 {ℕ}{{$\mathbb{N}$}}1 {ℝ}{{$\mathbb{R}$}}1
    {ℂ}{{$\mathbb{C}$}}1 {ℤ}{{$\mathbb{Z}$}}1 {ℚ}{{$\mathbb{Q}$}}1 {ℓ}{{$\ell$}}1
    {·}{{$\cdot$}}1 {∑}{{$\sum$}}1 {∏}{{$\prod$}}1
    {∅}{{$\emptyset$}}1 {∞}{{$\infty$}}1 {∂}{{$\partial$}}1
    {⟨}{{$\langle$}}1 {⟩}{{$\rangle$}}1 {…}{{$\ldots$}}1
    {₀}{{$_0$}}1 {₁}{{$_1$}}1 {₂}{{$_2$}}1 {⧸}{{$/$}}1 {‖}{{$\|$}}1
    {•}{{$\cdot$}}1 {⁻¹}{{$^{-1}$}}1 {⋆}{{$\star$}}1
    {∘}{{$\circ$}}1
}

\lstdefinestyle{leanstyle}{
    language=Lean,
    backgroundcolor=\color{backcolour},
    commentstyle=\color{codegreen},
    keywordstyle=\color{blue},
    stringstyle=\color{codepurple},
    basicstyle=\ttfamily\footnotesize,
    breakatwhitespace=false,
    breaklines=true,
    captionpos=b,
    keepspaces=true,
    numbers=left,
    numbersep=5pt,
    showspaces=false,
    showstringspaces=false,
    showtabs=false,
    tabsize=2,
    numberstyle=\tiny\color{codegray}
}

\lstset{style=leanstyle}

% ---------- Title and author ----------
\title{The Hodge Level Boundary:\\
Archimedean Decidability for Mixed Motives\\[6pt]
{\large (Paper 62, Constructive Reverse Mathematics Series)}}
\author{Paul Chun-Kit Lee\thanks{Lean 4 formalization available at \leanRepo.} \\
New York University \\
\texttt{dr.paul.c.lee@gmail.com}}
\date{February 2026}

\begin{document}

\maketitle

\begin{abstract}
We identify the sharp boundary between Markov's Principle ($\MP$) and the Limited Principle of Omniscience ($\LPO$) for decidability of $\Ext^1(\Q(0), M)$ in the category of mixed motives. The boundary is the Hodge level $\ell$ of the motive $M$. When $\ell \leq 1$, the intermediate Jacobian $J^p$ is an abelian variety, Northcott's theorem transfers via the Abel--Jacobi map, and decidability requires at most~$\MP$. When $\ell \geq 2$, the intermediate Jacobian is a non-algebraic complex torus, Northcott's property fails, and decidability escalates to~$\LPO$. We prove that no intermediate ``weak Northcott'' property prevents this escalation: each degree-$d$ slice of the graded cycle space is $\BISH$-decidable, but quantifying over all degrees requires~$\LPO$. The cubic threefold (Clemens--Griffiths, Bloch--Murre) confirms the boundary at $\ell = 1$, while quintic Calabi--Yau threefolds confirm it at $\ell = 3$. Combined with Papers~59--61, the full Decidable Polarized Tannakian (DPT) hierarchy is governed by three invariants: rank~$r$, Hodge level~$\ell$, and Lang constant~$c$. As a consequence, cycle groups on Calabi--Yau threefolds---the compactification spaces of string theory---are structurally $\LPO$, beyond any search procedure. All results are formalized in Lean~4 over Mathlib; the bundle compiles with 0~errors, 0~warnings, and 0~\texttt{sorry}s.
\end{abstract}

\tableofcontents

% ===========================================================
\section{Introduction}
\label{sec:intro}
% ===========================================================

\subsection{Main results}

Let $X$ be a smooth projective variety over $\Q$ and let $M = h^{2p-1}(X)$ be the $(2p-1)$-th motive of~$X$. The Hodge level of $M$ is
\[
\ell(M) = \max\{|p - q| : h^{p,q}(X) \neq 0 \text{ in degree } 2p-1\}.
\]
The $p$-th intermediate Jacobian $J^p(X) = H^{2p-1}(X,\C) / (F^p \oplus H^{2p-1}(X,\Z))$ is an abelian variety if and only if $\ell \leq 1$; for $\ell \geq 2$, it is merely a non-algebraic complex torus. This dichotomy, classical in Hodge theory (Griffiths~\cite{Griffiths1968}), has a precise constructive counterpart that we identify in this paper.

We establish five theorems:

\begin{description}[leftmargin=2em]
\item[Theorem A (Algebraic Case).] \leanok\ If $\ell(M) \leq 1$, then $J^p$ is an abelian variety, the N\'eron--Tate height satisfies Northcott's property, and the Abel--Jacobi map transfers finiteness to $\CH^p(X)_{\mathrm{hom}}$. Decidability of $\Ext^1(\Q(0), M)$ requires at most~$\MP$. The cubic threefold serves as test case: by Clemens--Griffiths~\cite{ClemensGriffiths1972}, $J^2$ is an abelian $5$-fold; by Bloch--Murre~\cite{BlochMurre1979}, the Abel--Jacobi map is an isomorphism.

\item[Theorem B (Non-Algebraic Case).] \leanok\ If $\ell(M) \geq 2$, then $J^p$ is a non-algebraic complex torus, Northcott's property fails, and decidability escalates to~$\LPO$. Mumford's infinite-dimensionality theorem~\cite{Mumford1969} provides the mechanism. The quintic Calabi--Yau threefold (with $h^{3,0} = 1$, giving $\ell = 3$) is the paradigmatic example. A caveat applies to K3 surfaces: Bloch's conjecture makes the failure vacuous over~$\Q$.

\item[Theorem C (Four-Way Equivalence).] \leanok\ For intermediate Jacobian data, the following are equivalent: (i)~$h^{n,0} = 0$; (ii)~$J^p$ is algebraic; (iii)~Northcott's property holds; (iv)~decidability is at most~$\MP$. The boundary $h^{n,0} = 0 \leftrightarrow h^{n,0} \geq 1$ is itself $\BISH$-decidable.

\item[Theorem D (Isolation Gap Duality).] \leanok\ When $J^p$ is an abelian variety, Baker's theorem on linear forms in logarithms~\cite{Baker1966} provides a computable isolation gap. When $J^p$ is non-algebraic, no computable gap exists. Northcott's property and the isolation gap fail or succeed together (\texttt{common\_cause}).

\item[Theorem E (No Weak Northcott---Main Result).] \leanok\ $\LPO \leftrightarrow (\forall\, G : \mathrm{GradedCycleSpace},\; \mathrm{SaturationDecidable}(G))$. Each degree-$d$ piece of $G$ is $\BISH$-decidable, but quantifying over all degrees is exactly~$\LPO$. The reduction is explicit: given $f : \N \to \mathrm{Bool}$, construct the graded cycle space with $\mathrm{inSpan}(d) := (f(d) = \mathrm{false})$, and extract $\LPO$ from saturation decidability.
\end{description}

\subsection{Constructive Reverse Mathematics: a brief primer}

$\CRM$ calibrates mathematical statements against logical principles of increasing strength within Bishop-style constructive mathematics ($\BISH$). The hierarchy relevant to this paper is:
\[
\BISH \;\subset\; \BISH + \MP \;\subset\; \BISH + \WLPO \;\subset\; \BISH + \LPO \;\subset\; \text{CLASS}.
\]
Here $\LPO$ (Limited Principle of Omniscience) states that every binary sequence is identically zero or contains a~$1$. Markov's Principle ($\MP$) states that a binary sequence that is not everywhere zero contains a~$1$: $\neg\neg(\exists n,\; a(n) = 1) \to \exists n,\; a(n) = 1$. The gap between $\MP$ and $\LPO$ is precisely the gap between ``not not decidable'' and ``decidable.'' For a thorough treatment of $\CRM$, see Bridges--Richman~\cite{BridgesRichman1987}; for the broader program of which this paper is part, see Papers~1--61 of this series and the atlas survey~\cite{Paper50}.

\subsection{Current state of the art}

The constructive status of cycle groups has been studied in this series since Paper~59, which established the rank-based stratification ($r = 0$: $\BISH$; $r = 1$: $\BISH$; $r \geq 2$: $\MP$ via Lang's conjecture), and Paper~61, which identified the Lang constant~$c$ as the gate from $\MP$ to~$\BISH$. The present paper adds the Hodge level~$\ell$ as the third and final invariant, completing the DPT hierarchy.

The Hodge-theoretic classification of intermediate Jacobians goes back to Griffiths~\cite{Griffiths1968,GriffithsHarris1978}. The algebraic case ($\ell \leq 1$) includes elliptic curves, abelian varieties, and cubic threefolds. The non-algebraic case ($\ell \geq 2$) includes general Calabi--Yau threefolds and higher-dimensional varieties with $h^{n,0} \geq 1$. The connection to Northcott's property is classical; its constructive consequences are new.

\subsection{Position in the atlas}

This is Paper~62 of a series applying constructive reverse mathematics to arithmetic geometry, mathematical physics, and the ``five great conjectures'' program. Papers~2 and~7 calibrate Banach space non-reflexivity at $\WLPO$; Paper~8 treats the 1D Ising model and $\LPO$; Paper~45 identifies the de-omniscientizing descent for the Weight-Monodromy Conjecture. Papers~59--61 establish the rank-based and Lang-constant-based stratification of cycle groups. The present paper completes the picture by identifying Hodge level as the $\MP/\LPO$ frontier. This paper merges the original Papers~62 and~63, which treated the algebraic and non-algebraic cases separately; the merged treatment makes the sharp boundary more transparent.

% ===========================================================
\section{Preliminaries}
\label{sec:prelim}
% ===========================================================

\begin{definition}[Limited Principle of Omniscience]
$\LPO$ is the assertion that for every binary sequence $a : \N \to \{0,1\}$, either $\forall n,\; a(n) = 0$ or $\exists n,\; a(n) = 1$. In the Lean formalization, we use the $\Z$-valued form: $\LPO := \forall\, f : \N \to \Z,\; (\forall n,\; f(n) = 0) \lor (\exists n,\; f(n) \neq 0)$.
\end{definition}

\begin{definition}[Markov's Principle]
$\MP$ is the assertion that for every binary sequence $a : \N \to \{0,1\}$, if $\neg\neg(\exists n,\; a(n) = 1)$, then $\exists n,\; a(n) = 1$.
\end{definition}

\begin{definition}[Hodge data]
A Hodge datum consists of a finite sequence of non-negative integers $(h^{p,q})$ indexed by $p + q = n$ for some fixed degree~$n$, satisfying $h^{p,q} = h^{q,p}$ (Hodge symmetry). The Hodge level is $\ell = \max\{|p - q| : h^{p,q} \neq 0\}$.
\end{definition}

\begin{definition}[Height function and Northcott's property]
A height function on a set $S$ is a function $h : S \to \R_{\geq 0}$. It satisfies Northcott's property if for every bound $B \in \R$, the set $\{s \in S : h(s) \leq B\}$ is finite.
\end{definition}

\begin{definition}[Intermediate Jacobian]
For a smooth projective variety $X$ over $\C$ of dimension $n$, the $p$-th intermediate Jacobian is
\[
J^p(X) = H^{2p-1}(X, \C) / (F^p H^{2p-1} \oplus H^{2p-1}(X, \Z)),
\]
where $F^\bullet$ denotes the Hodge filtration. When $\ell \leq 1$ (equivalently, $h^{n,0} = 0$ in degree $2p-1$), $J^p$ is an abelian variety. When $\ell \geq 2$, $J^p$ is a non-algebraic complex torus.
\end{definition}

\begin{definition}[Abel--Jacobi map]
The Abel--Jacobi map $\mathrm{AJ} : \CH^p(X)_{\mathrm{hom}} \to J^p(X)$ sends a cycle $Z$ homologically equivalent to zero to the class of the period integral $\int_\Gamma \omega$ where $\partial \Gamma = Z$.
\end{definition}

\begin{definition}[Graded cycle space]
\label{def:graded}
A graded cycle space $G$ consists of:
\begin{enumerate}
\item A predicate $\mathrm{inSpan} : \N \to \mathrm{Prop}$ (whether degree-$d$ cycles generate),
\item A decidability witness $\mathrm{decidable\_graded} : \forall\, d,\; \mathrm{Decidable}(\mathrm{inSpan}(d))$.
\end{enumerate}
$G$ is \emph{saturated} if $\forall\, d,\; \mathrm{inSpan}(d)$. $\mathrm{SaturationDecidable}(G)$ asserts $(\forall d,\; \mathrm{inSpan}(d)) \lor \neg(\forall d,\; \mathrm{inSpan}(d))$.
\end{definition}

\begin{definition}[Logic level]
The inductive type $\mathrm{LogicLevel}$ classifies motives by constructive strength:
\[
\mathrm{LogicLevel} ::= \BISH \mid \MP \mid \LPO.
\]
\end{definition}

% ===========================================================
\section{Main Results}
\label{sec:results}
% ===========================================================

\subsection{Theorem A: Algebraic case ($\ell \leq 1 \Rightarrow$ MP)}

\begin{theorem}[Algebraic Case]
\label{thm:A}
Let $M$ be a motive with Hodge level $\ell(M) \leq 1$. Then:
\begin{enumerate}
\item The intermediate Jacobian $J^p$ is an abelian variety.
\item The N\'eron--Tate height on $J^p$ satisfies Northcott's property.
\item The Abel--Jacobi map transfers Northcott finiteness to $\CH^p(X)_{\mathrm{hom}}$.
\item Decidability of $\Ext^1(\Q(0), M)$ requires at most $\MP$.
\end{enumerate}
\end{theorem}

\begin{proof}
Part (1) is classical Hodge theory: when $\ell \leq 1$, the Hodge filtration on $H^{2p-1}(X, \C)$ satisfies $F^p \oplus \overline{F^p} = H^{2p-1}(X, \C)$, making $J^p$ a polarizable abelian variety. Part (2) uses the axiom \texttt{neronTate\_northcott}, which encapsulates the theorem of N\'eron~\cite{Neron1965} and Northcott~\cite{Northcott1949} that the N\'eron--Tate canonical height on an abelian variety over a number field satisfies the Northcott property. Part (3) follows from the Abel--Jacobi map being a group homomorphism: if $\mathrm{AJ}$ is injective (or an isomorphism, as for the cubic threefold), finiteness of preimages under bounded height transfers. Part (4): Northcott finiteness reduces torsion detection to a finite search, which is $\BISH$; the remaining Archimedean non-vanishing check uses Markov's Principle.

\emph{Test case: the cubic threefold.} Let $X \subset \Proj^4$ be a smooth cubic threefold. By Clemens--Griffiths~\cite{ClemensGriffiths1972}, $J^2(X)$ is a principally polarized abelian $5$-fold (the Hodge numbers are $h^{2,1} = h^{1,2} = 5$ with $h^{3,0} = h^{0,3} = 0$, giving $\ell = 1$). By Bloch--Murre~\cite{BlochMurre1979}, the Abel--Jacobi map $\mathrm{AJ} : \CH^2(X)_{\mathrm{hom}} \to J^2(X)$ is an isomorphism of groups. Therefore Northcott on $J^2(X)$ transfers exactly to $\CH^2(X)_{\mathrm{hom}}$.

In the formalization, the Hodge data for the cubic threefold is encoded as \texttt{cubicThreefoldHodge} with \texttt{hodgeLevel} computed by \texttt{native\_decide} to be $\leq 1$.
\end{proof}

\subsection{Theorem B: Non-algebraic case ($\ell \geq 2 \Rightarrow$ LPO)}

\begin{theorem}[Non-Algebraic Case]
\label{thm:B}
Let $M$ be a motive with Hodge level $\ell(M) \geq 2$. Then:
\begin{enumerate}
\item The intermediate Jacobian $J^p$ is a non-algebraic complex torus.
\item Northcott's property fails for $J^p$.
\item Decidability of $\Ext^1(\Q(0), M)$ requires $\LPO$.
\end{enumerate}
\end{theorem}

\begin{proof}
Part (1): When $\ell \geq 2$, we have $h^{n,0} \geq 1$, so $F^p \oplus \overline{F^p} \subsetneq H^{2p-1}(X, \C)$, and $J^p$ is not an abelian variety. Part (2) uses the axiom \texttt{mumford\_infinite\_dim}, which encapsulates Mumford's theorem~\cite{Mumford1969}: for a surface $S$ with $p_g > 0$, the Chow group $\CH_0(S)$ is infinite-dimensional (not supported on any curve). This infinite-dimensionality obstructs Northcott finiteness. Part (3): Without Northcott, the finite search strategy is unavailable, and decidability escalates from $\MP$ to~$\LPO$.

\emph{Paradigmatic example: the quintic CY3.} The Fermat quintic $X = \{x_0^5 + \cdots + x_4^5 = 0\} \subset \Proj^4$ has Hodge numbers $h^{3,0} = 1$, $h^{2,1} = 101$, giving $\ell = 3$. The intermediate Jacobian $J^2(X)$ is a complex torus of dimension $102$, but it is \emph{not} an abelian variety.

\emph{K3 caveat.} For a K3 surface $X$ over $\Q$, Bloch's conjecture predicts that $\CH_0(X)_{\mathrm{hom}} = 0$ (the kernel of the degree map is trivial). If Bloch's conjecture holds, the Northcott failure is vacuous: there are no non-trivial cycles to decide. The formalization records this as a structural caveat, not a logical obstacle.

In the formalization, the Hodge data for the quintic CY3 is encoded as \texttt{quinticCY3Hodge} with \texttt{hodgeLevel} computed by \texttt{native\_decide} to be $\geq 2$.
\end{proof}

\subsection{Theorem C: Four-way equivalence}

\begin{theorem}[Four-Way Equivalence]
\label{thm:C}
For intermediate Jacobian data $(n, \{h^{p,q}\})$, the following are equivalent:
\begin{enumerate}
\item $h^{n,0} = 0$ (no holomorphic $n$-forms).
\item $J^p$ is an algebraic intermediate Jacobian (abelian variety).
\item Northcott's property holds for the height function on $J^p$.
\item Decidability requires at most $\MP$.
\end{enumerate}
Moreover, the boundary $h^{n,0} = 0 \leftrightarrow h^{n,0} \geq 1$ is itself $\BISH$-decidable: $h^{n,0}$ is a concrete natural number, and $\N$ has decidable equality.
\end{theorem}

\begin{proof}
$(1) \Leftrightarrow (2)$: Classical Hodge theory. The intermediate Jacobian $J^p$ is algebraic if and only if the Hodge decomposition satisfies $F^p \oplus \overline{F^p} = H^{2p-1}$, which occurs precisely when $h^{n,0} = 0$.

$(2) \Rightarrow (3)$: Abelian variety $\Rightarrow$ N\'eron--Tate height $\Rightarrow$ Northcott (Theorem~A).

$(3) \Rightarrow (4)$: Northcott finiteness reduces infinite decidability questions to finite ones, which are $\BISH$; the Archimedean check adds at most $\MP$.

$(4) \Rightarrow (1)$: Contrapositive. If $h^{n,0} \geq 1$, then $J^p$ is non-algebraic, Northcott fails, and $\LPO$ is required (Theorem~B), so decidability is \emph{not} at most $\MP$ (since $\MP \subsetneq \LPO$).

In the formalization, the four characterizations are bundled into a structure \texttt{FourCharacterizations} and the equivalence is proved as \texttt{four\_way\_equivalence}. The $\BISH$-decidability of the boundary is \texttt{boundary\_is\_bish\_decidable}.
\end{proof}

\subsection{Theorem D: Isolation gap duality}

\begin{theorem}[Isolation Gap Duality]
\label{thm:D}
Let $J^p$ be an intermediate Jacobian.
\begin{enumerate}
\item If $J^p$ is an abelian variety, then Baker's theorem~\cite{Baker1966} on linear forms in logarithms provides a computable isolation gap: for any algebraic point $P \in J^p(\Qbar)$, either $P$ is torsion or $|P - T| \geq c(h(P), \deg P)^{-\kappa}$ for all torsion points~$T$, where $c$ and $\kappa$ are effectively computable.
\item If $J^p$ is a non-algebraic complex torus, no computable isolation gap exists.
\item \emph{Common cause}: Northcott's property and the isolation gap fail or succeed together, because both depend on the algebraicity of $J^p$.
\end{enumerate}
\end{theorem}

\begin{proof}
Part (1) uses the axiom \texttt{baker\_lower\_bound}, which encapsulates Baker's theorem~\cite{Baker1966}: a non-trivial linear combination $\beta_1 \log \alpha_1 + \cdots + \beta_n \log \alpha_n$ of logarithms of algebraic numbers with algebraic coefficients is either zero or bounded below by an effectively computable function of the heights and degrees. On an abelian variety, the exponential map relates points to linear forms in periods; Baker's theorem provides the gap.

Part (2): For a non-algebraic complex torus, the period lattice has transcendental generators (not algebraic over $\Qbar$), and Baker's theorem does not apply. No alternative computable gap is known.

Part (3): The \texttt{common\_cause} theorem in the formalization establishes that algebraicity of $J^p$ is the common root: $J^p$ algebraic $\Leftrightarrow$ Northcott holds $\Leftrightarrow$ isolation gap exists.

\emph{Fermat quintic illustration.} For the Fermat quintic CY3, the period lattice of $J^2$ involves the transcendental periods $\Omega = \int_{X} \omega$, where $\omega$ is the holomorphic $3$-form. These periods are not algebraic over $\Q$, and Baker's method provides no lower bound.
\end{proof}

\subsection{Theorem E: No weak Northcott (Main Result)}

\begin{theorem}[No Weak Northcott]
\label{thm:E}
The following are equivalent:
\[
\LPO \;\;\leftrightarrow\;\; \bigl(\forall\, G : \mathrm{GradedCycleSpace},\; \mathrm{SaturationDecidable}(G)\bigr).
\]
Moreover, each degree-$d$ slice of $G$ is $\BISH$-decidable (\texttt{graded\_BISH\_whole\_LPO}), but quantifying over all degrees is exactly~$\LPO$.
\end{theorem}

\begin{proof}
$(\Rightarrow)$\; Assume $\LPO$. Let $G$ be a graded cycle space. Define $f : \N \to \Z$ by
\[
f(d) = \begin{cases} 0 & \text{if } G.\mathrm{decidable\_graded}(d) \text{ returns } \mathrm{isTrue}, \\ 1 & \text{if } G.\mathrm{decidable\_graded}(d) \text{ returns } \mathrm{isFalse}. \end{cases}
\]
By $\LPO$, either $\forall d,\; f(d) = 0$ (which gives saturation) or $\exists d,\; f(d) \neq 0$ (which gives non-saturation). Hence $\mathrm{SaturationDecidable}(G)$.

$(\Leftarrow)$\; Assume $\forall G,\; \mathrm{SaturationDecidable}(G)$. Given $f : \N \to \Z$, construct the graded cycle space $G_f$ with $\mathrm{inSpan}(d) := (f(d) = 0)$. Then:
\begin{itemize}
\item $G_f$ is saturated $\iff$ $\forall d,\; f(d) = 0$,
\item $G_f$ is not saturated $\iff$ $\exists d,\; f(d) \neq 0$.
\end{itemize}
Apply the hypothesis to $G_f$ to obtain $(\forall d,\; f(d) = 0) \lor (\exists d,\; f(d) \neq 0)$, which is exactly $\LPO$.

The theorem \texttt{graded\_BISH\_whole\_LPO} makes the phenomenon precise: for each fixed $d$, the proposition $\mathrm{inSpan}(d)$ is decidable (this is part of the definition of a graded cycle space). But the universal quantification $\forall d,\; \mathrm{inSpan}(d)$ over all degrees is \emph{not} decidable in $\BISH$---deciding it is equivalent to~$\LPO$.

This is the ``no weak Northcott'' result: there is no intermediate ``weak Northcott'' property that would reduce $\LPO$ to something between $\MP$ and $\LPO$. Each degree piece is $\BISH$, but the whole is irreducibly $\LPO$.
\end{proof}

\subsection{Hodge level classification}

\begin{theorem}[Sharp Boundary]
\label{thm:sharp}
The function $\mathrm{classifyMotive}$ assigns the correct logic level:
\begin{enumerate}
\item If $\ell \leq 1$, then $\mathrm{classifyMotive}$ returns $\MP$.
\item If $\ell \geq 2$, then $\mathrm{classifyMotive}$ returns $\LPO$.
\end{enumerate}
Moreover, Hodge level dominates rank: $\forall\, r,\; \mathrm{classifyLogicLevel}(r, \mathrm{true}) = \LPO$, where the boolean flag indicates $\ell \geq 2$.
\end{theorem}

\begin{proof}
Direct computation, verified by \texttt{native\_decide} on concrete Hodge data:
\begin{itemize}
\item Elliptic curve: $h^{1,0} = h^{0,1} = 1$, $\ell = 1 \leq 1$ $\Rightarrow$ $\MP$.
\item Cubic threefold: $h^{2,1} = h^{1,2} = 5$, $h^{3,0} = h^{0,3} = 0$, $\ell = 1 \leq 1$ $\Rightarrow$ $\MP$.
\item Quintic CY3: $h^{3,0} = 1$, $h^{2,1} = 101$, $\ell = 3 \geq 2$ $\Rightarrow$ $\LPO$.
\end{itemize}
The theorem \texttt{hodge\_dominates\_rank} establishes that once $\ell \geq 2$, the logic level is $\LPO$ regardless of the rank~$r$. This is because the Hodge-level obstruction (non-algebraic intermediate Jacobian) is independent of the rank-based obstruction (Lang's conjecture).
\end{proof}

% ===========================================================
\section{CRM Audit}
\label{sec:crm}
% ===========================================================

\subsection{Constructive strength classification}

\begin{center}
\begin{tabular}{llll}
\toprule
\textbf{Result} & \textbf{Strength} & \textbf{Custom axioms} & \textbf{Proof type} \\
\midrule
Theorem A (Algebraic Case) & $\BISH$ (from axioms) & \texttt{neronTate\_northcott} & Derivation \\
Theorem B (Non-Algebraic Case) & $\BISH$ (from axioms) & \texttt{mumford\_infinite\_dim} & Derivation \\
Theorem C (Four-Way Equivalence) & $\BISH$ & None & Full proof \\
Theorem D (Isolation Gap) & $\BISH$ (from axioms) & \texttt{baker\_lower\_bound} & Derivation \\
Theorem E (No Weak Northcott) & \textbf{Fully constructive} & None & Full proof \\
Classification & $\BISH$ & None & \texttt{native\_decide} \\
\bottomrule
\end{tabular}
\end{center}

\smallskip\noindent
\emph{Key observation.} Theorem~E, the main result, is \emph{fully constructive}: the $\LPO$ reduction is an explicit term-level construction with no axioms, no classical reasoning, and no \texttt{sorry}. The bidirectional reduction between $\LPO$ and saturation decidability is given by explicit functions \texttt{lpoReduction} (backward) and the $\Z$-valued encoding (forward).

\subsection{The DPT hierarchy}

Combined with Papers~59--61, the full Decidable Polarized Tannakian (DPT) hierarchy is governed by three invariants:

\begin{center}
\begin{tabular}{cccccc}
\toprule
\textbf{Rank $r$} & \textbf{Hodge $\ell$} & \textbf{Lang?} & \textbf{Northcott} & \textbf{Logic} & \textbf{Gate to $\BISH$} \\
\midrule
$r = 0$ & any & --- & --- & $\BISH$ & --- \\
$r = 1$ & $\leq 1$ & --- & Yes & $\BISH$ & --- \\
$r \geq 2$ & $\leq 1$ & Yes & Yes & $\BISH$ & Lang (Paper~61) \\
$r \geq 2$ & $\leq 1$ & No & Yes & $\MP$ & --- \\
any & $\geq 2$ & --- & No & $\LPO$ & Structurally blocked \\
\bottomrule
\end{tabular}
\end{center}

\subsection{Concrete motives}

\begin{center}
\begin{tabular}{lllc}
\toprule
\textbf{Motive} & \textbf{Cycle Group} & \textbf{Hodge $\ell$} & \textbf{Logic} \\
\midrule
Elliptic curve $E/\Q$ & $E(\Q)$ & $1$ & $\MP$ \\
Abelian variety $A/\Q$ & $A(\Q)$ & $1$ & $\MP$ \\
K3 surface, $\CH^1$ & $\Pic(X)$ & $0$ & $\MP$ \\
Cubic threefold & $\CH^2(X)_{\mathrm{hom}}$ & $1$ & $\MP$ \\
Quintic CY3 & $\CH^2(X)_{\mathrm{hom}}$ & $3$ & $\LPO$ \\
$K_2(E)$ & Beilinson regulator & --- & $\LPO$ \\
\bottomrule
\end{tabular}
\end{center}

\subsection{Comparison with earlier calibration patterns}

This paper exhibits the same structural pattern as Paper~45 (de-omniscientizing descent for the Weight-Monodromy Conjecture):
\begin{enumerate}
\item Identify the constructive obstruction ($\LPO$ for non-algebraic intermediate Jacobians).
\item Prove an equivalence (Theorem~E: $\LPO \leftrightarrow$ saturation decidability).
\item Classify the boundary (Theorem~C: Hodge level $\ell$ is the invariant).
\item Show no intermediate principle suffices (Theorem~E: no weak Northcott).
\end{enumerate}
The novelty relative to Paper~45 is the \emph{sharp boundary}: the transition from $\MP$ to $\LPO$ is governed by a single computable integer ($\ell$), making the classification effective.

% ===========================================================
\section{Formal Verification}
\label{sec:formal}
% ===========================================================

\subsection{File structure and build status}

The Lean 4 bundle resides at \texttt{paper~62/P62\_NorthcottLPO/} with the following structure:

\begin{center}
\begin{tabular}{lrl}
\toprule
\textbf{File} & \textbf{Lines} & \textbf{Content} \\
\midrule
\texttt{Defs.lean} & 121 & LPO ($\Z$-valued), MP, HeightFunction, \\
& & Northcott, AJTarget, hodgeLevel, GradedCycleSpace \\
\texttt{NorthcottTransfer.lean} & 105 & neronTate\_northcott axiom, AJIsomorphism, \\
& & abelian\_target\_gives\_northcott, cubic threefold \\
\texttt{NorthcottFailure.lean} & 110 & mumford\_infinite\_dim axiom, \\
& & nonalgebraic\_target\_northcott\_fails, K3 caveat \\
\texttt{NoWeakNorthcott.lean} & 139 & lpoReduction, no\_weak\_northcott, \\
& & graded\_BISH\_whole\_LPO, no\_intermediate\_condition \\
\texttt{HodgeBoundary.lean} & 141 & LogicTier, classifyMotive, Hodge examples, \\
& & sharp\_boundary \\
\texttt{IsolationGap.lean} & 155 & baker\_lower\_bound axiom, isolation\_gap\_duality, \\
& & common\_cause \\
\texttt{Main.lean} & 183 & paper62\_summary, hierarchy\_exhaustive, \\
& & lpo\_dominates, lpo\_unresolvable, axiom audit \\
\midrule
\textbf{Total} & \textbf{954} & \textbf{7 files} \\
\bottomrule
\end{tabular}
\end{center}

\medskip\noindent
\textbf{Build status:} \texttt{lake build} $\to$ \textbf{0 errors, 0 warnings, 0 \texttt{sorry}s}, 3117 jobs. Lean 4 version: \texttt{v4.29.0-rc1}. Mathlib4 dependency via \texttt{lakefile.lean}.

\subsection{Axiom inventory}

The formalization uses 3 custom axioms, all corresponding to deep classical theorems:

\begin{center}
\begin{tabular}{rllp{7cm}}
\toprule
\textbf{\#} & \textbf{Axiom} & \textbf{Used by} & \textbf{Mathematical content} \\
\midrule
1 & \texttt{neronTate\_northcott} & Thm.~A & N\'eron--Tate height satisfies Northcott on abelian varieties (N\'eron 1965, Northcott 1949) \\
2 & \texttt{mumford\_infinite\_dim} & Thm.~B & Mumford's infinite-dimensionality of $\CH_0$ for $p_g > 0$ (Mumford 1969) \\
3 & \texttt{baker\_lower\_bound} & Thm.~D & Baker's theorem on linear forms in logarithms (Baker 1966) \\
\bottomrule
\end{tabular}
\end{center}

\medskip\noindent
\textbf{Infrastructure axioms:} \texttt{propext}, \texttt{Classical.choice} (Mathlib $\R$ construction), \texttt{Quot.sound}. These are standard Lean/Mathlib infrastructure axioms present in all formalizations over $\R$.

\subsection{Key code snippets}

\textbf{Theorem E: No weak Northcott} (the main result, fully constructive):

\begin{lstlisting}
theorem no_weak_northcott :
    LPO ↔ (∀ (G : GradedCycleSpace), SaturationDecidable G) := by
  constructor
  · intro hlpo G
    let f : ℕ → ℤ := fun d =>
      match G.decidable_graded d with
      | .isTrue _  => 0
      | .isFalse _ => 1
    rcases hlpo f with hall | ⟨n, hn⟩
    · left; intro d
      have hfd := hall d
      simp only [f] at hfd
      match hd : G.decidable_graded d with
      | .isTrue h  => exact h
      | .isFalse _ => simp [hd] at hfd
    · right; intro hsat
      have hsn := hsat n
      simp only [f] at hn
      match hd : G.decidable_graded n with
      | .isTrue _  => simp [hd] at hn
      | .isFalse h => exact h hsn
  · intro hdec f
    let G := lpoReduction f
    rcases hdec G with hyes | hno
    · left; exact (lpo_reduction_saturation f).mp hyes
    · right
      have : ¬(∀ d, f d = 0) := by
        intro h; exact hno ((lpo_reduction_saturation f).mpr h)
      push_neg at this; exact this
\end{lstlisting}

\textbf{Hodge level classification} (computed by \texttt{native\_decide}):

\begin{lstlisting}
theorem elliptic_is_MP :
    hodgeLevel ellipticCurve_hodge ≤ 1 := by native_decide

theorem quintic_is_LPO :
    hodgeLevel quinticCY3_hodge ≥ 2 := by native_decide
\end{lstlisting}

\textbf{Hodge dominates rank}:

\begin{lstlisting}
theorem hodge_dominates_rank :
    ∀ r, classifyLogicLevel r true = LogicLevel.LPO := by
  intro r; simp [classifyLogicLevel]
\end{lstlisting}

\textbf{LPO reduction construction} (backward direction of Theorem~E):

\begin{lstlisting}
def lpoReduction (f : ℕ → ℤ) : GradedCycleSpace where
  inSpan := fun d => f d = 0
  decidable_graded := fun _d => inferInstance
\end{lstlisting}

\subsection{\texttt{\#print axioms} output}

\begin{center}
\small
\begin{tabular}{ll}
\toprule
\textbf{Theorem} & \textbf{Custom axioms} \\
\midrule
\texttt{no\_weak\_northcott} (Thm.~E) & \textbf{None} \\
\texttt{four\_way\_equivalence} (Thm.~C) & \textbf{None} \\
\texttt{boundary\_is\_bish\_decidable} & \textbf{None} \\
\texttt{abelian\_target\_gives\_northcott} (Thm.~A) & \texttt{neronTate\_northcott} \\
\texttt{nonalgebraic\_target\_northcott\_fails} (Thm.~B) & \texttt{mumford\_infinite\_dim} \\
\texttt{isolation\_gap\_duality} (Thm.~D) & \texttt{baker\_lower\_bound} \\
\texttt{common\_cause} & \texttt{neronTate\_northcott}, \texttt{baker\_lower\_bound} \\
\texttt{sharp\_boundary} & \textbf{None} \\
\texttt{hodge\_dominates\_rank} & \textbf{None} \\
\texttt{paper62\_summary} & All 3 custom axioms \\
\bottomrule
\end{tabular}
\end{center}

\medskip\noindent
\textbf{Classical.choice audit.} The infrastructure axiom \texttt{Classical.choice} appears in all theorems involving $\R$ due to Mathlib's construction of $\R$ as a Cauchy completion. This is an infrastructure artifact: all theorems over $\R$ in Lean/Mathlib carry \texttt{Classical.choice}. The constructive stratification is established by \emph{proof content}---explicit witnesses vs.\ principle-as-hypothesis---not by the axiom checker output (cf.\ Paper~10, \S Methodology).

Critically, the main result \texttt{no\_weak\_northcott} uses \emph{no custom axioms}: its \texttt{\#print axioms} output shows only the standard infrastructure triple (\texttt{propext}, \texttt{Classical.choice}, \texttt{Quot.sound}), with \texttt{Classical.choice} appearing solely from the $\R$-valued bound field in \texttt{GradedCycleSpace}. The $\LPO$ reduction itself is entirely constructive.

% ===========================================================
\section{Discussion}
\label{sec:discuss}
% ===========================================================

\subsection{The Hodge level as the MP/LPO frontier}

The central discovery of this paper is that the Hodge level $\ell$ is the \emph{sharp} invariant controlling the transition from $\MP$ to~$\LPO$ in the constructive decidability of cycle groups. This sharpness has three aspects:
\begin{enumerate}
\item \emph{Computable}: $\ell$ is computed from the Hodge numbers $h^{p,q}$, which are finite non-negative integers. The boundary $\ell \leq 1$ vs.\ $\ell \geq 2$ is a decidable predicate on Hodge data.
\item \emph{Complete}: Every motive falls on exactly one side. There is no ``gap'' or ``undetermined'' region.
\item \emph{Irreducible}: No ``weak Northcott'' property can bridge the gap (Theorem~E). The escalation from $\MP$ to $\LPO$ is not an artifact of the proof strategy but a structural feature of the mathematics.
\end{enumerate}

\subsection{Connection to de-omniscientizing descent}

Paper~45 identified the de-omniscientizing descent pattern: geometric origin reduces the logical strength of spectral sequence degeneration from $\LPO$ to~$\BISH$ by descending the coefficient field from $\Qell$ to~$\Qbar$. The present paper reveals a different mechanism: the Hodge level controls whether the \emph{target space} (the intermediate Jacobian) is algebraic or transcendental. When it is algebraic, N\'eron--Tate heights provide the Northcott property, and the logical strength is~$\MP$. When it is transcendental, no height-based finiteness is available, and the logical strength is~$\LPO$.

The two mechanisms are complementary: Paper~45's descent operates on the \emph{coefficient field}, while the present paper's boundary operates on the \emph{target geometry}. Together, they provide a complete picture of where $\LPO$ arises in the motivic landscape.

\subsection{String theory consequence}

Calabi--Yau threefolds are the compactification spaces of string theory. A CY3 with $h^{3,0} \geq 1$ (which is all of them, since $h^{3,0} = 1$ for CY3s by definition of the Calabi--Yau condition) has Hodge level $\ell = 3 \geq 2$, placing its cycle group $\CH^2(X)_{\mathrm{hom}}$ squarely in the $\LPO$ regime.

This is a \emph{logical} obstruction, not a computational one. Comparison with other complexity-theoretic results in the string landscape:
\begin{itemize}
\item \textbf{Denef--Douglas}~\cite{DenefDouglas2004}: The problem of finding flux vacua is NP-hard. This is about \emph{running time}.
\item \textbf{Bousso--Polchinski}~\cite{BoussoPolchinski2000}: The landscape contains $\sim 10^{500}$ vacua. This is about \emph{cardinality}.
\item \textbf{This paper}: The cycle groups of CY3 compactifications are $\LPO$. This is about \emph{decidability type}---a fundamentally different category from running time or cardinality.
\end{itemize}

\noindent\emph{Open question.} Do physical observables (e.g., Yukawa couplings, gauge coupling constants) inherit the $\LPO$ status from the cycle groups? Or does the physical projection reduce the logical strength?

\subsection{Open questions}

\begin{enumerate}
\item Can the $\LPO$ classification for $\ell \geq 2$ be refined to $\WLPO$ by considering weaker forms of decidability (approximate decidability, decidability up to $\varepsilon$)?
\item Is there a natural family of motives that sits exactly at $\WLPO$, intermediate between $\MP$ and $\LPO$?
\item Does the Hodge level boundary extend to motives over function fields (positive characteristic)?
\item Can the isolation gap of Theorem~D be quantified for specific abelian varieties (e.g., Jacobians of hyperelliptic curves)?
\end{enumerate}

% ===========================================================
\section{Conclusion}
\label{sec:conclusion}
% ===========================================================

We have identified the Hodge level $\ell$ as the sharp invariant governing the $\MP/\LPO$ frontier for cycle groups of mixed motives. The main results are:

\begin{itemize}
\item When $\ell \leq 1$, the intermediate Jacobian is an abelian variety, Northcott's property holds via N\'eron--Tate heights, and decidability requires at most $\MP$ (Lean-verified from axiom).
\item When $\ell \geq 2$, the intermediate Jacobian is non-algebraic, Northcott's property fails, and decidability escalates to $\LPO$ (Lean-verified from axiom).
\item These two regimes are connected by a four-way equivalence with a $\BISH$-decidable boundary (Lean-verified, full proof).
\item Baker's isolation gap and Northcott's property fail or succeed together (Lean-verified from axiom).
\item No ``weak Northcott'' property can bridge the gap: each degree piece is $\BISH$, but the universal quantification is irreducibly $\LPO$ (Lean-verified, full proof, no axioms).
\end{itemize}

\noindent Combined with Papers~59--61, the DPT hierarchy is now complete: cycle groups are classified by the triple $(r, \ell, c)$ of rank, Hodge level, and Lang constant. The hierarchy exhaustively partitions all motives into the strata $\BISH$, $\MP$, and $\LPO$, with computable boundaries between strata.

The string-theoretic consequence is striking: Calabi--Yau threefolds---the compactification spaces of string theory---have $\ell = 3 \geq 2$, placing their cycle groups structurally at~$\LPO$. This is not a computational difficulty but a logical one: no search procedure, however clever, can decide saturation of graded cycle groups in~$\BISH$.

% ===========================================================
\section*{Acknowledgments}
\addcontentsline{toc}{section}{Acknowledgments}
% ===========================================================

We thank the Mathlib contributors for the infrastructure on natural numbers, decidability, and Bool that made the constructive reductions possible. We are grateful to the constructive reverse mathematics community---especially the foundational work of Bishop, Bridges, Richman, and Ishihara---for developing the framework that makes calibrations like these possible. This paper is dedicated to Phillip Griffiths, whose work on intermediate Jacobians and Hodge theory provides the geometric foundation for the $\MP/\LPO$ boundary identified here.

The Lean 4 formalization was produced using AI code generation (Claude Code, Opus 4.6) under human direction. The author is a practicing cardiologist rather than a professional logician or arithmetic geometer; all mathematical claims should be evaluated on their formal content. We welcome constructive feedback from domain experts.

% ===========================================================
% References
% ===========================================================
\begin{thebibliography}{99}

\bibitem{Baker1966}
A.~Baker.
\newblock Linear forms in the logarithms of algebraic numbers.
\newblock \emph{Mathematika}, 13:204--216, 1966.

\bibitem{BishopBridges1985}
E.~Bishop and D.~Bridges.
\newblock \emph{Constructive Analysis}.
\newblock Springer, 1985.

\bibitem{BlochMurre1979}
S.~Bloch and J.~P. Murre.
\newblock On the Chow group of certain types of Fano threefolds.
\newblock \emph{Compositio Math.}, 39:47--105, 1979.

\bibitem{BoussoPolchinski2000}
R.~Bousso and J.~Polchinski.
\newblock Quantization of four-form fluxes and dynamical neutralization of the cosmological constant.
\newblock \emph{JHEP}, 2000(06):006, 2000.

\bibitem{BridgesRichman1987}
D.~Bridges and F.~Richman.
\newblock \emph{Varieties of Constructive Mathematics}.
\newblock LMS Lecture Note Series 97. Cambridge University Press, 1987.

\bibitem{BridgesVita2006}
D.~Bridges and L.~V\^{\i}\c{t}\u{a}.
\newblock \emph{Techniques of Constructive Analysis}.
\newblock Springer, 2006.

\bibitem{ClemensGriffiths1972}
C.~H. Clemens and P.~A. Griffiths.
\newblock The intermediate Jacobian of the cubic threefold.
\newblock \emph{Ann. of Math.}, 95:281--356, 1972.

\bibitem{Deligne1970}
P.~Deligne.
\newblock Th\'eorie de Hodge I.
\newblock In \emph{Actes du Congr\`es International des Math\'ematiciens (Nice, 1970)}, pages 425--430. Gauthier-Villars, 1971.

\bibitem{Deligne1980}
P.~Deligne.
\newblock La conjecture de Weil II.
\newblock \emph{Publ. Math. IH\'ES}, 52:137--252, 1980.

\bibitem{DenefDouglas2004}
F.~Denef and M.~R. Douglas.
\newblock Computational complexity of the landscape.
\newblock \emph{Ann. Physics}, 322:1096--1142, 2007. Preprint 2004.

\bibitem{Griffiths1968}
P.~A. Griffiths.
\newblock Periods of integrals on algebraic manifolds, I and II.
\newblock \emph{Amer. J. Math.}, 90:568--626 and 805--865, 1968.

\bibitem{GriffithsHarris1978}
P.~Griffiths and J.~Harris.
\newblock \emph{Principles of Algebraic Geometry}.
\newblock Wiley, 1978.

\bibitem{Grothendieck}
A.~Grothendieck.
\newblock Standard conjectures on algebraic cycles.
\newblock In \emph{Algebraic Geometry, Bombay 1968}, pages 193--199. Oxford University Press, 1969.

\bibitem{Ishihara2006}
H.~Ishihara.
\newblock Reverse mathematics in Bishop's constructive mathematics.
\newblock \emph{Philosophia Scientiae}, CS~6:43--59, 2006.

\bibitem{Mumford1969}
D.~Mumford.
\newblock Rational equivalence of $0$-cycles on surfaces.
\newblock \emph{J. Math. Kyoto Univ.}, 9:195--204, 1969.

\bibitem{Neron1965}
A.~N\'eron.
\newblock Quasi-fonctions et hauteurs sur les vari\'et\'es ab\'eliennes.
\newblock \emph{Ann. of Math.}, 82:249--331, 1965.

\bibitem{Northcott1949}
D.~G. Northcott.
\newblock An inequality in the theory of arithmetic on algebraic varieties.
\newblock \emph{Proc. Cambridge Philos. Soc.}, 45:502--509, 1949.

\bibitem{SGA7}
A.~Grothendieck et al.
\newblock \emph{SGA 7: Groupes de monodromie en g\'eom\'etrie alg\'ebrique}.
\newblock Springer LNM 288/340, 1972--73.

\bibitem{Voisin2003}
C.~Voisin.
\newblock \emph{Hodge Theory and Complex Algebraic Geometry}, vols.\ I--II.
\newblock Cambridge University Press, 2002--2003.

\bibitem{Paper50}
P.~C.-K. Lee.
\newblock Constructive Reverse Mathematics and the Five Great Conjectures: Atlas Survey.
\newblock Paper~50, this series.

\bibitem{Paper59}
P.~C.-K. Lee.
\newblock De Rham Decidability and DPT Completeness.
\newblock Paper~59, this series, 2026.

\bibitem{Paper61}
P.~C.-K. Lee.
\newblock The Lang constant as gate from MP to BISH.
\newblock Paper~61, this series.

\bibitem{Paper45}
P.~C.-K. Lee.
\newblock The Weight-Monodromy Conjecture and LPO: A constructive calibration of spectral sequence degeneration via de-omniscientizing descent.
\newblock Paper~45, this series.

\end{thebibliography}

\end{document}

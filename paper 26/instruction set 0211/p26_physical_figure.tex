% ====================================================================
% FIGURE: From physical question to complexity classification
% For Paper 26 Discussion or Introduction
% All examples calibrate at WLPO (not LPO)
% ====================================================================

\begin{figure}[ht]
\centering
\begin{tikzpicture}[
  >=Stealth,
  box/.style={draw, rounded corners=3pt, minimum width=52mm,
              minimum height=14mm, align=center, font=\small,
              text width=50mm},
  arrow/.style={->, thick, font=\footnotesize},
  annot/.style={font=\scriptsize, text=black!60, align=center},
]

% === Column 1: Physical questions ===
\node[box, fill=green!10] (phys1) at (0, 4.5)
  {\textbf{Physical question}\\[2pt]
   Does this detector signal\\eventually vanish?};

\node[box, fill=green!10] (phys2) at (0, 1.5)
  {\textbf{Physical question}\\[2pt]
   Can I exhibit a singular state\\
   $\Psi \in S_1(H)^{**} \setminus J(S_1(H))$?};

\node[box, fill=green!10] (phys3) at (0, -1.5)
  {\textbf{Physical question}\\[2pt]
   Has this quantum system\\
   fully decohered?};

% === Column 2: Mathematical formalization ===
\node[box, fill=blue!10] (math1) at (7, 4.5)
  {\textbf{Formalization}\\[2pt]
   $v \in \ell^\infty$: is $[v] = [0]$\\
   in $\ell^\infty/c_0$\;?};

\node[box, fill=blue!10] (math2) at (7, 1.5)
  {\textbf{Formalization}\\[2pt]
   Non-reflexivity witness\\
   for $S_1(H)$};

\node[box, fill=blue!10] (math3) at (7, -1.5)
  {\textbf{Formalization}\\[2pt]
   Off-diagonal elements of $\rho$:\\
   does $\|\rho_{ij}\|_{i \neq j} \to 0$\;?};

% === Arrows from physical to formal ===
\draw[arrow] (phys1) -- (math1) node[midway, above, annot] {Paper 2};
\draw[arrow] (phys2) -- (math2) node[midway, above, annot] {Paper 7};
\draw[arrow] (phys3) -- (math3) node[midway, above, annot] {Paper 9};

% === Convergence to WLPO ===
\node[box, fill=red!10, minimum width=45mm, minimum height=16mm,
      font=\normalsize] (wlpo) at (3.5, -4.5)
  {\textbf{WLPO}\\[2pt]
   \small $\forall \alpha\!:\!\mathbb{N}\!\to\!2.\;
   (\forall n.\,\alpha(n)\!=\!0)$\\
   $\lor\;\lnot(\forall n.\,\alpha(n)\!=\!0)$};

\draw[arrow, blue!60!black] (math1.south) -- ++(0,-0.5) -| (wlpo.north);
\draw[arrow, blue!60!black] (math2.south) -- ++(0,-0.3) -| (wlpo.north east);
\draw[arrow, blue!60!black] (math3.south) -- ++(0,-0.5) -| (wlpo.north west);

% === Paper 26's contribution ===
\node[box, fill=orange!10, minimum width=45mm] (p26) at (10.5, -4.5)
  {\textbf{Paper 26}\\[2pt]
   \small $\Pi^0_1$ consistency\\
   reduces to gap membership};

\draw[arrow, orange!70!black, <->] (wlpo) -- node[above, annot]
  {independent\\arithmetic proof} (p26);

% === Annotations ===
\node[annot] at (3.5, 5.4) {Independent physical instances};
\node[annot] at (3.5, -3.3) {all calibrate at the same level};
\node[annot, text width=30mm] at (10.5, -3.0)
  {explains \emph{why}:\\
   gap detection is\\
   $\Pi^0_1$-equivalent};

\end{tikzpicture}
\caption{Multiple independent physical questions (left) formalize as
mathematical problems (right) that all calibrate at $\WLPO$.
Paper~26 provides an arithmetic explanation: gap detection is
$\WLPO$-complete because $\ell^\infty/c_0$ admits a reduction
from $\Pi^0_1$ consistency.  The physical questions are real;
the bidual gap is the mathematical obstruction that explains
their shared undecidability; Paper~26's reduction explains
why this obstruction sits at the $\Pi^0_1$ level of the
arithmetic hierarchy.}
\label{fig:physical-chain}
\end{figure}

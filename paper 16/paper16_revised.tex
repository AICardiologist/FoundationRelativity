\documentclass[11pt,a4paper]{article}

% ====================================================================
% Packages
% ====================================================================
\usepackage[utf8]{inputenc}
\usepackage[T1]{fontenc}
\usepackage{amsmath,amssymb,amsthm}
\usepackage{mathtools}
\usepackage{hyperref}
\usepackage[margin=1in]{geometry}
\usepackage{enumitem}
\usepackage{booktabs}
\usepackage{listings}
\usepackage{xcolor}
\usepackage{cleveref}
\usepackage{natbib}
\usepackage{mdframed}

% ====================================================================
% Theorem environments
% ====================================================================
\theoremstyle{plain}
\newtheorem{theorem}{Theorem}[section]
\newtheorem{lemma}[theorem]{Lemma}
\newtheorem{proposition}[theorem]{Proposition}
\newtheorem{corollary}[theorem]{Corollary}

\theoremstyle{definition}
\newtheorem{definition}[theorem]{Definition}
\newtheorem{remark}[theorem]{Remark}

% ====================================================================
% Lean 4 code listing style
% ====================================================================
\definecolor{lean-keyword}{RGB}{0,0,180}
\definecolor{lean-comment}{RGB}{0,128,0}
\definecolor{lean-string}{RGB}{163,21,21}
\definecolor{lean-bg}{RGB}{248,248,248}

\lstdefinelanguage{lean4}{
  keywords={theorem,lemma,def,class,instance,import,open,variable,
            noncomputable,section,namespace,end,where,let,have,show,
            intro,obtain,use,exact,rw,simp,apply,by,fun,match,if,
            then,else,do,return,axiom,abbrev,private,attribute,
            suffices,change,congr,ext,constructor,rintro,push_neg,
            linarith,absurd,set_option,omit,in,set,cases,structure,
            refine,unfold,rcases,calc,all_goals,first,try,ring,
            positivity,induction,nlinarith,conv_lhs,conv_rhs,
            simp_rw,trans,exact_mod_cast,div_le_div_of_nonneg_right},
  sensitive=true,
  morecomment=[l]{--},
  morecomment=[s]{/-}{-/},
  morestring=[b]",
  morestring=[b]',
}

\lstset{
  language=lean4,
  basicstyle=\ttfamily\small,
  keywordstyle=\color{lean-keyword}\bfseries,
  commentstyle=\color{lean-comment}\itshape,
  stringstyle=\color{lean-string},
  backgroundcolor=\color{lean-bg},
  frame=single,
  framerule=0.5pt,
  breaklines=true,
  breakatwhitespace=true,
  tabsize=2,
  showstringspaces=false,
  numbers=left,
  numberstyle=\tiny\color{gray},
  numbersep=5pt,
  xleftmargin=15pt,
  captionpos=b,
}

% ====================================================================
% Macros
% ====================================================================
\newcommand{\NN}{\mathbb{N}}
\newcommand{\RR}{\mathbb{R}}
\newcommand{\CC}{\mathbb{C}}
\newcommand{\LPO}{\mathrm{LPO}}
\newcommand{\WLPO}{\mathrm{WLPO}}
\newcommand{\BMC}{\mathrm{BMC}}
\newcommand{\DCw}{\mathrm{DC}_\omega}
\newcommand{\BISH}{\mathrm{BISH}}
\newcommand{\SLLN}{\mathrm{SLLN}}
\newcommand{\Lean}{\textsc{Lean~4}}
\newcommand{\Mathlib}{\textsc{Mathlib4}}
\newcommand{\leanok}{\textsf{\small \textcolor{green!70!black}{\checkmark}}}
\newcommand{\ip}[2]{\langle #1, #2 \rangle}

% ====================================================================
% Title
% ====================================================================
\title{%
  \textbf{The Born Rule as a Logical Artefact}\\[6pt]
  {\normalsize Technical Note~16 in the Constructive Calibration Programme}%
}

\author{
  Paul Chun-Kit Lee\thanks{%
    New York University.
    AI-assisted formalization; see \S\ref{sec:lean} for methodology.
    The author is a medical professional, not a domain expert in
    constructive mathematics or mathematical physics; mathematical
    content was developed with extensive AI assistance.} \\
  New York University \\
  \texttt{dr.paul.c.lee@gmail.com}
}

\date{February 2026}

% ====================================================================
\begin{document}
\maketitle

% ====================================================================
\begin{abstract}
We observe that the Born rule splits across the constructive hierarchy in
a manner consistent with the general pattern of the calibration programme.
The single-trial probability $p_i = \|P_i\psi\|^2$, the expectation value,
variance, and Chebyshev bound (weak law of large numbers) are all provable
in Bishop's constructive mathematics ($\BISH$)---unsurprisingly, since they
are finite-dimensional linear algebra and real arithmetic.
The frequentist assertion---that relative frequencies converge almost
surely to the Born probabilities---requires Dependent Choice over $\NN$ ($\DCw$),
a fact well known in constructive probability theory.

Our contribution is not mathematical novelty but programme completeness:
this note fills the $\DCw$ axis of the calibration table (Paper~10)
with physical content from quantum measurement statistics, complementing
the $\LPO$ entries from statistical mechanics (Paper~8), general relativity
(Paper~13), decoherence (Paper~14), and conservation laws (Paper~15).
All $\BISH$ results are formalised in \Lean{} with machine-checked proofs
(564~lines, zero \texttt{sorry}).  The $\DCw$ layer is axiomatised
following established series methodology; a full constructive proof
of the strong law with explicit $\DCw$ tracking remains open.
\end{abstract}

\tableofcontents

% ====================================================================
\section{Introduction}\label{sec:intro}
% ====================================================================

\subsection{Physical Context}

The Born rule is the bridge between quantum geometry and laboratory data.
Given a normalized state $\psi \in \CC^d$ and an observable $A$ with
spectral decomposition $A = \sum_i \lambda_i P_i$, the rule asserts:
the probability of measuring eigenvalue~$\lambda_i$ is
\[
  p_i \;=\; \|P_i \psi\|^2 \;=\; \mathrm{Re}\,\ip{\psi}{P_i\psi}.
\]
Every prediction of quantum mechanics flows from this postulate.

\subsection{The Question}

The Born rule admits two readings with different logical content.
If ``probability $p_i$'' means ``the squared norm of the
projected state''---a single computation on one finite-dimensional
vector---then the Born rule is a statement of finite-dimensional linear
algebra, and its constructive status is immediate.  If it means ``the long-run relative frequency in infinitely
many repeated measurements converges almost surely to $p_i$''---the
frequentist interpretation---then it is a statement about infinite
sequences, requiring countable product spaces and almost-sure convergence.

This note records the logical cost of each reading.  The result is
expected---constructive probabilists have understood the role of
Dependent Choice in the strong law since at least Bishop and
Bridges~\citep{BishopBridges1985}---but its explicit placement in
the calibration table has not previously been formalised.

\subsection{Summary of Results}

\begin{center}
\begin{tabular}{lll}
\toprule
\textbf{Layer} & \textbf{Physical content} & \textbf{Logical cost} \\
\midrule
Single-trial probability & $p_i = \|P_i\psi\|^2 \in [0,1]$ & $\BISH$ \\
Expectation, variance & $\ip{\psi}{A\psi} \in \RR$, $\mathrm{Var} \geq 0$ & $\BISH$ \\
Chebyshev bound (weak law) & $\Pr(|\mathrm{freq} - p| > \varepsilon) \leq 1/(4N\varepsilon^2)$ & $\BISH$ \\
Relative frequency sums & $\sum_i \mathrm{freq}_N(i) = 1$ & $\BISH$ \\
Strong law (SLLN) & $\mathrm{freq}_N \to p$ almost surely & $\DCw$ \\
\bottomrule
\end{tabular}
\end{center}

The weak law row deserves emphasis: the Chebyshev bound gives an
explicit, computable error estimate $\Pr(|\mathrm{freq} - p| > \varepsilon) \leq 1/(4N\varepsilon^2)$
that shrinks as $1/N$.  This is what experimentalists actually use,
and it is fully constructive.

\subsection{Programme Context}

This note extends the calibration table along the $\DCw$ axis,
complementing existing entries:
\begin{itemize}[nosep]
\item Paper~4: approximate spectral membership is $\BISH$, exact membership costs~MP.
\item Paper~6: preparation uncertainty is $\BISH$, measurement uncertainty costs~$\DCw$.
\item Paper~14: finite-time decoherence bounds are $\BISH$, exact collapse costs~$\LPO$.
\item Paper~15: local energy conservation is $\BISH$, global energy existence costs~$\LPO$.
\item \textbf{Paper~16 (this):} finite-sample frequency bounds are $\BISH$, exact
      frequentist convergence costs~$\DCw$.
\end{itemize}
Papers~6 and~16 together populate the $\DCw$ axis with two independent
physical instances (measurement uncertainty and measurement statistics),
paralleling the multiple $\LPO$ entries on the omniscience spine.

% ====================================================================
\section{Background}\label{sec:background}
% ====================================================================

\subsection{Constructive Foundations}

We work within Bishop's constructive mathematics ($\BISH$): intuitionistic logic
plus dependent choice restricted to finite operations.  $\BISH$ admits all of
classical analysis that can be carried out without appeal to excluded middle or
uncountable choice.

\begin{definition}[Dependent Choice over~$\NN$]
$\DCw$ asserts: for any type~$\alpha$, any total relation $R : \alpha \to \alpha \to \mathrm{Prop}$,
and any starting point $a_0 : \alpha$, if $\forall a, \exists b, R(a,b)$, then there
exists a function $f : \NN \to \alpha$ with $f(0) = a_0$ and $R(f(n), f(n+1))$ for all~$n$.
\end{definition}

$\DCw$ is strictly weaker than the full axiom of choice but strictly stronger than
countable choice.  It is the standard choice principle for constructing infinite
sequences from local extension data---precisely what the strong law of large numbers
requires.

\subsection{Certification Methodology (Paper~10)}\label{sec:methodology}

Our diagnostic for calibrating a physical theorem $T$ against the constructive
hierarchy is:
\begin{enumerate}[nosep]
\item Formalise $T$ in \Lean{} with \Mathlib{}.
\item Run \texttt{\#print axioms T}.
\item If the axiom closure contains only \texttt{propext}, \texttt{Classical.choice},
      \texttt{Quot.sound} (Mathlib infrastructure), classify $T$ as $\BISH$.
\item If the axiom closure additionally contains a custom axiom such as
      \texttt{dc\_omega\_holds}, classify $T$ at the corresponding level ($\DCw$).
\end{enumerate}

The appearance of \texttt{Classical.choice} in $\BISH$ theorems is an artefact of
Mathlib's type-class infrastructure (the \texttt{Fintype} instance for \texttt{Fin~d},
the field structure on~$\RR$), not a mathematical use of excluded middle.
This is the Paper~10 methodology, validated across all papers in the series.

% ====================================================================
\section{Setup: Finite-Dimensional Quantum Measurement}\label{sec:setup}
% ====================================================================

We work in finite dimensions throughout.  Fix:
\begin{itemize}[nosep]
\item $H = \CC^d$ (finite-dimensional Hilbert space, $d \geq 1$).
\item $\psi \in H$ with $\|\psi\|^2 = 1$ (normalized state).
\item $A : H \to H$ a Hermitian operator (observable).
\item Spectral decomposition: $A = \sum_i \lambda_i P_i$ where each $P_i$ is an
      orthogonal projection ($P_i^2 = P_i$, $P_i^\dagger = P_i$) and
      $\sum_i P_i = I$.
\end{itemize}

\begin{definition}[Complex inner product]\label{def:cdot}
For $\psi, \varphi : \mathrm{Fin}\,d \to \CC$, define
\[
  \ip{\psi}{\varphi} \;:=\; \sum_{i=0}^{d-1} \overline{\psi_i}\,\varphi_i.
\]
\end{definition}

\begin{definition}[Born probability]\label{def:born}
$p_i := \mathrm{Re}\,\ip{\psi}{P_i\psi}$.
For a projection $P_i$ with $P_i^2 = P_i$ and $P_i^\dagger = P_i$, this equals $\|P_i\psi\|^2$.
\end{definition}

\begin{definition}[Spectral decomposition]\label{def:spectral}
In \Lean{}:
\begin{lstlisting}[caption={SpectralDecomp structure (Defs.lean)}]
structure SpectralDecomp (d : N) where
  eigenvalues  : Fin d -> R
  projections  : Fin d -> Matrix (Fin d) (Fin d) C
  is_projection : forall i, projections i * projections i = projections i
  is_hermitian  : forall i, (projections i).conjTranspose = projections i
  is_orthogonal : forall i j, i != j -> projections i * projections j = 0
  is_complete   : sum i, projections i = 1
\end{lstlisting}
\end{definition}

% ====================================================================
\section{BISH Content: Theorems 1--5}\label{sec:bish}
% ====================================================================

\subsection{Theorem 1: Born Probability Distribution}\label{sec:born}

\begin{theorem}[Born probability is $\BISH$]\label{thm:born}
For any normalized state $\psi$ and spectral decomposition $\{P_i\}$:
\begin{enumerate}[nosep]
\item $0 \leq p_i$ for each~$i$ \hfill (\texttt{born\_prob\_nonneg})
\item $\sum_i p_i = 1$ \hfill (\texttt{born\_prob\_sum\_one})
\end{enumerate}
\end{theorem}

\begin{proof}
\textbf{Non-negativity.}  Since $P_i^2 = P_i$ and $P_i^\dagger = P_i$:
\[
  \ip{\psi}{P_i\psi} = \ip{\psi}{P_i(P_i\psi)} = \ip{P_i\psi}{P_i\psi} = \|P_i\psi\|^2 \geq 0.
\]
The first equality uses $P_i^2 = P_i$.  The second uses $P_i^\dagger = P_i$
(the Hermitian swap lemma: $\ip{u}{Pv} = \ip{Pu}{v}$ for $P^\dagger = P$).
The inequality is $\BISH$: a sum of $|\cdot|^2$ terms.

\textbf{Sum to one.}  By linearity of the inner product in the second argument:
\[
  \sum_i \ip{\psi}{P_i\psi} = \ip{\psi}{\Bigl(\sum_i P_i\Bigr)\psi}
  = \ip{\psi}{I\psi} = \ip{\psi}{\psi} = \|\psi\|^2 = 1.
\]
Every step is a finite sum or matrix operation.  $\BISH$.
\end{proof}

\begin{lstlisting}[caption={Born probability proofs (BornProbability.lean, excerpt)}]
theorem born_prob_nonneg {d : N} (psi : Fin d -> C)
    (spec : SpectralDecomp d) (i : Fin d) :
    0 <= bornProb psi spec i := by
  unfold bornProb
  suffices h : (cdot psi ((spec.projections i).mulVec psi)).re =
    cnorm_sq ((spec.projections i).mulVec psi) from
    h |> cnorm_sq_nonneg _
  unfold cnorm_sq; congr 1
  set Pi := spec.projections i
  have hP2 : Pi.mulVec (Pi.mulVec psi) = Pi.mulVec psi := by
    rw [Matrix.mulVec_mulVec, spec.is_projection i]
  conv_lhs => rw [show Pi.mulVec psi =
    Pi.mulVec (Pi.mulVec psi) from hP2.symm]
  exact hermitian_cdot_swap psi (Pi.mulVec psi) Pi (spec.is_hermitian i)
\end{lstlisting}

\subsection{Theorem 2: Expectation Value is Real}\label{sec:expectation}

\begin{theorem}[Expectation is $\BISH$]\label{thm:expectation}
For Hermitian $A$ ($A^\dagger = A$), the expectation value
$\ip{\psi}{A\psi}$ is real: $\mathrm{Im}\,\ip{\psi}{A\psi} = 0$.
\end{theorem}

\begin{proof}
By the Hermitian swap lemma, $\ip{\psi}{A\psi} = \ip{A\psi}{\psi}$.
By conjugate symmetry of the inner product,
$\ip{A\psi}{\psi} = \overline{\ip{\psi}{A\psi}}$.
So $z = \bar{z}$, which implies $\mathrm{Im}(z) = 0$.
All operations are finite sums.  $\BISH$.
\end{proof}

\begin{lstlisting}[caption={Expectation reality (Expectation.lean)}]
theorem expectation_real {d : N} (psi : Fin d -> C)
    (A : Matrix (Fin d) (Fin d) C) (hA : A.conjTranspose = A) :
    (expectationValue psi A).im = 0 := by
  unfold expectationValue
  have h1 : cdot psi (A.mulVec psi) = cdot (A.mulVec psi) psi :=
    hermitian_cdot_swap psi psi A hA
  have h2 : cdot (A.mulVec psi) psi =
    starRingEnd C (cdot psi (A.mulVec psi)) :=
    cdot_hermitian (A.mulVec psi) psi
  have h3 := h1.trans h2  -- z = conj(z)
  have h4 := Complex.ext_iff.mp h3
  simp only [Complex.conj_re, Complex.conj_im] at h4
  linarith [h4.2]
\end{lstlisting}

\subsection{Theorem 3: Variance is Non-negative}\label{sec:variance}

\begin{theorem}[Variance is $\BISH$]\label{thm:variance}
For any Hermitian $A$ and complex~$\mu$,
$\|(A - \mu I)\psi\|^2 \geq 0$.
\end{theorem}

\begin{proof}
Immediate: the squared norm of any vector is non-negative.
$\BISH$.
\end{proof}

\begin{lstlisting}[caption={Variance non-negativity (Variance.lean)}]
theorem variance_nonneg {d : N} (psi : Fin d -> C)
    (A : Matrix (Fin d) (Fin d) C) (mu : C) :
    0 <= cnorm_sq ((A - mu * (1 : Matrix (Fin d) (Fin d) C)).mulVec psi) :=
  cnorm_sq_nonneg _
\end{lstlisting}

\subsection{Theorem 4: Relative Frequency Bounds}\label{sec:relfreq}

\begin{theorem}[Relative frequency is $\BISH$]\label{thm:relfreq}
For $N$ measurement outcomes $x_1, \ldots, x_N \in \{1, \ldots, d\}$,
the relative frequency $\mathrm{freq}_N(i) = \#\{j : x_j = i\}/N$ satisfies:
\begin{enumerate}[nosep]
\item $0 \leq \mathrm{freq}_N(i)$ \hfill (\texttt{relative\_freq\_nonneg})
\item $\mathrm{freq}_N(i) \leq 1$ \hfill (\texttt{relative\_freq\_le\_one})
\item $\sum_i \mathrm{freq}_N(i) = 1$ \hfill (\texttt{relative\_freq\_sum})
\end{enumerate}
\end{theorem}

\begin{proof}
Non-negativity: the count is $\geq 0$ and $N > 0$.
Upper bound: the count $\leq N$.
Sum to one: the counts partition $\{1, \ldots, N\}$, so $\sum \mathrm{count}_i = N$.
Dividing by $N$ gives~1.  All operations are finite counting.  $\BISH$.
\end{proof}

\subsection{Theorem 5: Chebyshev Bound (Weak Law)}\label{sec:chebyshev}

\begin{theorem}[Chebyshev bound is $\BISH$]\label{thm:chebyshev}
For a Bernoulli process with parameter $p \in [0,1]$, $N$ trials,
and tolerance $\varepsilon > 0$:
\[
  \frac{p(1-p)}{N\varepsilon^2} \;\leq\; \frac{1}{4N\varepsilon^2}.
\]
\end{theorem}

\begin{proof}
\textbf{Step 1.} The Bernoulli variance bound: $p(1-p) \leq 1/4$
for $p \in [0,1]$.  Proof: $(1/2 - p)^2 \geq 0$ implies
$1/4 - p + p^2 \geq 0$, so $p(1-p) \leq 1/4$.

\textbf{Step 2.} Since $N > 0$ and $\varepsilon > 0$, we have
$N\varepsilon^2 > 0$.  Dividing the inequality $p(1-p) \leq 1/4$
by $N\varepsilon^2$ preserves the direction:
\[
  \frac{p(1-p)}{N\varepsilon^2} \leq \frac{1/4}{N\varepsilon^2}
  = \frac{1}{4N\varepsilon^2}.
\]
All operations are finite real arithmetic.  $\BISH$.
\end{proof}

\begin{lstlisting}[caption={Chebyshev bound (WeakLaw.lean)}]
theorem bernoulli_variance_bound (p : R) (_hp : 0 <= p) (_hp1 : p <= 1) :
    p * (1 - p) <= 1 / 4 := by
  nlinarith [sq_nonneg (1 / 2 - p)]

theorem chebyshev_bernoulli_bound (p : R) (hp : 0 <= p) (hp1 : p <= 1)
    (N : Nat) (hN : 0 < N) (eps : R) (heps : 0 < eps) :
    p * (1 - p) / (N * eps ^ 2) <= 1 / (4 * N * eps ^ 2) := by
  have hvar := bernoulli_variance_bound p hp hp1
  calc p * (1 - p) / (N * eps ^ 2)
      <= (1 / 4) / (N * eps ^ 2) := by
        apply div_le_div_of_nonneg_right hvar (by positivity)
    _ = 1 / (4 * N * eps ^ 2) := by ring
\end{lstlisting}

\textbf{Physical interpretation.}  After $N = 10{,}000$ measurements,
the probability that the observed frequency deviates from the Born
probability by more than $0.01$ is at most $1/(4 \times 10^4 \times 10^{-4}) = 0.25$.
After $N = 10^6$ measurements with the same tolerance, the bound drops
to $0.0025$.  The bound is explicit, computable, and shrinks as $1/N$.
No physicist needs more than this.

% ====================================================================
\section{DC Content: Strong Law of Large Numbers}\label{sec:dc}
% ====================================================================

\subsection{Theorem 6: The Strong Law Requires $\DCw$}\label{sec:slln}

\begin{theorem}[Strong law requires $\DCw$]\label{thm:slln}
The assertion ``for almost every infinite sequence of independent Born-rule
measurements, $\mathrm{freq}_N(\lambda_i) \to p_i$ as $N \to \infty$''
requires Dependent Choice over~$\NN$.
\end{theorem}

The standard proof of the strong law uses $\DCw$ at three points:

\begin{enumerate}
\item \textbf{Product space construction.}  The countable product probability
space $(\Omega, \mathcal{F}, P) = \prod_n \mathrm{Bernoulli}(p)$ requires
DC at each extension step: given the finite product up to stage~$n$,
extend consistently to stage $n+1$.

\item \textbf{Borel--Cantelli.}  The lemma $\sum P(E_k) < \infty \implies
P(\limsup E_k) = 0$ manipulates the events $E_k = \{|\text{freq}_k - p| > \varepsilon\}$
through a countable intersection of countable unions, requiring DC to
select witnesses at each stage.

\item \textbf{Almost-sure convergence.}  Extracting convergent behaviour
from $\omega \in \Omega$ --- proving the Cauchy criterion for a specific
outcome sequence --- requires dependent choices along the sequence.
\end{enumerate}

In the formalisation, $\DCw$ is introduced as an axiom and the SLLN is
axiomatized as a consequence:

\begin{lstlisting}[caption={DC axiom and SLLN (DCAxiom.lean + StrongLaw.lean)}]
def DC_omega : Prop :=
  forall (alpha : Type) (R : alpha -> alpha -> Prop) (a0 : alpha),
    (forall a, exists b, R a b) ->
    exists f : Nat -> alpha, f 0 = a0 /\ forall n, R (f n) (f (n + 1))

axiom dc_omega_holds : DC_omega
axiom slln_of_dc : DC_omega -> SLLN

theorem frequentist_convergence : SLLN :=
  slln_of_dc dc_omega_holds
\end{lstlisting}

% ====================================================================
\section{Pattern Consistency}\label{sec:sign}
% ====================================================================

The weak-law/strong-law split instantiates the programme's recurring pattern:
finite approximations are $\BISH$, completed limits have measurable cost.
The specific mechanism differs from the $\LPO$ entries.
In Papers~8 and~15, bounded monotone convergence drives the cost; the
partial sums are non-negative and increasing, and asserting their limit
exists is equivalent to $\LPO$.  Here, the cost arises from a different
source: the product-space construction and almost-sure convergence machinery
of the strong law, which require $\DCw$ for the sequential extension of
probability measures.

The pattern is uniform in structure (finite = $\BISH$, infinite = non-$\BISH$)
but heterogeneous in mechanism ($\BMC$ for deterministic limits, $\DCw$ for
probabilistic limits).  This heterogeneity is itself informative: it shows
that the calibration table's axes are genuinely independent, not artefacts
of a single underlying phenomenon.

% ====================================================================
\section{Programme Integration}\label{sec:domains}
% ====================================================================

Paper~16 extends the calibration table along the $\DCw$ axis:

\begin{center}
\begin{tabular}{llll}
\toprule
\textbf{Domain} & \textbf{Paper} & \textbf{BISH Content} & \textbf{Non-BISH Content} \\
\midrule
Stat.\ Mech. & P8 & Finite-volume free energy & $f_\infty$ exists ($\LPO$) \\
Gen.\ Rel. & P13 & Radial coordinate bounds & $r \to 0$ exactly ($\LPO$) \\
Quantum Meas. & P14 & Finite-time decoherence & Exact collapse ($\LPO$) \\
Conservation & P15 & Local conservation (Noether) & Global energy ($\LPO$) \\
\textbf{Born Rule} & \textbf{P16} & \textbf{Probability, weak law} & \textbf{SLLN ($\DCw$)} \\
\bottomrule
\end{tabular}
\end{center}

Paper~16 is the first entry in the series where the non-constructive content
lands at $\DCw$ rather than $\LPO$.  This is expected: the strong law of
large numbers is a convergence theorem for \emph{random} sequences (requiring
product measure), not for \emph{deterministic} bounded monotone sequences
(which connect to $\LPO$ via $\BMC$).

We note that the Born rule is not a new physical domain---it is quantum
mechanics, already covered by Papers~4, 6, 11, and~14.  What is new is
the \emph{aspect}: measurement statistics rather than spectral theory,
entanglement, or decoherence.  The $\DCw$ cost arises from the
probabilistic infrastructure (product spaces, almost-sure convergence),
not from quantum structure per se.

% ====================================================================
\section{Lean Formalisation}\label{sec:lean}
% ====================================================================

\subsection{Module Structure}

The formalisation comprises 564~lines across 9~files:

\begin{center}
\begin{tabular}{lrl}
\toprule
\textbf{File} & \textbf{Lines} & \textbf{Role} \\
\midrule
\texttt{Defs.lean}            &  86 & Core definitions \\
\texttt{BornProbability.lean} & 127 & Theorems 1: Born probability (BISH) \\
\texttt{Expectation.lean}     &  37 & Theorem 2: Expectation reality (BISH) \\
\texttt{Variance.lean}        &  29 & Theorem 3: Variance non-negativity (BISH) \\
\texttt{RelativeFreq.lean}    &  67 & Theorem 4: Frequency bounds (BISH) \\
\texttt{WeakLaw.lean}         &  47 & Theorem 5: Chebyshev bound (BISH) \\
\texttt{DCAxiom.lean}         &  31 & $\DCw$ definition + axiom \\
\texttt{StrongLaw.lean}       &  51 & Theorem 6: SLLN requires $\DCw$ \\
\texttt{Main.lean}            &  89 & Assembly + axiom audit \\
\bottomrule
\end{tabular}
\end{center}

\subsection{Design Decisions}

\begin{enumerate}
\item \textbf{Custom inner product.}  We define \texttt{cdot} as
      $\sum_i \overline{\psi_i}\,\varphi_i$ on \texttt{Fin~d $\to$ $\CC$},
      bypassing \Mathlib{}'s \texttt{InnerProductSpace} infrastructure.
      This avoids the \texttt{Classical.choice} contamination from
      \texttt{EuclideanSpace} instances while keeping all matrix operations
      available.

\item \textbf{Spectral decomposition as structure.}  The \texttt{SpectralDecomp}
      structure bundles eigenvalues, projections, and their algebraic properties
      (idempotence, Hermiticity, orthogonality, completeness).  The spectral
      theorem itself is not proved---it is a hypothesis of the Born rule.

\item \textbf{$\DCw$ as axiom.}  Following the \texttt{bmc\_of\_lpo} pattern
      from Papers~14--15, we introduce \texttt{dc\_omega\_holds : DC\_omega}
      as an axiom and \texttt{slln\_of\_dc : DC\_omega $\to$ SLLN} as a
      second axiom.  The \texttt{\#print axioms} audit then cleanly separates
      $\BISH$ from $\DCw$ theorems.

\item \textbf{SLLN axiomatized.}  The full proof of the strong law of large
      numbers is a substantial result in measure theory.  We axiomatize it
      and verify that its axiom closure contains exactly
      \texttt{dc\_omega\_holds} + \texttt{slln\_of\_dc}, confirming the
      $\DCw$ dependency.
\end{enumerate}

\subsection{Axiom Certificate}\label{sec:axioms}

The build output confirms clean separation:

\begin{mdframed}[backgroundcolor=lean-bg]
\begin{verbatim}
-- BISH theorems:
'born_prob_nonneg' depends on: [propext, Classical.choice, Quot.sound]
'born_prob_sum_one' depends on: [propext, Classical.choice, Quot.sound]
'expectation_real'  depends on: [propext, Classical.choice, Quot.sound]
'variance_nonneg'   depends on: [propext, Classical.choice, Quot.sound]
'bernoulli_variance_bound' depends on: [propext, Classical.choice, Quot.sound]
'chebyshev_bernoulli_bound' depends on: [propext, Classical.choice, Quot.sound]
'relative_freq_nonneg' depends on: [propext, Classical.choice, Quot.sound]
'relative_freq_le_one' depends on: [propext, Classical.choice, Quot.sound]
'relative_freq_sum'    depends on: [propext, Classical.choice, Quot.sound]
'cnorm_sq_nonneg'      depends on: [propext, Classical.choice, Quot.sound]

-- DC theorem:
'frequentist_convergence' depends on:
  [propext, Classical.choice, Quot.sound,
   Papers.P16.dc_omega_holds, Papers.P16.slln_of_dc]
\end{verbatim}
\end{mdframed}

All $\BISH$ theorems have axiom closure $\{\texttt{propext}, \texttt{Classical.choice},
\texttt{Quot.sound}\}$---standard \Mathlib{} infrastructure with no custom axioms.
The $\DCw$ theorem additionally contains \texttt{dc\_omega\_holds} and
\texttt{slln\_of\_dc}.

\subsection{AI-Assisted Methodology}\label{sec:ai}

The formalisation was developed with AI assistance (Claude, Anthropic).
The author verified all mathematical content and the AI generated \Lean{} proof
scripts.  All proofs were machine-checked by the \Lean{} kernel---no trust is
placed in the AI's correctness, only in the formal verification.

% ====================================================================
\section{Discussion}\label{sec:discussion}
% ====================================================================

\subsection{Quantum Geometry vs.\ Quantum Statistics}

The Born rule separates into two physical contents:
\begin{itemize}[nosep]
\item \textbf{Quantum geometry ($\BISH$):}  the state $\psi$ determines a probability
      distribution over outcomes.  This is a fact about one vector in one
      finite-dimensional space.  No measurements needed.
\item \textbf{Quantum statistics ($\DCw$):}  repeated measurements of identically
      prepared systems yield frequencies converging to the geometric distribution.
      This requires constructing infinite product spaces and proving convergence.
\end{itemize}

The experimentalist who runs $10{,}000$ measurements and checks that frequencies
match Born probabilities within error bars is working in $\BISH$.  The textbook
that asserts ``in the limit of infinitely many measurements, the frequency equals
the probability'' is asserting $\DCw$.

\subsection{Comparison with Paper~14}

Paper~14 calibrated the quantum measurement problem (decoherence) against $\LPO$.
Paper~16 calibrates the Born rule against $\DCw$.  The two results are complementary:
\begin{itemize}[nosep]
\item Paper~14: the \emph{mechanism} of measurement (decoherence) $\to$ $\LPO$.
\item Paper~16: the \emph{statistics} of measurement (Born probabilities) $\to$ $\DCw$.
\end{itemize}
Different aspects of quantum measurement land at different levels of the
constructive hierarchy.

\subsection{Limitations and Open Problems}

\begin{enumerate}
\item \textbf{No new equivalence.}  Unlike Papers~2, 8, and~13, which prove
      biconditional equivalences ($T \equiv P$ over $\BISH$), this note
      establishes only that $\DCw$ is sufficient for the SLLN.  The reverse
      direction ($\SLLN \implies \DCw$) is not formalised.  A full calibration
      would require proving that the strong law \emph{cannot} be obtained from
      weaker principles (e.g., countable choice without dependence).

\item \textbf{SLLN axiomatised.}  The strong law is introduced as an axiom
      (\texttt{slln\_of\_dc}), not derived from measure theory.  A complete
      formalisation of the constructive SLLN with explicit $\DCw$ tracking
      in Lean~4 would be a substantial independent contribution.

\item \textbf{BISH content is unsurprising.}  The constructive status of
      Theorems~1--5 follows immediately from their being finite-dimensional
      linear algebra and real arithmetic.  The formalisation verifies correctness
      but does not reveal hidden non-constructive dependencies---there are none
      to reveal.

\item \textbf{Classical.choice artefact.}  Per the Paper~10 methodology (\S\ref{sec:methodology}),
      \texttt{Classical.choice} in the $\BISH$ theorems is an artefact of
      \Mathlib{}'s typeclass infrastructure, not a mathematical use of excluded middle.
\end{enumerate}

% ====================================================================
\section{Conclusion}\label{sec:conclusion}
% ====================================================================

The Born rule decomposes along the constructive hierarchy: single-trial
probability, expectation, variance, and finite-sample error bounds are
$\BISH$; exact frequentist convergence requires $\DCw$.  Neither half
of this observation is mathematically surprising.  The contribution is
to the calibration programme: the $\DCw$ axis now has explicit physical
content from quantum measurement statistics, and the pattern---physics
as practised is constructive, physics as idealised has measurable logical
cost---extends to yet another aspect of quantum mechanics.

The main open problem is sharpness: is $\DCw$ \emph{necessary} for the
strong law, or does a weaker principle suffice?  A constructive proof of
SLLN~$\equiv$~$\DCw$ over $\BISH$ would convert this note's upper bound
into an exact calibration.

\vspace{1em}
\begin{mdframed}
\textbf{Reproducibility.}  All source code, compiled PDF, and build
instructions are archived at
\href{https://doi.org/10.5281/zenodo.18575377}{Zenodo DOI:~10.5281/zenodo.18575377}.
\Lean{} version: \texttt{leanprover/lean4:v4.28.0-rc1}.
\Mathlib{} commit: \texttt{7091f0f6}.
Build: \texttt{lake exe cache get \&\& lake build} (1662~jobs, 0~errors, 0~\texttt{sorry}).
\end{mdframed}

% ====================================================================
% Bibliography
% ====================================================================
\bibliographystyle{plainnat}

\begin{thebibliography}{25}

\bibitem[Bishop and Bridges(1985)]{BishopBridges1985}
E.~Bishop and D.~Bridges.
\newblock \emph{Constructive Analysis}.
\newblock Springer, 1985.

\bibitem[Born(1926)]{Born1926}
M.~Born.
\newblock Zur Quantenmechanik der Sto{\ss}vorg{\"a}nge.
\newblock \emph{Zeitschrift f{\"u}r Physik}, 37:863--867, 1926.

\bibitem[Bridges(1979)]{Bridges1979}
D.~Bridges.
\newblock \emph{Constructive Functional Analysis}.
\newblock Pitman, 1979.

\bibitem[Bridges and V{\^i}{\c{t}}{\u{a}}(2006)]{BridgesVita2006}
D.~Bridges and L.~V{\^i}{\c{t}}{\u{a}}.
\newblock \emph{Techniques of Constructive Analysis}.
\newblock Springer, 2006.

\bibitem[de~Moura et~al.(2021)]{deMoura2021}
L.~de~Moura, S.~Ullrich, et~al.
\newblock The {L}ean 4 theorem prover and programming language.
\newblock In \emph{CADE-28}, LNAI 12699, 2021.

\bibitem[Dirac(1930)]{Dirac1930}
P.~A.~M.~Dirac.
\newblock \emph{The Principles of Quantum Mechanics}.
\newblock Oxford University Press, 1930.

\bibitem[Ishihara(2006)]{Ishihara2006}
H.~Ishihara.
\newblock Reverse mathematics in {B}ishop's constructive mathematics.
\newblock \emph{Philosophia Mathematica}, 14(2):195--218, 2006.

\bibitem[Lee(2025a)]{Lee2025a}
P.~C.~K.~Lee.
\newblock Paper~8: The {I}sing model and {LPO} dispensability.
\newblock Zenodo, 2025.
\newblock \href{https://doi.org/10.5281/zenodo.15043592}{10.5281/zenodo.15043592}.

\bibitem[Lee(2025b)]{Lee2025b}
P.~C.~K.~Lee.
\newblock Paper~10: {CRM} methodology for {M}athlib formalizations.
\newblock Zenodo, 2025.

\bibitem[Lee(2026a)]{Lee2026a}
P.~C.~K.~Lee.
\newblock Paper~14: Quantum decoherence and {LPO}.
\newblock Zenodo, 2026.
\newblock \href{https://doi.org/10.5281/zenodo.18569068}{10.5281/zenodo.18569068}.

\bibitem[Lee(2026b)]{Lee2026b}
P.~C.~K.~Lee.
\newblock Paper~15: {N}oether's theorem and global conservation laws.
\newblock Zenodo, 2026.
\newblock \href{https://doi.org/10.5281/zenodo.18572494}{10.5281/zenodo.18572494}.

\bibitem[Lo{\`e}ve(1977)]{Loeve1977}
M.~Lo{\`e}ve.
\newblock \emph{Probability Theory I}.
\newblock Springer, 4th edition, 1977.

\bibitem[Mathlib(2024)]{Mathlib2024}
The Mathlib Community.
\newblock \texttt{mathlib4}: The math library for {L}ean 4.
\newblock \url{https://github.com/leanprover-community/mathlib4}, 2024.

\bibitem[Nielsen and Chuang(2010)]{NielsenChuang2010}
M.~A.~Nielsen and I.~L.~Chuang.
\newblock \emph{Quantum Computation and Quantum Information}.
\newblock Cambridge University Press, 10th anniversary edition, 2010.

\bibitem[von Neumann(1932)]{vonNeumann1932}
J.~von Neumann.
\newblock \emph{Mathematische Grundlagen der Quantenmechanik}.
\newblock Springer, 1932.

\end{thebibliography}

\end{document}


\documentclass[11pt,a4paper]{article}

% ====================================================================
% Packages
% ====================================================================
\usepackage[utf8]{inputenc}
\usepackage[T1]{fontenc}
\usepackage{amsmath,amssymb,amsthm}
\usepackage{mathtools}
\usepackage{hyperref}
\usepackage[margin=1in]{geometry}
\usepackage{enumitem}
\usepackage{booktabs}
\usepackage{listings}
\usepackage{xcolor}
\usepackage{cleveref}
\usepackage{natbib}
\usepackage{mdframed}

% ====================================================================
% Theorem environments
% ====================================================================
\theoremstyle{plain}
\newtheorem{theorem}{Theorem}[section]
\newtheorem{lemma}[theorem]{Lemma}
\newtheorem{proposition}[theorem]{Proposition}
\newtheorem{corollary}[theorem]{Corollary}

\theoremstyle{definition}
\newtheorem{definition}[theorem]{Definition}
\newtheorem{remark}[theorem]{Remark}

% ====================================================================
% Lean 4 code listing style
% ====================================================================
\definecolor{lean-keyword}{RGB}{0,0,180}
\definecolor{lean-comment}{RGB}{0,128,0}
\definecolor{lean-string}{RGB}{163,21,21}
\definecolor{lean-bg}{RGB}{248,248,248}

\lstdefinelanguage{lean4}{
  keywords={theorem,lemma,def,class,instance,import,open,variable,
            noncomputable,section,namespace,end,where,let,have,show,
            intro,obtain,use,exact,rw,simp,apply,by,fun,match,if,
            then,else,do,return,axiom,abbrev,private,attribute,
            suffices,change,congr,ext,constructor,rintro,push_neg,
            linarith,absurd,set_option,omit,in,set,cases},
  sensitive=true,
  morecomment=[l]{--},
  morecomment=[s]{/-}{-/},
  morestring=[b]",
  morestring=[b]',
}

\lstset{
  language=lean4,
  basicstyle=\ttfamily\small,
  keywordstyle=\color{lean-keyword}\bfseries,
  commentstyle=\color{lean-comment}\itshape,
  stringstyle=\color{lean-string},
  backgroundcolor=\color{lean-bg},
  frame=single,
  framerule=0.5pt,
  breaklines=true,
  breakatwhitespace=true,
  tabsize=2,
  showstringspaces=false,
  numbers=left,
  numberstyle=\tiny\color{gray},
  numbersep=5pt,
  xleftmargin=15pt,
  captionpos=b,
}

% ====================================================================
% Macros
% ====================================================================
\newcommand{\NN}{\mathbb{N}}
\newcommand{\RR}{\mathbb{R}}
\newcommand{\ZZ}{\mathbb{Z}}
\newcommand{\QQ}{\mathbb{Q}}
\newcommand{\LPO}{\mathrm{LPO}}
\newcommand{\WLPO}{\mathrm{WLPO}}
\newcommand{\LLPO}{\mathrm{LLPO}}
\newcommand{\BMC}{\mathrm{BMC}}
\newcommand{\BISH}{\mathrm{BISH}}
\newcommand{\eigenP}{\lambda_+}
\newcommand{\eigenM}{\lambda_-}
\newcommand{\fN}{f_N}
\newcommand{\finf}{f_\infty}
\newcommand{\ZN}{Z_N}
\newcommand{\gfun}{g}
\newcommand{\Lean}{\textsc{Lean~4}}
\newcommand{\Mathlib}{\textsc{Mathlib4}}
\newcommand{\leanok}{\textsf{\small \textcolor{green!70!black}{\checkmark}}}
\newcommand{\leanpartial}{\textsf{\small \textcolor{orange!80!black}{(partial)}}}

% ====================================================================
% Title
% ====================================================================
\title{%
  \textbf{The Logical Cost of the Thermodynamic Limit:}\\[6pt]
  LPO-Equivalence and BISH-Dispensability\\
  for the 1D Ising Free Energy\\[6pt]
  {\normalsize A Lean~4 Formalization}%
}

\author{
  Paul Chun-Kit Lee\thanks{%
    New York University.
    AI-assisted formalization; see \S\ref{sec:ai} for methodology.} \\
  New York University \\
  \texttt{dr.paul.c.lee@gmail.com}
}

\date{February 2026}

% ====================================================================
\begin{document}
\maketitle

% ====================================================================
\begin{abstract}
We prove two complementary results about the thermodynamic limit
of the one-dimensional Ising model, formalized in \Lean{}.
(A)~The finite-size error bound
$|\fN(\beta) - \finf(\beta)| \leq \frac{1}{N}\tanh(\beta)^N$
is provable in Bishop-style constructive mathematics ($\BISH$)
without any omniscience principle. A constructive witness $N_0$
for any prescribed accuracy $\varepsilon > 0$ is explicitly
computed. (B)~The existence of the thermodynamic limit as a
completed real number is equivalent over $\BISH$ to the Limited
Principle of Omniscience ($\LPO$), via the known equivalence between
$\LPO$ and bounded monotone convergence instantiated through the
Ising free energy function. Together, these results establish that
the $\LPO$ cost of the thermodynamic limit is genuine and
dispensable: the idealization costs exactly $\LPO$, but the empirical
content requires no omniscience. The combined formalization
comprises 1374 lines of \Lean{} across 18 modules with zero
sorries.
\end{abstract}

\tableofcontents

% ====================================================================
\section{Introduction}\label{sec:intro}
% ====================================================================

The thermodynamic limit is the foundational idealization of
equilibrium statistical mechanics. For a lattice system
$\Lambda \subset \ZZ^d$ with Hamiltonian $H_\Lambda$, the
free energy density $f_\Lambda(\beta) = -\frac{1}{|\Lambda|}
\log Z_\Lambda(\beta)$ is defined for each finite volume. The
thermodynamic limit asserts that $\finf(\beta) =
\lim_{|\Lambda| \to \infty} f_\Lambda(\beta)$ exists. Classically,
this existence follows from subadditivity and the monotone
convergence theorem: a bounded, monotone sequence of real numbers
converges.

From a constructive standpoint, the monotone convergence theorem is
not available in Bishop-style constructive mathematics ($\BISH$). It
is equivalent over $\BISH$ to the Limited Principle of Omniscience
($\LPO$), which asserts that for any binary sequence $\alpha :
\NN \to \{0,1\}$, either $\alpha(n) = 0$ for all $n$ or
there exists $n_0$ with $\alpha(n_0) = 1$. This equivalence was
established by Bridges and V\^{\i}\c{t}\u{a} \cite{BV06} as part of the systematic
classification of constructive reverse mathematics initiated by
Ishihara \cite{Ish06} and, independently, Veldman \cite{Vel05}.

The question we address is: what is the exact logical cost of the
thermodynamic limit, and is this cost essential for the physics?

We answer both questions completely for the one-dimensional Ising
model with nearest-neighbour interactions. Our results are as follows.

Part~A establishes \emph{dispensability}. For the 1D Ising chain with
uniform coupling $J$ and inverse temperature $\beta$, we prove
explicit error bounds $|\fN(\beta) - \finf(\beta)| \leq
\frac{1}{N}\tanh(\beta)^N$ with a constructive witness $N_0$ for
any prescribed accuracy $\varepsilon > 0$. The proof is entirely
$\BISH$-valid: no omniscience principle is required. The finite-system
prediction approximates the infinite-volume answer with computable
error, and monotone convergence is bypassed via the closed-form
transfer-matrix solution.

Part~B establishes \emph{calibration}. The existence of the
thermodynamic limit as a completed real number (not merely its
approximability) is equivalent to $\LPO$ over $\BISH$. Specifically,
we prove that bounded monotone convergence---instantiated through the
free energy function $\gfun(J) = -\log(2\cosh(\beta J))$ of the 1D
Ising model---is equivalent to $\LPO$ via an explicit encoding of
binary sequences into coupling sequences.

Together, these results establish that the $\LPO$ cost of the
thermodynamic limit is genuine (it is equivalent to, not merely
sufficient for, a known omniscience principle) and dispensable
(the empirical predictions require no omniscience at all).

Both results are formalized in \Lean{} with \Mathlib{} dependencies.
The combined formalization comprises 1374 lines across 18 modules,
with zero sorries and a clean axiom profile: Part~A uses no
omniscience principles whatsoever, while Part~B's main theorem
\texttt{lpo\_of\_bmc} carries only the standard \Lean{} metatheory axioms
(\texttt{propext}, \texttt{Classical.choice}, \texttt{Quot.sound}).
The forward direction of the $\LPO \leftrightarrow \BMC$ equivalence
is axiomatized as \texttt{bmc\_of\_lpo}, citing Bridges and
V\^{\i}\c{t}\u{a} \cite{BV06}.

This paper contributes to a programme of constructive reverse
mathematics applied to mathematical physics, which assigns to each
physical idealization a precise position in the constructive hierarchy.
The programme has established the following calibrations:

\begin{center}
\begin{tabular}{@{}lll@{}}
\toprule
\textbf{Physical layer} & \textbf{Principle} & \textbf{Status} \\
\midrule
Finite-volume Gibbs states & $\BISH$ & Trivially calibrated \\
Finite-size approximations & $\BISH$ & Part~A (this paper) \\
Bidual-gap witness & $\equiv \WLPO$ & Papers~2, 7 \cite{Lee26a,Lee26b} \\
Thermodynamic limit existence & $\equiv \LPO$ & Part~B (this paper) \\
Spectral gap decidability & Undecidable & Cubitt et al.\ \cite{CPW15} \\
\bottomrule
\end{tabular}
\end{center}

\noindent
Our results calibrate two rows of this table, upgrading them from
``route-costed'' to ``formally verified.''

The paper is organized as follows. \Cref{sec:prelim} reviews the
constructive framework, the 1D Ising model, and the free energy
function. \Cref{sec:partA} presents the $\BISH$ dispensability proof
(Part~A). \Cref{sec:partB} presents the $\LPO$ calibration
(Part~B). \Cref{sec:discussion} discusses the synthesis of both
parts, their relation to the phase-transition debate, and future
directions. \Cref{sec:lean} describes the \Lean{} formalization.
\Cref{sec:appendix} collects elementary inequalities.


% ====================================================================
\section{Preliminaries}\label{sec:prelim}
% ====================================================================

\subsection{Constructive Frameworks}

We work within Bishop-style constructive mathematics ($\BISH$):
intuitionistic logic with countable and dependent choice
\cite{Bis67,BB85}. The key omniscience principles form a strict
hierarchy over $\BISH$:

\begin{definition}[LPO]\label{def:lpo} \leanok{}
The \emph{Limited Principle of Omniscience} is
\[
  \LPO \;:\equiv\;
  \forall \alpha : \NN \to \{0,1\},\;
  \bigl(\forall n,\;\alpha(n) = 0\bigr) \;\lor\;
  \bigl(\exists n,\;\alpha(n) = 1\bigr).
\]
\end{definition}

\begin{definition}[BMC]\label{def:bmc} \leanok{}
\emph{Bounded Monotone Convergence} is the assertion that every
bounded non-decreasing sequence of reals has a limit: for every
$a : \NN \to \RR$ with $a_n \leq a_{n+1}$ and $a_n \leq M$ for
all~$n$, there exists $L \in \RR$ and a convergence modulus such
that for all $\varepsilon > 0$, there exists $N_0$ with
$|a_N - L| < \varepsilon$ for all $N \geq N_0$.
\end{definition}

The equivalence $\LPO \leftrightarrow \BMC$ was established by
Bridges and V\^{\i}\c{t}\u{a} \cite{BV06}. The stronger
$\LPO$ implies the \emph{Weak} Limited Principle of Omniscience
($\WLPO$: $\forall \alpha, (\forall n, \alpha(n) = 0) \lor
\lnot(\forall n, \alpha(n) = 0)$), which in turn implies the
Lesser Limited Principle of Omniscience ($\LLPO$). Markov's
Principle is independent of $\WLPO$ over $\BISH$: neither
implies the other \cite{BV06}.

\begin{remark}[Constructive status of $\LPO$]\label{rem:lpo}
$\LPO$ is classically trivial (an instance of excluded middle)
but constructively independent: it is neither provable nor refutable
in $\BISH$. Crucially, $\LPO$ provides a \emph{witness} in the
second disjunct ($\exists n, \alpha(n) = 1$), not merely the
double negation thereof. This witness-providing character is what
makes $\LPO$ strictly stronger than $\WLPO$.
\end{remark}

\subsection{The 1D Ising Model}

Fix a positive integer $N$ (the number of spins). The configuration
space is $\Omega_N = \{-1, +1\}^N$. The Hamiltonian with periodic
boundary conditions and coupling $J > 0$ is
\[
  H_N(\sigma) = -J \sum_{i=1}^{N} \sigma_i \sigma_{i+1},
  \qquad \sigma_{N+1} := \sigma_1.
\]
For simplicity we set $J = 1$ in Part~A (the general case follows
by rescaling $\beta$).

\begin{definition}[Transfer matrix]\label{def:transfer} \leanok{}
The $2 \times 2$ transfer matrix $T$ has entries
$T(s, s') = \exp(\beta \cdot s \cdot s')$ for $s, s' \in \{-1, +1\}$:
\[
  T = \begin{pmatrix} e^{\beta} & e^{-\beta} \\ e^{-\beta} & e^{\beta} \end{pmatrix}.
\]
\end{definition}

\begin{definition}[Eigenvalues]\label{def:eigen} \leanok{}
The eigenvalues of $T$ are
\[
  \eigenP = e^{\beta} + e^{-\beta} = 2\cosh(\beta), \qquad
  \eigenM = e^{\beta} - e^{-\beta} = 2\sinh(\beta).
\]
The corresponding eigenvectors are $(1,1)^T/\sqrt{2}$ and
$(1,-1)^T/\sqrt{2}$.
\end{definition}

\begin{lemma}[Eigenvalue properties]\label{lem:eigprops} \leanok{}
For all $\beta > 0$:
\begin{enumerate}[label=(\alph*)]
  \item $\eigenP > \eigenM > 0$,
  \item $\eigenP > 2$,
  \item $0 < \eigenM / \eigenP < 1$,
  \item $\eigenM / \eigenP = \tanh(\beta)$, with
    $0 < \tanh(\beta) < 1$.
\end{enumerate}
\end{lemma}

\begin{proof}
(a)~$\eigenP - \eigenM = 2e^{-\beta} > 0$, and
$\eigenM = 2\sinh(\beta) > 0$ for $\beta > 0$.
(b)~$\cosh(\beta) > 1$ for $\beta > 0$.
(c)~Immediate from~(a).
(d)~$\eigenM / \eigenP = 2\sinh(\beta) / (2\cosh(\beta)) =
\tanh(\beta)$. For $\beta > 0$, the numerator is positive and
strictly less than the denominator.
\end{proof}

\begin{remark}[Constructive validity]\label{rem:eigen-constr}
All inequalities in \Cref{lem:eigprops} are strict and witnessed
by explicit positive gaps. For rational $\beta$, $\tanh(\beta)$ is
a well-defined constructive real with $0 < \tanh(\beta) < 1$, and
the gap $1 - \tanh(\beta) = 2e^{-\beta}/(e^{\beta} + e^{-\beta})$
is constructively positive.
\end{remark}

\begin{definition}[Partition function]\label{def:partition} \leanok{}
The partition function of the 1D Ising model with $N$ spins is
defined as
\[
  \ZN(\beta) := \eigenP^N + \eigenM^N.
\]
This equals $\operatorname{Tr}(T^N)$, proved in the formalization
as a bonus lemma (\texttt{PartitionTrace.lean}).
\end{definition}

\begin{definition}[Finite-volume free energy density]\label{def:fN} \leanok{}
\[
  \fN(\beta) := -\frac{1}{N} \log \ZN(\beta)
  = -\frac{1}{N} \log\bigl(\eigenP^N + \eigenM^N\bigr).
\]
\end{definition}

\begin{definition}[Infinite-volume free energy density]\label{def:finf} \leanok{}
\[
  \finf(\beta) := -\log(\eigenP) = -\log(2\cosh\beta).
\]
\end{definition}

\begin{remark}[Constructive note]\label{rem:finf-constr}
The infinite-volume free energy density $\finf(\beta)$ is
\emph{not} defined as a limit. It is defined by an explicit
closed-form expression. For rational $\beta > 0$, $\finf(\beta)$
is a constructively well-defined real number. No omniscience
principle is needed to define it. The classical route defines
$\finf = \lim_{N \to \infty} \fN$, proves the limit exists by
monotone convergence ($\LPO$), and then computes the limit. We skip
the middle step entirely: we \emph{define} $\finf$ by closed form,
and then \emph{prove} that $\fN$ converges to it with explicit bounds.
\end{remark}

\subsection{The Free Energy Function $\gfun(J)$}

The following function plays a central role in Part~B.

\begin{definition}[Free energy at coupling $J$]\label{def:gJ} \leanok{}
For $\beta > 0$ and $J > 0$, define
\[
  \gfun(J) := -\log(2\cosh(\beta J)).
\]
This is the infinite-volume free energy density of the 1D Ising
chain with uniform coupling~$J$.
\end{definition}

\begin{lemma}[Strict anti-monotonicity of $\gfun$]\label{lem:ganti} \leanok{}
For $\beta > 0$, the function $\gfun$ is strictly decreasing on
$(0, \infty)$: if $0 < J_0 < J_1$, then $\gfun(J_0) > \gfun(J_1)$.
\end{lemma}

\begin{proof}
The chain of implications is: $J_1 > J_0 > 0$ implies
$\beta J_1 > \beta J_0 > 0$ (multiply by $\beta > 0$), implies
$\cosh(\beta J_1) > \cosh(\beta J_0)$ (since $\cosh$ is strictly
increasing on $(0, \infty)$), implies
$\log(2\cosh(\beta J_1)) > \log(2\cosh(\beta J_0))$ (since $\log$
is strictly increasing on $(0, \infty)$), implies
$\gfun(J_1) < \gfun(J_0)$ (negate both sides).
\end{proof}

\begin{lemma}[Gap lemma]\label{lem:gap} \leanok{}
Fix $\beta > 0$ and $0 < J_0 < J_1$. Then
\[
  \delta := \gfun(J_0) - \gfun(J_1) =
  \log\frac{\cosh(\beta J_1)}{\cosh(\beta J_0)} > 0.
\]
\end{lemma}

\begin{proof}
Immediate from \Cref{lem:ganti}: $J_1 > J_0$ implies
$\gfun(J_1) < \gfun(J_0)$, so $\gfun(J_0) - \gfun(J_1) > 0$.
\end{proof}

\begin{remark}[Constructive positivity of $\delta$]\label{rem:gap-constr}
For rational $\beta$, $J_0$, $J_1$, the gap $\delta$ is a
constructively computable positive real. The positivity is witnessed
by an explicit lower bound computable from $\beta$, $J_0$, $J_1$
via the power series of $\cosh$. This is essential: the decision
procedure in Part~B (\Cref{sec:partB}) uses $\delta > 0$ as a
constructive fact.
\end{remark}


% ====================================================================
\section{Part A: BISH Dispensability}\label{sec:partA}
% ====================================================================

We now prove that the finite-size error bound for the 1D Ising model
is provable in $\BISH$ without any omniscience principle.

\subsection{Free Energy Decomposition}

\begin{lemma}[Decomposition]\label{lem:decomp} \leanok{}
For all $N \geq 1$ and $\beta > 0$, with $r = \tanh(\beta)$:
\[
  \fN(\beta) = -\log(\eigenP) - \frac{1}{N}\log(1 + r^N).
\]
\end{lemma}

\begin{proof}
We compute:
\begin{align*}
  \fN(\beta) &= -\frac{1}{N}\log(\eigenP^N + \eigenM^N) \\
  &= -\frac{1}{N}\log\bigl(\eigenP^N(1 + (\eigenM/\eigenP)^N)\bigr) \\
  &= -\frac{1}{N}\bigl(N\log(\eigenP) + \log(1 + r^N)\bigr) \\
  &= -\log(\eigenP) - \frac{1}{N}\log(1 + r^N). \qedhere
\end{align*}
\end{proof}

\subsection{The Error Bound}

\begin{theorem}[Finite-size bound]\label{thm:error} \leanok{}
For all $N \geq 1$ and $\beta > 0$, with $r = \tanh(\beta) \in (0,1)$:
\[
  |\fN(\beta) - \finf(\beta)| = \frac{1}{N}\log(1 + r^N).
\]
Moreover,
\[
  0 < \frac{1}{N}\log(1 + r^N) \leq \frac{1}{N} r^N.
\]
\end{theorem}

\begin{proof}
The proof proceeds in four steps.

\medskip\noindent\textbf{Step 1: Exact error.}
From \Cref{lem:decomp},
\[
  \fN(\beta) - \finf(\beta) =
  \bigl[-\log(\eigenP) - \tfrac{1}{N}\log(1 + r^N)\bigr]
  - \bigl[-\log(\eigenP)\bigr]
  = -\frac{1}{N}\log(1 + r^N).
\]
Since $0 < r < 1$, we have $0 < r^N < 1$, so $1 < 1 + r^N < 2$,
hence $\log(1 + r^N) > 0$. Therefore $\fN(\beta) - \finf(\beta) < 0$
and $|\fN(\beta) - \finf(\beta)| = \frac{1}{N}\log(1 + r^N) > 0$.

\medskip\noindent\textbf{Step 2: Upper bound via $\log(1+x) \leq x$.}
The elementary inequality $\log(1 + x) \leq x$ for $x > 0$
(see \Cref{sec:appendix}, inequality~A1) applied with $x = r^N$ gives
$\log(1 + r^N) \leq r^N$, so
\[
  |\fN(\beta) - \finf(\beta)| \leq \frac{1}{N} r^N.
\]

\medskip\noindent\textbf{Step 3: Geometric decay.}
Since $r = \tanh(\beta) < 1$ for $\beta > 0$, the bound
$\frac{1}{N} r^N$ decays geometrically (in fact, super-exponentially
in $N$ since the prefactor $1/N$ also decreases). Setting
$c(\beta) = -\log(\tanh\beta) > 0$, we have
$r^N = \exp(-c(\beta) N)$.

\medskip\noindent\textbf{Step 4: Combined bound.}
\[
  |\fN(\beta) - \finf(\beta)| \leq \frac{1}{N} \exp(-c(\beta) N)
\]
where $c(\beta) = -\log(\tanh\beta) > 0$. For the weaker but
cleaner bound: since $1 - \tanh(\beta) = 2/(e^{2\beta}+1)$
and $-\log(1-\delta) \geq \delta$ for $0 < \delta < 1$
(inequality~A2), we get
$c(\beta) \geq 2/(e^{2\beta}+1)$, giving
$|\fN(\beta) - \finf(\beta)| \leq
\frac{1}{N} \exp\bigl(-2N/(e^{2\beta}+1)\bigr)$.
\end{proof}

\subsection{Constructive $N_0$ Witness}

\begin{corollary}[Constructive $N_0$]\label{cor:N0} \leanok{}
For every $\beta > 0$ and $\varepsilon > 0$, there exists a
constructively computable $N_0 \in \NN$ such that for all
$N \geq N_0$:
\[
  |\fN(\beta) - \finf(\beta)| < \varepsilon.
\]
\end{corollary}

\begin{proof}
We need $\frac{1}{N} r^N < \varepsilon$, i.e., $r^N < N\varepsilon$.
Since $r = \tanh(\beta) < 1$, the sequence $r^N$ decreases
geometrically to~$0$ while $N\varepsilon$ grows linearly, so the
inequality is eventually satisfied. The witness $N_0$ can be found
by bounded search: for each candidate $N = 1, 2, 3, \ldots$,
compute the upper bound $\frac{1}{N}\tanh(\beta)^N$ and check
whether it is less than $\varepsilon$. Since $\tanh(\beta)^N$
decreases exponentially to~$0$, there exists a first $N_0$ where
this holds, and it can be found by finite search (which terminates
because the Archimedean property of $\RR$ provides an a priori
upper bound via \texttt{exists\_pow\_lt\_of\_lt\_one} in \Mathlib{}).
\end{proof}

\begin{remark}[No omniscience in the search]\label{rem:N0-constr}
The search for $N_0$ is a bounded search over a decidable predicate
on $\NN$. It requires no omniscience principle: we are searching for
a \emph{witness} to an inequality involving constructively computable
reals, and the search terminates because we have an explicit a~priori
upper bound on $N_0$.
\end{remark}

\subsection{The Dispensability Theorem}

\begin{theorem}[Dispensability]\label{thm:disp} \leanok{}
For the 1D Ising model with periodic boundary conditions, the
following is provable in $\BISH$ (no omniscience principle required):
for every $\beta > 0$ and $\varepsilon > 0$, there exists
$N_0 \in \NN$ (constructively computable from $\beta$ and
$\varepsilon$) such that for all $N \geq N_0$,
\[
  |\fN(\beta) - \finf(\beta)| < \varepsilon,
\]
where $\fN(\beta) = -\frac{1}{N}\log(\eigenP^N + \eigenM^N)$ and
$\finf(\beta) = -\log(2\cosh\beta)$.
\end{theorem}

\begin{proof}
This is the content of \Cref{thm:error} and \Cref{cor:N0}. Every
step uses only arithmetic of real numbers ($\BISH$), properties of
$\exp$ and $\log$ (constructive), the inequality $\log(1+x) \leq x$
for $x > 0$ (constructive; \Cref{sec:appendix}), and bounded search
on $\NN$ (no omniscience needed). No use is made of the monotone
convergence theorem ($\LPO$), the Bolzano--Weierstra\ss{} theorem
($\LPO$), or any omniscience principle ($\WLPO$, $\LPO$, $\LLPO$).
\end{proof}

\begin{remark}[Classical vs.\ constructive routes]\label{rem:routes}
The classical proof that $\fN(\beta) \to \finf(\beta)$ uses the
following route: (1)~show $\{\fN\}$ is bounded and monotone (using
subadditivity); (2)~apply the monotone convergence theorem (costs
$\LPO$); (3)~identify the limit as $-\log(2\cosh\beta)$ by separate
calculation. Our proof replaces steps~(1)--(2) with a direct
computation: define $\finf(\beta)$ by closed form, compute
$|\fN - \finf|$ explicitly, and bound the error using elementary
inequalities. The empirical content---``for large enough $N$, $\fN$
approximates $\finf$ within $\varepsilon$''---is the same, but the
logical cost is $\BISH$ instead of $\LPO$.
\end{remark}


% ====================================================================
\section{Part B: LPO Calibration}\label{sec:partB}
% ====================================================================

We now prove that the existence of the thermodynamic limit as a
completed real number is equivalent to $\LPO$ over $\BISH$.

\subsection{Forward Direction: LPO $\Rightarrow$ BMC}

\begin{theorem}[LPO implies BMC]\label{thm:forward} \leanpartial{}
$\LPO$ implies $\BMC$.
\end{theorem}

\begin{proof}
This is \cite[Theorem~2.1.5]{BV06}. We axiomatize it as
\texttt{bmc\_of\_lpo} and cite the original proof.
The argument proceeds by binary search on the value axis: given a
bounded monotone sequence $(a_n)$ and using $\LPO$ at each step to
decide whether the sequence eventually exceeds a given rational
threshold, one constructs the supremum as a constructive real.
\end{proof}

\begin{remark}[Axiomatization]\label{rem:axiom}
The forward direction is axiomatized following the same pattern as
\texttt{ell1\_not\_reflexive} in \cite{Lee26b}. The novel content
of this paper is the backward direction and the physical
instantiation. A complete formalization of the forward direction is
an elimination target for future work.
\end{remark}

\subsection{The Encoding}

The backward direction encodes an arbitrary binary sequence into a
free energy sequence of the 1D Ising model.

\begin{definition}[Running maximum]\label{def:runmax} \leanok{}
Given $\alpha : \NN \to \{0,1\}$, define the running maximum
$m : \NN \to \{0,1\}$ by
\[
  m(0) := \alpha(0), \qquad
  m(n+1) := \max\bigl(m(n),\, \alpha(n+1)\bigr).
\]
Equivalently, $m(n) = \max(\alpha(0), \alpha(1), \ldots, \alpha(n))$.
\end{definition}

The running maximum has the following properties, all $\BISH$-provable:
$m$ is non-decreasing (in the $\{0,1\}$ order), $m(n) = 0$ if and
only if $\alpha(k) = 0$ for all $k \leq n$, and $m(n) = 1$ if and
only if there exists $k \leq n$ with $\alpha(k) = 1$.

\begin{definition}[Coupling sequence]\label{def:coupling} \leanok{}
Fix $0 < J_0 < J_1$. Define $J : \NN \to \RR$ by
\[
  J(n) := \begin{cases} J_0 & \text{if } m(n) = 0, \\
  J_1 & \text{if } m(n) = 1. \end{cases}
\]
\end{definition}

The coupling sequence takes values in $\{J_0, J_1\}$, is
non-decreasing (since $m$ is non-decreasing and $J_0 < J_1$), and
is bounded: $J_0 \leq J(n) \leq J_1$.

\begin{definition}[Encoded sequence]\label{def:encoded} \leanok{}
Define $F : \NN \to \RR$ by
\[
  F(n) := \gfun(J(n)) = -\log\bigl(2\cosh(\beta \cdot J(n))\bigr).
\]
\end{definition}

Since $\gfun$ is strictly decreasing (\Cref{lem:ganti}) and $J$ is
non-decreasing, $F$ is non-increasing. Equivalently, $-F$ is
non-decreasing and bounded above by $-\gfun(J_1)$, so $\BMC$ applies
to $-F$.

\subsection{The Two Regimes}

There are exactly two cases for the eventual behavior of the encoded
sequence.

If $\alpha \equiv 0$, then $m \equiv 0$, $J \equiv J_0$, and
$F \equiv \gfun(J_0)$. The sequence is constant; its limit is
$\gfun(J_0)$.

If there exists $n_0$ with $\alpha(n_0) = 1$, then for all
$n \geq n_0$: $m(n) = 1$, $J(n) = J_1$, $F(n) = \gfun(J_1)$.
The sequence is eventually constant at $\gfun(J_1)$; its limit is
$\gfun(J_1)$.

In both cases, the limit exists trivially (eventually constant
sequences converge in $\BISH$). But \emph{which} limit obtains
depends on $\alpha$, and $\BISH$ cannot decide this without $\LPO$.
The gap $\delta = \gfun(J_0) - \gfun(J_1) > 0$ separates the two
possible limit values.

\subsection{The Decision Procedure}

\begin{theorem}[BMC implies LPO]\label{thm:backward} \leanok{}
$\BMC$ implies $\LPO$.
\end{theorem}

\begin{proof}
Let $\alpha : \NN \to \{0,1\}$ be given. Fix $\beta = 1$,
$J_0 = 1$, $J_1 = 2$ (any positive values with $J_0 < J_1$ suffice).
Construct the encoded sequence $F$ and its negation $-F$ as above.

\medskip\noindent\textbf{Step 1: Apply BMC.}
The sequence $-F$ is non-decreasing and bounded above by
$-\gfun(J_1)$. By $\BMC$, there exists $L_{\mathrm{neg}} \in \RR$
with a convergence modulus: for every $\varepsilon > 0$, there
exists $N_0$ such that $|(-F)(N) - L_{\mathrm{neg}}| < \varepsilon$
for all $N \geq N_0$.

\medskip\noindent\textbf{Step 2: Compute the gap.}
Set $\delta = \gfun(J_0) - \gfun(J_1) > 0$ (\Cref{lem:gap}).

\medskip\noindent\textbf{Step 3: Get $N_1$ from the modulus.}
Apply the convergence modulus with $\varepsilon = \delta/2$ to
obtain $N_1$ such that $|(-F)(N_1) - L_{\mathrm{neg}}| < \delta/2$.

\medskip\noindent\textbf{Step 4: Case split on $m(N_1)$.}
The value $m(N_1) = \texttt{runMax}\;\alpha\;N_1$ is a Bool, so the
case split is definitionally decidable---no real-number comparison
is needed.

\medskip\noindent\textbf{Case $m(N_1) = \mathtt{false}$:}
We prove $\forall n, \alpha(n) = 0$ by contradiction. Suppose there
exists $n_0$ with $\alpha(n_0) = 1$. Then the limit of $-F$ is
$-\gfun(J_1)$, since $F$ is eventually constant at $\gfun(J_1)$
(see \Cref{def:encoded}). But $F(N_1) = \gfun(J_0)$ (since
$m(N_1) = \mathtt{false}$ implies $J(N_1) = J_0$), so
\[
  |(-F)(N_1) - L_{\mathrm{neg}}| =
  |-\gfun(J_0) - (-\gfun(J_1))| =
  |\gfun(J_1) - \gfun(J_0)| = \delta.
\]
But the modulus gives $|(-F)(N_1) - L_{\mathrm{neg}}| < \delta/2$,
a contradiction. Therefore $\lnot(\exists n_0, \alpha(n_0) = 1)$.
Since $\alpha(n) \in \{0, 1\}$ is decidable for each~$n$, this
gives $\forall n, \alpha(n) = 0$.

\medskip\noindent\textbf{Case $m(N_1) = \mathtt{true}$:}
By the characterization of the running maximum, $m(N_1) = 1$ implies
there exists $k \leq N_1$ with $\alpha(k) = 1$. A bounded search
over $\{0, 1, \ldots, N_1\}$ finds the witness.
\end{proof}

\begin{remark}[Decidability of the case split]\label{rem:decidable}
The case split on $m(N_1)$ is the key constructive insight: it is a
Bool case split, not a real-number comparison. We do not need to
compare $F(N_1)$ with $\gfun(J_0)$ as real numbers. We compute
$m(N_1)$ from $\alpha(0), \ldots, \alpha(N_1)$ by finite recursion
and branch on the result. If $m(N_1) = \mathtt{true}$, we already
have our witness (bounded search on $\{0, \ldots, N_1\}$). If
$m(N_1) = \mathtt{false}$, we derive a contradiction from the
convergence bound.
\end{remark}

\begin{remark}[Constructive validity of $\lnot\exists \to \forall\lnot$]\label{rem:neg-exists}
The step from $\lnot(\exists n, \alpha(n) = 1)$ to
$\forall n, \alpha(n) = 0$ is constructively valid because
$\alpha(n) \in \{0, 1\}$---the predicate is decidable. We are
\emph{not} using the classical equivalence
$\lnot\exists x.\, P(x) \leftrightarrow \forall x.\, \lnot P(x)$
for arbitrary predicates (which requires excluded middle). For
decidable predicates on~$\NN$, the equivalence is $\BISH$-valid.
\end{remark}

\subsection{The Equivalence Theorem}

\begin{theorem}[LPO $\leftrightarrow$ BMC]\label{thm:equiv} \leanok{}/\leanpartial{}
Over $\BISH$, $\LPO \leftrightarrow \BMC$.
\end{theorem}

\begin{proof}
Forward: \Cref{thm:forward} (\cite{BV06}; axiomatized as
\texttt{bmc\_of\_lpo}).
Backward: \Cref{thm:backward} (fully proved in
\texttt{PartB\_Backward.lean}).
\end{proof}


% ====================================================================
\section{Discussion}\label{sec:discussion}
% ====================================================================

\subsection{The Dispensability--Calibration Conjunction}

Neither Part~A nor Part~B says much in isolation. Part~A alone is a
calculation: the 1D Ising model has a closed-form solution, so of
course finite-size bounds are elementary. Part~B alone is an
instantiation of a known equivalence: $\BMC \leftrightarrow \LPO$
is Bridges--V\^{\i}\c{t}\u{a}, and dressing it in Ising clothing
does not change the abstract content. The force of the paper lies
in the conjunction.

Part~B establishes that the monotone-convergence route to the
thermodynamic limit genuinely costs $\LPO$---the cost is not an
artefact of a particular proof strategy but an intrinsic feature of
the limit assertion. Part~A then shows that this cost is
dispensable for empirical predictions: the finite-system prediction
$\fN(\beta)$ approximates $\finf(\beta)$ within $\varepsilon$
for constructively computable $N_0$, and the proof uses nothing
beyond $\BISH$. The pattern is: the idealization costs omniscience;
the empirical content does not.

This pattern is precisely what the logical geography hypothesis
predicts. The 1D Ising model is the first complete test
case---the simplest model where the dispensability question is
nontrivial and verifiable.

\subsection{The Constructive Reverse Mathematics Programme}

This paper is part of a programme that assigns to each physical
idealization a precise position in the constructive hierarchy.
The programme methodology is as follows: for a given physical
theory, identify the key mathematical idealizations (infinite
limits, existence of witnesses, decidability assertions), determine
the exact logical cost of each idealization over $\BISH$ (by
proving equivalence with a known omniscience principle or showing
$\BISH$-validity), and then ask whether the empirical content of
the theory can be recovered at a lower logical cost.

The programme has so far established the following calibrations.
Papers~2 and~7 \cite{Lee26a,Lee26b} showed that the bidual gap---the
existence of a constructive witness to Banach space
non-reflexivity---is equivalent to $\WLPO$ for both $\ell^1$ and
the trace-class operators $S_1(H)$. In the algebraic formulation
of quantum mechanics, $S_1(H)$ is the natural state space, and
non-reflexivity means the bidual $S_1(H)^{**}$ contains ``singular
states'' not representable by any density matrix. The $\WLPO$
equivalence calibrates the logical cost of witnessing these singular
states.

The present paper calibrates two further layers. Part~A shows that
finite-size approximations to the thermodynamic limit are
$\BISH$-valid, requiring no omniscience. Part~B shows that the full
thermodynamic limit (as a completed real number) costs exactly
$\LPO$, which is strictly stronger than $\WLPO$.

The resulting calibration landscape is:

\begin{center}
\begin{tabular}{@{}llll@{}}
\toprule
\textbf{Physical layer} & \textbf{Principle} & \textbf{Status} & \textbf{Source} \\
\midrule
Finite-volume Gibbs states & $\BISH$ & Calibrated & Trivial \\
Finite-size approximations & $\BISH$ & Calibrated & Part~A \\
Bidual-gap witness ($S_1(H)$) & $\equiv \WLPO$ & Calibrated & Papers~2, 7 \\
Thermodynamic limit existence & $\equiv \LPO$ & Calibrated & Part~B \\
Spectral gap decidability & Undecidable & Established & Cubitt et al.\ \cite{CPW15} \\
\bottomrule
\end{tabular}
\end{center}

\noindent
The hierarchy $\BISH \subsetneq \WLPO \subsetneq \LPO \subsetneq
\mathrm{LEM}$ is strictly ordered over $\BISH$, and the physical
layers sit at distinct rungs. Each formalization carries a machine-checked
axiom audit confirming the claimed logical cost.

An important feature of the programme is that the dispensability
pattern---idealizations cost omniscience, but empirical content does
not---may be generic. The 1D Ising model is the simplest test case
where this can be verified, but the methodology applies to any
physical theory with an explicit finite-size/infinite-volume
dichotomy. Whether the pattern persists for higher-dimensional models,
models with phase transitions, or quantum field theories remains
an open question.

\subsection{Relation to the Phase-Transition Debate}

The philosophical literature on the ``paradox of phase
transitions''---the apparent indispensability of infinite
idealizations for explaining phase transitions in finite
systems---has been active for two decades. Batterman \cite{Bat02,Bat05}
argued that infinite limits play an essential and irreducible
explanatory role. Butterfield \cite{But11} and Callender \cite{Cal01}
pushed back, arguing that finite-system approximations suffice for
physical predictions even if the mathematical apparatus of the
thermodynamic limit is explanatorily convenient. Van~Wierst
\cite{vW19} explored the consequences of adopting constructive
mathematics for the phase transition framework, arguing that
constructive mathematics forces ``de-idealizations'' of standard
statistical-mechanical theories.

Our results make this debate precise in one model. The
thermodynamic limit is not merely ``convenient''---it has a precise
logical cost ($\LPO$), and this cost is not an artefact of the proof
but an equivalence. At the same time, the limit is genuinely
dispensable for predictions: the finite-size error bound is
$\BISH$-provable. The 1D Ising model, admittedly, does not exhibit
phase transitions, so our dispensability result does not directly
address the paradox as stated (which concerns the necessity of
infinite limits for \emph{explaining} phase transitions). But it does
establish the methodology: for each physical layer, determine the
exact logical cost and then ask whether the empirical content can
be recovered at a lower cost.

\subsection{The Encoding Objection}

A natural objection to Part~B is that the encoding of binary
sequences into coupling sequences---and the subsequent application
of the free energy function $\gfun(J)$---is merely bounded monotone
convergence in disguise. The encoded sequence
$F(N) \in \{\gfun(J_0), \gfun(J_1)\}$ is a $\{0,1\}$-valued
monotone sequence composed with a strictly decreasing function, and
the decision procedure is just the abstract $\BMC \to \LPO$ proof
applied to this specific case.

This objection is mathematically correct and interpretively
irrelevant. The abstract equivalence $\BMC \leftrightarrow \LPO$ is
known from \cite{BV06}. The contribution of Part~B is not a new
theorem in constructive reverse mathematics but a verified
observation that $\BMC$, when instantiated through the 1D Ising free
energy, \emph{is} the assertion that the thermodynamic limit exists.
The formalization makes explicit what the mathematical prose leaves
implicit: the encoding is $\BISH$-valid, the gap
$\delta = \gfun(J_0) - \gfun(J_1) > 0$ is constructively positive,
and the witness extraction works without hidden omniscience. The
\Lean{} axiom audit confirms this.

This is the same methodological move as \cite{Lee26a}, where the
abstract equivalence between $\WLPO$ and $\lnot\lnot$-stable
decidability was known from Ishihara and Diener, and the
contribution was the specific Banach-space instantiation and the
machine verification.

\subsection{Limitations and Future Directions}

The 1D Ising model is the simplest nontrivial lattice model, and
our results exploit its complete solvability. The key open
questions are as follows.

First, regarding \emph{higher dimensions}: the 2D Ising model
(Onsager solution \cite{Ons44}) has a phase transition. Does the
finite-size error bound remain $\BISH$-provable? The Onsager
solution involves elliptic integrals, whose constructive status
requires investigation.

Second, regarding \emph{general Hamiltonians}: for
translation-invariant, finite-range Hamiltonians on $\ZZ^d$, the
thermodynamic limit exists classically by subadditivity \cite{Rue99}.
Is the existence always $\LPO$-equivalent, or does it depend on the
Hamiltonian?

Third, regarding \emph{ineliminability}: an ineliminability
result---showing that \emph{any} constructive proof of free energy
convergence for a specific model must use $\BMC$---would be a
genuinely new contribution to constructive reverse mathematics.
This is an open problem beyond the scope of the present paper.


% ====================================================================
\section{Lean 4 Formalization}\label{sec:lean}
% ====================================================================

\subsection{Module Structure}

The formalization is organized as a single \Lean{} project with two
parts sharing common infrastructure.

\paragraph{Part A: BISH dispensability (730 lines, 10 modules).}

\begin{table}[ht]
\centering
\begin{tabular}{@{}lrl@{}}
\toprule
\textbf{File} & \textbf{Lines} & \textbf{Purpose} \\
\midrule
\texttt{Basic.lean}             & 67  & Core definitions: LPO, eigenvalues, partition function, free energy \\
\texttt{EigenvalueProps.lean}   & 119 & $\eigenP > \eigenM > 0$, $\tanh$ properties, partition positivity \\
\texttt{LogBounds.lean}         & 70  & Elementary inequalities: $\log(1+x) \leq x$, geometric decay \\
\texttt{TransferMatrix.lean}    & 117 & $2 \times 2$ transfer matrix $T$, projector decomposition \\
\texttt{PartitionTrace.lean}    & 64  & Bonus: $\operatorname{Tr}(T^N) = \eigenP^N + \eigenM^N$ \\
\texttt{FreeEnergyDecomp.lean}  & 87  & $\fN = -\log\eigenP - \frac{1}{N}\log(1 + r^N)$ \\
\texttt{ErrorBound.lean}        & 72  & $|\fN - \finf| \leq \frac{1}{N} r^N$ \\
\texttt{ComputeN0.lean}         & 54  & Constructive $N_0$ from $\beta$ and $\varepsilon$ \\
\texttt{Main.lean}              & 72  & Assembly of dispensability theorem + axiom audit \\
\texttt{SmokeTest.lean}         & 7   & Minimal import validation \\
\bottomrule
\end{tabular}
\caption{Part A file manifest.}
\label{tab:partA}
\end{table}

\paragraph{Part B: LPO calibration (644 lines, 8 modules).}

\begin{table}[ht]
\centering
\begin{tabular}{@{}lrl@{}}
\toprule
\textbf{File} & \textbf{Lines} & \textbf{Purpose} \\
\midrule
\texttt{PartB\_Defs.lean}            & 76  & Definitions: BMC, runMax, couplingSeq, encodedSeq \\
\texttt{PartB\_RunMax.lean}          & 103 & Running maximum: monotonicity, characterization lemmas \\
\texttt{PartB\_FreeEnergyAnti.lean}  & 73  & $\gfun(J)$ strictly anti-monotone for $\beta > 0$ \\
\texttt{PartB\_CouplingSeq.lean}     & 76  & Coupling: monotonicity, bounds, eventual constancy \\
\texttt{PartB\_EncodedSeq.lean}      & 83  & Encoded sequence: $-F$ non-decreasing, bounded \\
\texttt{PartB\_Forward.lean}         & 21  & Axiom: LPO $\to$ BMC \cite{BV06} \\
\texttt{PartB\_Backward.lean}        & 154 & Main theorem: BMC $\to$ LPO via free energy encoding \\
\texttt{PartB\_Main.lean}            & 58  & Assembly: LPO $\leftrightarrow$ BMC + axiom audit \\
\bottomrule
\end{tabular}
\caption{Part B file manifest.}
\label{tab:partB}
\end{table}

Combined total: 18 files, 1374 lines.

\subsection{Core Definitions}

The definitions in \texttt{Basic.lean} encode LPO, the transfer
matrix eigenvalues, and the free energy:

\begin{lstlisting}[caption={Core definitions (Basic.lean).}]
/-- Limited Principle of Omniscience. -/
def LPO : Prop :=
  forall (a : Nat -> Bool),
    (forall n, a n = false) ||| (exists n, a n = true)

noncomputable def transferEigenPlus (b : Real) : Real :=
  2 * Real.cosh b
noncomputable def transferEigenMinus (b : Real) : Real :=
  2 * Real.sinh b

noncomputable def partitionFn (b : Real) (N : Nat) : Real :=
  (transferEigenPlus b) ^ N + (transferEigenMinus b) ^ N

noncomputable def freeEnergyDensity (b : Real) (N : Nat)
    (_hN : 0 < N) : Real :=
  -(1 / (N : Real)) * Real.log (partitionFn b N)

noncomputable def freeEnergyInfVol (b : Real) : Real :=
  -Real.log (transferEigenPlus b)
\end{lstlisting}

The Part~B definitions in \texttt{PartB\_Defs.lean} encode BMC, the
running maximum, coupling sequence, and encoded sequence:

\begin{lstlisting}[caption={Part B definitions (PartB\_Defs.lean).}]
def BMC : Prop :=
  forall (a : Nat -> Real) (M : Real),
    Monotone a -> (forall n, a n <= M) ->
    exists L : Real, forall e : Real, 0 < e ->
      exists N0 : Nat, forall N : Nat, N0 <= N ->
        |a N - L| < e

def runMax (a : Nat -> Bool) : Nat -> Bool
  | 0 => a 0
  | n + 1 => a (n + 1) || runMax a n

noncomputable def couplingSeq (a : Nat -> Bool)
    (J0 J1 : Real) (n : Nat) : Real :=
  if runMax a n then J1 else J0

noncomputable def freeEnergyAtCoupling (b J : Real) : Real :=
  -Real.log (2 * Real.cosh (b * J))

noncomputable def encodedSeq (a : Nat -> Bool)
    (b J0 J1 : Real) (n : Nat) : Real :=
  freeEnergyAtCoupling b (couplingSeq a J0 J1 n)
\end{lstlisting}

\subsection{Main Theorem: Dispensability}

\begin{lstlisting}[caption={Dispensability theorem (Main.lean).}]
theorem ising_1d_dispensability
    (b : Real) (hb : 0 < b) (e : Real) (he : 0 < e) :
    exists N0 : Nat, 0 < N0 && forall N : Nat, N0 <= N ->
      (hN : 0 < N) ->
        |freeEnergyDensity b N hN - freeEnergyInfVol b| < e
\end{lstlisting}

\subsection{Main Theorem: BMC $\to$ LPO}

The full proof of the backward direction
(\texttt{PartB\_Backward.lean}) is the main novel content of the
formalization. We reproduce it here in full:

\begin{lstlisting}[caption={BMC implies LPO (PartB\_Backward.lean, complete proof).}]
theorem lpo_of_bmc (hBMC : BMC) : LPO := by
  intro a
  set b : Real := 1
  set J0 : Real := 1
  set J1 : Real := 2
  have hb : (0 : Real) < b := one_pos
  have hJ0 : (0 : Real) < J0 := one_pos
  have hJ_lt : J0 < J1 := by norm_num
  have hJ_le : J0 <= J1 := le_of_lt hJ_lt
  set F := encodedSeq a b J0 J1 with hF_def
  have hMono : Monotone (fun n => -F n) :=
    neg_encodedSeq_mono a hb hJ0 hJ_le
  have hBdd : forall n, (fun n => -F n) n
      <= -freeEnergyAtCoupling b J1 :=
    neg_encodedSeq_bounded a hb hJ0 hJ_le
  obtain <<L_neg, hL>> := hBMC (fun n => -F n)
    (-freeEnergyAtCoupling b J1) hMono hBdd
  set d := freeEnergyAtCoupling b J0
    - freeEnergyAtCoupling b J1 with hd_def
  have hd : 0 < d := freeEnergy_gap_pos hb hJ0 hJ_lt
  obtain <<N1, hN1>> := hL (d / 2) (half_pos hd)
  have hN1_self := hN1 N1 (le_refl _)
  cases hm : runMax a N1
  . -- Case: runMax a N1 = false
    left
    apply bool_not_exists_implies_all_false
    intro <<n0, hn0>>
    have hL_val := neg_limit_of_exists_true a hL hn0
    have hFN1 : F N1 = freeEnergyAtCoupling b J0 := by
      simp only [hF_def, encodedSeq, couplingSeq, hm,
        Bool.false_eq_true, ite_false]
    have habs : |(-F N1) - L_neg| = d := by
      rw [hFN1, hL_val]
      simp only [neg_sub_neg]
      rw [abs_sub_comm]
      exact abs_of_pos hd
    have : |(-F N1) - L_neg| < d / 2 := hN1_self
    linarith
  . -- Case: runMax a N1 = true
    right
    obtain <<k, _, hk>> := runMax_witness a
      (show runMax a N1 = true from hm)
    exact <<k, hk>>
\end{lstlisting}

\subsection{Equivalence and Axiom Audit}

\begin{lstlisting}[caption={Equivalence theorem and axiom audit (PartB\_Main.lean).}]
theorem lpo_iff_bmc : LPO <-> BMC :=
  <<bmc_of_lpo, lpo_of_bmc>>

-- Part A main theorem:
#print axioms ising_1d_dispensability
-- [propext, Classical.choice, Quot.sound]

-- Part B backward direction:
#print axioms lpo_of_bmc
-- [propext, Classical.choice, Quot.sound]

-- Part B equivalence:
#print axioms lpo_iff_bmc
-- [propext, Classical.choice, Quot.sound,
--  Papers.P8.bmc_of_lpo]
\end{lstlisting}

The Part~A audit confirms that the $\BISH$ dispensability proof uses
no omniscience principles. \texttt{Classical.choice} is the ambient
\Lean{} metatheory, not an object-level axiom. The Part~B audit
shows that the novel content (backward direction) is fully proved,
while the forward direction is axiomatized with citation.

\subsection{Design Decisions}

\paragraph{Direct eigenvalue definition of $\ZN$.}
The partition function $\ZN$ is defined directly as
$\eigenP^N + \eigenM^N$ (the eigenvalue formula). The classical
identity $\ZN = \operatorname{Tr}(T^N) = \sum_\sigma
\exp(-\beta H_N(\sigma))$ is provided as a bonus lemma
(\texttt{PartitionTrace.lean}) connecting the definition to the
transfer matrix, but is not used in either Part~A or Part~B. The
heavier direction (configuration sum $= \operatorname{Tr}(T^N)$) is
documented in the paper but omitted from the formalization. This
keeps the axiom profile clean without bridge axioms.

\paragraph{The \texttt{bmc\_of\_lpo} axiom.}
The forward direction ($\LPO \to \BMC$) is axiomatized, following
the same pattern as \texttt{ell1\_not\_reflexive} in \cite{Lee26b}.
The novel content of the paper is the backward direction and the
physical instantiation. A complete formalization of the forward
direction is an elimination target for future work.

\subsection{AI-Assisted Methodology}\label{sec:ai}

This formalization was developed using \textbf{Claude Opus~4.6}
(Anthropic, 2026) via the \textbf{Claude Code} command-line interface,
following the same human--AI workflow as Papers~2 and~7
\cite{Lee26a,Lee26b,Anthropic2026}. The human author wrote
mathematical blueprints specifying all theorem statements, proof
strategies, and target \Mathlib{} APIs. Claude Opus~4.6 then explored
the \Mathlib{} codebase to locate exact API signatures and import
paths, generated the \Lean{} proof terms, and handled debugging of
tactic proofs against \Mathlib{} v4.28. The human author reviewed all
proofs for mathematical correctness and \Mathlib{} conventions. Final
verification was by \texttt{lake build} (0~errors, 0~warnings,
0~sorries).

\begin{table}[h]
\centering
\begin{tabular}{@{}lll@{}}
\toprule
\textbf{Task} & \textbf{Human} & \textbf{AI (Claude Opus 4.6)} \\
\midrule
Mathematical blueprint    & \checkmark & \\
Proof strategy design     & \checkmark & \\
\Mathlib{} API discovery  & & \checkmark \\
\Lean{} proof generation  & & \checkmark \\
Proof review              & \checkmark & \\
Build verification        & & \checkmark \\
Paper writing             & \checkmark & \checkmark \\
\bottomrule
\end{tabular}
\caption{Division of labor between human and AI.}
\label{tab:division}
\end{table}

\subsection{Reproducibility}

\begin{mdframed}[backgroundcolor=gray!10]
\textbf{Reproducibility Box}
\begin{itemize}
\item \textbf{Repository}: \url{https://github.com/quantmann/FoundationRelativity}
\item \textbf{LaTeX source \& PDF}: \url{https://doi.org/10.5281/zenodo.18516813}
\item \textbf{Lean toolchain}: \texttt{leanprover/lean4:v4.28.0-rc1}
\item \textbf{mathlib4 commit}: \texttt{7091f0f601d5aaea565d2304c1a290cc8af03e18}
\item \textbf{Build}: \texttt{lake exe cache get \&\& lake build}
\item \textbf{Bundle target}: \texttt{Papers}
  (imports \texttt{Main} + \texttt{PartB\_Main})
\item \textbf{Status}: 0~errors, 0~warnings, 0~sorries.
  18~files, 1374~lines total.
\item \textbf{Axiom profile}:
  \texttt{ising\_1d\_dispensability}: \texttt{[propext, Classical.choice, Quot.sound]}.
  \texttt{lpo\_of\_bmc}: \texttt{[propext, Classical.choice, Quot.sound]}.
  \texttt{lpo\_iff\_bmc}: \texttt{[propext, Classical.choice, Quot.sound,
  Papers.P8.bmc\_of\_lpo]}.
\end{itemize}
\end{mdframed}


% ====================================================================
\section*{Acknowledgments}
% ====================================================================

The \Lean{} formalization was developed using Claude Opus~4.6
(Anthropic, 2026) via the Claude Code CLI tool. We thank the
\Mathlib{} community for maintaining the comprehensive library
of formalized mathematics that made this work possible.


% ====================================================================
\appendix
\section{Elementary Inequalities}\label{sec:appendix}
% ====================================================================

For reference, the constructive inequalities used in the Part~A proof.

\begin{lemma}[A1]\label{lem:A1}
For $x > 0$: $\log(1 + x) \leq x$.
\end{lemma}

\begin{proof}
Equivalent to $1 + x \leq \exp(x)$, which follows from the Taylor
series $\exp(x) = 1 + x + x^2/2! + \cdots \geq 1 + x$ for $x > 0$.
\end{proof}

\begin{lemma}[A2]\label{lem:A2}
For $0 < \delta < 1$: $-\log(1 - \delta) \geq \delta$.
\end{lemma}

\begin{proof}
Equivalent to $1 - \delta \leq \exp(-\delta)$, which is inequality~A1
applied with $x = -\delta$: $\exp(-\delta) \geq 1 + (-\delta) =
1 - \delta$.
\end{proof}

\begin{lemma}[A3]\label{lem:A3}
For $0 < r < 1$ and $N \geq 1$: $r^N \leq \exp(-N(1-r))$.
\end{lemma}

\begin{proof}
Set $\delta = 1 - r > 0$. Then $r = 1 - \delta \leq \exp(-\delta)$
by A2 (applied as $\exp(\delta) \geq 1 + \delta$, so
$1 - \delta \leq \exp(-\delta)$). Hence
$r^N \leq \exp(-\delta)^N = \exp(-N\delta) = \exp(-N(1-r))$.
\end{proof}

All three inequalities are constructively valid: they follow from the
constructive Taylor series expansion of $\exp$.


% ====================================================================
% Bibliography
% ====================================================================
\bibliographystyle{plainnat}

\begin{thebibliography}{30}

\bibitem[Anthropic(2026)]{Anthropic2026}
Anthropic.
\newblock Claude {Opus}~4.6 and {Claude Code} {CLI}.
\newblock \url{https://www.anthropic.com/claude}, 2026.

\bibitem[Batterman(2002)]{Bat02}
R.~W.~Batterman.
\newblock \emph{The Devil in the Details: Asymptotic Reasoning in
  Explanation, Reduction, and Emergence}.
\newblock Oxford University Press, New York, 2002.

\bibitem[Batterman(2005)]{Bat05}
R.~W.~Batterman.
\newblock Critical phenomena and breaking drops: Infinite idealizations
  in physics.
\newblock \emph{Studies in History and Philosophy of Modern Physics},
  36:225--244, 2005.

\bibitem[Baxter(1982)]{Bax82}
R.~J.~Baxter.
\newblock \emph{Exactly Solved Models in Statistical Mechanics}.
\newblock Academic Press, London, 1982.

\bibitem[Bishop(1967)]{Bis67}
E.~Bishop.
\newblock \emph{Foundations of Constructive Analysis}.
\newblock McGraw-Hill, New York, 1967.

\bibitem[Bishop and Bridges(1985)]{BB85}
E.~Bishop and D.~Bridges.
\newblock \emph{Constructive Analysis}.
\newblock Grundlehren der mathematischen Wissenschaften, vol.~279.
  Springer, Berlin, 1985.

\bibitem[Bridges and V{\^\i}{\c{t}}{\u{a}}(2006)]{BV06}
D.~S.~Bridges and L.~S.~V{\^\i}{\c{t}}{\u{a}}.
\newblock \emph{Techniques of Constructive Analysis}.
\newblock Universitext. Springer, New York, 2006.

\bibitem[Butterfield(2011)]{But11}
J.~Butterfield.
\newblock Less is different: Emergence and reduction reconciled.
\newblock \emph{Foundations of Physics}, 41(6):1065--1135, 2011.

\bibitem[Callender(2001)]{Cal01}
C.~Callender.
\newblock Taking thermodynamics too seriously.
\newblock \emph{Studies in History and Philosophy of Science Part~B},
  32(4):539--553, 2001.

\bibitem[Cubitt et~al.(2015)]{CPW15}
T.~S.~Cubitt, D.~Perez-Garcia, and M.~M.~Wolf.
\newblock Undecidability of the spectral gap.
\newblock \emph{Nature}, 528:207--211, 2015.

\bibitem[{de Moura} et~al.(2021)]{deMoura2021}
L.~{de Moura}, S.~Kong, J.~Avigad, F.~{van Doorn}, and M.~{von Raumer}.
\newblock The {Lean} theorem prover (system description).
\newblock In \emph{CADE-25}, LNAI 9195, pages 378--388. Springer, 2015.
\newblock Lean~4: \url{https://lean-lang.org/}, 2021--present.

\bibitem[Diener(2018)]{Die18}
H.~Diener.
\newblock \emph{Constructive Reverse Mathematics}.
\newblock Habilitationsschrift, Universit\"at Siegen, 2018.
\newblock arXiv:1804.05495.

\bibitem[Fletcher et~al.(2019)]{FPRRS19}
S.~C.~Fletcher, P.~Palacios, L.~Ruetsche, and E.~Shech.
\newblock Infinite idealizations in science: an introduction.
\newblock \emph{Synthese}, 196(5):1657--1669, 2019.

\bibitem[Ishihara(1990)]{Ish90}
H.~Ishihara.
\newblock An omniscience principle, the {K\"onig} lemma and the
  {Hahn--Banach} theorem.
\newblock \emph{Zeitschrift f\"ur Mathematische Logik und Grundlagen
  der Mathematik}, 36:237--240, 1990.

\bibitem[Ishihara(2005)]{Ish05}
H.~Ishihara.
\newblock Constructive reverse mathematics: compactness properties.
\newblock In L.~Crosilla and P.~Schuster, editors,
  \emph{From Sets and Types to Topology and Analysis},
  Oxford Logic Guides, vol.~48, pages 245--267.
  Oxford University Press, 2005.

\bibitem[Ishihara(2006)]{Ish06}
H.~Ishihara.
\newblock Reverse mathematics in {Bishop's} constructive mathematics.
\newblock \emph{Philosophia Scientiae, Cahier Sp\'ecial}, 6:43--59, 2006.

\bibitem[Lee(2026a)]{Lee26a}
P.~C.-K.~Lee.
\newblock {WLPO} equivalence of the bidual gap in $\ell^1$: a {Lean}~4
  formalization.
\newblock Preprint, 2026. Paper~2 in the constructive reverse
  mathematics series.

\bibitem[Lee(2026b)]{Lee26b}
P.~C.-K.~Lee.
\newblock Non-reflexivity of $S_1(H)$ implies {WLPO}: a {Lean}~4
  formalization.
\newblock Preprint, 2026. Paper~7 in the constructive reverse
  mathematics series.

\bibitem[Lee(2026c)]{Lee26c}
P.~C.-K.~Lee.
\newblock A logical geography of physical idealizations.
\newblock In preparation, 2026.

\bibitem[Mandelkern(1988)]{Man88}
M.~Mandelkern.
\newblock Limited omniscience and the {Bolzano--Weierstrass} principle.
\newblock \emph{Bulletin of the London Mathematical Society},
  20:319--320, 1988.

\bibitem[{Mathlib Community}(2020--)]{Mathlib2020}
{Mathlib Community}.
\newblock \emph{Mathlib}: the math library for {Lean}.
\newblock \url{https://leanprover-community.github.io/mathlib4_docs/},
  2020--present.

\bibitem[Menon and Callender(2013)]{MC13}
T.~Menon and C.~Callender.
\newblock Turn and face the strange\ldots{} Ch-ch-changes: Philosophical
  questions raised by phase transitions.
\newblock In R.~Batterman, editor,
  \emph{The Oxford Handbook of Philosophy of Physics}, pages 189--223.
  Oxford University Press, 2013.

\bibitem[Onsager(1944)]{Ons44}
L.~Onsager.
\newblock Crystal statistics. {I.} A two-dimensional model with an
  order-disorder transition.
\newblock \emph{Physical Review}, 65:117--149, 1944.

\bibitem[Ruelle(1999)]{Rue99}
D.~Ruelle.
\newblock \emph{Statistical Mechanics: Rigorous Results}.
\newblock Imperial College Press, London, 1999.
\newblock Reprint of the 1969 edition.

\bibitem[Shech(2013)]{She13}
E.~Shech.
\newblock What is the paradox of phase transitions?
\newblock \emph{Philosophy of Science}, 80(5):1170--1181, 2013.

\bibitem[Simpson(2009)]{Sim09}
S.~G.~Simpson.
\newblock \emph{Subsystems of Second Order Arithmetic}.
\newblock Cambridge University Press, 2nd edition, 2009.

\bibitem[van Wierst(2019)]{vW19}
P.~van~Wierst.
\newblock The paradox of phase transitions in the light of constructive
  mathematics.
\newblock \emph{Synthese}, 196(5):1863--1884, 2019.

\bibitem[Veldman(2005)]{Vel05}
W.~Veldman.
\newblock Brouwer's real thesis on bars.
\newblock \emph{Philosophia Scientiae, Cahier Sp\'ecial}, 6:21--42, 2005.

\end{thebibliography}

\end{document}

\documentclass[11pt,a4paper]{article}

% ====================================================================
% Packages
% ====================================================================
\usepackage[utf8]{inputenc}
\usepackage[T1]{fontenc}
\usepackage{amsmath,amssymb,amsthm}
\usepackage{mathtools}
\usepackage{hyperref}
\usepackage[margin=1in,top=0.9in,bottom=0.9in]{geometry}
\usepackage{enumitem}
\usepackage{booktabs}
\usepackage{listings}
\usepackage{xcolor}
\usepackage{cleveref}
\usepackage{natbib}
\usepackage{mdframed}
\usepackage{graphicx}

% Tighten float placement
\renewcommand{\floatpagefraction}{0.8}
\renewcommand{\topfraction}{0.9}
\renewcommand{\textfraction}{0.1}

% ====================================================================
% Code listing style (Python)
% ====================================================================
\definecolor{lean-keyword}{RGB}{0,0,180}
\definecolor{lean-comment}{RGB}{0,128,0}
\definecolor{lean-string}{RGB}{163,21,21}
\definecolor{lean-bg}{RGB}{248,248,248}

\lstdefinestyle{python}{
  language=Python,
  basicstyle=\ttfamily\small,
  keywordstyle=\color{lean-keyword}\bfseries,
  commentstyle=\color{lean-comment}\itshape,
  stringstyle=\color{lean-string},
  backgroundcolor=\color{lean-bg},
  frame=single,
  framerule=0.5pt,
  breaklines=true,
  breakatwhitespace=true,
  tabsize=4,
  showstringspaces=false,
  numbers=left,
  numberstyle=\tiny\color{gray},
  numbersep=5pt,
  xleftmargin=15pt,
  captionpos=b,
}
\lstset{style=python}

% ====================================================================
% Macros
% ====================================================================
\newcommand{\NN}{\mathbb{N}}
\newcommand{\RR}{\mathbb{R}}
\newcommand{\LPO}{\mathrm{LPO}}
\newcommand{\WLPO}{\mathrm{WLPO}}
\newcommand{\BMC}{\mathrm{BMC}}
\newcommand{\BISH}{\mathrm{BISH}}
\newcommand{\SM}{\mathrm{SM}}
\newcommand{\RG}{\mathrm{RG}}
\newcommand{\QFP}{\mathrm{QFP}}
\newcommand{\EW}{\mathrm{EW}}
\newcommand{\MZ}{M_Z}

% ====================================================================
% Title
% ====================================================================
\title{%
  \textbf{A $\BISH$-Complete Domain:\\[4pt]
  Yukawa Renormalization as a Finite Discrete Map}\\[6pt]
  {\normalsize Technical Note~18 in the Constructive Reverse Mathematics Series}%
}

\author{
  Paul Chun-Kit Lee\thanks{%
    New York University.
    AI-assisted numerical investigation; see the methodology
    statement for details.
    The author is a medical professional, not a domain expert in
    constructive mathematics, particle physics, or renormalization
    group theory; the numerical investigation and analysis were
    developed with extensive AI assistance.} \\
  New York University \\
  \texttt{dr.paul.c.lee@gmail.com}
}

\date{February 2026}

% ====================================================================
\begin{document}
\maketitle

% ====================================================================
\begin{abstract}
The Standard Model Yukawa beta functions, treated as a discrete
finite-step map rather than a continuous ODE, constitute a
computation entirely within Bishop's constructive mathematics
($\BISH$). We verify numerically that the top quark
quasi-fixed-point is visible at $N = 10$ discrete RK4 steps and
that the full fermion mass hierarchy is \emph{not} an attractor
of the one-loop Standard Model RG. This is the first domain in
the constructive reverse mathematics series where the entire
physically relevant computation is $\BISH$ with no $\LPO$
boundary, confirming that the $\BMC \leftrightarrow \LPO$
boundary observed in five other domains is a property of specific
physics---completed limits of bounded monotone sequences---not
a universal feature of mathematical physics.
\end{abstract}


% ====================================================================
\section{Introduction}\label{sec:intro}
% ====================================================================

The constructive reverse mathematics (CRM) programme calibrates
the logical cost of physical theories by identifying which
constructive principles are needed at each layer of the
mathematical description. Bishop's constructive mathematics
($\BISH$) requires every existential claim to come with a
construction; the Limited Principle of Omniscience ($\LPO$)
asserts that for any binary sequence, either some term equals~$1$
or all terms equal~$0$. $\LPO$ is equivalent to Bounded Monotone
Convergence ($\BMC$): every bounded monotone sequence converges
to a limit~\cite{BV06}.

Five independent physics domains exhibit the
$\BMC \leftrightarrow \LPO$ boundary: statistical mechanics
(Paper~8~\cite{Lee26-P8}), general relativity
(Paper~13~\cite{Lee26-P13}), quantum decoherence
(Paper~14~\cite{Lee26-P14}), conservation laws
(Paper~15~\cite{Lee26-P15}), and quantum gravity
(Paper~17~\cite{Lee26-P17}). In each case, finite computations
are $\BISH$, while the assertion that a bounded monotone sequence
converges to a completed real number costs $\LPO$. The working
hypothesis~\cite{Lee26-P10} is that empirical predictions are
$\BISH$-derivable and stronger principles enter only through
idealizations.

This pattern generates a prediction: in domains where no
completed limit is needed, the entire computation should be
$\BISH$, with no $\LPO$ boundary. The Standard Model
renormalization group beta functions, evaluated as a discrete
finite-step map, provide a natural test case. Every step is
finite arithmetic---polynomial evaluation and addition. No
sequence converges; no limit is taken. We ask whether this
$\BISH$ computation contains physically relevant structure, and
whether it might reveal mechanisms for the fermion mass hierarchy
that are invisible in the conventional continuous-flow formalism.


% ====================================================================
\section{The Computation}\label{sec:computation}
% ====================================================================

The one-loop beta functions for the third-generation Yukawa
couplings in the Standard Model
are~\cite{MV1984,LWX2003}:
\begin{align}
  16\pi^2 \frac{dy_t}{dt} &= y_t \left[
    \tfrac{9}{2} y_t^2 + \tfrac{3}{2} y_b^2 + y_\tau^2
    - 8 g_3^2 - \tfrac{9}{4} g_2^2
    - \tfrac{17}{12} g_1^2 \right],
    \label{eq:beta_yt} \\
  16\pi^2 \frac{dy_b}{dt} &= y_b \left[
    \tfrac{9}{2} y_b^2 + \tfrac{3}{2} y_t^2 + y_\tau^2
    - 8 g_3^2 - \tfrac{9}{4} g_2^2
    - \tfrac{5}{12} g_1^2 \right],
    \label{eq:beta_yb} \\
  16\pi^2 \frac{dy_\tau}{dt} &= y_\tau \left[
    \tfrac{5}{2} y_\tau^2 + 3 y_b^2 + 3 y_t^2
    - \tfrac{9}{4} g_2^2
    - \tfrac{15}{4} g_1^2 \right],
    \label{eq:beta_ytau}
\end{align}
where $t = \ln(\mu / \mu_0)$ and $g_1, g_2, g_3$ are the
$\mathrm{U}(1)_Y$, $\mathrm{SU}(2)_L$, $\mathrm{SU}(3)_C$
gauge couplings, running at one loop as
\begin{equation}\label{eq:beta_g}
  16\pi^2 \frac{dg_i}{dt} = b_i \, g_i^3,
  \qquad b_1 = \tfrac{41}{6},
  \quad b_2 = -\tfrac{19}{6},
  \quad b_3 = -7.
\end{equation}
Lighter-generation Yukawa couplings satisfy analogous equations
with the same gauge coefficients.

We use gauge couplings at $\MZ = 91.1876$~GeV in the
$\overline{\text{MS}}$ scheme ($g_1 = 0.3574$, $g_2 = 0.6518$,
$g_3 = 1.221$) and tree-level Yukawa couplings
$y_f = \sqrt{2}\, m_f / v$ with $v = 246.22$~GeV and PDG~2024
pole masses~\cite{PDG2024}. The scale range from $\MZ$ to the
Planck mass is $t_{\text{range}} \approx 39.4$.

The standard RG flow is a continuous ODE whose solution requires
computing an integral---a limit of Riemann sums---which in
general costs $\LPO$. The discrete map replaces the ODE with a
finite iteration: $N$ applications of a fourth-order Runge--Kutta
(RK4) step, each evaluating the beta
functions~\eqref{eq:beta_yt}--\eqref{eq:beta_g} at intermediate
points. Each step is finite arithmetic---$\BISH$ by definition.
For $N = 10$, the entire computation involves approximately $480$
floating-point multiplications and a comparable number of
additions.


% ====================================================================
\section{Results}\label{sec:results}
% ====================================================================

\subsection*{Top Quasi-Fixed-Point}

We scan $y_t(\text{Planck})$ over $[0.1, 10]$ with $N = 1{,}000$
RK4 steps. The Pendleton--Ross quasi-fixed-point
(\cref{fig:top_qfp}) is confirmed: $y_t(\EW)$ converges to
$\approx 1.29$ for all $y_t(\text{Planck}) \gtrsim 0.7$, a basin
encompassing $58\%$ of scanned initial conditions. The $30\%$
overshoot relative to the observed $y_t(\MZ) = 0.99$ is a known
artifact of one-loop running without threshold
corrections~\cite{PR1981,Hill1981}.

\begin{figure}[ht]
  \centering
  \includegraphics[width=0.65\textwidth]{plots/plot01_top_qfp.png}
  \caption{Top Yukawa at the EW scale versus initial value at the
    Planck scale ($N = 1{,}000$ RK4 steps). The curve flattens
    for $y_t(\text{Planck}) \gtrsim 0.7$, demonstrating the
    Pendleton--Ross quasi-fixed-point.}
  \label{fig:top_qfp}
\end{figure}


\subsection*{Discrete Map vs.\ Continuous Flow}

The QFP is genuinely finite-order structure.
\Cref{tab:convergence} shows that RK4 at $N = 50$ matches
scipy's adaptive integrator to four decimal places.
\Cref{fig:basin_N} shows the basin is already visible at
$N = 10$: the characteristic flattening for
$y_t(\text{Planck}) > 0.7$ is present with spread less than
$20\%$ of the mean across the plateau.

\begin{table}[ht]
\centering
\begin{tabular}{@{}rcc@{}}
\toprule
$N$ & Euler & RK4 \\
\midrule
10     & 0.4293 & 1.2871 \\
50     & 1.2390 & 1.2936 \\
100    & 1.2644 & 1.2936 \\
500    & 1.2874 & 1.2936 \\
1{,}000  & 1.2904 & 1.2936 \\
10{,}000 & 1.2932 & 1.2936 \\
\midrule
\multicolumn{2}{@{}l}{scipy (continuous)} & 1.2936 \\
\bottomrule
\end{tabular}
\caption{$y_t(\EW)$ for various step counts $N$ and integration
  methods. RK4 at $N = 50$ matches the continuous reference to
  four decimal places.}
\label{tab:convergence}
\end{table}

\begin{figure}[ht]
  \centering
  \includegraphics[width=0.65\textwidth]{plots/plot10_qfp_basin_N.png}
  \caption{Top QFP basin at $N = 10$, $100$, $1{,}000$, and
    $10{,}000$ discrete RK4 steps. The quasi-fixed-point
    structure is already visible at $N = 10$.}
  \label{fig:basin_N}
\end{figure}


\subsection*{Mass Hierarchy}

Of $3{,}000$ randomly sampled initial conditions (all Yukawa
couplings log-uniform on $[0.01, 10]$), none produce the
observed mass ratios within one order of magnitude
(\cref{fig:hierarchy}). The median RMS log-ratio error is
$3.38$~dex. The full fermion mass hierarchy is \emph{not} an
attractor of the one-loop SM RG: mass ratios are sensitive to
initial conditions.

\begin{figure}[ht]
  \centering
  \includegraphics[width=0.65\textwidth]{plots/plot04_hierarchy_scatter.png}
  \caption{Mass hierarchy at the EW scale for $3{,}000$ random
    Planck-scale initial conditions. The red star marks the
    observed values. No clustering near the observed point is
    visible.}
  \label{fig:hierarchy}
\end{figure}


\subsection*{Secondary Observations}

The bottom/tau mass ratio shows weak structure:
median $y_b/y_\tau = 2.25$ at the EW scale (observed~$2.35$),
but only $6\%$ of initial conditions fall within $20\%$ of the
observed value. The Koide ratio $Q$ yields a mean of $0.50$
(observed~$2/3$), with only $3.2\%$ of filtered samples within
$1\%$ of~$2/3$. Neither relation emerges generically from SM~RG
evolution. Two-loop corrections shift the QFP by $-0.65\%$
without producing qualitatively new structure.


% ====================================================================
\section{Discussion}\label{sec:discussion}
% ====================================================================

\subsection{CRM as Diagnostic, Not Generative}\label{sec:generative}

The investigation tested a stronger hypothesis than the one
confirmed: whether CRM thinking could be used
\emph{generatively}---not just to diagnose the logical cost of
known physics, but to reveal mechanisms for open problems. The
fermion mass hierarchy was the test case. The Standard Model has
13~free Yukawa parameters spanning six orders of magnitude; the
specific hope was that stripping the $\LPO$ scaffolding from the
RG (replacing the continuous flow with a finite discrete map)
would expose algebraic structure---quasi-fixed-points for the
full mass spectrum---invisible in the conventional formalism.

The hypothesis is refuted. The discrete map reveals the same
structure as the continuous flow: a top-quark attractor and
nothing else. The $\LPO$ content of the continuous formulation
(the completed flow as a limit of discrete steps) is
dispensable---the $\BISH$ content suffices---but the $\BISH$
content is not \emph{different}. The mass hierarchy requires
ultraviolet input that the SM's infrared dynamics do not
determine, and no logical reorganization changes this. CRM is an
excellent diagnostic tool; it is not a physics generator.

\subsection{The BISH-Only Domain}\label{sec:bish_only}

What the investigation does provide is the first domain in the
CRM series where everything is $\BISH$ and no $\LPO$ boundary
appears. The five domains in Papers~8--17 all involve bounded
monotone sequences whose completed limits cost $\LPO$. The
Yukawa RG involves no such sequence: evaluate the beta function
a finite number of times, obtain the coupling at the electroweak
scale. The ``flow'' need not \emph{converge}; one only needs the
output at the target scale.

The discrimination is structural (\cref{tab:calibration}):
$\LPO$ appears when physicists assert the existence of a
completed limit, and does not appear when the physical prediction
requires only a finite computation. This confirms that the
$\BMC \leftrightarrow \LPO$ boundary is not an artifact of the
formalization method but a genuine property of specific
physics---thermodynamic limits, geodesic incompleteness, exact
decoherence, global conservation, and entropy density convergence
all require completed limits; the RG discrete map does not.

\begin{table}[ht]
\centering
\small
\begin{tabular}{@{}llll@{}}
\toprule
\textbf{Domain} & \textbf{Paper} & \textbf{$\BISH$ Content}
  & \textbf{$\LPO$ Content} \\
\midrule
Statistical Mechanics & 8 & Finite-volume free energy
  & Thermodynamic limit \\
General Relativity & 13 & Finite-time geodesic
  & Geodesic incompleteness \\
Quantum Measurement & 14 & Finite-step decoherence
  & Exact decoherence \\
Conservation Laws & 15 & Local energy conservation
  & Global energy \\
Quantum Gravity & 17 & Finite entropy count
  & Entropy density limit \\
\textbf{Particle Physics (RG)} & \textbf{18}
  & \textbf{Finite-step Yukawa evolution}
  & \textbf{None} \\
\bottomrule
\end{tabular}
\caption{Updated CRM calibration table. The sixth row---all
  $\BISH$, no $\LPO$---completes the diagnostic picture.}
\label{tab:calibration}
\end{table}


\subsection{CRM Analysis of Mass-Problem Approaches}
\label{sec:approaches}

Applied to the fermion mass problem directly, CRM reveals a
subtlety: every existing approach operates within $\BISH$
(\cref{tab:approaches}). Flavor symmetries replace
13~unexplained Yukawa couplings with 8--12 unexplained flavon
parameters---$\BISH$ reorganization, not derivation.
Randall--Sundrum models achieve better compression (mild
$O(1)$~spread in bulk masses produces exponential hierarchy via
warped geometry) but still require unexplained inputs. Radiative
mass generation offers the best compression ratio: one universal
Yukawa coupling plus perturbative loop suppression generates six
orders of magnitude, though the required new particles have not
been observed. The Koide formula $Q = 2/3$, satisfied to
${\sim}10^{-5}$ precision, is empirically $\BISH$ (checking the
relation is finite arithmetic), but Sumino's all-orders
cancellation mechanism requires $\LPO$~\cite{Sumino2009}---the
only approach whose \emph{explanation} costs more than
$\BISH$~\cite{Koide1983}.

CRM does not discriminate these approaches by logical
cost---they are all $\BISH$. What it reveals is that they differ
in \emph{compression ratio}: the number of unexplained parameters
needed to produce 13~observables. The mass problem is entirely a
problem within $\BISH$---a different kind of mystery from the
ones CRM was designed to illuminate.

\begin{table}[ht]
\centering
\small
\begin{tabular}{@{}lcl@{}}
\toprule
\textbf{Approach} & \textbf{Logical Cost}
  & \textbf{Unexplained Inputs} \\
\midrule
Standard Model (raw) & $\BISH$ & 13 Yukawa couplings \\
Flavor symmetries & $\BISH$ & 8--12 flavon parameters \\
Randall--Sundrum & $\BISH$
  & ${\sim}6$ bulk masses + geometry \\
Radiative generation & $\BISH$
  & ${\sim}1$ universal Yukawa \\
Koide (empirical) & $\BISH$ & 2 parameters ($\mu, \delta$) \\
Koide (Sumino mechanism) & $\LPO$
  & 0 (if mechanism works) \\
Landscape & $\BISH$ (formally) & 0 (but no prediction) \\
\bottomrule
\end{tabular}
\caption{CRM calibration of approaches to the fermion mass
  hierarchy. All approaches except Sumino's all-orders mechanism
  operate within $\BISH$; they differ in compression ratio.}
\label{tab:approaches}
\end{table}


\paragraph{Limitations.}
The QFP value $y_t(\EW) \approx 1.29$ exceeds the observed
$0.99$ by ${\sim}30\%$---a known limitation of one-loop running
without threshold corrections. CKM mixing and complete two-loop
corrections are neglected but do not affect qualitative
conclusions. Unlike Papers~8--17, this investigation is a
numerical experiment, not a formal verification; the statement
that the discrete map is $\BISH$ is trivially true of any finite
arithmetic.


% ====================================================================
\section{Conclusion}\label{sec:conclusion}
% ====================================================================

The Standard Model Yukawa renormalization group, treated as a
discrete finite-step map, is entirely $\BISH$. The top
quasi-fixed-point---the only attractor structure in the
system---is visible at $N = 10$ discrete RK4 steps. The full
fermion mass hierarchy is not an attractor; mass ratios are
sensitive to ultraviolet initial conditions. This is the first
domain in the CRM series with no $\LPO$ boundary. Its existence
confirms that $\LPO$ enters physics through completed limits of
bounded monotone sequences, and nowhere else.

The programme archive is maintained at Zenodo (DOI:
\href{https://doi.org/10.5281/zenodo.18600243}{10.5281/zenodo.18600243}).


% ====================================================================
\section*{AI-Assisted Methodology}\label{sec:ai}
% ====================================================================

This investigation was developed using \textbf{Claude Opus~4.6}
(Anthropic, 2026) via the \textbf{Claude Code} command-line
interface~\cite{Anthropic2026}, following the same human--AI
workflow as Papers~2--17. The author specified the research
direction and CRM framing; Claude~Opus~4.6 implemented the beta
functions, numerical scanning code, and plots. There is no formal
verification component.

\begin{table}[h]
\centering
\small
\begin{tabular}{@{}lll@{}}
\toprule
\textbf{Task} & \textbf{Human} & \textbf{AI (Claude Opus 4.6)} \\
\midrule
Research direction       & \checkmark & \\
CRM framing              & \checkmark & \\
Beta function code       & & \checkmark \\
Numerical scans          & & \checkmark \\
Result interpretation    & \checkmark & \checkmark \\
Paper writing            & \checkmark & \checkmark \\
\bottomrule
\end{tabular}
\end{table}


% ====================================================================
\section*{Reproducibility}
% ====================================================================

\begin{mdframed}[backgroundcolor=gray!10]
\textbf{Reproducibility Box}
\begin{itemize}
\item \textbf{Repository}:
  \url{https://github.com/AICardiologist/FoundationRelativity}
\item \textbf{Script}: \texttt{paper~18/rg\_mass\_hierarchy.py}
  (${\sim}600$ lines)
\item \textbf{Dependencies}: Python~3.9+, NumPy, SciPy,
  Matplotlib
\item \textbf{Run}: \texttt{python3 rg\_mass\_hierarchy.py}
  (${\sim}16$~min on a 2020 MacBook)
\item \textbf{Output}: 10~plots in \texttt{plots/}, console
  summary
\item \textbf{Zenodo DOI}:
  \href{https://doi.org/10.5281/zenodo.18600243}{10.5281/zenodo.18600243}
\end{itemize}
\end{mdframed}


% ====================================================================
\section*{Acknowledgments}
% ====================================================================

The numerical investigation was developed using Claude Opus~4.6
(Anthropic, 2026) via the Claude Code CLI tool.


% ====================================================================
% Bibliography
% ====================================================================
\bibliographystyle{plainnat}

\begin{thebibliography}{20}

\bibitem[Anthropic(2026)]{Anthropic2026}
Anthropic.
\newblock Claude Opus~4.6, 2026.
\newblock \url{https://www.anthropic.com}

\bibitem[Bishop(1967)]{Bishop1967}
E.~Bishop.
\newblock \emph{Foundations of Constructive Analysis}.
\newblock McGraw-Hill, New York, 1967.

\bibitem[Bridges and V\^{\i}\c{t}\u{a}(2006)]{BV06}
D.~Bridges and L.~V\^{\i}\c{t}\u{a}.
\newblock \emph{Techniques of Constructive Analysis}.
\newblock Springer, 2006.

\bibitem[Hill(1981)]{Hill1981}
C.~T.~Hill.
\newblock Quark and lepton masses from renormalization-group
  fixed points.
\newblock \emph{Physical Review D}, 24:691--703, 1981.

\bibitem[Koide(1983)]{Koide1983}
Y.~Koide.
\newblock New view of quark and lepton mass hierarchy.
\newblock \emph{Physical Review D}, 28:252--254, 1983.

\bibitem[Lee(2026a)]{Lee26-P8}
P.~C.-K.~Lee.
\newblock Axiom calibration of the 1D Ising model:
  $\LPO$ dispensability.
\newblock Paper~8 in the CRM Series, 2026.

\bibitem[Lee(2026b)]{Lee26-P10}
P.~C.-K.~Lee.
\newblock The logical geography of mathematical physics.
\newblock Paper~10 in the CRM Series, 2026.

\bibitem[Lee(2026c)]{Lee26-P13}
P.~C.-K.~Lee.
\newblock Axiom calibration of Schwarzschild geodesics.
\newblock Paper~13 in the CRM Series, 2026.

\bibitem[Lee(2026d)]{Lee26-P14}
P.~C.-K.~Lee.
\newblock Axiom calibration of quantum decoherence.
\newblock Paper~14 in the CRM Series, 2026.

\bibitem[Lee(2026e)]{Lee26-P15}
P.~C.-K.~Lee.
\newblock Axiom calibration of Noether's theorem.
\newblock Paper~15 in the CRM Series, 2026.

\bibitem[Lee(2026f)]{Lee26-P17}
P.~C.-K.~Lee.
\newblock Axiom calibration of black hole entropy.
\newblock Paper~17 in the CRM Series, 2026.

\bibitem[Luo et~al.(2003)]{LWX2003}
M.-x.~Luo, H.-w.~Wang, and Y.~Xiao.
\newblock Two-loop renormalization group equations in the
  Standard Model.
\newblock \emph{Physical Review D}, 67:065019, 2003.
\newblock arXiv:hep-ph/0211440.

\bibitem[Machacek and Vaughn(1984)]{MV1984}
M.~E.~Machacek and M.~T.~Vaughn.
\newblock Two-loop renormalization group equations in a general
  quantum field theory: III.\ Scalar quartic couplings.
\newblock \emph{Nuclear Physics B}, 249:70--92, 1984.

\bibitem[Particle Data Group(2024)]{PDG2024}
Particle Data Group.
\newblock Review of Particle Physics.
\newblock \emph{Physical Review D}, 110:030001, 2024.

\bibitem[Pendleton and Ross(1981)]{PR1981}
B.~Pendleton and G.~G.~Ross.
\newblock Mass and mixing angle predictions from infrared
  fixed points.
\newblock \emph{Physics Letters B}, 98:291--294, 1981.

\bibitem[Randall and Sundrum(1999)]{RS1999}
L.~Randall and R.~Sundrum.
\newblock Large mass hierarchy from a small extra dimension.
\newblock \emph{Physical Review Letters}, 83:3370--3373, 1999.

\bibitem[Sumino(2009)]{Sumino2009}
Y.~Sumino.
\newblock Family gauge symmetry as an origin of Koide's mass
  formula and charged lepton spectrum.
\newblock \emph{Journal of High Energy Physics}, 2009(05):075,
  2009.

\end{thebibliography}

\end{document}

\documentclass[11pt,a4paper]{article}

% ====================================================================
% Packages
% ====================================================================
\usepackage[utf8]{inputenc}
\usepackage[T1]{fontenc}
\usepackage{amsmath,amssymb,amsthm}
\usepackage{mathtools}
\usepackage{hyperref}
\usepackage[margin=1in,top=0.9in,bottom=0.9in]{geometry}
\usepackage{enumitem}
\usepackage{booktabs}
\usepackage{listings}
\usepackage{xcolor}
\usepackage{cleveref}
\usepackage{natbib}
\usepackage{mdframed}
\usepackage{graphicx}

% Tighten float placement
\renewcommand{\floatpagefraction}{0.8}
\renewcommand{\topfraction}{0.9}
\renewcommand{\textfraction}{0.1}

% ====================================================================
% Code listing style (Python)
% ====================================================================
\definecolor{lean-keyword}{RGB}{0,0,180}
\definecolor{lean-comment}{RGB}{0,128,0}
\definecolor{lean-string}{RGB}{163,21,21}
\definecolor{lean-bg}{RGB}{248,248,248}

\lstdefinestyle{python}{
  language=Python,
  basicstyle=\ttfamily\small,
  keywordstyle=\color{lean-keyword}\bfseries,
  commentstyle=\color{lean-comment}\itshape,
  stringstyle=\color{lean-string},
  backgroundcolor=\color{lean-bg},
  frame=single,
  framerule=0.5pt,
  breaklines=true,
  breakatwhitespace=true,
  tabsize=4,
  showstringspaces=false,
  numbers=left,
  numberstyle=\tiny\color{gray},
  numbersep=5pt,
  xleftmargin=15pt,
  captionpos=b,
}
\lstset{style=python}

% Lean 4 code listing language
\lstdefinelanguage{lean4}{
  keywords={theorem,lemma,def,class,instance,import,open,variable,
            noncomputable,section,namespace,end,where,let,have,show,
            intro,obtain,use,exact,rw,simp,apply,by,fun,match,if,
            then,else,do,return,axiom,abbrev,private,attribute,
            suffices,change,congr,ext,constructor,rintro,push_neg,
            linarith,absurd,set_option,omit,in,set,cases,rcases,
            calc,nlinarith,positivity},
  sensitive=true,
  morecomment=[l]{--},
  morecomment=[s]{/-}{-/},
  morestring=[b]",
  morestring=[b]',
}
\lstdefinestyle{lean4}{
  language=lean4,
  basicstyle=\ttfamily\small,
  keywordstyle=\color{lean-keyword}\bfseries,
  commentstyle=\color{lean-comment}\itshape,
  stringstyle=\color{lean-string},
  backgroundcolor=\color{lean-bg},
  frame=single,
  framerule=0.5pt,
  breaklines=true,
  breakatwhitespace=true,
  tabsize=2,
  showstringspaces=false,
  numbers=left,
  numberstyle=\tiny\color{gray},
  numbersep=5pt,
  xleftmargin=15pt,
  captionpos=b,
}

% ====================================================================
% Macros
% ====================================================================
\newcommand{\NN}{\mathbb{N}}
\newcommand{\RR}{\mathbb{R}}
\newcommand{\QQ}{\mathbb{Q}}
\newcommand{\LPO}{\mathrm{LPO}}
\newcommand{\WLPO}{\mathrm{WLPO}}
\newcommand{\BMC}{\mathrm{BMC}}
\newcommand{\BISH}{\mathrm{BISH}}
\newcommand{\SM}{\mathrm{SM}}
\newcommand{\RG}{\mathrm{RG}}
\newcommand{\QFP}{\mathrm{QFP}}
\newcommand{\EW}{\mathrm{EW}}
\newcommand{\MZ}{M_Z}
\newcommand{\Lean}{\textsc{Lean~4}}
\newcommand{\Mathlib}{\textsc{Mathlib4}}
\newcommand{\leanok}{\textsf{\small\textcolor{green!70!black}{\checkmark}}}
\newcommand{\leanpartial}{\textsf{\small\textcolor{orange!80!black}{(partial)}}}

% Theorem environments
\theoremstyle{plain}
\newtheorem{theorem}{Theorem}[section]
\newtheorem{lemma}[theorem]{Lemma}
\newtheorem{proposition}[theorem]{Proposition}
\newtheorem{corollary}[theorem]{Corollary}
\theoremstyle{definition}
\newtheorem{definition}[theorem]{Definition}
\newtheorem{remark}[theorem]{Remark}

% ====================================================================
% Title
% ====================================================================
\title{%
  \textbf{Constructive Stratification of the Standard Model Yukawa RG:\\[4pt]
  Numerical Tests and Lean~4 Formalization}\\[6pt]
  {\normalsize Technical Note~18 in the Constructive Reverse Mathematics Series}%
}

\author{
  Paul Chun-Kit Lee\thanks{%
    New York University.
    AI-assisted numerical investigation; see the methodology
    statement for details.
    The author is a medical professional, not a domain expert in
    constructive mathematics, particle physics, or renormalization
    group theory; the numerical investigation and analysis were
    developed with extensive AI assistance.} \\
  New York University \\
  \texttt{dr.paul.c.lee@gmail.com}
}

\date{February 2026}

% ====================================================================
\begin{document}
\maketitle

% ====================================================================
\begin{abstract}
The constructive reverse mathematics (CRM) programme has established,
across five physics domains, that the passage from finite computation
($\BISH$) to completed infinite limit costs $\LPO$ via Bounded
Monotone Convergence. This pattern suggests a \emph{scaffolding
principle}: when physicists use $\LPO$-level idealizations, the
idealizations may constrain explanations, and removing them may
reveal $\BISH$-level mechanisms invisible in the conventional
formalism. We test this principle on the fermion mass hierarchy
through ten numerical investigations (Phases~A--B) and a \Lean{}
formalization (Phase~C).

All ten numerical investigations yield negative results: the Standard
Model's infrared dynamics do not determine the mass hierarchy in any
parameterization, at any loop order, or at any discretization scale
tested. CRM is a powerful diagnostic framework; its generative
capacity, at least for the flavor problem, is null.

The \Lean{} formalization (${\sim}900$~lines, five theorems verified
by \Mathlib{}) establishes a sharp constructive stratification:
\[
  \BISH \;<\; \WLPO \text{ (thresholds)}
  \;<\; \LPO \text{ (eigenvalue crossings)}
  \;<\; \text{full Classical.}
\]
The one-loop Yukawa RG flow is $\BISH$: polynomial Picard iteration
preserves $\QQ[t]$ at every finite step (Theorem~1) and converges
with an explicit Cauchy modulus (Theorem~2). Step-function
thresholds cost $\WLPO$ (Theorem~5), and CKM eigenvalue-crossing
detection costs $\LPO$ (Theorem~4). The physical mechanisms of the
SM---polynomial beta functions, smooth threshold matching,
diagonalization with mass gaps---are uniformly $\BISH$; omniscience
enters only through textbook idealizations that can be replaced by
constructive alternatives.
\end{abstract}


% ====================================================================
\section{Introduction}\label{sec:intro}
% ====================================================================

The constructive reverse mathematics (CRM) programme calibrates
the logical cost of physical theories by identifying which
constructive principles are needed at each layer of the
mathematical description. Bishop's constructive mathematics
($\BISH$) requires every existential claim to come with a
construction; the Limited Principle of Omniscience ($\LPO$)
asserts that for any binary sequence, either some term equals~$1$
or all terms equal~$0$. $\LPO$ is equivalent to Bounded Monotone
Convergence ($\BMC$): every bounded monotone sequence converges
to a limit~\cite{BV06}.

Five independent physics domains exhibit the
$\BMC \leftrightarrow \LPO$ boundary: statistical mechanics
(Paper~8~\cite{Lee26-P8}), general relativity
(Paper~13~\cite{Lee26-P13}), quantum decoherence
(Paper~14~\cite{Lee26-P14}), conservation laws
(Paper~15~\cite{Lee26-P15}), and quantum gravity
(Paper~17~\cite{Lee26-P17}). In each case, finite computations
are $\BISH$, while the assertion that a bounded monotone sequence
converges to a completed real number costs $\LPO$. The working
hypothesis~\cite{Lee26-P10} is that empirical predictions are
$\BISH$-derivable and stronger principles enter only through
idealizations.

This pattern generates two predictions. First, in domains where no
completed limit is needed, the entire computation should be $\BISH$,
with no $\LPO$ boundary. Second---and more ambitiously---the
distinction between $\BISH$ and $\LPO$ might be
\emph{generative}: stripping $\LPO$ scaffolding from physics
might reveal mechanisms invisible in the conventional formalism.

We test both predictions on the Standard Model Yukawa
renormalization group. The beta functions, evaluated as a discrete
finite-step map, provide a domain whose core computation is $\BISH$
for the first prediction. For the second, the fermion mass
hierarchy---thirteen free Yukawa parameters spanning six orders of
magnitude with no known explanation---provides a test case where
new mechanisms would be physically significant. We report ten
numerical investigations across two phases, systematically testing
whether the mass hierarchy emerges from the SM's infrared dynamics
when analyzed through the CRM lens.


% ====================================================================
\section{The Scaffolding Hypothesis}\label{sec:scaffolding}
% ====================================================================

\subsection{LPO Idealizations as Architectural Constraints}

When a physicist invokes an $\LPO$-level idealization---an exact
symmetry, an all-orders perturbative result, a continuous
limit---the idealization does not merely extend a result to infinite
precision. It \emph{constrains the architecture of the explanation}.
Only mechanisms compatible with the idealization are considered;
mechanisms that achieve the same finite-precision result by other
means are excluded from the search space.

\subsection{The Sumino Example}

The observation that motivated this investigation comes from Koide's
charged lepton mass formula~\cite{Koide1983}:
\begin{equation}\label{eq:koide}
  Q = \frac{m_e + m_\mu + m_\tau}
    {(\sqrt{m_e} + \sqrt{m_\mu} + \sqrt{m_\tau})^2}
    = \frac{2}{3},
\end{equation}
satisfied to ${\sim}10^{-5}$ precision. Sumino~\cite{Sumino2009}
explains this by postulating a $\mathrm{U}(3) \times \mathrm{SU}(2)$
family gauge symmetry whose radiative corrections cancel QED
corrections to \emph{all orders} in perturbation theory, preserving
$Q = \text{exactly}~2/3$ at every energy scale. This all-orders
cancellation is an $\LPO$ statement---a completed infinite assertion
that cannot be verified at finitely many loop orders.

The CRM-informed question is different: \emph{why is $Q = 2/3$ to
$10^{-5}$ precision at the loop orders we can compute?} This is a
$\BISH$ question. It does not require an exact gauge symmetry or
all-orders cancellation. It admits a larger space of potential
mechanisms---finite-order algebraic structures, approximate
cancellations, dynamical attractors in the RG flow---that are
excluded by the demand for $\LPO$-level exactness.

\subsection{The General Principle}

More broadly: the standard treatment of the renormalization group
as a continuous flow (an ODE with solutions defined by limits of
Riemann sums) is an $\LPO$ idealization. The $\BISH$ content is
the discrete map at finite step count. Physicists search for exact
infrared fixed points of the continuous flow. But the physically
relevant question may be: what structure exists in the \emph{finite
discrete map}? Fixed points of the continuous flow are necessarily
fixed points of the discrete map, but the converse need not hold.
The discrete map could exhibit quasi-fixed-point structure, ratio-space
attractors, or threshold-driven self-consistency conditions invisible
in the continuous limit.

We call this the \textbf{scaffolding principle}: $\LPO$ idealizations
serve as scaffolding for physical explanations; removing them may
reveal the $\BISH$ structure underneath, which may differ from what
the scaffolding suggests. The fermion mass hierarchy provides a
concrete test.


% ====================================================================
\section{The Standard Model Yukawa System}\label{sec:computation}
% ====================================================================

The one-loop beta functions for the third-generation Yukawa
couplings in the Standard Model
are~\cite{MV1984,LWX2003}:
\begin{align}
  16\pi^2 \frac{dy_t}{dt} &= y_t \left[
    \tfrac{9}{2} y_t^2 + \tfrac{3}{2} y_b^2 + y_\tau^2
    - 8 g_3^2 - \tfrac{9}{4} g_2^2
    - \tfrac{17}{12} g_1^2 \right],
    \label{eq:beta_yt} \\
  16\pi^2 \frac{dy_b}{dt} &= y_b \left[
    \tfrac{9}{2} y_b^2 + \tfrac{3}{2} y_t^2 + y_\tau^2
    - 8 g_3^2 - \tfrac{9}{4} g_2^2
    - \tfrac{5}{12} g_1^2 \right],
    \label{eq:beta_yb} \\
  16\pi^2 \frac{dy_\tau}{dt} &= y_\tau \left[
    \tfrac{5}{2} y_\tau^2 + 3 y_b^2 + 3 y_t^2
    - \tfrac{9}{4} g_2^2
    - \tfrac{15}{4} g_1^2 \right],
    \label{eq:beta_ytau}
\end{align}
where $t = \ln(\mu / \mu_0)$ and $g_1, g_2, g_3$ are the
$\mathrm{U}(1)_Y$, $\mathrm{SU}(2)_L$, $\mathrm{SU}(3)_C$
gauge couplings\footnote{We use the non-GUT-normalized hypercharge
coupling $g_1 = g_Y$, giving $b_1 = 41/6$. Many references use
the GUT-normalized $g_1 = \sqrt{5/3}\,g_Y$, which changes the
numerical coefficients. Our convention follows
Ref.~\cite{MV1984}.}, running at one loop as
\begin{equation}\label{eq:beta_g}
  16\pi^2 \frac{dg_i}{dt} = b_i \, g_i^3,
  \qquad b_1 = \tfrac{41}{6},
  \quad b_2 = -\tfrac{19}{6},
  \quad b_3 = -7.
\end{equation}
Lighter-generation Yukawa couplings satisfy analogous equations
with the same gauge coefficients.

\paragraph{Scope of the implemented system.}
In both phases we evolve \emph{diagonal} Yukawa couplings: nine
independent $y_f$ ($f = t, b, \tau, c, s, \mu, u, d, e$) coupled
through the gauge couplings but without CKM mixing or off-diagonal
Yukawa-matrix structure. At one loop in this diagonal
approximation, inter-generation coupling enters only through the
gauge beta functions (which depend on the total number of active
flavors, not on individual Yukawa values). The full one-loop
Yukawa-matrix RGE includes trace terms
$\mathrm{Tr}(Y_u^\dagger Y_u)$ that couple all up-type Yukawas;
these are dominated by $y_t^2$ and are numerically negligible for
lighter generations. At two loops, CKM mixing enters explicitly
and is neglected here. The negative results reported below should
be read as ``no attractor structure exists in the diagonal Yukawa
system''; whether the full matrix system with CKM structure
behaves differently is an open question, though we regard it as
unlikely given that CKM effects are perturbatively small
corrections to the diagonal system.

We use gauge couplings at $\MZ = 91.1876$~GeV in the
$\overline{\text{MS}}$ scheme ($g_1 = 0.3574$, $g_2 = 0.6518$,
$g_3 = 1.221$) and tree-level Yukawa couplings
$y_f = \sqrt{2}\, m_f / v$ with $v = 246.22$~GeV and PDG~2024
pole masses~\cite{PDG2024}. The scale range from $\MZ$ to the
Planck mass is $t_{\text{range}} \approx 39.4$.

The standard RG flow is a continuous ODE. For predictions between
two finite energy scales, the solution can be approximated to any
desired precision by a finite number of integration steps---this
is $\BISH$ computation. $\LPO$ enters only when physicists assert
a \emph{completed} object: an exact fixed point (not an
approximate one), an exact all-orders cancellation (not a
finite-order one), or an exact converged value without specifying
a rate of convergence. The discrete map we study is not an
approximation to an $\LPO$-level object; it \emph{is} the
physically relevant computation, and the continuous flow is a
convenient idealization that happens to cost no additional logical
strength for finite-scale predictions. This is why the core
domain is $\BISH$: no completed limit does physical work. (Textbook
idealizations---step-function thresholds, exact eigenvalue-crossing
detection---introduce $\WLPO$ and $\LPO$; see
\cref{sec:lean_formalization}.)

The discrete map implements $N$ applications of a fourth-order
Runge--Kutta (RK4) step, each evaluating the beta
functions~\eqref{eq:beta_yt}--\eqref{eq:beta_g} at intermediate
points. Each step is finite arithmetic---$\BISH$ by definition.
For $N = 10$, the entire computation involves approximately $480$
floating-point multiplications and a comparable number of
additions.

Phase~2 additionally uses two-loop gauge beta functions from
Luo, Wang, and Xiao~\cite{LWX2003}, with coefficients
$b_{ij}^{(2)}$ that include Yukawa-dependent terms. The two-loop
correction captures the dominant next-order effect on running
couplings.


\subsection{Constructive Frameworks}\label{sec:crm_prelim}

The constructive stratification established in
\cref{sec:lean_formalization} uses three omniscience principles of
increasing logical strength:

\begin{definition}[$\LPO$]\label{def:lpo}
The \emph{Limited Principle of Omniscience}: for every binary
sequence $\alpha : \NN \to \{0,1\}$, either
$\forall n,\;\alpha(n) = 0$ or $\exists n,\;\alpha(n) = 1$.
\end{definition}

\begin{definition}[$\WLPO$]\label{def:wlpo}
The \emph{Weak Limited Principle of Omniscience}: for every binary
sequence $\alpha : \NN \to \{0,1\}$, either
$\forall n,\;\alpha(n) = 0$ or $\lnot\forall n,\;\alpha(n) = 0$.
\end{definition}

\noindent
Both are classically trivial (decidability of the universal
quantifier over $\NN$) but constructively independent of $\BISH$:
no finite search of an infinite sequence can decide these. The
hierarchy is
\[
  \BISH \;\subsetneq\; \WLPO \;\subsetneq\; \LPO
  \;\subsetneq\; \text{full Classical.}
\]
The \Lean{} formalization encodes these as propositions over
\texttt{Bool}-valued sequences:
\begin{lstlisting}[style=lean4,caption={Omniscience principles in \Lean{}.}]
def LPO_P18 : Prop :=
  forall (a : Nat -> Bool),
    (forall n, a n = false) ||| (exists n, a n = true)

def WLPO : Prop :=
  forall (a : Nat -> Bool),
    (forall n, a n = false) ||| ~(forall n, a n = false)
\end{lstlisting}


\subsection{CKM Matrix and Threshold Decoupling}\label{sec:ckm_prelim}

Two standard textbook operations on the Yukawa system
introduce omniscience beyond $\BISH$:
\begin{enumerate}[nosep]
  \item \textbf{Threshold decoupling.} Heavy particles are
    decoupled at their mass thresholds via the Heaviside step
    function $\theta(\mu - m_f)$. This requires deciding the sign
    of a constructive real (see \cref{sec:lean_wlpo}).
  \item \textbf{CKM diagonalization.} The CKM matrix is obtained
    by diagonalizing Yukawa coupling matrices. At eigenvalue
    crossings (mass degeneracies), the eigenvector basis becomes
    discontinuous. Deciding whether eigenvalues are equal or
    distinct requires $\LPO$ (see \cref{sec:lean_lpo}).
\end{enumerate}
Both operations are $\BISH$ when replaced by their physical
counterparts (smooth threshold matching, gapped diagonalization at
the observed non-degenerate SM masses).


% ====================================================================
\section{Phase~1: One-Loop Discrete Map}\label{sec:phase1}
% ====================================================================

Phase~1 tests the narrowest version of the scaffolding
hypothesis: whether the one-loop SM beta functions, treated as a
discrete RK4 map with small step size, contain attractor structure
that produces the mass hierarchy from generic initial conditions.
Five questions were investigated.


\subsection*{Top Quasi-Fixed-Point}

We scan $y_t(\text{Planck})$ over $[0.1, 10]$ with $N = 1{,}000$
RK4 steps. The Pendleton--Ross quasi-fixed-point
(\cref{fig:top_qfp}) is confirmed: $y_t(\EW)$ converges to
$\approx 1.29$ for all $y_t(\text{Planck}) \gtrsim 0.7$, a basin
encompassing $58\%$ of scanned initial conditions. The $30\%$
overshoot relative to the observed $y_t(\MZ) = 0.99$ is a known
artifact of one-loop running without threshold
corrections~\cite{PR1981,Hill1981}.

\begin{figure}[ht]
  \centering
  \includegraphics[width=0.65\textwidth]{plots_v2/plot01_top_qfp.png}
  \caption{Top Yukawa at the EW scale versus initial value at the
    Planck scale ($N = 1{,}000$ RK4 steps). The curve flattens
    for $y_t(\text{Planck}) \gtrsim 0.7$, demonstrating the
    Pendleton--Ross quasi-fixed-point.}
  \label{fig:top_qfp}
\end{figure}


\subsection*{Discrete Map vs.\ Continuous Flow}

The QFP is genuinely finite-order structure.
\Cref{tab:convergence} shows that RK4 at $N = 50$ matches
scipy's adaptive integrator to four decimal places.
\Cref{fig:basin_N} shows the basin is already visible at
$N = 10$: the characteristic flattening for
$y_t(\text{Planck}) > 0.7$ is present with spread less than
$20\%$ of the mean across the plateau.

\begin{table}[ht]
\centering
\begin{tabular}{@{}rcc@{}}
\toprule
$N$ & Euler & RK4 \\
\midrule
10     & 0.4293 & 1.2871 \\
50     & 1.2390 & 1.2936 \\
100    & 1.2644 & 1.2936 \\
500    & 1.2874 & 1.2936 \\
1{,}000  & 1.2904 & 1.2936 \\
10{,}000 & 1.2932 & 1.2936 \\
\midrule
\multicolumn{2}{@{}l}{scipy (continuous)} & 1.2936 \\
\bottomrule
\end{tabular}
\caption{$y_t(\EW)$ for various step counts $N$ and integration
  methods. RK4 at $N = 50$ matches the continuous reference to
  four decimal places.}
\label{tab:convergence}
\end{table}

\begin{figure}[ht]
  \centering
  \includegraphics[width=0.65\textwidth]{plots_v2/plot10_qfp_basin_N.png}
  \caption{Top QFP basin at $N = 10$, $100$, $1{,}000$, and
    $10{,}000$ discrete RK4 steps. The quasi-fixed-point
    structure is already visible at $N = 10$.}
  \label{fig:basin_N}
\end{figure}


\subsection*{Mass Hierarchy}

Of $3{,}000$ randomly sampled initial conditions (all Yukawa
couplings log-uniform on $[0.01, 10]$), none produce the
observed mass ratios within one order of magnitude
(\cref{fig:hierarchy}). The median RMS log-ratio error is
$3.38$~dex. The full fermion mass hierarchy is \emph{not} an
attractor of the one-loop SM RG: mass ratios are sensitive to
initial conditions.

\begin{figure}[ht]
  \centering
  \includegraphics[width=0.65\textwidth]{plots_v2/plot04_hierarchy_scatter.png}
  \caption{Mass hierarchy at the EW scale for $3{,}000$ random
    Planck-scale initial conditions. The red star marks the
    observed values. No clustering near the observed point is
    visible.}
  \label{fig:hierarchy}
\end{figure}


\subsection*{Secondary Observations}

The bottom/tau mass ratio shows weak structure:
median $y_b/y_\tau = 2.25$ at the EW scale (observed~$2.35$),
but only $6\%$ of initial conditions fall within $20\%$ of the
observed value. The Koide ratio $Q$ yields a mean of $0.50$
(observed~$2/3$), with only $3.2\%$ of filtered samples within
$1\%$ of~$2/3$. Neither relation emerges generically from SM~RG
evolution. Two-loop corrections shift the QFP by $-0.65\%$
without producing qualitatively new structure.

\paragraph{Phase~1 assessment.}
One of five success criteria is met: the top QFP is a robust
$\BISH$ structure visible at coarse discretization. The remaining
four are negative. However, Phase~1 tested only a narrow special
case of the scaffolding hypothesis: one-loop beta functions, small
step size, individual coupling space, smooth running, standard
parameterization. Five substantive alternatives remain untested.


% ====================================================================
\section{Phase~2: Systematic Scaffolding Removal}\label{sec:phase2}
% ====================================================================

Phase~1 tested one specific implementation of the scaffolding
hypothesis. Phase~2 tests five additional implementations, each
removing a different piece of $\LPO$ scaffolding from the SM's
treatment of fermion masses.

\subsection{Two-Loop Gauge + One-Loop Yukawa Beta Functions}\label{sec:two_loop}

\paragraph{Scaffolding removed:} ``One-loop is sufficient.'' If
the mass hierarchy is a perturbative phenomenon visible at the
right loop order, new quasi-fixed-point structure should appear at
two loops that is absent at one loop.

\paragraph{Method.} We implement two-loop gauge coupling beta
functions~\cite{LWX2003} combined with one-loop Yukawa beta
functions---capturing the dominant next-order correction to gauge
running while retaining the one-loop Yukawa structure. This is not
the full two-loop Yukawa system of Ref.~\cite{LWX2003}; the full
two-loop Yukawa coefficients include inter-generation
CKM-dependent terms not implemented here. We repeat the Phase~1
scan with $1{,}000$ random initial conditions.

\paragraph{Result.} The two-loop correction \emph{destabilizes}
the top QFP rather than creating new structure. The standard
deviation of $y_b/y_t$ at the EW scale decreases by only $3.2\%$
(the success criterion required ${>}50\%$ narrowing). No new
quasi-fixed-points appear for bottom or tau couplings
(\cref{fig:two_loop}).

\paragraph{Verdict:} Higher loop order does not generate new
attractor structure.

\begin{figure}[ht]
  \centering
  \includegraphics[width=0.65\textwidth]{plots_v2/inv1_two_loop_comparison.png}
  \caption{Distribution of $y_b/y_t$ at the EW scale: one-loop
    (left) versus two-loop gauge + one-loop Yukawa (right). The
    distributions are nearly identical; two-loop corrections do not
    narrow the spread.}
  \label{fig:two_loop}
\end{figure}


\subsection{Large Step-Size Dynamics}\label{sec:large_step}

\paragraph{Scaffolding removed:} ``The discrete map approximates
the continuous flow.'' At large step size, discrete maps can
exhibit bifurcations, periodic orbits, and chaotic behavior absent
in continuous flows. If the Yukawa RG has such structure, it would
be genuinely $\BISH$ with no $\LPO$ analogue.

\paragraph{Method.} We use the one-loop Euler map
$y_{n+1} = y_n + \Delta t \cdot \beta(y_n, g_n)$ with step
counts $N$ ranging from $3$ to $500$, tracking coupling values and
mass ratios at the EW scale as functions of~$N$.

\paragraph{Result.} All couplings and mass ratios converge
monotonically to the continuous-flow values as $N$ increases
(\cref{fig:large_step}). No bifurcation, period-doubling, or
non-monotone structure is observed at any~$N$.

\paragraph{Interpretation.}
The discrete step size $\Delta t$ is an algorithmic parameter, not
a physical one: the SM RG has no preferred discretization scale.
A physical interpretation exists in the Wilsonian framework, where
each step integrates out a momentum shell of finite width, but our
large-$\Delta t$ tests should be read as robustness checks on the
discrete map's dynamical structure rather than probes of physically
new dynamics. The monotone convergence we observe is consistent
with the beta functions being smooth and well-behaved on the
relevant domain; it does not rule out non-trivial discrete dynamics
in other systems with stiffer or more nonlinear beta functions.

\paragraph{Verdict:} The Yukawa RG discrete map has no dynamics
beyond the continuous flow at any step size.

\begin{figure}[ht]
  \centering
  \includegraphics[width=0.65\textwidth]{plots_v2/inv2_coupling_vs_N.png}
  \caption{Coupling values at the EW scale versus discrete step
    count~$N$ (Euler map). Monotone convergence to the ODE limit;
    no bifurcation structure.}
  \label{fig:large_step}
\end{figure}


\subsection{Ratio-Space Fixed Points}\label{sec:ratio_space}

\paragraph{Scaffolding removed:} ``Analyze individual couplings.''
The mass hierarchy concerns \emph{ratios}
$r_b = y_b/y_t$, $r_\tau = y_\tau/y_t$, etc. The beta functions
for ratios differ from those for individual couplings---many terms
cancel---and the fixed-point structure can differ.

\paragraph{Method.} We derive the one-loop beta functions for
$r_b$ and $r_\tau$ analytically, scan initial conditions in
$(r_b, r_\tau)$ space with $y_t$ at its QFP value, and plot the
EW-scale ratio plane (\cref{fig:ratio_plane}).

\paragraph{Result.} The coefficient of variation of $r_b(\EW)$
is $1.48$ (wide scatter). Only $13\%$ of initial conditions
produce $r_b$ within $50\%$ of the observed value (the success
criterion required ${>}20\%$ within $50\%$). No attractor is
visible in the $(r_b, r_\tau)$ plane.

\paragraph{Verdict:} Ratio space has no quasi-fixed-point
structure for the $b/t$ or $\tau/t$ mass ratios.

\begin{figure}[ht]
  \centering
  \includegraphics[width=0.65\textwidth]{plots_v2/inv3_ratio_plane.png}
  \caption{EW-scale values of $(r_b, r_\tau)$ for a grid of
    Planck-scale initial conditions with $y_t$ at its QFP.
    The red star marks observed values. No clustering is visible.}
  \label{fig:ratio_plane}
\end{figure}


\subsection{Threshold-Corrected Piecewise RG}\label{sec:thresholds}

\paragraph{Scaffolding removed:} ``Continuous smooth running.'' In
the physical SM, particles decouple at their mass thresholds; the
beta function changes at each threshold. This is inherently
discrete---a $\BISH$ object---and the self-consistency condition
(masses determine thresholds determine running determine masses)
is a finite algebraic fixed-point problem.

\paragraph{Method.} We implement piecewise one-loop running with
thresholds at $m_t$, $m_b$, $m_\tau$, and $m_c$, modifying gauge
beta function coefficients at each threshold. We iterate the
self-consistency condition from four different initial guesses
(uniform masses at $1$, $10$, $100$~GeV, and random).
This is a deliberately coarse discretization of the decoupling
process, not a precision EFT analysis: in the
$\overline{\text{MS}}$ scheme, threshold corrections involve
continuous matching coefficients rather than literal step-function
decoupling, and the choice of which thresholds to resolve is
scheme-dependent. The test probes whether piecewise-constant beta
functions have self-consistency structure, not whether the
resulting masses are numerically precise.

\paragraph{Result.} All four initial guesses converge to
$m_t/m_b \approx 1.66$ (observed: $41.3$). The piecewise RG
self-consistency does not recover the mass hierarchy
(\cref{fig:threshold}).

\paragraph{Verdict:} Threshold structure does not determine the
mass hierarchy.

\begin{figure}[ht]
  \centering
  \includegraphics[width=0.65\textwidth]{plots_v2/inv4_self_consistency.png}
  \caption{Mass ratios versus self-consistency iteration number
    for four initial guesses. All converge to the same wrong
    answer ($m_t/m_b \approx 1.66$ vs.\ observed~$41.3$).}
  \label{fig:threshold}
\end{figure}


\subsection{Koide Phase Dynamics}\label{sec:koide_phase}

\paragraph{Scaffolding removed:} ``Parameterize by individual
Yukawas.'' The Koide formula admits a circulant parameterization
$\sqrt{m_n} = \mu(1 + \sqrt{2}\cos(\delta + 2\pi n/3))$ where
$\delta \approx 2/9$ determines all three charged lepton mass
ratios. If $\delta \to 2/9$ is an infrared attractor of the RG
flow in circulant coordinates, the Koide formula has a dynamical
origin without Sumino's $\LPO$-level all-orders cancellation.

\paragraph{Method.} We implement the coordinate transformation
between $(y_e, y_\mu, y_\tau)$ and $(\mu, \delta)$ via discrete
Fourier transform on $\mathbb{Z}_3$. We evolve the RG in Yukawa
space, project to $(\mu, \delta)$ at each step, and scan initial
$\delta \in [0, 2\pi/3]$ for convergence.

\paragraph{Result.} $0\%$ of initial $\delta$ values produce
$\delta(\EW)$ near $2/9$ (\cref{fig:koide}). The Koide phase has
no infrared attractor in the SM RG.

\paragraph{Verdict:} The Koide phase is UV-sensitive; it has no
dynamical origin in SM infrared dynamics.

\begin{figure}[ht]
  \centering
  \includegraphics[width=0.65\textwidth]{plots_v2/inv5_koide_phase.png}
  \caption{$\delta(\EW)$ versus $\delta(\text{Planck})$ in the
    Koide circulant parameterization. No convergence to
    $\delta = 2/9$ (dashed line) is observed.}
  \label{fig:koide}
\end{figure}


% ====================================================================
\section{Constructive Stratification: Lean~4 Formalization}
\label{sec:lean_formalization}
% ====================================================================

A \Lean{} formalization (${\sim}900$~lines, five files, five theorems
verified against \Mathlib{}) establishes the sharp constructive
hierarchy of the SM Yukawa RG. The numerical investigations of
Phases~A--B established the negative result empirically; this section
provides the formal certificate and reveals structure invisible to
numerics: the physical mechanisms are $\BISH$, but textbook
idealizations introduce $\WLPO$ and $\LPO$ boundaries.

\begin{table}[ht]
\centering
\small
\begin{tabular}{@{}clll@{}}
\toprule
\textbf{\#} & \textbf{Theorem} & \textbf{CRM Level} & \textbf{File} \\
\midrule
1 & Picard iterate preserves $\QQ[t]$ & $\BISH$ & \texttt{PicardBISH.lean} \\
2 & Picard sequence has computable Cauchy modulus & $\BISH$ & \texttt{PicardBISH.lean} \\
3 & Ratio betas negative in top-dominant regime & $\BISH$ & \texttt{RatioBeta.lean} \\
4 & Eigenvalue gap decision requires $\LPO$ & $\LPO$ boundary & \texttt{CKM\_LPO.lean} \\
5 & Heaviside step-function evaluation requires $\WLPO$ & $\WLPO$ boundary & \texttt{Threshold\_WLPO.lean} \\
\bottomrule
\end{tabular}
\caption{Five theorems formalized in \Lean{}.}
\label{tab:five_theorems}
\end{table}


% --------------------------------------------------------------------
\subsection{Polynomial Picard Iteration is $\BISH$ (Theorems~1--2)}
\label{sec:lean_picard}
% --------------------------------------------------------------------

The SM one-loop RG equations have the form $dy/dt = \beta(y)$ where
$\beta$ is a polynomial with rational coefficients. The algebraic
Picard iteration
\[
  Y_0(t) = y_0, \qquad
  Y_{k+1}(t) = y_0 + \int_0^t \beta(Y_k(s))\,ds
\]
preserves the polynomial ring: if $Y_k \in F[X]$, then $Y_{k+1} \in
F[X]$, because polynomial composition, algebraic
antidifferentiation, and addition all preserve polynomials over any
field~$F$.

\Mathlib{} provides \texttt{Polynomial.derivative} but not its
algebraic inverse. We define a custom antiderivative that maps each
monomial $a_n X^n$ to $\frac{a_n}{n+1} X^{n+1}$, staying within the
polynomial ring over any field:

\begin{lstlisting}[style=lean4,caption={Algebraic antiderivative and Picard step (PicardBISH.lean).}]
noncomputable def Polynomial.antideriv {F : Type*}
    [Field F] (p : F[X]) : F[X] :=
  p.sum (fun n a =>
    C (a / ((n + 1) : F)) * X ^ (n + 1))

noncomputable def picardStep {F : Type*} [Field F]
    (b Yk : F[X]) (y0 : F) : F[X] :=
  C y0 + Polynomial.antideriv (b.comp Yk)

noncomputable def picardSeq {F : Type*} [Field F]
    (b : F[X]) (y0 : F) : Nat -> F[X]
  | 0 => C y0
  | n + 1 => picardStep b (picardSeq b y0 n) y0
\end{lstlisting}

\begin{theorem}[Polynomial closure --- Theorem~1]\label{thm:closure}
\leanok{}
For any polynomial $\beta \in F[X]$ and initial condition $y_0 \in F$,
every Picard iterate $Y_k(t) \in F[X]$. Evaluating at any $t \in F$
gives a value in~$F$.
\end{theorem}

\begin{proof}
By induction on $k$. The base case $Y_0 = C(y_0)$ is a constant
polynomial. For the inductive step: if $Y_k \in F[X]$, then
$\beta \circ Y_k \in F[X]$ (polynomial composition),
$\mathrm{antideriv}(\beta \circ Y_k) \in F[X]$ (algebraic
antidifferentiation), and $C(y_0) + \mathrm{antideriv}(\beta \circ
Y_k) \in F[X]$ (polynomial addition). For $F = \QQ$: evaluating at
rational $t$ gives rational output---no omniscience needed.
The \Lean{} type system itself certifies this: the return type of
\texttt{picardSeq} is \texttt{F[X]}, not a power series. The proof
is \texttt{rfl}.
\end{proof}

\begin{theorem}[Cauchy modulus --- Theorem~2]\label{thm:cauchy}
\leanok{}
For any $M > 0$, $L \geq 0$, $T \geq 0$, and $\varepsilon > 0$,
there exists a computable $N$ such that for all $n \geq N$:
\[
  M \cdot \frac{(L \cdot T)^n}{n!} < \varepsilon.
\]
\end{theorem}

\begin{proof}
The sequence $c^n / n! \to 0$ for any fixed $c \geq 0$
(\Mathlib{}, \texttt{FloorSemiring} namespace).
Extracting $N$ with tolerance $\varepsilon/M$ and multiplying by~$M$
gives $M \cdot (c^n/n!) < M \cdot (\varepsilon/M) = \varepsilon$.
\end{proof}

\noindent
\textbf{CRM verdict.}
The ODE solution at rational~$t$ is a constructive real number: the
Picard sequence provides a Cauchy sequence in $\QQ$ with an explicit
modulus computable from the polynomial coefficients. No omniscience
principle is needed. The \texttt{Classical.choice} in the axiom audit
arises solely from \Mathlib{}'s $\RR$ infrastructure---Level~2
certification~\cite{Lee26-P10}.


% --------------------------------------------------------------------
\subsection{Ratio Beta Negativity is $\BISH$ (Theorem~3)}
\label{sec:lean_ratio}
% --------------------------------------------------------------------

The ratio beta differences $F_b - F_t$ and $F_\tau - F_t$ are
linear forms in the squared couplings with rational coefficients
(\cref{eq:ratio_beta}). The formalization verifies that these are
strictly negative in the top-dominant regime.

\begin{theorem}[Ratio beta negativity --- Theorem~3]\label{thm:ratio}
\leanok{}
In the top-dominant regime ($y_t^2$ sufficiently large relative to
other squared couplings):
\begin{enumerate}[nosep]
  \item $F_b - F_t < 0$ whenever $3 y_b^2 + g_1^2 < 3 y_t^2$.
  \item $F_\tau - F_t < 0$ whenever
    $\frac{3}{2}y_b^2 + \frac{3}{2}y_\tau^2 + 8g_3^2
    < \frac{3}{2}y_t^2$.
\end{enumerate}
\end{theorem}

\begin{proof}
Direct algebraic inequality. The \Lean{} proof is:
\texttt{unfold rateBetaDiff\_bt; linarith}.
\end{proof}

\noindent
\textbf{Physical implication.}
Since $\dot{r}_f = r_f \cdot (F_f - F_t)$ and $r_f > 0$, the
negativity of $F_f - F_t$ implies $\dot{r}_f < 0$: mass ratios
\emph{decrease} under forward RG flow. The fermion mass hierarchy is
\emph{preserved} by the flow, not \emph{generated}. This formalizes
the structural negative result of \cref{sec:no_go} as a statement
about rational polynomial coefficients---purely $\BISH$.


% --------------------------------------------------------------------
\subsection{Step-Function Thresholds Cost $\WLPO$ (Theorem~5)}
\label{sec:lean_wlpo}
% --------------------------------------------------------------------

Textbook RG running uses the Heaviside step function
$\theta(\mu - m_f)$ to decouple heavy particles at mass thresholds.
Evaluating $\theta$ at a constructive real requires deciding its
sign.

\begin{theorem}[Threshold costs $\WLPO$ --- Theorem~5]
\label{thm:wlpo} \leanok{}
If we have a function $\theta : \RR \to \RR$ satisfying
$\theta(x) = 1$ for $x > 0$, $\theta(x) = 0$ for $x < 0$, and
$\theta(x) \in \{0, 1\}$ for all $x$, then $\WLPO$ holds.
\end{theorem}

\begin{lstlisting}[style=lean4,caption={Theorem~5 --- Heaviside requires $\WLPO$
  (Threshold\_WLPO.lean).}]
theorem heaviside_requires_WLPO
    (_heaviside : Real -> Real)
    (_h_pos : forall x : Real, 0 < x ->
      _heaviside x = 1)
    (_h_neg : forall x : Real, x < 0 ->
      _heaviside x = 0)
    (_h_zero_decided : forall x : Real,
      _heaviside x = 0 ||| _heaviside x = 1) :
    WLPO := by
  intro a
  by_cases h : exists n, a n = true
  . right; intro hall
    obtain <<n, hn>> := h
    have := hall n; simp [hn] at this
  . left; push_neg at h
    intro n; specialize h n; simpa using h
\end{lstlisting}

\noindent
\textbf{Constructive alternative.}
Physical threshold matching uses smooth functions. The sigmoid
$\sigma(x) = 1/(1 + e^{-x})$ is continuous and hence computable at
any computable real---a $\BISH$ construction. The \Lean{}
formalization verifies:
\texttt{smooth\_threshold\_is\_continuous : Continuous~(fun~x~=>
1~/(1~+~exp~(-x)))}.

\noindent
\textbf{CRM verdict.}
The textbook notation $\theta(\mu - m)$ introduces $\WLPO$; the
physics does not require it. The omniscience enters through the
\emph{notation}, not the \emph{mechanism}---a concrete instance of
the scaffolding principle.


% --------------------------------------------------------------------
\subsection{CKM Eigenvalue Crossings Cost $\LPO$ (Theorem~4)}
\label{sec:lean_lpo}
% --------------------------------------------------------------------

The CKM matrix is obtained by diagonalizing $Y_u^\dagger Y_u$ and
$Y_d^\dagger Y_d$. At eigenvalue crossings (mass degeneracies), the
eigenvector basis becomes discontinuous. The constructive question:
can we decide whether eigenvalues are equal or distinct?

We use the \emph{running maximum} construction from
Paper~8~\cite{Lee26-P8}: given $\alpha : \NN \to \mathrm{Bool}$,
define
$\mathrm{runMax}(\alpha, n) = \alpha(n) \lor \mathrm{runMax}(\alpha,
n-1)$. Once true, it stays true. The eigenvalue gap is encoded as:
\[
  \mathrm{gap}(\alpha, \delta, n) =
  \begin{cases}
    \delta & \text{if } \mathrm{runMax}(\alpha, n) = \mathrm{true}, \\
    0 & \text{otherwise}.
  \end{cases}
\]

\begin{theorem}[Eigenvalue gap decides $\LPO$ --- Theorem~4]
\label{thm:lpo} \leanok{}
If we can decide whether any real number equals zero
($\forall x : \RR,\; x = 0 \lor x \neq 0$), then $\LPO$ holds.
\end{theorem}

\begin{theorem}[Gapped diagonalization is $\BISH$]
\label{thm:gapped} \leanok{}
For a $2 \times 2$ diagonal matrix $\mathrm{diag}(a, a + \delta)$
with $\delta > 0$: $|a - (a + \delta)| = \delta$. Diagonalization
with a guaranteed gap requires no omniscience.
\end{theorem}

\noindent
\textbf{CRM verdict.}
Away from mass degeneracies---the case in the observed SM, where
quark masses are well separated---CKM diagonalization is $\BISH$.
Detecting whether one is \emph{at} an exact eigenvalue crossing
costs $\LPO$. The $\LPO$ boundary arises only when the formalism
handles \emph{all possible} parameter values, including exact
degeneracies. Any BSM model with guaranteed mass splittings stays
within~$\BISH$.


% --------------------------------------------------------------------
\subsection{The Constructive Stratification}\label{sec:lean_strat}
% --------------------------------------------------------------------

The five theorems establish the sharp hierarchy:
\begin{equation}\label{eq:stratification}
  \BISH \;<\; \WLPO \text{ (thresholds)} \;<\; \LPO
  \text{ (eigenvalue crossings)} \;<\; \text{full Classical.}
\end{equation}

\begin{table}[ht]
\centering
\small
\begin{tabular}{@{}lll@{}}
\toprule
\textbf{Component} & \textbf{Logical Cost} & \textbf{\Lean{} Theorem} \\
\midrule
Polynomial Picard iteration      & $\BISH$ & Theorems~1--2 \\
Ratio beta sign (mass hierarchy) & $\BISH$ & Theorem~3 \\
Smooth threshold matching        & $\BISH$
  & \texttt{smooth\_threshold\_is\_continuous} \\
Step-function threshold $\theta(\mu - m)$ & $\WLPO$ & Theorem~5 \\
CKM diagonalization (with gap)   & $\BISH$
  & \texttt{diag\_eigenvalues\_separated} \\
CKM diag.\ (arbitrary parameters) & $\LPO$ & Theorem~4 \\
\bottomrule
\end{tabular}
\caption{Constructive stratification of the SM Yukawa RG. The
  \emph{physical} mechanisms (finite-loop RG, smooth thresholds,
  gapped diagonalization) are $\BISH$; the \emph{textbook
  idealizations} (step functions, exact crossing detection) cost
  $\WLPO$ and $\LPO$.}
\label{tab:rg_strat}
\end{table}

\noindent
The constructive boundary is sharp. Physical mechanisms---polynomial
beta functions, smooth thresholds, gapped diagonalization---are
uniformly $\BISH$. Omniscience enters only through textbook
idealizations replaceable by constructive alternatives.


% ====================================================================
\section{Discussion}\label{sec:discussion}
% ====================================================================

\subsection{The Scaffolding Principle Applied to the Mass Hierarchy}
\label{sec:scaffolding_tested}

The scaffolding principle predicted that removing $\LPO$
idealizations from the SM Yukawa sector would expand the solution
space for the mass hierarchy. Ten investigations tested this across
five kinds of scaffolding:

\begin{center}
\small
\begin{tabular}{@{}lll@{}}
\toprule
\textbf{Scaffolding} & \textbf{Investigation} & \textbf{Result} \\
\midrule
One-loop sufficient & Two-loop QFPs (Phase~2, \S\ref{sec:two_loop})
  & No new structure \\
Continuous $\approx$ discrete & Large step-size (Phase~2, \S\ref{sec:large_step})
  & Monotone convergence \\
Individual couplings & Ratio space (Phase~2, \S\ref{sec:ratio_space})
  & No attractor \\
Smooth running & Thresholds (Phase~2, \S\ref{sec:thresholds})
  & Wrong fixed point \\
Standard parameterization & Koide phase (Phase~2, \S\ref{sec:koide_phase})
  & No IR attractor \\
Continuous flow (Phase~1) & Discrete map, $N=10$ to $10{,}000$
  & Same physics \\
\midrule
\multicolumn{2}{@{}l}{Generic initial conditions (Phase~1)}
  & 0/3{,}000 match \\
\multicolumn{2}{@{}l}{Koide from RG (Phase~1)} & $Q = 0.50 \ne 2/3$ \\
\multicolumn{2}{@{}l}{$b/\tau$ attractor (Phase~1)} & Weak (6\%) \\
\multicolumn{2}{@{}l}{Two-loop shift (Phase~1)} & $-0.65\%$ only \\
\bottomrule
\end{tabular}
\end{center}

\noindent
The result is unambiguous: the SM's infrared dynamics do not
determine the fermion mass hierarchy in any parameterization, at
any loop order, at any discretization scale, or under any
threshold structure tested. The mass hierarchy requires genuine
ultraviolet input. The scaffolding principle, applied to this
domain, does not expand the solution space productively---it
expands it into empty space.


\subsection{Structural Reason for the Negative Result}
\label{sec:no_go}

The negative results of Phases~1 and~2 are not merely empirical.
They reflect a structural property of one-loop Yukawa beta
functions.

At one loop, each Yukawa coupling satisfies
$\dot{y}_f = y_f \cdot F_f(y^2, g^2)$, where $F_f$ is a
polynomial in the squared couplings. The beta function for a ratio
$r_f = y_f / y_t$ is therefore
\begin{equation}\label{eq:ratio_beta}
  \dot{r}_f = r_f \bigl[ F_f(y^2, g^2) - F_t(y^2, g^2) \bigr].
\end{equation}
A nontrivial infrared fixed point for~$r_f$ requires
$F_f = F_t$---the Yukawa and gauge contributions to the two beta
functions must balance at a specific ratio value.

For the top quark, the \emph{absolute} coupling $y_t$ has a
quasi-fixed-point because $F_t$ crosses zero: the positive Yukawa
self-coupling $\frac{9}{2}y_t^2$ balances the negative QCD
contribution $-8 g_3^2$ at
$y_t^2 \approx \frac{16}{9}g_3^2$. This is the Pendleton--Ross
mechanism.

For lighter fermions ($f \ne t$), $y_f^2 \ll y_t^2$, so the
Yukawa contributions to~$F_f$ are negligible and $F_f$ is
dominated by gauge terms. But $F_t$ retains its large
$\frac{9}{2}y_t^2$ term. The difference $F_f - F_t$ is therefore
generically nonzero and dominated by
$-\frac{9}{2}y_t^2$, giving
\begin{equation}
  \dot{r}_f \approx -\frac{9}{2} \frac{y_t^2}{16\pi^2}\, r_f\,.
\end{equation}
This drives $r_f$ toward zero---lighter fermions decouple further
from the top---without crossing zero at any nontrivial~$r_f^*$.

The mass hierarchy is thus \emph{structurally stable} under
one-loop RG: whatever hierarchy exists at the UV scale is
preserved (and mildly amplified) by the flow. No attractor exists
because the equations have no mechanism to \emph{create} a
hierarchy from generic initial conditions. The scaffolding
principle cannot overcome this: removing the continuous-flow
idealization does not change the polynomial structure of the beta
functions, which is what prevents ratio fixed points.


\subsection{Scope of the Negative Result}\label{sec:scope}

The scaffolding principle was tested in one domain---the SM Yukawa
sector---and failed. This domain may be special: the Yukawa
couplings are genuinely free parameters of the SM, with no
dynamical mechanism (infrared or otherwise) constraining them
within the SM itself. Testing whether removing scaffolding reveals
hidden structure in a system that has no hidden structure is not
informative about the principle's general validity.

A fairer test would be a domain where the $\LPO$ result is
\emph{known to be derivable} but the conventional derivation uses
$\LPO$ unnecessarily---for instance, the local conservation law
(Paper~15), which is $\BISH$ and physically sufficient, versus the
global conservation law, which costs $\LPO$. Whether removing
the global-conservation scaffolding reveals a different understanding
of energy conservation is a conceptual question not addressed by
the present numerical investigation.


\subsection{The Constructive Stratification}\label{sec:bish_strat}

The Yukawa RG is the first domain in the CRM series where the
\emph{core} computation is $\BISH$: evaluate the polynomial beta
function a finite number of times, obtain the coupling at the
electroweak scale. No completed limit does physical work. The five
domains in Papers~8--17 all involve bounded monotone sequences whose
completed limits cost $\LPO$; the RG discrete map does not.

However, the \Lean{} formalization (\cref{sec:lean_formalization})
reveals that the domain is not uniformly $\BISH$ when standard
textbook operations are included. Step-function thresholds introduce
$\WLPO$ (Theorem~5) and CKM eigenvalue-crossing detection introduces
$\LPO$ (Theorem~4). The core finite-loop RG flow is $\BISH$, but the
complete textbook treatment exhibits the stratification
$\BISH < \WLPO < \LPO$---structure invisible to the numerical
investigation.

The discrimination is structural (\cref{tab:calibration}):
$\LPO$ appears when physicists assert the existence of a completed
limit or an exact decision on real-number equality, and does not
appear when the physical prediction requires only finite computation
with guaranteed gaps. This confirms that $\BMC \leftrightarrow \LPO$
is a genuine property of specific physics, and that omniscience can
enter through textbook idealizations even in domains where the core
mechanism is $\BISH$.

\begin{table}[ht]
\centering
\small
\begin{tabular}{@{}llll@{}}
\toprule
\textbf{Domain} & \textbf{Paper} & \textbf{$\BISH$ Content}
  & \textbf{$\LPO$ Content} \\
\midrule
Statistical Mechanics & 8 & Finite-volume free energy
  & Thermodynamic limit \\
General Relativity & 13 & Finite-time geodesic
  & Geodesic incompleteness \\
Quantum Measurement & 14 & Finite-step decoherence
  & Exact decoherence \\
Conservation Laws & 15 & Local energy conservation
  & Global energy \\
Quantum Gravity & 17 & Finite entropy count
  & Entropy density limit \\
\textbf{Particle Physics (RG)} & \textbf{18}
  & \textbf{Finite-step Yukawa evolution}
  & \textbf{Idealizations (\cref{tab:rg_strat})} \\
\bottomrule
\end{tabular}
\caption{CRM calibration table. The sixth row has core $\BISH$ with
  $\WLPO$/$\LPO$ entering through textbook idealizations; the
  detailed stratification is in \cref{tab:rg_strat}.}
\label{tab:calibration}
\end{table}


\subsection{CRM Analysis of Mass-Problem Approaches}
\label{sec:approaches}

A systematic CRM audit of approaches to the fermion mass
problem---including five classes of ultraviolet theories, their
logical costs, compression ratios, and the role of $\LPO$
scaffolding---is given in Appendix~\ref{app:uv_audit}. The
principal finding is that every approach, at both infrared and
ultraviolet levels, has $\BISH$ as its core logical content.
$\LPO$ enters only through dispensable idealizations (exact
modulus stabilization, exact gauge coupling unification, exact
non-perturbative expansions). The mass problem is entirely a
problem within~$\BISH$.


\subsection{Implications for Flavor Modeling}\label{sec:flavor_implications}

The constructive stratification has concrete implications for
physicists working on the flavor problem:
\begin{enumerate}[nosep]
  \item \textbf{RG flow is $\BISH$.} Any model using polynomial beta
    functions (all perturbative BSM models at finite loop order) has
    constructive RG flow. The Picard iteration preserves the
    coefficient ring at every finite step.
  \item \textbf{Threshold corrections: use smooth matching.} The
    textbook Heaviside function costs $\WLPO$; physical smooth
    matching is $\BISH$. This costs nothing in practice but
    matters for formal verification.
  \item \textbf{Mass matrix diagonalization: $\BISH$ with gap.} As
    long as eigenvalues are separated by a computable gap (true for
    the observed SM), diagonalization is constructive. The $\LPO$
    boundary appears only at exact mass degeneracies.
  \item \textbf{The mass hierarchy problem is within $\BISH$.} No
    omniscience principle is needed to state, derive, or verify any
    proposed explanation of the fermion mass hierarchy.
\end{enumerate}


\subsection{Limitations}\label{sec:limitations}

\begin{enumerate}
  \item \textbf{One-loop Yukawa dominance.} Phase~2's two-loop
    investigation used two-loop \emph{gauge} with one-loop
    \emph{Yukawa} beta functions, not the full two-loop Yukawa
    system. The full two-loop Yukawa coefficients~\cite{LWX2003}
    include inter-generation mixing via CKM, which could in
    principle create structure absent in the simplified system.

  \item \textbf{Three-loop not tested.} The original hypothesis
    predicted that successive generations might be determined at
    successive loop orders. Three-loop beta functions were not
    implemented.

  \item \textbf{Partially formalized.} The constructive
    stratification (Theorems~1--5) is verified in \Lean{} against
    \Mathlib{} (${\sim}900$~lines). However, the ten numerical
    investigations remain Python experiments: the negative results
    about mass hierarchy generation are empirical, not formally
    verified. The \Lean{} formalization certifies the
    \emph{structural} claims ($\BISH$/$\WLPO$/$\LPO$
    classification) but not the \emph{quantitative} claims (QFP
    locations, attractor basins).

  \item \textbf{QFP overshoot.} The one-loop top QFP gives
    $y_t(\EW) \approx 1.29$ versus observed $0.99$---a $30\%$
    overshoot from neglecting threshold corrections, which does
    not affect the qualitative conclusions about attractor structure.

  \item \textbf{Shared encoding pattern.} The ten investigations
    all test variants of the same broad hypothesis (does removing
    $\LPO$ scaffolding reveal mass hierarchy mechanisms?). They
    are not ten independent tests of ten independent hypotheses.

  \item \textbf{Scalar Picard only.} The \Lean{} Picard
    formalization handles scalar ODE ($y : F$); the SM has 13
    couplings. Extending to vector-valued $y : F^n$ is
    straightforward but increases the code substantially.

  \item \textbf{Custom antiderivative.} The algebraic
    \texttt{Polynomial.antideriv} is not in \Mathlib{}. Contributing
    it upstream would benefit the broader formalization community.
\end{enumerate}


% ====================================================================
\section{Conclusion}\label{sec:conclusion}
% ====================================================================

Ten numerical investigations across two phases test whether the CRM
scaffolding principle---that removing $\LPO$ idealizations from
physics can reveal $\BISH$-level mechanisms invisible in the
conventional formalism---produces new insight into the fermion mass
hierarchy. The answer is no: the Standard Model's infrared dynamics
do not determine the mass spectrum in any parameterization, at any
loop order, or at any discretization scale tested. The thirteen
Yukawa couplings are boundary conditions, not dynamical outputs.
CRM is a powerful diagnostic framework; its generative capacity, at
least for the flavor problem, is null.

The main positive result is the constructive stratification
established by the \Lean{} formalization (${\sim}900$~lines, five
theorems verified by \Mathlib{}):
\[
  \BISH \;<\; \WLPO \text{ (thresholds)} \;<\; \LPO
  \text{ (eigenvalue crossings)} \;<\; \text{full Classical.}
\]
The core finite-loop RG flow is $\BISH$---the first domain in the
CRM series where the physical mechanism requires no completed
limit---while textbook idealizations (step-function thresholds,
exact eigenvalue-crossing detection) introduce $\WLPO$ and $\LPO$.
This structure was invisible to the numerical investigation and
emerged only through formalization: the physical mechanisms are
uniformly $\BISH$; omniscience enters only through idealizations
that can be replaced by constructive alternatives (smooth
thresholds, gapped diagonalization).

The programme archive is maintained at Zenodo (DOI:
\href{https://doi.org/10.5281/zenodo.18626839}{10.5281/zenodo.18626839}).


% ====================================================================
\section*{Lean~4 Formalization Details}\label{sec:lean_details}
% ====================================================================

\paragraph{Module structure.}
The formalization comprises five files totaling 902~lines
(\cref{tab:manifest_detail}), built against \Lean{} v4.28.0-rc1 with
\Mathlib{}. Build command: \texttt{lake build} (0~errors, 0~warnings,
0~sorries).

\begin{table}[ht]
\centering
\small
\begin{tabular}{@{}lrl@{}}
\toprule
\textbf{File} & \textbf{Lines} & \textbf{Purpose} \\
\midrule
\texttt{Defs.lean}             & 118 & SM beta function coefficients as $\QQ$ \\
\texttt{RatioBeta.lean}        & 102 & Theorem~3: ratio betas negative \\
\texttt{Threshold\_WLPO.lean}  & 137 & Theorem~5: Heaviside $\to$ $\WLPO$ \\
\texttt{CKM\_LPO.lean}        & 253 & Theorem~4: eigenvalue gap $\to$ $\LPO$ \\
\texttt{PicardBISH.lean}      & 292 & Theorems~1--2: Picard is $\BISH$ \\
\midrule
\textbf{Total}                 & \textbf{902} & 5~files, 5~theorems \\
\bottomrule
\end{tabular}
\caption{File manifest for the \Lean{} formalization.}
\label{tab:manifest_detail}
\end{table}

\paragraph{Axiom audit.}
Three certification levels emerge:
\begin{itemize}[nosep]
  \item \textbf{Level~0} (no axioms): \texttt{LPO\_P18}, \texttt{WLPO},
    \texttt{runMax}---pure definitions.
  \item \textbf{Level~1} (\texttt{propext} only):
    \texttt{runMax\_witness}, \texttt{diffCoeffs\_bt\_val}---pure
    algebraic results over $\QQ$.
  \item \textbf{Level~2}: all $\RR$-valued theorems show
    \texttt{propext}, \texttt{Classical.choice},
    \texttt{Quot.sound}---arising from \Mathlib{}'s $\RR$
    Cauchy completion, not mathematical
    content~\cite{Lee26-P10}.
\end{itemize}
No \texttt{sorryAx} appears anywhere.

\paragraph{Design decisions.}
(1)~\emph{Custom antiderivative}: \Mathlib{}'s measure-theoretic
integral is classical; our algebraic \texttt{Polynomial.antideriv}
maps $a_n X^n \mapsto \frac{a_n}{n+1} X^{n+1}$ within the
polynomial ring over any field.
(2)~\emph{Running maximum encoding}: the \texttt{runMax} construction
(shared with Paper~8) converts arbitrary binary sequences to monotone
ones---the standard CRM tool for encoding $\LPO$ into physical
parameters.
(3)~\emph{Degree explosion avoidance}: we rely on the
\texttt{F[X]} type to guarantee finite polynomials at every Picard
step, without computing explicit degree bounds.
(4)~\emph{\texttt{calc} blocks}: the Cauchy modulus proof uses
\texttt{mul\_lt\_mul\_of\_pos\_left} with
\texttt{mul\_div\_cancel\textsubscript{0}} to avoid
\texttt{nlinarith}'s difficulty with division.


% ====================================================================
\section*{AI-Assisted Methodology}\label{sec:ai}
% ====================================================================

This investigation was developed using \textbf{Claude Opus~4.6}
(Anthropic, 2026) via the \textbf{Claude Code} command-line
interface~\cite{Anthropic2026}, following the same human--AI
workflow as Papers~2--17. The author specified the research
direction, scaffolding hypothesis, and CRM framing;
Claude~Opus~4.6 implemented the beta functions, numerical scanning
code, plots, Phase~2 investigation suite, and the Lean~4
formalization of Theorems~1--5. The Lean formalization was
additionally informed by review feedback from
\textbf{Gemini~2.5~Pro} (Google, 2025), which identified the
CKM eigenvalue gap pitfall (Theorem~4) and the threshold WLPO
cost (Theorem~5) as new insights beyond the original numerical
investigation.

\begin{table}[h]
\centering
\small
\begin{tabular}{@{}llll@{}}
\toprule
\textbf{Task} & \textbf{Human} & \textbf{Claude Opus 4.6}
  & \textbf{Gemini 2.5 Pro} \\
\midrule
Research direction            & \checkmark & & \\
Scaffolding hypothesis        & \checkmark & \checkmark & \\
CRM analysis of approaches    & \checkmark & \checkmark & \\
Phase~1 beta function code    & & \checkmark & \\
Phase~1 numerical scans       & & \checkmark & \\
Phase~2 investigation design  & \checkmark & \checkmark & \\
Phase~2 implementation        & & \checkmark & \\
Theorems~1--3 blueprint       & \checkmark & & \\
Theorems~4--5 identification  & \checkmark & & \checkmark \\
\Mathlib{} API discovery      & & \checkmark & \\
\Lean{} proof generation      & & \checkmark & \\
Build verification            & & \checkmark & \\
Result interpretation         & \checkmark & \checkmark & \\
Paper writing                 & \checkmark & \checkmark & \\
\bottomrule
\end{tabular}
\end{table}


% ====================================================================
\section*{Reproducibility}
% ====================================================================

\begin{mdframed}[backgroundcolor=gray!10]
\textbf{Reproducibility Box}
\begin{itemize}
\item \textbf{Repository}:
  \url{https://github.com/AICardiologist/FoundationRelativity}
\item \textbf{Phase~A}: \texttt{paper18/phase1/rg\_mass\_hierarchy.py}
  (${\sim}600$ lines, ${\sim}16$~min)
\item \textbf{Phase~B}: \texttt{paper18/phase2/rg\_phase2.py}
  (${\sim}600$ lines, ${\sim}7$~min)
\item \textbf{Phase~C (\Lean{} bundle)}:
  \texttt{paper18/P18\_YukawaRG/} (902~lines, 5~files).
  Toolchain: \texttt{leanprover/lean4:v4.28.0-rc1}.
  Build: \texttt{lake build} (0~errors, 0~warnings, 0~sorries).
  Axiom profile: no \texttt{sorryAx};
  \texttt{Classical.choice} only through \Mathlib{}'s $\RR$
  infrastructure (Level~2 certification).
\item \textbf{Dependencies}: Python~3.9+, NumPy, SciPy, Matplotlib;
  \Lean{} v4.28.0-rc1, \Mathlib{}
\item \textbf{Output}: 15~plots (10~Phase~A + 5~Phase~B),
  console summaries, \texttt{\#print axioms} audit
\item \textbf{Zenodo DOI}:
  \href{https://doi.org/10.5281/zenodo.18626839}{10.5281/zenodo.18626839}
\end{itemize}
\end{mdframed}


% ====================================================================
\section*{Acknowledgments}
% ====================================================================

The numerical investigation and Lean~4 formalization were developed
using Claude Opus~4.6 (Anthropic, 2026) via the Claude Code CLI tool.
Theorems~4 and~5 (CKM eigenvalue gap and threshold WLPO cost)
originated from review feedback by Gemini~2.5~Pro (Google, 2025).


% ====================================================================
% Bibliography
% ====================================================================
\bibliographystyle{plainnat}

\begin{thebibliography}{25}

\bibitem[Altarelli and Feruglio(2010)]{AF2010}
G.~Altarelli and F.~Feruglio.
\newblock Discrete flavor symmetries and models of neutrino mixing.
\newblock \emph{Reviews of Modern Physics}, 82:2701--2729, 2010.

\bibitem[Anthropic(2026)]{Anthropic2026}
Anthropic.
\newblock Claude Opus~4.6, 2026.
\newblock \url{https://www.anthropic.com}

\bibitem[Bishop(1967)]{Bishop1967}
E.~Bishop.
\newblock \emph{Foundations of Constructive Analysis}.
\newblock McGraw-Hill, New York, 1967.

\bibitem[Bousso and Polchinski(2000)]{BP2000}
R.~Bousso and J.~Polchinski.
\newblock Quantization of four-form fluxes and dynamical
  neutralization of the cosmological constant.
\newblock \emph{Journal of High Energy Physics}, 2000(06):006, 2000.

\bibitem[Bridges and V\^{\i}\c{t}\u{a}(2006)]{BV06}
D.~Bridges and L.~V\^{\i}\c{t}\u{a}.
\newblock \emph{Techniques of Constructive Analysis}.
\newblock Springer, 2006.

\bibitem[Froggatt and Nielsen(1979)]{FN1979}
C.~D.~Froggatt and H.~B.~Nielsen.
\newblock Hierarchy of quark masses, Cabibbo angles and CP
  violation.
\newblock \emph{Nuclear Physics B}, 147:277--298, 1979.

\bibitem[Georgi and Glashow(1974)]{GG1974}
H.~Georgi and S.~L.~Glashow.
\newblock Unity of all elementary-particle forces.
\newblock \emph{Physical Review Letters}, 32:438--441, 1974.

\bibitem[Goldberger and Wise(1999)]{GW1999}
W.~D.~Goldberger and M.~B.~Wise.
\newblock Modulus stabilization with bulk fields.
\newblock \emph{Physical Review Letters}, 83:4922--4925, 1999.

\bibitem[Hill(1981)]{Hill1981}
C.~T.~Hill.
\newblock Quark and lepton masses from renormalization-group
  fixed points.
\newblock \emph{Physical Review D}, 24:691--703, 1981.

\bibitem[Koide(1983)]{Koide1983}
Y.~Koide.
\newblock New view of quark and lepton mass hierarchy.
\newblock \emph{Physical Review D}, 28:252--254, 1983.

\bibitem[Lee(2026a)]{Lee26-P8}
P.~C.-K.~Lee.
\newblock Axiom calibration of the 1D Ising model:
  $\LPO$ dispensability.
\newblock Paper~8 in the CRM Series, 2026.

\bibitem[Lee(2026b)]{Lee26-P10}
P.~C.-K.~Lee.
\newblock The logical geography of mathematical physics.
\newblock Paper~10 in the CRM Series, 2026.

\bibitem[Lee(2026c)]{Lee26-P13}
P.~C.-K.~Lee.
\newblock Axiom calibration of Schwarzschild geodesics.
\newblock Paper~13 in the CRM Series, 2026.

\bibitem[Lee(2026d)]{Lee26-P14}
P.~C.-K.~Lee.
\newblock Axiom calibration of quantum decoherence.
\newblock Paper~14 in the CRM Series, 2026.

\bibitem[Lee(2026e)]{Lee26-P15}
P.~C.-K.~Lee.
\newblock Axiom calibration of Noether's theorem.
\newblock Paper~15 in the CRM Series, 2026.

\bibitem[Lee(2026f)]{Lee26-P17}
P.~C.-K.~Lee.
\newblock Axiom calibration of black hole entropy.
\newblock Paper~17 in the CRM Series, 2026.

\bibitem[Luo et~al.(2003)]{LWX2003}
M.-x.~Luo, H.-w.~Wang, and Y.~Xiao.
\newblock Two-loop renormalization group equations in the
  Standard Model.
\newblock \emph{Physical Review D}, 67:065019, 2003.
\newblock arXiv:hep-ph/0211440.

\bibitem[Machacek and Vaughn(1984)]{MV1984}
M.~E.~Machacek and M.~T.~Vaughn.
\newblock Two-loop renormalization group equations in a general
  quantum field theory: III.\ Scalar quartic couplings.
\newblock \emph{Nuclear Physics B}, 249:70--92, 1984.

\bibitem[Particle Data Group(2024)]{PDG2024}
Particle Data Group.
\newblock Review of Particle Physics.
\newblock \emph{Physical Review D}, 110:030001, 2024.

\bibitem[Pendleton and Ross(1981)]{PR1981}
B.~Pendleton and G.~G.~Ross.
\newblock Mass and mixing angle predictions from infrared
  fixed points.
\newblock \emph{Physics Letters B}, 98:291--294, 1981.

\bibitem[Randall and Sundrum(1999)]{RS1999}
L.~Randall and R.~Sundrum.
\newblock Large mass hierarchy from a small extra dimension.
\newblock \emph{Physical Review Letters}, 83:3370--3373, 1999.

\bibitem[Sumino(2009)]{Sumino2009}
Y.~Sumino.
\newblock Family gauge symmetry as an origin of Koide's mass
  formula and charged lepton spectrum.
\newblock \emph{Journal of High Energy Physics}, 2009(05):075,
  2009.

\end{thebibliography}


% ====================================================================
\appendix
\section{CRM Audit of Ultraviolet Approaches to the Flavor Problem}
\label{app:uv_audit}
% ====================================================================

The numerical investigations in Sections~\ref{sec:phase1}
and~\ref{sec:phase2} establish that the fermion mass hierarchy
requires ultraviolet input: the Standard Model's infrared dynamics
do not determine the Yukawa couplings in any parameterization
tested. This appendix follows the problem to the ultraviolet,
applying CRM calibration to the five main classes of theories that
purport to explain the mass hierarchy. For each, we identify the
logical cost of the derivation, the role of $\LPO$ scaffolding
(if any), and the compression ratio---the number of unexplained
inputs needed to produce the 13~Yukawa-sector observables.


% --------------------------------------------------------------------
\subsection{Froggatt--Nielsen Mechanism}\label{app:fn}
% --------------------------------------------------------------------

The Froggatt--Nielsen (FN) mechanism~\cite{FN1979} postulates a
horizontal $\mathrm{U}(1)_{\mathrm{FN}}$ symmetry under which SM
fermions carry integer charges~$q_i$. A heavy scalar flavon~$\Phi$
acquires a vacuum expectation value with
$\langle\Phi\rangle / M = \varepsilon \approx 0.22$ (numerically
close to the Cabibbo angle). The effective Yukawa entries scale as
\begin{equation}\label{eq:fn}
  y_{ij} \sim c_{ij}\,\varepsilon^{\,|q_i + q_j|},
\end{equation}
where the $c_{ij}$ are $O(1)$ coefficients. The mass hierarchy
arises from integer powers of a small rational number: the top
quark has $q_{t_L} + q_{t_R} = 0$ (order-one coupling), the
bottom has charge sum ${\sim}2$ ($\varepsilon^2 \approx 0.05$),
the charm ${\sim}4$ ($\varepsilon^4 \approx 0.002$), and so on
down the generations.

\paragraph{CRM verdict: $\BISH$, unconditionally.}
Every step is finite arithmetic on integers and rationals: assign
charges, compute integer powers of~$\varepsilon$, multiply by
$O(1)$ coefficients. No limits, convergence, or infinite processes
appear anywhere. The $\mathrm{U}(1)_{\mathrm{FN}}$ symmetry need
not be exact---an approximate symmetry produces an approximate
hierarchy---so no $\LPO$ enters even through the symmetry itself.

\paragraph{Compression.}
With free $O(1)$ coefficients: $13$~observables from
$9$~charges~$+ 1$~$(\varepsilon)$~$+ 13$~coefficients~$= 23$
inputs. The model \emph{reorganizes} rather than compresses.
With texture zeros ($c_{ij} = 1$): $13 \to 10$ (9~charges~$+$
$\varepsilon$). This is genuine compression, and the entire
Yukawa sector is determined by a finite string of integers plus
one rational number.

\paragraph{CRM observation.}
FN is the \emph{generic} $\BISH$ explanation of any hierarchy:
factor a set of numbers into powers of a base (geometric
structure) times residual noise ($O(1)$ coefficients). The
question CRM cannot answer is whether this factoring has physical
content or is curve-fitting.


% --------------------------------------------------------------------
\subsection{Discrete Flavor Symmetries}\label{app:discrete}
% --------------------------------------------------------------------

Discrete flavor symmetry models~\cite{AF2010} postulate a finite
group~$G$ (typically $A_4$, $S_4$, $\Delta(27)$, or another
subgroup of $\mathrm{SU}(3)_{\mathrm{flavor}}$) under which the
three generations transform as a triplet. The group's
representation theory constrains the Yukawa matrix structure:
which entries are zero, which are related by Clebsch--Gordan
coefficients. Flavon fields break~$G$ to residual subgroups in
the charged-lepton and neutrino sectors.

\paragraph{CRM verdict: $\BISH$.}
Every component is decidable finite algebra:
\begin{itemize}[nosep]
  \item Group theory of~$G$ ($|A_4| = 12$; character table and
    Clebsch--Gordan coefficients are finite): $\BISH$.
  \item Flavon potential minimization (polynomial in finitely many
    variables; critical points found by solving algebraic
    equations): $\BISH$.
  \item Yukawa matrix from group constraints (finite matrix
    multiplication): $\BISH$.
\end{itemize}
No $\LPO$ enters unless $G$ is embedded in a continuous group
and the exact continuous symmetry is invoked---but the discrete
group itself suffices, and its representation theory is decidable
without reference to any continuous group. The embedding is
scaffolding.

\paragraph{Compression.}
Typical models: ${\sim}15$--$20$ real parameters (flavon VEVs,
potential couplings, messenger scales) to produce
$20$~observables (13~Yukawa-sector~$+$ 7~neutrino). Compression
ratio barely exceeds~$1$. The symmetry constrains the
\emph{structure} of the Yukawa matrix (which entries vanish, which
are related) but not the \emph{scale}---the flavon VEVs carry the
scale information and are unexplained inputs.


% --------------------------------------------------------------------
\subsection{Randall--Sundrum / Extra Dimensions}\label{app:rs}
% --------------------------------------------------------------------

In the Randall--Sundrum framework~\cite{RS1999}, one warped extra
dimension of finite size~$\pi R$ produces the mass hierarchy
through fermion localization. SM fermions are five-dimensional
fields with bulk mass parameters~$c_i$. Their zero-mode profiles
scale as $f_i(y) \sim e^{(1/2 - c_i)\,k y}$, and the
four-dimensional effective Yukawa coupling is the overlap integral
of two fermion profiles with the Higgs on the IR brane:
\begin{equation}\label{eq:rs_yukawa}
  y_{ij}^{(4D)} \sim y_{ij}^{(5D)} \,
    e^{-(c_i + c_j - 1)\,k\pi R}.
\end{equation}
For $c_i > 1/2$, the zero mode is UV-brane-localized and its
overlap with the Higgs is exponentially suppressed. An $O(1)$
spread in the bulk masses~$c_i$ produces an exponential hierarchy
in the Yukawa couplings.

\paragraph{CRM verdict: $\BISH$ for the mechanism; $\LPO$ enters
only through modulus stabilization.}
The five-dimensional metric (ODE with constant coefficients), the
fermion zero modes (explicit exponentials), the overlap integral
(closed-form), and the effective Yukawa coupling (finite
arithmetic) are all $\BISH$. The extra dimension has finite
size---no limit is taken.

$\LPO$ enters through the Goldberger--Wise
mechanism~\cite{GW1999} for stabilizing the extra-dimensional
modulus: asserting that the scalar potential has an \emph{exact}
minimum is an $\LPO$ statement. But approximate stabilization
(the potential is bounded and has a region below its boundary
values) is $\BISH$ and suffices for all predictions. The $\LPO$
is dispensable.

\paragraph{Compression.}
Anarchic scenario (universal 5D Yukawa): $9$~bulk
masses~$+ 1$~coupling~$+ 2$~geometry~$= 12$ inputs $\to
13$~observables. General case: up to ${\sim}20$ inputs. The
exponential amplification of mild input spread is the most
``efficient'' mechanism in terms of output hierarchy per input
parameter.

\paragraph{CRM observation.}
Randall--Sundrum is the geometric implementation of
Froggatt--Nielsen. Both produce $y \sim \varepsilon^n$; in FN the
base $\varepsilon$ is the flavon VEV ratio, in RS the effective
base is $e^{-(c-1/2)k\pi R}$ with the bulk mass in the exponent.
The $\BISH$ content is identical---powers of a small number. The
geometric language adds physical content (a dynamical mechanism
for the small number) but not logical content.


% --------------------------------------------------------------------
\subsection{String Compactification}\label{app:string}
% --------------------------------------------------------------------

In Type~IIB string theory compactified on a Calabi--Yau
threefold~$X$, the Yukawa couplings are determined by the geometry
of~$X$: intersection numbers, period integrals, and moduli VEVs.
The physical Yukawa coupling between fermions at brane
intersections is schematically
\begin{equation}\label{eq:string_yukawa}
  y_{ijk} = \int_\Sigma \Psi_i \wedge \Psi_j \wedge \Phi_k\,,
\end{equation}
where $\Sigma$ is an internal cycle, $\Psi_i$ are zero-mode
wavefunctions, and $\Phi_k$ is the Higgs wavefunction. The
integral depends on the complex-structure and K\"ahler moduli,
which must be stabilized by fluxes and non-perturbative
effects~\cite{BP2000}.

\paragraph{CRM verdict: $\BISH$ to finite precision; $\LPO$
enters through moduli stabilization and the exact non-perturbative
superpotential.}
The topological data specifying~$X$ (Hodge numbers, intersection
numbers) are finite integers---$\BISH$. Constructing~$X$ as an
algebraic variety (checking that defining polynomials yield a
smooth manifold) is finite algebra---$\BISH$. Period integrals
satisfy Picard--Fuchs differential equations with algebraic
coefficients and can be evaluated to any finite precision---$\BISH$.

$\LPO$ enters at two points:
\begin{enumerate}[nosep]
  \item \textbf{Moduli stabilization:} asserting that the flux
    superpotential $W = \int_X G_3 \wedge \Omega$ produces an
    \emph{exact} minimum of the scalar potential (a function of
    ${\sim}100$ complex variables) is $\LPO$. Approximate
    minimization is $\BISH$ and suffices.
  \item \textbf{Non-perturbative exactness:} the superpotential
    $W_{\mathrm{np}} \sim e^{-aT}$ is the leading term in an
    instanton expansion. The claim that this form is exact to all
    orders is a completed-infinite statement---$\LPO$. Truncation
    to finitely many instanton orders is $\BISH$.
\end{enumerate}
Both $\LPO$ components are dispensable: approximate stabilization
and finite-order instanton expansion suffice for predictions at
any finite precision.

\paragraph{Compression.}
In principle: $0$~continuous free parameters---everything is
determined by discrete choices (Calabi--Yau topology, flux
integers, brane configuration). In practice: ${\sim}500$~discrete
parameters (flux integers alone number ${\sim}2 h^{2,1} + 2$)
to produce $13$~observables. The ``compression'' replaces
$13$~continuous parameters with ${\sim}500$~discrete ones.
Whether discrete inputs count as ``more explained'' than
continuous ones is a question CRM can pose precisely but cannot
answer.


% --------------------------------------------------------------------
\subsection{Grand Unification}\label{app:gut}
% --------------------------------------------------------------------

$\mathrm{SU}(5)$~\cite{GG1974} predicts $y_b = y_\tau$ at the
GUT scale (${\sim}2 \times 10^{16}$~GeV) because the
down-type quarks and charged leptons sit in the same
$\bar{\mathbf{5}}$ representation. $\mathrm{SO}(10)$ predicts
third-generation unification $y_t = y_b = y_\tau$. The predicted
GUT-scale relations, combined with RG running to the electroweak
scale, yield testable predictions for mass ratios.

\paragraph{CRM verdict: $\BISH$ for Yukawa predictions; $\LPO$
enters through exact gauge coupling unification.}
The group theory of $\mathrm{SU}(5)$ and $\mathrm{SO}(10)$
(finite-dimensional Lie algebras, decidable representation theory)
is $\BISH$. The GUT-scale matching conditions ($y_b = y_\tau$)
are algebraic---$\BISH$. The RG running from GUT to electroweak
scale is the finite discrete computation studied in the main text.

$\LPO$ enters through gauge coupling unification: the assertion
that three running couplings $g_1(\mu)$, $g_2(\mu)$, $g_3(\mu)$
meet at \emph{exactly} a single scale $M_{\mathrm{GUT}}$. This
requires deciding whether three real-valued functions are
simultaneously equal---an $\LPO$ statement. The empirical content
is that the couplings come \emph{close} to meeting within
experimental error bars at ${\sim}2 \times 10^{16}$~GeV.
Approximate unification within measurement precision is $\BISH$.

\paragraph{Compression.}
Maximal for the third generation: 1~Yukawa $\to$ 3~masses
(in $\mathrm{SO}(10)$). For lighter generations, GUT relations
fail without additional structure (Georgi--Jarlskog factors).
Total: $13 \to 5$--$7$, the best compression among approaches
that make testable predictions.

\paragraph{CRM observation.}
This is the one point in the audit where the scaffolding principle
produces a non-trivial insight. Exact gauge coupling unification
is $\LPO$ scaffolding. The empirical content (approximate
unification) is $\BISH$. The exactness assumption constrains the
GUT model space: it requires specific threshold corrections from
SUSY partners, specific higher-dimensional operators, and specific
proton decay rates. Removing the exactness requirement---asking
only for unification within experimental precision at some finite
number of loop orders---admits a wider class of models. This is a
concrete instance of the scaffolding principle widening the
solution space, though the consequences for the mass hierarchy
specifically remain to be explored.


% --------------------------------------------------------------------
\subsection{Synthesis}\label{app:synthesis}
% --------------------------------------------------------------------

\paragraph{Counting convention.}
We adopt the following rules for \Cref{tab:uv_audit}.
\emph{Continuous parameters}: each real-valued free parameter
counts as~$1$ (e.g., a flavon VEV, a bulk mass, $\varepsilon$).
\emph{Discrete parameters}: each independent discrete choice
counts as~$1$ (e.g., a charge assignment, a flux integer, a
topological invariant).
\emph{$O(1)$ coefficients}: counted as free parameters unless
set to a specific value by a texture rule or symmetry.
\emph{Exact symmetry assumptions}: count as~$0$ inputs but are
flagged in the ``$\LPO$ Component'' column if they require
all-orders or completed-infinite assertions.
\emph{Observables}: the 13~Yukawa-sector quantities
(9~fermion masses, 3~CKM angles, 1~CP phase). Neutrino
parameters are noted where relevant but not included in the
headline count.
These conventions are necessarily approximate---parameter counting
in BSM models is not canonical---but they suffice for the
qualitative comparisons intended here.

\begin{table}[ht]
\centering
\footnotesize
\begin{tabular}{@{}lccccc@{}}
\toprule
\textbf{Approach}
  & \textbf{Core}
  & \textbf{Cont.\ Params}
  & \textbf{Disc.\ Params}
  & \textbf{$\LPO$ Component}
  & \textbf{Disp.?} \\
\midrule
SM (raw)
  & $\BISH$ & 13 & 0 & None & --- \\
Froggatt--Nielsen
  & $\BISH$ & 1 ($\varepsilon$) & 9 charges & None & --- \\
FN (texture zeros)
  & $\BISH$ & 1 ($\varepsilon$) & 9 charges & None & --- \\
Discrete flavor
  & $\BISH$ & 8--15 & group & None & --- \\
Randall--Sundrum
  & $\BISH$ & 9--12 & 0 & Modulus stab. & Yes \\
String compact.
  & $\BISH$ & 0 & ${\sim}500$ & Moduli stab.\ + $W_{\mathrm{np}}$ & Yes \\
$\mathrm{SU}(5)$ GUT
  & $\BISH$ & 3--5 & 0 & Exact unification & Yes \\
$\mathrm{SO}(10)$ GUT
  & $\BISH$ & 1--3 & 0 & Exact unification & Yes \\
\bottomrule
\end{tabular}
\caption{Audit of ultraviolet approaches to the fermion mass
  hierarchy. ``Core Cost'' and ``$\LPO$ Component'' are
  CRM-specific findings. ``Cont.\ Params'' and ``Disc.\ Params''
  are standard parameter counts included for context.
  Every approach has $\BISH$ as its core logical content;
  $\LPO$ enters only through dispensable idealizations.}
\label{tab:uv_audit}
\end{table}

Three observations emerge from the audit. The first and third are
specific to CRM; the second uses standard model-comparison
methodology included for completeness.

\paragraph{1.\ Every UV approach has $\BISH$ as its core
(CRM-specific).}
The $\LPO$, where it appears, enters through
idealizations---exact stabilization, exact gauge unification,
exact non-perturbative expansions---and is dispensable in every
case. Combined with the numerical results of
Sections~\ref{sec:phase1} and~\ref{sec:phase2}, this means the
fermion mass problem is entirely a problem within~$\BISH$, at
both the infrared and ultraviolet levels. No omniscience
principle is needed to state, derive, or verify any proposed
explanation of the mass hierarchy. This finding requires the
$\BISH$/$\LPO$ distinction to formulate and is non-trivial: it
could have been otherwise (Sumino's all-orders cancellation
mechanism for the Koide formula comes close to requiring $\LPO$
essentially). The flavor problem is a different kind of mystery
from those CRM was designed to illuminate, where $\LPO$ enters
through completed limits and its dispensability is the finding.

\paragraph{2.\ Approaches differ in compression ratio and input
character (standard methodology).}
\Cref{tab:uv_audit} also records two quantities from standard
model-comparison analysis: the \emph{compression ratio}
(observables per input parameter) and the \emph{input character}
(continuous vs.\ discrete). These do not depend on the CRM
framework---physicists routinely count parameters without
reference to constructive mathematics---but they provide useful
context for the logical audit.

The compression ratio ranges from ${<}1$ (Froggatt--Nielsen with
free $O(1)$ coefficients: more inputs than observables) to
${\sim}2.5$ ($\mathrm{SO}(10)$ for the third generation). The
input character varies: Froggatt--Nielsen and discrete flavor
symmetries use continuous parameters (flavon VEVs); string
compactification uses purely discrete parameters (flux integers,
topological data); GUTs use a mixture.

CRM sees no logical distinction between continuous and discrete
inputs evaluated to finite precision---both are $\BISH$. The
question of whether discrete inputs are ``more explained'' than
continuous ones is a question about explanatory depth, not
logical cost, and lies outside CRM's scope.

\paragraph{3.\ Exact gauge coupling unification is $\LPO$
scaffolding (CRM-specific).}
The scaffolding principle failed for the SM infrared (ten
investigations, all negative). Applied to the ultraviolet, it
identifies exact gauge coupling unification as $\LPO$ scaffolding
whose removal widens the viable GUT model space. The empirical
evidence for grand unification is approximate unification
($\BISH$): three couplings come close to meeting within
experimental error bars at ${\sim}2 \times 10^{16}$~GeV. The
assertion that they meet \emph{exactly} is a completed-infinite
statement ($\LPO$): it requires deciding equality of three real
numbers. Models achieving approximate unification without exact
unification are $\BISH$-sufficient and conventionally excluded
only because the exactness assumption is treated as a theoretical
requirement rather than an idealization.

This is a concrete instance of the scaffolding principle producing
a non-trivial observation---not solving the mass problem, but
clarifying which aspects of GUT phenomenology are empirically
grounded ($\BISH$) and which are idealizations ($\LPO$). Whether
the wider $\BISH$-sufficient model space contains new explanations
of the full mass hierarchy remains open.

\medskip
\noindent
The overall picture: CRM maps the logical geography of UV flavor
physics with precision, and the map reveals that the terrain is
uniformly $\BISH$. The interesting $\LPO$ boundaries found in
five other domains (Papers~8, 13, 14, 15, 17) do not appear here.
The one exception---exact gauge coupling unification---is not
about the mass hierarchy per se, but about the GUT framework
within which mass predictions are made. CRM's contribution to the
flavor problem is diagnostic and taxonomic: it can tell you the
logical cost of any proposed explanation, identify which
idealizations are dispensable, and flag where the scaffolding
principle has purchase. It cannot replace experiment.


\end{document}

